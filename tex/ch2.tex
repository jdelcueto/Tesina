\documentclass[../main.tex]{subfiles}
\begin{document}



\chapter{El Valor Cognoscitvo de la Fe y el problema del lenguaje religioso en la Filosofía Analítica}



Este capítulo sigue el estudio de F. Conesa en Creer y Conocer. En él se plantean las cuestiones y problemáticas fundamentales relacionadas con el valor cognoscitivo de la fe en el ámbito de la filosofía analítica.  \textquestiondown{}Cuál es la significatividad del lenguaje religioso? Si se admite que no carece de significado, \textquestiondown{}Qué valor cognoscitivo tiene? Si se afirma que es susceptible de ser verdadero o falso \textquestiondown{}existe un conocimiento religioso? \textquestiondown{}Cuál es su valor? Esta cuestión nos lleva a la afirmación que estudiaremos en Anscombe: el valor cognoscitivo de la Fe es el del saber por testimonio.


\section{Significatividad del Lenguaje Religioso}

\section{El Valor Cognoscitivo del Lenguaje Religioso}

\section{El Valor Cognoscitivo de la Fe Religiosa}

\section{La Fe como Saber Por Testimonio}

\end{document}