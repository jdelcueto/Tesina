\documentclass[../main.tex]{subfiles}
\begin{document}



\chapter{El Valor Cognoscitvo de la Fe y el problema del lenguaje religioso en la Filosofía Analítica}



Este capítulo sigue el estudio de F. Conesa en Creer y Conocer. En él se plantean las cuestiones y problemáticas fundamentales relacionadas con el valor cognoscitivo de la fe en el ámbito de la filosofía analítica.  \textquestiondown{}Cuál es la significatividad del lenguaje religioso? Si se admite que no carece de significado, \textquestiondown{}Qué valor cognoscitivo tiene? Si se afirma que es susceptible de ser verdadero o falso \textquestiondown{}existe un conocimiento religioso? \textquestiondown{}Cuál es su valor? Esta cuestión nos lleva a la afirmación que estudiaremos en Anscombe: el valor cognoscitivo de la Fe es el del saber por testimonio.


\section{Significatividad del Lenguaje Religioso}

\section{El Valor Cognoscitivo del Lenguaje Religioso}

Read Knowledge and Certainty:
What is operative here is not a sensible-verification theory, but the picture theory of significant description: both the proposition and its negation are supposed to describe a possibility, otherwise the status of the proposition is other than that of a significant description.

Psychology is no more akin to philosophy than any natural science. Theory of knowledge is the philosophy of psychology.(4.1121)
Wittgenstein and Frege avoided making theory of knowledge the cardinal theory of philosophy simply by cutting it dead; by doing none, and concentrating on the philosophy of logic.
But
The influence of the tractatus produced logical positivism, whose main doctrine is verificationism and in that doctrine theory of knowledge reigns supreme and a prominent position was given to the test for significance by asking for the observations that would verify a statement. 

Wittgenstein's ideas between the Tractatus and the time when he began writing the Philosophical Investigations were closely akin to logical positivists, less akin  before or after that time. 

\section{El Valor Cognoscitivo de la Fe Religiosa}

\section{La Fe como Saber Por Testimonio}

\end{document}