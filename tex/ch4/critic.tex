\blockquote[{\Cite[\S505]{wittgenstein1969oncertes}}]{Es ist immer von Gnaden der Natur, wenn man etwas weiß}. Para Wittgenstein, el juego de lenguaje, que es esencial en nuestra experiencia de conocer, es posible por una especie de providencia de la naturaleza. Con esto se refiere al hecho de que nuestras aseveraciones son posibles porque no ocurre constantemente que neguemos los fundamentos o justificaciones de afirmaciones que hemos llegado a considerar un juicio cierto y sólido. Añade que el \blockquote[{\Cite[\S509]{wittgenstein1969oncertes}}]{juego de lenguaje sólo es posible si se confía en algo}. Y con esto no quiere decir \enquote*{si es posible confiar}, sino si se confía de hecho, si se actúa en confianza.

Elizabeth Anscombe añadiría que este juego de lenguaje, y por tanto el conocer, es posible por la Gracia de Dios\footnote{\Cite[Cf.][224]{teichmann2008ans}: \enquote{It is `by favour of Nature' that assertion and knowledge are posible (cf. \emph{On Certainty, para. 505}); for Anselm and for Anscombe, it is (also) by the grace of God.}}. Con esto no estaríamos diciendo que ella simplemente cambiaría `Naturaleza' por `Dios' en la afirmación de Wittgenstein sobre la certeza, se refiere a otras cosas más.

Por un lado `Verdad' \blockquote[{\Cite[47]{torralbaynubiola2005fayeh:verdad}}]{es uno de los nombres de Dios} y \blockquote[{\Cite[47]{torralbaynubiola2005fayeh:verdad}}]{Hay verdad en muchas cosas}. Aquí ella también es `hermana intelectual' de San Anselmo, ambos comparten una noción trascendental de la verdad como rectitud que va a través de muchas cosas: proposiciones, el pensamiento, la voluntad, la acción y el ser de las cosas\footnote{\Cite[Cf.][197]{teichmann2008ans}: \enquote{In Anselm's account of how truth serves as the goal of assertion, he describes truth as `rightness perceptible to the intellect alone' \textelp{} a rightness that is to be found no only in propositions, but also in thought, will, action, and the being of things.}}. Y aquí podríamos atribuir a la Gracia Divina nuestra capacidad de reconocer y apreciar esta rectitud.

Esta idea nos conduce a otra en la que las reflexiones de Anscombe aluden a la Providencia Divina como fundamento del lenguaje. Ella se pregunta: \enquote*{¿es la humanidad la que produce las esencias experesadas en la gramática?}\footnote{\Cite[Cf.][72]{torralbaynubiola2005fayeh:esencia}} y propone que la respuesta a esto, según su parecer, no se encuentra en la humanidad misma, sino en \enquote{quien produjo la humanidad}; y añade: \blockquote[{\Cite[73]{torralbaynubiola2005fayeh:esencia}}]{Para mucha gente hoy día, esta respuesta equivale a ``la evolución''. Pero esto no es otra cosa que decir ``bueno, ocurrió y ya está''. Una respuesta más racional sería: la Inteligencia, que creó al hombre y que creó otras cosas por medio del \emph{logos} de su sabiduría. Aquel \emph{logos} constituye una infinidad de \emph{logos} de cosas posibles y reales, y también de las invenciones humanas}. Para ella la pregunta \enquote*{¿qué ha producido las esencias expresadas en el lenguaje humano?} es equivalente a \enquote*{¿qué es lo que ha producido el ser humano, capaz de aprender un lenguaje?}. Termina diciendo: \blockquote[{\Cite[74]{torralbaynubiola2005fayeh:esencia}}]{aquello que produce las inteligencias que producen tales cosas, y el resto del lenguaje también, es a su vez una inteligencia o unas inteligencias. Pero tendrá que ser una inteligencia de tipo distinto de la humana: porque si no, tendríamos un regreso al infinito. Es necesario que esta inteligencia (o estas inteligencias) sea capaz de inventar el lenguaje, incluso aunque tenga la habilidad de usar el lenguaje como lo hacemos los seres humanos}. Y según esto podríamos atribuir a la Gracia el que la humanidad posea el lenguaje del todo.

Pero, ¿Anscombe está hablando de Dios aquí? Es posible hacer esa conexión; habla de la `Intelgiencia' ``distinta de la humana'' que crea ``por medio del \emph{logos} de su sabiduria'', que es capaz de ``inventar el lenguaje'' y de ``usar el lenguaje como lo hacemos'' nosotros. Esto evoca ya el modo en que Elizabeth hablaba de la fe. Aquí estamos en la misma situación en la que nos dejan muchas de las expresiones de Anscombe presentes en este estudio. ¿Son nociones valiosas para la teología?, ¿nos dan ocasión para hablar de Dios y de su actuar? La respuesta a esto se encuentra, en las conexiones que nos permiten establecer.

Podríamos caracterizar el enfoque de la teología fundamental en clave `dogmático-fundacional' y `apologético-misionera'\footnote{\Cite[Cf.][80-85]{ninot2009tf}}. Desde esta perspectiva, el testimonio puede ser analizado según el primer aspecto como un modo de describir y comprender la Revelación y según el segundo como un modo de ``dar razón de nuestra esperanza'' en diálogo con la sociedad plural de la que formamos parte. Desde de este enfoque, ¿qué oportunidades ofrecen las reflexiones de Anscombe que hemos estudiado al análisis del testimonio? Una buena clave para situar su aportación es esta: \blockquote[{\Cite[451]{prades2015testimonio}}]{Ninguna esfera del saber humano puede prescindir <<absolutamente>> de la confianza en los propios sentidos, en la memoria, en la percepción sensible, en el otro, en la sociedad. El hombre vive de creencias, que no son contrarias al ejercicio crítico del saber, sino que se entrelazan inevitablemente con el mismo. Por este motivo la razón del hombre es una razón creyente. La plena estatura de esta razón creyente requiere llegar a distinguir la confianza de la mera credulidad}. Es llamativa la insistencia de Elizabeth de que el terreno de nuestro conocimiento esta lleno de creencias justificadas en lo que ella llamaría `fe'. Lo que ella denomina de este modo no es credulidad, sino un juicio que podemos llegar a hacer apoyados sobre otras creencias y que consiste en confiar en alguien sobre la verdad.

Elizabeth distingue entre la confianza depositada en lo que se juzga como un misterio de fe y la credulidad en lo que puede ser simple sinsentido enigmático o una superstición. Los fundamentos de esta confianza: lo que se espera porque se cree que Dios ha hablado y por la fe depositada en sus promesas, los signos que acompañan la enseñanza de Jesús y que dan testimonio de él como el Mesías esperado, la fuerza de la enseñanza de la Tradición de la Iglesia y su actuar, el juicio crítico de que las objeciones a esta creencia pueden ser respondidas,



Las discusiones de Anscombe sobre el saber, la verdad, el lenguaje, nuestras creencias y sus fundamentos describen una visión del lenguaje vivo en la actividad humana y del ser humano activo en el uso del lenguaje en el contexto de sus relaciones. Es dentro de ese panorama que la confianza como un elemento fundamental de la actividad del lenguaje muestra su fuerza para sostener juicios sólidos.



En el centro
la creencia de que Dios está involucrado en la actividad humana del lenguaje.



El recorrido por el sector de la obra de Anscombe que hemos hecho para indagar sobre el testimonio nos deja con una impresión sobre el lenguaje y sobre la naturaleza de distintos tipos de creencias nuestras y cómo estas quedan fundamentadas o justificadas. En estas descripciones se encuentran muchas oportunidades para establecer conexiones con

Entender la revelación de Dios como testimonio
Valor apologético del testimonio

La verdad práctica raz pract 189
Anscombe explica que la verdad práctica...

EG 7-8
No me cansaré de repetir aquellas palabras de Benedicto XVI que nos llevan al centro del Evangelio: «No se comienza a ser cristiano por una decisión ética o una gran idea, sino por el encuentro con un acontecimiento, con una Persona, que da un nuevo horizonte a la vida y, con ello, una orientación decisiva»[3].

8. Sólo gracias a ese encuentro —o reencuentro— con el amor de Dios, que se convierte en feliz amistad, somos rescatados de nuestra conciencia aislada y de la autorreferencialidad. Llegamos a ser plenamente humanos cuando somos más que humanos, cuando le permitimos a Dios que nos lleve más allá de nosotros mismos para alcanzar nuestro ser más verdadero. Allí está el manantial de la acción evangelizadora. Porque, si alguien ha acogido ese amor que le devuelve el sentido de la vida, ¿cómo puede contener el deseo de comunicarlo a otros?


\blockquote[][\,(St 1, 22-25)]{\emph{Poned en práctica la palabra y no os contentéis con oírla, engañándoos a vosotros mismos. Porque quien oye la palabra y no la pone en práctica, ese se parece al hombre que se miraba la cara en un espejo y, apenas se miraba, daba media vuelta y se olvidaba de cómo era. Pero el que se concentra en una ley perfecta, la de la libertad, y permanece en ella, no como oyente olvidadizo, sino poniéndola en práctica, ese será dichoso al practicarla}}.
