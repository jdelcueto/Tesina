\label{subsec:providencia}
\blockquote[{\Cite[\S505]{wittgenstein1969oncertes}}: \enquote{Cuando se sabe alguna cosa es siempre por gracia de la Naturaleza}. Texto alemán tomado de la edición bilingüe: {\Cite[\S505]{wittgenstein1969oncert}}.]{Es ist immer von Gnaden der Natur, wenn man etwas weiß}. Para Wittgenstein, el juego de lenguaje, que es esencial en nuestra experiencia de conocer, es posible por una especie de providencia de la naturaleza. Con esto se refiere al hecho de que nuestras aseveraciones son posibles porque no ocurre constantemente que neguemos los fundamentos o justificaciones de afirmaciones que hemos llegado a considerar un juicio cierto y sólido. Añade que el \blockquote[{\Cite[\S509]{wittgenstein1969oncertes}}.]{juego de lenguaje sólo es posible si se confía en algo}. Y con esto no quiere decir \enquote*{si es posible confiar}, sino si se confía de hecho, si se actúa en confianza.

Elizabeth Anscombe añadiría que este juego de lenguaje, y por tanto el conocer, es posible por la Gracia de Dios\footnote{\Cite[Cf.][224]{teichmann2008ans}: \enquote{It is `by favour of Nature' that assertion and knowledge are posible (cf. \emph{On Certainty, para. 505}); for Anselm and for Anscombe, it is (also) by the grace of God}.}. Con esto no estaríamos diciendo que ella simplemente cambiaría `Naturaleza' por `Dios' en la afirmación de Wittgenstein sobre la certeza, se refiere a otras cosas más.

Por un lado `Verdad' \blockquote[{\Cite[47]{torralbaynubiola2005fayeh:verdad}}.]{es uno de los nombres de Dios} y \blockquote[{\Cite[47]{torralbaynubiola2005fayeh:verdad}}.]{Hay verdad en muchas cosas}. Aquí ella también es `hermana intelectual' de San Anselmo, ambos comparten una noción trascendental de la verdad como rectitud que se da a través de muchas cosas: proposiciones, el pensamiento, la voluntad, la acción y el ser de las cosas\footnote{\Cite[Cf.][197]{teichmann2008ans}: \enquote{In Anselm's account of how truth serves as the goal of assertion, he describes truth as `rightness perceptible to the intellect alone' \textelp{} a rightness that is to be found not only in propositions, but also in thought, will, action, and the being of things}.}. Y aquí podríamos atribuir a la Gracia Divina nuestra capacidad de reconocer y apreciar esta rectitud.

Esta idea nos conduce a otra en la que las reflexiones de Anscombe aluden a la Providencia Divina como fundamento del lenguaje. Ella se pregunta: \blockquote[{\Cite[36]{anscombe2005ethics:hessence}}: \enquote{How did the essences expressed in these grammars come into being? Did mankind produce them?} La traducción al español de este artículo ha sido tomada de: {\Cite{torralbaynubiola2005fayeh:esencia}}.]{¿Cómo llegaron a existir las esencias expresadas en estas gramáticas? ¿Fueron producidas por la humanidad?} y propone que la respuesta a esto, según su parecer, no se encuentra en la humanidad misma, sino en \enquote{quien produjo la humanidad}; y añade: \blockquote[{\Cite[36]{anscombe2005ethics:hessence}}: \enquote{For many people of the present day, this answer would be equivalent to `Evolution'. But that is only a way of saying `Well it happened'. A more rational answer would be `Intelligence (or intelligences) which made men and other things through the logos of its wisdom.' That logos comprises an infinity of logoi of possible and actual things, and also of \emph{human} inventions}.]{Para mucha gente hoy día, esta respuesta equivale a ``la evolución''. Pero esto no es otra cosa que decir ``bueno, ocurrió y ya está''. Una respuesta más racional sería: la Inteligencia, que creó al hombre y que creó otras cosas por medio del \emph{logos} de su sabiduría. Aquel \emph{logos} constituye una infinidad de \emph{logos} de cosas posibles y reales, y también de las invenciones humanas}. Para ella la pregunta \enquote*{¿qué ha producido las esencias expresadas en el lenguaje humano?} es equivalente a \enquote*{¿qué es lo que ha producido el ser humano, capaz de aprender un lenguaje?}. Termina diciendo: \blockquote[{\Cite[74]{anscombe2005ethics:hessence}} \enquote{what produces the intelligences that \emph{produce} such things and the rest of language too is also intelligence or intelligences --- but of a different kind from the human. (Otherwise we'd have an infinite regress.) The intelligence (or intelligences) must be capable of inventing language even if it is not a language-user as human beings are}.]{aquello que produce las inteligencias que producen tales cosas, y el resto del lenguaje también, es a su vez una inteligencia o unas inteligencias. Pero tendrá que ser una inteligencia de tipo distinto de la humana: porque si no, tendríamos un regreso al infinito. Es necesario que esta inteligencia (o estas inteligencias) sea capaz de inventar el lenguaje, incluso aunque tenga la habilidad de usar el lenguaje como lo hacemos los seres humanos}. Y según esto podríamos atribuir a la Gracia el que la humanidad posea el lenguaje del todo.

Pero, ¿Anscombe está hablando de Dios aquí? Es posible hacer esa conexión; habla de la `Inteligencia' ``distinta de la humana'' que crea ``por medio del \emph{logos} de su sabiduría'', que es capaz de ``inventar el lenguaje'' y de ``usar el lenguaje como lo hacemos'' nosotros. Esto evoca ya el modo en que Elizabeth hablaba de la fe. Aquí estamos en la misma situación en la que nos dejan muchas de las expresiones de Anscombe presentes en este estudio. ¿Son nociones valiosas para la teología?, ¿nos dan ocasión para hablar de Dios y de su actuar? La respuesta a esto se encuentra en las conexiones que nos permiten establecer.

Un modo de caracterizar el enfoque de la teología fundamental es en clave `dogmático-fundacional' y `apologético-misionera'\footnote{\Cite[Cf.][80-85]{ninot2009tf}.}. Un análisis del testimonio desde esa perspectiva consistiría en estudiarlo como un modo de describir y comprender la Revelación según el primer aspecto y, de acuerdo al segundo, como un modo de ``dar razón de nuestra esperanza'' en diálogo con la sociedad plural de la que formamos parte. Desde este enfoque, ¿qué oportunidades ofrecen las reflexiones de Anscombe que hemos estudiado? Una buena clave para situar su aportación es esta: \blockquote[{\Cite[451]{prades2015testimonio}}.]{Ninguna esfera del saber humano puede prescindir <<absolutamente>> de la confianza en los propios sentidos, en la memoria, en la percepción sensible, en el otro, en la sociedad. El hombre vive de creencias, que no son contrarias al ejercicio crítico del saber, sino que se entrelazan inevitablemente con el mismo. Por este motivo la razón del hombre es una razón creyente. La plena estatura de esta razón creyente requiere llegar a distinguir la confianza de la mera credulidad}. Es llamativa la insistencia de Elizabeth de que el terreno de nuestro conocimiento esta lleno de creencias justificadas en lo que ella llamaría `fe', es decir, `creer a alguien'. Esta disposición que es el creer parte de un juicio en el que se determina confiar en alguien sobre la verdad. La solidez específica que ofrece esta confianza en contraste a la mera credulidad es una materia en donde las aportaciones de Anscombe son claras.
\label{subsec:confianza}

Dentro de este tema son interesantes los análisis que hace sobre los presupuestos o creencias involucradas en el juicio de llegar a creer el testimonio de alguien y cómo el contenido de estas creencias es distinto al de lo que se cree al creer a alguien propiamente. También es de gran interés la pregunta sobre las `relaciones no igualmente justificadas' de la falsedad y la verdad, que Wittgenstein y San Anselmo plantean respecto de las proposiciones, y que Anscombe aplica al testimonio: ¿por qué solo decimos que creemos a alguien cuando juzgamos que dice la verdad y es veraz? Ella construye su respuesta a partir de distintos elementos; la intención que puede atribuirse a la aserción, la rectitud del que habla, el enunciado y la cosa enunciada, todos estos aspectos de la comunicación están relacionados con el hecho de que atribuímos a la verdad una relación más justificada con nuestras afirmaciones, y con el testimonio también. La distinción entre conocimiento tradicional y conocer por testimonio y cómo ambos pueden llegar a constituir un fundamento para nuestras creencias e inferencias y cómo interactúan y se apoyan mutuamente es otro aspecto relevante del rol de la confianza en la formación de la razón creyente y los criterios que tenemos para juzgarla como distinta de la credulidad.

Cuando esta confianza es propiamente `fe', es decir, `creer a Dios', no es `contraria al ejercicio crítico del saber'. El elemento extraordinario hacia el que Anscombe dirige nuestra atención en su análisis sobre la fe es la creencia de que alguna voz, hecho o enseñanza \enquote*{viene a nosotros como palabra de Dios}. Esta creencia, de que \enquote*{el Eterno entra en el tiempo, el Todo se esconde en la parte} (FR 12), que constituye un juicio incondicional, no representa para Anscombe un creer sin fundamento. Es posible comparar los planteamientos de Elizabeth con otras propuestas relacionadas con esta materia. Para el cardenal Newman, \blockquote[{\Cite[276-277]{ninot2009tf}}.]{el paso hacia un juicio incondicional de la verdad se puede efectuar gracias a la convergencia de indicios o probabilidades históricas con ayuda del ``illative sense''}. En Rahner \blockquote[{\Cite[277]{ninot2009tf}}.]{el paso hacia este juicio se encuentra en la relación recíproca entre revelación trascendental \textelp{} y la revelación categorial \textelp{} siendo ambas comprendidas una a partir de la otra}. En H.U. von Balthasar la respuesta queda formulada en el desarrollo de la categoría del \emph{universale concretum} desde la metodología fenomenológica\footcite[277]{ninot2009tf}.

Las aportaciones de Anscombe en este tema se encuentran en sus discusiones sobre los milagros, las profecías, los misterios y el conocimiento común. Hay varios elementos valiosos en su análisis, desde la indagación en el valor de un testimonio en relación al grado de probabilidad del hecho que narra, hasta la `tesis de teología natural' inspirada en la promesa del Deuteronomio. Su objetivo constantemente es describir las `razones para no dudar' o la naturaleza de la disposición que se tiene cuando se cree que Dios ha dado testimonio de sí. Entre los aspectos más sobresalientes de sus respuestas merecen ser destacados los argumentos relacionados con el `conocimiento tradicional' como fundamento de nuestras inferencias, la noción de que la `esencia es expresada en la gramática' y lo que ella llama `necesidad aristotélica'. El terreno que estudian estas argumentaciones es similar al que describe Newman al hablar del \emph{illative sense}: \blockquote[{\Cite[293]{newman1870assent}}.]{en ningún género de raciocinio sobre cosas concretas, tanto si se trata de investigación histórica como de teología, podemos hallar un criterio último de la verdad o del error de nuestra inferencia, fuera de nuestra confianza en el sentido ilativo que la sanciona; a la manera como no hay criterio de la excelencia poética, la heroicidad de una acción o la caballerosidad de una conducta fuera del sentido mental peculiar, llámese genio, gusto, sentido de lo que está bien o sentido moral, al cual corresponden cada uno de estos objetos. Nuestro deber en cada uno de estos casos es reforzar y perfeccionar la facultad especial que constituye su regla viviente, y esto lo mejor que podamos}.

La diferencia clave dentro del pensamiento de Anscombe es que la confianza que se convierte en criterio no queda depositada en una facultad individual, sino en la actividad colectiva que da vida y contexto al lenguaje. Dentro de esta comprensión, la lógica constituye un modo de representación del uso que hacemos de la expresiones. La inferencia válida, como objeto de la lógica, se analiza desde su aplicación posible como parte de la gramática del lenguaje y la necesidad lógica se entiende como el `tener que' que constituiría un movimiento posible dentro del juego de lenguaje. Adicionalmente, hay proposiciones de conocimiento común que constituyen fundamentos o reglas que hacen posible el diálogo o las inferencias y en este sentido son `fundamentales' o `sólidas'.

Esto nos deja con un aspecto de las discusiones de Anscombe que ofrece más posibilidades de indagación futura. Los artículos escogidos para el estudio han estado relacionados con los aspectos más epistemológicos del testimonio. Solamente se ha aludido a su carácter performativo y su aspecto moral en la discusión sobre la enseñanza de los misterios de fe\footnote{\Cite[Cf.][450]{prades2015testimonio}: \enquote{\textins{el testimonio} reúne las dimensiones de palabra y gesto en lo que hemos identificado como carácter performativo del acto comunicativo; es a la vez un acto de conocimiento y un acto moral; comporta su ratificación mediante la responsabilidad ante lo testimoniado, que llega a la entrega de la vida en el caso eminente del martirio}.}. Sin embargo Anscombe tiene más que aportar sobre esta materia. Un aspecto de su pensamiento que nos limitamos a apuntar es la conexión entre el bien y el uso del lenguaje y entre la acción y la verdad.

Como fundamento complementario al `\emph{logical must}' expresado en la gramática que ordena el lenguaje, la llamada `necesidad Aristotélica' constituye un `\emph{non-logical must}' que justifica el orden de nuestro lenguaje desde la noción de `\emph{good for}', de lo que es bueno para nosotros. Esto implica que el criterio de la inferencia válida que consiste en su aplicabilidad real dentro de nuestra vida humana tiene como uno de sus fundamentos una orientación hacia el bien. La pregunta \enquote*{¿en virtud de qué puede tener una aplicación real una regla proposicional en nuestra vida?} puede ser respondida diciendo: \enquote*{porque está ordenada a alcanzar el bien y evitar el mal}. En esto Elizabeth identifica un aspecto moral en el uso recto del lenguaje.

La segunda conexión, que no hemos visto, es la noción de la `verdad práctica'. Un tema importante que se encuentra en la obra de Elizabeth está relacionado con el sentido en el que las acciones pueden ser verdaderas o falsas. Esta propiedad aplicada a la acción depende de la relación entre entendimiento y deseo en la configuración de la acción humana: \blockquote[{\Cite[189]{torralba2005accion}}.]{Anscombe explica que ``la verdad práctica es \emph{producida [brought about]} por medio de la deliberación bien construida [\emph{sound}] que lleva a la decisión y a la acción, y esto \emph{incluye} la verdad de la descripción `hacer lo bueno' [\emph{doing well}]. Por tanto, \emph{si} la decisión es consistente [\emph{sound}], lo que sucede ---la acción--- se corresponde con ella tal y como yo le he descrito: justamente como la descripción de `hacer lo bueno'''. La posibilidad de describir la acción como ``hacer lo bueno'', depende de que el deseo sea recto, es decir, de que el fin de la acción o la intención \emph{con la que} ---que es a lo que hace referencia el deseo--- también se puede describir como ``hacer lo bueno''. El examen de la verdad de esa descripción es la tarea fundamental de la ética}. Desde esta valoración es posible hablar de la acción del testigo como testimonio de la verdad y esta comprensión sería complementaria al `creer a alguien' como acceso a la realidad. La descripción de la enseñanza del misterio religioso como análoga a la enseñanza moral se apoya también en esta noción. En ambos casos la acogida de la verdad implica `poner por obra la palabra', participar de la acción verdadera `haciendo lo bueno'.

Este repaso final ha tenido como objetivo ofrecer una valoración del pensamiento de Anscombe como una aportación posible dentro de ciertas reflexiones de la Teología Fundamental. Esto como adición y complemento a los apartados dedicados a la descripción del valor epistemológico del testimonio, la justificación para valorar un hecho histórico como testimonio divino y la capacidad del lenguaje religioso para comunicar la verdad presentados en el capítulo anterior.

\vspace{2.83334em}
\vspace{1.41667em}
Hay una alegría característica de la que participa el testigo del Evangelio. Desde el pensamiento de Anscombe podríamos decir que es la alegría de reconocer a Dios involucrado en nuestra vida y en la actividad humana del lenguaje y creer a ese Dios que se comunica y actuar de acuerdo a sus promesas. En este sentido es la alegría del \blockquote[][\,(DCE 1; EG 7)]{encuentro con un acontecimiento, con una Persona, que da un nuevo horizonte a la vida y, con ello, una orientación decisiva}. Este encuentro con el amor de Dios es el que nos rescata de \blockquote[][\,(EG 8)]{nuestra conciencia aislada y de la autorreferencialidad}. La vida y la obra de Elizabeth nos dan una visión de esta conciencia puesta en relación y en comunicación con la pluralidad de su entorno. Podemos comprender así que la alegría del testigo del Evangelio también es que él mismo vive involucrado en la vida y el lenguaje humano para comunicar el amor que nos lleva más allá de nosotros mismos, porque \blockquote[][\,(Ibíd.)]{si alguien ha acogido ese amor que le devuelve el sentido de la vida, ¿cómo puede contener el deseo de comunicarlo a otros?} Sirva de aliento guardar el consejo del apóstol: \blockquote[][\,(St 1, 22-25)]{\emph{Poned en práctica la palabra y no os contentéis con oírla, engañándoos a vosotros mismos. Porque quien oye la palabra y no la pone en práctica, ese se parece al hombre que se miraba la cara en un espejo y, apenas se miraba, daba media vuelta y se olvidaba de cómo era. Pero el que se concentra en una ley perfecta, la de la libertad, y permanece en ella, no como oyente olvidadizo, sino poniéndola en práctica, ese será dichoso al practicarla}}.
