% SECCIÓN 1: La verdad
\section{Verdad y Significado}

\subsection{¿Qué es tener la verdad?}
Elizabeth Anscombe visitó muchas veces la Universidad de Navarra junto con Peter
Geach. Allí impartió algunos seminarios y participó de las Reuniones
Filosóficas.\footcite[cf.~][p.~15]{fa&esphom} En una de sus visitas, en octubre
de 1983, ofreció dos lecciones tituladas: ``Verdad'' y ``La unidad de la
verdad''. Las dos investigaciones estan apoyadas en algunas reflexiones de San
Anselmo cuyos argumentos sirven a Anscombe para explorar modos de hablar de
aquello de lo que decimos que tiene verdad. Anscombe dio inicio a su ponencia
planteando la cuestión como sigue: \citalitlar{Hay verdad en muchas cosas.
  Mirando a mi título `Truth' me quedo algo sobrecogida por él, pues lo que
  salta de la página hacia mi es uno de los nombres de Dios. <<He amado la
  verdad>> me dijo una vez un profesor moribundo, después de hablarme de la
  dificultad que sentía sobre la idea de amar a Dios. Sin embargo: <<He amado la
  verdad>>. Y luego, temiendo que yo no malentendiera su afirmación: <<No me
  refiero, cuando digo eso, que \emph{tenga} la verdad>>} \citalitlar{Tener la
  verdad, estar en la verdad---¿qué es esto? Y qué quiso decir Nuestro Señor al
  llamarse a \emph{sí mismo} la verdad? <<No hay tal cosa como la verdad, sólo
  hay verdades>>, decía mi suegro a la primera esposa de Bertrand Russell.
  Russell fue su maestro; la influencia se ve con facilidad.} \citalitlar{¿Pero
  cuáles son las cosas que tienen verdad en ellas? ¿Tiene la creación? ¿tienen
  las acciones? A qué se refería Aristóteles cuando dijo que el bien de la razón
  práctica era `verdad de acuerdo con el recto deseo'? ¿Las cosas hechas por los
  hombres tienen verdad en ellas? ¿Qué, de nuevo, quiso decir Aristóteles cuando
  afirmó que el arte o la habilidad es una disposición productiva con un logos
  verdadero? Mas allá todavía: Qué fuerza tiene contar la verdad entre los
  `trascendentales', esas cosas que `atraviesan' todas las categorías y todas
  las formas especiales de las cosas; y que no pertenecen cada uno a una
  categoría, como el color: amarillo; o el area: un acre; o el animal: un
  caballo.\footcite[~71]{anscombe2011plato:truth}}

\subsection{La primacia de la verdad sobre la falsedad}
Estos cuestionamientos llevan a Anscombe a indagar en una materia en la que
Wittgenstein y San Anselmo ---dice--- son `hermanos intelectuales': ¿cuál es la
primacía de la verdad sobre la falsedad?.

San Anselmo queda prendado de esta pregunta como consecuencia de su indagación
en \emph{De Veritate}: ¿Qué es la verdad de una proposición o declaración? Ha
elegido indagar en las proposiciones o las declaraciones como aquellas clases de
las cuales más naturalmente se puede pensar que contienen los posibles
portadores del predicado `verdadero'. Así lo expresa cuando dice
\citalitinterlin{Busquemos primero qué es la verdad en una proposición, dado que
  con frecuencia llamamos a éstas verdaderas o falsas.}\autocite{De Veritate c.
  2} Anscombe sigue esta misma consideración.

El primer movimiento que Elizabeth realiza en su análisis es presentar la
distinción entre significado y verdad.
Esta distinción es familiar en las elucidaciones del Tractatus:
\citalitinterlin{La proposición tiene un sentido que es independiente de los
  hechos} \autocite[\S~4.061]{wittgenstein1922tractatus} San Anselmo también lo
considera. Una proposición no pierde su significado cuando no es verdadera. Si
la \emph{significatio} de una proposición fuera su verdad ésta
\citalitinterlin{semper esset vera}, siempre sería verdadera. Sin embargo el
significado de una proposicion \citalitinterlin{manent \ldots et cum est quod
  enunciat, et cum non est}, permanece lo mismo cuando lo que se afirma es el
caso que es y cuando no lo es.

¿Qué es la verdad de una proposición? Se podría responder que es la
\citalitinterlin{res enunciata}, es decir, la realidad correspondiente, lo que
la proposición verdadera dice. Esta respuesta nos llevaría a confusión. <<La
verdad de una proposición es que significa estos hechos.>> Si esto es así,
entonces cuando deja de ser verdadera también pierde su significado, pues el
hecho que era su signifcado ya no es. Además, si la desaparición del hecho es la
desaparición del significado y la verdad, ¿no será entonces que el hecho es la
misma cosa que el significado y la verdad? Sin embargo no es así, el hecho es lo
que la hace verdadera: lo que la proposición dice, la \emph{res enunciata} es la
causa de la verdad de una proposición y no su verdad: \citalitinterlin{non eius
  veritas, sed causa veritatis eius dicenda est}

La distinción abre otra línea de consideraciones. El hecho o la \emph{res
  enunciata} por la proposición verdadera es la causa de la verdad del
enunciado. La proposición tiene significado independientemente de si es
verdadera o falsa. En este sentido, una proposición con significado puede
guardar relación de verdad o de falsedad con los hechos. Una proposición falsa
no carece de toda relación con el hecho, sino que contiene una descripción del
hecho que hace a la proposición contraria verdadera. Podríamos pensar, entonces,
que la proposición verdadera y la proposición falsa pueden intercambiar roles.

Wittgenstein sugiere esto cuando afirma que el hecho de que `\emph{p}' y
`$\sim$\emph{p}' pueden intercambiar roles es importante pues muestra que `no'
no representa nada en la realidad. Más aún `\emph{p}' y `$\sim$\emph{p}' son
opuestos en significado pero a ambos enunciados corresponde una sola realidad;
esto es el hecho, la \emph{res enunciata} por el enunciado verdadero. Esto
permitiría sostener que que verdadero y falso son tipos de relaciones entre el
signo y la cosa significada que están igualmente justificadas.

San Anselmo también llega a afrimar esta noción de relaciones igualmente
justificadas entre signo y lo que es significado en las proposiciones verdaderas
y falsas. El diálogo entre el discípulo y el maestro se desarrolla en este punto
como sigue:
\citalitlar{\emph{Maestro} ¿Qué te parece que es la verdad en el enunciado mismo?\\
\emph{Discípulo.} No sé más que esto: cuando significa ser lo que es, entonces es verdadero y hay
 verdad en él.\\
\emph{M.} ¿Para qué se hace una afirmación?\\
\emph{D.} Para significar que lo que es, es.\\
\emph{M.} Luego, debe significarlo.\\
\emph{D.} Es cierto.\\
\emph{M.} Cuando significa que lo que es, es, significa lo que debe.\\
\emph{D.} Es manifiesto.\\
\emph{M.} Y cuando significa lo que debe, significa rectamente.\\
\emph{D.} Así es.\\}

El maestro propone que la afirmación haciendo lo que debe significa rectamente,
y es lo mismo que la afirmación se recta y sea verdadera.

Dime qué he de responder si alguien dice que incluso cuando una expresión
significa que es algo que no es, está significando lo que debe. Puesto que se le
ha dado igualmente el significar como ser tanto lo que es como lo que no es.
Pues si no se le hubiera dado el significar como siendo incluso lo que no es, no
lo significaría. Así que incluso cuando significa que es lo que no es, está
significando lo que debe. Pero si es correcto y verdadero en significar lo que
debe, como has mostrado, entonces la expresión es verdadera incluso cuando dice
que es algo que no es.

algo es el caso que no es el caso que es,

En el diálogo con su discípulo le invita a examinar \emph{para qué} es
una aseveración o afirmación. Su respuesta será que es para significar
signifying that to be the case which is the case
significar aquello como siendo el caso que es el cuál es el caso que es.
que es
el caso


Al respecto, Wittgenstein plantea:
\citalitinterlin{¿Acaso no podríamos hacernos entender usando proposiciones
  falsas tal como hemos hecho hasta ahora por medio de las verdaderas---siempre
  y cuando sepamos que están significadas
  falsamente?\footcite[\S4.062]{wittgenstein1922tractatus}}

Anscombe compara este posible modo de actuar a una táctica de Santa Juana de
Arco. La Santa empleaba un código en las comunicaciones con sus generales
subordinados que consistía en que las cartas que ella marcaba con una cruz
contenían proposiciones que debían ser interpretadas en el sentido contrario.
Juana no mentía con este código y, si no estaba en error acerca de los hechos,
sus oraciones eran verdaderas y no falsas. El código es posible.

Hasta aquí parece que las proposiciones falsas y verdaderas tienen igualdad de
relación con la realidad significada. La pregunta ahora es ¿qué, entonces,
\emph{es} desigual entre ellas? ¿Cuál es la primacia de la verdad?
