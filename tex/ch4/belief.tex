\section{Belief}

En \emph{Philosophical Encounters} de Peter Geach aparece una investigación
breve de Elizabeth llamada ``On a Queer Pattern of Argument''. El escrito
ilustra bien unas palabras expresadas por Peter en el mismo libro acerca del
talante de Anscombe: \citalitlar{Como una filósofa madura, Elizabeth me parece
  ser una pensadora más intrépida que yo: es ella quien tiene ideas audaces y
  que a primera vista resultan meramente alocadas, a lo que en ocasiones he
  reaccionado con inicial indignación. (Cfr. sus escritos \emph{The
    Intentionality of Sensation} y \emph{The Fisrt Person}) Usualmente llego a
  ver como estas audaces ideas son más justificables de lo que originalmente
  suponía. \autocite[p.~11]{philaut}}

En esta ocasión, Anscombe se detiene a analizar un patrón lógico aplicándolo a
distintos tipos de argumento. El análisis terminará por levantar más problemas
que clarificaciones. Con frecuencia los argumentos terminan sonando como
locuras. Pero acaso no son validos? Y acaso no son ambas premisas y conclusiones
bastante posibles --dado, para cada caso, una historia apropiada?
La tercera premisa usualmente reclama una historia. Pero las historias son
suplidas con facilidad.

El patrón se puede expresar como sigue:
\begin{adjustwidth}{1.2cm}{}
  1. Si \emph{p}, entonces \emph{q}.\\
  2. Si \emph{r}, entonces no (si \emph{p} entonces \emph{q}).\\
  3. Si no \emph{p} entonces \emph{r}.\\
  $\therefore$ \emph{p} y \emph{q}.
\end{adjustwidth}

\begin{adjustwidth}{1.2cm}{}
  1. Si la luz roja está encendida, entonces tambíen lo está la verde.\\
  2. Si el inhibidor del verde esta encendido, no es verdad que si la luz roja
  esta encendida, también lo está la verde.\\
  3. Si la luz roja no está encendida el inhibidor de verde está encendido.\\
  $\therefore$ La luz roja está encendida y también lo está la verde.
\end{adjustwidth}

\begin{adjustwidth}{1.2cm}{}
  1. Si Dios ha prometido a mi padre que será el padre de una gran nación por
  medio de mi, entonces mi padre lo será\\
  2. Si mi Padre me mata, no es cierto que si Dios le ha prometido que él sera
  el Padre de una gran nación por medio de mi, entonces el lo será.
  (Por lo tanto no me matará.)\\
  3. Si Dios no ha prometido a mi padre que el será el padre de una gran nación
  por medio de mí, mi padre va a matarme.\\
  $\therefore$ Dios ha prometido a mi padre y esto será cumplido\\
\end{adjustwidth}

Uno de estos argumentos que Anscombe considera se convierte en el
punto de partida de su investigación sobre qué significa creer a alguien.

Queda construido como
sigue: \citalitlar{Había tres hombres, A, B y C, hablando en cierta aldea. A
  dijo: ``Si ese árbol cae, interrumpira el paso por el camino por un largo
  tiempo.'' ``No será así si hay funcionando alguna máquina que sirva para
  remover árboles'', dijo B. C destacó: ``Habrá una, si el árbol no cae.'' El
  famoso sofista Eutídemo, un extraño en el lugar, estaba escuchando.
  Inmediatamente dijo: ``Les creo a todos. Así que infiero que el árbol caerá y
  el camino quedará interrumpido al paso.''}

¿Qué está mal en Eutídemo? Si se evalúa la lógica del argumento antes expuesto
no aparece ninguna contradicción, sin embargo hay algo extraño en la afirmación
``les creo a todos'' de Eutídemo. Si la lógica del argumento parece permitir que
la inferencia de Eutídemo sea posible, por qué suena tan extraña la posibilidad
de que les crea a todos?

Creer con un objeto personal.
¿Qué relevancia puede tener una investigación sobre la gramática de la expresión
creer a x que p? ¿Acaso no es un fenómeno tremendamente familiar? Si me dices
que has comido salchichas para el desayuno, te creería, no tiene nada de
extraño. Para Anscombe, sin embargo, creer a alguien es un tema de gran
importancia para la vida y la filosofía, además de que es un tema en sí mismo
suficientemente problemático para ameritar investigación filosófica.

Esta investigación de Anscombe, además, establece un nexo entre la gramática del
creer, de la fe y del acceder al mundo más allá de las experiencias o relaciones
de memorias por medio del testimonio.
El primer vínculo que establece es entre creer a y fe. \citalitinterlin{Si las
  palabras mantuvieran sus viejos significados hubiera llamado a mi tópico fe}
El término ha sido usado en el pasado justo con este sentido, de creer a
alguien.
Hubo una época en la que hubo el siguiente modo de hablar:
fe se distinguía como humana y divina
fe humana era creer a un mero ser human;
fe divina era creer a Dios.

El segundo vínculo queda establecido cuando Anscombe describe la importancia de
creer con un objeto personal para la teoría del conocimiento.

Creer a x que p es un tema, no sólo importante para la teología y para la
filosofía de la religión. También es de inmensa importancia para la teoría del
conocimiento.

Una decripción o teoría sobre como conocemos que no tenga en cuenta lo que
conocemos por testimonio ignora una gran parte de nuestro conocimiento del
mundo.

creer a x que p es una acepción de la palabra fe


meaning is use
grammar is essence
grammar is expressed in a tradition of use
