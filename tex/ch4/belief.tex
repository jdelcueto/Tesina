\section{Acerca del creer y su estructura}

\subsection{Un Peculiar Patrón de Argumento.}
Peter Geach dedica un breve apartado a Anscombe en su ``Autobiografía
Filosófica'', entre otras cosas que relata sobre ella, destaca el talante
filosófico de su esposa; dice: \citalitlar{Como una filósofa madura,
  Elizabeth me parece ser una pensadora más intrépida que yo: es ella quien
  tiene ideas audaces y que a primera vista resultan meramente alocadas, a lo
  que en ocasiones he reaccionado con inicial indignación. (Cf. sus escritos
  \emph{The Intentionality of Sensation} y \emph{The First Person}) Usualmente
  llego a ver cómo estas audaces ideas son más justificables de lo que
  originalmente suponía.\footnote{\cite[11]{geach1991philaut} <<As a mature
    philosopher, Elizabeth strikes me as a more adventurous thinker than I am:
    it is she who gets bold and at first sight merely zany ideas, to which I
    sometimes reacted with initial outrage. (Cfr. her papers `The Intentionality
    of Sensation' and `The First Person') Usually I come to think these bold
    ideas are more defensible than I had originally supposed.>>}} En el mismo
libro que recoge estas memorias de Geach, hay un breve artículo de Elizabeth
titulado \emph{On a Queer Pattern of Argument}\footnote{\cite{anscombe1991aqp}
  En adelante la referencia al artículo será como aparece en:
  \cite{anscombe2015logic:qpa}} que ejemplifica adecudamente las palabras de
Geach.

En esta ocasión la consideración intrépida consitirá en indagar sobre la validez
de principios lógicos familiares aplicándolos a diversos ejemplos de
argumentaciones. El extraño patrón de argumento que da título a la investigación
queda expresado de este modo:
  \begin{adjustwidth}{1.2cm}{}
    1.\hspace{.5cm}Si $p$, entonces $q$.\\
    2.\hspace{.5cm}Si $r$, entonces no (si $p$ entonces $q$).\\
    3.\hspace{.5cm}Si no $p$ entonces $r$.\\
    $\therefore$\hspace{.5cm}$p$ y $q$.
  \end{adjustwidth}

  Se obtiene `no $r$' de las primeras dos premisas y entonces `$p$' de `no $r$'
  y la tercera premisa; con la primera premisa nuevamente y `$p$' obtenemos la
  conclusión.{\footnote{\cite[299]{anscombe2015logic:qpa} <<We get `not $r$'
      from the first two premises and then `$p$' from `not $r$' and the third;
      with the first one again this gives us the conclusion>>.}} Hecha esta
  descripción, Anscombe entonces invita a considerar el siguiente argumento
  construido según el patrón anterior:
  \begin{adjustwidth}{1.2cm}{}
    1.\hspace{.5cm}Si ese árbol cae, entonces interrumpirá el paso por el camino
    durante mucho tiempo.\\
    2.\hspace{.5cm}Eso no es verdad si hay una maquina para remover árboles
    funcionando.\\
    3.\hspace{.5cm}Si el árbol no cae, habrá una maquina para remover árboles
    funcionando.\\
    $\therefore$\hspace{.5cm}El árbol caerá e interrumpirá el paso por el camino
    durante mucho tiempo.
  \end{adjustwidth}

  ¿Qué resultado se obtiene si se intenta formar un juicio razonable o
  conocimiento desde este argumento? <<Si ese árbol cae entonces interrumpirá el
  camino y si hay una maquina para remover árboles funcionando entonces no será
  verdad que si el árbol cae entonces interrumpira el camino.>> (`Si $p$
  entonces $q$ y si $r$ entonces no [si $p$ entonces $q$]'). De esta conjunción
  se sigue `no habrá una maquina para remover árboles funcionando' (`no $r$'),
  pero ¿se podría considerar esta deducción un juicio razonable?. La segunda
  premisa se lee como arrojando duda sobre la primera, y la tercera premisa
  expresa la pertinencia de la segunda. Descartar la duda y afirmar la primera
  sugiere que ya se cree la primera premisa antes de evaluar la segunda. Pero en
  ese caso el argumento mismo no explicaría los fundamentos para la conclusión.
  Aún cuando se estuviera asintiendo a las otras dos premisas porque ya se cree
  la primera, estas trabajan junto a un hipotético para sostener la
  creencia\autocite[Cf.~][300]{anscombe2015logic:qpa}.

  Anscombe entonces propone: \citalitinterlin{Si todo esto es correcto, tenemos
    aquí un caso bastante interesante de una serie de proposiciones que implican
    una conclusión pero no son fundamentos posibles para llegar a esa
    conclusión}\footnote{\cite[300]{anscombe2015logic:qpa} <<If all this is
    right, we have here a rather interesting case of a set of propositions which
    entail a conclusion but are impossible grounds for coming to that
    conclusion>>}. El argumento no necesita que se juzgue como verdadera la
  conclusión o parte de ella para considerar verdadera alguna de las premisas,
  pero sí reclama que parte de la conclusión sea fundamento para aceptar la
  combinación de modo que se pueda formar un juicio
  razonable\autocite[Cf.~][301]{anscombe2015logic:qpa}.

El curioso patrón de argumento aparece como punto de partida en otra
investigación de Elizabeth titulada ``What is it to believe someone?''. En esta
ocasión cada premisa aparece atribuida a una persona distinta y la conclusión a
un cuarto personaje. El pequeño relato aparece como sigue: \citalitlar{Había
  tres hombres, $A$, $B$ y $C$, hablando en cierta aldea. $A$ dijo: ``Si ese
  árbol cae, interrumpirá el paso por el camino durante mucho tiempo.'' ``No
  será así si hay alguna máquina para remover árboles funcionando'', dijo $B$.
  $C$ destacó: ``\emph{Habrá} una, si el árbol no cae.'' El famoso sofista
  Eutidemo, un extraño en el lugar, estaba escuchando. Inmediatamente dijo:
  ``Les creo a todos. Así que infiero que el árbol caerá e interrumpirá el paso
  por el camino.'' \footnote{\cite[1]{anscombe2008faith:tobelieve} <<There were
    three men, $A$, $B$ and $C$, talking in a certain village. $A$ said ``If
    that tree falls down, it'll block the road for a long time.'' ``That's not
    so if there's a tree-clearing machine working'', said $B$. $C$ remarked
    ``There \emph{will} be one, if the tree doesn't fall down.'' The famous
    sophist Euthydemus, a stranger in the place, was listening. He immediately
    said ``I believe you all. So I infer that the tree will fall and the road
    will be blocked.''>>}}

  ¿En qué está mal Eutidemo? Si se evalúa la lógica del argumento antes expuesto
  no aparece ninguna contradicción, sin embargo hay algo extraño en la afirmación
  ``les creo a todos''. Si la lógica del argumento parece permitir que la
  inferencia de Eutidemo sea posible, por qué suena tan extraña la posibilidad de
  que les crea a todos?

\subsection{Un tema importante para la teoría del conocimiento}
¿Qué relevancia puede tener una investigación sobre la gramática de la expresión
`creer a $X$ que $p$'? ¿Acaso no es un fenómeno tremendamente familiar? <<Si me
dices que has comido salchichas para el desayuno, te
creería>>\autocite[1]{anscombe2008faith:tobelieve}, no tiene nada de extraño.

No es difícil imaginar una motivación de fe a esta investigación de Elizabeth,
sin embargo su objetivo es más general. El tema en sí mismo es suficientemente
problemático como para merecer investigación filosófica. Creer a alguien es,
además, una herramienta que empleamos tan ampliamente que representa un tema de
gran impotancia para la vida y debe ocupar un lugar en la teoría del
conocimiento.

 Atender la pregunta que es eso que llamamos creer a alguien
consistirá para anscombe en una descripción de la estructura del creer
desde un análisis de su gramática.

El objetivo de Elizabeth es analizar la expresión `creer' con un objeto
personal.



\subsection{Nexos}
Esta investigación de Anscombe establece varios nexos importantes. La gramática
de `creer a $X$ que $p$' está relacionada en puntos importantes con la gramática
de la fe. El testimonio es descrito como el complemento (in)directo en la
expresión `Creer a $X$ que $p$'. Adicionalmente, el análisis de Anscombe
presenta una descripción de la estructura de creer que es útil para la
descripción del testimonio.

El primer vínculo que establece es entre `creer a' y la fe. \citalitinterlin{Si
  las palabras mantuvieran sus viejos significados habría llamado a mi tópico
  fe}. Hoy la palabra se usa para significar lo mismo que religión, o
posiblemente creencia religiosa. Según este uso, creer en Dios --creer que Dios
es, no que pueda ayudarnos, por ejemplo-- se llamaría fe. Esto ha tenido un
efecto dañino para el pensamiento sobre la religión. En el pasado, sin embargo,
el término ha sido usado justo con el sentido de `creer a alguien'. Cuando se
usaba de este modo, fe se distinguía como humana y divina, según se usara para
hablar de creer a un ser humano o creer a Dios.

El segundo vínculo queda establecido cuando Anscombe describe la importancia de
creer con un objeto personal para la teoría del conocimiento. `Creer a $X$ que
$p$' es un tema importante, no sólo para la teología y para la filosofía de la
religión, sino también para la teoría del conocimiento. Una descripción o teoría
sobre cómo conocemos que no tenga en cuenta lo que conocemos por testimonio
ignora una gran parte de nuestro modo de conocer el mundo. \citalitinterlin{La
  mayor parte de nuestro conocimiento de la realidad descansa sobre la creencia
  que depositamos en las cosas que se nos han dicho y enseñado.}

En tercer lugar Anscombe rechaza la teoría de Hume sobre nuestro acceso a la
realidad más allá de nuestra experiencia o relación de ideas y su descripción de
la estructura del creer en el testimonio. La descripción de Hume consiste en
subsumir el creer en el testimonio bajo nuestra creencia en causas y efectos. Su
pensamiento era que creemos en las causas porque percibimos sus efectos y causa
y efecto siempre se han encontrado yendo juntos. Similarmente creemos en el
testimonio porque percibimos el testimonio y hemos encontrado que siempre (¡al
menos con frecuencia!) testimonio y verdad van juntos. Es así que la idea de
causa y efecto es nuestro puente para llegar a cualquier idea del mundo más allá
de nuestra experiencia personal. Anscombe tacha la propuesta de absurda y
plantea: \citalitlar{Hemos de reconocer al testimonio como el que nos da nuestro
  mundo más grande en no menor grado, o incluso en un grado mayor, que la
  relación de causa y efecto; y creerlo es bastante distinto en estructura que
  el creer en causas y efectos. Tampoco es lo que el testimonio nos da una parte
  completamente desprendible, como el fleco de grasa en un pedazo de filete. Es
  más bien como los flequillos y rayas de grasa que están distribuidos a través
  de la buena carne; aunque hay grumos de pura grasa también.}

Establecidos estos preámbulos, Anscombe se adentra propiamente en la gramática
de creer con un objeto personal y, por tanto, su análisis de la estructura del
creer el testimonio de alguien.

Varias preguntas dirigen la investigación de Anscombe:
Creer a alguien es creer lo que la persona dice?
Puedo creer algo que otra persona cree, y no porque le crea. Un maestro de
filosofía, por ejemplo, no espera que sus discípulos le crean, sino que lleguen
a ver lo que está diciendo. En este sentido creer a alguien no es solamente
creer lo que la persona dice.

Entonces, ¿creer a alguien es creer en algo porque la persona dice que es así?
Una persona puede mentirme y a la vez creer lo que no es. No podría decir aquí
que le estoy creyendo.

¿Puedo decir que me creo a mi mismo? Creer con un objeto personal no puede ser
reflexivo, puesto que creer a alguien es creer que NN cree lo que dice. No
podría decir que me creo que x porque no estaría creyendo que creo lo que digo.

Esta última característica es la que hace que la afirmación de Eutidemo suene
como una locura, creer a alguien no es simplemente creer lo que dijo porque lo
dijo, sino creer que NN cree lo que dice.
