\section{Belief}

En \emph{Philosophical Encounters} de Peter Geach aparece una investigación
breve de Elizabeth llamada ``On a Queer Pattern of Argument''. El escrito
ilustra bien unas palabras expresadas por Peter en el mismo libro acerca del
talante de Anscombe: \citalitlar{Como una filósofa madura, Elizabeth me parece
  ser una pensadora más intrépida que yo: es ella quien tiene ideas audaces y
  que a primera vista resultan meramente alocadas, a lo que en ocasiones he
  reaccionado con inicial indignación. (Cfr. sus escritos \emph{The
    Intentionality of Sensation} y \emph{The Fisrt Person}) Usualmente llego a
  ver como estas audaces ideas son más justificables de lo que originalmente
  suponía. \autocite[p.~11]{philaut}}

En esta ocasión, Anscombe se detiene a analizar un patrón lógico aplicándolo a
distintos tipos de argumento. El análisis terminará por levantar más problemas
que clarificaciones. Con frecuencia los argumentos terminan sonando como
locuras. Pero acaso no son validos? Y acaso no son ambas premisas y conclusiones
bastante posibles --dado, para cada caso, una historia apropiada?
La tercera premisa usualmente reclama una historia. Pero las historias son
suplidas con facilidad.

El patrón se puede expresar como sigue:
\begin{adjustwidth}{1.2cm}{}
  1. Si \emph{p}, entonces \emph{q}.\\
  2. Si \emph{r}, entonces no (si \emph{p} entonces \emph{q}).\\
  3. Si no \emph{p} entonces \emph{r}.\\
  $\therefore$ \emph{p} y \emph{q}.
\end{adjustwidth}

\begin{adjustwidth}{1.2cm}{}
  1. Si la luz roja está encendida, entonces tambíen lo está la verde.\\
  2. Si el inhibidor del verde esta encendido, no es verdad que si la luz roja
  esta encendida, también lo está la verde.\\
  3. Si la luz roja no está encendida el inhibidor de verde está encendido.\\
  $\therefore$ La luz roja está encendida y también lo está la verde.
\end{adjustwidth}

\begin{adjustwidth}{1.2cm}{}
  1. Si Dios ha prometido a mi padre que será el padre de una gran nación por
  medio de mi, entonces mi padre lo será\\
  2. Si mi Padre me mata, no es cierto que si Dios le ha prometido que él sera
  el Padre de una gran nación por medio de mi, entonces el lo será.
  (Por lo tanto no me matará.)\\
  3. Si Dios no ha prometido a mi padre que el será el padre de una gran nación
  por medio de mí, mi padre va a matarme.\\
  $\therefore$ Dios ha prometido a mi padre y esto será cumplido\\
\end{adjustwidth}

Uno de estos argumentos que Anscombe considera se convierte en el
punto de partida de su investigación sobre qué significa creer a alguien.

Queda construido como
sigue: \citalitlar{Había tres hombres, A, B y C, hablando en cierta aldea. A
  dijo: ``Si ese árbol cae, interrumpira el paso por el camino por un largo
  tiempo.'' ``No será así si hay funcionando alguna máquina que sirva para
  remover árboles'', dijo B. C destacó: ``Habrá una, si el árbol no cae.'' El
  famoso sofista Eutídemo, un extraño en el lugar, estaba escuchando.
  Inmediatamente dijo: ``Les creo a todos. Así que infiero que el árbol caerá y
  el camino quedará interrumpido al paso.''}

¿Qué está mal en Eutídemo? Si se evalúa la lógica del argumento antes expuesto
no aparece ninguna contradicción, sin embargo hay algo extraño en la afirmación
``les creo a todos'' de Eutídemo. Si la lógica del argumento parece permitir que
la inferencia de Eutídemo sea posible, por qué suena tan extraña la posibilidad
de que les crea a todos?

El objetivo de Elizabeth es analizar la expresión `creer' con un objeto
personal. ¿Qué relevancia puede tener una investigación sobre la gramática de la
expresión `creer a $x$ que $p$'? ¿Acaso no es un fenómeno tremendamente
familiar? Si me dices que has comido salchichas para el desayuno, te creería, no
tiene nada de extraño. Para Anscombe, sin embargo, creer a alguien es un tema de
gran importancia para la vida y la filosofía, además de que es un tema en sí
mismo suficientemente problemático para ameritar investigación filosófica.

Esta investigación de Anscombe establece varios nexos importantes. La gramática
de `creer a $x$ que $p$' está relacionada en puntos importantes con la gramática
de la fe. El testimonio es descrito como el complemento (in)directo en la
expresión `Creer a $x$ que $p$'. Adicionalmente, el análisis de Anscombe
presenta una descripción de la estructura de creer que es útil para la
descripción del testimonio.

El primer vínculo que establece es entre `creer a' y la fe. \citalitinterlin{Si
  las palabras mantuvieran sus viejos significados hubiera llamado a mi tópico
  fe}. Hoy la palabra se usa para significar lo mismo que religión, o
posiblemente creencia religiosa. Según este uso, creer en Dios --creer que Dios
es, no que pueda ayudarnos, por ejemplo-- se llamaría fe. Esto ha tenido un
efecto dañino para el pensamiento sobre la religión. En el pasado, sin embargo,
el término ha sido usado justo con el sentido de `creer a alguien'. Cuando se
usaba de este modo, fe se distinguía como humana y divina, según se usara para
hablar de creer a un ser humano o creer a Dios.

El segundo vínculo queda establecido cuando Anscombe describe la importancia de
creer con un objeto personal para la teoría del conocimiento. `Creer a $x$ que
$p$' es un tema importante, no sólo para la teología y para la filosofía de la
religión, sino también para la teoría del conocimiento. Una descripción o teoría
sobre cómo conocemos que no tenga en cuenta lo que conocemos por testimonio
ignora una gran parte de nuestro modo de conocer el mundo. \citalitinterlin{La
  mayor parte de nuestro conocimiento de la realidad descansa sobre la creencia
  que depositamos en las cosas que se nos han dicho y enseñado.}

En tercer lugar Anscombe rechaza la teoría de Hume sobre nuestro acceso a la
realidad más allá de nuestra experiencia o relación de ideas y su descripción de
la estructura del creer en el testimonio. La descripción de Hume consiste en
subsumir el creer en el testimonio bajo nuestra creencia en causas y efectos. Su
pensamiento era que creemos en las causas porque percibimos sus efectos y causa
y efecto siempre se han encontrado yendo juntos. Similarmente creemos en el
testimonio porque percibimos el testimonio y hemos encontrado que siempre (¡al
menos con frecuencia!) testimonio y verdad van juntos. Es así que la idea de
causa y efecto es nuestro puente para llegar a cualquier idea del mundo más allá
de nuestra experiencia personal. Anscombe tacha la propuesta de absurda y
plantea: \citalitlar{Hemos de reconocer al testimonio como el que nos da nuestro
  mundo más grande en no menor grado, o incluso en un grado mayor, que la
  relación de causa y efecto; y creerlo es bastante distinto en estructura que
  el creer en causas y efectos. Tampoco es lo que el testimonio nos da una parte
  completamente desprendible, como el fleco de grasa en un pedazo de filete. Es
  más bien como los flequillos y rayas de grasa que están distribuidos a través
  de la buena carne; aunque hay grumos de pura grasa también.}

Establecidos estos preámbulos, Anscombe se adentra propiamente en la gramática
de creer con un objeto personal y, por tanto, su análisis de la estructura del
creer el testimonio de alguien.

Varias preguntas dirigen la investigación de Anscombe:
Creer a alguien es creer lo que la persona dice?
Puedo creer algo que otra persona cree, y no porque le crea. Un maestro de
filosofía, por ejemplo, no espera que sus discípulos le crean, sino que lleguen
a ver lo que está diciendo. En este sentido creer a alguien no es solamente
creer lo que la persona dice.

Entonces, ¿creer a alguien es creer en algo porque la persona dice que es así?
Una persona puede mentirme y a la vez creer lo que no es. No podría decir aquí
que le estoy creyendo.

¿Puedo decir que me creo a mi mismo? Creer con un objeto personal no puede ser
reflexivo, puesto que creer a alguien es creer que NN cree lo que dice. No
podría decir que me creo que x porque no estaría creyendo que creo lo que digo.

Esta última característica es la que hace que la afirmación de Eutidemo suene
como una locura, creer a alguien no es simplemente creer lo que dijo porque lo
dijo, sino creer que NN cree lo que dice.
