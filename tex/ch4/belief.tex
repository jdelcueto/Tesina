\section{Acerca del creer y su estructura}

\subsection{Un Peculiar Patrón de Argumento.}
Peter Geach dedica un breve apartado a Anscombe en su ``Autobiografía
Filosófica''. Ambos se dedicaban a la filosofía y era común que cuestionaran a
uno sobre el pensamiento del otro, sin embargo no era raro que no supieran cómo
contestar. Los dos tenían distintos intereses en sus investigaciones y tambíen
un estilo diferente al acercarse a los problemas filosóficos. Geach lo describe
así: \citalitlar{Como una filósofa madura, Elizabeth me parece ser una pensadora
  más intrépida que yo: es ella quien tiene ideas audaces y que a primera vista
  resultan meramente alocadas, a lo que en ocasiones he reaccionado con inicial
  indignación. (Cfr. sus escritos \emph{The Intentionality of Sensation} y
  \emph{The First Person}) Usualmente llego a ver cómo estas audaces ideas son
  más justificables de lo que originalmente
  suponía\footnote{\cite[11]{geach1991philaut}: <<As a mature philosopher,
    Elizabeth strikes me as a more adventurous thinker than I am: it is she who
    gets bold and at first sight merely zany ideas, to which I sometimes reacted
    with initial outrage. (Cfr. her papers `The Intentionality of Sensation' and
    `The First Person') Usually I come to think these bold ideas are more
    defensible than I had originally supposed.>>}.} El mismo libro que recoge
estas memorias de Geach contiene un breve artículo de Elizabeth titulado
\emph{On a Queer Pattern of Argument}\footnote{\cite{anscombe1991aqp} En
  adelante la referencia al artículo será como aparece en:
  \cite{anscombe2015logic:qpa}} que ejemplifica adecudamente las palabras antes
referidas sobre su esposa.

En esta ocasión la consideración intrépida consitirá en indagar sobre la validez
de principios lógicos familiares aplicándolos a diversos ejemplos de
argumentaciones. El extraño patrón de argumento que da título a la investigación
queda expresado de este modo:
  \begin{adjustwidth}{1.2cm}{}
    1.\hspace{.459cm}Si $p$, entonces $q$.\\
    2.\hspace{.459cm}Si $r$, entonces no (si $p$ entonces $q$).\\
    3.\hspace{.459cm}Si no $p$ entonces $r$.\\
    $\therefore$\hspace{.459cm}$p$ y $q$.
  \end{adjustwidth}

  Se obtiene `no $r$' de las primeras dos premisas y entonces `$p$' de `no $r$'
  y la tercera premisa; con la primera premisa nuevamente y `$p$' obtenemos la
  conclusión.{\footnote{\cite[299]{anscombe2015logic:qpa} <<We get `not $r$'
      from the first two premises and then `$p$' from `not $r$' and the third;
      with the first one again this gives us the conclusion>>.}} Hecha esta
  descripción, Anscombe entonces invita a considerar el siguiente argumento
  construido según el patrón anterior:
  \begin{adjustwidth}{1.2cm}{}
    1.\hspace{.459cm}Si ese árbol cae, entonces interrumpirá el paso por el camino
    durante mucho tiempo.\\
    2.\hspace{.459cm}Eso no es verdad si hay una máquina para remover árboles
    funcionando.\\
    3.\hspace{.459cm}Si el árbol no cae, habrá una máquina para remover árboles
    funcionando.\\
    $\therefore$\hspace{.459cm}El árbol caerá e interrumpirá el paso por el camino
    durante mucho tiempo.
  \end{adjustwidth}

  ¿Qué resultado se obtiene si se intenta formar un juicio razonable o
  conocimiento desde este argumento? <<Si ese árbol cae entonces interrumpirá el
  camino y si hay una maquina para remover árboles funcionando entonces no será
  verdad que si el árbol cae entonces interrumpira el camino.>> (`Si $p$
  entonces $q$ y si $r$ entonces no [si $p$ entonces $q$]'). De esta conjunción
  se sigue `no habrá una maquina para remover árboles funcionando' (`no $r$'),
  pero ¿se podría considerar esta deducción un juicio razonable?. La segunda
  premisa se lee como arrojando duda sobre la primera, y la tercera premisa
  expresa la pertinencia de la segunda. Descartar la duda y afirmar la primera
  sugiere que ya se cree la primera premisa antes de evaluar la segunda. Pero en
  ese caso el argumento mismo no explicaría los fundamentos para la conclusión.
  Aún cuando se estuviera asintiendo a las otras dos premisas porque ya se cree
  la primera, estas trabajan junto a un hipotético para sostener la
  creencia\autocite[Cf.~][300]{anscombe2015logic:qpa}.

  Anscombe entonces propone: \citalitinterlin{Si todo esto es correcto, tenemos
    aquí un caso bastante interesante de una serie de proposiciones que implican
    una conclusión pero no son fundamentos posibles para llegar a esa
    conclusión}\footnote{\cite[300]{anscombe2015logic:qpa} <<If all this is
    right, we have here a rather interesting case of a set of propositions which
    entail a conclusion but are impossible grounds for coming to that
    conclusion>>}. El argumento no necesita que se juzgue como verdadera la
  conclusión o parte de ella para considerar verdadera alguna de las premisas,
  pero sí reclama que parte de la conclusión sea fundamento para aceptar la
  combinación de modo que se pueda formar conocimiento o un juicio
  razonable\footnote{\cite[Cf.~][301]{anscombe2015logic:qpa}}.

  En este caso `$q$' no se sigue necesariamente de `$p$' y así, al no ser una
  verdad necesaria, sólo se puede aceptar la conjunción de las primeras dos
  premisas si se está independientemente seguro de que `no $r$'. No es común que
  estemos en la situación de pensar que `si $p$ entonces $q$' y que sólo por eso
  esté claro que `si $r$ entonces no (si $p$ entonces $q$)' y entonces poder
  deducir razonablemente de esto que `no $r$'. Es más común que al juzgar la
  conjunción de las primeras dos premisas, el antecedente de la segunda pierda
  fuerza. El punto de la segunda premisa es arrojar duda sobre la primera; la
  conjunción de la segunda premisa y la tercera refuerzan la pertinencia de la
  segunda. Sin embargo la segunda premisa sólo tendrá la fuerza de poner en duda
  la primera premisa ---y no al revés--- si, además de ser verdadera y
  pertinente, resulta imposible de descartar porque resulta necesario tomar en
  serio su antecedente `si $r$'\autocite[Cf.~][301]{anscombe2015logic:qpa}.

  Anscombe acuña la expresión `revocabilidad
  esencial'\footnote{\cite[Cf.~][301]{anscombe2015logic:qpa}: <<Then we have
    perhaps discovered the special character of (theoretical) hypotheticals
    whose consquents don't follow logically from their antecedents. We might
    call this character `essential defeasibility'>>.} para denominar al carácter
  especial de hipotéticos teoréticos cuyos consecuentes no se siguen lógicamente
  de sus antecedentes. En este caso esta característica es la que hace que
  incluso cuando `no $r$' se sigue de `si $p$ entonces $q$ y si $r$, entonces no
  (si $p$ entonces $q$)', no sería razonable deducir `no $r$' de esa conjunción.
  Elizabeth además observa que hay un gran número de juicios que son así. Al
  hacer una afirmación categórica con la seguridad apropiada, frecuentemente se
  descarta inmediatamente lo que la falsificaría sólo porque se sabe que ésta es
  verdadera. Sin embargo existe toda una clase de juicios como el que se ha
  analizado que al ser hecho no se descarta implicitamente como falso todo lo
  que los falsificaría\autocite[Cf.~][302]{anscombe2015logic:qpa}.

\subsection{¿Qué es creer a alguien?}
\subsubsection{Cuestión preliminar}
En el análisis anterior Anscombe ha descrito un escenario en el que combinar
varias premisas como conocimiento o juicio razonable resulta problemático a la
hora de justificar el fundamento de la conclusión apoyándose sólo en las
premisas y su relación lógica.

En su investigación titulada \emph{What is it to believe someone?} Anscombe
comienza describiendo un escenario basado en el mismo argumento, situándose así
en un escenario que plantea la misma dificultad; también en el creer a alguien
el fundamento para la combinación de las premisas en un juicio razonable parece
estar más allá de las mismas premisas y sus relaciones. En esta ocasión cada
premisa aparece atribuida a una persona distinta y la conclusión a un cuarto
personaje. El pequeño relato aparece como sigue: \citalitlar{Había tres hombres,
  $A$, $B$ y $C$, hablando en cierta aldea. $A$ dijo: ``Si ese árbol cae,
  interrumpirá el paso por el camino durante mucho tiempo.'' ``No será así si
  hay alguna máquina para remover árboles funcionando'', dijo $B$. $C$ destacó:
  ``\emph{Habrá} una, si el árbol no cae.'' El famoso sofista Eutidemo, un
  extraño en el lugar, estaba escuchando. Inmediatamente dijo: ``Les creo a
  todos. Así que infiero que el árbol caerá e interrumpirá el paso por el
  camino.'' \footnote{\cite[1]{anscombe2008faith:tobelieve} <<There were three
    men, $A$, $B$ and $C$, talking in a certain village. $A$ said ``If that tree
    falls down, it'll block the road for a long time.'' ``That's not so if
    there's a tree-clearing machine working'', said $B$. $C$ remarked ``There
    \emph{will} be one, if the tree doesn't fall down.'' The famous sophist
    Euthydemus, a stranger in the place, was listening. He immediately said ``I
    believe you all. So I infer that the tree will fall and the road will be
    blocked.''>>}}

¿En qué está mal Eutidemo? Si se evalúa la lógica del argumento antes expuesto
no aparece ninguna contradicción, sin embargo hay algo extraño en la afirmación
``les creo a todos''. Si la lógica del argumento parece permitir que la
inferencia de Eutidemo sea posible, ¿por qué suena tan extraña la posibilidad de
que les crea a todos y juzgue esa conclusión?

\subsubsection{Naturaleza de la Investigación}
Es útil recordar aquí en términos generales el modo en el que Anscombe actua en
una investigación filosófica. Wittgenstein inicialmente describió el análisis
del lenguaje bajo la concepción de que la lógica conforma el orden que está
debajo y que sostiene todo lenguaje posible. El trabajo del filósofo es analizar
el lenguaje para sacar al descubierto el orden lógico que está debajo del
lenguaje ordinario y que es la forma de la realidad. Wittgenstein abandonó esta
concepción; en Investigaciones Filosóficas exclama: \citalitlar{Cuanto más de
  cerca examinamos el lenguaje actual, más crece el conflicto entre éste y
  nuestro requisito. (Pues la pureza cristalina de la lógica no era, por
  supuesto, algo que yo hubiera \emph{descubierto}: era un requisito.) El
  conflicto se hace intolerable; el requisito llega ahora a estar en peligro de
  tornarse vacuo. --- Nos hemos situado en hielo resbaladizo donde no hay
  fricción, y así, en cierto sentido, las condiciones son ideales; pero también,
  justo por eso, no somos capaces de caminar. Queremos caminar: así que
  necesitamos \emph{fricción}. ¡De vuelta al terreno
  escarpado!\footnote{\cite[\S107]{wittgenstein1953phiinv}: <<The more closely
    we examine actual language, the greater becomes the conflict between it and
    our requirement. (For the crystalline purity of logic was, of course, not
    something I had \emph{discovered}: it was a requirement.) The conflict
    becomes intolerable; the requirement is in danger of becoming vacuous. ---
    We have got on to slippery ice where there is no friction, and so, in a
    certain sense, the conditions are ideal; but also, just because of that, we
    are unable to walk. We want to walk: so we need \emph{friction}. Back to the
    rough ground!>>}.}

Los nombres, las proposiciones, el lenguaje, no tienen una forma esencial para
ser puesta al descubierto por el análisis, sino que son familias de estructuras
que están a plena vista y que pueden ser clarificadas por medio de la
descripción\autocite[Cf.~][12]{bakerhacker2009understanding}. Wittgenstein le
\citalitinterlin{da la vuelta a la
  busqueda}\autocite[\S108]{wittgenstein1953phiinv}, y trata a la lógica no como
lo que está debajo del lenguaje para ser descubierto, sino como
\citalitinterlin{una cuadrícula que imponemos sobre los argumentos para probar y
  demostrar su validez}\footnote{\cite[12]{bakerhacker2009understanding}: <<a
  grid we impose upon arguments to test and demonstrate their validity>>}.

Descartada esta concepción sublime, Wittgenstein describe los problemas
filosóficos como formas de malentendidos o falta de entendimiento que pueden ser
disueltos por medio de descripciones de los usos de las palabras. La tarea de la
filosofía es la \citalitinterlin{clarificación gramatical que disuelve la
  perplejidad conceptual y ofrece una visión amplia o representación estudiable
  de un segmento de la gramática de nuestro
  lenguaje}\footnote{\cite[12]{bakerhacker2009understanding}: <<grammatical
  clarification that dissolves conceptual puzzlement and gives an overview of or
  surveyable representation of a segment of the grammar of our language>>}. Esta
metodología, por tanto, no pretende ofrecer teorías explicativas fruto de la
deducción o la hipótesis; tampoco pretende ofrecer tesis dogmáticas o
esencialistas. Más bien busca describir usos familiares de las palabras y
ordenarlas de tal manera que los patrones de su uso sean
estudiables\autocite[Cf.~][12]{bakerhacker2009understanding}. La metodología de
Elizabeth está basada en esto.

\subsubsection{Investigación Gramática de `creer a $x$ que $p$'.}
Anscombe pone el interés de su investigación en la forma de la expresión `creer
a $x$ que $p$'\autocite[Cf.~][2]{anscombe2008faith:tobelieve}. Su análisis se va
desenvolviendo a lo largo de la descripción de los usos de la expresión.

\citalitinterlin{Si me dijeras `Napoleón perdió la batalla de Waterloo' y te
  digo `te creo' sería una
  broma}\footnote{\cite[4]{anscombe2008faith:tobelieve}: <<If you tell me
  `Napoleon lost the battle of Waterloo' and I say `I believe you' that is a
  joke.>>}. A primer golpe `creer a $x$ que $p$' parece que significa
simplemente creer lo que alguien me dice, o creer que lo que me dice es
verdadero. Sin embargo esto no es suficiente. Puede ser que ya crea lo que
alguien me venga a decir. Puede ser que la comunicación suscite que forme mi
propio juicio acerca de la verdad comunicada, pero aquí no podría decir que
estoy creyendo al que comunica o que estoy contando con él para mi creer que
$p$.

¿Entonces creer a alguien es creer algo apoyado en el hecho de que lo ha dicho?
\citalitinterlin{Puede que se le pregunte a un testigo `¿Por qué pensó que aquel
  hombre se estaba muriendo?' y que éste responda `Porque el doctor me lo dijo'
  [\ldots] `no me hice ninguna opinión propia --- yo sólo creí al
  doctor'}\footnote{\cite[4]{anscombe2008faith:tobelieve}: <<A witness might be
  asked `Why did you think the man was dying?' and reply `Because the doctor
  told me'. If asked further what his own judgement was, he may reply `I had no
  opinion of my own --- I just believed the doctor'.>>}. Este puede ser un
ejemplo de contar con $x$ para la verdad de $p$. Esto, sin embargo, tampoco
parece ser suficiente. Puedo imaginar el caso en el que esté convencido de que
alguien a la vez cree lo opuesto a la verdad de $p$ y quiera mentirme. Según
este cálculo podría decir que creo en lo que ha dicho por el hecho de que me lo
ha dicho, pero no estaría diciendo que le creo a él.

¿Qué se puede decir del <<les creo a todos>> de Eutidemo en la cuestión
preliminar? Anscombe juzga que la exclamación no expresa simplemente una opinión
apresurada o excesiva credulidad, sino más bien suena a
locura\autocite[5]{anscombe2008faith:tobelieve}. Eutidemo no puede estar
diciendo la verdad cuando dice que les cree a todos. La expresión de $C$ da
pertinencia a lo que dice $B$, y la manera natural de entender lo que dice $B$
es como arrojando duda sobre lo que $A$ ha dicho. ¿Se puede pensar que $A$
todavía cree lo que ha dicho inicialmente? ¿Eutidemo puede creer a $A$ sin saber
cuál es su reacción a lo que $B$ y $C$ han dicho? Entonces Anscombe concluye,
\citalitinterlin{Para creer a $N$ uno debe creer que $N$ mismo cree lo que está
  diciendo}\footnote{\cite[5]{anscombe2008faith:tobelieve}: <<To believe $N$ one
  must believe that $N$ himself believes what he is saying>>.} Creer a $N$ sin
saber si $N$ cree lo que dice le suena a Elizabeth como una locura.

En este punto queda expuesta a la luz una segunda creencia involucrada en el
creer a $x$ que $p$. Anscombe fija su atención en esto. Creer a $x$ que $p$
conlleva otras creencias, éstas son presuposiciones implicadas en llegar a
plantearse si creer o no. En primer lugar, si se cree a alguien, tiene que ser
el caso que se cree que una comunicación es de
alguien\autocite[Cf.~][6]{anscombe2008faith:tobelieve}. Esta presuposición no
parece tan problemática si se piensa en las ocasiones en las que creemos a
alguien que es percibido. Sin embargo tiene más profundidad si se considera que
con frecuencia recibimos la comunicación sin que esté presente el que habla,
como cuando leemos un libro\autocite[Cf.~][5]{anscombe2008faith:tobelieve}.

Se puede imaginar aquí una situación problemática. Supongamos que alguien recibe
una carta en la que el autor no es el comunicador ostensible o aparente, es
decir, quien firma la carta no es quien la ha escrito. ¿Se puede decir que el
que recibe la carta cree o descree al autor o al comunicador ostensible? Creer
al autor, afirma Anscombe, conlleva un tipo de juicio y especulación que no son
mediaciones ordinarias en el creer a
alguien\autocite[Cf.~][7]{anscombe2008faith:tobelieve}. Para decir que creo al
autor tendría que discernir que la comunicación que viene bajo otro nombre es
realmente de esta otra persona que además me quiere decir esto.

Respecto de la posibilidad de decir que se cree al comunicador ostensible
Anscombe distingue entre un comunicador ostensible que exista o no. Ante una
comunicación que viene de parte de un comunicador aparente que no existe alguien
puede responder diciendo que cree o descree al comunicador aparente, pero la
decisión de decir esto ---dice Anscombe--- \citalitinterlin{es una decisión de
  dar a estos verbos un uso `intencional', como el verbo `ir
  tras'}\footnote{\cite[7]{anscombe2008faith:tobelieve}: <<is a decision to give
  those verbs an `intentional' use like the verb `to look for'>> Anscombe hace
  una descripción más detallada de lo que significa usar un verbo
  intencionalmente en \emph{The Intentionality of Sensation}. Ahí propone que un
  verbo es usado intencionalmente cuando tiene como objeto directo un `objeto
  intencional' (`objeto' no en el sentido material, sino de finalidad).}. Esto
lo ilustra añadiendo: \citalitlar{Y así uno podría hablar de alguien como
  creyendo al dios (Apolo, digamos), cuando consultó el oráculo del dios -- sin
  que por esto uno estuviera implicando que uno mismo cree en la existencia del
  dios. Todo lo que queremos es que necesitamos saber lo que es llamado que el
  dios le diga algo\footnote{\cite[7]{anscombe2008faith:tobelieve}: <<And so we
    might speak of someone as believing the god (Apollo, say), when he consulted
    the oracle of the god -- without thereby implying that one believed in the
    existence of the gos oneself. All we want is that we should know what is
    called the god's telling him something>>}.} `Creer' usado aquí
intencionalmente viene a decir que se busca o se desea creer a $x$ (Apolo en
este caso) cuando se recibe aquello que alguien entiende como una comunicación
suya.

En el caso de que el comunicador ostensivo sí exista, la noción de creerle
manifiesta una cierta oscilación. Una tercera persona podría decir que `aquel,
pensando que $N$ dijo esto, le creyo', o el comunicador aparente puede decir
`veo que pensaste que fui yo quien dijo esto y me creiste', sin embargo, si el
que ha recibido la comunicación dijera `naturalmente te creí', el comunicador
aparente podría contestar `ya que no lo he dicho yo, no me estabas creyendo a
mi'\autocite[Cf.~][8]{anscombe2008faith:tobelieve}.

Estas consideraciones llevan a Anscombe a distinguir entre el que habla en una
comunicación y el productor inmediato de la
comunicación\autocite[Cf.~][8]{anscombe2008faith:tobelieve}. Éste puede ser
cualquiera que pase hacia adelante alguna comunicación, un maestro o mensajero,
o un interprete o traductor; éste es \citalitinterlin{el productor inmediato de
  aquello que se entiende, o incluye una reclamación interna de ser entendido
  como una comunicación de $NN$}\footnote{\cite[8]{anscombe2008faith:tobelieve}:
  <<we can speak of the immediate producer of what is taken, or makes an
  internal claim to be taken, as a communication from $NN$>>}. Si digo que creo
a un intérprete estoy afirmando que creo lo que ha dicho su principal, y mi
contar con el intérprete consiste en la creencia de que ha reproducido lo que
aquel ha dicho. En este sentido el intérprete no le falta rectitud si dice algo
que no es verdadero pero no ha representado falsamente lo que ha dicho su
principal. Por el contrario, al maestro sí le faltaría rectitud si lo que dice
no es verdadero. Cuando se cree al maestro, aún en el caso que no sea de ninguna
manera autoridad original de lo que comunica, se le cree a él sobre lo que
transmite. Para Anscombe no es necesario que cuando se cree a alguien se le
trate como una autoridad
original\autocite[Cf.~][5]{anscombe2008faith:tobelieve}. En esto el ejemplo del
maestro como distinto del intérprete es ilustrativo. Un maestro puede conocer lo
que enseña porque lo ha recibido de alguna tradición de información y al
transmitir lo que enseña se le está creyendo a él.

Asoma aquí otro aspecto relacionado con esta presuposición. Al creer que una
comunicación es de alguien se cree a una persona que puede tener distintos
grados de autoridad sobre lo que dice. El maestro del que se ha hablado antes
podría afirmar <<Leonardo da Vinci dibujó diseños para una máquina voladora>> y
en esto no es para nada una autoridad
original\autocite[Cf.~][6]{anscombe2008faith:tobelieve}. Conoce esto porque lo
ha escuchado, incluso si ha visto los diseños; y aún cuando los hubiera
descubierto él mismo tendría que haber contado con alguna información recibida
de que esos diseños que ve son de Leonardo. En este caso sí seria una autoridad
original en notar que estos diseños que ha escuchado que son de Leonardo son de
máquinas voladoras. Anscombe explica la distinción diciendo:
\citalitlar{[Alguien] es \emph{una} autoridad original en aquello que él mismo
  ha hecho y visto y oido: digo \emph{una} autoridad original porque sólo quiero
  decir que él mismo sí contribuye algo, es algún tipo de testigo por ejemplo,
  en lugar de alguien que sólo transmite información recibida. Pero su informe
  de aquello de lo que es testigo es con frecuencia [\ldots] fuertemente
  influenciado o más bien casi del todo formado por la información que \emph{él}
  ha recibido\footnote{\cite[5]{anscombe2008faith:tobelieve}: <<He is \emph{an}
    original authority on what he himself has done and seen and heard: I say
    \emph{an} original authority because I only mean that he does himself
    contribute something, e.g. is in some sort a witness, as oposed to one who
    only transmits information received. But his account of what he is a witness
    to is very often [\ldots] heavily affected or ratherl all but completely
    formed by what information \emph{he} had received.>>}.} Además de ser
\emph{una} autoridad original sobre algún hecho una persona puede ser una
autoridad \emph{totalmente} original. Si la distinción entre alguien que no es
una autoridad original y alguien que sí lo es ha sido descrita como la
contribución de algo propio que junto con la información recibida permite
construir un informe, lo particular de una autoridad totalmente original es que
no se apoya en ninguna información recibida para construir su informe de los
hechos. Anscombe no entiende el lenguaje como información recibida. Pone como
ejemplo de informe de una autoridad totalmente original a alguien que dice `esta
mañana comí una manzana' y dice: \citalitlar{si él está en la situación usual
  entre nosotros, el sabe lo que una manzana es --- es decir, puede reconocer
  una. Así que aún cuando se le ha `enseñado el concepto' al aprender a usar el
  lenguaje en la vida ordinaria, no cuento esto como un caso de depender en
  información recibida.\footnote{\cite[6]{anscombe2008faith:tobelieve}: <<if he
    is in the situation usual among us, he knows what an apple is --- i.e. can
    recognise one. So though he was `taught the concept' in learning to use
    language in everyday life, I do not count that as a case of reliance on
    information received.>>}}

Hasta aquí se ha visto que el creer a $x$ que $p$ implica otras creencias que
son presuposiciones a la pregunta sobre si se cree o se descree a alguien y se
ha descrito lo que tiene que ver con la creencia de que una comunicación viene
de alguien. Anscombe examina otras presuposiciones más. También tiene que ser el
caso que creamos que por la comunicación, la persona que habla quiere decir
\emph{esto}.
