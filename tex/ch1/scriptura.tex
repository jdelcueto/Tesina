\section{El testimonio en la Escritura: la Revelación como acción testimonial de Dios}

\subsection{La Revelación descrita con estrucura testimonial}

La Iglesia de hoy, como María, conserva el Evangelio meditándolo en su corazón (Cf. Lc 2,19). Así está presente en el centro de la comunidad creyente el anuncio de Cristo vivo como fundamento de su esperanza en cada etapa de la historia. Este motivo de esperanza conservado es también compartido y expresado, según la enseñanza del apóstol: \blockquote[][\,(1Pe 3,15)]{glorificad a Cristo en vuestros corazones, dispuestos siempre a dar explicación a todo el que os pida una razón de vuestra esperanza}.

Este Evangelio atesorado como fundamento en el centro de la vida de la comunidad eclesial, así como Buena Nueva proclamada y transmitida en el tiempo y en el mundo puede ser comprendido como tres testimonios que son uno: \enquote*{Palabra vivida en el Espíritu}\footnote{\Cite[Cf.][110]{latourelle1975et}: \enquote{Car c'est L'Esprit qui posse l'Eglise à poursuivre son oeuvre d'évangelisation; c'est l'Esprit qui inspire la foi, la nourrit et l'approfondit. C'est l'Esprit qui relie entre eux ces trois témoignages qui n'en font qu'un: celui de la parole-vécue-dans-l'Esprit. Par son témoignage, l'Esprit intériorise le témoignage extérieur de la Bonne Nouvelle du salut en Jésus-Christ et le porte à l'accomplissement de la foi, qui est la réponse d'amour de l'humanité à l'appel d'amour du Père par le Christ.} Ver también \Cite[582]{ninot2009tf}, donde este triple testimonio sirve para orientar la reflexión sobre el testimonio como vía empírica de la credibilidad de la Iglesia.}.

La Evangelización puede ser entendida en este sentido como testimonio de la `palabra de vida' (1Jn 1,1) que los apóstoles anuncian como testigos de lo que han contemplado y palpado. Es también el testimonio de los cristianos que, acogiendo esta palabra, la viven, poniendo por obra lo que ella enseña. Es además testimonio del Espíritu Santo que interioriza el testimonio externo de la Buena Noticia y lo lleva al cumplimiento de la fe en cada persona\footcite[Cf.][110]{latourelle1975et}. Es el Espíritu el que santifica y fecunda la acción de los cristianos, es tambíen el que impulsa y sostiene la acción de la Iglesia; es el Espíritu el que inspira la fe, la nutre y la profundiza\footcite[Cf.][110]{latourelle1975et}.

Este dinamismo fundamental que puede encontrarse vivo hoy en la comunidad de la Iglesia ha actuado en ella desde su origen y le ha acompañado en cada época. Según esto es posible valorar lo que se transmite en la tradición eclesial como la perpetuación de la actividad de Cristo y los apóstoles, que es a su vez proyección del testimonio divino\footnote{\Cite[Cf.][573]{ninot2009tf}: \enquote{el testimonio divino se proyecta luego en el apostólico y se perpetúa en el testimonio eclesial. Por eso, el testimonio es revelación en la actividad de Cristo y de los apóstoles y es transmisión de la revelación en la tradición eclesial}.}.

En la actividad de Cristo el testimonio divino queda proyectado como interpelación a la libertad realizada por la identidad propia de Jesús: \blockquote[][\,(Jn 4,10)]{Si conocieras el don de Dios y quién es el que te dice <<dame de beber>> le pedirías tu, y él te daría agua viva}; \blockquote{``¿Crees tú en el Hijo del hombre?''\textelp{} ``¿Y quién es, Señor, para que crea en él?''\textelp{} ``Lo estás viendo: el que te está hablando, ese es''} (Jn 9,35-37). En la actividad apostólica, el testimonio divino sigue interpelando la libertad humana como manifestación de Jesús Resucitado. Los apóstoles actúan como testigos de los acontecimientos de la Pascua de Jesús y su valor salvífico\footcite[Cf.][576]{ninot2009tf} y este testimonio es descrito como acción del Espíritu que impulsa la tarea apostólica y que da nueva vida a los que acogen el anuncio de la Buena Noticia.

Puede encontrarse un ejemplo en el testimonio de Felipe. El apóstol sale más allá de Jerusalén hacia Samaria, y todavía llega más lejos, al compartir la Buena Noticia de Jesús con un extranjero etíope: \blockquote[][\,(Hch 8, 29-39)]{El Espíritu dijo a Felipe: <<Acércate y pégate a la carroza>>. Felipe se acercó corriendo, le oyó leer el profeta Isaías, y le preguntó: <<¿Entiendes lo que estás leyendo?>>. Contestó: <<¿Y cómo voy a entenderlo si nadie me guía?>>. E invitó a Felipe a subir y a sentarse con él. El pasaje de la Escritura que estaba leyendo era este: \emph{Como cordero fue llevado al matadero, como oveja muda ante el esquilador, así no abre su boca. En su humillación no se le hizo justicia. ¿Quién podrá contar su descendencia? Pues su vida ha sido arrancada de la tierra.} El eunuco preguntó a Felipe: <<Por favor, ¿de quién dice esto el profeta?; ¿de él mismo o de otro?>>. Felipe se puso a hablarle y, tomando pie de este pasaje, le anunció la Buena Nueva de Jesús. Continuando el camino, llegaron a un sitio donde había agua, y dijo el eunuco: <<Mira, agua. ¿Qué dificultad hay en que me bautice?>>. Mandó parar la carroza, bajaron los dos al agua, Felipe y el eunuco, y lo bautizó. Cuando salieron del agua, el Espíritu del Señor arrebató a Felipe. El eunuco no volvió a verlo, y siguió su camino lleno de alegría}. Además de ser ejemplo de la actividad apostólica, este relato puede servir como síntesis del modo en que la categoría del testimonio está presente en la Escritura.

El testimonio comienza con la iniciativa de Dios mismo que impulsa tanto la palabra profética del Antiguo Testamento como el anuncio apostólico del Nuevo Testamento. Esta iniciativa de Dios tiende hacia el testimonio de la Palabra definitiva del Padre que es Cristo resucitado. En aquellos que creen en el testimonio de Dios se engendra alegría y vida nueva. En palabras de R. Latourelle: \blockquote[{\Cite[1530]{latourelle2000testimonio}}.]{En el trato de las tres personas divinas con los hombres existe un intercambio de testimonios que tiene la finalidad de proponer la revelación y de alimentar la fe. Son tres los que revelan o dan testimonio, y esos tres no son más que uno. Cristo da testimonio del Padre, mientras que el Padre y el Espíritu dan testimonio del Hijo. Los apóstoles a su vez dan testimonio de lo que han visto y oído del verbo de la vida. Pero su testimonio no es la comunicación de una ideología, de un descubrimiento científico, de una técnica inédita, sino la proclamación de la salvación prometida y finalmente realizada}.

De este modo el anuncio del apóstol Felipe sirve aquí como un ejemplo específico del testimonio, que ilustra una noción que \blockquote[{\Cite[109]{prades2015testimonio}}.]{atraviesa toda la Escritura y se corresponde con la estructura misma de la revelación}. El testimonio está presente a lo largo de la Escritura junto a otras categorías como pueden ser la de `alianza', `palabra', `paternidad' o `filiación', como parte del \blockquote[{\Cite[1523]{latourelle2000testimonio}}.]{grupo de analogías empleadas por la Escritura para introducir al hombre en las riquezas del misterio divino}.

Esta clave servirá para dar enfoque a un examen sobre la categoría del testimonio en la Escritura. ¿Qué nos dice el Antiguo y el Nuevo Testamento de la revelación como acto testimonial de Dios? Esta pregunta supone que la revelación comparte los rasgos de la actividad humana que es el testimonio, sin embargo, como Latourelle adiverte: \blockquote[{\Cite[1526]{latourelle2000testimonio}}.]{globalmente se puede decir que el testimonio bíblico asume, pero al mismo tiempo exalta hasta sublimarlos, los rasgos del testimonio humano}.

Cabe añadir una última consideración. La revelación de Dios entendida como acto testimonial suyo tiene como expresión definitiva el misterio pascual de Cristo\footnote{\Cite[128]{prades2015testimonio}: \enquote{el misterio pascual al cual tiende toda la existencia terrena de Cristo, constituye el acto testimonial por excelencia de Dios}.}. Este misterio ocupa el lugar principal en el testimonio bíblico: \blockquote[{\Cite[404]{ninot2009tf}}.]{la Resurrección como ``final'' de la unicidad del acontecimiento de Jesucristo, encarnado, muerto y resucitado, subraya específicamente la definitividad de la existencia humana salvada por Dios en la carne de Jesús de Nazaret, ya que la autocomunicación de Dios ha alcanzado su palabra última en la Resurrección de Jesucristo, y por eso es prenda de la resurrección de todos los hombres}. Como tal, parece justo tratar el testimonio que es el misterio pascual en un apartado propio. Y será este precisamente el punto de partida para la descripción de la categoría del testimonio en la Escritura.

En el Antiguo Testamento la Revelación también puede ser comprendida como el `intercambio de testimonios' que existe en el trato de Dios con los hombres\footcite[Cf.][1530]{latourelle2000testimonio}. También aquí la acción testimonial divina se despliega de diversos modos. En la vida del pueblo de la alianza YHWH da testimonio de sí a través de la creación, la ley y, de modo eminente, en personas elegidas y enviadas por él\footcite[Cf.][114-115]{prades2015testimonio}. Esta manifestación divina implica como testigo al mismo pueblo, hacia quien ha sido dirigida la voz del Señor.

La literatura sapiencial recoge la profundización en la experiencia de Dios que ha tenido el pueblo de Israel. En ella se describe el acceso posible al conocimiento de Dios a partir de los bienes visibles o de sus obras: \blockquote[][\,(Sab 13,1-5)]{Son necios por naturaleza todos los hombres que han ignorado a Dios y no han sido capaces de conocer al que es a partir de los bienes visibles, ni de reconocer al artífice fijándose en sus obras, sino que tuvieron por dioses al fuego, al viento, al aire ligero, a la bóveda estrellada, al agua impetuosa y a los luceros del cielo, regidores del mundo. Si, cautivados por su hermosura, los creyeron dioses, sepan cuánto los aventaja su Señor, pues los creó el mismo autor de la belleza. Y si los asombró su poder y energía, calculen cuánto más poderoso es quien los hizo, pues por la grandeza y hermosura de las criaturas se descubre por analogía a su creador}.

El Dios que puede ser reconocido por analogía en el asombro y belleza de las criaturas es un Dios personal que concede sabiduría al piadoso: \blockquote[][\,(Eclo 43,32-3)]{Aún quedan misterios mucho más grandes: tan solo hemos visto algo de sus obras. Porque el Señor lo ha hecho todo y a los piadosos les ha dado la sabiduría}. Esta sabiduría es justicia y raíz de inmortalidad: \blockquote[][\,(Sab 15,1-3)]{Pero tú, Dios nuestro, eres bueno y fiel, eres paciente y todo lo gobiernas con misericordia. Aunque pequemos, somos tuyos y reconocemos tu poder, pero no pecaremos, sabiendo que te pertenecemos. Conocerte a ti es justicia perfecta y reconocer tu poder es la raíz de la inmortalidad}. En este sentido la misma creación es acto testimonial de Dios donde se comunica su misterio y la vida que Él ofrece.
