\section{El testimonio en la Escritura: la Revelación como acción testimonial de Dios}

\subsection{La Revelación descrita con estrucura testimonial}

La Iglesia de hoy, como María, conserva el Evangelio meditándolo en su corazón (Cf.~Lc 2,19). Así está presente en el centro de la comunidad creyente el anuncio de Cristo vivo como fundamento de su esperanza en cada etapa de la historia. Este motivo de esperanza conservado es también compartido y expresado, según la enseñanza del apóstol: \blockquote[][\,(1Pe 3,15)]{glorificad a Cristo en vuestros corazones, dispuestos siempre a dar explicación a todo el que os pida una razón de vuestra esperanza}.

Este Evangelio atesorado como fundamento en el centro de la vida de la comunidad eclesial, así como Buena Nueva proclamada y transmitida en el tiempo y en el mundo puede ser comprendido como tres testimonios que son uno: \enquote*{Palabra vivida en el Espíritu}\footnote{\cite[Cf.~][110]{latourelle1975et}: \enquote{Car c'est L'Esprit qui posse l'Eglise à poursuivre son oeuvre d'évangelisation; c'est l'Esprit qui inspire la foi, la nourrit et l'approfondit. C'est l'Esprit qui relie entre eux ces trois témoignages qui n'en font qu'un: celui de la parole-vécue-dans-l'Esprit. Par son témoignage, l'Esprit intériorise le témoignage extérieur de la Bonne Nouvelle du salut en Jésus-Christ et le porte à l'accomplissement de la foi, qui est la réponse d'amour de l'humanité à l'appel d'amour du Père par le Christ.} Ver también \cite[582]{ninot2009tf} donde este triple testimonio sirve para orientar la reflexión sobre el testimonio como vía empírica de la credibilidad de la Iglesia.}.

La Evangelización puede ser entendida en este sentido como testimonio de la `palabra de vida' (1Jn 1,1) que los apóstoles anuncian como testigos de lo que han contemplado y palpado. Es también el testimonio de los cristianos que, acogiendo esta palabra, la viven, poniendo por obra lo que ella enseña. Es además testimonio del Espíritu Santo que interioriza el testimonio externo de la Buena Noticia y lo lleva al cumplimiento de la fe en cada persona\autocite[Cf.~][110]{latourelle1975et}. Es el Espíritu el que santifica y fecunda la acción de los cristianos, es tambíen el que impulsa y sostiene la acción de la Iglesia; es el Espíritu el que inspira la fe, la nutre y la profundiza\autocite[Cf.~][110]{latourelle1975et}.

Este dinamismo fundamental que puede encontrarse vivo hoy en la comunidad de la Iglesia ha actuado en ella desde su origen y le ha acompañado en cada época. Según esto es posible valorar lo que se transmite en la tradición eclesial como la perpetuación de la actividad de Cristo y los apóstoles, que es a su vez proyección del testimonio divino\footnote{\cite[Cf.~][573]{ninot2009tf}: \enquote{el testimonio divino se proyecta luego en el apostólico y se perpetúa en el testimonio eclesial. Por eso, el testimonio es revelación en la actividad de Cristo y de los apóstoles y es transmisión de la revelación en la tradición eclesial.}}.

En la actividad de Cristo el testimonio divino queda proyectado como interpelación a la libertad realizada por la identidad propia de Jesús: \blockquote[][\,(Jn 4,10)]{Si conocieras el don de Dios y quién es el que te dice <<dame de beber>> le pedirías tu, y él te daría agua viva}; \blockquote{``¿Crees tú en el Hijo del hombre?''\textelp{} ``¿Y quién es, Señor, para que crea en él?''\textelp{} ``Lo estás viendo: el que te está hablando, ese es''} (Jn 9,35-37). En la actividad apostólica, el testimonio divino sigue interpelando la libertad humana como manifestación de Jesús Resucitado. Los apóstoles actúan como testigos de los acontecimientos de la Pascua de Jesús y su valor salvífico\autocite[Cf.][576]{ninot2009tf} y este testimonio es descrito como acción del Espíritu que impulsa la tarea apostólica y que da nueva vida a los que acogen el anuncio de la Buena Noticia.

Puede encontrarse un ejemplo en el testimonio de Felipe. El apóstol sale más allá de Jerusalén hacia Samaria, y todavía llega más lejos, al compartir la Buena Noticia de Jesús con un extranjero etíope: \blockquote[][\,(Hch 8, 29-39)]{El Espíritu dijo a Felipe: <<Acércate y pégate a la carroza>>. Felipe se acercó corriendo, le oyó leer el profeta Isaías, y le preguntó: <<¿Entiendes lo que estás leyendo?>>. Contestó: <<¿Y cómo voy a entenderlo si nadie me guía?>>. E invitó a Felipe a subir y a sentarse con él. El pasaje de la Escritura que estaba leyendo era este: \emph{Como cordero fue llevado al matadero, como oveja muda ante el esquilador, así no abre su boca. En su humillación no se le hizo justicia. ¿Quién podrá contar su descendencia? Pues su vida ha sido arrancada de la tierra.} El eunuco preguntó a Felipe: <<Por favor, ¿de quién dice esto el profeta?; ¿de él mismo o de otro?>>. Felipe se puso a hablarle y, tomando pie de este pasaje, le anunció la Buena Nueva de Jesús. Continuando el camino, llegaron a un sitio donde había agua, y dijo el eunuco: <<Mira, agua. ¿Qué dificultad hay en que me bautice?>>. Mandó parar la carroza, bajaron los dos al agua, Felipe y el eunuco, y lo bautizó. Cuando salieron del agua, el Espíritu del Señor arrebató a Felipe. El eunuco no volvió a verlo, y siguió su camino lleno de alegría}. Además de ser ejemplo de la actividad apostólica, este relato puede servir como síntesis del modo en que la categoría del testimonio está presente en la Escritura.

El testimonio comienza con la iniciativa de Dios mismo que impulsa tanto la palabra profética del Antiguo Testamento como el anuncio apostólico del Nuevo Testamento. Esta iniciativa de Dios tiende hacia el testimonio de la Palabra definitiva del Padre que es Cristo resucitado. En aquellos que creen en el testimonio de Dios se engendra alegría y vida nueva. En palabras de R. Latourelle: \blockquote[{\cite[1530]{latourelle2000testimonio}}]{En el trato de las tres personas divinas con los hombres existe un intercambio de testimonios que tiene la finalidad de proponer la revelación y de alimentar la fe. Son tres los que revelan o dan testimonio, y esos tres no son más que uno. Cristo da testimonio del Padre, mientras que el Padre y el Espíritu dan testimonio del Hijo. Los apóstoles a su vez dan testimonio de lo que han visto y oído del verbo de la vida. Pero su testimonio no es la comunicación de una ideología, de un descubrimiento científico, de una técnica inédita, sino la proclamación de la salvación prometida y finalmente realizada}.

De este modo el anuncio del apóstol Felipe sirve aquí como un ejemplo específico del testimonio, que ilustra una noción que \blockquote[{\cite[109]{prades2015testimonio}}]{atraviesa toda la Escritura y se corresponde con la estructura misma de la revelación}. El testimonio está presente a lo largo de la Escritura junto a otras categorías como pueden ser la de `alianza', `palabra', `paternidad' o `filiación', como parte del \blockquote[{\cite[1523]{latourelle2000testimonio}}]{grupo de analogías empleadas por la Escritura para introducir al hombre en las riquezas del misterio divino}.

Esta clave servirá para dar enfoque a un examen sobre la categoría del testimonio en la Escritura. ¿Qué nos dice el Antiguo y el Nuevo Testamento de la revelación como acto testimonial de Dios? Esta pregunta supone que la revelación comparte los rasgos de la actividad humana que es el testimonio, sin embargo, como Latourelle adiverte: \blockquote[{\cite[1526]{latourelle2000testimonio}}]{globalmente se puede decir que el testimonio bíblico asume, pero al mismo tiempo exalta hasta sublimarlos, los rasgos del testimonio humano}.

Cabe añadir una última consideración. La revelación de Dios entendida como acto testimonial suyo tiene como expresión definitiva el misterio pascual de Cristo\footnote{\cite[128]{prades2015testimonio}: \enquote{el misterio pascual al cual tiende toda la existencia terrena de Cristo, constituye el acto testimonial por excelencia de Dios.}}. Este misterio ocupa el lugar principal en el testimonio bíblico: \blockquote[{\cite[404]{ninot2009tf}}]{la Resurrección como ``final'' de la unicidad del acontecimiento de Jesucristo, encarnado, muerto y resucitado, subraya específicamente la definitividad de la existencia humana salvada por Dios en la carne de Jesús de Nazaret, ya que la autocomunicación de Dios ha alcanzado su palabra última en la Resurrección de Jesucristo, y por eso es prenda de la resurrección de todos los hombres}. Como tal, parece justo tratar el testimonio que es el misterio pascual en un apartado propio. Y será este precisamente el punto de partida para la descripción de la categoría del testimonio en la Escritura.

\subsection{El testimonio en el misterio y anuncio pascual}

\enquote*{Cristo ha resucitado} (Cf.~1Tes 4,15; 1Cor 15,12-20; Rom 6,4) es la confesión que está en el núcleo del anuncio más primitivo del evangelio\autocite[Cf.][403]{ninot2009tf}. Creer en esta noticia conlleva acoger la manifestación más plena de la Revelación y la motivación más definitiva para creer. En este sentido: \blockquote[{\cite[405]{ninot2009tf}}]{La Resurrección de Jesús mirada desde la perspectiva de la teología fundamental presupone un estatuto epistemológico peculiar, puesto que es el punto culminante y objeto de la Revelación y, a su vez, es su acreditación suprema y máximo motivo de credibilidad, tal como recuerda el texto citado de Pablo ``si Cristo no ha resucitado, nuestra predicación es vana y vana es nuestra fe'' (1 Cor 15,14)}.

El misterio pascual no aparece como desconectado del conjunto de la vida y misión de Jesús, sino que hacia él tienden sus obras y palabras desde el comienzo. Cristo pasó por el mundo haciendo el bien, como testimonio de la bondad de Dios, y esta acción va orientada a ese punto culminante que es su pasión, muerte y resurrección; \blockquote[{\cite[127]{prades2015testimonio}}]{el testimonio que Jesús va ofreciendo durante su vida pública le va a reclamar una entrega definitiva a favor de los que lo han acogido y frente a la resistencia que ha generado en quienes le rechazan}.

A lo largo de este camino Jesús manifiesta su confianza en el Padre: \blockquote[][\,(Jn 11,41b-42a)]{Padre, te doy gracias porque me has escuchado; yo sé que tu me escuchas siempre}; esta relación queda afirmada plenamente ante la pasión como confianza puesta en su voluntad: \blockquote[][\,(Lc 22,42)]{Padre \textelp{} que no se haga mi voluntad, sino la tuya}. De este modo en el misterio pascual queda atestiguada la plena unidad de Cristo con el Padre, en la mayor confianza imaginable\autocite[Cf.~][127]{prades2015testimonio}.

A lo largo de su misión, Cristo dio testimonio del amor del Padre \blockquote[][\,(Jn 13,1)]{habiendo amado a los suyos que estaban en el mundo\ldots}. En el misterio pascual, donde \blockquote[][\,(ibíd.)]{los amó hasta el extremo}, queda confirmado definitivamente como testigo del Padre. Con su entrega ofrece el testimonio pleno del amor salvador del Padre: \blockquote[][\,(Jn 3,16)]{Porque tanto amó Dios al mundo, que entregó a su Unigénito, para que todo el que cree en él no perezca, sino que tenga vida eterna}.

A lo largo de su vida, Cristo también es testigo de la necesidad del camino salvífico ofrecido como decisión trinitaria libre e irrevocable de redimir a los hombres\autocite[Cf.~][128]{prades2015testimonio}. \blockquote[][\,(Mc 8,31)]{El hijo del hombre tiene que padecer mucho, ser reprobado por los ancianos, sumos sacerdotes y escribas, ser ejecutado y resucitar a los tres días}. Este testimonio de la voluntad divina es comprendido por los discípulos a la luz del Resucitado; \blockquote[][\,(Cf.~Lc 24,45-47a)]{les abrió el entendimiento para comprender las Escrituras \textelp{} ``así está escrito: el Mesías padecerá, resucitará de entre los muertos al tercer día y en su nombre se proclamará la conversión''}.

La intencionalidad de este testimonio que Jesús ofrece a lo largo de su vida hasta llegar al acto testimonial definitivo de Dios al mundo que es el misterio pascual aparece con claridad en la respuesta de Cristo a Pilato antes de la Pasión: \blockquote[][\,(Jn 18,37)]{Yo para esto he nacido y para esto he venido al mundo: para dar testimonio de la verdad. Todo el que es de la verdad escucha mi voz}. En su vida pública y en su misión Cristo ha actuado como profeta que anuncia la verdad; da a conocer al Padre, a quien nadie ha visto nunca, pero que el Hijo sí conoce\footnote{Cf.~ Jn 1,18; Ver también \cite[28]{ratzinger2007jdenaz}: \enquote{En Jesús se cumple la promesa del nuevo profeta. En Él se ha hecho plenamente realidad lo que en Moisés era sólo imperfecto: Él vive ante el rostro de Dios no sólo como amigo, sino como Hijo; vive en la más íntima unidad con el Padre.}}. En el misterio pascual Jesús se manifiesta como verdadero profeta, acreditado por el hecho mismo de la Resurrección donde se ha realizado en él mismo lo que ha revelado y prometido\autocite[128]{prades2015testimonio}.

La resurrección de Cristo no sólo acredita su propio testimonio, sino que sostiene el testimonio apostólico. Si Cristo no ha resucitado sería vana cualquier argumentación, sin embargo, Jesús es `el Viviente', estuvo muerto, pero vive por los siglos de los siglos (Cf.~Ap 1,17-18).

Los apóstoles son testigos de la vida de Cristo, de sus palabras y acciones, muerte y resurrección. De tal modo, son testigos en continuidad con el testimonio de Cristo. El testimonio apostólico es un anuncio de estos hechos que ellos conocen y cuyo valor han reconocido por la fe. Así Pedro proclama estas cosas el día de Pentecostés: \blockquote[][\,(Hch 2,32)]{A este Jesús lo resucitó Dios, de lo cual todos nosotros somos testigos}. El apóstol es testigo en la fe sobre un acontecimiento enraizado en la historia\autocite[Cf.~][402; 406]{ninot2009tf}.

Así mismo es presentado el testimonio de Pedro en casa de Cornelio donde el centurión y todos lo que lo acompañaban esperaban reunidos para escuchar lo que el Señor quisiera comunicarles por medio del apóstol. Pedro, comprendiendo que la verdad de Dios no hace acepción de personas, narra los hechos que él bien conoce: \blockquote[][\,(Hch 10,37-41)]{Vosotros conocéis lo que sucedió en toda Judea, comenzando por Galilea, después del bautismo que predicó Juan. Me refiero a Jesús de Nazaret, ungido por Dios con la fuerza del Espíritu Santo, que pasó haciendo el bien y curando a todos los oprimidos por el diablo, porque Dios estaba con él. Nosotros somos testigos de todo lo que hizo en la tierra de los judíos y en Jerusalén. A este lo mataron, colgándolo de un madero. Pero Dios lo resucitó al tercer día y le concedió la gracia de manifestarse, no a todo el pueblo, sino a los testigos designados por Dios: a nosotros, que hemos comido y bebido con él después de su resurrección de entre los muertos}. Este testimonio de los hechos es iluminado en su sentido profundo porque Pedro conoce a Jesús a quien los apóstoles y el pueblo vieron y escucharon, y que es ahora juez de vivos y muertos: \blockquote[][\,(Hch 10,42-43)]{Nos encargó predicar al pueblo, dando solemne testimonio de que Dios lo ha constituido juez de vivos y muertos. De él dan testimonio todos los profetas: que todos los que creen en él reciben, por su nombre, el perdón de los pecados}.

El apóstol entiende estos hechos y su alcance religioso y salvífico interpretándolos en continuidad con la voluntad de Dios manifestada en su acción en favor del pueblo judío a quién habló por medio de los profetas; voluntad hecha manifiesta en \blockquote[][\,(Hch 2,22)]{Jesús el Nazareno, varón acreditado por Dios ante vosotros con los milagros, prodigios y signos que Dios realizó por medio de él, como vosotros mismos sabéis}.

Este anuncio es experiencia del Resucitado que comió y bebió con ellos; él mismo se apareció a los que él quiso dando testimonio de su resurrección. \blockquote[{\cite[129]{prades2015testimonio}}]{Cristo glorificado manifiesta su verdad a los que él quiere y esta manifestación es simultáneamente testimonio de su identidad y testimonio de que él es la Vida (1Jn 5,11)}.

El misterio divino que se manifiesta en la Pascua de Jesús no deja de expresarse en el anuncio pascual realizado por los apóstoles. Ellos son testigos de un hecho enraizado en la historia, que tiene un alcance religioso y salvífico y que es interpretado desde la voluntad de Dios manifestada en los hechos y palabras de Cristo. Sin las obras que Jesús realizó, el testimonio apostólico se derrumba, no existe\autocite[Cf.][1529]{latourelle2000testimonio}. Sin la vida y obra, muerte y resurrección de Jesús \blockquote[][\,(1Cor 15,15)]{resultamos unos falsos testigos de Dios, porque hemos dado testimonio contra él, diciendo que ha resucitado a Cristo, a quien no ha resucitado}.

En Cristo, testigo acreditado por su Resurrección, encuentra su cumplimiento la promesa hecha al pueblo de Israel: \blockquote[Dt 18,15 y Hch 3,22; {\cite[Cf.~][24-29]{ratzinger2007jdenaz}}]{El Señor, tu Dios, te suscitará de entre los tuyos, de entre tus hermanos, un profeta como yo. A él lo escucharéis}. Así como el misterio pascual y su anuncio no están desconectados de la vida de Cristo, tampoco lo están de la acción salvadora de Dios en el AT. Como veremos, el misterio divino se manifiesta a un pueblo que también está llamado a dar testimonio, reconociendo desde la confianza en Dios el valor salvífico de los sucesos de su historia.

\subsection{La acción testimonial de Dios en el Antiguo Testamento}

En el Antiguo Testamento la Revelación también puede ser comprendida como el `intercambio de testimonios' que existe en el trato de Dios con los hombres\autocite[Cf.~][1530]{latourelle2000testimonio}. También aquí la acción testimonial divina se despliega de diversos modos. En la vida del pueblo de la alianza YHWH da testimonio de sí a través de la creación, la ley y, de modo eminente, en personas elegidas y enviadas por él\autocite[Cf.~][114s]{prades2015testimonio}. Esta manifestación divina implica como testigo al mismo pueblo, hacia quien ha sido dirigida la voz del Señor.

La literatura sapiencial recoge la profundización en la experiencia de Dios que ha tenido el pueblo de Israel. En ella se describe el acceso posible al conocimiento de Dios a partir de los bienes visibles o de sus obras: \blockquote[][\,(Sab 13,1-5)]{Son necios por naturaleza todos los hombres que han ignorado a Dios y no han sido capaces de conocer al que es a partir de los bienes visibles, ni de reconocer al artífice fijándose en sus obras, sino que tuvieron por dioses al fuego, al viento, al aire ligero, a la bóveda estrellada, al agua impetuosa y a los luceros del cielo, regidores del mundo. Si, cautivados por su hermosura, los creyeron dioses, sepan cuánto los aventaja su Señor, pues los creó el mismo autor de la belleza. Y si los asombró su poder y energía, calculen cuánto más poderoso es quien los hizo, pues por la grandeza y hermosura de las criaturas se descubre por analogía a su creador}.

El Dios que puede ser reconocido por analogía en el asombro y belleza de las criaturas es un Dios personal que concede sabiduría al piadoso: \blockquote[][\,(Eclo 43,32-3)]{Aún quedan misterios mucho más grandes: tan solo hemos visto algo de sus obras. Porque el Señor lo ha hecho todo y a los piadosos les ha dado la sabiduría}. Esta sabiduría es justicia y raíz de inmortalidad: \blockquote[][\,(Sab 15,1-3]){Pero tú, Dios nuestro, eres bueno y fiel, eres paciente y todo lo gobiernas con misericordia. Aunque pequemos, somos tuyos y reconocemos tu poder, pero no pecaremos, sabiendo que te pertenecemos. Conocerte a ti es justicia perfecta y reconocer tu poder es la raíz de la inmortalidad}. En este sentido la misma creación es acto testimonial de Dios donde se comunica su misterio y la vida que Él ofrece.

YHWH también aparece en el Antiguo Testamento como testigo de los mandamientos contenidos en la Ley\autocite[Cf.~][115]{prades2015testimonio}. Esta queda grabada en las `tablas del testimonio' y confiadas a Moisés: \blockquote[][\,(Ex 31,18)]{Cuando acabó de hablar con Moisés en la montaña del Sinaí, le dio las dos tablas del Testimonio, tablas de piedra escritas por el dedo de Dios}. Este testimonio se enfrenta a un pueblo con el corazón extraviado: \blockquote[][\,(Ex 32,19)]{Al acercarse al campamento y ver el becerro y las danzas, Moisés, encendido en ira, tiró las tablas y las rompió al pie de la montaña}. Sin embargo Dios no se detiene ante la dureza del pueblo. Las tablas del testimonio son reconstruidas: \blockquote[][\,(Ex 34,1.27)]{El Señor dijo a Moisés: <<Labra dos tablas de piedra como las primeras y yo escribiré en ellas las palabras que había en las primeras tablas que tú rompiste.>> \textelp{} <<Escribe estas palabras: de acuerdo con estas palabras concierto alianza contigo y con Israel>>}. Moisés, que conoció el nombre misterioso del Señor (Ex 3,13s), y habló con Él como un amigo (Ex 33,11), aparece ante el pueblo como testigo del único Dios, y de su lealtad con el pueblo. Pertenece a aquellos que el Señor elige como testigos suyos en cada etapa de la historia del pueblo de Israel como testimonio suyo y de su fidelidad.

Este es el modo eminente en que el AT describe el testimonio que Dios dirige al pueblo. Los profetas y ungidos por YHWH son testigos del Señor y de su compromiso con el pueblo. La vida totalmente comprometida del profeta expresa tanto a Dios, absoluto que comunica, como su lealtad: \blockquote[{\cite[116s]{prades2015testimonio}}]{es Dios quien da testimonio de sí mismo y de sus obras y designios a través de las personas elegidas, que se comprometen en su integridad como testigos de YHWH incluso hasta la muerte si el testimonio les lleva a ello. Por eso, la autoridad del testimonio no descansa en los testigos, sino en el mismo YHWH, que es quien los escoge y envía}. En tanto que testigos, la acción de estos escogidos puede ser descrita según los rasgos que tiene la actividad humana de dar testimonio, sin embargo la noción de testigo que se aplica a estos elegidos de Dios va más allá de la que encontraríamos en el lenguaje ordinario. La vida del profeta queda comprometida con un testimonio que no le pertenece, sino que \blockquote[{\cite[118]{prades2015testimonio}}]{procede de una iniciativa absoluta, en cuanto a su origen y en cuanto a su contenido} puesto que viene de Dios y es testimonio de sí mismo. Aquí la categoría de testimonio significa mas allá de su uso ordinario en la actividad humana y adquiere un sentido religioso como dimensión totalmente nueva\autocite[Cf.~][118]{prades2015testimonio}.

El testimonio de YHWH que el profeta proclama con su actividad y el compromiso de su vida implica al pueblo y le hace testigo: \blockquote[][\,(Is 43,8-12)]{Saca afuera a un pueblo que tiene ojos, pero está ciego, que tiene oídos, pero está sordo. Que todas las naciones se congreguen y todos los pueblos se reúnan. ¿Quién de entre ellos podría anunciar esto, o proclamar los hechos antiguos? Que presenten sus testigos para justificarse, que los oigan y digan: es verdad. Vosotros sois mis testigos --—oráculo del Señor--—, y también mi siervo, al que yo escogí, para que sepáis y creáis y comprendáis que yo soy Dios. Antes de mí no había sido formado ningún dios, ni lo habrá después. Yo, yo soy el Señor, fuera de mí no hay salvador. Yo lo anuncié y os salvé; lo anuncié y no hubo entre vosotros dios extranjero. Vosotros sois mis testigos --—oráculo del Señor--—: yo soy Dios}. El siervo es testigo al que el Señor ha escogido para que el pueblo sepa, crea y comprenda que YHWH es el único Dios verdadero. Al compartir este saber de Dios con el pueblo, éstos también están llamados a ser testigos. Ninguna otra nación podría anunciar como ellos lo que YHWH ha hecho para proveer, liberar, salvar.

Así como el profeta, el pueblo es escogido y enviado por YHWH y por medio de él el Señor da testimonio de sí mismo y se propone como quien da sentido y consistencia a toda la realidad humana. Este testimonio tiene importancia social puesto que está llamado a ser proclamado, y esta proclamación implica el compromiso de los actos y la vida del testigo, es decir, del profeta y todo el pueblo\autocite[Cf.][1526-1527]{latourelle2000testimonio}.

El testimonio de Dios a través de personas escogidas por Él en el AT queda constituido por la narración de hechos que acontecen en la historia, estos hechos son interpretados en su valor absoluto y carácter redentor, y son confesados como actuación de Dios en la vida humana\autocite[Cf.][119]{prades2015testimonio}. Esto vuelve a ponernos en conexión con la figura de Cristo como profeta acreditado por su Resurrección y los apóstoles como verdaderos testigos de un hecho enraizado en la historia, confesado desde la fe e interpretado desde la acción de Dios en Jesús. Esta sintonía anticipa lo que se verá a continuación sobre el testimonio en el Nuevo Testamento. En él la acción testimonial de Dios se describe en continuidad con la tradición veterotestamentaria y llegará a su manifestación plena en el misterio pascual.

\subsection{La acción testimonial de Dios en el Nuevo Testamento}

El Evangelio de Mateo enseña que cuando Jesús llegó a Cafarnaún a comenzar su predicación se cumplieron las promesas que Dios había hecho por medio de los profetas. El Reino de los cielos se desvelaba en su cercanía. Allí la vida de los primeros discípulos cambió. El testimonio de Cristo no es sobre cualquier anuncio o cualquier hecho, sino que tiene un valor absoluto. Jesús de Nazaret \blockquote[{\cite[126]{prades2015testimonio}}]{no se limita a proponer una cierta inspiración espiritual o un cierto sentido ético para el obrar de la persona o del pueblo, sino que pretende ser radicalmente <<testimonio de la verdad>> (Jn 18,37) de alcance universal}.

Jesús es testimonio de carácter singular, en quien se da a conocer el momento de la plenitud de la salvación, presencia del hombre nuevo y `paradigma universal de humanidad'\autocite[Cf.~][279; 290--291]{ninot2009tf}. Este valor universal de la verdad que se comunica en Jesús se desarrolla y se manifiesta en sus acciones concretas: comiendo con los pecadores o sanando a los enfermos es donde se muestra \blockquote[][\,(Cf.~Jn 14,6)]{el camino, la verdad y la vida} para todos.

Este testimonio de Cristo, su vida, actos y palabras, fue sometido al juicio de sus contemporáneos. Asombrados porque no enseña como los demás y por las signos que realiza, se cuestionan sobre su autoridad y poder. Entonces Jesús también tiene que ofrecer testimonio de su credibilidad. La respuesta a este juicio del pueblo se halla en su ministerio en sintonía con las Escrituras: \blockquote[][\,(Lc 4,21)]{Hoy se ha cumplido esta Escritura que acabáis de oír}; donde el pueblo puede encontrar la vida y el sentido que buscan: \blockquote[][\,(Jn 5,39-40)]{estudiáis las Escrituras pensando encontrar en ellas vida eterna; pues ellas están dando testimonio de mi, ¡y no queréis venir a mí para tener vida!}. El testimonio de credibilidad de Jesús ante el pueblo se encuentra también en sus obras, que son las obras del Padre y son confirmación y realización de sus enseñanzas: \blockquote[][\,(Jn 10,38)]{Si no hago las obras de mi Padre, no me creáis, pero si las hago, aunque no me creáis a mí, creed a las obras, para que comprendáis y sepáis que el Padre está en mí y yo en el Padre}.

El singular testimonio de Cristo es comunicación de la verdad con valor universal. El testimonio de Cristo es también su actividad e identidad que hacen creíble lo que comunica. De este modo entre lo que Jesús testimonia y la credibilidad que suscita su testimonio hay una circularidad constante: \blockquote[{\cite[124]{prades2015testimonio}}]{La pretensión única que encerraba su testimonio resultaba tan exorbitante que hubiera sido inaceptable para los hombres si no fuera porque sus obras, sus palabras y, en rigor, su presencia misma, lo hacían profundamente razonable en su singularidad}.

Acoger el testimonio de Jesús es escuchar la Escritura y creer en las obras del Padre. Sin embargo la palabra de Cristo choca con el odio de aquellos que son hostiles a la verdad y que, rechazando su testimonio, se juzgan a sí mismos\footnote{\cite[Cf.~][1530]{latourelle2000testimonio}
%: \enquote{Pero la palabra de Cristo choca con la contestación y el odio. Enfrentados con Cristo, los judíos, que representan al conjunto del mundo hostil a la verdad, rechazan su testimonio y se juzgan a sí mismos.}
}. \blockquote[][\,(Jn 15,22-24)]{Si yo no hubiera venido y no les hubiera hablado, no tendrían pecado, pero ahora no tienen excusas de su pecado. El que me odia a mí, odia también a mi Padre. Si yo no hubiera hecho en medio de ellos obras que ningún otro ha hecho, no tendrían pecado, pero ahora las han visto y me han odiado a mí y a mi Padre}.

Jesús es \blockquote[][\,(Jn 1,5)]{la luz que brilla en la tiniebla y la tiniebla no la recibió}. Jesús es el \blockquote[][\,(Jn 1,18)]{unigénito, que está en el seno del Padre, es quien lo ha dado a conocer}. Este testimonio es manifestación de la comunión trinitaria. Cristo revela al Padre y comunica al Espíritu, y su identidad de Hijo es manifestada como acción del Padre y del Espíritu: \blockquote[][\,(Mt 4,16-17)]{Apenas se bautizó Jesús, salió del agua; se abrieron los cielos y vio que el Espíritu de Dios bajaba como una paloma y se posaba sobre él. Y vino una voz de los cielos que decía: ``Este es mi Hijo amado, en quien me complazco''}.

La acción testimonial de Dios que se describe en el Nuevo Testamento está concentrada en la persona de Cristo, y en su relación manifiesta con el Padre y el Espíritu se expresa el testimonio de la Trinidad misma: \blockquote[{\cite[410]{latourelle1999rev}}. Ver también: {\cite[131]{prades2015testimonio}}]{la Escritura describe la actividad reveladora de la trinidad en forma de testimonios mutuos. El Hijo es el testigo del Padre, y como tal se da a conocer a los apóstoles. A su vez, el Padre da también testimonio de que Cristo es el Hijo, por la atracción que produce en las almas, por las obras que da al Hijo para que las realice y sobre todo por la resurrección, testimonio decisivo del Padre en favor del Hijo. El Hijo da testimonio del Espíritu porque promete enviarlo como educador, consolador, santificador. Y el Espíritu viene y da testimonio del hijo porque le recuerda, le da a conocer, descubre la plenitud de sentido de sus palabras, lo insinúa en las almas}. Esta actividad reveladora de la trinidad introduce al ser humano en la comunión trinitaria. Dios trino se comunica al ser humano actuando en su interior, atrayendo, inspirando; también se comunica externamente por las obras que realiza. Esta participación en la comunión divina viene bien expresada en la finalidad del testimonio apostólico: \blockquote[][\,(1Jn 1,3)]{Eso que hemos visto y oído os lo anunciamos, para que estéis en comunión con nosotros y nuestra comunión es con el Padre y con su Hijo Jesucristo}.

Jesús es el fundamento, testigo fiel y veraz para todo tiempo y lugar\footnote{Cf. Ap 1,15; 3,14. Ver también: \cite[132]{prades2015testimonio}}. Creer su testimonio es acoger al absoluto en la historia, esta confianza la hace posible el Espíritu: \blockquote[{\cite{latourelle2000testimonio}}]{Cristo es, por tanto, el testigo absoluto, el que lleva en sí mismo la garantía de su testimonio. El hombre, sin embargo, no sería capaz de acoger por la fe este testimonio del absoluto, manifestado en la carne y el lenguaje de Jesús, sin una atracción interior (Jn 6,44), que es un don del Padre y un testimonio del Espíritu (1Jn 5,9-10)}.

Aquellos que creen en Cristo no sólo encuentran una respuesta a su búsqueda de vida y de sentido, sino que \blockquote[][\,(Jn 7,38)]{de sus entrañas manarán ríos de agua viva}. Y esto Jesús lo dice \blockquote[][\,(Jn 7,39)]{refiriéndose al Espíritu que habían de recibir los que creyeran en él}. Esta promesa del Espíritu acontece en Pentecostés y sin ese testimonio postpascual del Espíritu quedaría incompleta la comunicación de Dios en el misterio Pascual\autocite[Cf.~][135]{prades2015testimonio}. El envío y la acción del Espíritu prometido completa la acción testimonial de Dios: \blockquote[{\cite[134-135]{prades2015testimonio}}]{Al haber <<acompañado>> al Hijo en la tierra de una manera singular desde el momento de su unción en el Jordán, que dispone al Hijo ---concebido por obra del Espíritu Santo--- para la misión en la carne, el Espíritu Santo vuelve al Padre portando en sí todo el misterio redentor del Hijo. De este modo, cuando el Resucitado lo envía a la Iglesia, el Espíritu vuelve como Testigo de la verdad completa, que incluye la perfecta glorificación de la carne del Hijo como plenitud de lo humano}.

El Espíritu enviado por Cristo lleva a la verdad plena a los apóstoles: \blockquote[][\,(Jn 16,13)]{cuando venga él, el Espíritu de la verdad, os guiará hasta la verdad plena. Pues no hablará por cuenta propia, sino que hablará de lo que oye y os comunicará lo que está por venir}. Este testimonio del Espíritu completa también el testimonio de los apóstoles: \blockquote[][\,(Jn 15,26-27)]{Cuando venga el Paráclito, que os enviaré desde el Padre, el Espíritu de la verdad, que procede del Padre, él dará testimonio de mí; y también vosotros daréis testimonio, porque desde el principio estáis conmigo}. Ellos han estado desde el principio con Cristo, así son testigos que pueden narrar lo que han visto y oído; su testimonio queda perfeccionado por el Espíritu que les introduce en el misterio del Hijo encarnado y les permite interpretar y comprender la verdad del Hijo, y por éste, la del Padre\autocite[Cf.~][139]{prades2015testimonio}.

Los que han compartido con Jesús desde el principio son testigos del Evangelio, pero el Resucitado sigue eligiendo apóstoles y en virtud de la acción del Espíritu éstos son testigos del mismo misterio\autocite[Cf.~][576]{ninot2009tf}. Así Matías no sólo es \blockquote[][\,(Hch 1,21)]{uno de los que nos acompañaron todo el tiempo que convivió con nosotros el Señor Jesús}, sino que es elegido por el Resucitado (Cf.~Hch 1,24-26). Igualmente Pablo es constituido testigo por la llamada del Resucitado, así puede decir \blockquote[][\,(1Cor 2,1)]{Yo mismo hermanos cuando vine a vosotros a anunciaros el testimonio de Dios\ldots}. De este modo la transmisión viva del testimonio cristiano esta constituida por un momento fundacional en la convivencia con Jesús y un momento continuante como dos aspectos históricos inseparables.\autocite[Cf.~][148]{prades2015testimonio} Este momento continuante está compuesto por los que han sido testigos oculares, y por los que no lo han sido: \blockquote[{\cite[148]{prades2015testimonio}}]{unos y otros son elegidos, llamados y enviados por Cristo, el Cristo histórico los primeros y el Cristo glorioso los segundos}. Aquel que recibe este testimonio y cree en él encuentra la vida nueva. ``¿Qué dificultad hay en que me bautice?'', decide aquel hombre que recibió el testimonio de Felipe y ``siguió su camino lleno de alegría'' después de haber encontrado a Dios. Considerar la revelación divina como acción testimonial de Dios conduce en definitiva a estimar la revelación misma como forma de amor y libertad de Dios que interpela el amor y libertad humano. En tanto que comunicación libre y amorosa, el testimonio de Dios atiende la naturaleza humana de su beneficiario; en tanto que don divino queda desvelado su origen y meta más allá de lo humano\autocite[Cf.][152]{prades2015testimonio}.
