\section{La Categoría del Testimonio en la Sagrada Escritura}

La Iglesia de hoy, como María, conserva el Evangelio meditándolo en su
corazón.\footnote{Cf.~Lc 2,19} Así está presente en el centro de la comunidad
creyente el anuncio de Cristo vivo como fundamento de su esperanza en cada etapa
de la historia. Este motivo de esperanza conservado es también compartido y
expresado, según la enseñanza del apóstol:
\blockquote[1Pe 3,15]{glorificad a Cristo en vuestros corazones, dispuestos
  siempre a dar explicación a todo el que os pida una razón de vuestra
  esperanza}.

Este Evangelio atesorado como fundamento en el centro de la vida de la comunidad
eclesial, así como Buena Nueva proclamada y transmitida en el tiempo y en el
mundo puede ser comprendido como tres testimonios que son uno: \enquote{palabra
  vivida en el Espíritu}.\footnote{\cite[Cf.~][110]{latourelle1975et}: Car c'est
  L'Esprit qui posse l'Eglise à poursuivre son oeuvre d'évangelisation; c'est
  l'Esprit qui inspire la foi, la nourrit et l'approfondit. C'est l'Esprit qui
  relie entre eux ces trois témoignages qui n'en font qu'un: celui de la
  parole-vécue-dans-l'Esprit. Par son témoignage, l'Esprit intériorise le
  témoignage extérieur de la Bonne Nouvelle du salut en Jésus-Christ et le porte
  à l'accomplissement de la foi, qui est la réponse d'amour de l'humanité à
  l'appel d'amour du Père par le Christ. Ver también \cite[582]{ninot2009tf}
  donde este triple testimonio sirve para orientar la reflexión sobre el
  testimonio como vía empírica de la credibilidad de la Iglesia.}

La Evangelización puede ser entendida en este sentido como testimonio de la
\enquote{palabra de vida}\footnote{1Jn 1,1} que los apóstoles anuncian como
testigos de lo que han contemplado y palpado\footnote{ibíd.}. Es también el
testimonio de los cristianos que, acogiendo esta palabra, la viven, poniendo por
obra lo que ella enseña. Es además testimonio del Espíritu Santo que interioriza
el testimonio externo de la Buena Noticia y lo lleva al cumplimiento de la fe en
cada persona.\autocite[Cf.~][110]{latourelle1975et} Es el Espíritu el que
santifica y fecunda la acción de los cristianos, es tambíen el que impulsa y
sostiene la acción de la Iglesia; es el Espíritu el que inspira la fe, la nutre
y la profundiza.\autocite[Cf.~][110]{latourelle1975et}

Este dinamísmo fundamental que puede encontrarse vivo hoy en la comunidad de la
Iglesia ha actuado en ella desde su origen y le ha acompañado en cada época.
Según esto es posible valorar lo que se transmite en la tradición eclesial como
la perpetuación de la actividad de Cristo y los apóstoles, que es a su vez
proyección del testimonio divino.\footnote{\cite[Cf.~][573]{ninot2009tf}:
  \enquote{el testimonio divino se proyecta luego en el apostólico y se perpetúa
    en el testimonio eclesial. Por eso, el testimonio es revelación en la
    actividad de Cristo y de los apóstoles y es transmisión de la revelación en
    la tradición eclesial.}}

En la actividad de Cristo el testimonio divino queda proyectado como
interpelación a la libertad realizada por la identidad propia de Jesús:
\blockquote[Jn 4,10]{Si conocieras el don de Dios y quién es el que te dice
  \enquote{dame de beber} le pedirías tu, y él te daría agua viva};
\blockquote{\enquote{¿Crees tú en el Hijo del hombre?}\textelp{} \enquote{¿Y
    quién es, Señor, para que crea en él?}\textelp{} \enquote{Lo estás viendo:
    el que te está hablando, ese es}}.\footnote{Jn 9, 35--37} En la actividad
apostólica, el testimonio divino sigue interpelando la libertad humana como
manifestación de Jesús Resucitado. Los apóstoles actuan como testigos de los
acontecimientos de la Pascua de Jesús y su valor
salvífico\autocite[Cf.][576]{ninot2009tf} y este testimonio es descrito como
acción del Espíritu que impulsa la tarea apostólica y que da nueva vida a los
que acogen el anuncio de la Buena Noticia.

Puede encontrarse un ejemplo de esto en el testimonio de Felipe. El apóstol sale
más allá de Jerusalén hacia Samaria, y todavía llega más lejos, al compartir la
Buena Noticia de Jesús con un extranjero Etíope:

\blockquote[Hch 8, 29--39]{El Espíritu dijo a Felipe: \enquote{Acércate y pégate
    a la carroza}. Felipe se acercó corriendo, le oyó leer el profeta Isaías, y
  le preguntó: \enquote{¿Entiendes lo que estás leyendo?}. Contestó: \enquote{¿Y
    cómo voy a entenderlo si nadie me guía?}. E invitó a Felipe a subir y a
  sentarse con él. El pasaje de la Escritura que estaba leyendo era este:
  \emph{Como cordero fue llevado al matadero, como oveja muda ante el
    esquilador, así no abre su boca. En su humillación no se le hizo justicia.
    ¿Quién podrá contar su descendencia? Pues su vida ha sido arrancada de la
    tierra.} El eunuco preguntó a Felipe: \enquote{Por favor, ¿de quién dice
    esto el profeta?; ¿de él mismo o de otro?}. Felipe se puso a hablarle y,
  tomando pie de este pasaje, le anunció la Buena Nueva de Jesús. Continuando el
  camino, llegaron a un sitio donde había agua, y dijo el eunuco: \enquote{Mira,
    agua. ¿Qué dificultad hay en que me bautice?}. Mandó parar la carroza,
  bajaron los dos al agua, Felipe y el eunuco, y lo bautizó. Cuando salieron del
  agua, el Espíritu del Señor arrebató a Felipe. El eunuco no volvió a verlo, y
  siguió su camino lleno de alegría.}

Además de ser ejemplo de la actividad apostólica, este relato puede servir como
síntesis del modo en que la categoría del testimonio está presente en la
Escritura.

El testimonio comienza con la iniciativa de Dios mismo que impulsa tanto la
palabra profética del Antiguo Testamento como el anuncio apostólico del Nuevo
Testamento. Esta iniciativa de Dios tiende hacia el testimonio de la Palabra
definitiva del Padre que es Cristo resucitado. En aquellos que creen en el
testimonio de Dios se engendra alegría y vida nueva. En palabras de R.
Latourelle:
\blockquote[{\cite[1531]{latourelle2000testimonio}}]{En el trato de las tres
  personas divinas con los hombres existe un intercambio de testimonios que
  tiene la finalidad de proponer la revelación y de alimentar la fe. Son tres
  los que revelan o dan testimonio, y esos tres son más que uno. Cristo da
  testimonio del Padre, mientras que el Padre y el Espíritu dan testimonio del
  Hijo. Los apóstoles a su vez dan testimonio de lo que han visto y oído del
  verbo de la vida. Pero su testimonio no es la comunicación de una ideología,
  de un descubrimiento científico, de una técnica inédita, sino la proclamación
  de la salvación prometida y finalmente realizada.}

De este modo el anuncio del apóstol Felipe sirve aquí como un ejemplo específico
del testimonio, que ilustra sin embargo, una noción que
\blockquote[{\cite[109]{prades2015testimonio}}]{atraviesa toda la Escritura y se
  corresponde con la estructura misma de la revelación}. El testimonio está
presente a lo largo de la Escritura junto a otras categorías como pueden ser la
de \enquote{alianza}, \enquote{palabra}, \enquote{paternidad} o
\enquote{filiación}, como parte del
\blockquote[{\cite[1523]{latourelle2000testimonio}}]{grupo de analogías
  empleadas por la Escritura para introducir al hombre en las riquezas del
  misterio divino}.

Esta clave servirá para dar enfoque a un examen sobre la categoría del
testimonio en la Escritura. ¿Qué nos dice el Antiguo y el Nuevo Testamento de la
revelación como acto testimonial de Dios? Esta pregunta supone que la revelación
comparte los rasgos de la actividad humana que es el testimonio, sin embargo,
como Latourelle adiverte:
\blockquote[{\cite[1526]{latourelle2000testimonio}}]{globalmente se puede decir
  que el testimonio bíblico asume pero al mismo tiempo exalta hasta sublimarlos,
  los rasgos del testimonio humano}.

Cabe añadir una última consideración. La revelación de Dios entendida como acto
testimonial suyo tiene como expresión definitiva el misterio pascual de
Cristo.\footnote{\cite[128]{prades2015testimonio}: el misterio pascual al cual
  tiende toda la existencia terrena de Cristo, constituye el acto testimonial
  por excelencia de Dios.} Este misterio ocupa el lugar principal en el
testimonio bíblico:
\blockquote[{\cite[404]{ninot2009tf}}]{la Resurrección como \enquote{final} de
  la unicidad del acontecimiento de Jesucristo, encarnado, muerto y resucitado,
  subraya específicamente la definitividad de la existencia humana salvada por
  Dios en la carne de Jesús de Nazaret, ya que la autocomunicación de Dios ha
  alcanzado su palabra última en la Resurrección de Jesucristo, y por eso es
  prenda de la resurrección de todos los hombres.}
Como tal, parece justo tratar el testimonio que es el misterio pascual en su
propio apartado. Y será éste precisamente el punto de partida para esta
descripción de la categoría del testimonio en la Escritura.
