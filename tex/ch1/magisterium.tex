\section{Iglesia como signo sacramental: el Testimonio en el Magisterio Reciente}

Nuestro recorrido comenzó al inicio de este capítulo tomando como punto de
partida a la Iglesia como signo visible. La vida de la comunidad eclesial, sus
costumbres y actitudes, son presencia histórica y realidad perceptible. La
Iglesia puede ser reconocida hoy actuando según su costumbre de reunirse en
torno a la Palabra de Dios para celebrarla y conocer la verdad para su vida. Lo
que se vive hoy y se ha transmitido en la tradición eclesial lo hemos valorado
como perpetuación de la actividad de Cristo y de los apóstoles y, por tanto,
como proyección del testimonio divino. En este sentido hemos considerado la
presencia de la Revelación divina en el corazón y anuncio de la Iglesia como
triple testimonio usando la expresión de Latourelle: \enquote{palabra vivida en
  el Espíritu}. Esta reflexión ha querido servir para describir la naturaleza de
la Revelación como experiencia familiar en la vida de la Iglesia. La noción de
la categoría del testimonio que atraviesa la escritura ha servido para valorar
la naturaleza de la Revelación según su estructura testimonial.

Así como la categoría del testimonio ha servido para decir algo sobre la
Revelación en la Escritura, ahora se pretende decir algo sobre lo que la
categoría del testimonio puede aportar para comprender la identidad de la
Iglesia y su misión en el mundo y cómo ésta forma parte del dinamismo de la
Revelación divina.

Con Latourelle se ha dicho que el testimonio es una de esas categorías que la
escritura emplea como analogía para introducirnos al misterio divino. El
Concilio nos regala otra analogía que va de la mano con la categoría del
testimonio en la comprensión de la Iglesia y su misión:
\blockquote[LG 8]{la sociedad provista de sus órganos jerárquicos y el Cuerpo
  místico de Cristo, la asamblea visible y la comunidad espiritual, la Iglesia
  terrestre y la Iglesia enriquecida con los bienes celestiales, no deben ser
  consideradas como dos cosas distintas, sino que más bien forman una realidad
  compleja que está integrada de un elemento humano y otro divino. Por eso se la
  compara, por una notable analogía, al misterio del Verbo encarnado, pues así
  como la naturaleza asumida sirve al Verbo divino como de instrumento vivo de
  salvación unido indisolublemente a Él, de modo semejante la articulación
  social de la Iglesia sirve al Espíritu Santo}

La visibilidad de la Iglesia está al servicio del Espíritu Santo de modo que su
naturaleza humana sirve a la presencia divina como instrumento vivo de
salvación. La presencia de la articulación social de la Iglesia en el mundo
actúa de manera análoga a la presencia de Cristo. De este modo
\blockquote[{\cite[566]{ninot2009tf}}]{la eclesiología se resuelve en la
  Cristología y por eso el \enquote{lugar} de la Iglesia en el acto de creer
  será \enquote{análogo} al de Cristo}. Esta relación con Cristo y el Espíritu
otorgan a la Iglesia un valor sacramental:
\blockquote[LG 48]{Porque Cristo, levantado sobre la tierra, atrajo hacia sí a
  todos (cf. Jn 12, 32 gr.); habiendo resucitado de entre los muertos (Rm 6, 9),
  envió sobre los discípulos a su Espíritu vivificador, y por Él hizo a su
  Cuerpo, que es la Iglesia, sacramento universal de salvación; estando sentado
  a la derecha del Padre, actúa sin cesar en el mundo para conducir a los
  hombres a la Iglesia y, por medio de ella, unirlos a sí más estrechamente y
  para hacerlos partícipes de su vida gloriosa alimentándolos con su cuerpo y
  sangre. Así que la restauración prometida que esperamos, ya comenzó en Cristo,
  es impulsada con la misión del Espíritu Santo y por Él continúa en la Iglesia,
  en la cual por la fe somos instruidos también acerca del sentido de nuestra
  vida temporal, mientras que con la esperanza de los bienes futuros llevamos a
  cabo la obra que el Padre nos encomendó en el mundo y labramos nuestra
  salvación (cf. Flp 2, 12).}
Esta Iglesia que es sacramento es mediación de acción salvadora de Dios;
comunica los dones de la gracia y manifiesta el misterio de Dios:
\blockquote[GS 45]{Todo el bien que el Pueblo de Dios puede dar a la familia
  humana al tiempo de su peregrinación en la tierra, deriva del hecho de que la
  Iglesia es ``sacramento universal de salvación'', que manifiesta y al mismo
  tiempo realiza el misterio del amor de Dios al hombre.}

La Iglesia en el mundo es así uno de los signos contenidos en la Revelación que
ayudan a la razón que busca la comprensión del misterio. El signo sacramental
que es la Iglesia permite atestiguar desde la fe el misterio de Dios que en ella
se expresa del mismo modo que ocurre con la Eucaristía o la presencia de Cristo
encarnado:
\blockquote[FR 13]{Podemos fijarnos, en cierto modo, en el horizonte sacramental
  de la Revelación y, en particular, en el signo eucarístico donde la unidad
  inseparable entre la realidad y su significado permite captar la profundidad
  del misterio. Cristo en la Eucaristía está verdaderamente presente y vivo, y
  actúa con su Espíritu, pero como acertadamente decía Santo Tomás, \enquote{lo
    que no comprendes y no ves, lo atestigua una fe viva, fuera de todo el orden
    de la naturaleza. Lo que aparece es un signo: esconde en el misterio
    realidades sublimes}. A este respecto escribe el filósofo Pascal:
  \enquote{Como Jesucristo permaneció desconocido entre los hombres, del mismo
    modo su verdad permanece, entre las opiniones comunes, sin diferencia
    exterior. Así queda la Eucaristía entre el pan común}.}

El misterio sublime que aparece en un signo puede ser atestiguado por la fe
viva. El asentimiento al signo sacramental por la fe supone el reconocimiento de
que viene de Dios y por tanto es creer a quien es garante de su propia verdad.
Este asentimiento implica a la persona por completo:
\blockquote[FR 13]{Desde la fe el hombre da su asentimiento a ese testimonio
  divino. Ello quiere decir que reconoce plena e integralmente la verdad de lo
  revelado, porque Dios mismo es su garante. Esta verdad, ofrecida al hombre y
  que él no puede exigir, se inserta en el horizonte de la comunicación
  interpersonal e impulsa a la razón a abrirse a la misma y a acoger su sentido
  profundo. Por esto el acto con el que uno confía en Dios siempre ha sido
  considerado por la Iglesia como un momento de elección fundamental, en la cual
  está implicada toda la persona. Inteligencia y voluntad desarrollan al máximo
  su naturaleza espiritual para permitir que el sujeto cumpla un acto en el cual
  la libertad personal se vive de modo pleno.}
La acogida del misterio divino comunicado en el signo sacramental es así un acto
de libertad plena que no sólo permite reconocer el misterio de Dios, sino que
nos desvela nuestra vocación de comunión con Él, que es nuestro sentido más
auténtico:
\blockquote[FR 13]{El conocimiento de fe, en definitiva, no anula el misterio;
  sólo lo hace más evidente y lo manifiesta como hecho esencial para la vida del
  hombre: Cristo, el Señor, \enquote{en la misma revelación del misterio del
    Padre y de su amor, manifiesta plenamente el hombre al propio hombre y le
    descubre la grandeza de su vocación}, que es participar en el misterio de la
  vida trinitaria de Dios.}

La Iglesia es signo sacramental unido inseparablemente al misterio divino que
comunica, de modo análogo a la unión del Verbo divino y la naturaleza asumida
por Él. El conocimiento de la fe abre la razón humana a la verdad revelada como
comunicación interpersonal de Dios realizada por medio de este signo sacramental
que es la Iglesia. Este acto de confianza es movimiento de la libertad como
asentimiento y elección de Dios que se revela y acogida de su llamada a
participar de la comunión trinitaria. Aquí sacramento y testimonio son
categorías que interactúan para describir el acceso al misterio divino que se
comunica a través de signos. Esta Iglesia que es signo sacramental es signo
creíble por el testimonio de la vida comprometida con el misterio de amor que
significa:
\blockquote[SCa 85]{La misión primera y fundamental que recibimos de los santos
  Misterios que celebramos es la de dar testimonio con nuestra vida. El asombro
  por el don que Dios nos ha hecho en Cristo infunde en nuestra vida un
  dinamismo nuevo, comprometiéndonos a ser testigos de su amor. Nos convertimos
  en testigos cuando, por nuestras acciones, palabras y modo de ser, aparece
  Otro y se comunica. Se puede decir que el testimonio es el medio con el que la
  verdad del amor de Dios llega al hombre en la historia, invitándolo a acoger
  libremente esta novedad radical. En el testimonio Dios, por así decir, se
  expone al riesgo de la libertad del hombre. Jesús mismo es el testigo fiel y
  veraz (cf. Ap 1,5; 3,14); vino para dar testimonio de la verdad (cf. Jn
  18,37). Con estas reflexiones deseo recordar un concepto muy querido por los
  primeros cristianos, pero que también nos afecta a nosotros, cristianos de
  hoy: el testimonio hasta el don de sí mismos, hasta el martirio, ha sido
  considerado siempre en la historia de la Iglesia como la cumbre del nuevo
  culto espiritual: <<Ofreced vuestros cuerpos>> (Rm 12,1). Se puede recordar,
  por ejemplo, el relato del martirio de san Policarpo de Esmirna, discípulo de
  san Juan: todo el acontecimiento dramático es descrito como una liturgia, más
  aún como si el mártir mismo se convirtiera en Eucaristía. Pensemos también en
  la conciencia eucarística que san Ignacio de Antioquía expresa ante su
  martirio: él se considera <<trigo de Dios>> y desea llegar a ser en el
  martirio <<pan puro de Cristo>>. El cristiano que ofrece su vida en el
  martirio entra en plena comunión con la Pascua de Jesucristo y así se
  convierte con Él en Eucaristía. Tampoco faltan hoy en la Iglesia mártires en
  los que se manifiesta de modo supremo el amor de Dios. Sin embargo, aun cuando
  no se requiera la prueba del martirio, sabemos que el culto agradable a Dios
  implica también interiormente esta disponibilidad, y se manifiesta en el
  testimonio alegre y convencido ante el mundo de una vida cristiana coherente
  allí donde el Señor nos llama a anunciarlo.}
El testimonio hasta el don de nosotros mismos se convierte en signo sacramental,
el cristiano que ofrece su vida por completo como testigo entra en comunión con
la Pascua y se convierte con Cristo en Eucaristía. La vida entregada, este signo
sacramental, es el medio adecuado para comunicar la comunión con Dios:
\blockquote[LF 40]{En efecto, la fe necesita un ámbito en el que se pueda
  testimoniar y comunicar, un ámbito adecuado y proporcionado a lo que se
  comunica. Para transmitir un contenido meramente doctrinal, una idea, quizás
  sería suficiente un libro, o la reproducción de un mensaje oral. Pero lo que
  se comunica en la Iglesia, lo que se transmite en su Tradición viva, es la luz
  nueva que nace del encuentro con el Dios vivo, una luz que toca la persona en
  su centro, en el corazón, implicando su mente, su voluntad y su afectividad,
  abriéndola a relaciones vivas en la comunión con Dios y con los otros. Para
  transmitir esta riqueza hay un medio particular, que pone en juego a toda la
  persona, cuerpo, espíritu, interioridad y relaciones. Este medio son los
  sacramentos, celebrados en la liturgia de la Iglesia. En ellos se comunica una
  memoria encarnada, ligada a los tiempos y lugares de la vida, asociada a todos
  los sentidos; implican a la persona, como miembro de un sujeto vivo, de un
  tejido de relaciones comunitarias. Por eso, si bien, por una parte, los
  sacramentos son sacramentos de la fe, también se debe decir que la fe tiene
  una estructura sacramental. El despertar de la fe pasa por el despertar de un
  nuevo sentido sacramental de la vida del hombre y de la existencia cristiana,
  en el que lo visible y material está abierto al misterio de lo eterno.}.
Al celebrar los sacramentos con fe viva, la comunidad eclesial se deja implicar
por completo por la luz del Dios vivo que se comunica y el memorial que se
encarna. Despertar a la fe en los sacramentos es también despertar al sentido
sacramental que tiene la propia vida cristiana. Así como en los sacramentos los
signos visibles comunican la luz de Dios, también la propia existencia del
cristiano puede arrojar esa luz.

Este valor sacramental de la vida del cristiano y de la comunidad eclesial hace
de su propia existencia un testimonio kerygmático:
\blockquote[EN 21]{La Buena Nueva debe ser proclamada en primer lugar, mediante
  el testimonio. Supongamos un cristiano o un grupo de cristianos que, dentro de
  la comunidad humana donde viven, manifiestan su capacidad de comprensión y de
  aceptación, su comunión de vida y de destino con los demás, su solidaridad en
  los esfuerzos de todos en cuanto existe de noble y bueno. Supongamos además
  que irradian de manera sencilla y espontánea su fe en los valores que van más
  allá de los valores corrientes, y su esperanza en algo que no se ve ni osarían
  soñar. A través de este testimonio sin palabras, estos cristianos hacen
  plantearse, a quienes contemplan su vida, interrogantes irresistibles: ¿Por
  qué son así? ¿Por qué viven de esa manera? ¿Qué es o quién es el que los
  inspira? ¿Por qué están con nosotros? Pues bien, este testimonio constituye ya
  de por sí una proclamación silenciosa, pero también muy clara y eficaz, de la
  Buena Nueva. Hay en ello un gesto inicial de evangelización. Son posiblemente
  las primeras preguntas que se plantearán muchos no cristianos, bien se trate
  de personas a las que Cristo no había sido nunca anunciado, de bautizados no
  practicantes, de gentes que viven en una sociedad cristiana pero según
  principios no cristianos, bien se trate de gentes que buscan, no sin
  sufrimiento, algo o a Alguien que ellos adivinan pero sin poder darle un
  nombre. Surgirán otros interrogantes, más profundos y más comprometedores,
  provocados por este testimonio que comporta presencia, participación,
  solidaridad y que es un elemento esencial, en general al primero absolutamente
  en la evangelización.}
La acción testimonial de Dios que se manifiesta en Cristo y en los sacramentos
instituidos por Él está análogamente presente en la vida comprometida del
cristiano. El testimonio humano es respuesta de fe de aquellos que han
reconocido a Dios en los signos que le encarnan y que corresponden con palabras
y obras que quieren significar la vida nueva que viene del Señor. En esta
correspondencia están hundidas las raíces de la misión de proclamar la Buena
Nueva.

El testimonio es así acción propia de todo bautizado que ha quedado unido a
Cristo y a la Iglesia.\autocite[Cf.][188]{prades2015testimonio} Toda la Iglesia
tiene la misión de dar testimonio; los obispos ofrecen al mundo el rostro de la
Iglesia con su trato y trabajo pastoral (GS 43), los presbíteros, creciendo en
el amor por el desempeño de su oficio, han de ser un vivo testimonio de Dios (LG
41), los fieles han de dar testimonio de verdad como testigos de la resurrección
(LG 28 y LG 38), los religiosos deben ofrecer un testimonio sostenido por la
integridad de la fe, por la caridad y el amor a la cruz y la esperanza de la
gloria futura (PC 25), los profesores han de dar testimonio tanto con su vida
como con su doctrina (GE 8), los misioneros han de ofrecer testimonio con una
vida enteramente evangélica, con paciencia, longanimidad, suavidad, caridad
sincera, y si es necesario hasta con la propia sangre(AG 24).

El signo que es la vida de los cristianos y, por tanto la Iglesia, esta llamado
a purificarse y crecer. La contradicción entre la fe y la vida de los cristianos
puede constituir un motivo de tropiezo, en lugar de dar a conocer la luz de
Dios. El testimonio de la vida entregada, aún cuando ha sido estimado según su
valor sacramental, es un signo imperfecto que debe ser madurado con una actitud
vigilante:
\blockquote[GS 34]{Aunque la Iglesia, por la virtud del Espíritu Santo, se ha
  mantenido como esposa fiel de su Señor y nunca ha cesado de ser signo de
  salvación en el mundo, sabe, sin embargo, muy bien que no siempre, a lo largo
  de su prolongada historia, fueron todos sus miembros, clérigos o laicos,
  fieles al espíritu de Dios. Sabe también la Iglesia que aún hoy día es mucha
  la distancia que se da entre el mensaje que ella anuncia y la fragilidad
  humana de los mensajeros a quienes está confiado el Evangelio. Dejando a un
  lado el juicio de la historia sobre estas deficiencias, debemos, sin embargo,
  tener conciencia de ellas y combatirlas con máxima energía para que no dañen a
  la difusión del Evangelio. De igual manera comprende la Iglesia cuánto le
  queda aún por madurar, por su experiencia de siglos, en la relación que debe
  mantener con el mundo. Dirigida por el Espíritu Santo, la Iglesia, como madre,
  no cesa de ``exhortar a sus hijos a la purificación y a la renovación para que
  brille con mayor claridad la señal de Cristo en el rostro de la Iglesia''}.
Es así que la vida de la Iglesia es peregrinación de maduración y
perfeccionamiento sostenida por el Espíritu. Como afirma DV 8: \blockquote{la
  Iglesia, en el decurso de los siglos, tiende constantemente a la plenitud de
  la verdad divina, hasta que en ella se cumplan las palabras de Dios}.

La categoría del testimonio ha servido para acercarnos a algunos textos
magisteriales y describir la vida de la Iglesia como signo sacramental. Las
luminosas palabras de K. Wojtyła pueden servirnos aquí para concluir:
\blockquote[Para una discusión más amplia de la lectura de Wojtyła véase
{\cite[194--197]{prades2015testimonio}}]{El significado del testimonio en la
  doctrina del Vaticano~II es explícitamente analógico, puesto que el Concilio
  habla del testimonio de Dios y del hombre, que, de diversa manera, corresponde
  al divino, y a una respuesta multiforme a la revelación. En todo caso, sin
  embargo, la respuesta es testimonio y el testimonio, respuesta.}

Este recorrido a través de algunos modos de emplear la categoría del testimonio
en la Escritura y la doctrina magisterial ha servido para describir los
dinamismos de la Revelación como acción libre y amorosa del Padre encarnada en
la naturaleza humana asumida por el Verbo y sostenida por la acción interior del
Espíritu. Esta acción de la libertad divina ha encontrado la correspondencia de
la libertad humana que acoge la invitación al amor y se compromete por completo
a la comunión con Dios. Este intercambio testimonial comunica el amor divino.
