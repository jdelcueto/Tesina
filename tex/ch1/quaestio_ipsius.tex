\section{La Categoría del Testimonio como Problema}

Hasta ahora se ha empleado la categoría del testimonio sin problematizarla. Es
decir, hemos tratado el testimonio como \enquote{cosa familiar y conocida}
empleada ordinariamente en nuestras conversaciones. Es aquí donde nos permitimos
tratar al testimonio como algo que hay que esclarecer, algo sobre lo que se
plantean preguntas, de modo que hay que traer a la mente una explicación
adecuada. En palabras de Latourelle:
\blockquote[{\cite[1523]{latourelle2000testimonio}}]{Si la revelación misma se
  apoya en la experiencia humana del testimonio para expresar una de las
  relaciones fundamentales que unen al hombre con Dios, la reflexión teológica
  se encuentra entonces autorizada a explorar los datos de esta experiencia.}

Hasta el momento solo hemos formulado una pregunta, que si ampliamos un poco
queda: \enquote{¿qué es conocer una verdad para la vida por el testimonio de la
  revelación divina?}. Esta formulación puede servir como punto de partida y, si
tenemos en cuenta la reflexión filosófica en torno al testimonio, podemos
expandirla más.

Aún cuando el testimonio ocupa un lugar vital en nuestro contacto con el mundo,
no siempre ha gozado del interés de la investigación filosófica. Más
recientemente, sin embargo, su importancia ha sido mejor apreciada y así lo
refleja la variedad literatura que puede encontrarse en la filosofía
contemporánea.\footnote{Cf.~\cite[1]{lackeysosa2006eptest}: Despite the vital
  role that testimony occupies in our epistemic lives, traditional
  epistemological theories focused primarily on other sources, such as sense
  perception, memory, and reason, with relatively little attention devoted
  specifically to testimony. In recent years, however, the epistemic
  significance of testimony has been more fully appreciated, and the current
  literature has benefited from the publication of a considerable amount of
  interesting and innovative work in this area.}
Esta época mas prolija en discusiones no es, sin embargo, el origen de algunas
posturas propuestas en torno al testimonio; éste lo encontramos más bien en la
época moderna. Recurriremos, por tanto, a algunas aportaciones y desafíos
ofrecidos por la filosofía moderna y contemporánea para expandir nuestra
anterior pregunta y formular las cuestiones principales que servirán luego para
navegar en el pensamiento de Elizabeth Anscombe.

\subsection{¿Cuál es el valor epistemológico del testimonio?}
Corresponde a la epistemología la tarea de estudiar la naturaleza del conocer y
su justificación. ¿Cuáles son los componentes del conocimiento? ¿sus fuentes o
condiciones? ¿sus límites?\footnote{Cf.~\cite[3]{moser2002ep}: Epistemology,
  characterized broadly, is an account of knowledge. Within the discipline of
  philosophy, epistemology is the study of the nature of knowledge and
  justification: in particular, the study of (a) the defining components, (b)
  the substantive conditions or sources, and (c) the limits of knowledge and
  justification.} La pregunta sobre el valor epistemológico del testimonio
consiste en juzgar el lugar que éste ocupa en una descripción del conocimiento;
¿qué se puede decir del testimonio como estrategia para adquirir la verdad y
evitar el error?\footnote{Cf.~\cite[14]{moser2002ep}: Any standard or strategy
  worthy of the title ``epistemic'' must have as its fundamental goal the
  acquisition of truth and the avoidance of error.}

Podemos recurrir al análisis tradicional empleado para hablar del conocimiento
proposicional y entenderlo como \enquote{creencia verdadera
  justificada}.\footnote{\cite[4]{moser2002ep}: Ever since Plato's Theaetetus,
  epipstemologists have tried to identify the essential, defining components of
  propositional knowledge. These components will yield an analysis of
  propositional knowledge. An influential traditional view, inspired by Plato
  and Kant among others, is that propositional knowledge has three individually
  necessary and jointly sufficient components: justification, truth, and belief.
  On this view, propositional knowledge is, by definition, justified true
  belief. This tripartite definition has come to be called ``the standard
  analysis''.} Según esta composición tripartita la pregunta sobre el valor
epistemológico del testimonio se puede plantear diciendo: dada una comunicación
que cualifique como testimonio y que sea al caso que la creencia formada desde
esta comunicación está basada enteramente en el testimonio
recibido,\footnote{Cf.~\cite[4]{lackeysosa2006eptest}: Even if an expression of
  thought qualifies as testimony and the resulting belief formed is entirely
  testimonially based for the hearer, however, there is the further question of
  how precisely such a belief successfully counts as justified belief or an
  instance of knowledge.} ¿cómo adquirimos efectivamente una creencia verdadera
justificada sobre la base de lo que alguien nos ha
dicho?,\footnote{Cf.~\cite[2]{lackeysosa2006eptest}: how we successfully acquire
  justified belief or knowledge on the basis of what other people tell us. This,
  rather than what testimony is, is often taken to be the issue of central
  import from an epistemological point of view.} es decir, ¿cómo, precisamente,
una creencia como esta puede ser contada satisfactoriamente como creencia
justificada o una instancia de conocimiento?
\footnote{Cf.~\cite[4]{lackeysosa2006eptest}: how precisely such a belief
  successfully counts as justified belief or an instance of knowledge}

Las respuestas a esta pregunta central sobre la epistemología del testimonio se
han situado en dos posturas que se han denominado \enquote{reduccionista} y
\enquote{no-reduccionista}.\footnote{Cf.~\cite[4]{lackeysosa2006eptest}: Indeed,
  this is the question at the center of the epistemology of testimony, and the
  current philosophical literature contains two central options for answering
  it: non-reductionism and reductionism.} Las raíces históricas de la primera
postura se le suelen atribuir a Hume y de la segunda a Thomas Reid.

De acuerdo a los no-reduccionistas el testimonio es simplemente una fuente de
justificación como lo sería la percepción de los sentidos, la memoria o la
inferencia. Según esto, siempre que no haya una justificación contraria
suficientemente relevante, el que escucha tiene justificación verdadera para
creer las proposiciones del testimonio del que
habla.\footnote{Cf.~\cite[4]{lackeysosa2006eptest}: According to
  non-reductionists ---whose historical roots are standardly traced back to
  Reid--- testimony is just as basic a source of justification (warrant,
  entitlement, knowledge, etc.) as sense perception, memory, inference, and the
  like. Accordingly, so long as there are no relevant defeaters, hearers can
  justifiedly accept the assertions of speakers merely on the basis of a
  speaker's testimony.}

Hume, por su parte, \blockquote[{\cite[79]{coady1992test}}: is one of the few
philosophers who has offered anything like a sustained account of testimony and
if any view has a claim to the title of `the received view' it is his]{es uno de
  los pocos filósofos que ha ofrecido algo así como una descripción sostenida
  acerca del testimonio y si alguna perspectiva puede reclamar el título de `el
  punto de vista adoptado' es la suya}. En la base de su valoración del
testimonio está su estima de la relación de causa y efecto como fundamento de
cualquier razonamiento concerniente a cuestiones de hecho.

Distinto a las relaciones de ideas, la evidencia de la veracidad de una cuestión
de hecho no se demuestra a priori, sino que ha de ser descubierta en la
experiencia. Ahora bien, ¿cuál es la naturaleza de la evidencia de aquellas
cuestiones de hecho que están más allá de la percepción de nuestros sentidos o
de las impresiones de nuestra memoria?\footnote{Matters of fact, which are the
  second objects of human reason, are not ascertained in the same manner; nor is
  our evidence of their truth, however great, of a like nature with the
  foregoing (relations of ideas) [\ldots] The contrary of every matter of fact
  is still possible [\ldots] We should, in vain, therefore attempt to
  demonstrate its falsehood. Were it demonstratively false, it would imply a
  contradiction, and could never be distinctly conceived by the mind [\ldots]
  what is the nature of that evidence which assures us of any real existence and
  matter of fact, beyond the present testimony of our senses, or the records of
  our memory. enquiry ch IV part I} Nuestros razonamientos relacionados con
algún hecho se componen de inferencias realizadas a partir del conocimiento que
tenemos de que a una causa se sigue su efecto.\footnote{All our reasonings
  concerning fact are of the same nature; and here it is constantly supposed
  that there is a connection between the present fact and that which is inferred
  from it. Were there nothing to bind them together, the inference would be
  entirely precarious.} Este conocimiento de la relación causa y efecto, a su
vez, no consiste en un razonamiento a priori, \citalitinterlin{sino que surge
  completamente de la experiencia, cuando descubrimos que cualesquiera objetos
  particulares están constantemente unidos entre sí}.\footnote{that the
  knowledge of this relation is not, in any instance, attained by reasonings a
  priori, but arises entirely from experience, when we find that any particular
  objects are constantly conjoined with each other. enquiry ch IV part I} Es así
que \citalitinterlin{por medio de esta relación solamente, podemos ir más allá
  de nuestra memoria y sentidos}. \footnote{By means of that relation alone, we
  can go beyond the evidence of our memory and senses.}

Esta misma línea de razonamiento es la que se sigue en la descripción acerca del
testimonio y su valor: \citalitlar{there is no species of reasoning more common,
  more useful, and even necessary to human life, than that which is derived from
  the testimony of men, and the reports of eye witnesses and spectators. This
  species of reasoning, perhaps, one may deny to be founded on the relation of
  cause and effect. I shall not dispute about a word. It will be sufficient to
  observe, that our assurance in any argument of this kind, is derived from no
  other principle than our observation of the veracity of human testimony, and
  of the usual conformity of facts to the reports of witnesses. It being a
  general maxim, that no objects have any discoverable connection together, and
  that all the inferences which we can draw from one to another, are founded
  merely on our experience of their constant and regular conjunction; it is
  evident, that we ought not to make an exception to this maxim in favour of
  human testimony, whose connection with any event seems, in itself, as little
  necessary as any other. Were not the memory tenacious to a certain degree; had
  not men commonly an inclination to truth and a principle of probity; were they
  not sensible to shame, when detected in a falsehood; were not these, I say,
  discovered by experience to be qualities inherent in human nature, we should
  never repose the least confidence in human testimony. A man delirious, or
  noted for falsehood and villany, has no manner of authority with
  us.\footnote{enquiry ch X part I}} Así como nuestra habitual experiencia de la
relación de causa y efecto nos permite hacer inferencias acerca de las
cuestiones de hecho que están más allá de nuestros sentidos, la conformidad que
usualmente experimentamos entre los hechos y el informe que un testigo nos da de
ellos nos permite inferir su veracidad. Según el análasis ofrecido por
C.\,A.\,J.~Coady, la teoría de Hume: \citalitlar{constitutes a reduction of
  testimony as a form of evidence or support to the status of a species (one
  might almost say, a mutation) of inductive inference. And, again, in so far as
  inductive inference is reduced by Hume to a species of observation and
  consequences attendant upon observations, then in a like fashion testimony
  meets the same fate.\footnote{testimony p. 79}}

La valoración epistemológica del testimonio y la perspectiva ofrecida por Hume
nos deja así con un primer desafío: \citalitlar{en la vida social cabe aceptar
  un conocimiento por testimonio a condición de que su grado de certeza se
  limite a la probabilidad, y a condición de que pueda ser siempre reconducido a
  una verificación por conocimiento
  directo}.\autocite[294]{prades2015testimonio} Estas consideraciones añaden
algunos elementos a nuestro cuestionamiento original. Conocer una verdad para la
vida desde el testimonio implica que pueda obtenerse una creencia verdadera
justificada basada en lo que una persona ha comunicado. La visión de Hume es que
la evidencia que puede ofrecer un testimonio para justificar una creencia no es
mayor que la probabilidad y esta evidencia está basada en la inferencia que nos
permite la habitual experiencia de que el testimonio comunicado y la verdad de
los hechos suelen ir unidos. Más adelante veremos qué tiene que decir Anscombe
ante este desafío. Todavía podemos plantear una segunda cuestión esta vez
relacionada con la segunda parte de nuestra pregunta original.

\subsection{¿Tiene fuerza un testimonio histórico del absoluto?}
El contexto de la reflexión de Hume sobre el testimonio es precisamente el de la
creencia en los milagros. La preocupación de Hume es que el `hombre sabio' pueda
verificar sus creencias de modo que no sea víctima de `engaños supersticiosos'.
Para esto, estima, que ha encontrado un argumento que servirá para distinguir
superstición de verdad.\footnote{I flatter myself, that I have discovered an
  argument of a like nature, which, if just, will, with the wise and learned, be
  an everlasting check to all kinds of superstitious delusion, and consequently
  will be useful as long as the world endures.} Dice: \citalitinterlin{in our
  reasonings concerning matter of fact, there are all imaginable degrees of
  assurance, from the highest certainty to the lowest species of moral evidence.
  A wise man, therefore, proportions his belief to the evidence}. Entonces
sugiere un criterio que permite ajustar las creencias a la evidencia:
\citalitlar{“That no testimony is sufficient to establish a miracle, unless the
  testimony be of such a kind, that its falsehood would be more miraculous than
  the fact which it endeavours to establish; and, even in that case, there is a
  mutual destruction of arguments; and the superior only gives us an assurance
  suitable to that degree of force which remains after deducting the inferior.”}
Esto tiene como consecuencia que lo razonable sea abandonar la razonabilidad de
las verdades cristianas, comprendiendo que solo pueden ser contempladas desde la
fe. Empleando su criterio ofrece una valoración de la revelación de la escritura
como sigue:
\citalitlar{I am the better pleased with the method of reasoning here delivered,
  as I think it may serve to confound those dangerous friends, or disguised
  enemies to the Christian religion, who have undertaken to defend it by the
  principles of human reason. Our most holy religion is founded on faith, not on
  reason; and it is a sure method of exposing it, to put it to such a trial as
  it is by no means fitted to endure. To make this more evident, let us examine
  those miracles related in Scripture; and, not to lose ourselves in too wide a
  field, let us confine ourselves to such as we find in the Pentateuch, which we
  shall examine according to the principles of these pretended Christians, not
  as the word or testimony of God himself, but as the production of a mere human
  writer and historian. Here then we are first to consider a book, presented to
  us by a barbarous and ignorant people, written in an age when they were still
  more barbarous, and in all probability long after the facts which it relates,
  corroborated by no concurring testimony, and resembling those fabulous
  accounts which every nation gives of its origin. Upon reading this book, we
  find it full of prodigies and miracles. It gives an account of a state of the
  world and of human nature entirely different from the present: of our fall
  from that state; of the age of man extended to near a thousand years; of the
  destruction of the world by a deluge; of the arbitrary choice of one people,
  as the favourites of heaven, and that people the countrymen of the author; of
  their deliverance from
  bondage by prodigies the most astonishing imaginable.\\
  I desire any one to lay his hand upon his heart, and, after a serious
  consideration, declare, whether he thinks that the falsehood of such a book,
  supported by such a testimony, would be more extraordinary and miraculous than
  all the miracles it relates; which is, however, necessary to make it be
  received according to the measures of probability above established.}
¿Se puede afirmar que sería más ``milagrosa'' la falsedad de los milagros que
atestigua la escritura? La posibilidad de recibir este testimonio como evidencia
de alguna verdad descansaría sobre esta condición y una persona razonable
debería medir la probabilidad de veracidad de estos relatos teniendo en cuenta
que el estado de las cosas que describe es distinto al que experimentamos en el
presente.

En una línea similar de pensamiento encontramos las reflexiones de
G.\,E.~Lessing. Dos cuestiones expresadas en ``On the proof of the spirit and
power'' merecen ser destacadas:
\citalitlar{The problem is that reports of fulfilled prophecies are not
  fullfiled prophecies; that reports of miracles are not miracles. These, the
  prophecies fulfilled before my eyes, the miracles that occur before my eyes,
  are immediate in their effect. But those---the reports of fulfilled prophecies
  and miracles, have to work through a medium which takes away all their force}
Lo que debería tener la fuerza para justificar la credibilidad queda debilitado
por su medio de transmisión, entonces \citalitinterlin{the problem is that this
  proof of the spirit and of power no longer has any spirit or power, but has
  sunk to the level of human testimonies of spirit and power}. Tal como lo
plantea Lessing y teniendo en cuenta el criterio propuesto por Hume, el
testimonio, en tanto que dinamismo humano, no tiene fuerza suficiente para
justificar razonablemente creencias sobre Dios como verdadero conocimiento.

Esta objeción nos lleva a la siguiente:
\citalitlar{the reports which we have of these prophecies and miracles are as
  reliable as historical truths can ever be [\ldots] But if they are as reliable
  as this, why are they treated as if they were infinitely more reliable?
  [\ldots] If no historical truth can be demonstrated, then nothing can be
  demonstrated by means of historical truths. That is: \emph{accidental truths
    of history can never become proof of necessary truths of reason.}}
El punto que Lessing señala es infranqueable para su propio intento de
comprometerse con la verdad que la creencia cristiana pretende comunicar. La
singularidad de la persona y obra de Jesús como manifestación de la realidad de
Dios pierde para él toda su fuerza, puesto que no puede estimar estas verdades
históricas como fundamento para una verdad necesaria como lo es la verdad de
Dios.

Esto nos deja con una segunda problemática: \citalitinterlin{no se puede tener
  conocimiento directo de milagros y profecias [\ldots] no se puede aceptar una
  comunicación divina que no sea inmediatamente dirigida al
  individuo}\autocite[294]{prades2015testimonio}. Este desafío viene a poner en
cuestión que un hecho histórico de la vida personal o colectiva pueda ser
estimado como testimonio del absoluto. La revelación de Dios por medio de
testigos no es un fenómeno que tenga justificación razonable para su veracidad,
y por tanto sólo puede ser acogida por una fe desconectada de la razón.

\subsection{¿Tiene carácter veritativo el lenguaje teológico?}
Una problemática adicional está representada en la crítica al lenguaje religioso
planteada por el Círculo de Viena. A\,J.~Ayer lo expresa como sigue:
\blockquote[{\cite[155]{dominguez2009at}}]{Si la existencia de tal dios fuese
  probable, la proposición de que existiera sería una hipótesis empírica. Y, en
  ese caso, sería posible deducir de ella, y de otras hipótesis científicas,
  ciertas proposiciones experienciales que no fuesen deducibles de esas otras
  hipótesis solas. Pero, en realidadm esto no es posible. [\ldots] Porque decir
  que ``Dios existe'' es realizar una expresión metafísica que no pude ser ni
  verdadera ni falsa. Y, según el mismo criterio, ninguna oración que pretenda
  describir la naturaleza de un Dios trascendente puede poseer ninguna
  significación literal.}

La intención del Círculo es la unificación de la ciencia y como herramienta para
este trabajo, empleó el análisis del lenguaje. Un análisis de la expresión
linguística empleada en el conocimiento científico permite distinguir las
afirmaciones que pueden tener valor veritativo de las que no contienen esta
posibilidad y, por tanto, no son lenguaje significativo. A. Flew, por ejemplo,
considera que la Teología no es un lenguaje significativo poruqe no es posible
su falsabilidad. De este modo:
\blockquote[{\cite[155]{dominguez2009at}}]{La crítica del Círculo de Viena no se
  suma al ``Dios ha muerto'' de Nietzsche, sino que va aún más allá: lo que ha
  muerto es la misma palabra: ``Dios''. Nos encontramos ante lo que podemos
  considerar una nueva y refinada especie de ateísmo: el ateísmo semántico. Esta
  forma de ateísmo se sustenta en un equivocismo hermenéutico. No cabe comparar,
  arguyen los equivocistas, los nombres de supuestas realidades trascendentes
  con los de las realidades empíricas.}
