\section{Naturaleza de la pregunta sobre el testimonio}
  Es una experiencia familiar en nuestras comunidades reunirnos en torno a la
  Sagrada Escritura y compartir la Palabra buscando en ella luz para nuestro
  presente. Una escena evangélica en torno a la cual muchos se han reunido a
  escuchar al Señor es la narración de Mateo del comienzo de la misión pública de
  Jesús y la llamada de los primeros discípulos:

  \citalitlar{Al enterarse Jesús de que habían arrestado a Juan se retiró a
    Galilea. Dejando Nazaret se estableció en Cafarnaún, junto al mar, en el
    territorio de Zabulón y Neftalí, para que se cumpliera lo dicho por medio del
    profeta Isaías:\\
    <<Tierra de Zabulón y tierra de Neftalí, camino del mar, al otro lado del
    Jordán, Galilea de los gentiles. El pueblo que habitaba en tinieblas vio una
    luz grande; a los que habitaban en tierra y sombras de
    muerte, una luz les brilló>>.\\
    Desde entonces comenzó Jesús a predicar diciendo: <<Convertíos, porque está
    cerca el reino de los cielos>>.\\
    Paseando junto al mar de Galilea vio a dos hermanos, a Simón, llamado Pedro, y
    a Andrés, que estaban echando la red en el mar, pues eran pescadores. Les
    dijo: <<Venid en pos de mí y os haré pescadores de hombres>>. Inmediatamente
    dejaron las redes y lo siguieron. Y pasando adelante vio a otros dos hermanos,
    a Santiago, hijo de Zebedeo, y a Juan, su hermano, que estaban en la barca
    repasando las redes con Zebedeo, su padre, y los llamó. Inmediatamente dejaron
    la barca y a su padre y lo siguieron.\footnote{Mt~4,12--22}}

  No sería difícil ahora visualizar una variedad de escenarios en los que este
  texto pueda ser discutido en nuestro contexto eclesial. Es proclamado, por
  ejemplo, en el ciclo A el III Domingo del Tiempo Ordinario. Es así que puede
  escucharse en las reflexiones del Papa Francisco en el Ángelus en la Plaza de
  San Pedro, donde destaca el hecho de que la misión de Jesús comience en una zona
  periférica:
  \citalitlar{Es una tierra de frontera, una zona de tránsito donde se encuentran
    personas diversas por raza, cultura y religión. La Galilea se convierte así en
    el lugar simbólico para la apertura del Evangelio a todos los pueblos. Desde
    este punto de vista, Galilea se asemeja al mundo de hoy: presencia simultánea
    de diversas culturas, necesidad de confrontación y necesidad de encuentro.
    También nosotros estamos inmersos cada día en una <<Galilea de los gentiles>>,
    y en este tipo de contexto podemos asustarnos y ceder a la tentación de
    construir recintos para estar más seguros, más protegidos. Pero Jesús nos
    enseña que la Buena Noticia, que Él trae, no está reservada a una parte de la
    humanidad, sino que se ha de comunicar a todos. Es un feliz anuncio destinado
    a quienes lo esperan, pero también a quienes tal vez ya no esperan nada y no
    tienen ni siquiera la fuerza de buscar y pedir.\footnote{PAPA FRANCISCO
      ÁNGELUS Plaza de San Pedro Domingo 26 de enero de 2014}}

  Tambíen el Papa Benedicto XVI ofreció su comentario y se fijó en la fuerza de esa
  noticia que Cristo comenzaba a anunciar:
  \citalitlar{El término ``evangelio'', en tiempos de Jesús, lo usaban los
    emperadores romanos para sus proclamas. Independientemente de su contenido, se
    definían ``buenas nuevas'', es decir, anuncios de salvación, porque el
    emperador era considerado el señor del mundo, y sus edictos, buenos presagios.
    Por eso, aplicar esta palabra a la predicación de Jesús asumió un sentido
    fuertemente crítico, como para decir: Dios, no el emperador, es el Señor del
    mundo, y el
    verdadero Evangelio es el de Jesucristo.\\
    La ``buena nueva'' que Jesús proclama se resume en estas palabras: ``El reino
    de Dios —-o reino de los cielos-— está cerca''. ¿Qué significa esta expresión?
    Ciertamente, no indica un reino terreno, delimitado en el espacio y en el
    tiempo; anuncia que Dios es quien reina, que Dios es el Señor,
    y que su señorío está presente, es actual, se está realizando.\\
    Por tanto, la novedad del mensaje de Cristo es que en él Dios se ha hecho
    cercano, que ya reina en medio de nosotros, como lo demuestran los milagros y
    las curaciones que realiza.\footnote{BENEDICTO XVI ÁNGELUS Plaza de San Pedro
      Domingo 27 de enero de 2008}}

  No sólo en San Pedro, sino que también podría encontrarse este texto en la
  celebración de la eucaristía domincal resonando en las comunidades y parroquias;
  en las homilias, oraciones, reflexiones o cánticos, invitando a la conversión y
  haciendo nueva la invitación de Jesús: <<Convertíos, porque está cerca el reino
  de los cielos>>. Quizás tambíen se le oiga entre algún grupo juvenil donde
  Simón, Andrés, Santiago y Juan sean tratados como modelos de vocación a la vida
  consagrada o al apostolado, atendiendo con entusiasmo cómo lo dejaron todo en el
  momento para seguir a Jesús. Seguramente algún joven reconociendo aquella
  llamada: <<Venid en pos de mí y os haré pescadores de hombres>> sonando como voz
  dentro de sí.

  El texto de la Escritura es tratado en estos contextos como testimonio de la
  vida de Jesucristo y de la vida de aquellos que le llaman maestro y que
  participan de su misión. No son, sin embargo, tratados como historias del
  pasado, sino como palabras para el presente. Es hoy que la Buena Noticia no está
  reservada a una parte de la humanidad, sino que ha de comunicarse a todos como
  insiste el Papa Francisco. Es hoy que Dios se hace cercano en Cristo para reinar
  en medio de nosotros como enseñó Benedicto XVI. Es hoy que Jesús nos invita a la
  conversión y a ir en pos de él.

  Es sobre esta costumbre de la Iglesia que ha de formularse ahora una pregunta.
  Resultará apropiado apelar aquí a otra costumbre de la Iglesia y buscar luz para
  esto en las Confesiones de San Agustín. Pensando en Dios y pensando en el
  tiempo, Agustín queda inquieto por una serie de preguntas: \citalitlar{¿Qué es,
    pues, el tiempo? ¿Quién podrá explicar esto fácil y brevemente? ¿Quién podrá
    comprenderlo con el pensamiento, para hablar luego de él? Y, sin embargo, ¿qué
    cosa más familiar y conocida mentamos en nuestras conversaciones que el
    tiempo? Y cuando hablamos de él, sabemos sin duda qué es, como sabemos o
    entendemos lo que es cuando lo oímos pronunciar a otro. ¿Qué es, pues, el
    tiempo? Si nadie me lo pregunta, lo sé; pero si quiero explicárselo al que me
    lo pregunta, no lo sé.\footnote{De las confesiones xi.14 (n. 17)}}

  Agustín expresa su extrañeza de que un concepto empleado ordinariamente se
  torne tan desconocido cuando llega la hora de explicarlo. ``¿Qué es el
  tiempo?'' o ``¿qué es conocer?'', ``¿la libertad?'' y ``¿qué es la fe?'' son
  preguntas de este tipo; distintas, por ejemplo, a ``¿cuál es el peso exacto de
  este objeto?'' o ``¿quién será la próxima persona en entrar por esa
  puerta?''.\footnote{cf. Wittgenstein BT. p.304} Preguntar ``¿qué es conocer una
  verdad para la vida por el testimonio de la Escritura?'' sería, como la pregunta
  agustiniana sobre el tiempo, una pregunta sobre la naturaleza o esencia de
  este fenómeno. Un concepto familiar en la vida de la Iglesia como el
  testimonio queda enmarcado como problema cuando nos acercamos a él queriendo
  comprender su esencia.

  Para continuar explorando la naturaleza de la pregunta sobre el testimonio
  resultará útil recurrir aquí al modo en que el psicólogo William James formula
  algunas preguntas sobre la Escritura al comienzo de sus conferencias sobre la
  \emph{religion natural}. Apelando a la literatura de lógica de su época a
  comienzos del siglo XX distingue dos niveles de investigación sobre cualquier
  tema: aquellas preguntas que se resuelven por medio de prposiciones
  \emph{existenciales}, como ``¿qué constitución, qué origen, qué historia tiene
  esto?'' o ``¿cómo se ha realizado esto?''; en segundo lugar las preguntas que se
  responden con proposiciones de \emph{valor} como ``¿cuál es la importancia,
  sentido o significado actual de esto?''. A este segundo juicio James lo denomina
  \emph{juicio espiritual}. Aplicando esta distinción a la Biblia se cuestiona:

  \citalitlar{ <<¿Bajo qué condiciones biográficas los escritores sagrados aportan
    sus diferentes contribuciones al volumen sacro?>>, <<¿Cúal era exactamente el
    contenido intelectual de sus declaraciones en cada caso particular?>>. Por
    supuesto, éstas son preguntas sobre hechos históricos y no vemos cómo las
    respuestas pueden resolver, de súbito, la última pregunta: <<¿De qué modo este
    libro, que nace de la forma descrita, puede ser una guía para nuestra vida y
    una revelación?>>. Para contestar habríamos de poseer alguna teoría general
    que nos mostrara con qué peculiaridades ha de contar una cosa para adquirir
    valor en lo que concierne a la revelación; y, en ella misma, tal teoría sería
    lo que antes hemos denominado un juicio espiritual.\footnote{William James
      Variedades de la Experiencia Religiosa p. 27} }

  Desde esta perspectiva la pregunta sobre cómo el testimonio de la escritura
  puede ser una guía para nuestra vida es una investigación sobre la importancia,
  sentido o significado que ésta tiene actualmente. La respuesta emitida en
  conclusión sería un juicio de valor sobre el fenómeno del testimonio. James
  propone que sería necesaria una teoría general que explicara qué características
  ha de tener alguna cosa para que merezca ser valorada como revelación. Así
  planteado, la pregunta sobre el testimonio sería atendida adecuadamente por
  medio de una investigación que indagara dentro de este fenómeno para descubrir
  los elementos que le otorgan el valor adecuado como para ser considerado guía
  para nuestra vida o una revelación. La explicación de dichos elementos
  configurarían una teoría que nos permitiría juzgar un testimonio concreto como
  valioso, o no, como guía o revelación para nuestras vidas.

  La ruta sugerida por este modo de conducir esta investigación, sin embargo, nos
  dejaría lejos del modo en que Elizabeth Anscombe se plantea un problema
  filosófico. En el trasfondo de su metodología filosófica está la propuesta por
  Ludwig Wittgenstein. Aunque se verá con más detalle qué implica esto, es
  necesario anticipar ahora algo acerca del modo en que ambos se encaminan a la
  hora de atender una investigación filosófica.

  En \emph{Investigaciones Filosóficas} \S89 Wittgenstein hace referencia al
  texto antes citado de las Confesiones para describir la peculiaridad de las
  preguntas filosóficas:
  \citalitlar{ Augustine says in \emph{Confessions} XI. 14, ``quid est ergo
    tempus? si nemo ex me quaerat scio; si quaerenti explicare velim nescio''.
    --This could not be said about a question of natural science (``What is the
    specific gravity of hydrogen'', for instance). Something that one knows when
    nobody asks one but no longer knows when one is asked to explain it, is
    something that has to be \emph{called to mind}. (And it is obviously
    something which, for some reason, it is difficult to call to mind.)}

  Para Wittgenstein es de gran importancia atender el paso que damos para
  resolver la perplejidad causada por el reclamo de explicar un fenómeno. El
  deseo de aclararlo nos puede impulsar a buscar una explicación dentro del
  fenómeno mismo, o cómo él diría: \citalitinterlin{We feel as if we had to see
    right into phenomena}.\footnote{\S90} Esta predisposición nos puede conducir
  a ignorar la amplitud del modo en que el lenguaje sobre esto es empleado en la
  actividad humana y a enfocarnos sólo en un elemento particular del lenguaje
  sobre este fenómeno y tomarlo como un ejemplo paradigmático para construir un
  modelo abstrayendo explicaciones y generalizaciones sobre él. Esta manera de
  indagar, le parece a Wittgenstein, nos hunde cada vez más profundamente en un
  estado de frustración y confusión filosófica de modo que llegamos a imaginar
  que para alcanzar claridad \citalitinterlin{we have to describe extreme
    subtleties, which again we are quite unable to describe with the means at
    our disposal. We feel as if we had to repair a torn spider's web with our
    fingers.}\footnote{\S106}

  La alternativa que Wittgenstein propone es una investigación que no esté
  dirigida hacia dentro del fenómeno, sino \citalitinterlin{as one might say,
    towards the \emph{`possibilities'} of phenomena. What that means is that we
    call to mind the \emph{kinds of statement} that we make about phenomena}. A
  este esfuerzo le denomina ``investigación gramática''. La describe de este modo:
  \citalitlar{ Our inquiry is therefore a grammatical one. And this inquiry sheds
    light on our problem by clearing misunderstandings away. Misunderstandings
    concerning the use of words, brought about, among other things, by certain
    analogies between the forms of expression in different regions of our
    language. -- Some of them can be removed by substituting one form of
    expression for another; this may be called `analysing' our forms of
    expression, for sometimes this procedure resembles taking things
    apart.\footnote{\S90}} El modo de salir de nuestra perplejidad, por tanto,
  consiste en prestar cuidadosa atención al uso que hacemos de hecho con las
  palabras y la aplicación que empleamos de las expresiones. Esto está al
  descubierto en nuestro uso del lenguaje de modo que la dificultad para
  \emph{traer a la mente} aquello que aclare un fenómeno no está en descubrir algo
  oculto en éste, sino en aprender a valorar lo que tenemos ante nuestra vista:
  \citalitinterlin{The aspects of things that are most important for us are hidden
    because of their simplicity and familiarity. (One is unable to notice
    something -- because it is always before one's eyes.)}\footnote{\S129} La
  descripción de los hechos concernientes al uso del lenguaje en nuestra actividad
  humana ordinaria componen los pasos del tipo de investigación sugerido por
  Wittgenstein. Hay cierta insatisfacción en este modo de proceder, como él mismo
  afirma: \citalitlar{Where does this investigation get its importance from, given
    that it seems only to destroy everything interesting: that is, all that is
    great and important? (As it were, all the buildings, leaving behind only bits
    of stone and rubble.) But what we are destroying are only houses of cards, and
    we are
    clearing up the ground of language on which they stood.\\
    The results of philosophy are the discovery of some piece of plain nonsense
    and the bumps that the understanding has got running up against the limit of
    language. They -- these bumps -- make us see the value of that discovery.}

  Anscombe, al igual que Wittgenstein, no se limita a emplear un sólo método para
  hacer filosofía, como afirma el mismo Wittgenstein: \citalitinterlin{There is
    not a single philosophical method, though there are indeed methods, different
    therapies as it were}.\footnote{\S133} Sin embargo si atendemos a su modo de
  hacer filosofía podemos encontrarla empleando lenguajes o juegos de lenguaje
  imaginarios para arrojar luz sobre modos actuales de usar el lenguaje o esquemas
  conceptuales; del mismo modo su trabajo esta lleno de ejemplos donde la
  encontramos examinando con detenimiento el uso que de hecho hacemos del
  lenguaje.\footnote{cf. teichmann p. 228-229} Es visible en ella ese
  \citalitinterlin{modo característicamente Wittgensteniano de rebatir la
    tendencia del filósofo de explicar alguna cuestión filosóficamente enigmática
    inventando una entidad o evento que la causa, así como los físicos inventan
    partículas como el gravitón}.\footnote{There is however a somehow
    chracteristically Wittgenstenian way of countering the philosopher's tendency
    to explain a philosophically puzzling thing by inventing an entity or event
    which causes it, as physicists invent particles like the graviton. From plato
    to witt intro xix}

  Según el título de este trabajo ha prometido, el análisis sobre el testimonio
  que será expuesto es el que se encuentra desarrollado en el pensamiento de
  Elizabeth Anscombe. La pregunta planteada al inicio: ¿qué es conocer una verdad
  para la vida por el testimonio de la Escritura?, entendida como investigación
  filosófica, será examinada en las descripiciones que Anscombe realiza sobre el
  modo de usar el lenguaje sobre el creer, la confianza, la verdad, la fe y otros
  fenómenos relacionados con el conocer por testimonio. Nuestro título adiverte
  además que ésta es una investigación en perspectiva teólogica, cabe
  inmendiatamente añadir algo breve al respecto.

  ¿Qué es teología?, se preguntaba Joseph Ratzinger en su alocución en el 75
  aniversario del nacimiento del cardenal Hermann Volk en 1978, e introducía
  suscintamente su respuesta a esa pregunta tan grande diciendo:

  \citalitlar{Cuando se intenta decir algo sobre esta materia, precisamente como
    tributo al cardenal Volk y a su pensamiento, se asocian, poco menos que
    automáticamente, dos ideas. Me viene a las mientes, por un lado, su divisa (y
    título de uno de sus libros): \emph{Dios todo en todos}, y el programa
    espiritual contenido en ella; por otra parte, se aviva el recuerdo de lo que
    ya antes se ha insinuado: un modo de interrogar total y absolutamente
    filosófico, que no se detiene en reales o supuestas comprobaciones históricas,
    en diagnósticos sociológicos o en técnicas pastorales, sino que se lanza
    implacablemente a la busqueda de los fundamentos.\\
    Según esto, cabría formular ya dos tesis que pueden servirnos de hilo
    conductor para nuestro interrogante sobre la esencia de la teología:\\
    1. La teología se refiere a Dios.\\
    2. El pensamiento teológico está vinculado al modo de cuestionar filosófico
    como a su método fundamental.\footnote{teoría de los principios teológicos, p
      380}}
  Esta investigación sobre el testimonio como parte de la vida de la Iglesia será
  realizada atendiendo al modo de cuestionar filosófico realizado por Elizabeth
  Anscombe como método, examinando esta experiencia en referencia a Dios, es
  decir, como vivencia de su ser y de su obrar.

  Hasta aquí simplemente se ha descrito un modo de andar a través de la discusión
  acerca de la categoría del testimonio atendiendo el hecho de que tanto la
  temática como la figura de Anscombe otorgan a este camino peculiaridades que hay
  que tener en cuenta. Siendo concientes de estas particularidades podríamos ahora
  ampliar más el horizonte respecto de dos cuestiones brevemente expuestas
  anteriormente. En primer lugar es necesario ampliar la descripción hecha hasta
  aquí del fenómeno del testimonio en la vida de la Iglesia, ya que aunque nos
  resulte familiar relacionarlo con el testimonio de la Sagrada Escritura, tanto
  en el Magisterio de la Iglesia como en la propia Escritura se haya presente la
  categoría del testimonio con una riqueza que merece la pena explorar. En segundo
  lugar habría que detallar todavía mejor lo problemático del testimonio, sobre
  todo cuando se considera su importancia en la transmisión de la fe y el anuncio
  del Evangelio en el mundo.

\section{La categoría del Testimonio en la Sagrada Escritura}
La Iglesia de hoy, como María, conserva el Evangelio meditándolo en su
corazón.\footnote{Lc 2,19} Así está presente en el centro de la comunidad
creyente el anuncio de Cristo vivo como fundamento de su esperanza en cada etapa
de la historia. Este motivo de esperanza conservado es también compartido y
expresado, según la enseñanza del apóstol:\citalitinterlin{glorificad a Cristo
  en vuestros corazones, dispuestos siempre a dar explicación a todo el que os
  pida una razón de vuestra esperanza}.\footnote{1Pe 3, 15} Este Evangelio
atesorado como fundamento en el centro de la vida de la comunidad eclesial, así
como Buena Nueva proclamada y transmitida en el tiempo y en el mundo puede ser
comprendido como tres testimonios que son uno:<<palabra vivida en el
Espíritu>>\footnote{cf. Porque es el Espíritu el que impulsa a la Iglesia a
  perseguir son obras de evangelización; es el Espíritu quien santifica y
  fecunda el testimonio de su vida; y es el Espíritu el que inspira la fe, la
  nutre y la profundiza. Es el Espíritu quien alivia entre estos tres
  testimonios que son uno: el de la palabra vivida en el Espíritu. A través del
  testimonio, el Espíritu internaliza el testimonio externo de la Buena Nueva de
  la salvación en Jesucristo y lo lleva al cumplimiento de la fe, que es la
  respuesta del amor del verdadero amor de la humanidad a través del Padre.
  Cristo; Latourelle Evangelisation et temoignage ninot 582}.

La Evangelización puede ser entendida en este sentido como testimonio de la
<<palabra de vida>>\footnote{1Jn 1,1} que los apóstoles anuncian como testigos
de lo que han contemplado y palpado\footnote{1Jn 1,1}. Es también el testimonio
del modo de vivir de los cristianos que, acogiendo esta palabra, la viven,
poniendo por obra lo que ella enseña. Es además testimonio del Espíritu Santo
que internaliza el testimonio externo de la Buena Noticia y lo lleva al
cumplimiento de la fe en cada persona.\footnote{cf. latourelle, ninot 582} Es el
Espíritu el que santifica y fecunda la acción de los cristianos, es tambíen el
que impulsa y sostiene la acción de la Iglesia; es el Espíritu el que inspira la
fe, la nutre y la profundiza.\footnote{latourelle evangelisation et temoignage}

Este dinamísmo fundamental que puede encontrarse vivo hoy en la comunidad de la
Iglesia ha actuado en ella desde su origen y le ha acompañado en cada época.
Según esto es posible valorar lo que se transmite en la tradición eclesial como
la perpetuación de la actividad de Cristo y los apóstoles, que es a su vez
proyección del testimonio divino.\footnote{ el testimonio divino se proyecta
  luego en el apostólico y se perpetúa en el testimonio eclesial. Por eso, el
  testimonio es revelación en la actividad de Cristo y de los apóstoles y es
  transmisión de la revelación en la tradición eclesial. ninot 573}

En la actividad de Cristo el testimonio divino queda proyectado como
interpelación a la libertad realizada por la identidad propia de Jesús:
\citalitinterlin{Si conocieras el don de Dios y quién es el que te dice ``dame
  de beber'' le pedirías tu, y él te daría agua viva}\footnote{Jn 4, 10};
\citalitinterlin{``¿Crees tú en el Hijo del hombre?''\ldots ``¿Y quién es,
  Señor, para que crea en él?''\ldots ``Lo estás viendo: el que te está
  hablando, ese es''}\footnote{Jn 9, 35--37}. En la actividad apostólica, el
testimonio divino sigue interpelando la libertad humana como manifestación de
Jesús Resucitado. Los apóstoles actuan como testigos de los acontecimientos de
la Pascua de Jesús y su valor salvífico\footnote{cf. ninot 576} y este
testimonio es descrito como acción del Espíritu que impulsa la tarea apostólica
y que da nueva vida a los que acogen el anuncio de la Buena Noticia.

Puede encontrarse un ejemplo de esto en el testimonio de Felipe. El apóstol sale
más allá de Jerusalén hacia Samaria, y todavía llega más lejos, al compartir la
Buena Noticia de Jesús con un extranjero Etíope: \citalitlar{El Espíritu dijo a
  Felipe: <<Acércate y pégate a la carroza>>. Felipe se acercó corriendo, le oyó
  leer el profeta Isaías, y le preguntó: <<¿Entiendes lo que estás leyendo?>>.
  Contestó: <<¿Y cómo voy a entenderlo si nadie me guía?>>. E invitó a Felipe a
  subir y a sentarse con él. El pasaje de la Escritura que estaba leyendo era
  este: \emph{Como cordero fue llevado al matadero, como oveja muda ante el
    esquilador, así no abre su boca. En su humillación no se le hizo justicia.
    ¿Quién podrá contar su descendencia? Pues su vida ha sido arrancada de la
    tierra.} El eunuco preguntó a Felipe: <<Por favor, ¿de quién dice esto el
  profeta?; ¿de él mismo o de otro?>>. Felipe se puso a hablarle y, tomando pie
  de este pasaje, le anunció la Buena Nueva de Jesús. Continuando el camino,
  llegaron a un sitio donde había agua, y dijo el eunuco: «Mira, agua. ¿Qué
  dificultad hay en que me bautice?». Mandó parar la carroza, bajaron los dos al
  agua, Felipe y el eunuco, y lo bautizó. Cuando salieron del agua, el Espíritu
  del Señor arrebató a Felipe. El eunuco no volvió a verlo, y siguió su camino
  lleno de alegría. \footnote{Hch 8, 29--39}} Además de ser ejemplo de la
actividad apostólica, este relato puede servir como síntesis del modo en que la
categoría del testimonio está presente en la Escritura.

El testimonio comienza con la iniciativa de Dios mismo que impulsa tanto la
palabra profética del Antiguo Testamento como el anuncio apostólico del Nuevo
Testamento. Esta iniciativa de Dios tiende hacia el testimonio de la Palabra
definitiva del Padre que es Cristo resucitado. En aquellos que creen en el
testimonio de Dios se engendra alegría y vida nueva. En palabras de R.
Latourelle:
\citalitlar{En el trato de las tres personas divinas con los hombres existe un
  intercambio de testimonios que tiene la finalidad de proponer la revelación y
  de alimentar la fe. Son tres los que revelan o dan testimonio, y esos tres son
  más que uno. Cristo da testimonio del Padre, mientras que el Padre y el
  Espíritu dan testimonio del Hijo. Los apóstoles a su vez dan testimonio de lo
  que han visto y oído del verbo de la vida. Pero su testimonio no es la
  comunicación de una ideología, de un descubrimiento científico, de una técnica
  inédita, sino la proclamación de la salvación prometida y finalmente
  realizada.\footnote{diccion testimonio p.1531}}
De este modo el anuncio del apóstol Felipe sirve aquí como un ejemplo específico
del testimonio, que ilustra sin embargo, una noción
que\citalitinterlin{atraviesa toda la Escritura y se corresponde con la
  estructura misma de la revelación.}\footnote{la noción de testimonio atraviesa
  la Escritura y se coresponde con la estructura misma de la Revelación: <<la
  Escritura describe la revelación como una economía del testimonio>>. prades
  109} El testimonio está presente a lo largo de la Escritura junto a otras
categorías como pueden ser la de `alianza', `palabra', `paternidad' o
`filiación', como parte del \citalitinterlin{grupo de analogías empleadas por la
  Escritura para introducir al hombre en las riquezas del misterio
  divino}.\footnote{latourelle p. 1523}

Esta clave servirá para dar enfoque a un examen sobre la categoría del
testimonio en la Escritura. ¿Qué nos dice el Antiguo y el Nuevo Testamento de la
revelación como acto testimonial de Dios? Esta pregunta supone que la revelación
comparte los rasgos de la actividad humana que es el testimonio, sin embargo,
como Latourelle adiverte: \citalitinterlin{globalmente se puede decir que el
  testimonio bíblico asume pero al mismo tiempo exalta hasta sublimarlos, los
  rasgos del testimonio humano.}\footnote{cf. latourelle 1526 Globalmente se
  puede decir que el testimonio bíblico asume pero al mismo tiempo exalta hasta
  sublimarlos, los rasgos del testimonio humano. latourelle 1526}

Cabe añadir una última consideración. La revelación de Dios entendida como acto
testimonial suyo tiene como expresión definitiva el misterio pascual de
Cristo.\footnote{cf. el misterio pascual al cual tiende toda la existencia
  terrena de Cristo, constituye el acto testimonial por excelencia de Dios
  prades 128} Este misterio ocupa el lugar principal en el testimonio bíblico:
\citalitlar{la Resurrección como ``final'' de la unicidad del acontecimiento de
  Jesucristo, encarnado, muerto y resucitado, subraya específicamente la
  definitividad de la existencia humana salvada por Dios en la carne de Jesús de
  Nazaret, ya que la autocomunicación de Dios ha alcanzado su palabra última en
  la Resurrección de Jesucristo, y por eso es prenda de la resurrección de todos
  los hombres.\footnote{ninot 404}}
Como tal, parece justo tratar el testimonio que es el misterio pascual en su
propio apartado. Y será éste precisamente el punto de partida para esta
descripción de la categoría del testimonio en la Escritura.

\subsection{El testimonio en el misterio y anuncio pascual}

\ifdraft{\subsubsection{Estatuto espistemológico especial}}{}
<<Cristo ha resucitado>>\footnote{Cf. 1 Tes 4,15; 1Cor 15,12--20; Rom 6,4} es la
confesión que está en el núcleo del más primitivo anuncio del
evangelio.\footnote{ninot 403} Creer en esta noticia conlleva acoger la
manifestación más plena de la Revelación y la motivación más definitiva para
creer. En este sentido:
\citalitlar{La Resurrección de Jesús mirada desde la perspectiva de la teología
  fundamental presupone un estatuto epistemológico peculiar, puesto que es el
  punto culminante y objeto de la Revelación y, a su vez, es su acreditación
  suprema y máximo motivo de credibilidad, tal como recuerda el texto citado de
  Pablo ``si Cristo no ha resucitado, nuestra predicación es vana y vana es
  nuestra fe'' (1 Cor 15,14).\footnote{ninot 405}}

Este misterio pascual no aparece como hecho desconectado del conjunto de la vida
y misión de Jesús, sino que hacia él tienden sus obras y palabras desde el
comienzo. Cristo pasó por el mundo haciendo el bien, como testimonio de la
bondad de Dios, y esta acción va orientada a ese punto culminante que es su
pasión, muerte y resurrección; \citalitinterlin{el testimonio que Jesús va
  ofreciendo durante su vida pública le va a reclamar una entrega definitiva a
  favor de los que o han acogido y frente a la resistencia que ha generado en
  quienes le rechazan.}\footnote{prades 127}

\ifdraft{\subsubsection{Es testimonio y motivo de credibilidad definitvo de
    confianza absoluta en el Padre}}{}

A lo largo de este camino Jesús manifiesta su confianza en el Padre:
\citalitinterlin{Padre, te doy gracias porque me has escuchado; yo sé que tu me
  escuchas siempre}\footnote{Jn 11, 41b-42a}; esta relación queda afirmada
plenamente ante la pasión como confianza puesta en la voluntad del Padre:
\citalitinterlin{Padre\ldots que no se haga mi voluntad, sino la
  tuya}\footnote{Lc 22,42}. De este modo en el misterio pascual queda
atestiguada la plena unidad de Cristo con el Padre, en la mayor confianza
imaginable.\footnote{prades 127}

\ifdraft{\subsubsection{Es testimonio y motivo de credibilidad definitvo del
    Amor de Dios}}{}

\ifdraft{\subsubsection{Es testimonio y motivo de credibilidad definitvo del
    proyecto de salvación deseado libremente por Dios}}{}

Si Cristo no ha resucitado
sería vana cualquier argumentación, sin embargo, Jesús es <<el Viviente>>,
estuvo muerto, pero vive por los siglos de los siglos.\footnote{Ap 1, 17--18}
Así Pedro da testimonio de ésto el día de Pentecostés: \citalitinterlin{A este
  Jesús lo resucitó Dios, de lo cual todos nosotros somos
  testigos}.\footnote{Hch 2, 32} El apóstol es testigo en la fe sobre un
acontecimiento enraizado en la historia.\footnote{ninot 402 y 406 enraizado}

Así mismo es presentado el testimonio de Pedro en casa de Cornelio donde el
centurión y todos lo que lo acompañaban esperaban reunidos para escuchar lo que
el Señor quisiera comunicarles por medio del apóstol. Pedro, comprendiendo que
la verdad de Dios no hace acepción de personas, narra los hechos que él bien
conoce: \citalitlar{<<Vosotros conocéis lo que sucedió en toda Judea, comenzando
  por Galilea, después del bautismo que predicó Juan. Me refiero a Jesús de
  Nazaret, ungido por Dios con la fuerza del Espíritu Santo, que pasó haciendo
  el bien y curando a todos los oprimidos por el diablo, porque Dios estaba con
  él. Nosotros somos testigos de todo lo que hizo en la tierra de los judíos y
  en Jerusalén. A este lo mataron, colgándolo de un madero. Pero Dios lo
  resucitó al tercer día y le concedió la gracia de manifestarse, no a todo el
  pueblo, sino a los testigos designados por Dios: a nosotros, que hemos comido
  y bebido con él después de su resurrección de entre los
  muertos.>>\footnote{Hch 10, 37--41}} Este testimonio de los hechos queda
enlazado con un testimonio de fe sobre el sentido profundo de lo que Pedro
conoce, Jesús, a quien los apóstoles y el pueblo vieron y escucharon es ahora
juez de vivos y muertos:
\citalitlar{<<Nos encargó predicar al pueblo, dando solemne testimonio de que
  Dios lo ha constituido juez de vivos y muertos. De él dan testimonio todos los
  profetas: que todos los que creen en él reciben, por su nombre, el perdón de
  los pecados.>>\footnote{Hch 10, 42-43}}

El apóstol entiende estos hechos y su alcance religioso y salvífico
interpretándolos en continuidad con la voluntad de Dios manifestada en su acción
en favor del pueblo judío a quién habló por medio de los profetas; voluntad
hecha manifiesta en definitva en \citalitinterlin{Jesús el Nazareno, varón
  acreditado por Dios ante vosotros con los milagros, prodigios y signos que
  Dios realizó por medio de él, como vosotros mismos sabéis}.\footnote{Hch 2,22}

La categoría del testimonio en el anuncio pascual es sobre un hecho enraizado en
la historia, que tiene un alcance religioso y salvífico y que es interpretado
desde la voluntad de Dios manifestada en los hechos y palabras de Cristo. Sin
las obras que Jesús realizó, el testimonio apostólico se derrumba, no
existe.\footnote{Latourelle 1529} Sin la vida y obra, muerte y resurrección de
Jesús \citalitinterlin{resultamos unos falsos testigos de Dios, porque hemos dado
  testimonio contra él, diciendo que ha resucitado a Cristo, a quien no ha
  resucitado}.\footnote{1 Cor 15, 15} Sin embargo, Jesús es testigo verdadero y
acreditado, en su resurreción se realiza en acto lo que ha
prometido\footnote{prades 128}, de modo que el testimonio apostólico queda
acreditado.

\subsection{El testimonio en el Antiguo Testamento}

  I. en el AT se nos describe la revelación como acto testimonial de Dios en estos términos:
  prades 114-115:

  Yahvé da testimonio de sí
  en las obras de la creación
  en la ley que son testimonio
  hombres escogidos como moisés

  II. en el AT la revelación es acto testimonial de Dios ante todo a través de los
  profetas

  latourelle: en el AT el testigo es ante todo el profeta

  el testigo es también el pueblo de Israel

Is 43, 8--12

  la autoridad del testigo no viene de él, sino de su vocación privilegiada y de
  su envío. prades 117

  lo que escinde este nuevo sentido del testimonio de todos sus usos en el
  lenguaje ordinario es que el testimonio no pertenece al testigo. Este procede de
  una iniciativa absoluta, en cuanto a su origen y en cuanto a su contenido.
  prades 118 - ricoeur

  en el AT el testimonio en su sentido más denso y sublime, es el de Dios mismo a
  través de personas escogidas por Él, que reminten continuamente a hechos
  acontecidos en la historia y a la interpretación que los acompaña, para
  reconocer con ello la presencia y actuación de Dios en la historia humana.

  Hay peligro de falsos profetas y también de corazones sordos. El pueblo es
  infiel y testarudo y actua irracionalmente rechazando las múltiples pruebas de
  la predilección divina.

\subsection{El testimonio en el Nuevo Testamento}
  El testimonio exterior va acompañado de un testimonio interior del Espíritu que
  hace al hombre capaz de abrirse al evangelio y de adherirse a él por la fe. 1530


!!!Aquí terminar otra vez con mateo y hablando de el universal concreto
