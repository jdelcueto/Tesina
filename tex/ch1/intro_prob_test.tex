\section{Naturaleza de la pregunta sobre el testimonio}
  Es una experiencia familiar en nuestras comunidades reunirnos en torno a la
  Sagrada Escritura y compartir la Palabra buscando en ella luz para nuestro
  presente. Una escena evangélica en torno a la cual muchos se han reunido a
  escuchar al Señor es la narración de Mateo del comienzo de la misión pública de
  Jesús y la llamada de los primeros discípulos:

  \citalitlar{Al enterarse Jesús de que habían arrestado a Juan se retiró a
    Galilea. Dejando Nazaret se estableció en Cafarnaún, junto al mar, en el
    territorio de Zabulón y Neftalí, para que se cumpliera lo dicho por medio del
    profeta Isaías:\\
    <<Tierra de Zabulón y tierra de Neftalí, camino del mar, al otro lado del
    Jordán, Galilea de los gentiles. El pueblo que habitaba en tinieblas vio una
    luz grande; a los que habitaban en tierra y sombras de
    muerte, una luz les brilló>>.\\
    Desde entonces comenzó Jesús a predicar diciendo: <<Convertíos, porque está
    cerca el reino de los cielos>>.\\
    Paseando junto al mar de Galilea vio a dos hermanos, a Simón, llamado Pedro, y
    a Andrés, que estaban echando la red en el mar, pues eran pescadores. Les
    dijo: <<Venid en pos de mí y os haré pescadores de hombres>>. Inmediatamente
    dejaron las redes y lo siguieron. Y pasando adelante vio a otros dos hermanos,
    a Santiago, hijo de Zebedeo, y a Juan, su hermano, que estaban en la barca
    repasando las redes con Zebedeo, su padre, y los llamó. Inmediatamente dejaron
    la barca y a su padre y lo siguieron.\footnote{Mt~4,12--22}}

  No sería difícil ahora visualizar una variedad de escenarios en los que este
  texto pueda ser discutido en nuestro contexto eclesial. Es proclamado, por
  ejemplo, en el ciclo A el III Domingo del Tiempo Ordinario. Es así que puede
  escucharse en las reflexiones del Papa Francisco en el Ángelus en la Plaza de
  San Pedro, donde destaca el hecho de que la misión de Jesús comience en una zona
  periférica:
  \citalitlar{Es una tierra de frontera, una zona de tránsito donde se
    encuentran personas diversas por raza, cultura y religión. La Galilea se
    convierte así en el lugar simbólico para la apertura del Evangelio a todos
    los pueblos. Desde este punto de vista, Galilea se asemeja al mundo de hoy:
    presencia simultánea de diversas culturas, necesidad de confrontación y
    necesidad de encuentro. También nosotros estamos inmersos cada día en una
    <<Galilea de los gentiles>>, y en este tipo de contexto podemos asustarnos y
    ceder a la tentación de construir recintos para estar más seguros, más
    protegidos. Pero Jesús nos enseña que la Buena Noticia, que Él trae, no está
    reservada a una parte de la humanidad, sino que se ha de comunicar a todos.
    Es un feliz anuncio destinado a quienes lo esperan, pero también a quienes
    tal vez ya no esperan nada y no tienen ni siquiera la fuerza de buscar y
    pedir.\autocite{francisco2014angelus}}

  Tambíen el Papa Benedicto XVI ofreció su comentario y se fijó en la fuerza de esa
  noticia que Cristo comenzaba a anunciar:
  \citalitlar{El término ``evangelio'', en tiempos de Jesús, lo usaban los
    emperadores romanos para sus proclamas. Independientemente de su contenido, se
    definían ``buenas nuevas'', es decir, anuncios de salvación, porque el
    emperador era considerado el señor del mundo, y sus edictos, buenos presagios.
    Por eso, aplicar esta palabra a la predicación de Jesús asumió un sentido
    fuertemente crítico, como para decir: Dios, no el emperador, es el Señor del
    mundo, y el
    verdadero Evangelio es el de Jesucristo.\\
    La ``buena nueva'' que Jesús proclama se resume en estas palabras: ``El reino
    de Dios —--o reino de los cielos-—- está cerca''. ¿Qué significa esta expresión?
    Ciertamente, no indica un reino terreno, delimitado en el espacio y en el
    tiempo; anuncia que Dios es quien reina, que Dios es el Señor,
    y que su señorío está presente, es actual, se está realizando.\\
    Por tanto, la novedad del mensaje de Cristo es que en él Dios se ha hecho
    cercano, que ya reina en medio de nosotros, como lo demuestran los milagros y
    las curaciones que realiza.\autocite{benedicto2008angelus}}

  No sólo en San Pedro, sino que también podría encontrarse este texto en la
  celebración de la eucaristía domincal resonando en las comunidades y parroquias;
  en las homilias, oraciones, reflexiones o cánticos, invitando a la conversión y
  haciendo nueva la invitación de Jesús: <<Convertíos, porque está cerca el reino
  de los cielos>>. Quizás tambíen se le oiga entre algún grupo juvenil donde
  Simón, Andrés, Santiago y Juan sean tratados como modelos de vocación a la vida
  consagrada o al apostolado, atendiendo con entusiasmo cómo lo dejaron todo en el
  momento para seguir a Jesús. Seguramente algún joven reconociendo aquella
  llamada: <<Venid en pos de mí y os haré pescadores de hombres>> sonando como voz
  dentro de sí.

  El texto de la Escritura es tratado en estos contextos como testimonio de la
  vida de Jesucristo y de la vida de aquellos que le llaman maestro y que
  participan de su misión. No son, sin embargo, tratados como historias del
  pasado, sino como palabras para el presente. Es hoy que la Buena Noticia no está
  reservada a una parte de la humanidad, sino que ha de comunicarse a todos como
  insiste el Papa Francisco. Es hoy que Dios se hace cercano en Cristo para reinar
  en medio de nosotros como enseñó Benedicto XVI. Es hoy que Jesús nos invita a la
  conversión y a ir en pos de él.

  Es sobre esta costumbre de la Iglesia que ha de formularse ahora una pregunta.
  Resultará apropiado apelar aquí a otra costumbre de la Iglesia y buscar luz para
  esto en las Confesiones de San Agustín. Pensando en Dios y pensando en el
  tiempo, Agustín queda inquieto por una serie de preguntas: \citalitlar{¿Qué es,
    pues, el tiempo? ¿Quién podrá explicar esto fácil y brevemente? ¿Quién podrá
    comprenderlo con el pensamiento, para hablar luego de él? Y, sin embargo, ¿qué
    cosa más familiar y conocida mentamos en nuestras conversaciones que el
    tiempo? Y cuando hablamos de él, sabemos sin duda qué es, como sabemos o
    entendemos lo que es cuando lo oímos pronunciar a otro. ¿Qué es, pues, el
    tiempo? Si nadie me lo pregunta, lo sé; pero si quiero explicárselo al que me
    lo pregunta, no lo sé.\footnote{De las confesiones xi.14 (n. 17)}}

  Agustín expresa su extrañeza de que un concepto empleado ordinariamente se
  torne tan desconocido cuando llega la hora de explicarlo. ``¿Qué es el
  tiempo?'' o ``¿qué es conocer?'', ``¿la libertad?'' y ``¿qué es la fe?'' son
  preguntas de este tipo; distintas, por ejemplo, a ``¿cuál es el peso exacto de
  este objeto?'' o ``¿quién será la próxima persona en entrar por esa
  puerta?''.\footnote{cf. Wittgenstein BT. p.304} Preguntar ``¿qué es conocer una
  verdad para la vida por el testimonio de la Escritura?'' sería, como la pregunta
  agustiniana sobre el tiempo, una pregunta sobre la naturaleza o esencia de
  este fenómeno. Un concepto familiar en la vida de la Iglesia como el
  testimonio queda enmarcado como problema cuando nos acercamos a él queriendo
  comprender su esencia.

  Para continuar explorando la naturaleza de la pregunta sobre el testimonio
  resultará útil recurrir aquí al modo en que el psicólogo William James formula
  algunas preguntas sobre la Escritura al comienzo de sus conferencias sobre la
  \emph{religion natural}. Apelando a la literatura de lógica de su época a
  comienzos del siglo XX distingue dos niveles de investigación sobre cualquier
  tema: aquellas preguntas que se resuelven por medio de prposiciones
  \emph{existenciales}, como ``¿qué constitución, qué origen, qué historia tiene
  esto?'' o ``¿cómo se ha realizado esto?''; en segundo lugar las preguntas que se
  responden con proposiciones de \emph{valor} como ``¿cuál es la importancia,
  sentido o significado actual de esto?''. A este segundo juicio James lo denomina
  \emph{juicio espiritual}. Aplicando esta distinción a la Biblia se cuestiona:

  \citalitlar{ <<¿Bajo qué condiciones biográficas los escritores sagrados aportan
    sus diferentes contribuciones al volumen sacro?>>, <<¿Cúal era exactamente el
    contenido intelectual de sus declaraciones en cada caso particular?>>. Por
    supuesto, éstas son preguntas sobre hechos históricos y no vemos cómo las
    respuestas pueden resolver, de súbito, la última pregunta: <<¿De qué modo este
    libro, que nace de la forma descrita, puede ser una guía para nuestra vida y
    una revelación?>>. Para contestar habríamos de poseer alguna teoría general
    que nos mostrara con qué peculiaridades ha de contar una cosa para adquirir
    valor en lo que concierne a la revelación; y, en ella misma, tal teoría sería
    lo que antes hemos denominado un juicio espiritual.\footnote{William James
      Variedades de la Experiencia Religiosa p. 27} }

  Desde esta perspectiva la pregunta sobre cómo el testimonio de la escritura
  puede ser una guía para nuestra vida es una investigación sobre la importancia,
  sentido o significado que ésta tiene actualmente. La respuesta emitida en
  conclusión sería un juicio de valor sobre el fenómeno del testimonio. James
  propone que sería necesaria una teoría general que explicara qué características
  ha de tener alguna cosa para que merezca ser valorada como revelación. Así
  planteado, la pregunta sobre el testimonio sería atendida adecuadamente por
  medio de una investigación que indagara dentro de este fenómeno para descubrir
  los elementos que le otorgan el valor adecuado como para ser considerado guía
  para nuestra vida o una revelación. La explicación de dichos elementos
  configurarían una teoría que nos permitiría juzgar un testimonio concreto como
  valioso, o no, como guía o revelación para nuestras vidas.

  La ruta sugerida por este modo de conducir la investigación, sin embargo, nos
  dejaría apartados de la manera en que Elizabeth Anscombe se plantea un problema
  filosófico. En el trasfondo de su metodología filosófica está la propuesta por
  Ludwig Wittgenstein. Aunque se verá con más detalle qué implica esto, es
  necesario anticipar ahora algo acerca del modo en que ambos se encaminan a la
  hora de atender una investigación filosófica.

  En \emph{Investigaciones Filosóficas} \S89 Wittgenstein hace referencia al
  texto antes citado de las Confesiones para describir la peculiaridad de las
  preguntas filosóficas:
  \citalitlar{ Augustine says in \emph{Confessions} XI. 14, ``quid est ergo
    tempus? si nemo ex me quaerat scio; si quaerenti explicare velim nescio''.
    --This could not be said about a question of natural science (``What is the
    specific gravity of hydrogen'', for instance). Something that one knows when
    nobody asks one but no longer knows when one is asked to explain it, is
    something that has to be \emph{called to mind}. (And it is obviously
    something which, for some reason, it is difficult to call to mind.)}

  Para Wittgenstein es de gran importancia atender el paso que damos para
  resolver la perplejidad causada por el reclamo de explicar un fenómeno. El
  deseo de aclararlo nos puede impulsar a buscar una explicación dentro del
  fenómeno mismo, o como él diría: \citalitinterlin{We feel as if we had to see
    right into phenomena}.\footnote{\S90} Esta predisposición nos puede conducir
  a ignorar la amplitud del modo en que el lenguaje sobre esto es empleado en la
  actividad humana y a enfocarnos sólo en un elemento particular del lenguaje
  sobre este fenómeno y tomarlo como un ejemplo paradigmático para construir un
  modelo abstrayendo explicaciones y generalizaciones sobre él. Esta manera de
  indagar, le parece a Wittgenstein, nos hunde cada vez más profundamente en un
  estado de frustración y confusión filosófica de modo que llegamos a imaginar
  que para alcanzar claridad \citalitinterlin{we have to describe extreme
    subtleties, which again we are quite unable to describe with the means at
    our disposal. We feel as if we had to repair a torn spider's web with our
    fingers.}\footnote{\S106}

  La alternativa que Wittgenstein propone es una investigación que no esté
  dirigida hacia dentro del fenómeno, sino \citalitinterlin{as one might say,
    towards the \emph{`possibilities'} of phenomena. What that means is that we
    call to mind the \emph{kinds of statement} that we make about phenomena}. A
  este esfuerzo le denomina ``investigación gramática''. La describe de este modo:
  \citalitlar{ Our inquiry is therefore a grammatical one. And this inquiry sheds
    light on our problem by clearing misunderstandings away. Misunderstandings
    concerning the use of words, brought about, among other things, by certain
    analogies between the forms of expression in different regions of our
    language. -- Some of them can be removed by substituting one form of
    expression for another; this may be called `analysing' our forms of
    expression, for sometimes this procedure resembles taking things
    apart.\footnote{\S90}} El modo de salir de nuestra perplejidad, por tanto,
  consiste en prestar cuidadosa atención al uso que hacemos de hecho con las
  palabras y la aplicación que empleamos de las expresiones. Esto está al
  descubierto en nuestro uso del lenguaje de modo que la dificultad para
  \emph{traer a la mente} aquello que aclare un fenómeno no está en descubrir algo
  oculto en éste, sino en aprender a valorar lo que tenemos ante nuestra vista:
  \citalitinterlin{The aspects of things that are most important for us are hidden
    because of their simplicity and familiarity. (One is unable to notice
    something -- because it is always before one's eyes.)}\footnote{\S129} La
  descripción de los hechos concernientes al uso del lenguaje en nuestra actividad
  humana ordinaria componen los pasos del tipo de investigación sugerido por
  Wittgenstein. Hay cierta insatisfacción en este modo de proceder, como él mismo
  afirma: \citalitlar{Where does this investigation get its importance from, given
    that it seems only to destroy everything interesting: that is, all that is
    great and important? (As it were, all the buildings, leaving behind only bits
    of stone and rubble.) But what we are destroying are only houses of cards, and
    we are
    clearing up the ground of language on which they stood.\\
    The results of philosophy are the discovery of some piece of plain nonsense
    and the bumps that the understanding has got running up against the limit of
    language. They -- these bumps -- make us see the value of that discovery.}

  Anscombe, al igual que Wittgenstein, no se limita a emplear un sólo método para
  hacer filosofía, como afirma el mismo Wittgenstein: \citalitinterlin{There is
    not a single philosophical method, though there are indeed methods, different
    therapies as it were}.\footnote{\S133} Sin embargo si atendemos a su modo de
  hacer filosofía podemos encontrarla empleando lenguajes o juegos de lenguaje
  imaginarios para arrojar luz sobre modos actuales de usar el lenguaje o esquemas
  conceptuales; del mismo modo su trabajo esta lleno de ejemplos donde la
  encontramos examinando con detenimiento el uso que de hecho hacemos del
  lenguaje.\footnote{cf. teichmann p. 228-229} Es visible en ella ese
  \citalitinterlin{modo característicamente Wittgensteniano de rebatir la
    tendencia del filósofo de explicar alguna cuestión filosóficamente enigmática
    inventando una entidad o evento que la causa, así como los físicos inventan
    partículas como el gravitón}.\footnote{There is however a somehow
    chracteristically Wittgenstenian way of countering the philosopher's tendency
    to explain a philosophically puzzling thing by inventing an entity or event
    which causes it, as physicists invent particles like the graviton. From plato
    to witt intro xix}

  Según el título de este trabajo ha prometido, el análisis sobre el testimonio
  que será expuesto es el que se encuentra desarrollado en el pensamiento de
  Elizabeth Anscombe. La pregunta planteada al inicio: ¿qué es conocer una verdad
  para la vida por el testimonio de la Escritura?, entendida como investigación
  filosófica, será examinada en las descripiciones que Anscombe realiza sobre el
  modo de usar el lenguaje sobre el creer, la confianza, la verdad, la fe y otros
  fenómenos relacionados con el conocer por testimonio. Nuestro título adiverte
  además que ésta es una investigación en perspectiva teólogica, cabe
  inmendiatamente añadir algo breve al respecto.

  ¿Qué es teología?, se preguntaba Joseph Ratzinger en su alocución en el 75
  aniversario del nacimiento del cardenal Hermann Volk en 1978, e introducía
  suscintamente su respuesta a esa pregunta tan grande diciendo:

  \citalitlar{Cuando se intenta decir algo sobre esta materia, precisamente como
    tributo al cardenal Volk y a su pensamiento, se asocian, poco menos que
    automáticamente, dos ideas. Me viene a las mientes, por un lado, su divisa (y
    título de uno de sus libros): \emph{Dios todo en todos}, y el programa
    espiritual contenido en ella; por otra parte, se aviva el recuerdo de lo que
    ya antes se ha insinuado: un modo de interrogar total y absolutamente
    filosófico, que no se detiene en reales o supuestas comprobaciones históricas,
    en diagnósticos sociológicos o en técnicas pastorales, sino que se lanza
    implacablemente a la busqueda de los fundamentos.\\
    Según esto, cabría formular ya dos tesis que pueden servirnos de hilo
    conductor para nuestro interrogante sobre la esencia de la teología:\\
    1. La teología se refiere a Dios.\\
    2. El pensamiento teológico está vinculado al modo de cuestionar filosófico
    como a su método fundamental.\footnote{teoría de los principios teológicos, p
      380}}
  Esta investigación sobre el testimonio como parte de la vida de la Iglesia será
  realizada atendiendo al modo de cuestionar filosófico realizado por Elizabeth
  Anscombe como método, examinando esta experiencia en referencia a Dios, es
  decir, como vivencia de su ser y de su obrar.

  Hasta aquí simplemente se ha descrito un modo de andar a través de la discusión
  acerca de la categoría del testimonio atendiendo el hecho de que tanto la
  temática como la figura de Anscombe otorgan a este camino peculiaridades que hay
  que tener en cuenta. Siendo concientes de estas particularidades podríamos ahora
  ampliar más el horizonte respecto de dos cuestiones brevemente expuestas
  anteriormente. En primer lugar es necesario ampliar la descripción hecha hasta
  aquí del fenómeno del testimonio en la vida de la Iglesia, ya que aunque nos
  resulte familiar relacionarlo con el testimonio de la Sagrada Escritura, tanto
  en el Magisterio de la Iglesia como en la propia Escritura se haya presente la
  categoría del testimonio con una riqueza que merece la pena explorar. En segundo
  lugar habría que detallar todavía mejor lo problemático del testimonio, sobre
  todo cuando se considera su importancia en la transmisión de la fe y el anuncio
  del Evangelio en el mundo.

\section{La Categoría del Testimonio en la Sagrada Escritura}
La Iglesia de hoy, como María, conserva el Evangelio meditándolo en su
corazón.\footnote{Lc 2,19} Así está presente en el centro de la comunidad
creyente el anuncio de Cristo vivo como fundamento de su esperanza en cada etapa
de la historia. Este motivo de esperanza conservado es también compartido y
expresado, según la enseñanza del apóstol:\citalitinterlin{glorificad a Cristo
  en vuestros corazones, dispuestos siempre a dar explicación a todo el que os
  pida una razón de vuestra esperanza}.\footnote{1Pe 3, 15} Este Evangelio
atesorado como fundamento en el centro de la vida de la comunidad eclesial, así
como Buena Nueva proclamada y transmitida en el tiempo y en el mundo puede ser
comprendido como tres testimonios que son uno:<<palabra vivida en el
Espíritu>>\footnote{cf. Porque es el Espíritu el que impulsa a la Iglesia a
  perseguir son obras de evangelización; es el Espíritu quien santifica y
  fecunda el testimonio de su vida; y es el Espíritu el que inspira la fe, la
  nutre y la profundiza. Es el Espíritu quien alivia entre estos tres
  testimonios que son uno: el de la palabra vivida en el Espíritu. A través del
  testimonio, el Espíritu internaliza el testimonio externo de la Buena Nueva de
  la salvación en Jesucristo y lo lleva al cumplimiento de la fe, que es la
  respuesta del amor del verdadero amor de la humanidad a través del Padre.
  Cristo; Latourelle Evangelisation et temoignage ninot 582}.

La Evangelización puede ser entendida en este sentido como testimonio de la
<<palabra de vida>>\footnote{1Jn 1,1} que los apóstoles anuncian como testigos
de lo que han contemplado y palpado\footnote{1Jn 1,1}. Es también el testimonio
de los cristianos que, acogiendo esta palabra, la viven,
poniendo por obra lo que ella enseña. Es además testimonio del Espíritu Santo
que internaliza el testimonio externo de la Buena Noticia y lo lleva al
cumplimiento de la fe en cada persona.\footnote{cf. latourelle, ninot 582} Es el
Espíritu el que santifica y fecunda la acción de los cristianos, es tambíen el
que impulsa y sostiene la acción de la Iglesia; es el Espíritu el que inspira la
fe, la nutre y la profundiza.\footnote{latourelle evangelisation et temoignage}

Este dinamísmo fundamental que puede encontrarse vivo hoy en la comunidad de la
Iglesia ha actuado en ella desde su origen y le ha acompañado en cada época.
Según esto es posible valorar lo que se transmite en la tradición eclesial como
la perpetuación de la actividad de Cristo y los apóstoles, que es a su vez
proyección del testimonio divino.\footnote{ el testimonio divino se proyecta
  luego en el apostólico y se perpetúa en el testimonio eclesial. Por eso, el
  testimonio es revelación en la actividad de Cristo y de los apóstoles y es
  transmisión de la revelación en la tradición eclesial. ninot 573}

En la actividad de Cristo el testimonio divino queda proyectado como
interpelación a la libertad realizada por la identidad propia de Jesús:
\citalitinterlin{Si conocieras el don de Dios y quién es el que te dice ``dame
  de beber'' le pedirías tu, y él te daría agua viva}\footnote{Jn 4, 10};
\citalitinterlin{``¿Crees tú en el Hijo del hombre?''\ldots ``¿Y quién es,
  Señor, para que crea en él?''\ldots ``Lo estás viendo: el que te está
  hablando, ese es''}\footnote{Jn 9, 35--37}. En la actividad apostólica, el
testimonio divino sigue interpelando la libertad humana como manifestación de
Jesús Resucitado. Los apóstoles actuan como testigos de los acontecimientos de
la Pascua de Jesús y su valor salvífico\autocite[Cf.][576]{ninot2009tf} y este
testimonio es descrito como acción del Espíritu que impulsa la tarea apostólica
y que da nueva vida a los que acogen el anuncio de la Buena Noticia.

Puede encontrarse un ejemplo de esto en el testimonio de Felipe. El apóstol sale
más allá de Jerusalén hacia Samaria, y todavía llega más lejos, al compartir la
Buena Noticia de Jesús con un extranjero Etíope: \citalitlar{El Espíritu dijo a
  Felipe: <<Acércate y pégate a la carroza>>. Felipe se acercó corriendo, le oyó
  leer el profeta Isaías, y le preguntó: <<¿Entiendes lo que estás leyendo?>>.
  Contestó: <<¿Y cómo voy a entenderlo si nadie me guía?>>. E invitó a Felipe a
  subir y a sentarse con él. El pasaje de la Escritura que estaba leyendo era
  este: \emph{Como cordero fue llevado al matadero, como oveja muda ante el
    esquilador, así no abre su boca. En su humillación no se le hizo justicia.
    ¿Quién podrá contar su descendencia? Pues su vida ha sido arrancada de la
    tierra.} El eunuco preguntó a Felipe: <<Por favor, ¿de quién dice esto el
  profeta?; ¿de él mismo o de otro?>>. Felipe se puso a hablarle y, tomando pie
  de este pasaje, le anunció la Buena Nueva de Jesús. Continuando el camino,
  llegaron a un sitio donde había agua, y dijo el eunuco: «Mira, agua. ¿Qué
  dificultad hay en que me bautice?». Mandó parar la carroza, bajaron los dos al
  agua, Felipe y el eunuco, y lo bautizó. Cuando salieron del agua, el Espíritu
  del Señor arrebató a Felipe. El eunuco no volvió a verlo, y siguió su camino
  lleno de alegría. \footnote{Hch 8, 29--39}} Además de ser ejemplo de la
actividad apostólica, este relato puede servir como síntesis del modo en que la
categoría del testimonio está presente en la Escritura.

El testimonio comienza con la iniciativa de Dios mismo que impulsa tanto la
palabra profética del Antiguo Testamento como el anuncio apostólico del Nuevo
Testamento. Esta iniciativa de Dios tiende hacia el testimonio de la Palabra
definitiva del Padre que es Cristo resucitado. En aquellos que creen en el
testimonio de Dios se engendra alegría y vida nueva. En palabras de R.
Latourelle:
\citalitlar{En el trato de las tres personas divinas con los hombres existe un
  intercambio de testimonios que tiene la finalidad de proponer la revelación y
  de alimentar la fe. Son tres los que revelan o dan testimonio, y esos tres son
  más que uno. Cristo da testimonio del Padre, mientras que el Padre y el
  Espíritu dan testimonio del Hijo. Los apóstoles a su vez dan testimonio de lo
  que han visto y oído del verbo de la vida. Pero su testimonio no es la
  comunicación de una ideología, de un descubrimiento científico, de una técnica
  inédita, sino la proclamación de la salvación prometida y finalmente
  realizada.\autocite[1531]{latourelle2000testimonio}}
De este modo el anuncio del apóstol Felipe sirve aquí como un ejemplo específico
del testimonio, que ilustra sin embargo, una noción
que\citalitinterlin{atraviesa toda la Escritura y se corresponde con la
  estructura misma de la revelación.}\footnote{la noción de testimonio atraviesa
  la Escritura y se coresponde con la estructura misma de la Revelación: <<la
  Escritura describe la revelación como una economía del testimonio>>.
  \autocite[109]{prades2015testimonio}} El testimonio está presente a lo largo
de la Escritura junto a otras categorías como pueden ser la de `alianza',
`palabra', `paternidad' o `filiación', como parte del \citalitinterlin{grupo de
  analogías empleadas por la Escritura para introducir al hombre en las riquezas
  del misterio divino}.\footnote{latourelle p. 1523}

Esta clave servirá para dar enfoque a un examen sobre la categoría del
testimonio en la Escritura. ¿Qué nos dice el Antiguo y el Nuevo Testamento de la
revelación como acto testimonial de Dios? Esta pregunta supone que la revelación
comparte los rasgos de la actividad humana que es el testimonio, sin embargo,
como Latourelle adiverte: \citalitinterlin{globalmente se puede decir que el
  testimonio bíblico asume pero al mismo tiempo exalta hasta sublimarlos, los
  rasgos del testimonio humano.}\footnote{cf. latourelle 1526 Globalmente se
  puede decir que el testimonio bíblico asume pero al mismo tiempo exalta hasta
  sublimarlos, los rasgos del testimonio humano. latourelle 1526}

Cabe añadir una última consideración. La revelación de Dios entendida como acto
testimonial suyo tiene como expresión definitiva el misterio pascual de
Cristo.\footnote{cf. el misterio pascual al cual tiende toda la existencia
  terrena de Cristo, constituye el acto testimonial por excelencia de Dios
  prades 128} Este misterio ocupa el lugar principal en el testimonio bíblico:
\citalitlar{la Resurrección como ``final'' de la unicidad del acontecimiento de
  Jesucristo, encarnado, muerto y resucitado, subraya específicamente la
  definitividad de la existencia humana salvada por Dios en la carne de Jesús de
  Nazaret, ya que la autocomunicación de Dios ha alcanzado su palabra última en
  la Resurrección de Jesucristo, y por eso es prenda de la resurrección de todos
  los hombres.\footnote{ninot 404}}
Como tal, parece justo tratar el testimonio que es el misterio pascual en su
propio apartado. Y será éste precisamente el punto de partida para esta
descripción de la categoría del testimonio en la Escritura.

\subsection{El testimonio en el misterio y anuncio pascual}

<<Cristo ha resucitado>>\footnote{Cf.~1Tes 4,15; 1Cor 15,12--20; Rom 6,4} es la
confesión que está en el núcleo del más primitivo anuncio del
evangelio.\autocite[Cf.][403]{ninot2009tf} Creer en esta noticia conlleva acoger la
manifestación más plena de la Revelación y la motivación más definitiva para
creer. En este sentido:
\citalitlar{La Resurrección de Jesús mirada desde la perspectiva de la teología
  fundamental presupone un estatuto epistemológico peculiar, puesto que es el
  punto culminante y objeto de la Revelación y, a su vez, es su acreditación
  suprema y máximo motivo de credibilidad, tal como recuerda el texto citado de
  Pablo ``si Cristo no ha resucitado, nuestra predicación es vana y vana es
  nuestra fe'' (1 Cor 15,14).\autocite[405]{ninot2009tf}}

Este misterio pascual no aparece como hecho desconectado del conjunto de la vida
y misión de Jesús, sino que hacia él tienden sus obras y palabras desde el
comienzo. Cristo pasó por el mundo haciendo el bien, como testimonio de la
bondad de Dios, y esta acción va orientada a ese punto culminante que es su
pasión, muerte y resurrección; \citalitinterlin{el testimonio que Jesús va
  ofreciendo durante su vida pública le va a reclamar una entrega definitiva a
  favor de los que lo han acogido y frente a la resistencia que ha generado en
  quienes le rechazan.}\autocite[127]{prades2015testimonio}

A lo largo de este camino Jesús manifiesta su confianza en el Padre:
\citalitinterlin{Padre, te doy gracias porque me has escuchado; yo sé que tu me
  escuchas siempre}\footnote{Jn 11, 41b-42a}; esta relación queda afirmada
plenamente ante la pasión como confianza puesta en su voluntad:
\citalitinterlin{Padre\ldots que no se haga mi voluntad, sino la
  tuya}\footnote{Lc 22,42}. De este modo en el misterio pascual queda
atestiguada la plena unidad de Cristo con el Padre, en la mayor confianza
imaginable.\footnote{prades 127}

A lo largo de su misión, Cristo dió testimonio del amor del Padre
\citalitinterlin{habiendo amado a los suyos que estaban en el
  mundo\ldots}\footnote{Jn 13,1}. En el misterio pascual, donde
\citalitinterlin{los amó hasta el extremo}\footnote{Jn 13, 1}, queda confirmado
definitivamente como testigo del Padre. Con su entrega ofrece el testimonio
pleno del amor salvador del Padre: \citalitinterlin{Porque tanto amó Dios al
  mundo, que entregó a su Unigénito, para que todo el que cree en él no perezca,
  sino que tenga vida eterna.}\footnote{Jn 3,16}

A lo largo de su vida, Cristo también es testigo de la necesidad del camino
salvífico que es libre e irrevocable decisión trinitaria de redimir a los
hombres\footnote{prades 128}: \citalitinterlin{¿No sabíais que yo debía estar en
  las cosas de mi Padre?}\footnote{Lc 2, 49}; \citalitinterlin{El hijo del
  hombre tiene que padecer mucho, ser reprobado por los ancianos, sumos
  sacerdotes y escribas, ser ejecutado y resucitar a los tres días.}\footnote{Mc
  8, 31} Este testimonio de la voluntad divina es comprendido por los discípulos
por la luz del Resucitado; \citalitinterlin{les abrió el entendimiento para
  comprender las Escrituras\ldots ``así está escrito: el Mesías padecerá,
  resucitaráde entre los muertos al tercer día y en su nombre se proclamará la
  conversión''}.\footnote{Lc 24, 45-47a}

La intencionalidad de este testimonio que Jesús ofrece a lo largo de su vida
hasta llegar al acto testimonial definitivo de Dios al mundo que es el misterio
pascual aparece con claridad en la respuesta de Cristo a Pilato antes de la
Pasión: \citalitinterlin{Yo para esto he nacido y para esto he venido al mundo:
  para dar testimonio de la verdad. Todo el que es de la verdad escucha mi
  voz.}\footnote{Jn 18,37} En su vida pública y en su misión Cristo ha actuado
como profeta que anuncia la verdad; da a conocer al Padre, a quien nadie ha
visto nunca, pero que el Hijo sí conoce.\footnote{cf. Jn 1,18 vease también
  Jesús de Nazaret 24} En el misterio pascual Jesús se manifiesta como verdadero
profeta, acreditado por el hecho mismo de la Resurrección donde se ha realizado
en él mismo lo que ha revelado y prometido. \footnote{prades 128}

La resurrección de Cristo no sólo acredita su propio testimonio, sino que
sostiene el testimonio apostólico. Si Cristo no ha resucitado sería vana
cualquier argumentación, sin embargo, Jesús es <<el Viviente>>, estuvo muerto,
pero vive por los siglos de los siglos.\footnote{Ap 1, 17--18}

Los apóstoles son testigos de la vida de Cristo, de sus palabras y acciones,
muerte y resurrección. De tal modo, son testigos en continuidad con el testimonio
de Cristo. El testimonio apostólico es un anuncio de estos hechos que ellos
conocen y cuyo valor han reconocido por la fe. Así Pedro proclama estas cosas el
día de Pentecostés: \citalitinterlin{A este Jesús lo resucitó Dios, de lo cual
  todos nosotros somos testigos}.\footnote{Hch 2, 32} El apóstol es testigo en
la fe sobre un acontecimiento enraizado en la historia.\footnote{ninot 402 y 406
  enraizado}

Así mismo es presentado el testimonio de Pedro en casa de Cornelio donde el
centurión y todos lo que lo acompañaban esperaban reunidos para escuchar lo que
el Señor quisiera comunicarles por medio del apóstol. Pedro, comprendiendo que
la verdad de Dios no hace acepción de personas, narra los hechos que él bien
conoce: \citalitlar{<<Vosotros conocéis lo que sucedió en toda Judea, comenzando
  por Galilea, después del bautismo que predicó Juan. Me refiero a Jesús de
  Nazaret, ungido por Dios con la fuerza del Espíritu Santo, que pasó haciendo
  el bien y curando a todos los oprimidos por el diablo, porque Dios estaba con
  él. Nosotros somos testigos de todo lo que hizo en la tierra de los judíos y
  en Jerusalén. A este lo mataron, colgándolo de un madero. Pero Dios lo
  resucitó al tercer día y le concedió la gracia de manifestarse, no a todo el
  pueblo, sino a los testigos designados por Dios: a nosotros, que hemos comido
  y bebido con él después de su resurrección de entre los
  muertos.>>\footnote{Hch 10,37--41}} Este testimonio de los hechos queda
enlazado con un testimonio de fe sobre el sentido profundo de lo que Pedro
conoce, Jesús, a quien los apóstoles y el pueblo vieron y escucharon, es ahora
juez de vivos y muertos:
\citalitlar{<<Nos encargó predicar al pueblo, dando solemne testimonio de que
  Dios lo ha constituido juez de vivos y muertos. De él dan testimonio todos los
  profetas: que todos los que creen en él reciben, por su nombre, el perdón de
  los pecados.>>\footnote{Hch 10,42--43}}

El apóstol entiende estos hechos y su alcance religioso y salvífico
interpretándolos en continuidad con la voluntad de Dios manifestada en su acción
en favor del pueblo judío a quién habló por medio de los profetas; voluntad
hecha manifiesta en definitva en \citalitinterlin{Jesús el Nazareno, varón
  acreditado por Dios ante vosotros con los milagros, prodigios y signos que
  Dios realizó por medio de él, como vosotros mismos sabéis}.\footnote{Hch 2,22}

Este anuncio es experiencia del Resucitado que comió y bebió con ellos;
él mismo se apareció a los que él quiso dando testimonio de su
resurrección. \citalitinterlin{Cristo glorificado manifiesta su verdad a los que
  él quiere y esta manifestación es simultaneamente testimonio de su identidad y
  testimonio de que él es la Vida (1Jn 5,11)}\autocite[129]{prades2015testimonio}

El misterio divino que se manifiesta en la Pascua de Jesús no deja de expresarse
en el anuncio pascual realizado por los apóstoles. Ellos son testigos de un
hecho enraizado en la historia, que tiene un alcance religioso y salvífico y que
es interpretado desde la voluntad de Dios manifestada en los hechos y palabras
de Cristo. Sin las obras que Jesús realizó, el testimonio apostólico se
derrumba, no existe.\autocite[Cf.][1529]{latourelle2000testimonio} Sin la vida y
obra, muerte y resurrección de Jesús \citalitinterlin{resultamos unos falsos
  testigos de Dios, porque hemos dado testimonio contra él, diciendo que ha
  resucitado a Cristo, a quien no ha resucitado}.\footnote{1Cor 15,15}

En Cristo, testigo acreditado por su Resurrección, encuentra su cumplimiento
la promesa hecha al pueblo de Israel: \citalitinterlin{El Señor, tu Dios, te
  suscitará de entre los tuyos, de entre tus hermanos, un profeta como yo. A
  él lo escucharéis}.\footnote{Dt 18, 15 véase intro Jesús de Nazaret} Así
como el misterio pascual y su anuncio no están desconectados de la vida de
Cristo, tampoco lo están de la acción salvadora de Dios en el AT. Como veremos,
el misterio divino se manifiesta a un pueblo que también está llamado a dar
testimonio, reconociendo desde la confianza en Dios el valor salvífico de los
sucesos de su historia.



\subsection{La acción testimonial de Dios en el Antiguo Testamento}

En el AT encontramos ese <<intercambio de testimonios>> que existe en el trato
de las tres personas divinas con los
hombres.\autocite[Cf.][1531]{latourelle2000testimonio} También aquí la acción
testimonial divina se despliega de diversos modos. En la vida del pueblo de la
alianza YHWH da testimonio de sí a través de la creación, la ley y, de modo
eminente, en personas elegidas y enviadas por
él.\autocite[Cf.][114s]{prades2015testimonio} Esta manifestiación divina implica
como testigo al mismo pueblo, hacia quien ha sido dirigida la voz del Señor.

La literatura sapiencial recoge la profundización en la experiencia de
Dios que ha tenido el pueblo de Israel. En ella se describe el acceso posible al
conocimiento de Dios a partir de los bienes visibles o de sus obras:
\citalitlar{Son necios por naturaleza todos los hombres que han ignorado a Dios
  y no han sido capaces de conocer al que es a partir de los bienes visibles, ni
  de reconocer al artífice fijándose en sus obras, sino que tuvieron por dioses
  al fuego, al viento, al aire ligero, a la bóveda estrellada, al agua impetuosa
  y a los luceros del cielo, regidores del mundo. Si, cautivados por su
  hermosura, los creyeron dioses, sepan cuánto los aventaja su Señor, pues los
  creó el mismo autor de la belleza. Y si los asombró su poder y energía,
  calculen cuánto más poderoso es quien los hizo, pues por la grandeza y
  hermosura de las criaturas se descubre por analogía a su creador.\footnote{Sab
    13,1--5}}
El Dios que puede ser reconocido por analogía en el asombro y belleza de las
ciraturas es un Dios personal que concede sabiduría al
piadoso:\citalitinterlin{Aún quedan misterios mucho más grandes: tan solo hemos
  visto algo de sus obras. Porque el Señor lo ha hecho todo y a los piadosos les
  ha dado la sabiduría.}\footnote{Eclo 43,32--33} Esta sabiduria es justicia y
raíz de inmortalidad:
\citalitlar{Pero tú, Dios nuestro, eres bueno y fiel, eres paciente y todo lo
  gobiernas con misericordia. Aunque pequemos, somos tuyos y reconocemos tu
  poder, pero no pecaremos, sabiendo que te pertenecemos. Conocerte a ti es
  justicia perfecta y reconocer tu poder es la raíz de la
  inmortalidad.\footnote{Sab 15,1--3}}
En este sentido la misma creación es acto testimonial de Dios donde se comunica
su misterio y la vida que Él ofrece.

YHWH también aparece en el AT como testigo de los mandamientos contenidos en la
Ley.\autocite[Cf.][115]{prades2015testimonio} Ésta queda grabada en las ``tablas
del testimonio'' y confiadas a Moisés:\citalitinterlin{Cuando acabó de hablar
  con Moisés en la montaña del Sinaí, le dio las dos tablas del Testimonio,
  tablas de piedra escritas por el dedo de Dios.}\footnote{Ex 31,18} Este
testimonio se enfrenta a un pueblo con el corazón extraviado:\citalitinterlin{Al
  acercarse al campamento y ver el becerro y las danzas, Moisés, encendido en
  ira, tiró las tablas y las rompió al pie de la montaña.}\footnote{Ex 32,19}
Sin embargo Dios no se detiene ante la dureza del pueblo. Las tablas del
testimonio son reconstruidas:
\citalitlar{El Señor dijo a Moisés: <<Labra dos tablas de piedra como las
  primeras y yo escribiré en ellas las palabras que había en las primeras tablas
  que tú rompiste.>>\ldots~<<Escribe estas palabras: de acuerdo con estas
  palabras concierto alianza contigo y con Israel>>.\footnote{Ex 34,1.27}}
Moisés, que conoció el nombre del Señor (Ex 3,13s), y habló con Él como un amigo
(Ex 33,11), aparece ante el pueblo como testigo del único Dios, y de su lealtad
con el pueblo. Pertenece a
aquellos que el Señor elige como testigos suyos en cada etapa de la historia del
pueblo de Israel como testimonio suyo y de su fidelidad.

Este es el modo eminente en que el AT describe el testimonio que Dios dirige al
pueblo. Los profetas y ungidos por YHWH son testigos del Señor y de su
compromiso con el pueblo. La vida totalmente comprometida del profeta expresa
tanto a Dios, absoluto que comunica, como su lealtad:
\citalitlar{es Dios quien da testimonio de sí mismo y de sus obras y designios a
  través de las personas elegidas, que se comprometen en su integridad como
  testigos de YHWH incluso hasta la muerte si el testimonio les lleva a ello.
  Por eso, la autoridad del testimonio no descansa en los testigos, sino en el
  mismo YHWH, que es quien los escoge y
  envía.\autocite[116s]{prades2015testimonio}}
En tanto que testigos, la acción de estos escogidos puede ser descrita según los
rasgos que tiene la actividad humana de dar testimonio, sin embargo la noción de
testigo que aplica a estos elegidos de Dios va más allá de la que encontraríamos
en el lenguaje ordinario. La vida del profeta queda comprometida con un
testimonio que no le pertenece, sino que \citalitinterlin{procede de una
  iniciativa absoluta, en cuanto a su origen y en cuanto a su
  contenido}\autocite[118]{prades2015testimonio} puesto que viene de Dios y es
testimonio de sí. Aquí la categoría de testimonio significa mas allá de su uso
ordinario en la actividad humana y adquiere un sentido religioso como dimensión
totalmente nueva\autocite[Cf.][118]{prades2015testimonio}.

El testimonio de YHWH que el profeta proclama con su actividad y el compromiso
de su vida implica al pueblo y le hace testigo:
\citalitlar{Saca afuera a un pueblo que tiene ojos, pero está ciego, que tiene
  oídos, pero está sordo. Que todas las naciones se congreguen y todos los
  pueblos se reúnan. ¿Quién de entre ellos podría anunciar esto, o proclamar los
  hechos antiguos? Que presenten sus testigos para justificarse, que los oigan y
  digan: es verdad. Vosotros sois mis testigos --—oráculo del Señor--—, y
  también mi siervo, al que yo escogí, para que sepáis y creáis y comprendáis
  que yo soy Dios. Antes de mí no había sido formado ningún dios, ni lo habrá
  después. Yo, yo soy el Señor, fuera de mí no hay salvador. Yo lo anuncié y os
  salvé; lo anuncié y no hubo entre vosotros dios extranjero. Vosotros sois mis
  testigos --—oráculo del Señor--—: yo soy Dios.\footnote{Is 43,8--12}}
El siervo es testigo que el Señor ha escogido para que el pueblo sepa, crea y
comprenda que YHWH es el único Dios verdadero. Al compartir este saber de Dios
con el pueblo, éstos también están llamados a ser testigos. Ninguna otra nación
podría anunciar como ellos lo que YHWH ha hecho para proveer, liberar, salvar.

Así como el profeta, el pueblo es escogido y enviado por YHWH y por medio de él
el Señor da testimonio de sí mismo y se propone como quien da sentido y
consistencia a toda la realidad humana. Este testimonio tiene importancia social
puesto que está llamado a ser proclamado, y esta proclamación implica el
compromiso de los actos y la vida del testigo, es decir, del profeta y todo el
pueblo.\autocite[Cf.][1526s]{latourelle2000testimonio}

El testimonio de Dios a través de personas escogidas por Él en el AT queda
constituido por la narración de hechos que acontecen en la historia, estos
hechos son interpretados en su valor absoluto y carácter redentor, y son
confesados como actuación de Dios en la vida
humana.\autocite[Cf.][119]{prades2015testimonio} Esto vuelve a ponernos en
conexión con la figura de Cristo como profeta acreditado por su Resurrección y
los apóstoles como verdaderos testigos de un hecho enraizado en la historia,
confesado desde la fe e interpretado desde la acción de Dios en Jesús. Esta
sintonía anticipa lo que se verá a continuación sobre el testimonio en el Nuevo
Testamento. En él la acción testimonial de Dios se describe en continuidad con
la tradición veterotestamentaria y llegará a su manifestación plena en el
misterio pascual.

\subsection{La acción testimonial de Dios en el Nuevo Testamento}
El Evangelio de Mateo enseña que el día que Jesús llegó a Cafarnaún a comenzar
su predicación se cumplieron las promesas que Dios había hecho por medio de los
profetas. Ese día el Reino de los cielos quedó desvelado en su cercanía. Allí la
vida de los primeros discípulos cambió al punto y definitivamente. El testimonio
de Cristo no es cualquier anuncio o cualquier hecho, sino que tiene un valor
absoluto. Jesús de Nazaret \citalitinterlin{no se limita a proponer una cierta
  inspiración espiritual o un cierto sentido ético para el obrar de la persona o
  del pueblo, sino que pretende ser radicalmente ``testimonio de la verdad'' (Jn
  18,37) de alcance universal.}\autocite[126]{prades2015testimonio} Jesús es
testimonio de carácter singular,\autocite[Cf.][279]{ninot2009tf} en quien se da
a conocer el momento de la plenitud de la
salvación,\autocite[Cf.][290]{ninot2009tf} presencia del hombre nuevo y
``paradigma universal de humanidad''\autocite[Cf.][291]{ninot2009tf}. Este valor
universal de la verdad que se comunica en Jesús se desarrolla y se manifiesta en
sus acciones concretas: comiendo con los pecadores o sanando a los enfermos es
donde se muestra \citalitinterlin{el camino, la verdad y la
  vida}\footnote{Cf.~Jn 14,6} para todos.

Este testimonio de Cristo, su vida, actos y palabras, fue sometido al juicio de
sus contemporaneos. Asombrados porque no enseña como los demás y por las signos
que realiza, se cuestionan sobre su autoridad y poder. Entonces Jesús también
tiene que ofrecer testimonio de su credibilidad. La respuesta a este juicio del
pueblo se halla en su ministerio en sintonía con las Escrituras:
\citalitinterlin{Hoy se ha cumplido esta Escritura que acabáis de
  oir}\footnote{Lc 4,21}; donde el pueblo puede encontrar la vida y el sentido
que buscan: \citalitinterlin{estudiáis las Escrituras pensando encontrar en
  ellas vida eterna; pues ellas están dando testimonio de mi, ¡y no queréis
  venir a mí para tener vida!}\footnote{Jn 5, 39--40}. El testimonio de
credibilidad de Jesús ante el pueblo se encuentra también en sus obras, que son
las obras del Padre y son confirmación y realización de sus enseñanzas:
\citalitinterlin{Si no hago las obras de mi Padre, no me creáis, pero si las
  hago, aunque no me creáis a mí, creed a las obras, para que comprendáis y
  sepáis que el Padre está en mí y yo en el Padre}\footnote{Jn 10,38}.

El singular testimonio de Cristo es comunicación de la verdad con valor
universal. El testimonio de Cristo es también su actividad e identidad que hacen
creíble lo que comunica. De este modo entre lo que Jesús testimonia y la
credibilidad que suscita su testimonio hay una circularidad constante:
\citalitlar{La pretensión única que encerraba su testimonio resultaba tan
  exorbitante que hubiera sido inaceptable para los hombres si no fuera porque
  sus obras, sus palabras y, en rigor, su presencia misma, lo hacían
  profundamente razonable en su
  singularidad.\autocite[124]{prades2015testimonio}}

Acoger el testimonio de Jesús es escuchar la Escritura y creer en las obras del
Padre. Sin embargo la palabra de Cristo choca con el odio de aquellos que son
hostiles a la verdad y que, rechazando su testimonio, se juzgan a sí
mismos.\footnote{Pero la palabra de Cristo choca con la contestación y el odio.
  Enfrentados con Cristo, los judíos, que representan al conjunto del mundo
  hostil a la verdad, rechazan su testimonio y se juzgan a sí
  mismos.\autocite[1530]{latourelle2000testimonio}} \citalitlar{Si yo no hubiera
  venido y no les hubiera hablado, no tendrían pecado, pero ahora no tienen
  excusas de su pecado. El que me odia a mí, odia también a mi Padre. Si yo no
  hubiera hecho en medio de ellos obras que ningún otro ha hecho, no tendrían
  pecado, pero ahora las han visto y me han odiado a mí y a mi Padre\footnote{Jn
    15,22--24}}

Jesús es \citalitinterlin{la luz que brilla en la tiniebla y la tiniebla no la
  recibió}\footnote{Jn 1,5}. Jesús es el \citalitinterlin{unigénito, que está en
  el seno del Padre, es quien lo ha dado a conocer}\footnote{Jn 1,18}. Este
testimonio es manifestación de la comunión trinitaria. Cristo revela al Padre y
comunica al Espíritu, y su identidad de Hijo es manifestada como acción del
Padre y del Espíritu: \citalitinterlin{Apenas se bautizó Jesús, salió del agua;
  se abrieron los cielos y vio que el Espíritu de Dios bajaba como una paloma y
  se posaba sobre él. Y vino una voz de los cielos que decía: <<Este es mi Hijo
  amado, en quien me complazco>>.}\footnote{Mt 4,16--17}

La acción testimonial de Dios que se describe en el Nuevo Testamento está
concentrada en la persona de Cristo y en su relación manifiesta con el Padre y
el Espíritu se expresa el testimonio de la Trinidad misma:
\citalitlar{la Escritura describe la actividad reveladora de la trinidad en
  forma de testimonios mutuos. El Hijo es el testigo del padre, y como tal se da
  a conocer a los apóstoles. A su vez, el Padre da también testimonio de que
  Cristo es el Hijo, por la atracción que produce en las almas, por las obras
  que da al Hijo para que las realice y sobre todo por la resurrección,
  testimonio decisivo del Padre en favor del Hijo. El Hijo da testimonio del
  Espíritu porque promete enviarlo como educador, consolador, santificador. Y el
  Espíritu viene y da testimonio del hijo porque le recuerda, le da a conocer,
  descubre la plenitud de sentido de sus palabras, lo insinua en las
  almas.\footnote{de Latourelle, Teología de la Revelación 410
    en\autocite[131]{prades2015testimonio}}}
Esta actividad reveladora de la trinidad introduce al ser humano en la comunión
trinitaria. Dios trino se comunica al ser humano actuando en su interior,
atrayendo, inspirando; también se comunica externamente por las obras que
realiza. Esta participación en la comunión divina viene bien expresada en la
finalidad del testimonio apostólico: \citalitinterlin{Eso que hemos visto y oído
  os lo anunciamos, para que estéis en comunión con nosotros y nuestra comunión
  es con el Padre y con su Hijo Jesucristo.}\footnote{1Jn 1,3}

Jesús es el fundamento, testigo fiel y veraz para todo tiempo y
lugar.\autocite[Cf.][132]{prades2015testimonio} Creer su testimonio es acoger al
absoluto en la historia, esta confianza la hace posible el Espíritu:
\citalitlar{Cristo es, por tanto, el testigo absoluto, el que lleva en sí mismo
  la garantía de su testimonio. El hombre, sin embargo, no sería capaz de acoger
  por la fe este testimonio del absoluto, manifestado en la carne y el lenguaje
  de Jesús, sin una atracción interior (Jn 6,44), que es un don del Padre y un
  testimonio del Espíritu (1Jn 5,9--10).\autocite{latourelle2000testimonio}}
Aquellos que creen en Cristo no sólo encuentran una respuesta a su busqueda de
vida y sentido, sino que \citalitinterlin{de sus entrañas manarán ríos de agua
  viva}.\footnote{Jn 7,38} Y esto Jesús lo dice \citalitinterlin{refiriéndose al
  Espíritu que habían de recibir los que creyeran en él}.\footnote{Jn 7,39} Esta
promesa del Espíritu acontece en Pentecostés y sin ese testimonio postpascual
del Espíritu quedaría incompleta la comunicación de Dios en el misterio
Pascual.\autocite[Cf.][135]{prades2015testimonio} El envío y la acción del
Espíritu prometido completa la acción testimonial de Dios:
\citalitlar{Al haber ``acompañado'' al Hijo en la tierra de una manera singular
  desde el momento de su unción en el Jordán, que dispone al Hijo ---concebido
  por obra del Espíritu Santo--- para la misión en la carne, el Espíritu Santo
  vueve al Padre portando en sí todo el misterio redentor del Hijo. De este
  modo, cuando el Resucitado lo envía a la Iglesia, el Espíritu vuelve como
  Testigo de la verdad completa, que incluye la perfecta glorificación de la
  carne del Hijo como plenitud de lo
  humano.\autocite[134s]{prades2015testimonio}}

El Espíritu enviado por Cristo lleva a la verdad plena a los apóstoles:
\citalitinterlin{cuando venga él, el Espíritu de la verdad, os guiará hasta la
  verdad plena. Pues no hablará por cuenta propia, sino que hablará de lo que
  oye y os comunicará lo que está por venir}.\footnote{Jn 16,13} Este testimonio
del Espíritu completa tambíen el testimonio de los apóstoles:
\citalitinterlin{Cuando venga el Paráclito, que os enviaré desde el Padre, el
  Espíritu de la verdad, que procede del Padre, él dará testimonio de mí; y
  también vosotros daréis testimonio, porque desde el principio estáis
  conmigo}.\footnote{Jn 15,26--27} Ellos han estado desde el principio con
Cristo, así son testigos que pueden narrar lo que han visto y oído; su testimono
queda perfeccionado por el Espíritu que les introduce en el misterio del Hijo
encarnado y les permite interpretar y comprender la verdad del Hijo, y por éste,
la del Padre.\autocite[Cf.][139]{prades2015testimonio}

Los que han compartido con Jesús desde el principio son testigos del Evangelio,
pero el Resucitado sigue eligiendo apóstoles y en virtud de la acción del
Espíritu éstos son testigos del mismo misterio.\autocite[Cf.][576]{ninot2009tf}
Así Matías no sólo es \citalitinterlin{uno de los que nos acompañaron todo el
  tiempo que convivió con nosotros el Señor Jesús}\footnote{Hch 1,21}, sino que
es elegido por el Resucitado.\footnote{Cf.~Hch 1,24--26} Igualmente Pablo es
constituido testigo por la llamada del Resucitado, asi puede decir
\citalitinterlin{Yo mismo hermanos cuando vine a vosotros anunciaros el
  testimonio de Dios\ldots}\footnote{1\,Cor 2,1}. De este modo la transmisión
viva del testimonio cristiano esta constituida por un momento fundacional en la
convivencia con Jesús y un momento continuante como dos aspectos históricos
inseparables.\autocite[Cf.][148]{prades2015testimonio} Este momento continuante
esta compuesto por los que han sido testigos oculares, como por los que no:
\citalitinterlin{unos y otrs son elegidos, llamados y enviados por Cristo, el
  Cristo histórico los primeros y el Cristo glorioso los
  segundos}.\autocite[148]{prades2015testimonio}

Aquel que recibe este testimonio y cree en él encuentra la vida nueva. <<¿Qué
dificultad hay en que me bautice?>>, decide aquel hombre que recibió el
testimonio de Felipe y <<siguió su camino lleno de alegría>> después de haber
encontrado a Dios. Considerar la revelación divina como acción testimonial de
Dios conduce en definitiva a estimar la revelación misma como forma de amor y
libertad de Dios que interpela el amor y libertad humano. En tanto que
comunicación libre y amorosa, el testimonio de Dios atiende la naturaleza humana
de su beneficiario; en tanto que don divino queda desvelado su origen y meta más
allá de lo humano.\autocite[Cf.][152]{prades2015testimonio}

\section{Iglesia como signo sacramental el Testimonio en el Magisterio Reciente}
Nuestro recorrido comenzó al inicio de este capítulo tomando como punto de
partida a la Iglesia como signo visibile. La vida de la comunidad eclesial, sus
costumbres y actitudes, son presencia histórica y realidad perceptible. La
Iglesia puede ser reconocida hoy actuando según su costumbre de reunirse en
torno a la Palabra de Dios para celebrarla y conocer la verdad para su vida. Lo
que se vive hoy y se ha transmitido en la tradición eclesial lo hemos valorado
como perpetuación de la actividad de Cristo y de los apóstoles y, por tanto,
como proyección del testimonio divino. En este sentido hemos considerado la
presencia de la Revelación divina en el corazón y anuncio de la Iglesia como
triple testimonio usando la expresión de Latourelle: <<palabra vivida en el
Espíritu>>. Esta reflexión ha querido servir para describir la naturaleza de la
Revelación como experiencia familiar en la vida de la Iglesia. La noción de la
categoría del testimonio que atraviesa la escritura ha servido para valorar la
naturaleza de la Revelación según su estructura testimonial.

Así como la categoría del testimonio ha servido para decir algo sobre la
Revelación en la Escritura, ahora se pretende decir algo sobre lo que la
categoría del testimonio puede aportar para comprender la identidad de la
Iglesia y su misión en el mundo y cómo ésta forma parte del dinamismo de la
Revelación divina.

Con Latourelle se ha dicho que el testimonio es una de esas categorías que la
escritura emplea como analogía para introducirnos al misterio divino. El
Concilio nos regala otra analogía que va de la mano con la categoría del
testimonio en la comprensión de la Iglesia y su misión:
\citalitlar{la sociedad provista de sus órganos jerárquicos y el Cuerpo místico de
Cristo, la asamblea visible y la comunidad espiritual, la Iglesia terrestre y la
Iglesia enriquecida con los bienes celestiales, no deben ser consideradas como
dos cosas distintas, sino que más bien forman una realidad compleja que está
integrada de un elemento humano y otro divino. Por eso se la compara, por una
notable analogía, al misterio del Verbo encarnado, pues así como la naturaleza
asumida sirve al Verbo divino como de instrumento vivo de salvación unido
indisolublemente a Él, de modo semejante la articulación social de la Iglesia
sirve al Espíritu Santo\footnote{LG 8}}

La visibilidad de la Iglesia está al servicio del Espíritu Santo de modo que su
naturaleza humana sirve a la presencia divina como instrumento vivo de
salvación. La presencia de la articulación social de la Iglesia actua de manera
análoga a la presencia de Cristo. Según esto se puede decir que
\citalitinterlin{la eclesiología se resuelve en la Cristología y por eso el
  ``lugar'' de la Iglesia en el acto de creer será ``análogo'' al de
  Cristo}.\autocite[566]{ninot2009tf} Esta relación con Cristo y el Espíritu
otorgan a la Iglesia valor sacramental:
\citalitlar{Porque Cristo, levantado sobre la tierra, atrajo hacia sí a todos
  (cf. Jn 12, 32 gr.); habiendo resucitado de entre los muertos (Rm 6, 9), envió
  sobre los discípulos a su Espíritu vivificador, y por El hizo a su Cuerpo, que
  es la Iglesia, sacramento universal de salvación; estando sentado a la derecha
  del Padre, actúa sin cesar en el mundo para conducir a los hombres a la
  Iglesia y, por medio de ella, unirlos a sí más estrechamente y para hacerlos
  partícipes de su vida gloriosa alimentándolos con su cuerpo y sangre. Así que
  la restauración prometida que esperamos, ya comenzó en Cristo, es impulsada
  con la misión del Espíritu Santo y por Él continúa en la Iglesia, en la cual
  por la fe somos instruidos también acerca del sentido de nuestra vida
  temporal, mientras que con la esperanza de los bienes futuros llevamos a cabo
  la obra que el Padre nos encomendó en el mundo y labramos nuestra salvación
  (cf. Flp 2, 12).\footnote{LG 48}}
Esta Iglesia que es sacramento es mediación de acción salvadora de Dios;
comunica los dones de la gracia y manifiesta el misterio de Dios:
\citalitlar{Todo el bien que el Pueblo de Dios puede dar a la familia humana al
  tiempo de su peregrinación en la tierra, deriva del hecho de que la Iglesia es
  "sacramento universal de salvación", que manifiesta y al mismo tiempo realiza
  el misterio del amor de Dios al hombre.\footnote{GS 45}}

La Iglesia en el mundo es así uno de los signos contenidos en la Revelación que
ayudan a la razón que busca la comprensión del misterio. El signo sacramental
que es la Iglesia permite atestiguar desde la fe el misterio de Dios que en ella
se expresa del mismo modo que ocurre con la Eucaristía o la presencia de Cristo
encarnado:
\citalitlar{Podemos fijarnos, en cierto modo, en el horizonte sacramental de la
  Revelación y, en particular, en el signo eucarístico donde la unidad
  inseparable entre la realidad y su significado permite captar la profundidad
  del misterio. Cristo en la Eucaristía está verdaderamente presente y vivo, y
  actúa con su Espíritu, pero como acertadamente decía Santo Tomás, <<lo que no
  comprendes y no ves, lo atestigua una fe viva, fuera de todo el orden de la
  naturaleza. Lo que aparece es un signo: esconde en el misterio realidades
  sublimes>>. A este respecto escribe el filósofo Pascal: <<Como Jesucristo
  permaneció desconocido entre los hombres, del mismo modo su verdad permanece,
  entre las opiniones comunes, sin diferencia exterior. Así queda la Eucaristía
  entre el pan común>>.\footnote{FR 13}}
El misterio sublime que aparece en un signo puede ser atestiguado por la fe
viva. El asentimiento al signo sacramental por la fe supone el reconocimiento de
que viene de Dios y por tanto es creer a quien es garante de su propia verdad.
Este asentimiento implica a la persona por completo:
\citalitlar{Desde la fe el hombre da su asentimiento a ese testimonio divino.
  Ello quiere decir que reconoce plena e integralmente la verdad de lo revelado,
  porque Dios mismo es su garante. Esta verdad, ofrecida al hombre y que él no
  puede exigir, se inserta en el horizonte de la comunicación interpersonal e
  impulsa a la razón a abrirse a la misma y a acoger su sentido profundo. Por
  esto el acto con el que uno confía en Dios siempre ha sido considerado por la
  Iglesia como un momento de elección fundamental, en la cual está implicada
  toda la persona. Inteligencia y voluntad desarrollan al máximo su naturaleza
  espiritual para permitir que el sujeto cumpla un acto en el cual la libertad
  personal se vive de modo pleno.\footnote{FR 13}}
La acogida del misterio divino comunicado en el signo sacramental es así un acto
de libertad plena que no sólo permite reconocer el misterio de Dios, sino que
nos desvela nuestra vocación de comunión con Él, que es nuestro sentido más
auténtico:
\citalitlar{El conocimiento de fe, en definitiva, no anula el misterio; sólo lo
  hace más evidente y lo manifiesta como hecho esencial para la vida del hombre:
  Cristo, el Señor, <<en la misma revelación del misterio del Padre y de su
  amor, manifiesta plenamente el hombre al propio hombre y le descubre la
  grandeza de su vocación>>, que es participar en el misterio de la vida
  trinitaria de Dios.\footnote{FR 13}}

La Iglesia es signo sacramental unido inseparablemente al misterio divino que
comunica, de modo análogo a la unión del Verbo divino y la naturaleza asumida
por Él. El conocimiento de la fe abre la razón humana a la verdad revelada como
comunicación interpersonal de Dios realizada por medio de este signo sacramental
que es la Iglesia. Este acto de confianza es movimiento de la libertad como
asentimiento y elección de Dios que se revela y acogida de su llamada a
participar de la comunión trinitaria. Aquí sacramento y testimonio son
categorías que interactuan para describir el acceso al misterio divino que se
comunica a través de signos. Esta Iglesia que es signo sacramental es signo
creíble por el testimonio de la vida comprometida con el misterio de amor que
significa: \citalitlar{La misión primera y fundamental que recibimos de los
  santos Misterios que celebramos es la de dar testimonio con nuestra vida. El
  asombro por el don que Dios nos ha hecho en Cristo infunde en nuestra vida un
  dinamismo nuevo, comprometiéndonos a ser testigos de su amor. Nos convertimos
  en testigos cuando, por nuestras acciones, palabras y modo de ser, aparece
  Otro y se comunica. Se puede decir que el testimonio es el medio con el que la
  verdad del amor de Dios llega al hombre en la historia, invitándolo a acoger
  libremente esta novedad radical. En el testimonio Dios, por así decir, se
  expone al riesgo de la libertad del hombre. Jesús mismo es el testigo fiel y
  veraz (cf. Ap 1,5; 3,14); vino para dar testimonio de la verdad (cf. Jn
  18,37). Con estas reflexiones deseo recordar un concepto muy querido por los
  primeros cristianos, pero que también nos afecta a nosotros, cristianos de
  hoy: el testimonio hasta el don de sí mismos, hasta el martirio, ha sido
  considerado siempre en la historia de la Iglesia como la cumbre del nuevo
  culto espiritual: <<Ofreced vuestros cuerpos>> (Rm 12,1). Se puede recordar,
  por ejemplo, el relato del martirio de san Policarpo de Esmirna, discípulo de
  san Juan: todo el acontecimiento dramático es descrito como una liturgia, más
  aún como si el mártir mismo se convirtiera en Eucaristía. Pensemos también en
  la conciencia eucarística que san Ignacio de Antioquía expresa ante su
  martirio: él se considera <<trigo de Dios>> y desea llegar a ser en el
  martirio <<pan puro de Cristo>>. El cristiano que ofrece su vida en el
  martirio entra en plena comunión con la Pascua de Jesucristo y así se
  convierte con Él en Eucaristía. Tampoco faltan hoy en la Iglesia mártires en
  los que se manifiesta de modo supremo el amor de Dios. Sin embargo, aun cuando
  no se requiera la prueba del martirio, sabemos que el culto agradable a Dios
  implica también interiormente esta disponibilidad, y se manifiesta en el
  testimonio alegre y convencido ante el mundo de una vida cristiana coherente
  allí donde el Señor nos llama a anunciarlo.\footnote{SCa 85}}
El testimonio hasta el don de nosotros mismos se convierte en signo sacramental,
el cristiano que ofrece su vida por completo como testigo entra en comunión con
la Pascua y se convierte con Cristo en Eucaristía. La vida entregada, este signo
sacramental, es el medio adecuado para comunicar la comunión con Dios:
\citalitlar{En efecto, la fe necesita un ámbito en el que se pueda testimoniar y
  comunicar, un ámbito adecuado y proporcionado a lo que se comunica. Para
  transmitir un contenido meramente doctrinal, una idea, quizás sería suficiente
  un libro, o la reproducción de un mensaje oral. Pero lo que se comunica en la
  Iglesia, lo que se transmite en su Tradición viva, es la luz nueva que nace
  del encuentro con el Dios vivo, una luz que toca la persona en su centro, en
  el corazón, implicando su mente, su voluntad y su afectividad, abriéndola a
  relaciones vivas en la comunión con Dios y con los otros. Para transmitir esta
  riqueza hay un medio particular, que pone en juego a toda la persona, cuerpo,
  espíritu, interioridad y relaciones. Este medio son los sacramentos,
  celebrados en la liturgia de la Iglesia. En ellos se comunica una memoria
  encarnada, ligada a los tiempos y lugares de la vida, asociada a todos los
  sentidos; implican a la persona, como miembro de un sujeto vivo, de un tejido
  de relaciones comunitarias. Por eso, si bien, por una parte, los sacramentos
  son sacramentos de la fe, también se debe decir que la fe tiene una estructura
  sacramental. El despertar de la fe pasa por el despertar de un nuevo sentido
  sacramental de la vida del hombre y de la existencia cristiana, en el que lo
  visible y material está abierto al misterio de lo eterno. \footnote{LF 40}}
Al celebrar los sacramentos con fe viva, la comunidad eclesial se deja implicar
por completo por la luz del Dios vivo que se comunica y el memorial que se
encarna. Despertar a la fe en los sacramentos es también despertar al sentido
sacramental que tiene la propia vida cristiana. Así como en los sacramentos los
signos visibles comunican la luz de Dios, también la propia existencia del
cristiano puede arrojar esa luz.

Este valor sacramental de la vida del cristiano y de la comunidad eclesial hace
de su propia existencia un testimonio kerygmático:
\citalitlar{La Buena Nueva debe ser proclamada en primer lugar, mediante el
  testimonio. Supongamos un cristiano o un grupo de cristianos que, dentro de la
  comunidad humana donde viven, manifiestan su capacidad de comprensión y de
  aceptación, su comunión de vida y de destino con los demás, su solidaridad en
  los esfuerzos de todos en cuanto existe de noble y bueno. Supongamos además
  que irradian de manera sencilla y espontánea su fe en los valores que van más
  allá de los valores corrientes, y su esperanza en algo que no se ve ni osarían
  soñar. A través de este testimonio sin palabras, estos cristianos hacen
  plantearse, a quienes contemplan su vida, interrogantes irresistibles: ¿Por
  qué son así? ¿Por qué viven de esa manera? ¿Qué es o quién es el que los
  inspira? ¿Por qué están con nosotros? Pues bien, este testimonio constituye ya
  de por sí una proclamación silenciosa, pero también muy clara y eficaz, de la
  Buena Nueva. Hay en ello un gesto inicial de evangelización. Son posiblemente
  las primeras preguntas que se plantearán muchos no cristianos, bien se trate
  de personas a las que Cristo no había sido nunca anunciado, de bautizados no
  practicantes, de gentes que viven en una sociedad cristiana pero según
  principios no cristianos, bien se trate de gentes que buscan, no sin
  sufrimiento, algo o a Alguien que ellos adivinan pero sin poder darle un
  nombre. Surgirán otros interrogantes, más profundos y más comprometedores,
  provocados por este testimonio que comporta presencia, participación,
  solidaridad y que es un elemento esencial, en general al primero absolutamente
  en la evangelización.\footnote{EN 21}}
La acción testimonial de Dios que se manifiesta en Cristo y en los sacramentos
instituidos por Él está analogamente presente en la vida comprometida del
cirstiano. El testimonio humano es respuesta de fe de aquellos que han
reconocido a Dios en los signos que le encarnan y que corresponden con palabras
y obras que quieren significar la vida nueva que viene del Señor. En esta
correspondencia se unden las raíces de la misión de proclamar la Buena Nueva.

El testimonio es así acción propia de todo bautizado que ha quedado unido a
Cristo y a la Iglesia.\autocite[Cf.][188]{prades2015testimonio} Toda la Iglesia
tiene la misión de dar testimonio; los obispos ofrecen al mundo el rostro de la
Iglesia con su trato y trabajo pastoral\footnote{GS 43}, los presbíteros,
creciendo en el amor por el desempeño de su oficio, han de ser un vivo
testimonio de Dios\footnote{LG 41}, los fieles han de dar testimonio de verdad
como testigos de la resurrección\footnote{LG 28 y LG 38}, los religiosos deben
ofrecer un testimonio sostenido por la integridad de la fe, por la caridad y el
amor a la cruz y la esperanza de la gloria futura\footnote{PC 25}, los
profesores han de dar testimonio tanto con su vida como con su
doctrina\footnote{GE 8}, los misioneros han de ofrecer testimonio con una vida
enteramente evangélica, con paciencia, longanimidad, suavidad, caridad sincera,
y si es necesario hasta con la propia sangre.\footnote{AG 24}

El signo que es la vida de los cristianos y, por tanto la Iglesia, esta llamado
a purificarse y crecer. La contradicción entre la fe y la vida de los cristianos
puede constituir un motivo de tropiezo, en lugar de dar a conocer la luz de
Dios. El testimonio de la vida entregada, aún cuando ha sido estimado según su
valor sacramental, es un signo imperfecto que debe ser madurado con una actitud
vigilante:
\citalitlar{Aunque la Iglesia, por la virtud del Espíritu Santo, se ha mantenido
  como esposa fiel de su Señor y nunca ha cesado de ser signo de salvación en el
  mundo, sabe, sin embargo, muy bien que no siempre, a lo largo de su prolongada
  historia, fueron todos sus miembros, clérigos o laicos, fieles al espíritu de
  Dios. Sabe también la Iglesia que aún hoy día es mucha la distancia que se da
  entre el mensaje que ella anuncia y la fragilidad humana de los mensajeros a
  quienes está confiado el Evangelio. Dejando a un lado el juicio de la historia
  sobre estas deficiencias, debemos, sin embargo, tener conciencia de ellas y
  combatirlas con máxima energía para que no dañen a la difusión del Evangelio.
  De igual manera comprende la Iglesia cuánto le queda aún por madurar, por su
  experiencia de siglos, en la relación que debe mantener con el mundo. Dirigida
  por el Espíritu Santo, la Iglesia, como madre, no cesa de ``exhortar a sus
  hijos a la purificación y a la renovación para que brille con mayor claridad
  la señal de Cristo en el rostro de la Iglesia''\footnote{GS 34}}
La vida de la Iglesia está marcada por esa llamada a este enriquecimiento
constante. Como afirma DV 8: \citalitinterlin{la Iglesia, en el decurso de los
  siglos, tiende constantemente a la plenitud de la verdad divina, hasta que en
  ella se cumplan las palabras de Dios.}

La categoría del testimonio ha servido para acercarnos a algunos textos
magisteriales y describir la vida de la Iglesia como signo sacramental. A modo
de conclusión son luminosas las palabras de K. Wojtyła:
\citalitlar{El significado del testimonio en la doctrina del Vaticano II es
  explícitamente analógico, puesto que el Concilio habla del testimonio de Dios
  y del hombre, que, de diversa manera, corresponde al divino, y a una respuesta
  multiforme a la revelación. En todo caso, sin embargo, la respuesta es
  testimonio y el testimonio, respuesta. \footnote{Para una discusión más amplia
    de la lectura de Wojtyła véase \cite[194--197]{prades2015testimonio}}}

Este recorrido a través de algunos modos de emplear la categoría del testimonio
en la Escritura y la doctrina magisterial ha servido para describir los
dinamísmos de la Revelación como acción libre y amorosa del Padre encarnada en
en la naturaleza humana asumida por el Verbo y sostenida por la acción interior
del Espíritu. Esta acción de la libertad divina ha encontrado la correspondencia
de la libertad humana que acoge la invitación al amor y se compromete por
completo a la comunión con Dios. Este intercambio testimonial comunica el amor
divino.

\section{La Categoría del Testimonio como problema}

Hasta ahora se ha empleado la categoría del testimonio sin problematizarla. Es
decir, hemos tratado el testimonio como <<cosa familiar y conocida>> empleada
ordinariamente en nuestras conversaciones. Es aquí donde nos permitimos tratar
al testimonio como algo que hay que esclarecer, algo sobre lo que se plantean
preguntas, de modo que hay que traer a la mente una explicación adecuada. En
palabras de Latourelle:
\citalitlar{Si la revelación misma se apoya en la experiencia humana del
  testimonio para expresar una de las relaciones fundamentales que unen al
  hombre con Dios, la reflexión teológica se encuentra entonces autorizada a
  explorar los datos de esta
  experiencia.\autocite[1523]{latourelle2000testimonio}}
Hasta el momento sólo hemos formulado una pregunta, que si ampliamos un poco
queda: ``¿qué es conocer una verdad para la vida por el testimonio de la
revelación divina?''. Esta formulación puede servir como punto de partida y, si
tenemos en cuenta la reflexión filosófica en torno al testimonio, podemos
expandirla más.

Aún cuando el testimonio ocupa un lugar vital en nuestro contacto con el mundo,
no siempre ha gozado del interés de la investigación filosófica. Más
recientemente, sin embargo, su importancia ha sido mejor apreciada y así lo
refleja la variedad literatura que puede encontrarse en la filosofía
contemporanea.\footnote{Cf. Jennifer Lackey Despite the vital role that
  testimony occupies in our epistemic lives, traditional epistemological
  theories focused primarily on other sources, such as sense perception, memory,
  and reason, with relatively little attention devoted specifically to
  testimony. In recent years, however, the epistemic significance of testimony
  has been more fully appreciated, and the current literature has benefited from
  the publication of a considerable amount of interesting and innovative work in
  this area.}
Esta época mas prolija en discusiones no es, sin embargo, el origen de algunas
posturas propuestas en torno al testimonio, éste lo encontramos más bien en la
época moderna. Recurriremos, por tanto, a algunas aportaciones y desafíos
ofrecidos por la filosofía moderna y contemporánea para expandir nuestra
anterior pregunta y formular las cuestiones principales que serviran luego para
navegar en el pensamiento de Elizabeth Anscombe.

\subsection{¿Cuál es el valor epistemológico del testimonio?}
Corresponde a la epistemología la tarea de estudiar la naturaleza del conocer y
su justificación. ¿Cuáles son los componentes del conocimiento? ¿sus fuentes o
condiciones? ¿sus límites?\footnote{Cf. Oxford Epistemology, characterized
  broadly, is an account of knowledge. Within the discipline of philosophy,
  epistemology is the study of the nature of knowledge and justification: in
  particular, the study of (a) the defining components, (b) the substantive
  conditions or sources, and (c) the limits of knowledge and justification.} La
pregunta sobre el valor epistemológico del testimonio consiste en juzgar el
lugar que éste ocupa en una descripición del conocimiento; ¿qué se puede decir
del testimonio como estrategia para adquirir la verdad y evitar el
error?\footnote{Cf. Any standard or strategy worthy of the title ``epistemic''
  must have as its fundamental goal the acquisition of truth and the avoidance
  of error.}

Un análisis tradicional empleado para hablar del conocimiento proposicional es
el que le describe como `creencia verdadera justificada'.\footnote{Ever since
  Plato's Theaetetus, epipstemologists have tried to identify the essential,
  defining components of propositional knowledge. These components will yield an
  analysis of propositional knowledge. An influential traditional view, inspired
  by Plato and Kant among others, is that propositional knowledge has three
  individually necessary and jointly sufficient components: justification,
  truth, and belief. On this view, propositional knowledge is, by definition,
  justified true belief. This tripartite definition has come to be called ``the
  standard analysis''.} Según esta composición tripartita la pregunta sobre el
valor epistemológico del testimonio se puede plantear diciendo: dada una
comunicación que cualifique como testimonio y que sea al caso que la creencia
formada desde esta comunicación está basada enteramente en el testimonio
recibido,\footnote{Even if an expression of thought qualifies as testimony and
  the resulting belief formed is entirely testimonially based for the hearer,
  however, there is the further question of how precisely such a belief
  successfully counts as justified belief or an instance of knowledge.} ¿cómo
adquirimos efectivamente una creencia verdadera justificada sobre la base de lo
que alguien nos ha dicho?\footnote{how we successfully acquire justified belief
  or knowledge on the basis of what other people tell us. This, rather than what
  testimony is, is often taken to be the issue of central import from an
  epistemological point of view.}, es decir, ¿cómo, precisamente, una creencia
como esta puede ser contada satisfactoriamente como creencia justificada o una
instancia de conocimiento? \footnote{how precisely such a belief successfully
  counts as justified belief or an instance of knowledge}

Las respuestas a esta pregunta central sobre la epistemología del testimonio se
han situado en lo que se ha llamado la postura reduccionista y la
no-reduccionista.\footnote{Indeed, this is the question at the center of the
  epistemology of testimony, and the current philosophical literature contains
  two central options for answering it: non-reductionism and reductionism.} Las
raices históricas de la primera postura se le suelen atribuir a Hume y de la
segunda a Thomas Reid.

De acuerdo a los no-reduccionistas el testimonio es simplemente una fuente de
justificación como lo sería la percepción de los sentidos, la memoria o la
inferencia. Según esto, siempre que no haya una justificación contraria
suficientemente relevante, el que escucha tiene justificación verdadera para
creer las proposiciones del testimonio del que habla.\footnote{According to
  non-reductionists ---whose historical roots are standardly traced back to
  Reid--- testimony is just as basic a source of justification (warrant,
  entitlement, knowledge, etc.) as sense perception, memory, inference, and the
  like. Accordingly, so long as there are no relevant defeaters, hearers can
  justifiedly accept the assertions of speakers merely on the basis of a
  speaker's testimony.}

Hume, por su parte \citalitinterlin{es uno de los pocos filósofos que ha
  ofrecido algo así como una descripción sostenida acerca del testimonio y si alguna
  perspectiva puede reclamar el título de `el punto de vista adoptado' es la
  suya. }\footnote{is one of the few philosophers who has offered anything like
  a sustained account of testimony and if any view has a claim to the title of
  `the received view' it is his testimony p 79}


Las reflexiones de Anscombe le llevarán a dialogar más con las preocupaciones de
Hume.

Conviene anticipar algo sobre sus propuestas. ¿Qué lugar ocupa el
testimonio en la descripción que hace sobre el conocimiento?

En primer lugar cabe destacar que le considera un tema poco desarrollado en la
filosofía antigua y moderna:
\citalitlar{It may, therefore, be a subject worthy of curiosity, to enquire what
  is the nature of that evidence which assures us of any real existence and
  matter of fact, beyond the present testimony of our senses, or the records of
  our memory. This part of philosophy, it is observable, has been little
  cultivated, either by the ancients or moderns}



To apply these principles to a particular instance, we may observe, that there
is no species of reasoning more common, more useful, and even necessary to human
life, than that which is derived from the testimony of men, and the reports of
eye witnesses and spectators. This species of reasoning, perhaps, one may deny
to be founded on the relation of cause and effect. I shall not dispute about a
word. It will be sufficient to observe, that our assurance in any argument of
this kind, is derived from no other principle than our observation of the
veracity of human testimony, and of the usual conformity of facts to the reports
of witnesses. It being a general maxim, that no objects have any discoverable
connection together, and that all the inferences which we can draw from one to
another, are founded merely on our experience of their constant and regular
conjunction; it is evident, that we ought not to make an exception to this maxim
in favour of human testimony, whose connection with any event seems, in itself,
as little necessary as any other. Were not the memory tenacious to a certain
degree; had not men commonly an inclination to truth and a principle of probity;
were they not sensible to shame, when detected in a falsehood; were not these, I
say, discovered by experience to be qualities inherent in human nature, we
should never repose the least confidence in human testimony. A man delirious, or
noted for falsehood and villany, has no manner of authority with us.


en la vida social cabe aceptar un conocimiento por testimonio a condición de que
su grado de certeza se limite a la probabilidad

\subsection{Puede haber un hecho histórico que signifique una realidad absoluta?}
no se puede aceptar una comunicación divina que no sea inmediatamente dirigida
al individuo

\subsection{Lessing es lo mismo relatos de milagros etc...}
no se puede tener conocimiento directo de milagros
o profecías

I am the better pleased with the method of reasoning here delivered, as I think
it may serve to confound those dangerous friends, or disguised enemies to the
Christian religion, who have undertaken to defend it by the principles of human
reason. Our most holy religion is founded on faith, not on reason; and it is a
sure method of exposing it, to put it to such a trial as it is by no means
fitted to endure. To make this more evident, let us examine those miracles
related in Scrip ture; and, not to lose ourselves in too wide a field, let us
con fine ourselves to such as we find in the Pentateuch, which we shall examine
according to the principles of these pretended Christians, not as the word or
testimony of God himself, but as the production of a mere human writer and
historian. Here then we are first to consider a book, presented to us by a
barbarous and ignorant people, written in an age when they were still more
barbarous, and in all probability long after the facts which it relates,
corroborated by no concurring testimony, and resembling those fabulous accounts
which every nation gives of its origin. Upon reading this book, we find it full
of prodigies and miracles. It gives an account of a state of the world and of
human nature entirely different from the present: of our fall from that state;
of the age of man extended to near a thousand years; of the destruction of the
world by a deluge; of the arbitrary choice of one people, as the favourites of
heaven, and that people the countrymen of the author; of their deliverance from
bondage by prodigies the most astonishing imaginable.

I desire any one to lay his hand upon his heart, and, after a serious
consideration, declare, whether he thinks that the falsehood of such a book,
supported by such a testimony, would be more extraordinary and miraculous than
all the miracles it relates; which is, however, necessary to make it be received
according to the measures of probability above established.

\subsection{El lenguaje religioso es capaz de transmitir alguna verdad?}
Ciruclo de viena, falsabilidad como cuarto atributo de el conocimiento
proposicional. cita de analogía teológica.
