\section{Naturaleza de la pregunta sobre el testimonio}
Es una experiencia familiar en nuestras comunidades reunirnos en torno a la
Sagrada Escritura y compartir la Palabra buscando en ella luz para nuestro
presente. Podemos apelar a esta experiencia familiar para apoyar el primer paso
de nuestra investigación sobre el testimonio. Imaginemos un domingo,
específicamente un tercer domingo del Tiempo Ordinario. En el ciclo A, el
Evangelio que se proclama ese día es este:
\blockquote[Mt~4,12--23]{Al enterarse Jesús de que habían arrestado a Juan se
  retiró a Galilea. Dejando Nazaret se estableció en Cafarnaún, junto al mar, en
  el territorio de Zabulón y Neftalí, para que se cumpliera lo dicho por medio
  del profeta Isaías:

  \enquote{Tierra de Zabulón y tierra de Neftalí, camino del mar, al otro lado
    del Jordán, Galilea de los gentiles. El pueblo que habitaba en tinieblas vio
    una luz grande; a los que habitaban en tierra y sombras de muerte, una luz
    les brilló}.

  Desde entonces comenzó Jesús a predicar diciendo: \enquote{Convertíos, porque
    está cerca el reino de los cielos}.

  Paseando junto al mar de Galilea vio a dos hermanos, a Simón, llamado Pedro, y
  a Andrés, que estaban echando la red en el mar, pues eran pescadores. Les
  dijo: \enquote{Venid en pos de mí y os haré pescadores de hombres}.
  Inmediatamente dejaron las redes y lo siguieron. Y pasando adelante vio a
  otros dos hermanos, a Santiago, hijo de Zebedeo, y a Juan, su hermano, que
  estaban en la barca repasando las redes con Zebedeo, su padre, y los llamó.
  Inmediatamente dejaron la barca y a su padre y lo siguieron. Jesús recorría
  toda Galilea enseñando en sus sinagogas, proclamando el evangelio del reino y
  curando toda enfermedad y toda dolencia en el pueblo.}

No sería difícil ahora visualizar una variedad de escenarios en los que este
texto pueda ser discutido en nuestro contexto eclesial. En enero de 2014 el Papa
Francisco lo reflexionaba en el Ángelus en la Plaza de San Pedro y destacaba que
la misión de Jesús comenzara en una zona periférica:

\blockquote[{\cite{francisco2014angelus}}]{Es una tierra de frontera, una zona
  de tránsito donde se encuentran personas diversas por raza, cultura y
  religión. La Galilea se convierte así en el lugar simbólico para la apertura
  del Evangelio a todos los pueblos. Desde este punto de vista, Galilea se
  asemeja al mundo de hoy: presencia simultánea de diversas culturas, necesidad
  de confrontación y necesidad de encuentro. También nosotros estamos inmersos
  cada día en una \enquote{Galilea de los gentiles}, y en este tipo de contexto
  podemos asustarnos y ceder a la tentación de construir recintos para estar más
  seguros, más protegidos. Pero Jesús nos enseña que la Buena Noticia, que Él
  trae, no está reservada a una parte de la humanidad, sino que se ha de
  comunicar a todos. Es un feliz anuncio destinado a quienes lo esperan, pero
  también a quienes tal vez ya no esperan nada y no tienen ni siquiera la fuerza
  de buscar y pedir.}

El Papa Benedicto XVI también había ofrecido su comentario unos años antes. En
su caso el acento del relato lo encontró en la fuerza singular de esa
\enquote{buena nueva} que Cristo comenzaba a anunciar:

\blockquote[{\cite{benedicto2008angelus}}]{El término \enquote{evangelio}, en
  tiempos de Jesús, lo usaban los emperadores romanos para sus proclamas.
  Independientemente de su contenido, se definían \enquote{buenas nuevas}, es
  decir, anuncios de salvación, porque el emperador era considerado el señor del
  mundo, y sus edictos, buenos presagios. Por eso, aplicar esta
  palabra a la predicación de Jesús asumió un sentido fuertemente crítico, como
  para decir: Dios, no el emperador, es el Señor del mundo, y el verdadero
  Evangelio es el de Jesucristo.

  La \enquote{buena nueva} que Jesús proclama se resume en estas palabras:
  \enquote{El reino de Dios ---o reino de los cielos--- está cerca}. ¿Qué
  significa esta expresión? Ciertamente, no indica un reino terreno, delimitado
  en el espacio y en el tiempo; anuncia que Dios es quien reina, que Dios es el
  Señor, y que su señorío está presente, es actual, se está realizando.

  Por tanto, la novedad del mensaje de Cristo es que en él Dios se ha hecho
  cercano, que ya reina en medio de nosotros, como lo demuestran los milagros y
  las curaciones que realiza.}

Ciertamente este texto no se le encontraría solamente en San Pedro, sino que
estaría presente en la celebración de la eucaristía domincal resonando en las
comunidades y parroquias alrededor del mundo; en las homilias, oraciones,
reflexiones o cánticos, invitando a la conversión y haciendo nueva la invitación
de Jesús: \enquote{Convertíos, porque está cerca el reino de los cielos}. Quizás
tambíen se le oiga entre algún grupo juvenil donde Simón, Andrés, Santiago y
Juan sean tratados como modelos de vocación a la vida consagrada o al
apostolado, atendiendo con entusiasmo cómo lo dejaron todo en el momento para
seguir a Jesús. Seguramente algún joven reconociendo aquella llamada:
\enquote{Venid en pos de mí y os haré pescadores de hombres} sonando como voz
dentro de sí.

El texto de la Escritura es tratado en estos contextos como testimonio de la
vida de Jesucristo y de la vida de aquellos que le llaman maestro y que
participan de su misión. No son, sin embargo, tratados como historias del
pasado, sino como palabras para el presente. Es hoy que la Buena Noticia no está
reservada a una parte de la humanidad, sino que ha de comunicarse a todos como
insiste el Papa Francisco. Es hoy que Dios se hace cercano en Cristo para reinar
en medio de nosotros como enseñó Benedicto XVI. Es hoy que Jesús nos invita a la
conversión y a ir en pos de él.

Es sobre esta costumbre de la Iglesia que quisieramos formular una pregunta que
ponga en marcha nuestra investigación. Para esto nos servirá emplear otra
costumbre eclesial y acudir al pensamiento de San Agustín para encontrar algo de
luz. En el capítulo XI de las \emph{Confesiones} nos lo encontramos inquieto
---como siempre--- esta vez pensando en Dios y pensando en el tiempo, asaltado
por una serie de preguntas:

\blockquote[{\cite[XI.14 n.17]{confesiones}}]{¿Qué es, pues, el tiempo? ¿Quién
  podrá explicar esto fácil y brevemente? ¿Quién podrá comprenderlo con el
  pensamiento, para hablar luego de él? Y, sin embargo, ¿qué cosa más familiar y
  conocida mentamos en nuestras conversaciones que el tiempo? Y cuando hablamos
  de él, sabemos sin duda qué es, como sabemos o entendemos lo que es cuando lo
  oímos pronunciar a otro. ¿Qué es, pues, el tiempo? Si nadie me lo pregunta, lo
  sé; pero si quiero explicárselo al que me lo pregunta, no lo sé.}

Agustín expresa su extrañeza de que un concepto empleado ordinariamente se torne
tan desconocido cuando llega la hora de explicarlo. \enquote{¿Qué es el tiempo?}
o \enquote{¿qué es conocer?}, \enquote{¿la libertad?} y \enquote{¿qué es la fe?}
son preguntas de este tipo; distintas, por ejemplo, a \enquote{¿cuál es el peso
  exacto de este objeto?} o \enquote{¿quién será la próxima persona en entrar
  por esa puerta?}.\autocite[Cf.][304]{wittgenstein2005bt} Preguntar
\enquote{¿qué es conocer una verdad para la vida por el testimonio de la
  Escritura?} sería, como la pregunta agustiniana sobre el tiempo, una pregunta
sobre la naturaleza o esencia de este fenómeno. Un concepto familiar en la vida
de la Iglesia como el testimonio queda enmarcado como problema cuando nos
acercamos a él queriendo comprender su esencia.

Esto ya nos da una pista sobre el tipo de pregunta que queremos hacer. El
siguiente paso ahora será precisar un poco cómo Elizabeth Anscombe se conduce a
través de este tipo de cuestiones. A modo de telón de fondo que nos sirva de
contraste podemos emplear una investigación similar realizada a comienzos del
siglo XX por el psicólogo William James.
Al comienzo de sus conferencias sobre
algunas preguntas sobre la Escritura al comienzo de sus conferencias sobre la
\emph{religión natural}. Apelando a la literatura de lógica de su época a
comienzos del siglo XX distingue dos niveles de investigación sobre cualquier
tema: aquellas preguntas que se resuelven por medio de prposiciones
\emph{existenciales}, como \enquote{¿qué constitución, qué origen, qué historia
  tiene esto?} o \enquote{¿cómo se ha realizado esto?}; en segundo lugar las
preguntas que se responden con proposiciones de \emph{valor} como \enquote{¿cuál
  es la importancia, sentido o significado actual de esto?}. A este segundo
juicio James lo denomina \emph{juicio espiritual}. Aplicando esta distinción a
la Biblia se cuestiona:

\blockquote[{\cite[27]{james2002variedades}}]{\enquote{¿Bajo qué condiciones
    biográficas los escritores sagrados aportan sus diferentes contribuciones al
    volumen sacro?}, \enquote{¿Cúal era exactamente el contenido intelectual de
    sus declaraciones en cada caso particular?}. Por supuesto, éstas son
  preguntas sobre hechos históricos y no vemos cómo las respuestas pueden
  resolver, de súbito, la última pregunta: \enquote{¿De qué modo este libro, que
    nace de la forma descrita, puede ser una guía para nuestra vida y una
    revelación?}. Para contestar habríamos de poseer alguna teoría general que
  nos mostrara con qué peculiaridades ha de contar una cosa para adquirir valor
  en lo que concierne a la revelación; y, en ella misma, tal teoría sería lo que
  antes hemos denominado un juicio espiritual.}

Desde esta perspectiva la pregunta sobre cómo el testimonio de la escritura
puede ser una guía para nuestra vida es una investigación sobre la importancia,
sentido o significado que ésta tiene actualmente. La respuesta emitida en
conclusión sería un juicio de valor sobre el fenómeno del testimonio. James
propone que sería necesaria una teoría general que explicara qué características
ha de tener alguna cosa para que merezca ser valorada como revelación. Así
planteado, la pregunta sobre el testimonio sería atendida adecuadamente por
medio de una investigación que indagara dentro de este fenómeno para descubrir
los elementos que le otorgan el valor adecuado como para ser considerado guía
para nuestra vida o una revelación. La explicación de dichos elementos
configurarían una teoría que nos permitiría juzgar un testimonio concreto como
valioso, o no, como guía o revelación para nuestras vidas.

      La ruta sugerida por este modo de conducir la investigación, sin embargo, nos
      dejaría apartados de la manera en que Elizabeth Anscombe se plantea un problema
      filosófico. En el trasfondo de su metodología filosófica está la propuesta por
      Ludwig Wittgenstein. Aunque se verá con más detalle qué implica esto, es
      necesario anticipar ahora algo acerca del modo en que ambos se encaminan a la
      hora de atender una investigación filosófica.

      En \emph{Investigaciones Filosóficas} \S89 Wittgenstein hace referencia al
      texto antes citado de las Confesiones para describir la peculiaridad de las
      preguntas filosóficas:
      \citalitlar{ Augustine says in \emph{Confessions} XI. 14, ``quid est ergo
        tempus? si nemo ex me quaerat scio; si quaerenti explicare velim nescio''.
        --This could not be said about a question of natural science (``What is the
        specific gravity of hydrogen'', for instance). Something that one knows when
        nobody asks one but no longer knows when one is asked to explain it, is
        something that has to be \emph{called to mind}. (And it is obviously
        something which, for some reason, it is difficult to call to mind.)}

      Para Wittgenstein es de gran importancia atender el paso que damos para
      resolver la perplejidad causada por el reclamo de explicar un fenómeno. El
      deseo de aclararlo nos puede impulsar a buscar una explicación dentro del
      fenómeno mismo, o como él diría: \citalitinterlin{We feel as if we had to see
        right into phenomena}.\footnote{\S90} Esta predisposición nos puede conducir
      a ignorar la amplitud del modo en que el lenguaje sobre esto es empleado en la
      actividad humana y a enfocarnos sólo en un elemento particular del lenguaje
      sobre este fenómeno y tomarlo como un ejemplo paradigmático para construir un
      modelo abstrayendo explicaciones y generalizaciones sobre él. Esta manera de
      indagar, le parece a Wittgenstein, nos hunde cada vez más profundamente en un
      estado de frustración y confusión filosófica de modo que llegamos a imaginar
      que para alcanzar claridad \citalitinterlin{we have to describe extreme
        subtleties, which again we are quite unable to describe with the means at
        our disposal. We feel as if we had to repair a torn spider's web with our
        fingers.}\footnote{\S106}

      La alternativa que Wittgenstein propone es una investigación que no esté
      dirigida hacia dentro del fenómeno, sino \citalitinterlin{as one might say,
        towards the \emph{`possibilities'} of phenomena. What that means is that we
        call to mind the \emph{kinds of statement} that we make about phenomena}. A
      este esfuerzo le denomina ``investigación gramática''. La describe de este modo:
      \citalitlar{ Our inquiry is therefore a grammatical one. And this inquiry sheds
        light on our problem by clearing misunderstandings away. Misunderstandings
        concerning the use of words, brought about, among other things, by certain
        analogies between the forms of expression in different regions of our
        language. -- Some of them can be removed by substituting one form of
        expression for another; this may be called `analysing' our forms of
        expression, for sometimes this procedure resembles taking things
        apart.\footnote{\S90}} El modo de salir de nuestra perplejidad, por tanto,
      consiste en prestar cuidadosa atención al uso que hacemos de hecho con las
      palabras y la aplicación que empleamos de las expresiones. Esto está al
      descubierto en nuestro uso del lenguaje de modo que la dificultad para
      \emph{traer a la mente} aquello que aclare un fenómeno no está en descubrir algo
      oculto en éste, sino en aprender a valorar lo que tenemos ante nuestra vista:
      \citalitinterlin{The aspects of things that are most important for us are hidden
        because of their simplicity and familiarity. (One is unable to notice
        something -- because it is always before one's eyes.)}\footnote{\S129} La
      descripción de los hechos concernientes al uso del lenguaje en nuestra actividad
      humana ordinaria componen los pasos del tipo de investigación sugerido por
      Wittgenstein. Hay cierta insatisfacción en este modo de proceder, como él mismo
      afirma: \citalitlar{Where does this investigation get its importance from, given
        that it seems only to destroy everything interesting: that is, all that is
        great and important? (As it were, all the buildings, leaving behind only bits
        of stone and rubble.) But what we are destroying are only houses of cards, and
        we are
        clearing up the ground of language on which they stood.\\
        The results of philosophy are the discovery of some piece of plain nonsense
        and the bumps that the understanding has got running up against the limit of
        language. They -- these bumps -- make us see the value of that discovery.}

      Anscombe, al igual que Wittgenstein, no se limita a emplear un sólo método para
      hacer filosofía, como afirma el mismo Wittgenstein: \citalitinterlin{There is
        not a single philosophical method, though there are indeed methods, different
        therapies as it were}.\footnote{\S133} Sin embargo si atendemos a su modo de
      hacer filosofía podemos encontrarla empleando lenguajes o juegos de lenguaje
      imaginarios para arrojar luz sobre modos actuales de usar el lenguaje o esquemas
      conceptuales; del mismo modo su trabajo esta lleno de ejemplos donde la
      encontramos examinando con detenimiento el uso que de hecho hacemos del
      lenguaje.\footnote{cf. teichmann p. 228-229} Es visible en ella ese
      \citalitinterlin{modo característicamente Wittgensteniano de rebatir la
        tendencia del filósofo de explicar alguna cuestión filosóficamente enigmática
        inventando una entidad o evento que la causa, así como los físicos inventan
        partículas como el gravitón}.\footnote{There is however a somehow
        chracteristically Wittgenstenian way of countering the philosopher's tendency
        to explain a philosophically puzzling thing by inventing an entity or event
        which causes it, as physicists invent particles like the graviton. From plato
        to witt intro xix}

      Según el título de este trabajo ha prometido, el análisis sobre el testimonio
      que será expuesto es el que se encuentra desarrollado en el pensamiento de
      Elizabeth Anscombe. La pregunta planteada al inicio: ¿qué es conocer una verdad
      para la vida por el testimonio de la Escritura?, entendida como investigación
      filosófica, será examinada en las descripiciones que Anscombe realiza sobre el
      modo de usar el lenguaje sobre el creer, la confianza, la verdad, la fe y otros
      fenómenos relacionados con el conocer por testimonio. Nuestro título adiverte
      además que ésta es una investigación en perspectiva teólogica, cabe
      inmendiatamente añadir algo breve al respecto.

      ¿Qué es teología?, se preguntaba Joseph Ratzinger en su alocución en el 75
      aniversario del nacimiento del cardenal Hermann Volk en 1978, e introducía
      suscintamente su respuesta a esa pregunta tan grande diciendo:

      \citalitlar{Cuando se intenta decir algo sobre esta materia, precisamente como
        tributo al cardenal Volk y a su pensamiento, se asocian, poco menos que
        automáticamente, dos ideas. Me viene a las mientes, por un lado, su divisa (y
        título de uno de sus libros): \emph{Dios todo en todos}, y el programa
        espiritual contenido en ella; por otra parte, se aviva el recuerdo de lo que
        ya antes se ha insinuado: un modo de interrogar total y absolutamente
        filosófico, que no se detiene en reales o supuestas comprobaciones históricas,
        en diagnósticos sociológicos o en técnicas pastorales, sino que se lanza
        implacablemente a la busqueda de los fundamentos.\\
        Según esto, cabría formular ya dos tesis que pueden servirnos de hilo
        conductor para nuestro interrogante sobre la esencia de la teología:\\
        1. La teología se refiere a Dios.\\
        2. El pensamiento teológico está vinculado al modo de cuestionar filosófico
        como a su método fundamental.\footnote{teoría de los principios teológicos, p
          380}}
      Esta investigación sobre el testimonio como parte de la vida de la Iglesia será
      realizada atendiendo al modo de cuestionar filosófico realizado por Elizabeth
      Anscombe como método, examinando esta experiencia en referencia a Dios, es
      decir, como vivencia de su ser y de su obrar.

      Hasta aquí simplemente se ha descrito un modo de andar a través de la discusión
      acerca de la categoría del testimonio atendiendo el hecho de que tanto la
      temática como la figura de Anscombe otorgan a este camino peculiaridades que hay
      que tener en cuenta. Siendo concientes de estas particularidades podríamos ahora
      ampliar más el horizonte respecto de dos cuestiones brevemente expuestas
      anteriormente. En primer lugar es necesario ampliar la descripción hecha hasta
      aquí del fenómeno del testimonio en la vida de la Iglesia, ya que aunque nos
      resulte familiar relacionarlo con el testimonio de la Sagrada Escritura, tanto
      en el Magisterio de la Iglesia como en la propia Escritura se haya presente la
      categoría del testimonio con una riqueza que merece la pena explorar. En segundo
      lugar habría que detallar todavía mejor lo problemático del testimonio, sobre
      todo cuando se considera su importancia en la transmisión de la fe y el anuncio
      del Evangelio en el mundo.
