% SECCIÓN 1: La categoría del testimonio en la tradición
\section{La categoría del testimonio en la tradición}

Rm 10,8-15: 

Entonces, ¿qué dice? 'Cerca de ti está la palabra: en tu boca y en tu corazón,'
es decir, la palabra de la fe que nosotros proclamamos. Porque, si confiesas con
tu boca que Jesús es Señor y crees en tu corazón que Dios le resucitó de entre
los muertos, serás salvo. Pues con el corazón se cree para conseguir la
justicia, y con la boca se confiesa para conseguir la salvación.

Porque dice la Escritura: 'Todo el que crea en él no será confundido.' Que no
hay distinción entre judío y griego, pues uno mismo es el Señor de todos, rico
para todos los que le invocan. Pues 'todo el que invoque el nombre del Señor se
salvará.' Pero ¿cómo invocarán a aquel en quien no han creído? ¿Cómo creerán en
aquel a quien no han oído? ¿Cómo oirán sin que se les predique? Y ¿cómo
predicarán si no son enviados? Como dice la Escritura: '¡Cuán hermosos los pies
de los que anuncian el bien!'

\subsection{Cristo Testigo}

\subsection{La transmisión de la revelación: el Testimonio apostólico}

\subsection{El testimonio de la fe en el martirio}

\subsection{Revalorización del testimonio a partir del Concilio Vaticano II}

\subsubsection{Vaticano I}

\subsubsection{Vaticano II}

\subsubsection{Magisterio Post-Conciliar}

\subsubsection{Nueva Evangelización}

\section{El Testimonio como categoría hermeneútica}

\section{El Testimonio como misión}
