Gertrude Elizabeth Margaret Anscombe nació el 18 de marzo de 1919, la tercera
hija de Gertrude Elizabeth y Alan Wells Anscombe. Aquel año la familia se
hallaba en Irlanda donde el Capitán Anscombe había sido asignado como parte de
un regimiento Británico instalado en Limerick. Al terminar la guerra la familia
regresó a Londres donde Alan era profesor de secundaria en Dulwich
College.\footcite[cf.~][p.~31]{biofellows}

Elizabeth hizo sus estudios de bachillerato en Sydenham High School, una
escuela independiente localizada a las afueras de Londres y fundada en 1887 por
la Girl's Public Day School Trust con el fin de ofrecer oportunidades de
educación para mujeres. Se graduó en el curso 1936--1937.

Con doce años de edad Anscombe descubrió el Catolicismo leyendo testimonios de
las obras y sufrimientos de los sacerdotes recusantes en Inglaterra durante la
época Isabelina. Esta y otras lecturas realizadas entre los doce y los quince
motivaron su conversión a la fe católica.\footcite[cf.~][p.~33]{biofellows}

Tras su graduación de Sydenham recibió una beca y fue admitida en St.~Hugh's
College en la Universidad de Oxford. Allí estudió \emph{Litterae Humaniores}, un
programa de cuatro años dividido en dos periodos: \emph{Classical Honour
  Moderations} (`Mods') y \emph{Final Honour School} (`Greats'). En 1939
Anscombe recibió Second Class en `Mods' compuesto por estudios en latín y griego
y literatura antigua que servían como preparación para el segundo periodo. En
1941 recibió First Class en \emph{Litterae Humaniores} cuando culminó los
exámenes de 'Greats' que comprendía estudios de filosofía y de historia.

Durante su primer año en Oxford recibió formación en la fe del sacerdote
dominico Richard Kehoe, profesor del Blackfriar's Private Hall, centro docente
perteneciente a la Orden de Predicadores. El 27 de abril de 1938 fue admitida en
la Iglesia Católica.

En la procesión del \emph{Corpus Christi} de aquel año conoció a otro catecumeno
del Padre Kehoe, su nombre era Peter Geach. Había recibido su admisión a la
Iglesia unas semanas después de ella, su madre era polaca, su padre maestro de
filosofía. Había sido instruido en lógica por su padre teniendo como libros de
texto \emph{Formal Logic} de Neville Keynes y \emph{Principia Mathematica} de
Bertrand Russell. A los pocos meses de conocerse se habían comprometido y el 26
de diciembre de 1941 Elizabeth y Peter se casaron en el Brompton Oratory de
Londres.\footcite[cf.~][p.~33]{biofellows}

En el tiempo en el que Anscombe estuvo en St.~Hugh's el programa de lecciones
manifestaba la transformación ocurrida en la universidad durante los últimos
cincuenta años; desde una docencia e interés de carácter teológico hacia una
orientación más secular. En el periodo de `Greats' los estudios de filosofía se
fundaban en la República de Platón y la Ética Nicomáquea de Aristóteles. Además
de las lecciones dedicadas a los clásicos se estudiaba a filósofos modernos como
Berkeley, Locke, Hume y Kant. Al estudio de la Crítica de la Razón Pura se le
dedicaban lecciones que ocupaban los tres periodos lectivos de un año académico.
Había interes por temas de ética y teoría del conocimiento, así como por temas
relacionados con psicología y ética: motivación, acción, libertad. Se estudiaba
también a Hobbes y Rosseau y teoría política. Sin embargo, había pocas lecciones
dedicadas a cuestiones metafísicas o estéticas. De filosofía medieval se ofrecía
solo una lección dedicada a Tomás de Aquino.\footcite[cf.~][pp.~23-24]{accint}

Los estudiantes de Oxford contaban con un tutor en la preparación de sus
materias. Anscombe contó con la supervisión de G.~Ryle quien en 1939 ofreció el
curso de introducción a la filosofía y también otro curso sobre el
\emph{Tractatus} de Wittgenstein, junto con el joven
A.~J.~Ayer.\footcite[cf.~][p.~24]{accint}


En 1941 Anscombe continuó en Oxford como `Research Student' y en 1942 obtuvo una
`Research Fellowship' en el Newnham College en Cambrdige.
