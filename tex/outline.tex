\documentclass[./main.tex]{subfiles}
\begin{document}

\setcounter{chapter}{2}
\chapter{
Desarrollo filosófico de Elizabeth Anscombe
}

\section{Wittgenstein en el Transfondo de la Filosofía de Anscombe}

\subsection{El método filosófico como esclarecimiento del lenguaje}

\subsection{Dos Cortes en la Filosofía}

\subsection{Desarrollo del Tractatus}

\subsection{Las elucidaciones del Tractatus}

\subsubsection{Filosofía como actividad}

\subsubsection{El pensamiento como representación}

\subsubsection{Proposiciones elementales}

\subsubsection{Proposiciones como proyecciones con polos de verdad-falsedad}

\subsubsection{La distinción entre el decir y el mostrar}

\subsection{El método filosófico como medicina para la confusión}

\subsection{Investigaciones Filosóficas}

\subsubsection{Descripción agustiniana del lenguaje}

\subsubsection{El uso como significado}

\subsubsection{La gramática como esencia}

\subsubsection{Juegos de Lenguaje}

\subsubsection{Seguir una regla}

\subsubsection{Definiciones Ostensivas Privadas}

\newpage
\section{Actividad filosófica de Elizabeth Anscombe}

\subsection{Conversión y primeras reflexiones filosóficas}

\subsubsection{Causalidad}

\subsubsection{Percepción}

\subsection{Estudiante en Oxford}

\subsubsection{H.H. Price y Hume: Fenomenalismo y Escepticismo}

\subsubsection{\emph{The Justice of the Present War Examined}}

\subsection{Estudiante en Cambridge}

\subsection{Profesora en Oxford}

\subsubsection{\emph{A Reply to Mr C. S. Lewis's Argument that Naturalism is Self-Refuting}}

\subsubsection{Traducción y Publicación de Obra Póstuma de Wittgenstein}

\subsubsection{Intention}

\subsubsection{\emph{An Introduction to Wittgenstein's Tractatus}}

\subsubsection{\emph{Three Philosophers} con Peter Geach}

\subsection{Cátedra de Filosofía en Cambridge}

\subsubsection{\emph{Causality and Determination}}

\subsubsection{\emph{Collected Philosophical papers}}

\subsection{Actividad académica tras su retiro}

\setcounter{chapter}{3}
\chapter{
La Categoría del Testimonio en el Pensamiento de Elizabeth Anscombe
}

\section{Verdad}

\subsection{``Truth'' y ``Truth Sense and Assertion''}

¿Qué es la primacía de la verdad sobre la falsedad? Anscombe recorre el análisis
en torno a esta cuestión realizado por San Anselmo en el Capítulo 2 del \emph{De
  Veritate} y por Wittgenstein en el \emph{Tractatus}.

\subsection{``The Unity of Truth''}

\subsection{``Making True''}

\section{Creer}

\subsection{``What is it to believe someone?''}

\subsection{``Belief and Thought''}

\subsection{``Grounds of belief''}

\subsection{``Motives for beliefs of all sorts''}

\section{Fe}

\subsection{``Faith''}

\subsection{``Parmenides, contradiction and Mystery''}

¿Podemos llegar a decir de alguna proposición particular `esto es verdad, pero
lo que afirma es irreductiblemente enigmático'? ¿Debería descartarse este tipo
de afirmación amparados en la idea de que `lo que puede ser dicho ha de ser
dicho claramente'? Esto implicaría que todos los misterios, incluyendo los
misterios centrales del Cristianismo son meramente ilusiones.

Anscombe le dedica breves palabras al rechazo de esta postura. Afirma que no
hay fundamento para el punto de vista de que nada que no pueda ser captado en
el pensamiento puede ser verdad. Esto es más bien un tipo de prejuicio.

\subsection{``Hume on Miracles''}

\subsection{``Prophecy and Miracles''}

\subsection{``On Transubstantiation''}

\end{document}
