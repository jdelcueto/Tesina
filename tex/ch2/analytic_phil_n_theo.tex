\section{La filosofía analítica y la teología del testimonio}

Antes de adentrarnos en el pensamiento de Anscombe simplemente anotamos aquí algo sobre la motivación para estudiar la categoría teológica del testimonio dentro de la filosofía analítica. Quizás las últimas consideraciones del apartado anterior anticipan una pista al respecto.

Elizabeth misma explica que `filosofía analítica' representa más un modo de investigar que un contenido doctrinal\footnote{\cite[66]{anscombe2008faith:twenty}: \enquote{Analytical philosophy is more characterised by styles of argument and investigation than by doctrinal content. It is thus possible for people of widely different beliefs to be practitioners of this sort of philosophy. It ought not to surprise anyone that a seriously believing Catholic Christian should also be an analytical philosopher}}. El tipo de análisis del lenguaje propuesto en \emph{Investigaciones Filosóficas} ofrece un paradigma de estudio que está bien en sintonía con la naturaleza del fenómeno del testimonio puesto que nos dirige a la actividad pública de la vida humana, donde al testimonio se le encuentra en acción.

C.\,A.\,J. Coady, en su investigación sobre el testimonio, nota que este es un tema relativamente poco atendido en épocas pasadas del pensamiento filosófico. Entonces se pregunta si esta negligencia se debe al hecho de que el testimonio verdaderamente no juega un papel significativo en la formación de creencias razonables. Su conclusión será que no. Aún cuando el testimonio tiene de hecho un papel importante en el conocimiento humano, se ha dado por supuesto su poca importancia. Una de las explicaciones que Coady ofrece para este supuesto es la perspectiva individualista dominante en el pensamiento y la práctica política, social y económica del mundo occidental tras el renacimiento. Así como estas ideas e ideales han tenido efecto sobre nociones como la libertad y la sociedad, también han influenciado nuestro modo de pensar sobre el conocimiento, la verdad o la racionalidad\footnote{\cite[Cf.~][6-13]{coady1992test}}. En este sentido, un estudio dedicado al testimonio está motivado por el deseo de salir de una concepción y retórica individualista del conocimiento.

En sintonía con este interés es de mucho valor la aportación de Fergus Kerr en \emph{Theology after Wittgenstein}. Una de sus contiendas principales en el libro es que la teología moderna está saturada por presupuestos individualistas cartesianos\footnote{\cite[Cf.~][8]{kerr1997theo}: \enquote{My argument in this book is that, far from still having to incorporate Cartesian assumptions about the self, as Rahner supposed, modern theology is already saturated with them.}}. Su impresión es que muchos teólogos han pactado con este hecho:\blockquote[{\cite[10]{kerr1997theo}}: \enquote{Theologians are thus well aware of the difficulties that the modern philosophy of the self has created. My suspicion, however, is that versions of the mental ego of Cartesianism are ensconced in a great deal of Christian thinking, and that many theologians regard this as inevitable and even desirable.}]{Los teólogos están claramente conscientes de las dificultades que la filosofía moderna del ego ha creado. Mi sospecha, sin embargo, es que versiones del ego mental del cartesianismo están acomodadas en una gran cantidad de pensamiento cristiano, y que muchos teólogos consideran esto como algo inevitable o deseable}. Sin pretender hacer aquí una valoración o juicio de la imagen que Kerr describe sobre la teología, tomaremos una tésis central de su discusión, esto es, que las reflexiones de las etapas más tardías del pensamiento de Wittgenstein pueden servir a la teología para desmitificar el rol del ego desconectado del mundo y del lenguaje como paradigma teológico\footnote{\cite[Cf.~][23]{kerr1997theo}: \enquote{My claim is that the most illuminating exploration of the continuing power of the myth of the worldless (and often essentially wordless) ego is to be found in the later writings of Ludwig Wittgenstein.}}.

La propuesta de Kerr merecería una consideración más detallada, pero aquí nos limitaremos a referir dos de sus premisas relacionadas con \emph{Investigaciones Filosóficas}. Lo primero es que es llamativa la cita de San Agustín en el punto de partida del análisis de Wittgenstein. Fergus destaca que Ludwig pudo usar el argumento de algún otro autor para establecer el mismo punto: \blockquote[{\cite[42]{kerr1997theo}}: \enquote{he could have found it in many other philosophers: he might have been thinking of the author of the \emph{Tractatus}. Finally however, he chose to direct his readers to the greatest autobiography in the Christian tradition. To probe the epistemological predicament of the soul in the \emph{Confessions} was to open up a seam in the theological anthropology that has shaped Christian self-understanding since the fifth century. It is difficult to believe that Wittgenstein did not know what he was doing.}]{pudo haberlo encontrado en otros filósofos: pudo haber pensado en el autor del \emph{Tractatus}. Finalmente, sin embargo, escogió dirigir a sus lectores a la más grande autobiografía de la tradición cristiana. Al sondear el dilema epistemológico del alma en las \emph{Confesiones} accedió a una veta en la antropología teológica que ha dado forma a la autocomprensión cristiana desde el siglo quinto. Es difícil creer que Wittgenstein no sabía lo que estaba haciendo}. El análisis que Ludwig hace sobre la relación entre la realidad, el lenguaje y el pensamiento no deja de ser un intento de reconocer definitivamente cómo somos realmente\footnote{\cite[Cf.~][23]{kerr1997theo}: \enquote{Wittgenstein invites us to remember ourselves as we really are. Once and for all, that is to say, we need to give up comparing ourselves with ethereal beings that enjoy unmediated communion with one another.}}.

Otro punto intersante tiene que ver con la breve referencia a la teología que Wittgenstein anota en \emph{Investigaciones Filosóficas} en el contexto de la discusión sobre la esencia de los conceptos. En \S371 dice: \blockquote[{\cite[\S371]{wittgenstein1953phiinv}}: \enquote{\emph{Essence} is expressed in grammar.}]{La \emph{esencia} se expresa en la gramática.} y continúa en \S373: \blockquote[{\cite[\S371]{wittgenstein1953phiinv}}: \enquote{Grammar tells what kind of object anything is. (Theology as grammar.)}]{La gramática nos dice qué tipo de objeto es cualquier cosa. (Teología como gramática.)} Además de esta mención no se encuentra otra explicación de esta noción, pero Kerr interpreta el comentario desde las discusiones sostenidas por Ludwig en sus lecciones: \blockquote[{\cite[32]{ambrose2001lectures}}: \enquote{What happens with the words ``God'' and ``soul'' is what happens with the word ``number''. Even though we give up explaining these words ostensively, by pointing, we don't give up explaining them in substantival terms. \textelp{} Luther said that theology is the grammar of the word ``God''. I interpret this to mean that an investigation of the word would be a grammatical one. For example, people might dispute by denying that one could talk about arms of God. This would throw light on the use of the word. What is ridiculous or blasphemous also shows the grammar of the word.}]{Lo que ocurre con las palabras ``Dios'' y ``alma'' es lo que ocurre con la palabra ``número''. Aún cuando renunciamos a explicarlas ostensivamente, a recurrir a apuntar a algo, no renunciamos a explicarlas en términos sustantivos. \textelp{} Lutero dijo que la teología es la gramática de la palabra ``Dios''. Yo interpreto que esto quiere decir que una investigación de esta palabra sería gramática. Por ejemplo, la gente puede debatir negando que sea posible hablar de los brazos de Dios. Esto arrojaría luz sobre el uso de la palabra. Lo que es ridículo o blasfemo también mostraría la gramática de la palabra}. Con esto Wittgenstein orienta el esfuerzo por indagar en las esencias de estas palabras, no a fenómenos ocultos en la realidad o el pensamiento, sino al fenómeno mismo del uso del lenguaje: \blockquote[{\cite[148-149]{kerr1997theo}}: \enquote{In effect, by remarking that theology is grammar, he is reminding us that it is only by listening to what we say about God (what has been said for many generations), and to how what is said about God ties in with what we say and do in innumerable other connections, that we have any chance of understanding what we mean when we speak of God.}]{En efecto, al observar que la teología es gramática, nos está recordando que es solo por medio de la escucha de lo que decimos sobre Dios (lo que se ha dicho por muchas generaciones), y de cómo lo que se dice sobre Dios está relacionado con lo que hacemos y decimos en otras innumerables conexiones, que tenemos alguna oportunidad de entender lo que decimos cuando hablamos de Dios}.
