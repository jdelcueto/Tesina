%SECCIÓN 2:
\section{Actividad Filosófica de Elizabeth Anscombe}

\subsection{Los primeros arduos esfuerzos}

\ifdraft{\subsubsection{Causalidad reflexiones iniciales de Anscombe}}{}

Por aquella época de mediados de los 30 la joven Gertrude Elizabeth Margaret
Anscombe, andaba buscando un buen argumento que demostrara que todo lo que
existe tiene que tener una causa. ¿Por qué cuando algo ocurre estamos seguros de
que tiene una causa? Nadie sabía darle una respuesta.\autocite[cf.~][p.~vii
]{anscombe1981metaphysicsintro} Así, sin darse cuenta, se iniciaba en lo que
sería para ella ardua actividad: la filosofía. Rigurosa y enérgica desde el
principio.

El origen de su peculiar curiosidad por la causalidad se hallaba en una obra
llamada `Teología Natural' escrita por un jesuita del siglo XIX. Había llegado a
este libro motivada por su conversión a la Iglesia
Católica.\autocite[cf.~][p.~vii]{anscombe1981metaphysicsintro} El tratado le
resultó problemático en dos cuestiones.

La primera fue la doctrina de la \emph{`scientia media'}, según la cual Dios
tiene conocimiento, por ejemplo, de lo que alguien podría haber hecho si no
hubiera muerto cuando murió. A Elizabeth le parecía que lo que hubiera ocurrido
si lo que pasó no hubiera pasado simplemente no existe; no hay qué conocer. Y no
podía creer esto. Anscombe tuvo la oportunidad de discutir esta preocupación con
Richard Kehoe durante su preparación religiosa en su primer año en Oxford. La
dificultad para creer aquella doctrina le parecía un límite para aceptar la fe
católica. Richard le aclaró que no hacía falta que creyera en eso. Con el tiempo
entendió que se trataba de una discusión de escuela, en la que los jesuítas y
dominicos entablaron una ardua disputa y que la postura que ella había adoptado
era la qu había sido defendida por los
dominicos.\autocite[cf.~][p.~vii]{anscombe1981metaphysicsintro}

La segunda cuestión problematica la encontró en un argumento sobre la existencia
de la `Causa Primera'. El tratado ofrecía como preliminar al argumento una
demostración de un `principio de causalidad' según el cual todo cuanto existe
tiene que tener una causa. Anscombe notó, escasamente escondido en una premisa,
un presupuesto de la conclusión del propio argumento. Aquel \emph{petitio
  principii} le pareció un simple descuido y resolvió, por tanto, escribir una
versión mejorada de la demostración. Durante los siguientes dos o tres años
produjo unas cinco versiones que le parecían satisfactorias, sin embargo
eventualmente descubría que contenían la misma falacia, cada vez disimulada más
astutamente. Todo este esfuerzo lo realizó sin ninguna enseñanza formal en
filosofía, incluso su último intento de argumento lo hizo antes de estudiar
`Greats'.\autocite[cf.~][p.~vii]{anscombe1981metaphysicsintro}

\ifdraft{\subsubsection{Oxford: La Percepción y el fenomenalismo de Price}}{}

Sus lecturas en torno a su conversión fueron motivo de más reflexiones. Esta
vez, como fruto de \emph{The Nature of Belief} de Martin D'Arcy, se interesó por
el tema de la percepción. Durante años ocupaba su tiempo, en cafeterías, por
ejemplo, mirando fijamente objetos, diciendose a sí misma: <<Veo un paquete.
¿Pero qué veo realmente? ¿Cómo puedo decir que veo algo más que una extensión
amarilla?>>\autocite[cf.~][p.~viii]{anscombe1981metaphysicsintro}

Al principio su impresión era que lo que veía eran objetos:
\citalitinterlin{Estaba segura de que veía objetos, como paquetes de cigarrillos
  o tazas o\ldots~cualquier cosa más o menos sustancial
  servía.}\autocite[p.~viii]{anscombe1981metaphysicsintro} Además creía que
debemos de conocer la categoría de un objeto cuando hablamos de él, eso
corresponde a la lógica del término usado para hablar del objeto y no de algún
descubrimiento empírico. Estas ideas, sin embargo, las había desarrollado
fijándose en artefactos urbanos. Los ejemplos de percepción de la naturaleza que
más la impactaron fueron `madera' y el cielo. Este último le hizo retractarse de
su creencia sobre el conocimiento lógico de la categoría de los
objetos.\autocite[cf.~][p.~viii]{anscombe1981metaphysicsintro}

Sus indagaciones sobre la percepción, así como le ocurrió con la causalidad,
fueron previas al periodo de `Greats' donde estudiaría formalmente la filosofía.
Ya desde `Mods' asistía a las lecciones de H.~H.~Price sobre percepción y
fenomenalismo. De todos los que escuchó en Oxford fue quién le inspiró mayor
respeto, no porque estuviera de acuerdo con lo que decía, sino porque hablaba de
lo que había que hablar. El único libro suyo que le pareció realmente bueno fue
\emph{Hume's Theory of the External World} y lo leyó sin interrupción de
principio a fin. Fue Price quien despertó en ella un intenso interés por el
capítulo de Hume sobre ``Del escepticismo con respecto a los sentidos''. Aunque
le parecía que Price tendía a suavizar a Hume, el hecho de que escribiera sobre
él le parecia que era escribir sobre las cosas mismas que merecía la pena
discutir. Asncombe, sin embargo, odiaba el fenomenalismo y se sentía atrapada
por él, pero no sabía salir de él, o rebatirlo. La postura escéptica tampoco la
convencía como para adoptarla y no la dejaba satisfecha. Esta insatisfacción no
haría más que crecer en sus años en Oxford.
\autocites[cf.~][p.~viii]{anscombe1981metaphysicsintro}
[~y~][p.~26]{torralba2005accion}

\ifdraft{\subsubsection{En Cambrdige con Wittgenstein}}{}

En las lecciones con Wittgenstein en Cambridge fue que el pensamiento central
<<Tengo \emph{esto}, y defino `amarillo' como \emph{esto}>> fue efectivamente
atacado. Anscombe misma lo narra usando dos ejemplos:

Anscombe conoció a Wittgenstein en los años culminantes de su pensamiento
     filosófico.
     Al comienzo de sus lecciones en 1944 Wittgenstein escribía a su amigo Rush Rhees:
     \citalitinterlin{
         \ldots mis clases no han ido tan mal. Thouless esta asistiendo, y una mujer,
         'Mrs so and so'
         que se llama a sí misma
         'Miss Anscombe',
         que ciertamente es inteligente, aunque no del calibre de Kreisel.
         \autocite[p.~371]{cambridgeletters}
     }
     Un año mas tarde escribía a Norman Malcolm:
     \citalitinterlin{
         \ldots mi clase ahora es bastante grande, 19 personas. \ldots Smythies esta
         viniendo, y una mujer que es muy buena, es decir, más que solamente
         inteligente\ldots
         \autocite[p.~388]{cambridgeletters}
     }
     Aquellos años no sólo creció en Wittgenstein la apreciación de la capacidad de
     Anscombe, sino que se afianzó entre ellos una estrecha amistad.

     La influencia de Wittgenstein fue decisiva para el desarrollo filosófico de
     Elizabeth. Las lecciones con Wittgenstein eran directas y con franqueza. Esta
     metodología carente de cualquier parafernalia era inquietante para algunos,
     inspiradora para otros, pero fue tremendamente liberadora para
     ella.\autocite[loc 9853 Chapter 4, Section 24, \S5]{monk} Esta libertad
     quedaba demostrada en que Anscombe no se contentaba con repetir lo que decía
     Wittgenstein, sino que pensaba por sí misma; en esto precisamente era más fiel
     al espíritu de la filosofía que había aprendido de él. Sobre esta relación,
     Phillipa Foot, amiga de ambos, cuenta que durante mucho tiempo sostuvo
     objeciones a las afirmaciones de Wittgenstein, eventualmente, un comentario de
     Norman Malcom la hizo pensar que podía haber valor en lo que Wittgenstein decía.
     Cuestionó entonces a Anscombe:
     ``¿Por qué no me dijiste?'', ella le contestó: ``Porque es importante que uno
     tenga sus resistencias''. Anscombe evidentemente pensaba ---continúa Foot:
     \citalitlar{
         que un largo periodo de vigorosa objeción era la mejor manera de entender a
         Wittgenstein. Aun cuando era su amiga cercana y albacea literaria, y una de
         los primeros en reconocer su grandeza, nada podía ser más lejano de su
         carácter y modo de pensamiento que el discipulado.\autocite[p.~4]{teichmann}
     }

     Peter geach que dice que les ayudó que estudiaron otros filósofos antes de
     Wittgenstein.

\pnote{introducir algunos contrastes y relaciones entre
       Anscombe y Wittgenstein para explicar la incursión en la vida/pensamiento
       de W.}
