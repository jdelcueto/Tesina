\section{La Filosofía de Wittgenstein como trasfondo de la Filosofía de Anscombe}

\subsection{Naturaleza y método de la filosofía}

Para comprender el método filosófico de Anscombe hay que tener en cuenta la filosofía de Wittgenstein. En este apartado se ofrece una descripción, que no pretende ser exhaustiva, de la manera en que Ludwig entiende la filosofía y de algunos puntos fundamentales de las dos etapas de su pensamiento.

Una de las constantes importantes del pensamiento de Wittgenstein fue su definición de la naturaleza de los problemas filosóficos. Para él las cuestiones de la filosofía no son problemáticas por ser erróneas, sino por no tener significado\autocite[Cf.~][\S4.003]{wittgenstein1922tractatuses}. Una proposición sin significado que no es puesta al descubierto como tal atrapa al filósofo dentro de una confusión del lenguaje que no le permite acceder a la realidad. Salir de la confusión no consiste en refutar una doctrina y plantear una teoría alternativa, sino en examinar las operaciones hechas con las palabras para llegar a manejar una visión clara del empleo de nuestras expresiones. La filosofía no es un cuerpo doctrinal, sino una actividad\autocite[Cf.~][\S4.112]{wittgenstein1922tractatuses} y una terapia\autocite[Cf.~][\S133]{wittgenstein1953phiinv}.

La actitud terapéutica adoptada por Wittgenstein en su atención de las confusiones filosóficas fue su respuesta más definitiva a la naturaleza de estos problemas. Para ello halló los más eficaces remedios en sus investigaciones sobre el significado y el sentido del lenguaje. Ordinariamente tomamos parte en esta actividad humana que es el lenguaje. Jugamos el juego del lenguaje. ---¿Jugarlo es entenderlo?--- A la vista de Wittgenstein saltaban extraños problemas sobre las reglas de este juego; entonces no podía evitar escudriñarlas al detalle\autocite[Cf.~][356]{monk1991duty}. En este análisis del lenguaje está la raíz de sus ideas sobre el sentido, el significado y la verdad.

Durante su vida sostuvo dos grandes descripciones del significado. Originalmente describió el lenguaje como una imagen que representa el posible estado de las cosas en el mundo. En una segunda etapa se distanció de esta analogía para describir al lenguaje como una herramienta cuyo significado consiste en la suma de las múltiples semejanzas familiares que aparecen en los distintos usos para los cuales el lenguaje es empleado en la actividad humana. Dentro de la primera descripción una expresión sin significado es una cuyos elementos no componen una representación del posible estado de las cosas. Dentro de la segunda descripción una expresión sin significado resulta del empleo de una expresión propia de un ``juego del lenguaje'' fuera de su contexto.

Estas dos etapas del pensamiento de Wittgenstein son representadas por dos importantes tratados. El \emph{'Tractatus Logico\=/Philosophicus'}, publicado en 1921, recoge sus esfuerzos por elaborar un gran tratado filosófico comenzados en 1911 y culminados durante la Primera Guerra Mundial. El segundo, \emph{'Philosophische Untersuchungen'}, o \emph{'Investigaciones Filosóficas'}, traducido por Anscombe y publicado posthumamente en 1953, fue elaborado a partir de múltiples manuscritos desarrollados por Wittgenstein desde su regreso a Cambridge en 1929 hasta su muerte en 1951.

\blockquote[{\cite[181]{anscombe2011plato:twocuts}}: \enquote{Wittgenstein is extraordinary among philosophers for having made two epochs, or cuts, in the history of philosophy.}]{Wittgenstein es extraordinario entre los filósofos por haber generado dos épocas, o cortes, en la historia de la filosofía}. Con estas palabras Anscombe comenzaría su discurso inaugural para el Sexto Simposio Internacional de Wittgenstein unos treinta años después de la publicación de las \emph{'Investigaciones Filosóficas'}. Y explica: \blockquote[{\cite[181]{anscombe2011plato:twocuts}}: \enquote{a philosopher makes a cut if he makes a difference to the way philosophy is done: philosophy after the cut cannot be the same as before.}]{un filósofo hace un corte si genera un cambio en el modo en que la filosofía es hecha: la filosofía tras el corte no puede ser la misma de antes}.

Estos cambios de época generados por la influencia de Wittgenstein vinieron caracterizados por el esfuerzo de comprender cada libro tras su publicación, tarea complicada en ambos casos por la dificultad intrínseca de los tratados, ofuscada a su vez por los prejuicios filosóficos proyectados a cada obra por sus lectores.  Elizabeth explica que: \blockquote[{\cite[Cf.~][183]{anscombe2011plato:twocuts}}: \enquote{the assumption that the \emph{Philosophical Investigations} presents us a theory of language ---a theory, say, of how sounds become significant speech--- will quickly place us at a distance from the very questions which Wittgenstein is occupied with.}]{la presunción, por ejemplo, de que \emph{'Investigaciones Filosóficas'} presenta una teoría del lenguaje ---quizás sobre cómo los sonidos se tornan en discursos significativos--- nos dejaría situados lejos de las preguntas que genuinamente ocupan a Wittgenstein}. Ahora bien, la comprensión adecuada de su pensamiento y método trae consigo, a juicio de Anscombe, cierto efecto curativo.

Según Anscombe el método general adecuado de discutir los problemas filosóficos propuesto por Wittgenstein consiste en mostrar que la persona no ha provisto significado (o referencia) para ciertos signos en sus expresiones\footnote{\cite[Cf.~][151]{anscombe1959iwt}: \enquote{The general method that Wittgenstein does suggest is that of `shewing that a man has supplied no meaning [or perhaps: ``no reference''] for certain signs in his sentences'.}}. Creía que el camino que lleva a formular estos problemas está frecuentemente trazado por la mala comprensión de la lógica de nuestro lenguaje.

Cada obra de Wittgenstein representa su esfuerzo de superar estas confusiones y propone un método para remediarlas. Su primera propuesta plantea que el modo de aclarar las confusiones de los problemas filosóficos consiste en identificar en el lenguaje el límite de lo que expresa pensamiento; lo que queda al otro lado de esta frontera sería simplemente sinsentido. En otras palabras: \blockquote[{\cite[11]{wittgenstein1922tractatuses}}]{lo que siquiera puede ser dicho, puede ser dicho claramente; y de lo que no se puede hablar, hay que callar}. Con esta expresión Wittgenstein resumió el sentido de la obra que ahora examinaremos.

\subsection{Las elucidaciones del \emph{Tractatus}}
% Este párrafo resume los cuatro puntos del Tractatus que se desglosarán en los próximos párrafos
Desde las proposiciones principales del Tractatus queda claro que el tema central del libro es la conexión entre el lenguaje, o el pensamiento, y la realidad.
% 1.Filosofía como actividad
En este nexo es donde la actividad filosófica ha de buscar esclarecer el pensamiento.
% 2.El pensamiento como representación
La tesis básica sobre esta relación consiste en que las proposiciones, o su equivalente en la mente, son imágenes de los hechos.
% 3.Las proposiciones como proyecciones con polos de verdad-falsedad
La proposición es la misma imagen tanto si es cierta como si es falsa, es decir, es la misma imagen sin importar que lo que se corresponde a esta es el caso que es cierto o no. El mundo es la totalidad de los hechos, a saber, de lo equivalente en la realidad a las proposiciones verdaderas.
% 4.La distinción entre el decir y el mostrar
Solo las situaciones que pueden ser plasmadas en imágenes pueden ser afirmadas en proposiciones. Adicionalmente hay mucho que es inexpresable, lo cual no debemos intentar enunciar, sino más bien contemplar sin palabras\footnote{\cite[Cf.~][19]{anscombe1959iwt}: \enquote{There is indeed much that is inexpressible --- which we must not try to state, but mus contemplate without words.}}.

\subsubsection{La filosofía como actividad}

La filosofía es la actividad que tiene como objeto la clarificación lógica de los pensamientos\autocite[\S4.112]{wittgenstein1922tractatuses}. El problema de muchas de las proposiciones y preguntas que se han escrito acerca de asuntos filosóficos no es que sean falsas, sino carentes de significado. Wittgenstein continúa: \blockquote[{\cite[\S4.003]{wittgenstein1922tractatuses}}]{De ahí que no podamos dar respuesta en absoluto a interrogantes de este tipo, sino solo constatar su condición de absurdos. La mayor parte de los interrogantes y proposiciones de los filósofos estriban en nuestra falta de comprensión de nuestra lógica lingüística. (Son del tipo del interrogante acerca de si lo bueno es más o menos idéntico que lo bello). Y no es de extrañar que los más profundos problemas \emph{no} sean problema \emph{alguno}}. Es así que el precipitado de la reflexión filosófica que el Tractatus recoge no pretende componer un cuerpo doctrinal articulado por proposiciones filosóficas, sino más bien ofrecer `elucidaciones' que sirven como etapas escalonadas y transitorias que al ser superadas conducen a ver el mundo correctamente. Este esfuerzo hace de pensamientos opacos e indistintos unos claros y con límites bien definidos\autocite[Cf.~][\S4.112 y \S6.54]{wittgenstein1922tractatuses}. La posibilidad de llegar a una visión clara del mundo es fruto de la posibilidad de lograr aclarar la lógica del lenguaje. El lenguaje, a su vez, está compuesto de la totalidad de las proposiciones, y estas, cuando tienen sentido, representan el pensamiento\autocite[Cf.~][\S4 y \S4.001]{wittgenstein1922tractatuses}. Sin embargo, el mismo lenguaje que puede expresar el pensamiento lo disfraza: \blockquote[{\cite[\S4.002]{wittgenstein1922tractatuses}}]{El lenguaje disfraza el pensamiento. Y de un modo tal, en efecto, que de la forma externa del ropaje no puede deducirse la forma del pensamiento disfrazado; porque la forma externa del ropaje está construida de cara a objetivos totalmente distintos que el de \emph{permitir} reconocer la forma del cuerpo}.

El intento de llegar desde el lenguaje al pensamiento por medio de las proposiciones con significado es el esfuerzo por conocer una imagen de la realidad. El pensamiento es la imagen lógica de los hechos, en él se contiene la posibilidad del estado de las cosas que son pensadas y la totalidad de los pensamientos verdaderos es una imagen del mundo\autocite[Cf.~][\S3 y \S3.001]{wittgenstein1922tractatuses}.

\subsubsection{El pensamiento como representación}

El pensamiento es representación de la realidad por la identidad existente entre la posibilidad de la estructura de una proposición y la posibilidad de la estructura un hecho: \blockquote[{\cite[171]{anscombe2011plato:simplicity}}: \enquote{Objects ---which are simples--- combine into elementary situations. The kind of way they hang together in such a situation is its \emph{Structure}. \emph{Form} is the possibility of the structure. Not all possible structures are actual: one that is actual is an `elementary fact'. We form pictures of facts, of possible facts indeed, but some of them are actual too. A picture consists in \emph{its} elements combining in a particular kind of way. Their doing so presents the objects named by them as combined in just that way. The combination of the elements of the picture ---the presenting combination--- is called \emph{its} structure and its possibility the form of representation of the picture. This `form of representation' is the possibility that things are combined as are the elements of the picture.}; {\cite[Cf.~][\S2.15]{wittgenstein1922tractatuses}}]{Los objetos ---que son simples--- se combinan en situaciones elementales. El modo en el que se sujetan juntos en una situación tal es su \emph{Estructura}. \emph{Forma} es la posibilidad de esa estructura. No todas las estructuras posibles son actuales: una que es actual es un `hecho elemental'. Nosotros formamos imágenes de los hechos, de hechos posibles ciertamente, pero algunos de ellos son actuales también. Una imagen consiste en \emph{sus} elementos combinados en un modo específico. Al estar así presentan a los objetos denominados por ellos como combinados específicamente en ese mismo modo. La combinación de los elementos de la imagen ---la combinación siendo presentada--- se llama su estructura y su posibilidad se llama la forma de representación de la imagen.

Esta `forma de representación' es la posibilidad de que las cosas están combinadas como lo están los elementos de la imagen}.

La representación y los hechos tienen en común la forma lógica: \blockquote[{\cite[\S2.18]{wittgenstein1922tractatuses}}]{Lo que cualquier figura, sea cual fuere su forma, ha de tener en común con la realidad para poder siquiera ---correcta o falsamente--- figurarla, es la forma lógica, esto es, la forma de la realidad}.

\subsubsection{Las proposiciones como proyecciones con polos de verdad-falsedad}

La imagen de la realidad se convierte en proposición en el momento en que nosotros correlacionamos sus elementos con las cosas actuales\footnote{\cite[Cf.~][73]{anscombe1959iwt}: \enquote{a picture (in the ordinary sense) becomes a proposition the moment we correlate its elements with actual things.}}. La condición de posibilidad de entablar dicha correlación es la relación interna entre los elementos de la imagen en una estructura con sentido\footnote{\cite[Cf.~][68]{anscombe1959iwt}: \enquote{only if significant relations hold among the elements of the picture \emph{can} they be correlated with objects outside so as to stand for them}}. De este modo: \blockquote[{\cite[\S5.4733]{wittgenstein1922tractatuses}}]{Frege dice: cualquier proposición formada correctamente debe tener un sentido; y yo digo: cualquier proposición posible está correctamente formada y si carece de sentido ello solo puede deberse a que no hemos dado \emph{significado} a algunas de sus partes integrantes}.

La proposición expresa el pensamiento perceptiblemente por medio de signos. Usamos los signos de las proposiciones como proyecciones del estado de las cosas y las proposiciones son el signo proposicional en su relación proyectiva con el mundo. A la proposición le corresponde todo lo que le corresponde a la proyección, pero no lo que es proyectado, de tal modo, que la proposición no contiene aún su sentido, sino la posibilidad de expresarlo; la forma de su sentido, pero no su contenido\autocite[Cf.~][\S3.1, \S3.11--\S3.13]{wittgenstein1922tractatuses}.

La proposición no `contiene su sentido' porque la correlación la hacemos nosotros, al `pensar su sentido'. Hacemos esto cuando usamos los elementos de la proposición para representar los objetos cuya posible configuración estamos reproduciendo en la disposición de los elementos de la proposición. Esto es lo que significa que la proposición sea llamada una imagen de la realidad\footnote{\cite[cf.~][69]{anscombe1959iwt}: \enquote{The reason why the proposition doesn't `contain its sense' is that the correlations are made by us; we mean the objects by the components of the proposition in `thinking its sense'}}.

Toda proposición-imagen tiene dos acepciones. Puede ser una descripción de la existencia de una configuración de objetos o puede ser una descripción de la no-existencia de una configuración de objetos\footnote{\cite[Cf.~][72]{anscombe1959iwt}: \enquote{Every picture-proposition has two senses, in one of which it is a description of the existence, in the other of the non-existence, of a configuration of objects; and it is that by being a projection.}}. Esta doble acepción es el resultado de que la proposición-imagen puede ser una proyección hecha en sentido positivo o negativo\footnote{\cite[Cf.~][74]{anscombe1959iwt}: \enquote{Thus we can consider the T and F poles of the picture-proposition as giving two senses, positive and negative (as it were, the different methods of projection), in which the picture-proposition can be thought.}}. Esto queda ilustrado en una analogía: \blockquote[{\cite[\S4.463]{wittgenstein1922tractatuses}}]{La proposición, la figura, el modelo, son, en sentido negativo, como un cuerpo sólido que limita la libertad de movimiento de los demás; en sentido positivo, como el espacio limitado por substancia sólida, en el que un cuerpo ocupa un lugar}.

De este modo toda proposición-imagen tiene dos polos; de verdad y de falsedad. Las tautologías y las contradicciones, por su parte, no son imágenes de la realidad ya que no representan ningún posible estado de las cosas. Así continúa la ilustración anterior:
\blockquote[{\cite[\S4.463]{wittgenstein1922tractatuses}}]{La tautología deja a la realidad el espacio lógico entero ---infinito---; la contradicción llena todo el espacio lógico y no deja a la realidad punto alguno. De ahí que ninguna de las dos pueda determinar en modo alguno la realidad}.

La verdad de las proposiciones es posible, de las tautologías es cierta y de las contradicciones imposible. La tautología y la contradicción son los casos límite de la combinación de signos ---específicamente--- su disolución\autocite[Cf.~][\S4.464 y \S4.466]{wittgenstein1922tractatuses}. Las tautologías son proposiciones sin sentido (carecen de polos de verdad y falsedad), su negación son las contradicciones. Los intentos de decir lo que solo puede ser mostrado resultan en esto, en formaciones de palabras que carecen de sentido, es decir, son formaciones que parecen oraciones, cuyos componentes resultan no tener significado en esa forma de oración\footnote{\cite[Cf.~][163]{anscombe1959iwt}: \enquote{attempts to say what is `shewn' produce `\emph{non-sensical}' formations of words---i.e. sentence-like formations whose constituents turn out not to have any meaning in those forms of sentences}}.

\subsubsection{La distinción entre el decir y el mostrar}

La conexión entre las tautologías y aquello que no se puede decir, sino mostrar, es que estas ---siendo proposiciones lógicas sin sentido--- muestran la 'lógica del mundo'\footnote{\cite[Cf.~][163]{anscombe1959iwt}: \enquote{tautologies shew the `logic of the world'. But what they shew is not what they are an attempt to say: for Wittgenstein does not regard them as an attempt to say anything.}}. Esta 'lógica del mundo' o 'de los hechos' es la que más prominentemente aparece en el Tractatus entre las cosas que no pueden ser dichas, sino mostradas. Esta lógica no solo se muestra en las tautologías, sino en todas las proposiciones. Queda exhibida en las proposiciones diciendo aquello que pueden decir.

La forma lógica no puede expresarse desde el lenguaje, pues es la forma del lenguaje mismo, se hace manifiesta en este, no es representativa de los objetos y tampoco puede ser representada por signos, tiene que ser mostrada: \blockquote[{\cite[\S4.0312]{wittgenstein1922tractatuses}}]{La posibilidad de la proposición descansa sobre el principio de la representación de objetos por medio de signos. Mi idea fundamental es que las \enquote{constantes lógicas} no representan nada. Que la \emph{lógica} de los hechos no puede representarse}.

La lógica es, por tanto, trascendental, no en el sentido de que las proposiciones sobre lógica afirmen verdades trascendentales, sino en que todas las proposiciones muestran algo que permea todo lo decible, pero es en sí mismo indecible\footnote{\cite[Cf.~][166 \S2]{anscombe1959iwt}: \enquote{when the \emph{Tractatus} tells us that `Logic is transcendental', it does not mean that the propositions of logic state transcendental truths; it means that they, like all other propositions, shew something that pervades everything sayable and is itself unsayable.}}.

Otra cuestión notoria entre aquello que no puede ser dicho, sino mostrado es la cuestión acerca de la verdad del solipsismo. Los límites del mundo son los límites de la lógica, lo que no podemos pensar, no podemos pensarlo, y por tanto tampoco decirlo. Los límites de mi lenguaje significan los límites de mi mundo\autocite[Cf~.][\S5.6 y \S5.61]{wittgenstein1922tractatuses}. De este modo: \blockquote[{\cite[\S5.62]{wittgenstein1922tractatuses}}]{lo que el solipsismo \emph{entiende} es plenamente correcto, solo que eso no se puede \emph{decir}, sino que se muestra.

Que el mundo es \emph{mi} mundo se muestra en que los límites \emph{del} lenguaje (del lenguaje que solo yo entiendo) significan los límites de \emph{mi} mundo}. Así como la lógica del mundo y la verdad del solipsismo quedan mostradas, también, las verdades éticas y religiosas, aunque no expresables, se manifiestan a sí mismas en la vida. Existe, por tanto lo inexpresable que se muestra a sí mismo, esto es lo místico\autocite[Cf.~][\S6.522]{wittgenstein1922tractatuses}.

De la voluntad como sujeto de la ética no podemos hablar\autocite[Cf.~][\S6.423]{wittgenstein1922tractatuses}. El mundo es independiente de nuestra voluntad ya que no hay conexión lógica entre esta y los hechos. La voluntad y la acción como fenómenos, por tanto, interesan solo a la psicología\footnote{\cite[cf.~][171]{anscombe1959iwt}: \enquote{there is no logical connection between will and world \textelp{} In so far as an event in the world can be described as voluntary, and volition be studied, the will, and therefore action, is `a phenomenon, of interest only to psychology'.}}.

 El significado del mundo tiene que estar fuera del mundo\autocite[Cf.~][\S6.41]{wittgenstein1922tractatuses} y Dios no se revela \emph{en} el mundo\autocite[Cf.~][\S6.432]{wittgenstein1922tractatuses}. Esto se sigue de la teoría de la representación; una proposición y su negación son ambas posibles, cuál es verdad es accidental\footnote{\cite[Cf.~][170]{anscombe1959iwt}: \enquote{This follows from the picture theory; a proposition and its negation are both possible; which one is true is accidental.}}. Si hay un valor que valga la pena para el mundo tiene que estar fuera de lo que es el caso que es; lo que hace que el mundo tenga un valor no-accidental tiene que estar fuera de lo accidental, tiene que estar fuera del mundo\autocite[Cf.~][\S6.41]{wittgenstein1922tractatuses}.

Finalmente, aplicar el límite de lo que puede ser expresado a la actividad filosófica significa que: \blockquote[{\cite[\S6.53]{wittgenstein1922tractatuses}}]{El método correcto de la filosofía seria propiamente este: no decir nada más que lo que se puede decir, o sea, proposiciones de la ciencia natural ---o sea, algo que nada tiene que ver con la filosofía---, y entonces, cuantas veces alguien quisiera decir algo metafísico, probarle que en sus proposiciones no había dado significado a ciertos signos. Este método le resultaría insatisfactorio ---no tendría el sentimiento de que le enseñábamos filosofía---, pero sería el único estrictamente correcto}. La frase usada para describir la obra: \enquote*{de lo que no podemos hablar, de eso hemos de guardar silencio}, pretende expresar tanto una verdad logico-filosófica como un precepto ético. El sinsentido que resulta de tratar de decir lo que solo puede ser mostrado no solo es lógicamente insostenible, sino éticamente indeseable\footnote{\cite[Cf.~][156]{monk1991duty}: \enquote{The famous last sentence of the book ---`Whereof one cannot speak, thereof one must be silent'--- expresses both a logico-philosophical truth and an ethical precept.}}. Wittgenstein explicó esta finalidad ética de su obra en una carta a Ludwig von Ficker de este modo:
\blockquote[{\cite[22--23]{monk2005howto}}: \enquote{the point of the book is ethical. I once wanted to give a few words in the foreword which now are actually not in it, which, however, I'll write to you now because they might be a key for you: I wanted to write that my work consists on two parts: of the one which is here, and of everything which I have \emph{not} written. And precisely this second part is the important one. For the Ethical is delimited from within, as it were, by my book; and I'm convinced that, \emph{strictly} speaking, it can ONLY be delimited in this way. In brief, I think: All of that whcih \emph{many} are \emph{babbling} today, I have defined in my book by remaning silent about it.}]{el punto del libro es ético. Hubo un tiempo en que quise ofrecer en el prefacio algunas palabras que ya no están ahí, estas, sin embargo, quiero escribírtelas ahora porque pueden ser clave para ti: quise escribir que mi trabajo consiste en dos partes: en la que está aquí, y en todo lo que \emph{no} he escrito. Y precisamente esta segunda parte es la importante. Pues lo Ético es delimitado desde dentro, por así decirlo, por mi libro; y estoy convencido de que, \emph{estrictamente} hablando, este SOLO puede ser delimitado de este modo. En resumen, pienso que: todo de lo que \emph{muchos} están \emph{mascullando} hoy en día, lo he definido en mi libro al mantenerme en silencio sobre eso}.

\subsection{\emph{Investigaciones Filosóficas} y el nuevo método de Wittgenstein}

Anscombe conoció a Wittgenstein en la segunda etapa de su pensamiento, y trabajó con él para traducir \emph{Investigaciones Filosóficas}, así que hemos de atribuir a esta etapa tardía la mayor influencia en el pensamiento de Elizabeth. Sin embargo, como vimos en el apartado anterior, \emph{An Introduction to Wittgenstein's Tractatus}, constituye una de las discusiones más amplias del pensamiento de Wittgenstein en la obra de Anscombe. El mismo Wittgenstein reiteró que su pensamiento tardío solo puede entenderse a la luz del \emph{Tractatus}, sin embargo esto no terminaría de explicar el interés de Anscombe en esa obra. Quizás es correcto decir que el \emph{Tractatus}, con su énfasis en el tema de la verdad, no dejó de ser una reflexión con mérito para Elizabeth como complemento de la atención que presta \emph{Investigaciones Filosóficas} al tema del sentido\autocite[Cf.~][191--193]{teichmann2008ans}. En este apartado veremos algunos aspectos de las discusiones de Wittgenstein en esta segunda obra. La descripción será más general que la del \emph{Tractatus} ya que el análisis de los artículos de Anscombe en el capítulo siguiente nos dará la oportunidad de profundizar en algunos elementos que no se tratarán aquí.

Para ilustrar el cambio que hay en la concepción del lenguaje entre el \emph{Tractatus} e \emph{Investigaciones Filosóficas} podemos recurrir a algunas reflexiones de Wittgenstein sobre los fundamentos de las matemáticas hechas entre 1937 y 1938. Él se plantea la siguiente pregunta: \enquote*{¿Cómo sé que al calcular la serie $+2$ debo escribir `$20004$, $20006$' y no `$20004$, $20008$'?} La pregunta tiene que ver con el modo en el que actuamos según una regla. Al calcular esta serie se ha ofrecido $+2$ como norma para el cálculo. Ahora la pregunta es cómo se sabe qué hacer con ese conocimiento previo cuando llega el momento de ponerlo en acto. Si se ha comprendido la guía inicial se tendrá certeza sobre qué hacer después de $20004$, y esta certeza no implica que $20006$ haya quedado determinado de antemano, sino en que ante cualquier número ofrecido se tiene la capacidad de ofrecer el siguiente. Entonces continúa: \blockquote[Esta larga cita se ha tomado de la traducción al inglés realizada por Anscombe: {\cite[I, \S4]{wittgenstein1956remmath}}; una traducción española puede encontrarse en: {\cite[17--18]{wittgenstein1956remmathes}}]{``¿Pero entonces en qué consiste la peculiar inexorabilidad de las matemáticas?''\,---\,¿No será acaso la inexorabilidad con la que dos sigue a uno y tres a dos un buen ejemplo?\,---\,Pero presuntamente esto significa: se sigue así en la \emph{serie de números cardinales}; pues en una serie distinta se seguiría de un modo distinto. Pero ¿acaso esta serie no está definida precisamente por esta secuencia?\,---\,``¿Hay que suponer que esto significa que cualquier modo en el que una persona cuente es igualmente correcto, y que cualquiera puede contar en el orden que quiera?''\,---\,Probablemente no lo llamaríamos `contar' si todo el mundo dijera los números uno después de otro \emph{de cualquier manera}; pero por supuesto esto no se trata simplemente de un problema sobre el nombre que se usa. Pues lo que llamamos `contar' es una parte importante de las actividades de nuestras vidas. Contar y calcular no son ---por ejemplo--- un simple pasatiempo. Contar (y eso significa: contar \emph{así}) es una técnica que es empleada diariamente en las operaciones más variadas de nuestras vidas. Y por eso es que aprendemos a contar como lo hacemos: con prácticas interminables, con despiadada exactitud; por eso es que es inexorablemente insistido que hemos de decir ``dos'' después de ``uno'', ``tres'' después de ``dos'' y así sucesivamente.\,---\,``Pero entonces este contar es sólo un \emph{uso}; ¿acaso no hay alguna verdad que se corresponda con esta secuencia?'' La \emph{verdad} es que contar ha demostrado que paga.\,---\,``Entonces quieres decir que `ser verdad' significa: ser utilizable (o útil)?''\,---\,No, no eso; pero que no puede ser dicho de la serie de números naturales\,---\,y tampoco de nuestro lenguaje\,---\,que es verdad, pero: que es utilizable, y, sobre todo que \emph{se usa de hecho}}. La discusión de \emph{Investigaciones Filosóficas} comienza con una cita de \emph{Confesiones} I,8 donde se encuentra una descripción de una imagen de la `esencia del lenguaje humano' que Wittgenstein considera que pertenece a la tradición que culminó en la teoría del \emph{Tractatus}. Allí la necesidad que le atribuimos a ciertas verdades y nuestra capacidad de reconocer esta necesidad a priori se explicó por la forma lógica común al pensamiento y la realidad y que queda expresada en el lenguaje. Sin embargo, esta tradición se equivocó al cuestionarse qué hace a estas verdades necesarias. La investigación adecuada parte de la pregunta sobre qué es que una proposición \emph{sea} necesaria y la respuesta se encuentra examinando y describiendo el papel que juegan estas proposiciones en las transacciones que hacemos con nuestro lenguaje\footnote{\cite[Cf.~][242--243]{bakerhacker2014rules}: \enquote{Wittgenstein, when composing the early draft of the \emph{Investigations} in 1936/7, approached the task of mapping out this terrain from a unique vantage point --- namely his elucidation of internal relations by reference ot human practices of using signs. His examination of the concept of following a rule provides the background for clarifying the character of mathemathical propositions, of what he called grammatical propositions and hence too of putative metaphysical propositions, and of the propositions of logic. He gave a detailed and comprehensive account of their peculiar status, an account which explains both why we conceive of them as necessary truths and what sense can be made of that conception. The questions of what makes them necessary (what is the source of their necessity) and how a priori knowledge of them is possible (how do we recognize them) lead us astray before we have begun. The prior question is: what is it for a proposition to \emph{be} a `necessary proposition', i.e. to be a proposition of mathematics, to be a logical proposition, or to be what Wittgenstein called a grammatical proposition? If this is answered by examining and properly describing the roles of such propositions in our linguistic transactions, the traditional questions can be resolved or dissolved. Wittgenstein's account is as bold as it is original.}}.

Con la pregunta sobre cómo continuar la serie, Wittgenstein está cuestionando en qué consiste la necesidad matemática que rige la secuencia. Similarmente habla de la necesidad en relación con la gramática. Tras cuestionarse sobre el modo en que calculamos la serie, añade la observación: \enquote{la pregunta ``¿cómo sé que este color es `rojo'?'' es similar.} La cuestión planteada no solo tiene que ver con el modo en el que vamos según una serie, sino con las operaciones que hacemos con las palabras. También con las palabras hay una comprensión inicial de su uso que luego se aplica en cada caso. ¿Cómo sé que en esta ocasión estoy empleando una expresión según la regla que es su uso? Wittgenstein dirá que hay una relación entre necesidad, gramática y uso en la actividad humana que constituyen lo que podríamos considerar la esencia de las palabras.

Esta manera de analizar en lenguaje tiene como consecuencia que no podemos pensar en los conceptos como entidades privadas en nuestro pensamiento. En \emph{Investigaciones Filosóficas} \S380 encontramos: \blockquote[{\cite[\S380]{wittgenstein1953phiinv}}: \enquote{How do I recognize that this is red?\,---\,``I see that it is \emph{this}; and then I know that that is what is called.'' This?\,---\,What?! What kind of answer to this question makes sense? (You keep on steering towards an inner ostensive explanation.) I could not apply any rules to a \emph{private} transition from what is seen to words. Here the rules really would hang in the air; for the institution of their application is lacking.}]{¿Cómo reconozco que esto es rojo?\,---\,``Veo que es \emph{esto}; y entonces sé que eso es lo que esto es llamado'' ¿Esto?\,---\,¡¿Qué?! ¿Qué tipo de respuesta a esta pregunta tiene sentido? (Sigues girando hacia una explicación ostensiva interna.) No podría aplicar ninguna regla a una transición \emph{privada} desde lo que es visto a las palabras. Aquí las reglas realmente quedarían suspendidas en el aire; pues la institución para su aplicación está ausente}.

Y añade en \S381: \blockquote[{\cite[\S380]{wittgenstein1953phiinv}}: \enquote{How do I recognize that this colour is red?\,---\,One answer would be: ``I have learnt English.''}]{¿Cómo reconozco que este color es rojo?\,---\,Una respuesta sería: ``He aprendido [castellano]''}. Ir según una regla es ir según una costumbre, un uso, una institución; \blockquote[{\cite[\S199]{wittgenstein1953phiinv}}: \enquote{To understand a sentence means to understand a language. To understand a language means to have mastered a technique.}]{Entender una oración significa entender un lenguaje, entender un lenguaje significa dominar una técnica.} La gramática de la expresión `seguir una regla' supone la existencia de una práctica, una regularidad, un comportamiento normativo. Sólo cuando esta red de comportamientos está en juego se puede hablar de que existe una regla\footnote{\cite[Cf.~][p.~14]{bakerhacker2009understanding}: \enquote{The internal relation is forged by the existence of a practice, a regularity in applying the rule, and the normative behaviour (of justification, criticism, correction of mistakes, etc.) that surrounds the practice. Only when such complex forms of behaviour are in play does it make sense to speak of \emph{there being} a rule at all}}. No es posible que haya una sola persona que en una sola ocasión `siguió una regla', esta consideración no es correspondiente con la gramática de la expresión\footnote{\cite[Cf.~][\S199]{wittgenstein1953phiinv}: \enquote{Is what we call ``following a rule'' something that it would be possible for only \emph{one} person, only \emph{once} in a lifetime, to do?}}.

Cuando Elizabeth Anscombe participó de estas discusiones en las clases con Wittgenstein encontró una perspectiva liberadora en la noción de que el significado de las palabras queda expresado en definitiva en el uso que hacemos de ellas: \blockquote[{\cite[viii]{anscombe1981metaphysics}}: \enquote{At one point in these classes Wittgenstein was discussing the interpretation of the sign-post, and it burst upon me that the way you go by it is the final interpretation.}]{En cierto punto Wittgenstein estaba discutiendo en sus clases la interpretación del letrero (sign-post), y estalló en mi que el modo en que vas según éste es la interpretación final}. Un letrero es una expresión de una regla ante la que hemos sido entrenados a reaccionar de un modo particular. Pensar que se está siguiendo una regla no es seguir una regla, y por eso no es posible seguir una regla `privadamente'\footnote{\cite[Cf.~][\S202]{wittgenstein1953phiinv}: \enquote{That's why `following a rule' is a practice. And to \emph{think} one is following a rule is not to follow a rule. And that's why it's not possible to follow a rule `privately'; otherwise, thinking one was following a rule would be the same thing as following it.}}. La interpretación definitiva de una expresión de una regla es cómo se actúa ante ella. Durante sus estudios en Oxford, Anscombe había rechazado con fuerza un realismo representativo lockeano que insistía que los colores como ella los veía no son parte del mundo externo. Como reacción contraria tendía a identificar estas sensaciones con \emph{esto} (this), como si `azul' o `amarillo' fueran artículos que `están ahí'. Esta noción también le parecía equivocada, pero no lograba librarse de ella: \blockquote[{\cite[viii]{anscombe1981metaphysics}}: \enquote{At another \textins{point} I came out with ``But I still want to say: Blue is there.'' Older hands smiled or laughed but Wittgenstein checked them by taking it seriously, saying ``Let me think what medicine you need\ldots Suppose that we had the word `painy' as a word for the property of some surfaces.'' The `medicine' was effective \textelp{} If ``painy'' were a possible secondary quality word, then wouldn't just the same motive drive me to say: ``Painy is there'' as drove me to say ``Blue is there''?}]{En otra \textins{ocasión} salí con: ``Pero todavía quiero decir: Azul esta ahí''. Manos más veteranas sonrieron o rieron, pero Wittgenstein las detuvo tomándolo en serio, diciendo: ``Déjame pensar qué medicina necesitas\ldots'' ``Supón que tenemos la palabra `\emph{painy}', como una palabra para la propiedad de ciertas superficies''. La `medicina' fue efectiva \textelp{} Si ``\emph{painy}'' fuera una palabra posible para una cualidad secundaria, ¿no podría el mismo motivo conducirme a decir: ``\emph{Painy} está ahí'' que lo que me condujo a decir ``Azul está ahí''?} La solución a la dificultad de Anscombe no consiste tampoco en identificar `azul' o `painy' con `esta sensación', sino precisamente en desligar estos conceptos tanto de `algo que está ahí', como de `esta sensación que tengo', el significado se encuentra en su uso: \blockquote[{\cite[114]{anscombe1981parmenides:qli}}: \enquote{``You learned the \emph{concept} pain when you learned language.'' That is, it is not experiencing pain that gives you the meaning of the word ``pain''. How could an experience dictate the grammar of a word? \textelp{} doesn't it make certain demands on the grammar, if the word is to be the word for \emph{that} experience?}]{``Aprendimos el \emph{concepto} dolor cuando aprendimos el lenguaje.'' Esto es, no ha sido experimentar el dolor lo que nos ha dado el significado de la palabra ``dolor''. ¿Cómo podría una experiencia dictar la gramática de una palabra? \textelp{} ¿acaso no implica ciertas exigencias a la gramática, si la palabra tiene que ser la palabra de \emph{esa} experiencia?}

El cambio ocurrido en Anscombe al encontrarse con el método propuesto por Wittgenstein es representativo del problema de la filosofía que él quiso resolver: \blockquote[{\cite[Cf.~][213]{diamond2004crisscross}}: \enquote{Before the `medicine', Anscombe's problem is one of philosphy's Big Questions. It is a form of the question how our thought is able to connect with reality. She is aware of, has in her mind, \emph{this}, the blue; is it or is it not \emph{there}, in the world?}]{Antes de la `medicina', el problema de Anscombe es una de las Grandes Preguntas de la filosofía. Es una forma de la pregunta sobre cómo nuestro pensamiento tiene la capacidad de conectar con la realidad. Ella está consciente de, tiene en su mente, \emph{esto}, el azul; ¿está o no está \emph{ahí}, en el mundo?} La respuesta del \emph{Tractatus} pensó en esta como una conexión metafísica presente en el orden lógico que sostiene todo lenguaje posible. El trabajo del filósofo según esta concepción consiste en analizar las expresiones para sacar al descubierto el orden lógico que está debajo del lenguaje ordinario y que es la forma de la realidad. Ahora la ruta es distinta, en \emph{Investigaciones Filosóficas} exclama: \blockquote[{\cite[\S107]{wittgenstein1953phiinv}}: \enquote{The more closely we examine actual language, the greater becomes the conflict between it and our requirement. (For the crystalline purity of logic was, of course, not something I had \emph{discovered}: it was a requirement.) The conflict becomes intolerable; the requirement is in danger of becoming vacuous.\,---\,We have got on to slippery ice where there is no friction, and so, in a certain sense, the conditions are ideal; but also, just because of that, we are unable to walk. We want to walk: so we need \emph{friction}. Back to the rough ground!}]{Cuanto más de cerca examinamos el lenguaje actual, más crece el conflicto entre éste y nuestro requisito. (Pues la pureza cristalina de la lógica no era, por supuesto, algo que yo hubiera \emph{descubierto}: era un requisito.) El conflicto se hace intolerable; el requisito llega ahora a estar en peligro de tornarse vacuo.\,---\,Nos hemos situado en hielo resbaladizo donde no hay fricción, y así, en cierto sentido, las condiciones son ideales; pero también, justo por eso, no somos capaces de caminar. Queremos caminar: así que necesitamos \emph{fricción}. ¡De vuelta al terreno escarpado!} El análisis del lenguaje tiene que considerarlo integrado a la actividad de la vida humana. Ahí es donde el lenguaje está funcionando, está vivo, tiene `fricción'. En ese sentido, todo lo que necesitamos para entender el lenguaje está ante nosotros, a la vista, es nuestra manera de vivir\footnote{\cite[Cf.~][48]{mcginn2013guide}: \enquote{Instead of approaching language as a system of signs with meaning, we are prompted to imagine it in situ, embedded in the lives of those who speak it. The tendency to isolate language, or abstract it from the context in which it ordinarily lives, is connected with our desire to say what the essence of language is, and with our urge to explain how these mere signs (mere marks) acquire their extraordinary power to mean or represent something. Wittgenstein’s aim is to show us that in this act of abstraction we turn our backs on everything that is essential to language’s signifying in the way that it does; it is our act of abstracting language from its employment within our ordinary lives that turns it into something dead, whose ability to represent now cries out for explanation. Thus, the sense of a need to explain how language (conceived as a system of signs) has the magical power to represent the world is connected with our failure to look at language where it is actually functioning. Wittgenstein does not set out to satisfy our sense of a need for a theory of representation (a theory that explains how the dead sign acquires meaning), but to dispel this sense of a need through getting us to look at language where it is actually doing work, and where we can see its essence fully displayed. In directing us, through the concept of a language-game, to ‘the spatial and temporal phenomenon of language, not [to] some non-spatial, a temporal non-entity’ (PI §108), Wittgenstein hopes gradually to bring us to see that ‘nothing extraordinary is involved’ (PI §94), that everything that we need to understand the essence of language ‘already lies open to view’ (PI §126).}}.
