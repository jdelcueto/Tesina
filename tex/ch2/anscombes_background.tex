\subsection{El método de Wittgenstein}

Anscombe escuchó alguna vez a Wittgenstein explicar en sus clases cómo pretendía
ofrecer ejemplos de `ejercicios para cinco dedos', como los que se emplean para
el piano. Eran ejercicios para pensar. <<Soy como un maestro de piano>>
--decía-- <<intento enseñar un estilo de pensar, una técnica, no una materia>>.
Lo que se escuchaba en sus lecciones, sin embargo, no eran piezas musicales,
sino más bien las prácticas donde el pianista afina los movimientos que van
dirigidos a construir su
concierto.\autocite[cf.~][p.~357]{KlaggeNordman2003pubnpriv}

En cierta ocasión, Wittgenstein recibió a Anscombe con una pregunta: <<¿Por qué
la gente dice que era natural pensar que el sol giraba alrededor de la tierra en
lugar de que la tierra rotaba en su eje?>> Elizabeth contestó: <<Supongo que
porque se veía como si el sol girara alrededor de la tierra.>> <<Bueno\ldots>>,
añadió Wittgenstein, <<¿cómo se hubiera visto si se hubiera \emph{visto} como si
la tierra rotara en su propio eje?>> Anscombe reaccionó extendiendo las manos
delante de ella con las palmas hacia arriba y, levantándolas desde sus rodillas
con un movimiento circular, se inclinó hacia atrás asumiendo una expresión de
mareo. <<¡Exactamente!>> exclamó
Wittgenstein.\autocite[cf.~][p.~151]{anscombe1959iwt}

Anscombe se percató del problema; la pregunta de Wittgenstein había puesto en
evidencia que hasta aquél momento no había ofrecido ningún significado relevante
para su expresión \emph{``se veía como si''} en su respuesta \emph{``se veía
  como si el sol girara alrededor de la tierra''}. En ocasiones como ésta
Elizabeth descubría cómo no podía ofrecer ningún significado que no fuera
respaldado por una concepción ingenua, y ésta podía ser destruida fácilmente por
una pregunta. \citalitinterlin{<<La concepción ingenua es realmente descuido en
  el pensamiento, pero puede necesitar el poder de un Copérnico para
  cuestionarla efectivamente.>>\autocite[p.~151]{anscombe1959iwt}}

¿Qué tipo de problema es éste? ¿Qué falta cuando una expresión carece de
significado?

\subsection{El arte de hacer filosofía}

\ifdraft{\subsubsection{Vida salvaje luchando por emerger abiertamente}}{}
Wittgenstein pensaba que
% Within all great art there is a WILD animal: tamed.
\citalitinterlin{ dentro de todo buen arte hay un animal salvaje
  domado}\autocite[p.~43e]{wittgenstein1998cnv}. Su talante artístico, sin
embargo, no manifestaba esta primitiva vitalidad; o como él mismo decía:
% In my artistic activities I have merely good manners
\citalitinterlin{ en mis actividades artísticas tengo meramente buenos
  modales.}\autocite[p.~29e]{wittgenstein1998cnv}

Ejemplo de estos ``buenos modales'' fue el diseño que realizó para la casa de su
hermana Margaret en Viena, terminada en 1928.
% my house for Gretl is the product of a sensitive ear, good manners, the
% expression of great understanding... wild life striving to erupt in the open
% is lacking... health is lacking (Kierkergaard)
Trabajó como arquitecto de la casa con exhaustiva minuciosidad y el producto
manifestaba gran entendimiento, ``buen oido'', pero le escaseaba ``salud'',
pensaba él.\autocite[p.~43e]{wittgenstein1998cnv}

% Even in music... feeling, he showed above all great understanding, rather than
% manifesting wild life... When he played music with others... his interest was
% in getting it right... When he played, he was not expressing himself... but
% the thoughts... of others. He was probably right to regard himself not as
% creative but as reproductive ...It was only in philosophy that his creativity
% could really be awakened. Only then, as Russell had long ago noticed, does one
% see in him 'wild life striving to erupt in the open''

También en la música, arte por la que tenía la mayor afición, era llamativa su
recia exactitud. Cuando tocaba con otros ponía su mayor interés en lograr una
expresión exacta y correcta, recreando música y pensamientos ajenos, más que
expresándose a sí mismo. Perseguía reproducir más que
crear.\autocite[loc.˜]{monk1991duty}

Esta fuerza creativa ausente en su rigurosa actitud hacia la actividad artística
estallaba, sin embargo, en su actividad filosófica. Aquella cualidad que él
encontraba característica del buen arte, esa ``vida salvaje luchando por emerger
abiertamente'',\autocite[cf.˜][loc.˜]{monk1991duty} quedaba expresada en su
quehacer filosífico.

\ifdraft{\subsubsection{Filosofía emergente}}{}
La filosofía nació así en Ludwig. Como una fuerza violenta. Se hallaba
estudiando ingeniería en Manchester y se interesó por los fundamentos de las
matemáticas. Este interés no tardó en convertirse en el deseo de elaborar un
trabajo filosófico. Su hermana Hermine le describe así en sus memorias de la
familia Wittgenstein
\footnote{Hermine Wittgenstein escribió la historia y memorias de su familia
  ``Familienerinnerungen'' durante la segunda Guerra Mundial.}:

\citalitlar{Fue repentinamente agarrado por la filosofía ---es decir, por la
  reflexión en problemas filosóficos--- tan violentamente y tan en contra de su
  voluntad que sufrió severamente por la doble y conflictiva llamada interior y
  se veía a sí mismo como roto en dos. Una de muchas transformaciones por las
  que pasaría en su vida había venido sobre él y le estremeció hasta lo más
  profundo. Estaba concentrado en escribir un trabajo filosófico y finalmente
  determinó mostrar el plan de su obra al Profesor Frege en Jena, quien había
  discutido preguntas similares. [\ldots] Frege alentó a Ludwig en su búsqueda
  filosófica y le aconsejó que fuera a Cambridge como alumno del Profesor
  Russell, cosa que Ludwig ciertamente hizo.\autocite[p. 73]{mcguinness}}

La investigación filosófica comenzada en aquel momento se convirtió en la tarea
del resto de su vida. Sus incipientes ideas filosóficas pasarían por diversas
transformaciones, pero expresaban ya desde el principio una preocupación por los
problemas fundamentales. Por las reglas del juego, se podría decir.

\ifdraft{\subsubsection{La Naturaleza de los problemas Filosóficos}}{}
Entre esas cuestiones fundamentales se halla una de las constantes importantes
en su pensamiento. Ésta es su definición de la naturaleza de los problemas
filosóficos. Para Wittgenstein las cuestiones de la filosofía no son
problemáticas por ser erróneas, sino por no tener
significado.\autocite[cf.~][4.003]{wittgenstein1922tractatus}

Una proposición sin significado que no es puesta al descubierto como tal atrapa
al filósofo dentro de una confusión del lenguaje que no le permite acceder a la
realidad. Salir de la confusión no consiste en refutar una doctrina y plantear
una teoría alternativa, sino en examinar las operaciones hechas con las palabras
para llegar a manejar una visión clara del empleo de nuestras expresiones. La
filosofía no es un cuerpo doctrinal, sino una
actividad\autocite[cf.~][4.112]{wittgenstein1922tractatus}y una
terapia\autocite[cf.~][\S133]{wittgenstein1953phiinv}.

La actitud terapéutica adoptada por Wittgenstein en su atención de las
confusiones filosóficas fue su respuesta más definitiva a la naturaleza de estos
problemas. Para ello halló los más eficaces remedios en sus investigaciones
sobre el significado y el sentido del lenguaje.

Ordinariamente tomamos parte en esta actividad humana que es el lenguaje.
Jugamos el juego del lenguaje. ---¿Jugarlo es entenderlo?--- A la vista de
Wittgenstein saltaban extraños problemas sobre las reglas de este juego;
entonces no podía evitar escudriñarlas al
detalle.\autocite[cf.~][loc.7099]{monk1991duty} En este análisis del lenguaje está la
raíz de sus ideas sobre el sentido, el significado y la verdad.

Durante su vida sostuvo dos grandes descripciones del significado. Originalmente
describió el lenguaje como una imagen que representa el posible estado de las
cosas en el mundo. En una segunda etapa se distanció de esta analogía para
describir al lenguaje como una herramienta cuyo significado consiste en la suma
de las múltiples semejanzas familiares que aparecen en los distintos usos para
los cuales el lenguaje es empleado en la actividad humana. Dentro de la primera
descripción una expresión sin significado es una cuyos elementos no componen una
representación del posible estado de las cosas. Dentro de la segunda descripción
una expresión sin significado resulta del empleo de una expresión propia de un
``juego del lenguaje'' fuera de su contexto.

\subsection{Dos Cortes en la Filosofía}
 Estas dos etapas del pensamiento de Wittgenstein son representadas por dos
 importantes tratados. El \emph{'Tractatus Logico\=/Philosophicus'}, publicado en
 1921, recoge sus esfuerzos por elaborar un gran tratado filosófico comenzados en
 1911 y culminados durante la Primera Guerra Mundial. El segundo,
 \emph{'Philosophische Untersuchungen'}, o \emph{'Investigaciones Filosóficas'},
 traducido por Anscombe y publicado posthumamente en 1953, fue elaborado a partir
 de múltiples manuscritos desarrollados por Wittgenstein desde su regreso a
 Cambridge en 1929 hasta su muerte en 1951.

 \citalitinterlin{Wittgenstein es extraordinario entre los filósofos por haber
   generado dos épocas, o cortes\footnote{Anscombe toma el termino 'corte' de
     Boguslaw Wolniewicz, filósofo polaco y amigo.}, en la historia de la
   filosofía.}\autocite[p.~181]{anscombe2011plato:twocuts}
 Con estas palabras Anscombe comenzaría su discurso inaugural para el Sexto
 Simposio Internacional de Wittgenstein unos treinta años después de la
 publicación de las \emph{'Investigaciones Filosóficas'}. Y explica:
 \citalitinterlin{un filósofo hace un corte si genera un cambio en el modo en que
   la filosofía es hecha: la filosofía tras el corte no puede ser la misma de
   antes.}\autocite[p.~181]{anscombe2011plato:twocuts}

 Estos cambios de época generados por la influencia de Wittgenstein vinieron
 caracterizados por el esfuerzo de comprender cada libro tras su publicación,
 tarea complicada en ambos casos por la dificultad intrínseca de los tratados,
 ofuscada a su vez por los prejuicios filosóficos proyectados a cada obra por sus
 lectores. La presunción, por ejemplo, de que \emph{'Investigaciones
   Filosóficas'} presenta una teoría del lenguaje ---quizás sobre cómo los
 sonidos se tornan en discursos significativos--- nos dejaría situados lejos de
 las preguntas que genuinamente ocupan a
 Wittgenstein.\autocite[cf.~][p.~183]{anscombe2011plato:twocuts} Ahora bien, la comprensión
 adecuada de su pensamiento y método trae consigo, a juicio de Anscombe, cierto
 efecto curativo.

\ifdraft{\subsubsection{Ver el mundo claramente}}{}
Quedar 'curados' es quedar liberados de la trampa de ciertas inclinaciones que
impiden llegar a concepciones verdaderas. El trabajo de Wittgenstein busca tener
este efecto en la filosofía. ¿Lo logra?

Elizabeth analiza uno de estos esfuerzos. Es una aflicción extendida entre los
filósofos la excesiva dependencia en explicaciones o conexiones necesarias. ¿Han
podido quedar curados los que han estudiado a Wittgenstein? Y añade:
\citalitlar{La filosofía profesional es en gran medida una gran fábrica para la
  manufactura de necesidades---sólo las necesidades nos dan paz mental. No es de
  extrañarse que Wittgenstein despierte cierto odio entre nosotros. Amenaza
  privarnos de nuestro empleo en la fábrica.\autocite[p~.184]{anscombe2011plato:twocuts}}

La dependencia en estas explicaciones que \emph{`deben de ser'} para justificar
nuestras proposiciones nos impide tener una concepción clara del panorama de la
realidad. Anscombe lo ilustra de este modo:
\citalitlar{La descripción detallada de la distribución de manchas de color en
  un canvas no nos revela la imagen que está en él, sin embargo, si dices:
  ``Pero la imagen es \emph{también}. \emph{¿En qué consiste?} \emph{debe de}
  haber ahí algo más además de pintura en un canvas''--estarías embarcandote en
  una busqueda ilusoria. El vasto número de cosas que conocemos y hacemos y que
  indagamos son como la imagen en el canvas. Las realidades acerca de nuestro
  conocer, nuestro hacer y nuestro indagar son enormemente interesantes; pero
  necesidades de tipo absolutamente \emph{a priori} no pueden ser encontradas
  para justificar nuestras aserciones.\autocite[p.~185]{anscombe2011plato:twocuts}}

En contraste con este uso engañoso de la necesidad hay un uso inocuo de ese
\emph{`deber de'} que ocurre en regiones más especializadas. Un ejemplo
notable es el modo en el que hacemos cuentas en una serie, o el modo en el que
calculamos el valor de una variable $\mathcal{Y}$ dado un cierto valor para
$\mathcal{X}$ en una fórmula. Podríamos decir que la serie está determinada ya
de antemano por la fórmula, al calcularla sólo ponemos en tinta, por así
decirlo, la parte de la serie que estamos computando. Aquí no estamos
exactamente manufacturando una necesidad, sino más bien
\citalitinterlin{tratando de formular el ideal de una necesidad que está siendo
  imitada por los cálculos cuando son de resultados que son `determinados', en
  ese sentido inofensivo de necesidad \autocite[p.~185]{anscombe2011plato:twocuts}}.

Pues bien, para Wittgenstein la pregunta sobre la manera adecuada de continuar
una serie es la misma pregunta sobre cómo usar la palabra `rojo'. Así como la
serie tiene una cierta determinación por su formula, la palabra tiene una cierta
determinación por su uso. En este sentido, conocer el significado de una palabra
consiste en comprender ese \emph{`deber de'} que determina su futura aplicación.

Este camino en la busqueda del significado de las proposiciones puede ser
ocasión de otra inclinación:
\citalitinterlin{Aquí no estamos tan tentados de inventar o manufacturar
  necesidades, sino de descansar conformes con las que creemos haber
  comprendido.\autocite[p.~185]{anscombe2011plato:twocuts}}

Esta podría ser nuestra actitud respecto de nuestro uso de las proposiciones
hasta que alguien nos interrumpe con una pregunta sobre la necesidad de estar en
lo correcto cuando usamos una palabra de cierto modo. Esta pregunta sería
esceptica sólo para aquel que asumiera que sus presunciones son
irrefragablemente correctas y la base del significado y la
verdad.\autocite[cfr.~][p.~186]{anscombe2011plato:twocuts}

El impacto de Wittgenstein en la filosofía es para Anscombe una ruta que permite
llegar a concepciones verdaderas. Nos permite ver la pintura con claridad.
Siguiendo la anterior ilustración:

\citalitlar{Es un impedimento para llegar a mirar la imagen, si estás aferrado a
  la convicción de que debes una de dos; extraer la imagen desde la descripción
  del color de cada mancha de pintura en una fina cuadrícula extendida sobre
  esta, o que debes tener una teoría de lo que la imagen es aparte de lo que esa
  descripción describe. Si renuncias a ambas inclinaciones podrás llegar a mirar
  a la pintura y haciéndolo podrías encontrarte lleno de asombro. O, como
  Wittgenstein una vez lo dijera, puedes encontrarte a tí mismo `caminando en
  una montaña de maravillas'}

Según Anscombe el método general adecuado de discutir los problemas filosóficos
propuesto por Wittgenstein consiste en mostrar que la persona no ha provisto
significado (o referencia) para ciertos signos en sus expresiones.\autocite[cf.
p. 151]{anscombe1959iwt} Creía que el camino que lleva a formular estos problemas está
frecuentemente trazado por la mala comprensión de la lógica de nuestro lenguaje.

Cada obra de Wittgenstein representa su esfuerzo de superar estas confusiones
y propone un método para remediarlas. Su primera propuesta plantea que el modo
de aclarar las confusiones de los problemas filosóficos consiste en
identificar en el lenguaje el límite de lo que expresa pensamiento; lo que
queda al otro lado de esta frontera sería simplemente sinsentido. En otras
palabras: \citalitinterlin{
  % What can be said at all
  Lo que puede ser dicho en absoluto puede ser dicho claramente; y de lo que uno
  no puede hablar, de eso, uno debe guardar silencio}.
\autocite[prefacio]{wittgenstein1922tractatus}

Con esta expresión Wittgenstein resumió el sentido del \emph{`Tractatus
Logico\=/Philosophicus'}.

\subsection{Las elucidaciones del Tractatus}
% Este párrafo resume los cuatro puntos del Tractatus que se desglosarán en los
% próximos párrafos
Desde las proposiciones principales del Tractatus queda claro que el tema
central del libro es la conexión entre el lenguaje, o el pensamiento, y la
realidad.
% 1.Filosofía como actividad
En este nexo es donde la actividad filosófica ha de buscar esclarecer el
pensamiento.
% 2.El pensamiento como representación
La tesis básica sobre esta relación consiste en que las proposiciones, o su
equivalente en la mente, son imágenes de los hechos.
% 3.Las proposiciones como proyecciones con polos de verdad-falsedad
La proposición es la misma imagen tanto si es cierta como si es falsa, es decir,
es la misma imagen sin importar que lo que se corresponde a ésta es el caso que
es cierto o no. El mundo es la totalidad de los hechos, a saber, de lo
equivalente en la realidad a las proposiciones verdaderas.
% 4.La distinción entre el decir y el mostrar
Sólo las situaciones que pueden ser plasmadas en imágenes pueden ser afirmadas
en proposiciones. Adicionalmente hay mucho que es inexpresable, lo cual no
debemos intentar enunciar, sino más bien contemplar sin palabras.\autocite[cf.
p.19]{anscombe1959iwt}

\subsubsection{La filosofía como actividad}

La filosofía es la actividad que tiene como objeto la clarificación lógica
de los pensamientos.\autocite[4.112 p. 52]{wittgenstein1922tractatus} El problema de muchas de
las proposiciones y preguntas que se han escrito acerca de asuntos filosóficos
no es que sean falsas, sino carentes de significado. Wittgenstein continúa:
\citalitlar{4.003~En consecuencia no podemos dar respuesta a preguntas de este
    tipo, sino exponer su falta de sentido. Muchas cuestiones y proposiciones de
    los filósofos resultan del hecho de que no entendemos la lógica de nuestro
    lenguaje. (Son del mismo genero que la pregunta sobre si lo Bueno es más o
    menos idéntico a lo Bello). Y así no hay que sorprenderse ante el hecho de
    que los problemas más profundos realmente no son problemas.\autocite[4.003
    p. 45]{wittgenstein1922tractatus}}

Es así que el precipitado de la reflexión filosófica que el Tractatus recoge no
pretende componer un cuerpo doctrinal articulado por proposiciones filosóficas,
sino más bien ofrecer `elucidaciones' que sirven como etapas escalonadas y
transitorias que al ser superadas conducen a ver el mundo correctamente. Este
esfuerzo hace de pensamientos opacos e indistintos unos claros y con límites
bien definidos.\autocite[cf. 4.112 y 6.54]{wittgenstein1922tractatus}
La posibilidad de llegar a una visión clara del mundo es fruto de la posibilidad
de lograr aclarar la lógica del lenguaje. El lenguaje, a su vez, está compuesto
de la totalidad de las proposiciones, y éstas, cuando tienen sentido,
representan el pensamiento.\autocite[cf. 4 y 4.001]{wittgenstein1922tractatus}
Sin embargo, el mismo lenguaje que puede expresar el pensamiento lo disfraza:

\citalitlar{4.002~El lenguaje disfraza el pensamiento; de tal manera que de la
    forma externa de sus ropajes uno no puede inferir la forma del pensamiento
    que estos revisten, porque la forma externa de la vestimenta esta elaborada
    con un propósito bastante distinto al de favorecer que la forma del cuerpo
    sea conocida.}

El intento de llegar desde el lenguaje al pensamiento por medio de las
proposiciones con significado es el esfuerzo por conocer una imagen de la
realidad. El pensamiento es la imagen lógica de los hechos, en él se contiene la
posibilidad del estado de las cosas que son pensadas y la totalidad de los
pensamientos verdaderos es una imagen del mundo.\autocite[cf.][3 y
3.001]{wittgenstein1922tractatus}

\subsubsection{El pensamiento como representación}

El pensamiento es representación de la realidad por la identidad existente entre
la posibilidad de la estructura de una proposición y la posibilidad de la
estructura un hecho:

\citalitlar{Los objetos ---que son simples--- se combinan en situaciones
    elementales. El modo en el que se sujetan juntos en una situación tal es su
    estructura. Forma es la posibilidad de esa estructura. No todas las
    estructuras posibles son actuales: una que es actual es un `hecho
    elemental'. Nosotros formamos imágenes de los hechos, de hechos posibles
    ciertamente, pero algunos de ellos son actuales también. Una imagen consiste
    en sus elementos combinados en un modo específico. Al estar así presentan a
    los objetos denominados por ellos como combinados específicamente en ese
    mismo modo. La combinación de los elementos de la imagen ---la combinación
    siendo presentada--- se llama su estructura y su posibilidad se llama la
    forma de representación de la imagen.
    Esta `forma de representación' es la posibilidad de que las cosas están
    combinadas como lo están los elementos de la imagen.
    \footnote{\cite[cf.][p.~171]{simplicity}; \cite[n.~2.15]{wittgenstein1922tractatus}}}

La representación y los hechos tienen en común la forma lógica:
\citalitlar{2.18~Lo que toda representación, de una forma cualquiera, debe tener
    en común con la realidad, de manera que pueda representarla ---cierta o
    falsamente--- de algún modo, es su forma lógica, esto es, la forma de la
    realidad.\autocite[p.34]{wittgenstein1922tractatus}}

\subsubsection{Las proposiciones como proyecciones con polos de verdad-falsedad}
    La imagen de la realidad se convierte en proposición en el momento en que
    nosotros correlacionamos sus elementos con las cosas
    actuales.\autocite[cf.~][p.\,73]{anscombe1959iwt}
    La condición de posibilidad de entablar dicha correlación es la relación interna
    entre los elementos de la imagen en una estructura con
    sentido.\autocite[cf.~][p.~68]{anscombe1959iwt}
    De este modo:
    \citalitlar{5.4733~Frege dice: Toda proposición legítimamente construida tiene
        que tener un sentido; y yo digo: Toda proposición posible está legítimamente
        construida, y si ésta no tiene sentido es sólo porque no hemos dado
        significado a alguna de sus partes constitutivas. (Incluso cuando pensemos
        que lo hemos hecho.)\autocite[p.~78]{wittgenstein1922tractatus}}

    La proposición expresa el pensamiento perceptiblemente por medio de signos.
    Usamos los signos de las proposiciones como proyecciones del estado de las cosas
    y las proposiciones son el signo proposicional en su relación proyectiva con el
    mundo. A la proposición le corresponde todo lo que le corresponde a la
    proyección, pero no lo que es proyectado, de tal modo, que la proposición no
    contiene aún su sentido, sino la posibilidad de expresarlo; la forma de su
    sentido, pero no su contenido.\autocite[cf.~][3.1,3.11-3.13]{wittgenstein1922tractatus}

    La proposición no `contiene su sentido' porque la correlación la hacemos nosotros,
    al `pensar su sentido'. Hacemos esto cuando usamos los elementos de la
    proposición para representar los objetos cuya posible configuración estamos
    reproduciendo en la disposición de los elementos de la proposición. Esto es lo
    que significa que la proposición sea llamada una imagen de la
    realidad.\autocite[cf.~][p.69]{anscombe1959iwt}

    Toda proposición-imagen tiene dos acepciones. Puede ser una descripción de
    la existencia de una configuración de objetos o puede ser una descripción de la
    no-existencia de una configuración de objetos.\autocite[cf.~][p.~72]{anscombe1959iwt}
    %Es una peculiaridad de la proyección el que de ésta y del método de proyección
    %se puede decir qué es lo que se está proyectando, sin que sea necesario que tal
    %cosa exista físicamente.\autocite[cf.~][p.~72]{anscombe1959iwt}
    %La idea de la proyección es peculiarmente apta para explicar el carácter de una
    %proposición como teniendo sentido independientemente de los hechos, como
    %inteligible aún antes de que se sepa que es cierta; como algo que concierne lo
    %que se puede cuestionar sobre si es verdad, y saber lo que se pregunta antes de
    %conocer la respuesta.\autocite[cf.~][p.~73]{anscombe1959iwt}
    Esta doble acepción es el resultado de que la proposición-imagen puede ser una
    proyección hecha en sentido positivo o negativo.\autocite[cf.~][p.~74]{anscombe1959iwt} Esto
    queda ilustrado en una analogía:

    \citalitlar{4.463~La proposición, la imagen, el modelo, son en el sentido
        negativo como un cuerpo solido, que restringe el libre movimiento de otro:
        en el sentido positivo, son como un espacio limitado por una sustancia
        sólida, en la cual un cuerpo puede ser colocado.\autocite[p.~63]{wittgenstein1922tractatus}}

    De este modo toda proposición-imagen tiene dos polos; de verdad y de falsedad.
    Las tautologías y las contradicciones, por su parte, no son imagenes de la
    realidad ya que no representan ningún posible estado de las cosas. Así continúa
    la ilustración anterior:

    \citalitlar{4.463~Una tautología deja abierto para la realidad el total infinito
        del espacio lógico; una contradicción llena el total del espacio lógico no
        dejando ningún punto de él para la realidad. Así pues ninguna de las dos
        puede determinar la realidad de ningún modo.\autocite[p.~78]{wittgenstein1922tractatus}}

    La verdad de las proposiciones es posible, de las tautologías es cierta y de las
    contradicciones imposible. La tautología y la contradicción son los casos límite
    de la combinación de signos ---específicamente--- su
    disolución.\autocite[cf.~][4.464 y 4.466]{wittgenstein1922tractatus} Las tautologías son
    proposiciones sin sentido (carecen de polos de verdad y falsedad), su negación son
    las contradicciones. Los intentos de decir lo que sólo puede ser mostrado
    resultan en esto, en formaciones de palabras que carecen de sentido, es decir,
    son formaciones que parecen oraciones, cuyos componentes resultan no tener
    significado en esa forma de oración.\autocite[cf.~][p.~163~\S2]{anscombe1959iwt}.

\subsubsection{La distinción entre el decir y el mostrar}
      La conexión entre las tautologías y aquello que no se puede decir, sino
      mostrar, es que éstas ---siendo proposiciones lógicas sin sentido--- muestran
      la 'lógica del mundo'.\autocite[cf.~][p.~163~\S3]{anscombe1959iwt}. Esta 'lógica del
      mundo' o 'de los hechos' es la que más prominentemente aparece en el Tractatus
      entre las cosas que no pueden ser dichas, sino mostradas. Esta lógica no solo
      se muestra en las tautologías, sino en todas las proposiciones. Queda exhibida
      en las proposiciones diciendo aquello que pueden decir.

      La forma lógica no puede expresarse desde el lenguaje, pues es la forma del
      lenguaje mismo, se hace manifiesta en éste, no es representativa de los objetos
      y tampoco puede ser representada por signos, tiene que ser mostrada:
      \citalitlar{4.0312~La posibilidad de las proposiciones se basa en el principio de
          la representación de los objetos por medio de signos. Mi pensamiento
          fundamental es que las ``constantes lógicas'' no son representativas. Que la
          lógica de los hechos no puede ser representada.\autocite[p.~48]{wittgenstein1922tractatus}}

      La lógica es, por tanto, trascendental, no en el sentido de que las
      proposiciones sobre lógica afirmen verdades trascendentales, sino en que todas
      las proposiciones muestran algo que permea todo lo decible, pero es en sí mismo
      indecible.\autocite[cf.~][p.~166 \S2]{anscombe1959iwt}

      Otra cuestión notoria entre aquello que no puede ser dicho, sino mostrado es la
      cuestión acerca de la verdad del solipsismo. Los limites del mundo son los
      límites de la lógica, lo que no podemos pensar, no podemos pensarlo, y por tanto
      tampoco decirlo. Los límites de mi lenguaje significan los límites de mi
      mundo.\autocite[cf~.][5.6~y~5.61]{wittgenstein1922tractatus} De este modo:
      \citalitlar{5.62~[\ldots]Lo que el solipsismo \emph{significa}, es ciertamente
          correcto, sólo que no puede ser \emph{dicho}, pero se muestra a sí
          mismo. Que el mundo es \emph{mi} mundo, se muestra a sí mismo en el hecho
          de que los limites del lenguaje (de \emph{aquel} lenguaje que yo
          entiendo) significan los límites de mi
          mundo.\autocite[cf~.][p.~89]{wittgenstein1922tractatus}}

      Así como la lógica del mundo y la verdad del solipsismo quedan mostradas,
      también, las verdades éticas y religiosas, aunque no expresables, se manifiestan
      a sí mismas en la vida.

      Existe, por tanto lo inexpresable que se muestra a sí mismo, esto es lo
      místico.\autocite[cf.~][6.522]{wittgenstein1922tractatus}

      De la voluntad como sujeto de la ética no podemos
      hablar\autocite[cf.~][6.423]{wittgenstein1922tractatus}. El mundo es independiente de nuestra
      voluntad ya que no hay conexión lógica entre ésta y los hechos.
      La voluntad y la acción como fenómenos, por tanto, interesan sólo a la
      psicología.\autocite[cf.~][p.171 \S3]{anscombe1959iwt}

      El significado del mundo tiene que estar fuera del
      mundo\autocite[cf.~][6.41]{wittgenstein1922tractatus} y Dios no se revela \emph{en} el
      mundo\autocite[cf.~][6.432]{wittgenstein1922tractatus}.
      Esto se sigue de la teoría de la representación; una proposición y su negación
      son ambas posibles, cuál es verdad es accidental.\autocite[cf.~][p.170 \S4]{anscombe1959iwt}
      Si hay un valor que valga la pena para el mundo tiene que estar fuera de lo que
      es el caso que es; lo que hace que el mundo tenga un valor no-accidental tiene
      que estar fuera de lo accidental, tiene que estar fuera del
      mundo.\autocite[cf.~][6.41]{wittgenstein1922tractatus}

      Finalmente, aplicar el límite de lo que puede ser expresado a la actividad
      filosófica significa que:
      \citalitlar{6.53~El método correcto para la filosofía sería este. No decir nada
          excepto lo que pueda ser dicho, esto es, proposiciones de la ciencia
          natural, es decir, algo que no tiene nada que ver con la filosofía: y luego
          siempre, cuando alguien quiera decir algo metafísico, demostrarle que no ha
          logrado dar significado a ciertos signos en sus proposiciones. Este método
          sería insatisfactorio para la otra persona ---no tendría la impresión de que
          le estuviéramos enseñando filosofía--- pero este método sería el único
          estrictamente correcto.\autocite[p. 107--108]{wittgenstein1922tractatus}}

        La frase usada para describir la obra: \citalitinterlin{de lo que no podemos
          hablar, de eso hemos de guardar silencio}, pertende expresar tanto una
        verdad logico-filosófica como un precepto ético. El sinsentido que resulta de
        tratar de decir lo que sólo puede ser mostrado no sólo es lógicamente
        insostenible, sino éticamente indeseable.\autocite[cf.~][p.~156]{monk1991duty}
        Wittgenstein explicó esta finalidad ética de su obra en una carta a Ludwig von
        Ficker de este modo: \citalitlar{[\ldots] el punto del libro es ético. Hubo un
          tiempo en que quise ofrecer en el prefacio algunas palabras que ya no están
          ahí, éstas, sin embargo, quiero escribirtelas ahora porque pueden ser clave
          para ti: quise escribir que mi trabajo consiste en dos partes: en la que
          está aquí, y en todo lo que \emph{no} he escrito. Y precisamente esta
          segunda parte es la importante. Pues lo ético es delimitado desde dentro,
          por así decirlo, por mi libro; y estoy convencido de que,
          \emph{estrictamente} hablando, éste SÓLO puede ser delimitado de este modo.
          En resumen, pienso que: todo de lo que \emph{muchos} están
          \emph{mascullando} hoy en día, lo he definido en mi libro al mantenerme en
          silencio sobre ello.\autocite[p.~22-23]{howtoread}}

\subsection{Del \emph{Tractatus} a \emph{Investigaciones Filosóficas}}
Aún como prisionero en Cassino, Wittgenstein había decidido que a su regreso a
Viena se prepararía para ser profesor de escuela
elemental\autocite[cf.~][p.~158]{monk1991duty}. Fue liberado en agosto de 1919
y, según su propósito, se enlistó en el \emph{Lehrerbildungsanhalt} para recibir
formación en enseñanza. En septiembre de 1920 estaría en el pequeño pueblo de
Trattenbach en Noruega como profesor de escuela elemental. A lo largo de aquel
año intentó sin éxito la publicación del Tractatus y tuvo que dejar la tarea en
manos de Russell al partir hacia Noruega. En 1922 el libro de Wittgenstein sería
finalmente publicado.

En 1929 Wittgenstein regresó a la tarea filosófica. Presentó el \emph{Tractatus
  Logico\=/Philosophicus} como su tesis doctoral en Cambridge y recibió un
fellowship de cinco años en ``Trinity College''. Comenzó sus lecciones en el
periodo Lent de 1930. Terminó su fellowship en el curso 1935-1936 y tomó un
receso. Regresó a ofrecer lecciones en Cambridge en 1938. El 11 de febrero de
1939 fue nombrado a la cátedra de filosofía en Cambridge tras el retiro de
G.\,E.\,Moore. Permanecería en esta labor hasta su retiro en 1947.

Cuando Wittgenstein regresó a la filosofía en 1929 encontró grandes defectos en
las tesis lógicas y metafísicas del Tractatus. Esto le llevó a abandonar
principios relacionados con la idea central de su teoría de la imagen. Rechazó
la noción de los objetos simples como significados de los nombres simples, la
concepción de los hechos y las ideas como compartiendo la forma lógica o la
propuesta de que toda inferencia lógica depende de una composición de función de
verdad\autocite[cf.][p.~44]{bakerhacker2014rules}.

Una idea que no abandonó inicialmente, sino que reforzó, fue la del lenguaje
como un cálculo de reglas. En el \emph{Tractatus} había propuesto que cualquier
lenguaje posible tiene como base la estructura de un cálculo lógico--sintáctico
conectado a la realidad por nombres lógicamente apropiados cuyos significados
son objetos simples que constituyen la sustancia del mundo. Su argumentación
ahora es que cualquier lenguaje posible es un calculo autónomo de reglas y el
significado es otorgado a los signos primitivos indefinibles, en parte, por
medio de definiciones ostensivas. Las muestras empleadas en la definición
ostensiva son ellas mismas parte de los medios de representación. Según esto el
significado de una expresión no es un objeto en la realidad, sino que consiste
en la totalidad de las reglas que determinan su uso dentro del cálculo del
lenguaje. El significado de una palabra es su lugar en la gramática, su rol en
el cálculo\autocite[cf.~][p.44]{bakerhacker2014rules}.

En 1931 empezaría a proponer que el hablar un lenguaje es un sistema
multifacético de actividades gobernadas por reglas, abandonando la idea de que
hay un sistema de reglas que rigen un cálculo que está debajo y sostiene todo
discurso significativo. Entonces fue dejando de hablar del cálculo del lenguaje
y empezó a usar el calcular como una analogía para describir el uso del
lenguaje. La operación de hacer un cálculo y seguir las reglas que éste sugiere
guarda relación con el modo en el que operamos cuando usamos el lenguaje y
seguimos las reglas que éste nos presenta.

Subsecuentemente abandonaría incluso la analogía del cálculo. En 1930 había
empezado a comparar el lenguaje con un juego de ajedrez al reflexionar en el
debate entre Frege y formalistas matemáticos como Heine, Thomae y
Weyl.\autocite[cf.~][p.134]{bakerhacker2014rules} En 1931 empezó a preferir esta
analogía a la del cálculo. Al igual que al hacer un cálculo, al jugar un juego
se siguen reglas que gobiernan las operaciones realizadas dentro de éste. Las
palabras son como piezas de ajedrez, las explicaciones de los significados de
las palabras son como las reglas del ajedrez y los significados de las palabras
son como el potencial de movimiento y captura de las piezas de ajedrez. La
analogía del ajedrez para hablar del lenguaje resultó fructífera precisamente
porque se trata de un juego. El uso de las expresiones es involucrarse en un
juego de lenguaje.

Fue así como Wittgenstein fue cambiando su atención hacia los usos de las
expresiones en las prácticas humanas y su investigación empezó a girar en torno
al hablar como una actividad integrada en la vida humana, entretejida con otra
multitud de acciones, actividades, relaciones y respuestas.

Wittgenstein llegará a sostener, como queda atestiguado en \emph{Investigaciones
  Filosóficas} \S90, que la filosofía es una investigación gramática en la que
los problemas filosóficos son resueltos por medio de la descripción del uso de
las palabras, clarificando la gramática de las expresiones y tabulando reglas.
Con Moore, se podría objetar que gramática es el tipo de cosas que se enseña a
los niños en la escuela, por ejemplo: <<no se dice ``tres hombres \emph{estaba}
en el campo'', sino ``tres hombres \emph{estaban} en el campo''>> ---eso es
gramática. Y ¿qué tiene que ver eso con filosofía? A lo que Wittgenstein
contestaría: efectivamente este ejemplo no tiene nada que ver con filosofía, ya
que en él todo está claro. Pero qué tal si dijéramos ``Dios el Padre, Dios el
Hijo y Dios el Espíritu Santo''; ¿\emph{estaban} en el campo o \emph{estaba} en
el campo?\autocite[cf.~][55]{bakerhacker2014rules}

%Esta metodología resultante de la evolución en la filosofía de Wittgenstein será
%en la que tomaría parte Elizabeth Anscombe cuando llegó a sus lecciones en 1942.

\subsection{El nuevo método de Wittgenstein}
En sus reflexiones sobre los fundamentos de las matemáticas entre 1937 y 1938,
Wittgenstein plantea la siguiente pregunta: \citalitinterlin{¿Cómo sé que al
  calcular la serie $+2$ debo escribir `$20004$, $20006$' y no `$20004$,
  $20008$'?}

La pregunta tiene que ver con el modo en el que actuamos según una regla. Al
calcular esta serie se ha ofrecido $+2$ como norma para el cálculo. Ahora la
pregunta es cómo se sabe qué hacer con ese conocimiento previo cuando llega el
momento de ponerlo en acto. Si se ha comprendido la guia inicial se tendrá
certeza sobre qué hacer después de $20004$, y esta certeza no implica que
$20006$ haya quedado determinado de antemano, sino que en que ante cualquier
número ofrecido se tiene la capacidad de ofrecer el siguiente. Entonces
continua:
\citalitlar{<<¿Pero entonces en qué consiste la peculiar inexorabilidad de las
  matemáticas?>> ---¿No será acaso la inexorabilidad con la que dos sigue a uno
  y tres a dos un buen ejemplo? ---Pero presuntamente esto significa: se sigue
  así en la \emph{serie de números cardinales}; pues en una serie distinta se
  seguiría de un modo distinto. Pero ¿acaso esta serie no está definida
  precisamente por esta secuencia? ---<<¿Hay que suponer que esto significa que
  cualquier modo en el que una persona cuente es igualmente correcto, y que
  cualquiera puede contar en el orden que quiera?>> ---Probablemente no lo
  llamaríamos `contar' si todo el mundo dijera los números uno después de otro
  \emph{de cualquier manera}; pero por supuesto esto no se trata simplemente de
  un problema sobre el nombre que se usa. Pues lo que llamamos `contar' es una
  parte importante de las actividades de nuestras vidas. Contar y calcular no
  son --por ejemplo-- un simple pasatiempo. Contar (y eso significa: contar
  \emph{así}) es una técnica que es empleada diariamente en las operaciones más
  variadas de nuestras vidas. Y por eso es que aprendemos a contar como lo
  hacemos: con prácticas interminables, con despiadada exactitud; por eso es que
  es inexorablemente insistido que hemos de decir `dos' después de `uno', `tres'
  después de `dos' y así sucesivamente. ---<<Pero entonces este contar es sólo
  un uso; ¿acaso no hay alguna verdad que se corresponda con esta secuencia?>>
  La \emph{verdad} es que contar ha demostrado que paga. ---<<Entonces quieres
  decir que `ser verdad' significa: ser utilizable (o útil)?>> ---No, no eso;
  pero que no puede ser dicho de la serie de números naturales --y tampoco de
  nuestro lenguaje-- que es verdad, pero: que es utilizable, y, sobre todo que
  \emph{se usa de hecho}.\autocite[p.~37 \S4]{wittgenstein1956remmath}}

A la pregunta sobre cómo continuar la serie, Wittgenstein añade la observación:
\citalitinterlin{la pregunta <<¿cómo sé que este color es `rojo'?>> es similar.}
La cuestión planteada no solo tiene que ver con el modo en el que vamos según
una serie, sino con las operaciones que hacemos con las palabras. Tambíen con
las palabras hay una comprensión inicial de su uso que luego se aplica en cada
caso. ¿Cómo sé que en esta ocasión estoy empleando una expresión según la regla
que es su uso?

En \emph{Investigaciones Filosóficas} \S380 encontramos:
\citalitlar{¿Cómo reconozco que esto es rojo? ---``Veo que es \emph{esto}; y
  entonces sé que eso es lo que esto es llamado'' ¿Esto? ---¡¿Qué?! ¿Qué tipo de
  respuesta a esta pregunta tiene sentido? (Sigues girando hacia una explicación
  ostensiva interna.) No podría aplicar ninguna regla a una transición
  \emph{privada} desde lo que es visto a las palabras. Aquí las reglas realmente
  quedarían suspendidas en el aire; pues la institución para su aplicación esta
  ausente.}

Y añade en \S381: \citalitinterlin{¿Cómo reconozco que este color es rojo?
  ---Una respuesta sería: <<He aprendido [castellano]>>.} Ir según una regla es
ir según una costumbre, un uso, una institución; \citalitinterlin{entender una
  oración significa entender un lenguaje, entender un lenguaje significa dominar
  una técnica\autocite[p.~87 \S9]{wittgenstein1953phiinv}.} La gramática de la
expresión `seguir una regla' supone la existencia de una prática, una
regularidad, un comportamiento normativo. Sólo cuando esta red de
comportamientos está en juego se puede hablar de que existe una
regla\autocite[cf.~][p.~14]{bakerhacker2009understanding}. No es posible que
haya una sola persona que en una sola ocasión `siguió una regla', esta
consideración no es correspondiente con la gramática de la
expresión\autocite[cf.~][p.~87 \S1 199]{wittgenstein1953phiinv}.

Cuando Elizabeth Anscombe participó de estas discusiones en las clases con
Wittgenstein encontró una ruta para sus propias indagaciones filosóficas.
\citalitinterlin{En cierto punto Wittgenstein estaba discutiendo en sus clases
  la interpretación del letrero (sign-post), y estalló en mi que el modo en que
  vas según éste es la interpretación
  final.\autocite[p.~viii]{anscombe1981metaphysicsintro}} Un letrero es una
expresión de una regla ante la que hemos sido entrenados a reaccionar de un modo
particular. Pensar que se está siguiendo una regla no es seguir una regla, y por
eso no es posible seguir una regla `privadamente' \autocite[cf.~][p.87 /S1
202]{wittgenstein1953phiinv}. La interpretación definitiva de una expresión de
una regla es cómo se actua ante ella.

Durante sus estudios en Oxford Anscombe había rechazado con fuerza un realismo
representativo lockeano que insistía que los colores como ella los veía no son
parte del mundo externo. Como reacción contraria tendía a identificar estas
sensaciones con \emph{esto} (this), como si `azul' o `amarillo' fueran artículos
que `están ahí'. Esta noción también le parecía equivocada, pero no lograba
librarse de ella\autocite[cf.][210]{diamond2004crisscross}: \citalitlar{En otra
  ocasión salí con: <<Pero todavía quiero decir: ``Azul esta ahí''>>. Manos más
  veteranas sonrieron o rieron, pero Wittgenstein las detuvo tomándolo en serio,
  diciendo: <<Déjame pensar qué medicina necesitas\ldots>> <<Supón que tenemos
  la palabra \emph{`painy'}, como una palabra para la propiedad de ciertas
  superficies>>. La `medicina' fue efectiva\ldots}
  % y la historia ilustra la habilidad de Wittgenstein para comprender el
  % pensamiento que se le estaba siendo ofrecido en objeción.
\citalitlar{
  % Uno podría protestar, desde luego, que precisamente ésto es equivocado en la
  % asimilación que hace Locke de las cualidades secundarias al dolor: puedes
  % esbozar el funcionamiento de ``dolor'' como una palabra para una cualidad
  % secundaria, pero no puedes hacer la operación inversa. Pero la `medicina' no
  % implicaba que podrías.
  [\ldots] Si \emph{`painy'} fuera una palabra posible para una cualidad
  secundaria, ¿no podría el mismo motivo conducirme a decir: \emph{`painy'} está
  aquí que lo que me condujo a decir `azul' está aquí?
  % Mi expresión no significaba que `azul' es el nombre de esta sensación que
  % estoy teniendo, ni cambié a ese pensamiento.
  \autocite[p.~viii]{andcombe1981metaphysicsintro}}

¿Qué cambió en la comprensión del lenguaje para Anscombe?

\subsection{Investigaciones Filosóficas}
% Al igual que con la introducción al análisis presentado para el Tractatus
% resumimos en este parrafo los puntos que se trataran sobre Investigaciones
% Filosóficas.
Las primeras lineas del prefacio de \emph{Investigaciones Filosóficas} leen:
\citalitinterlin{Los pensamientos que publico en lo que sigue son el precipitado
  de investigaciones filosóficas que me han ocupado durante los últimos
  dieciseis años.} El prefacio fue escrito en 1945.

Qué vamos a ver?

Estructura general según baker and hacker:

1-27a Explicación preliminar de concepcion agustiniana del lenguaje

27b-64 malentendidos acerca de los nombres y el uso de los nombres bajo la
concepción agustiniana

65-88 investigación sobre concepción de nombres simples ligados a objetos
simples que son los constituyentes últimos de la realidad

89-108 crítica de los principios metodológicos más profundos que guiaron el
tractatus y repudio de una concepción sublime de la filosofía y la investigación
lógica que lo informó

109-133 bosquejo de la nueva concepción de la filosofía y de sus métodos

133-142 transición desde la discusión de doble faz de la filosofía y la
subsecuente investigación sobre el comprender

143-184 contra una idea de que comprender es un estado que implica que la
aplicación está comprendida previo a su uso, esto para aclarar el status
categorial de comprender

185-242 complementa la secuencia de comentarios anterior y clarifica la relación
entre entender una expresión, el significado o uso de esta y la explicación de
lo que significa, que es una regla para su uso

243-315
incorpora los argumentos sobre el lenguaje privado

316-362 on thinking

363-397 on imagination

398-427 mundo subjetivo de sensación experiencia y imaginación, el yo y auto
referencia y conceptos de conciencia y auto conciencia

428-65 el malentendido de que el significado de los signos, su habilidad para
representar lo que representan depende de procesos mentales de pensar

466-490 discusión breve sobre el problema de la justificación del razonamiento
inductivo

491-570 examen de significado y otros problemas relacionados

571-693 conceptos psicológicos
