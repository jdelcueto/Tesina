\section{Filosofía analítica y método filosófico de Elizabeth Anscombe}

%\subsection{Dos `cortes' en la filosofía.}

Para comprender el método filosófico de Anscombe hay que tener en cuenta algunas nociones básicas del método de filosofía analítica empleado por L. Wittgenstein\footnote{Para este estudio hemos tenido muy en cuenta las aportaciones de Wittgenstein, puesto que están muy presentes en la filisofía de Anscombe. Ha servido como bibliografía secundaria tanto: \cite{wittgenstein1922tractatuses}, como \cite{wittgenstein1953phiinv}. También han sido importantes: \cite{wittgenstein1969oncertes} y además: \cite{wittgenstein1956remmathes}. Igualmente ha sido fundamental la consulta a comentarios: \cite{anscombe1959iwt} y también: \cite{bakerhacker2009understanding}, \cite{bakerhacker2014rules} y \cite{hacker2000mind}.}. Un elemento de esta metodología que fue importante para la madurez filosófica de Elizabeth y que constituye además una de las constantes del pensamiento de Wittgenstein fue su definición de la naturaleza de los problemas filosóficos. Para él las cuestiones de la filosofía no son problemáticas por ser erróneas, sino por no tener significado\footcite[Cf.][\S4.003]{wittgenstein1922tractatuses}. Una proposición sin significado que no es puesta al descubierto como tal atrapa al filósofo dentro de una confusión del lenguaje que no le permite acceder a la realidad. Salir de la confusión no consiste en refutar una doctrina y plantear una teoría alternativa, sino en examinar las operaciones hechas con las palabras para llegar a manejar una visión clara del empleo de nuestras expresiones. La filosofía no es un cuerpo doctrinal, sino una actividad\footcite[Cf.][\S4.112]{wittgenstein1922tractatuses} y una terapia\footcite[Cf.][\S133]{wittgenstein1953phiinv}.

La actitud terapéutica adoptada por Wittgenstein en su atención de las confusiones filosóficas fue su respuesta más definitiva a la naturaleza de estos problemas. Para ello encontró remedio en sus investigaciones sobre el significado y el sentido del lenguaje.
%Ordinariamente tomamos parte en esta actividad humana que es el lenguaje. Jugamos el juego del lenguaje. ---¿Jugarlo es entenderlo?--- A la vista de Wittgenstein saltaban extraños problemas sobre las reglas de este juego; entonces no podía evitar escudriñarlas al detalle\footcite[Cf.][356]{monk1991duty}. En este análisis del lenguaje está la raíz de sus ideas sobre el sentido, el significado y la verdad.
Durante su vida sostuvo dos grandes descripciones del significado. Originalmente describió el lenguaje como una imagen que representa el posible estado de las cosas en el mundo\footcite[Cf.][\S4.021-4.023]{wittgenstein1922tractatuses}. En una segunda etapa se distanció de esta analogía\footcite[Cf.][\S114-115;120]{wittgenstein1953phiinv} para describir al lenguaje como una herramienta cuyo significado consiste en la suma de las múltiples semejanzas familiares que aparecen en los distintos usos para los cuales el lenguaje es empleado en la actividad humana\footcite[Cf.][\S65-67; 77; 122-133; 569]{wittgenstein1953phiinv}. Dentro de la primera descripción una expresión sin significado es una cuyos elementos no componen una representación del posible estado de las cosas\footcite[Cf.][\S4.03; 4.06; 4.064; 4.112]{wittgenstein1922tractatuses}. Dentro de la segunda descripción una expresión sin significado es una que no tiene una aplicación posible dentro del contexto de un ``juego del lenguaje''\footcite[Cf.][\S80-85; 496-500; 559-568]{wittgenstein1953phiinv}.

Estas dos etapas del pensamiento de Wittgenstein son representadas por dos importantes tratados. El \emph{'Tractatus Logico-Philosophicus'}, publicado en 1921, recoge sus esfuerzos por elaborar un gran tratado filosófico comenzados en 1911 y culminados durante la Primera Guerra Mundial. El segundo, \emph{'Philosophische Untersuchungen'}, o \emph{'Investigaciones Filosóficas'}, traducido por Anscombe y publicado póstumamente en 1953, fue elaborado a partir de múltiples manuscritos desarrollados por Wittgenstein desde su regreso a Cambridge en 1929 hasta su muerte en 1951.

\blockquote[{\Cite[181]{anscombe2011plato:twocuts}}: \enquote{Wittgenstein is extraordinary among philosophers for having made two epochs, or cuts, in the history of philosophy}.]{Wittgenstein es extraordinario entre los filósofos por haber generado dos épocas, o cortes, en la historia de la filosofía}. Con estas palabras Anscombe comenzaría su discurso inaugural para el Sexto Simposio Internacional de Wittgenstein unos treinta años después de la publicación de las \emph{'Investigaciones Filosóficas'}. Y explica: \blockquote[{\Cite[181]{anscombe2011plato:twocuts}}: \enquote{a philosopher makes a cut if he makes a difference to the way philosophy is done: philosophy after the cut cannot be the same as before}.]{un filósofo hace un corte si genera un cambio en el modo en que la filosofía es hecha: la filosofía tras el corte no puede ser la misma de antes}.

Estos cambios de época generados por la influencia de Wittgenstein vinieron caracterizados por el esfuerzo de comprender cada libro tras su publicación, tarea complicada en ambos casos por la dificultad intrínseca de los tratados, ofuscada a su vez por los prejuicios filosóficos proyectados a cada obra por sus lectores\footnote{\cite[Cf.][183]{anscombe2011plato:twocuts}: \enquote{the assumption that the \emph{Philosophical Investigations} presents us a theory of language ---a theory, say, of how sounds become significant speech--- will quickly place us at a distance from the very questions which Wittgenstein is occupied with}.}.
%Elizabeth explica que: \blockquote[{\Cite[Cf.][183]{anscombe2011plato:twocuts}}: \enquote{the assumption that the \emph{Philosophical Investigations} presents us a theory of language ---a theory, say, of how sounds become significant speech--- will quickly place us at a distance from the very questions which Wittgenstein is occupied with}.]{la presunción, por ejemplo, de que \emph{'Investigaciones Filosóficas'} presenta una teoría del lenguaje ---quizás sobre cómo los sonidos se tornan en discursos significativos--- nos dejaría situados lejos de las preguntas que genuinamente ocupan a Wittgenstein}.
Ahora bien, la comprensión adecuada de su pensamiento y método trae consigo, a juicio de Anscombe, cierto efecto curativo.

%Según Anscombe el método general adecuado de discutir los problemas filosóficos propuesto por Wittgenstein consiste en mostrar que la persona no ha provisto significado (o referencia) para ciertos signos en sus expresiones\footnote{\cite[Cf.][151]{anscombe1959iwt}: \enquote{The general method that Wittgenstein does suggest is that of `shewing that a man has supplied no meaning [or perhaps: ``no reference''] for certain signs in his sentences'}.}. Creía que el camino que lleva a formular estos problemas está frecuentemente trazado por la mala comprensión de la lógica de nuestro lenguaje.

%Cada obra de Wittgenstein representa su esfuerzo de superar estas confusiones y propone un método para remediarlas. Su primera propuesta plantea que el modo de aclarar las confusiones de los problemas filosóficos consiste en identificar en el lenguaje el límite de lo que expresa pensamiento; lo que queda al otro lado de esta frontera sería simplemente sinsentido. En otras palabras: \blockquote[{\Cite[11]{wittgenstein1922tractatuses}}.]{lo que siquiera puede ser dicho, puede ser dicho claramente; y de lo que no se puede hablar, hay que callar}. Con esta expresión Wittgenstein resumió el sentido de la obra que ahora examinaremos.

%\subsection{Las elucidaciones del \emph{Tractatus}}
% Este párrafo resume los cuatro puntos del Tractatus que se desglosarán en los próximos párrafos
%Desde las proposiciones principales del \emph{Tractatus} queda claro que el tema central del libro es la conexión entre el lenguaje, o el pensamiento, y la realidad.
% 1.Filosofía como actividad
%En este nexo es donde la actividad filosófica ha de buscar esclarecer el pensamiento.
% 2.El pensamiento como representación
%La tesis básica sobre esta relación consiste en que las proposiciones, o su equivalente en la mente, son imágenes de los hechos.
% 3.Las proposiciones como proyecciones con polos de verdad-falsedad
%La proposición es la misma imagen tanto si es cierta como si es falsa, es decir, es la misma imagen sin importar que lo que se corresponde a esta es el caso que es cierto o no. El mundo es la totalidad de los hechos, a saber, de lo equivalente en la realidad a las proposiciones verdaderas.
% 4.La distinción entre el decir y el mostrar
%Solo las situaciones que pueden ser plasmadas en imágenes pueden ser afirmadas en proposiciones. Adicionalmente hay mucho que es inexpresable, lo cual no debemos intentar enunciar, sino más bien contemplar sin palabras\footnote{\cite[Cf.][19]{anscombe1959iwt}: \enquote{There is indeed much that is inexpressible --- which we must not try to state, but must contemplate without words}.}.

%\subsection{\emph{Investigaciones Filosóficas} y el nuevo método de Wittgenstein}

%Anscombe conoció a Wittgenstein en la segunda etapa de su pensamiento, y trabajó con él para traducir \emph{Investigaciones Filosóficas}, así que hemos de atribuir a esta etapa tardía la mayor influencia en el pensamiento de Elizabeth. Sin embargo una de las discusiones más amplias del pensamiento de Wittgenstein en la obra de Anscombe se encuentra en \emph{An Introduction to Wittgenstein's Tractatus}. El mismo Wittgenstein reiteró que su pensamiento tardío solo puede entenderse a la luz del \emph{Tractatus}, sin embargo esto no terminaría de explicar el interés de Anscombe en esa obra. Quizás es más correcto decir que el \emph{Tractatus}, con su énfasis en el tema de la verdad, no dejó de ser una reflexión con mérito para Elizabeth como complemento de la atención que presta \emph{Investigaciones Filosóficas} al tema del sentido\footcite[Cf.][191-193]{teichmann2008ans}.

%\emph{Tractatus} queda claro que el tema central del libro es la conexión entre el lenguaje, o el pensamiento, y la realidad. En este nexo es donde la actividad filosófica ha de buscar esclarecer el pensamiento. La tesis básica sobre esta relación consiste en que las proposiciones, o su equivalente en la mente, son imágenes de los hechos. La proposición es la misma imagen tanto si es cierta como si es falsa, es decir, es la misma imagen sin importar que lo que se corresponde a esta es el caso que es cierto o no. El mundo es la totalidad de los hechos, a saber, de lo equivalente en la realidad a las proposiciones verdaderas. Solo las situaciones que pueden ser plasmadas en imágenes pueden ser afirmadas en proposiciones. Adicionalmente hay mucho que es inexpresable, lo cual no debemos intentar enunciar, sino más bien contemplar sin palabras\footnote{\cite[Cf.][19]{anscombe1959iwt}: \enquote{There is indeed much that is inexpressible --- which we must not try to state, but must contemplate without words}.}.

%En este apartado veremos algunos aspectos de las discusiones de Wittgenstein en esta segunda obra. La descripción será más general que la del \emph{Tractatus} ya que el análisis de los artículos de Anscombe en el capítulo siguiente nos dará la oportunidad de profundizar en algunos elementos que no se tratarán aquí.

%Para ilustrar el cambio que hay en la concepción del lenguaje entre el \emph{Tractatus} e \emph{Investigaciones Filosóficas} podemos recurrir a algunas reflexiones de Wittgenstein sobre los fundamentos de las matemáticas hechas entre 1937 y 1938. Él se plantea la siguiente pregunta: \enquote*{¿Cómo sé que al calcular la serie $+2$ debo escribir `$20004$, $20006$' y no `$20004$, $20008$'?} La pregunta tiene que ver con el modo en el que actuamos según una regla. Al calcular esta serie se ha ofrecido $+2$ como norma para el cálculo. Ahora la pregunta es cómo se sabe qué hacer con ese conocimiento previo cuando llega el momento de ponerlo en acto. Si se ha comprendido la guía inicial se tendrá certeza sobre qué hacer después de $20004$, y esta certeza no implica que $20006$ haya quedado determinado de antemano, sino en que ante cualquier número ofrecido se tiene la capacidad de ofrecer el siguiente. Entonces continúa: \blockquote[Esta larga cita se ha tomado de la traducción al inglés realizada por Anscombe: {\cite[I, \S4]{wittgenstein1956remmath}}; una traducción española puede encontrarse en: {\cite[17-18]{wittgenstein1956remmathes}}.]{``¿Pero entonces en qué consiste la peculiar inexorabilidad de las matemáticas?''\,---\,¿No será acaso la inexorabilidad con la que dos sigue a uno y tres a dos un buen ejemplo?\,---\,Pero presuntamente esto significa: se sigue así en la \emph{serie de números cardinales}; pues en una serie distinta se seguiría de un modo distinto. Pero ¿acaso esta serie no está definida precisamente por esta secuencia?\,---\,``¿Hay que suponer que esto significa que cualquier modo en el que una persona cuente es igualmente correcto, y que cualquiera puede contar en el orden que quiera?''\,---\,Probablemente no lo llamaríamos `contar' si todo el mundo dijera los números uno después de otro \emph{de cualquier manera}; pero por supuesto esto no se trata simplemente de un problema sobre el nombre que se usa. Pues lo que llamamos `contar' es una parte importante de las actividades de nuestras vidas. Contar y calcular no son ---por ejemplo--- un simple pasatiempo. Contar (y eso significa: contar \emph{así}) es una técnica que es empleada diariamente en las operaciones más variadas de nuestras vidas. Y por eso es que aprendemos a contar como lo hacemos: con prácticas interminables, con despiadada exactitud; por eso es que es inexorablemente insistido que hemos de decir ``dos'' después de ``uno'', ``tres'' después de ``dos'' y así sucesivamente.\,---\,``Pero entonces este contar es sólo un \emph{uso}; ¿acaso no hay alguna verdad que se corresponda con esta secuencia?'' La \emph{verdad} es que contar ha demostrado que paga.\,---\,``Entonces quieres decir que `ser verdad' significa: ser utilizable (o útil)?''\,---\,No, no eso; pero que no puede ser dicho de la serie de números naturales\,---\,y tampoco de nuestro lenguaje\,---\,que es verdad, pero: que es utilizable, y, sobre todo que \emph{se usa de hecho}}. La discusión de \emph{Investigaciones Filosóficas} comienza con una cita de \emph{Confesiones} I,8 donde se encuentra una descripción de una imagen de la `esencia del lenguaje humano' que Wittgenstein considera que pertenece a la tradición que culminó en la teoría del \emph{Tractatus}. Allí la necesidad que le atribuimos a ciertas verdades y nuestra capacidad de reconocer esta necesidad a priori se explicó por la forma lógica común al pensamiento y la realidad y que queda expresada en el lenguaje. Sin embargo, esta tradición se equivocó al cuestionarse qué hace a estas verdades necesarias. La investigación adecuada parte de la pregunta sobre qué es que una proposición \emph{sea} necesaria y la respuesta se encuentra examinando y describiendo el papel que juegan estas proposiciones en las transacciones que hacemos con nuestro lenguaje\footnote{\cite[Cf.][242-243]{bakerhacker2014rules}: \enquote{Wittgenstein, when composing the early draft of the \emph{Investigations} in 1936/7, approached the task of mapping out this terrain from a unique vantage point\,---\,namely his elucidation of internal relations by reference to human practices of using signs. His examination of the concept of following a rule provides the background for clarifying the character of mathematical propositions, of what he called grammatical propositions and hence too of putative metaphysical propositions, and of the propositions of logic. He gave a detailed and comprehensive account of their peculiar status, an account which explains both why we conceive of them as necessary truths and what sense can be made of that conception. The questions of what makes them necessary (what is the source of their necessity) and how a priori knowledge of them is possible (how do we recognize them) lead us astray before we have begun. The prior question is: what is it for a proposition to \emph{be} a `necessary proposition', i.e. to be a proposition of mathematics, to be a logical proposition, or to be what Wittgenstein called a grammatical proposition? If this is answered by examining and properly describing the roles of such propositions in our linguistic transactions, the traditional questions can be resolved or dissolved. Wittgenstein's account is as bold as it is original}.}.

%Con la pregunta sobre cómo continuar la serie, Wittgenstein está cuestionando en qué consiste la necesidad matemática que rige la secuencia. Similarmente habla de la necesidad en relación con la gramática. Tras cuestionarse sobre el modo en que calculamos la serie, añade la observación: \enquote{la pregunta ``¿cómo sé que este color es `rojo'?'' es similar.} La cuestión planteada no solo tiene que ver con el modo en el que vamos según una serie, sino con las operaciones que hacemos con las palabras. También con las palabras hay una comprensión inicial de su uso que luego se aplica en cada caso. ¿Cómo sé que en esta ocasión estoy empleando una expresión según la regla que es su uso? Wittgenstein dirá que hay una relación entre necesidad, gramática y uso en la actividad humana que constituyen lo que podríamos considerar la esencia de las palabras.

%Esta manera de analizar en lenguaje tiene como consecuencia que no podemos pensar en los conceptos como entidades privadas en nuestro pensamiento. En \emph{Investigaciones Filosóficas} \S380 encontramos: \blockquote[{\Cite[\S380]{wittgenstein1953phiinv}}: \enquote{How do I recognize that this is red?\,---\,``I see that it is \emph{this}; and then I know that that is what is called.'' This?\,---\,What?! What kind of answer to this question makes sense? (You keep on steering towards an inner ostensive explanation.) I could not apply any rules to a \emph{private} transition from what is seen to words. Here the rules really would hang in the air; for the institution of their application is lacking}.]{¿Cómo reconozco que esto es rojo?\,---\,``Veo que es \emph{esto}; y entonces sé que eso es lo que esto es llamado'' ¿Esto?\,---\,¡¿Qué?! ¿Qué tipo de respuesta a esta pregunta tiene sentido? (Sigues girando hacia una explicación ostensiva interna.) No podría aplicar ninguna regla a una transición \emph{privada} desde lo que es visto a las palabras. Aquí las reglas realmente quedarían suspendidas en el aire; pues la institución para su aplicación está ausente}.

%Y añade en \S381: \blockquote[{\Cite[\S381]{wittgenstein1953phiinv}}: \enquote{How do I recognize that this colour is red?\,---\,One answer would be: ``I have learnt English.''}.]{¿Cómo reconozco que este color es rojo?\,---\,Una respuesta sería: ``He aprendido [español]''}. Ir según una regla es ir según una costumbre, un uso, una institución; \blockquote[{\Cite[\S199]{wittgenstein1953phiinv}}: \enquote{To understand a sentence means to understand a language. To understand a language means to have mastered a technique}.]{Entender una oración significa entender un lenguaje, entender un lenguaje significa dominar una técnica.} La gramática de la expresión `seguir una regla' supone la existencia de una práctica, una regularidad, un comportamiento normativo. Solo cuando esta red de comportamientos está en juego se puede hablar de que existe una regla\footnote{\cite[Cf.][p.~14]{bakerhacker2009understanding}: \enquote{The internal relation is forged by the existence of a practice, a regularity in applying the rule, and the normative behaviour (of justification, criticism, correction of mistakes, etc.) that surrounds the practice. Only when such complex forms of behaviour are in play does it make sense to speak of \emph{there being} a rule at all}.}. No es posible que haya una sola persona que en una sola ocasión `siguió una regla', esta consideración no es correspondiente con la gramática de la expresión\footnote{\cite[Cf.][\S199]{wittgenstein1953phiinv}: \enquote{Is what we call ``following a rule'' something that it would be possible for only \emph{one} person, only \emph{once} in a lifetime, to do?}.}.

Elizabeth también experimentó una especie de `corte' en su desarrollo filosófico cuando participó de las lecciones de Wittgenstein en Cambridge\footnote{Un estudio más detallado de la influencia de Wittgenstein en la filosofía de Anscombe se encuentra en: \Cite[37-95]{grimi2014dl}}. Allí encontró una perspectiva liberadora en la noción de que el significado de las palabras queda expresado en definitiva en el uso que hacemos de ellas: \blockquote[{\Cite[viii]{anscombe1981metaphysics}}: \enquote{At one point in these classes Wittgenstein was discussing the interpretation of the sign-post, and it burst upon me that the way you go by it is the final interpretation}.]{En cierto punto Wittgenstein estaba discutiendo en sus clases la interpretación del letrero (\emph{sign-post}), y comprendí de repente que el modo en que vas según este es la interpretación final}. Un letrero es una expresión de una regla ante la que hemos sido entrenados a reaccionar de un modo particular. Pensar que se está siguiendo una regla no es seguir una regla, y por eso no es posible seguir una regla `privadamente'\footnote{\cite[Cf.][\S202]{wittgenstein1953phiinv}: \enquote{That's why `following a rule' is a practice. And to \emph{think} one is following a rule is not to follow a rule. And that's why it's not possible to follow a rule `privately'; otherwise, thinking one was following a rule would be the same thing as following it}.}. La interpretación definitiva de una expresión de una regla es cómo se actúa ante ella.

Durante sus estudios en Oxford, Anscombe había rechazado con fuerza un realismo representativo lockeano que insistía que los colores como ella los veía no son parte del mundo externo. Como reacción contraria tendía a identificar estas sensaciones con \emph{esto} (\emph{this}), como si `azul' o `amarillo' fueran artículos que `están ahí'. Esta noción también le parecía equivocada, pero no lograba librarse de ella: \blockquote[{\Cite[viii]{anscombe1981metaphysics}}: \enquote{At another \textins{point} I came out with ``But I still want to say: Blue is there.'' Older hands smiled or laughed but Wittgenstein checked them by taking it seriously, saying ``Let me think what medicine you need\ldots Suppose that we had the word `painy' as a word for the property of some surfaces.'' The `medicine' was effective \textelp{} If ``painy'' were a possible secondary quality word, then wouldn't just the same motive drive me to say: ``Painy is there'' as drove me to say ``Blue is there''?}]{En otra \textins{ocasión} salí con: ``Pero todavía quiero decir: Azul esta ahí''. Manos más veteranas sonrieron o rieron, pero Wittgenstein las detuvo tomándolo en serio, diciendo: ``Déjame pensar qué medicina necesitas\ldots'' ``Supón que tenemos la palabra `\emph{painy}', como una palabra para la propiedad de ciertas superficies''. La `medicina' fue efectiva \textelp{} Si ``\emph{painy}'' fuera una palabra posible para una cualidad secundaria, ¿no podría el mismo motivo conducirme a decir: ``\emph{Painy} está ahí'' que lo que me condujo a decir ``Azul está ahí''?} La solución a la dificultad de Anscombe no consiste tampoco en identificar `azul' o `painy' con `esta sensación', sino precisamente en desligar estos conceptos tanto de `algo que está ahí', como de `esta sensación que tengo', el significado se encuentra en su uso: \blockquote[{\Cite[114]{anscombe1981parmenides:qli}}: \enquote{``You learned the \emph{concept} pain when you learned language.'' That is, it is not experiencing pain that gives you the meaning of the word ``pain''. How could an experience dictate the grammar of a word? \textelp{} doesn't it make certain demands on the grammar, if the word is to be the word for \emph{that} experience?}]{``Aprendimos el \emph{concepto} dolor cuando aprendimos el lenguaje.'' Esto es, no ha sido experimentar el dolor lo que nos ha dado el significado de la palabra ``dolor''. ¿Cómo podría una experiencia dictar la gramática de una palabra? \textelp{} ¿acaso no implica ciertas exigencias a la gramática, si la palabra tiene que ser la palabra de \emph{esa} experiencia?}

El cambio ocurrido en Anscombe al encontrarse con el método propuesto por Wittgenstein es representativo de uno de los problemas de la filosofía que él quiso resolver: \blockquote[{\Cite[Cf.][213]{diamond2004crisscross}}: \enquote{Before the `medicine', Anscombe's problem is one of philosphy's Big Questions. It is a form of the question how our thought is able to connect with reality. She is aware of, has in her mind, \emph{this}, the blue; is it or is it not \emph{there}, in the world?}]{Antes de la `medicina', el problema de Anscombe es una de las Grandes preguntas de la filosofía. Es una forma de la pregunta sobre cómo nuestro pensamiento tiene la capacidad de conectar con la realidad. Ella está consciente de, tiene en su mente, \emph{esto}, el azul; ¿está o no está \emph{ahí}, en el mundo?} La respuesta del \emph{Tractatus} pensó en esta como una conexión metafísica presente en el orden lógico que sostiene todo lenguaje posible. El trabajo del filósofo según esta concepción consiste en analizar las expresiones para sacar al descubierto el orden lógico que está debajo del lenguaje ordinario y que es la forma de la realidad. Ahora la ruta es distinta, en \emph{Investigaciones Filosóficas} exclama: \blockquote[{\Cite[\S107]{wittgenstein1953phiinv}}: \enquote{The more closely we examine actual language, the greater becomes the conflict between it and our requirement. (For the crystalline purity of logic was, of course, not something I had \emph{discovered}: it was a requirement.) The conflict becomes intolerable; the requirement is in danger of becoming vacuous.\,---\,We have got on to slippery ice where there is no friction, and so, in a certain sense, the conditions are ideal; but also, just because of that, we are unable to walk. We want to walk: so we need \emph{friction}. Back to the rough ground!}]{Cuanto más de cerca examinamos el lenguaje actual, más crece el conflicto entre este y nuestro requisito. (Pues la pureza cristalina de la lógica no era, por supuesto, algo que yo hubiera \emph{descubierto}: era un requisito.) El conflicto se hace intolerable; el requisito llega ahora a estar en peligro de tornarse vacuo.\,---\,Nos hemos situado en hielo resbaladizo donde no hay fricción, y así, en cierto sentido, las condiciones son ideales; pero también, justo por eso, no somos capaces de caminar. Queremos caminar: así que necesitamos \emph{fricción}. ¡De vuelta al terreno escarpado!} El análisis del lenguaje tiene que considerarlo integrado a la actividad de la vida humana. Ahí es donde el lenguaje está funcionando, está vivo, tiene `fricción'. En ese sentido, todo lo que necesitamos para entender el lenguaje está ante nosotros, a la vista, es nuestra manera de vivir\footnote{\cite[Cf.][48]{mcginn2013guide}: \enquote{Instead of approaching language as a system of signs with meaning, we are prompted to imagine it in situ, embedded in the lives of those who speak it. The tendency to isolate language, or abstract it from the context in which it ordinarily lives, is connected with our desire to say what the essence of language is, and with our urge to explain how these mere signs (mere marks) acquire their extraordinary power to mean or represent something. Wittgenstein’s aim is to show us that in this act of abstraction we turn our backs on everything that is essential to language’s signifying in the way that it does; it is our act of abstracting language from its employment within our ordinary lives that turns it into something dead, whose ability to represent now cries out for explanation. Thus, the sense of a need to explain how language (conceived as a system of signs) has the magical power to represent the world is connected with our failure to look at language where it is actually functioning. Wittgenstein does not set out to satisfy our sense of a need for a theory of representation (a theory that explains how the dead sign acquires meaning), but to dispel this sense of a need through getting us to look at language where it is actually doing work, and where we can see its essence fully displayed. In directing us, through the concept of a language-game, to ‘the spatial and temporal phenomenon of language, not [to] some non-spatial, atemporal non-entity’ (PI §108), Wittgenstein hopes gradually to bring us to see that ‘nothing extraordinary is involved’ (PI §94), that everything that we need to understand the essence of language ‘already lies open to view’ (PI §126)}.}.
Este enfoque hace de la filosofía de Anscombe una de especial interés para el estudio del testimonio. En el siguiente apartado exploraremos también cómo representa una oportunidad interesante para la investigación teológica.
