%!TEX program = xelatex
%!TEX TS-program = xelatex
%!TEX encoding = UTF-8 Unicode

%Base Document Class
\documentclass[
10pt,twoside,
%draft
final
]{report}

%Document Info
\usepackage{portadasandamaso}
\title{La Categoría del Testimonio en el pensamiento de Elizabeth
    Anscombe.\space Valoración y crítica en perspectiva teológico-fundamental.}
\author{Joel E. Del Cueto}
\supervisor{Prof. Javier Prades}
\project{Tesina de Licenciatura}
\faculty{Facultad de Teología}
\date{2019}

%PACKAGES
%subfiles to include the chapters
\usepackage{subfiles}
%csquotes for advanced facilities for inline and display quotations
\usepackage[autostyle,spanish=spanish,maxlevel=2]{csquotes}

%\SetCiteCommand{\footcite}
\SetBlockEnvironment{quotation}
%\renewcommand{\mkcitation}[1]{\footnote{#1}}

\renewcommand{\mkcitation}[1]{\footnote{#1}}
\renewcommand{\mktextquote}[6]{#1#2#4#3#5#6}
\renewcommand{\mkblockquote}[4]{#1#2#3#4}


\SetBlockThreshold{3}


%\makeatletter
%\DeclareQuoteStyle{threelevel}
%    {\ifnum\csq@qlevel<2 \guillemotleft\else\textquotedblleft\fi}% opening outer mark
%    {\ifnum\csq@qlevel<2 \guillemotright\else\textquotedblright\fi}% closing outer mark
%    {\textquoteleft}% opening inner mark
%    {\textquoteright}% closing inner mark
%\makeatother

%\ExecuteQuoteOptions{style=threelevel}

%hyperref to handle cross-referencing commands
\usepackage[hyperfootnotes=false]{hyperref}
%ifdraft for notes and todos activation
\usepackage{ifdraft}
%verbatim for multiline comments
\usepackage{verbatim}
%Easy Lists
\usepackage[ampersand]{easylist}


% STYLE
%Load ./style.sty
\usepackage{style}

%BIBLIOGRAPHY
%OLD
%bibstyle=./bibliography/authortitle-custom,
%citestyle=./bibliography/verbose-ibid-custom,
%refsection=section,
%useprefix=true,
%block=none,
%giveninits=true,
%dashed=true,
%labeldate=false,
%sorting=nyvt,
%autocite=footnote,
%isbn=false,
%url=false

\usepackage[
backend=biber,
%bibstyle=./bibliography/authortitle-custom,
%citestyle=./bibliography/verbose-ibid-custom,
%style=authortitle-dw,
style=authortitle,
%xref=true,
%firstfull=true,
giveninits=true,
%idembibformat=dash,
%namefont=smallcaps,
autocite=footnote,
isbn=false,
url=false,
%edsuper=true,
]{biblatex}

\DefineBibliographyStrings{spanish}{%
    andothers = {et al.},
    in = {\lowercase{e}n:},
    editor = {(\lowercase{e}d.),},
    editors = {(\lowercase{e}ds.),},
}

%Biblatex using biber engine to manage the bibliography
%\usepackage[
%backend=biber,
%style=verbose,
%citestyle=verbose
%]{biblatex}
%The bibliography database
\addbibresource{../bibliography/bibliodb/primary.bib}
\addbibresource{../bibliography/bibliodb/secondary.bib}

\begin{document}

%Limit hyphenation to long words
%\lefthyphenmin=4
\righthyphenmin=3

%Title Page
\ifdraft{
    \begin{center}
        \large{
        La Categoría del Testimonio en el pensamiento de Elizabeth Anscombe.\\
        Valoración y crítica en perspectiva teológico-fundamental.\\
        \emph{--//DRAFT//--}}
    \end{center}
}
{
%    \begin{titlepage}

\huge La Categoría del Testimonio en el pensamiento de Elizabeth Anscombe.\\
\large Un estudio en perspectiva teológico-fundamental.

\end{titlepage}
%\blankpage
%\afterpage{\blankpage}
%\maketitle
}

%Dedicatoria
%\ifdraft{}{
%\subfile{./dedicatoria}
%\afterpage{\blankpage}
%}
%Table of Contents
%\ifdraft{}{
    \tableofcontents
%}
% OLD   Outline
%    \subfile{./outline}
%    Intro
%\ifdraft{}{
%    \subfile{./intro}
%}
%Capítulo 1:
%\ifdraft{}{
%    \subfile{./ch1}
%}
%Capítulo 2:
%\ifdraft{}{
%    \subfile{./ch2}
%}
%Capítulo 3:
%\ifdraft{}{
%    \subfile{./ch3}
%}

%Capítulo 4:
%\ifdraft{}{
%    \subfile{./ch4}
%}

%Capítulo 5:
%\ifdraft{}{
%    \subfile{./ch5}
%}


%CAPÍTULO 1:
%Introducción General al Problema del Testimonio
\subfile{./ch1}
%CAPÍTULO 2:
%Vida y Pensamiento Propio de Elizabeth Anscombe
%\subfile{./ch2}
%CAPÍTULO 3:
%La concepción de Elizabeth Anscombe sobre el Testimonio
\subfile{./ch3}
%CAPÍTULO 4:
%Conclusión General
%\subfile{./ch4}


%Bibliography
%\clearpage
%\addcontentsline{toc}{chapter}{Bibliography}
\subfile{./bibliography}

\end{document}

%%% Local Variables:
%%% coding: utf-8
%%% mode: latex
%%% TeX-engine: xetex
%%% End:
