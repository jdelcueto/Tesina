\documentclass[../main.tex]{subfiles}
\begin{document}



\chapter*{La Categoría del Testimonio en el Pensamiento de Elizabeth Anscombe}


En una ocasión Wittgenstein recibió a Anscombe con la pregunta: <<¿Por qué la gente dice que era natural pensar que el sol giraba alrededor de la tierra en lugar de que la tierra giraba en su eje?>> Elizabeth contestó: <<Supongo que porque se veía como si el sol girara alrededor de la tierra.>> <<Bueno\ldots>>, añadió Wittgenstein, <<¿cómo se hubiera visto si se hubiera \emph{visto} como si la tierra girara en su propio eje?>> A esta pregunta Anscombe reaccionó extendiendo las manos delante de ella con las palmas hacia arriba y, levantándolas desde sus rodillas con un movimiento circular, se inclinó hacia atrás asumiendo una expresión de mareo. <<¡Exactamente!>> exclamó Wittgenstein.\footcite[151]{IWT}

Anscombe estaba familiarizada con este método de análisis de las proposiciones; Wittgenstein buscaba mostrar que ella no había provisto significado o referencia para ciertos signos en su afirmación y éste es el propósito de su segunda pregunta. Al cuestionar ``¿cómo se hubiera visto como si la tierra girara en su propio eje?'' sale a relucir que hasta aquél momento Anscombe no había ofrecido ningún significado relevante para su expresión ``se veía como si'' en su respuesta ``se veía como si el sol girara alrededor de la tierra''. 

Este modo de criticar una oración desvelando que no expresa un pensamiento verdadero refleja los principios propuestos en el \emph{Tractatus}


\end{document}