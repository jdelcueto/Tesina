\documentclass[./main.tex]{subfiles}
\begin{document}
\setcounter{chapter}{2}

% \setcounter{chapter}{2}
%\chapter{Desarrollo filosófico de Elizabeth Anscombe}

%ESTE CAPÍTULO SE OFRECE COMO UN TESTIMONIO DE LA VIDA, EL PENSAMIENTO Y LA
%FILOSOFÍA DE ANSCOMBE
%¿Cómo Elizabeth Anscombe hizo filosofía?
%Peter Geach tenía estas palabras que
%decir sobre su esposa:\citalitlar{Como una filósofa madura, Elizabeth me parece
%    ser una pensadora más intrépida que yo: es ella quien tiene ideas audaces y
%    que a primera vista resultan meramente alocadas, a lo que en ocasiones he
%    reaccionado con inicial indignación. (Cfr. sus escritos \emph{The
%        Intentionality of Sensation} y \emph{The Fisrt Person}) Usualmente llego
%    a ver como estas audaces ideas son más justificables de lo que originalmente
%    suponía. \autocite[p.~11]{philaut}}

%    Bernard Williams, cuya relación intelectual con Anscombe no estuvo libre de
%    fricciones, hablando sobre su experiencia en Oxford en los años 50,
%    dijo\autocite[p.~228]{teichmann}: \citalitlar{Otra persona que tuvo un tipo
%    de influencia sobre mi ---¡aunque me alegra decir que pienso que no me ha
%    influido de otros modos!--- fue Elizabeth Anscombe. Una cosa que hacía, que
%    sacó de Wittgenstein, era que imprimía sobre uno que el ser ingenioso no
%    era suficiente. La filosofía de Oxford, y esto todavía es cierto hasta
%    cierto punto, tenía una gran tendencia a ser ingeniosa. Era muy erística:
%    había
%    mucho intercambio dialéctico competitivo, y mucho demostrar que los otros
%    estaban equivocados. Yo era muy bueno en todo eso. Pero Elizabeth transmitía
%    un fuerte sentido de seriedad al tema, y cómo éste era difícil en modos para
%    los que ser ingenioso simplemente no era suficiente.\autocite[Bernard
%    Williams en entrevista con el Harvard Review of philosophy, 2004]{ref}}

%Roger Teichmann propone también: \citalitlar{Hay otra razón para la falta de
%    aparente sistematicidad en los escritos de Anscombe, y esto es que su
%    propósito al escribir era típicamente llegar a algún sitiio en sus propios
%    pensamientos sobre algún tema; usalmente dedica poco o nigún tiempo en
%    proveer un trasfondo, o en justificar su principales `presupuestos',
%    prefiriendo empezar \emph{in media res}.\autocite[p.~1]{teichmann}}

%Anscombe protestó públicamente la guerra y también la legalización del aborto.

%Y añade: \citalitlar{Un modo en el que Anscombe se diferencia considerablemente
%    de Wittgenstein es en su actitud hacia los males político y sociales.
%    [...]Ella era en totalmente una de esas personas que se ganan el epíteto de
%    `ser francos', y sus amigos más indulgentes tendrían que admitir que sus
%    modos podían en ocasiones ser inquietantemente
%    bruscos.\autocite[p.4]{teichmann}}

%Describiremos su quehacer filosófico en dos pasos: atendiendo el impacto que
%Wittgenstein causa en la filosofía con el Tractatus y luego atendiendo las
%cuestiones filosóficas que Anscombe confronta.

%SECCIÓN 1: ANSCOMBE Y WITTGENSTEIN
%\section{Filosofía analítica y método filosófico de Elizabeth Anscombe}

%\subsection{Dos `cortes' en la filosofía.}

Para comprender el método filosófico de Anscombe hay que tener en cuenta algunas nociones básicas del método de filosofía analítica empleado por L. Wittgenstein. Un elemento de esta metodología que fue importante para la madurez filosófica de Elizabeth y que constituye además una de las constantes importantes del pensamiento de Wittgenstein fue su definición de la naturaleza de los problemas filosóficos. Para él las cuestiones de la filosofía no son problemáticas por ser erróneas, sino por no tener significado\footcite[Cf.][\S4.003]{wittgenstein1922tractatuses}. Una proposición sin significado que no es puesta al descubierto como tal atrapa al filósofo dentro de una confusión del lenguaje que no le permite acceder a la realidad. Salir de la confusión no consiste en refutar una doctrina y plantear una teoría alternativa, sino en examinar las operaciones hechas con las palabras para llegar a manejar una visión clara del empleo de nuestras expresiones. La filosofía no es un cuerpo doctrinal, sino una actividad\footcite[Cf.][\S4.112]{wittgenstein1922tractatuses} y una terapia\footcite[Cf.][\S133]{wittgenstein1953phiinv}.

La actitud terapéutica adoptada por Wittgenstein en su atención de las confusiones filosóficas fue su respuesta más definitiva a la naturaleza de estos problemas. Para ello encontró los más eficaces remedios en sus investigaciones sobre el significado y el sentido del lenguaje.
%Ordinariamente tomamos parte en esta actividad humana que es el lenguaje. Jugamos el juego del lenguaje. ---¿Jugarlo es entenderlo?--- A la vista de Wittgenstein saltaban extraños problemas sobre las reglas de este juego; entonces no podía evitar escudriñarlas al detalle\footcite[Cf.][356]{monk1991duty}. En este análisis del lenguaje está la raíz de sus ideas sobre el sentido, el significado y la verdad.
Durante su vida sostuvo dos grandes descripciones del significado. Originalmente describió el lenguaje como una imagen que representa el posible estado de las cosas en el mundo. En una segunda etapa se distanció de esta analogía para describir al lenguaje como una herramienta cuyo significado consiste en la suma de las múltiples semejanzas familiares que aparecen en los distintos usos para los cuales el lenguaje es empleado en la actividad humana. Dentro de la primera descripción una expresión sin significado es una cuyos elementos no componen una representación del posible estado de las cosas. Dentro de la segunda descripción una expresión sin significado resulta del empleo de una expresión propia de un ``juego del lenguaje'' fuera de su contexto.

Estas dos etapas del pensamiento de Wittgenstein son representadas por dos importantes tratados. El \emph{'Tractatus Logico-Philosophicus'}, publicado en 1921, recoge sus esfuerzos por elaborar un gran tratado filosófico comenzados en 1911 y culminados durante la Primera Guerra Mundial. El segundo, \emph{'Philosophische Untersuchungen'}, o \emph{'Investigaciones Filosóficas'}, traducido por Anscombe y publicado póstumamente en 1953, fue elaborado a partir de múltiples manuscritos desarrollados por Wittgenstein desde su regreso a Cambridge en 1929 hasta su muerte en 1951.

\blockquote[{\Cite[181]{anscombe2011plato:twocuts}}: \enquote{Wittgenstein is extraordinary among philosophers for having made two epochs, or cuts, in the history of philosophy}.]{Wittgenstein es extraordinario entre los filósofos por haber generado dos épocas, o cortes, en la historia de la filosofía}. Con estas palabras Anscombe comenzaría su discurso inaugural para el Sexto Simposio Internacional de Wittgenstein unos treinta años después de la publicación de las \emph{'Investigaciones Filosóficas'}. Y explica: \blockquote[{\Cite[181]{anscombe2011plato:twocuts}}: \enquote{a philosopher makes a cut if he makes a difference to the way philosophy is done: philosophy after the cut cannot be the same as before}.]{un filósofo hace un corte si genera un cambio en el modo en que la filosofía es hecha: la filosofía tras el corte no puede ser la misma de antes}.

Estos cambios de época generados por la influencia de Wittgenstein vinieron caracterizados por el esfuerzo de comprender cada libro tras su publicación, tarea complicada en ambos casos por la dificultad intrínseca de los tratados, ofuscada a su vez por los prejuicios filosóficos proyectados a cada obra por sus lectores\footnote{\cite[Cf.][183]{anscombe2011plato:twocuts}: \enquote{the assumption that the \emph{Philosophical Investigations} presents us a theory of language ---a theory, say, of how sounds become significant speech--- will quickly place us at a distance from the very questions which Wittgenstein is occupied with}.}.
%Elizabeth explica que: \blockquote[{\Cite[Cf.][183]{anscombe2011plato:twocuts}}: \enquote{the assumption that the \emph{Philosophical Investigations} presents us a theory of language ---a theory, say, of how sounds become significant speech--- will quickly place us at a distance from the very questions which Wittgenstein is occupied with}.]{la presunción, por ejemplo, de que \emph{'Investigaciones Filosóficas'} presenta una teoría del lenguaje ---quizás sobre cómo los sonidos se tornan en discursos significativos--- nos dejaría situados lejos de las preguntas que genuinamente ocupan a Wittgenstein}.
Ahora bien, la comprensión adecuada de su pensamiento y método trae consigo, a juicio de Anscombe, cierto efecto curativo.

%Según Anscombe el método general adecuado de discutir los problemas filosóficos propuesto por Wittgenstein consiste en mostrar que la persona no ha provisto significado (o referencia) para ciertos signos en sus expresiones\footnote{\cite[Cf.][151]{anscombe1959iwt}: \enquote{The general method that Wittgenstein does suggest is that of `shewing that a man has supplied no meaning [or perhaps: ``no reference''] for certain signs in his sentences'}.}. Creía que el camino que lleva a formular estos problemas está frecuentemente trazado por la mala comprensión de la lógica de nuestro lenguaje.

%Cada obra de Wittgenstein representa su esfuerzo de superar estas confusiones y propone un método para remediarlas. Su primera propuesta plantea que el modo de aclarar las confusiones de los problemas filosóficos consiste en identificar en el lenguaje el límite de lo que expresa pensamiento; lo que queda al otro lado de esta frontera sería simplemente sinsentido. En otras palabras: \blockquote[{\Cite[11]{wittgenstein1922tractatuses}}.]{lo que siquiera puede ser dicho, puede ser dicho claramente; y de lo que no se puede hablar, hay que callar}. Con esta expresión Wittgenstein resumió el sentido de la obra que ahora examinaremos.

%\subsection{Las elucidaciones del \emph{Tractatus}}
% Este párrafo resume los cuatro puntos del Tractatus que se desglosarán en los próximos párrafos
%Desde las proposiciones principales del \emph{Tractatus} queda claro que el tema central del libro es la conexión entre el lenguaje, o el pensamiento, y la realidad.
% 1.Filosofía como actividad
%En este nexo es donde la actividad filosófica ha de buscar esclarecer el pensamiento.
% 2.El pensamiento como representación
%La tesis básica sobre esta relación consiste en que las proposiciones, o su equivalente en la mente, son imágenes de los hechos.
% 3.Las proposiciones como proyecciones con polos de verdad-falsedad
%La proposición es la misma imagen tanto si es cierta como si es falsa, es decir, es la misma imagen sin importar que lo que se corresponde a esta es el caso que es cierto o no. El mundo es la totalidad de los hechos, a saber, de lo equivalente en la realidad a las proposiciones verdaderas.
% 4.La distinción entre el decir y el mostrar
%Solo las situaciones que pueden ser plasmadas en imágenes pueden ser afirmadas en proposiciones. Adicionalmente hay mucho que es inexpresable, lo cual no debemos intentar enunciar, sino más bien contemplar sin palabras\footnote{\cite[Cf.][19]{anscombe1959iwt}: \enquote{There is indeed much that is inexpressible --- which we must not try to state, but must contemplate without words}.}.

%\subsection{\emph{Investigaciones Filosóficas} y el nuevo método de Wittgenstein}

%Anscombe conoció a Wittgenstein en la segunda etapa de su pensamiento, y trabajó con él para traducir \emph{Investigaciones Filosóficas}, así que hemos de atribuir a esta etapa tardía la mayor influencia en el pensamiento de Elizabeth. Sin embargo una de las discusiones más amplias del pensamiento de Wittgenstein en la obra de Anscombe se encuentra en \emph{An Introduction to Wittgenstein's Tractatus}. El mismo Wittgenstein reiteró que su pensamiento tardío solo puede entenderse a la luz del \emph{Tractatus}, sin embargo esto no terminaría de explicar el interés de Anscombe en esa obra. Quizás es más correcto decir que el \emph{Tractatus}, con su énfasis en el tema de la verdad, no dejó de ser una reflexión con mérito para Elizabeth como complemento de la atención que presta \emph{Investigaciones Filosóficas} al tema del sentido\footcite[Cf.][191-193]{teichmann2008ans}.

%\emph{Tractatus} queda claro que el tema central del libro es la conexión entre el lenguaje, o el pensamiento, y la realidad. En este nexo es donde la actividad filosófica ha de buscar esclarecer el pensamiento. La tesis básica sobre esta relación consiste en que las proposiciones, o su equivalente en la mente, son imágenes de los hechos. La proposición es la misma imagen tanto si es cierta como si es falsa, es decir, es la misma imagen sin importar que lo que se corresponde a esta es el caso que es cierto o no. El mundo es la totalidad de los hechos, a saber, de lo equivalente en la realidad a las proposiciones verdaderas. Solo las situaciones que pueden ser plasmadas en imágenes pueden ser afirmadas en proposiciones. Adicionalmente hay mucho que es inexpresable, lo cual no debemos intentar enunciar, sino más bien contemplar sin palabras\footnote{\cite[Cf.][19]{anscombe1959iwt}: \enquote{There is indeed much that is inexpressible --- which we must not try to state, but must contemplate without words}.}.

%En este apartado veremos algunos aspectos de las discusiones de Wittgenstein en esta segunda obra. La descripción será más general que la del \emph{Tractatus} ya que el análisis de los artículos de Anscombe en el capítulo siguiente nos dará la oportunidad de profundizar en algunos elementos que no se tratarán aquí.

%Para ilustrar el cambio que hay en la concepción del lenguaje entre el \emph{Tractatus} e \emph{Investigaciones Filosóficas} podemos recurrir a algunas reflexiones de Wittgenstein sobre los fundamentos de las matemáticas hechas entre 1937 y 1938. Él se plantea la siguiente pregunta: \enquote*{¿Cómo sé que al calcular la serie $+2$ debo escribir `$20004$, $20006$' y no `$20004$, $20008$'?} La pregunta tiene que ver con el modo en el que actuamos según una regla. Al calcular esta serie se ha ofrecido $+2$ como norma para el cálculo. Ahora la pregunta es cómo se sabe qué hacer con ese conocimiento previo cuando llega el momento de ponerlo en acto. Si se ha comprendido la guía inicial se tendrá certeza sobre qué hacer después de $20004$, y esta certeza no implica que $20006$ haya quedado determinado de antemano, sino en que ante cualquier número ofrecido se tiene la capacidad de ofrecer el siguiente. Entonces continúa: \blockquote[Esta larga cita se ha tomado de la traducción al inglés realizada por Anscombe: {\cite[I, \S4]{wittgenstein1956remmath}}; una traducción española puede encontrarse en: {\cite[17-18]{wittgenstein1956remmathes}}.]{``¿Pero entonces en qué consiste la peculiar inexorabilidad de las matemáticas?''\,---\,¿No será acaso la inexorabilidad con la que dos sigue a uno y tres a dos un buen ejemplo?\,---\,Pero presuntamente esto significa: se sigue así en la \emph{serie de números cardinales}; pues en una serie distinta se seguiría de un modo distinto. Pero ¿acaso esta serie no está definida precisamente por esta secuencia?\,---\,``¿Hay que suponer que esto significa que cualquier modo en el que una persona cuente es igualmente correcto, y que cualquiera puede contar en el orden que quiera?''\,---\,Probablemente no lo llamaríamos `contar' si todo el mundo dijera los números uno después de otro \emph{de cualquier manera}; pero por supuesto esto no se trata simplemente de un problema sobre el nombre que se usa. Pues lo que llamamos `contar' es una parte importante de las actividades de nuestras vidas. Contar y calcular no son ---por ejemplo--- un simple pasatiempo. Contar (y eso significa: contar \emph{así}) es una técnica que es empleada diariamente en las operaciones más variadas de nuestras vidas. Y por eso es que aprendemos a contar como lo hacemos: con prácticas interminables, con despiadada exactitud; por eso es que es inexorablemente insistido que hemos de decir ``dos'' después de ``uno'', ``tres'' después de ``dos'' y así sucesivamente.\,---\,``Pero entonces este contar es sólo un \emph{uso}; ¿acaso no hay alguna verdad que se corresponda con esta secuencia?'' La \emph{verdad} es que contar ha demostrado que paga.\,---\,``Entonces quieres decir que `ser verdad' significa: ser utilizable (o útil)?''\,---\,No, no eso; pero que no puede ser dicho de la serie de números naturales\,---\,y tampoco de nuestro lenguaje\,---\,que es verdad, pero: que es utilizable, y, sobre todo que \emph{se usa de hecho}}. La discusión de \emph{Investigaciones Filosóficas} comienza con una cita de \emph{Confesiones} I,8 donde se encuentra una descripción de una imagen de la `esencia del lenguaje humano' que Wittgenstein considera que pertenece a la tradición que culminó en la teoría del \emph{Tractatus}. Allí la necesidad que le atribuimos a ciertas verdades y nuestra capacidad de reconocer esta necesidad a priori se explicó por la forma lógica común al pensamiento y la realidad y que queda expresada en el lenguaje. Sin embargo, esta tradición se equivocó al cuestionarse qué hace a estas verdades necesarias. La investigación adecuada parte de la pregunta sobre qué es que una proposición \emph{sea} necesaria y la respuesta se encuentra examinando y describiendo el papel que juegan estas proposiciones en las transacciones que hacemos con nuestro lenguaje\footnote{\cite[Cf.][242-243]{bakerhacker2014rules}: \enquote{Wittgenstein, when composing the early draft of the \emph{Investigations} in 1936/7, approached the task of mapping out this terrain from a unique vantage point\,---\,namely his elucidation of internal relations by reference to human practices of using signs. His examination of the concept of following a rule provides the background for clarifying the character of mathematical propositions, of what he called grammatical propositions and hence too of putative metaphysical propositions, and of the propositions of logic. He gave a detailed and comprehensive account of their peculiar status, an account which explains both why we conceive of them as necessary truths and what sense can be made of that conception. The questions of what makes them necessary (what is the source of their necessity) and how a priori knowledge of them is possible (how do we recognize them) lead us astray before we have begun. The prior question is: what is it for a proposition to \emph{be} a `necessary proposition', i.e. to be a proposition of mathematics, to be a logical proposition, or to be what Wittgenstein called a grammatical proposition? If this is answered by examining and properly describing the roles of such propositions in our linguistic transactions, the traditional questions can be resolved or dissolved. Wittgenstein's account is as bold as it is original}.}.

%Con la pregunta sobre cómo continuar la serie, Wittgenstein está cuestionando en qué consiste la necesidad matemática que rige la secuencia. Similarmente habla de la necesidad en relación con la gramática. Tras cuestionarse sobre el modo en que calculamos la serie, añade la observación: \enquote{la pregunta ``¿cómo sé que este color es `rojo'?'' es similar.} La cuestión planteada no solo tiene que ver con el modo en el que vamos según una serie, sino con las operaciones que hacemos con las palabras. También con las palabras hay una comprensión inicial de su uso que luego se aplica en cada caso. ¿Cómo sé que en esta ocasión estoy empleando una expresión según la regla que es su uso? Wittgenstein dirá que hay una relación entre necesidad, gramática y uso en la actividad humana que constituyen lo que podríamos considerar la esencia de las palabras.

%Esta manera de analizar en lenguaje tiene como consecuencia que no podemos pensar en los conceptos como entidades privadas en nuestro pensamiento. En \emph{Investigaciones Filosóficas} \S380 encontramos: \blockquote[{\Cite[\S380]{wittgenstein1953phiinv}}: \enquote{How do I recognize that this is red?\,---\,``I see that it is \emph{this}; and then I know that that is what is called.'' This?\,---\,What?! What kind of answer to this question makes sense? (You keep on steering towards an inner ostensive explanation.) I could not apply any rules to a \emph{private} transition from what is seen to words. Here the rules really would hang in the air; for the institution of their application is lacking}.]{¿Cómo reconozco que esto es rojo?\,---\,``Veo que es \emph{esto}; y entonces sé que eso es lo que esto es llamado'' ¿Esto?\,---\,¡¿Qué?! ¿Qué tipo de respuesta a esta pregunta tiene sentido? (Sigues girando hacia una explicación ostensiva interna.) No podría aplicar ninguna regla a una transición \emph{privada} desde lo que es visto a las palabras. Aquí las reglas realmente quedarían suspendidas en el aire; pues la institución para su aplicación está ausente}.

%Y añade en \S381: \blockquote[{\Cite[\S381]{wittgenstein1953phiinv}}: \enquote{How do I recognize that this colour is red?\,---\,One answer would be: ``I have learnt English.''}.]{¿Cómo reconozco que este color es rojo?\,---\,Una respuesta sería: ``He aprendido [español]''}. Ir según una regla es ir según una costumbre, un uso, una institución; \blockquote[{\Cite[\S199]{wittgenstein1953phiinv}}: \enquote{To understand a sentence means to understand a language. To understand a language means to have mastered a technique}.]{Entender una oración significa entender un lenguaje, entender un lenguaje significa dominar una técnica.} La gramática de la expresión `seguir una regla' supone la existencia de una práctica, una regularidad, un comportamiento normativo. Solo cuando esta red de comportamientos está en juego se puede hablar de que existe una regla\footnote{\cite[Cf.][p.~14]{bakerhacker2009understanding}: \enquote{The internal relation is forged by the existence of a practice, a regularity in applying the rule, and the normative behaviour (of justification, criticism, correction of mistakes, etc.) that surrounds the practice. Only when such complex forms of behaviour are in play does it make sense to speak of \emph{there being} a rule at all}.}. No es posible que haya una sola persona que en una sola ocasión `siguió una regla', esta consideración no es correspondiente con la gramática de la expresión\footnote{\cite[Cf.][\S199]{wittgenstein1953phiinv}: \enquote{Is what we call ``following a rule'' something that it would be possible for only \emph{one} person, only \emph{once} in a lifetime, to do?}.}.

Elizabeth también experimentó una especie de `corte' en su desarrollo filosófico cuando participó de las lecciones de Wittgenstein en Cambridge. Allí encontró una perspectiva liberadora en la noción de que el significado de las palabras queda expresado en definitiva en el uso que hacemos de ellas: \blockquote[{\Cite[viii]{anscombe1981metaphysics}}: \enquote{At one point in these classes Wittgenstein was discussing the interpretation of the sign-post, and it burst upon me that the way you go by it is the final interpretation}.]{En cierto punto Wittgenstein estaba discutiendo en sus clases la interpretación del letrero (sign-post), y estalló en mi que el modo en que vas según este es la interpretación final}. Un letrero es una expresión de una regla ante la que hemos sido entrenados a reaccionar de un modo particular. Pensar que se está siguiendo una regla no es seguir una regla, y por eso no es posible seguir una regla `privadamente'\footnote{\cite[Cf.][\S202]{wittgenstein1953phiinv}: \enquote{That's why `following a rule' is a practice. And to \emph{think} one is following a rule is not to follow a rule. And that's why it's not possible to follow a rule `privately'; otherwise, thinking one was following a rule would be the same thing as following it}.}. La interpretación definitiva de una expresión de una regla es cómo se actúa ante ella.

Durante sus estudios en Oxford, Anscombe había rechazado con fuerza un realismo representativo lockeano que insistía que los colores como ella los veía no son parte del mundo externo. Como reacción contraria tendía a identificar estas sensaciones con \emph{esto} (this), como si `azul' o `amarillo' fueran artículos que `están ahí'. Esta noción también le parecía equivocada, pero no lograba librarse de ella: \blockquote[{\Cite[viii]{anscombe1981metaphysics}}: \enquote{At another \textins{point} I came out with ``But I still want to say: Blue is there.'' Older hands smiled or laughed but Wittgenstein checked them by taking it seriously, saying ``Let me think what medicine you need\ldots Suppose that we had the word `painy' as a word for the property of some surfaces.'' The `medicine' was effective \textelp{} If ``painy'' were a possible secondary quality word, then wouldn't just the same motive drive me to say: ``Painy is there'' as drove me to say ``Blue is there''?}]{En otra \textins{ocasión} salí con: ``Pero todavía quiero decir: Azul esta ahí''. Manos más veteranas sonrieron o rieron, pero Wittgenstein las detuvo tomándolo en serio, diciendo: ``Déjame pensar qué medicina necesitas\ldots'' ``Supón que tenemos la palabra `\emph{painy}', como una palabra para la propiedad de ciertas superficies''. La `medicina' fue efectiva \textelp{} Si ``\emph{painy}'' fuera una palabra posible para una cualidad secundaria, ¿no podría el mismo motivo conducirme a decir: ``\emph{Painy} está ahí'' que lo que me condujo a decir ``Azul está ahí''?} La solución a la dificultad de Anscombe no consiste tampoco en identificar `azul' o `painy' con `esta sensación', sino precisamente en desligar estos conceptos tanto de `algo que está ahí', como de `esta sensación que tengo', el significado se encuentra en su uso: \blockquote[{\Cite[114]{anscombe1981parmenides:qli}}: \enquote{``You learned the \emph{concept} pain when you learned language.'' That is, it is not experiencing pain that gives you the meaning of the word ``pain''. How could an experience dictate the grammar of a word? \textelp{} doesn't it make certain demands on the grammar, if the word is to be the word for \emph{that} experience?}]{``Aprendimos el \emph{concepto} dolor cuando aprendimos el lenguaje.'' Esto es, no ha sido experimentar el dolor lo que nos ha dado el significado de la palabra ``dolor''. ¿Cómo podría una experiencia dictar la gramática de una palabra? \textelp{} ¿acaso no implica ciertas exigencias a la gramática, si la palabra tiene que ser la palabra de \emph{esa} experiencia?}

El cambio ocurrido en Anscombe al encontrarse con el método propuesto por Wittgenstein es representativo del problema de la filosofía que él quiso resolver: \blockquote[{\Cite[Cf.][213]{diamond2004crisscross}}: \enquote{Before the `medicine', Anscombe's problem is one of philosphy's Big Questions. It is a form of the question how our thought is able to connect with reality. She is aware of, has in her mind, \emph{this}, the blue; is it or is it not \emph{there}, in the world?}]{Antes de la `medicina', el problema de Anscombe es una de las Grandes Preguntas de la filosofía. Es una forma de la pregunta sobre cómo nuestro pensamiento tiene la capacidad de conectar con la realidad. Ella está consciente de, tiene en su mente, \emph{esto}, el azul; ¿está o no está \emph{ahí}, en el mundo?} La respuesta del \emph{Tractatus} pensó en esta como una conexión metafísica presente en el orden lógico que sostiene todo lenguaje posible. El trabajo del filósofo según esta concepción consiste en analizar las expresiones para sacar al descubierto el orden lógico que está debajo del lenguaje ordinario y que es la forma de la realidad. Ahora la ruta es distinta, en \emph{Investigaciones Filosóficas} exclama: \blockquote[{\Cite[\S107]{wittgenstein1953phiinv}}: \enquote{The more closely we examine actual language, the greater becomes the conflict between it and our requirement. (For the crystalline purity of logic was, of course, not something I had \emph{discovered}: it was a requirement.) The conflict becomes intolerable; the requirement is in danger of becoming vacuous.\,---\,We have got on to slippery ice where there is no friction, and so, in a certain sense, the conditions are ideal; but also, just because of that, we are unable to walk. We want to walk: so we need \emph{friction}. Back to the rough ground!}]{Cuanto más de cerca examinamos el lenguaje actual, más crece el conflicto entre éste y nuestro requisito. (Pues la pureza cristalina de la lógica no era, por supuesto, algo que yo hubiera \emph{descubierto}: era un requisito.) El conflicto se hace intolerable; el requisito llega ahora a estar en peligro de tornarse vacuo.\,---\,Nos hemos situado en hielo resbaladizo donde no hay fricción, y así, en cierto sentido, las condiciones son ideales; pero también, justo por eso, no somos capaces de caminar. Queremos caminar: así que necesitamos \emph{fricción}. ¡De vuelta al terreno escarpado!} El análisis del lenguaje tiene que considerarlo integrado a la actividad de la vida humana. Ahí es donde el lenguaje está funcionando, está vivo, tiene `fricción'. En ese sentido, todo lo que necesitamos para entender el lenguaje está ante nosotros, a la vista, es nuestra manera de vivir\footnote{\cite[Cf.][48]{mcginn2013guide}: \enquote{Instead of approaching language as a system of signs with meaning, we are prompted to imagine it in situ, embedded in the lives of those who speak it. The tendency to isolate language, or abstract it from the context in which it ordinarily lives, is connected with our desire to say what the essence of language is, and with our urge to explain how these mere signs (mere marks) acquire their extraordinary power to mean or represent something. Wittgenstein’s aim is to show us that in this act of abstraction we turn our backs on everything that is essential to language’s signifying in the way that it does; it is our act of abstracting language from its employment within our ordinary lives that turns it into something dead, whose ability to represent now cries out for explanation. Thus, the sense of a need to explain how language (conceived as a system of signs) has the magical power to represent the world is connected with our failure to look at language where it is actually functioning. Wittgenstein does not set out to satisfy our sense of a need for a theory of representation (a theory that explains how the dead sign acquires meaning), but to dispel this sense of a need through getting us to look at language where it is actually doing work, and where we can see its essence fully displayed. In directing us, through the concept of a language-game, to ‘the spatial and temporal phenomenon of language, not [to] some non-spatial, atemporal non-entity’ (PI §108), Wittgenstein hopes gradually to bring us to see that ‘nothing extraordinary is involved’ (PI §94), that everything that we need to understand the essence of language ‘already lies open to view’ (PI §126)}.}.

Estas consideraciones nos ayudan a v


%SECCIÓN 2: TRUTH
%%SECCIÓN 1: 
\section{Actividad Filosófica de Elizabeth Anscombe}

\subsection{Los primeros arduos esfuerzos}

\ifdraft{\subsubsection{De Wittgenstein a Anscombe}}{} 
En el 1929 Wittgenstein presentó el Tractatus Logico\=/Philosophicus como su
tesis doctoral en Cambridge. Ese mismo año fue designado como profesor en
``Trinity College'', allí estaría hasta 1936.

\ifdraft{\subsubsection{Causalidad reflexiones iniciales de Anscombe}}{}

Por aquella época de mediados de los 30 la joven Gertrude Elizabeth Margaret
Anscombe, andaba buscando un buen argumento que demostrara que todo lo que
existe tiene que tener una causa. ¿Por qué cuando algo ocurre estamos seguros de
que tiene una causa? Nadie sabía darle una respuesta.\autocite[cf.~][p.~vii
]{anscombe1981metaphysicsintro} Así, sin darse cuenta, se iniciaba en la ardua
tarea de la filosofía. Rigurosa y enérgica desde el principio.

El origen de su peculiar curiosidad por la causalidad se hallaba en una obra
llamada `Teología Natural' escrita por un jesuita del siglo XIX. Había llegado a
este libro motivada por su conversión a la Iglesia
Católica.\autocite[cf.~][p.~vii]{anscombe1981metaphysicsintro} El tratado le
resultó problemático en dos cuestiones.

La primera fue la doctrina de la \emph{`scientia media'}, según la cual Dios
tiene conocimiento, por ejemplo, de lo que alguien podría haber hecho si no
hubiera muerto cuando murió. A Elizabeth le parecía que lo que hubiera ocurrido
si lo que pasó no hubiera pasado simplemente no existe; no hay qué conocer. Y no
podía creer esto. Anscombe tuvo la oportunidad de discutir esta preocupación con
Richard Kehoe durante su preparación religiosa en su primer año en Oxford. La
dificultad para creer aquella doctrina le parecía un límite para aceptar la fe
católica. Richard le aclaró que no hacía falta que creyera en eso. Con el tiempo
entendió que se trataba de una discusión de escuela, en la que los jesuítas y
dominicos entablaron una ardua disputa y que la postura que ella había adoptado
era la qu había sido defendida por los
dominicos.\autocite[cf.~][p.~vii]{anscombe1981metaphysicsintro}

La segunda cuestión problematica la encontró en un argumento sobre la existencia
de la `Causa Primera'. El tratado ofrecía como preliminar al argumento una
demostración de un `principio de causalidad' según el cual todo cuanto existe
tiene que tener una causa. Anscombe notó, escasamente escondido en una premisa,
un presupuesto de la conclusión del propio argumento. Aquel \emph{petitio
  principii} le pareció un simple descuido y resolvió, por tanto, escribir una
versión mejorada de la demostración. Durante los siguientes dos o tres años
produjo unas cinco versiones que le parecían satisfactorias, sin embargo
eventualmente descubría que contenían la misma falacia, cada vez disimulada más
astutamente. Todo este esfuerzo lo realizó sin ninguna enseñanza formal en
filosofía, incluso su último intento de argumento lo hizo antes de estudiar
`Greats'.\autocite[cf.~][p.~vii]{anscombe1981metaphysicsintro}

\ifdraft{\subsubsection{Oxford: La Percepción y el fenomenalismo de Price}}{}

Sus lecturas en torno a su conversión fueron motivo de más reflexiones. Esta
vez, como fruto de \emph{The Nature of Belief} de Martin D'Arcy, se interesó por
el tema de la percepción. Durante años ocupaba su tiempo, en cafeterías, por
ejemplo, mirando fijamente objetos, diciendose a sí misma: <<Veo un paquete.
¿Pero qué veo realmente? ¿Cómo puedo decir que veo algo más que una extensión
amarilla?>>\autocite[cf.~][p.~viii]{anscombe1981metaphysicsintro}

Al principio su impresión era que lo que veía eran objetos:
\citalitinterlin{Estaba segura de que veía objetos, como paquetes de cigarrillos
  o tazas o\ldots~cualquier cosa más o menos sustancial
  servía.}\autocite[p.~viii]{anscombe1981metaphysicsintro} Además creía que
debemos de conocer la categoría de un objeto cuando hablamos de él, eso
corresponde a la lógica del término usado para hablar del objeto y no de algún
descubrimiento empírico. Estas ideas, sin embargo, las había desarrollado
fijándose en artefactos urbanos. Los ejemplos de percepción de la naturaleza que
más la impactaron fueron `madera' y el cielo. Este último le hizo retractarse de
su creencia sobre el conocimiento lógico de la categoría de los
objetos.\autocite[cf.~][p.~viii]{anscombe1981metaphysicsintro}

Sus indagaciones sobre la percepción, así como le ocurrió con la causalidad,
fueron previas al periodo de `Greats' donde estudiaría formalmente la filosofía.
Ya desde `Mods' asistía a las lecciones de H.H.~Price sobre percepción y
fenomenalismo. De todos los que escuchó en Oxford fue quién le inspiró mayor
respeto, no porque estuviera de acuerdo con lo que decía, sino porque hablaba de
lo que había que hablar. El único libro suyo que le pareció realmente bueno fue
\emph{Hume's Theory of the External World} y lo leyó sin interrupción de
principio a fin. Fue Price quien despertó en ella un intenso interés por el
capítulo de Hume sobre ``Del escepticismo con respecto a los sentidos''. Aunque
le parecía que Price tendía a suavizar a Hume, el hecho de que escribiera sobre
él le parecia que era escribir sobre las cosas mismas que merecía la pena
discutir. Asncombe, sin embargo, odiaba el fenomenalismo y se sentía atrapada
por él, pero no sabía salir de él, o rebatirlo. La postura escéptica tampoco la
convencía como para adoptarla y no la dejaba satisfecha. Esta insatisfacción no
haría más que crecer en sus años en Oxford.
\autocites[cf.~][p.~viii]{anscombe1981metaphysicsintro}
[~y~][p.~26]{torralba2005accion}

\ifdraft{\subsubsection{En Cambrdige con Wittgenstein}}{}

  En las lecciones con Wittgenstein en Cambridge fue que el pensamiento central
  <<Tengo \emph{esto}, y defino `amarillo' como \emph{esto}>> fue efectivamente
  atacado. Anscombe misma lo narra usando dos ejemplos:

  \citalitlar{En cierto punto Wittgenstein estaba discutiendo en sus clases la
    interpretación del letrero (sign-post), y estalló en mi que el modo en que
    vas según éste es la interpretación
    final.\autocite[p.~viii]{andcombe1981metaphysicsintro}}

En \emph{Investigaciones Filosóficas} \S198 

toda interpretación queda sostenida en el aire junto con lo que interpreta, y no
puede darle a ésto ningún apoyo. Las interpretaciones por sí solas no determinan
el significado.

  Aquí Elizabeth se refiere a \autocite[p.~86~\S198]{PI}

  \citalitlar{En otra ocasión salí con: <<Pero todavía quiero decir: ``Azul esta
    ahí''>>. [\ldots] [Wittgenstein] dijo: <<Déjame pensar qué medicina
    necesitas\ldots>> <<Supón que tenemos la palabra \emph{`painy'}, como una
    palabra para la propiedad de ciertas superficies>>. La `medicina' fue efectiva
    [\ldots] Si \emph{`painy'} fuera una palabra posible para una cualidad
    secundaria, ¿no podría el mismo motivo conducirme a decir: \emph{`painy'} esta
    aquí que lo que me condujo a decir azul está aquí? Mi expresión no significaba
    que `azul' es el nombre de esta sensación que estoy teniendo, ni cambié a ese
    pensamiento.\autocite[p.~viii]{andcombe1981metaphysicsintro}}

  The issue's significance can be seen by considering how the argument is
  embedded in the structure of Philosophical Investigations. Immediately
  prior to the introduction of the argument (§§241f), Wittgenstein suggests
  that the existence of the rules governing the use of language and making
  communication possible depends on agreement in human behaviour—such as the
  uniformity in normal human reaction which makes it possible to train most
  children to look at something by pointing at it. (Unlike cats, which react
  in a seemingly random variety of ways to pointing.) One function of the
  private language argument is to show that not only actual languages but
  the very possibility of language and concept formation depends on the
  possibility of such agreement.

  Another, related, function is to oppose the idea that metaphysical
  absolutes are within our reach, that we can find at least part of the
  world as it really is in the sense that any other way of conceiving that
  part must be wrong (cf. Philosophical Investigations p. 230). Philosophers
  are especially tempted to suppose that numbers and sensations are examples
  of such absolutes, self-identifying objects which themselves force upon us
  the rules for the use of their names. Wittgenstein discusses numbers in
  earlier sections on rules (185–242). Some of his points have analogues in
  his discussion of sensations, for there is a common underlying confusion
  about how the act of meaning determines the future application of a
  formula or name. In the case of numbers, one temptation is to confuse the
  mathematical sense of ‘determine’ in which, say, the formula y = 2x
  determines the numerical value of y for a given value of x (in contrast
  with y > 2x, which does not) with a causal sense in which a certain
  training in mathematics determines that normal people will always write
  the same value for y given both the first formula and a value for x—in
  contrast with creatures for which such training might produce a variety of
  outcomes (cf. §189). This confusion produces the illusion that the result
  of an actual properly conducted calculation is the inevitable outcome of
  the mathematical determining, as though the formula's meaning itself were
  shaping the course of events.

  In the case of sensations, the parallel temptation is to suppose that they
  are self-intimating. Itching, for example, seems like this: one just feels
  what it is directly; if one then gives the sensation a name, the rules for
  that name's subsequent use are already determined by the sensation itself.
  Wittgenstein tries to show that this impression is illusory, that even
  itching derives its identity only from a sharable practice of expression,
  reaction and use of language. If itching were a metaphysical absolute,
  forcing its identity upon me in the way described, then the possibility of
  such a shared practice would be irrelevant to the concept of itching: the
  nature of itching would be revealed to me in a single mental act of naming
  it (the kind of mental act which Russell called ‘acquaintance’); all
  subsequent facts concerning the use of the name would be irrelevant to how
  that name was meant; and the name could be private. The private language
  argument is intended to show that such subsequent facts could not be
  irrelevant, that no names could be private, and that the notion of having
  the true identity of a sensation revealed in a single act of acquaintance
  is a confusion.




    \begin{revision}
       ``For a large class of cases of the employment of the word ‘meaning’—though not
       for all—this way can be explained in this way: the meaning of a word is its use
       in the language'' (PI 43). This basic statement is what underlies the change of
       perspective most typical of the later phase of Wittgenstein's thought: a change
       from a conception of meaning as representation to a view which looks to use as
       the crux of the investigation. 
       \end{revision}

      \begin{revision}
      Philosophical Investigations:
      --Undertake an investigation, leading, not to the construction of new and
      surprising theories or explanations, but the examination of our life with
      language. This is a grammatical investigation PI~\S90 
      --The ideas of explanation and discovery are misleading and inappropiate when
      applied to questions like: what is meaning?
      --We feel as if we had to repair a spider web with our fingers PI~\S106
      --PI~\S129
      --By putting details together in the right way or by using a new analogy or
      comparison to prompt us to see our practice of using language in a new light, we
      find that we achieve the understanding that we thought would only come with the
      construction of an explanatory account. RFGB, p.30
      --Philosopher's questions must be treated like an illness is treated. PI~\S133
      and \S255.
      --The aim of grammatical investigations is perspicious representation PI~\S122
      --Meaning is use.
      --The question of a philosopher is: how do I go about this?
      \end{revision}


      \begin{revision}
      What marks the transition from early to later Wittgenstein can be summed up as
      the total rejection of dogmatism, i.e., as the working out of all the
      consequences of this rejection. The move from the realm of logic to that of
      ordinary language as the center of the philosopher's attention; from an emphasis
      on definition and analysis to ‘family resemblance’ and ‘language-games’; and
      from systematic philosophical writing to an aphoristic style—all have to do with
      this transition towards anti-dogmatism in its extreme. It is in the
      Philosophical Investigations that the working out of the transitions comes to
      culmination. Other writings of the same period, though, manifest the same
      anti-dogmatic stance, as it is applied, e.g., to the philosophy of mathematics
      or to philosophical psychology.
      \end{revision}




      2. La metodología terapéutica y franca de Wittgenstein fue liberadora
      \begin{revision}



      El método terapeútico de Wittgenstein tuvo éxito en liberarla de confusiones
      filosóficas donde otras metodologíás mas teoréticas habían fallado. En sus
      estudios en St. Hugh's escuchaba a Price/ldots
      \end{revision}


      \begin{revision}
      Este modo de criticar una proposición desvelando que no expresa un pensamiento
      verdadero ilustra los principios propuestos en el \emph{Tractatus} y recuerda
      una de sus tesis más conocidas: 
      En el prefacio de las Investigaciones Filosóficas, con fecha de enero de 1945
      Wittgenstein dice que los pensamientos que publica en el libro son el
      precipitado de invetigaciones filosóficas que le han ocupado durante los pasados
      16 años. En enero 1929 Wittgenstein estaba regresando a Cambridge.
      \end{revision}


      \begin{revision}
      En ocasiones como esta la
      discusión con Wittgenstein llevaba a Anscombe a afirmaciones para las cuales no
      podía ofrecer mejor significado que los sugeridos por concepciones ingenuas. Una
      concepción así no es otra cosa que ausencia de pensamiento, pero su falta de
      significado no es evidente, sino que requiere de la fuerza de un `Copérnico'
      para ponerla en cuestión efectivamente.\autocite[cf. 151]{IWT} 
      \end{revision}

Anscombe conoció a Wittgenstein en los años culminantes de su pensamiento
     filosófico. 
     Al comienzo de sus lecciones en 1944 Wittgenstein escribía a su amigo Rush Rhees:
     \citalitinterlin{
         \ldots mis clases no han ido tan mal. Thouless esta asistiendo, y una mujer, 
         'Mrs so and so'
         que se llama a sí misma 
         'Miss Anscombe',
         que ciertamente es inteligente, aunque no del calibre de Kreisel.
         \autocite[p.~371]{cambridgeletters}
     }
     Un año mas tarde escribía a Norman Malcolm:
     \citalitinterlin{
         \ldots mi clase ahora es bastante grande, 19 personas. \ldots Smythies esta
         viniendo, y una mujer que es muy buena, es decir, más que solamente
         inteligente\ldots 
         \autocite[p.~388]{cambridgeletters}
     }
     Aquellos años no sólo creció en Wittgenstein la apreciación de la capacidad de
     Anscombe, sino que se afianzó entre ellos una estrecha amistad. 

     La influencia de Wittgenstein fue decisiva para el desarrollo filosófico de
     Elizabeth. Las lecciones con Wittgenstein eran directas y con franqueza. Esta
     metodología carente de cualquier parafernalia era inquietante para algunos,
     inspiradora para otros, pero fue tremendamente liberadora para
     ella.\autocite[loc 9853 Chapter 4, Section 24, \S5]{monk} Esta libertad
     quedaba demostrada en que Anscombe no se contentaba con repetir lo que decía
     Wittgenstein, sino que pensaba por sí misma; en esto precisamente era más fiel
     al espíritu de la filosofía que había aprendido de él. Sobre esta relación,
     Phillipa Foot, amiga de ambos, cuenta que durante mucho tiempo sostuvo
     objeciones a las afirmaciones de Wittgenstein, eventualmente, un comentario de
     Norman Malcom la hizo pensar que podía haber valor en lo que Wittgenstein decía.
     Cuestionó entonces a Anscombe: 
     ``¿Por qué no me dijiste?'', ella le contestó: ``Porque es importante que uno
     tenga sus resistencias''. Anscombe evidentemente pensaba ---continúa Foot: 
     \citalitlar{
         que un largo periodo de vigorosa objeción era la mejor manera de entender a
         Wittgenstein. Aun cuando era su amiga cercana y albacea literaria, y una de
         los primeros en reconocer su grandeza, nada podía ser más lejano de su
         carácter y modo de pensamiento que el discipulado.\autocite[p.~4]{teichmann}
     }

     Peter geach que dice que les ayudó que estudiaron otros filósofos antes de
     Wittgenstein.

\pnote{introducir algunos contrastes y relaciones entre
       Anscombe y Wittgenstein para explicar la incursión en la vida/pensamiento
       de W.}


%SECCIÓN 3: FAITH
%\subsection{Que se puede entender de la fe sin tenerla}
En Oscott College, el seminario de la Archidiócesis de Birmingham, se comenzaron
a celebrar las conferencias llamadas Wiseman Lectures en 1971. Para estas
lecciones ofrecidas anualmente en memoria de Nicholas Wiseman se invitaba un
ponente que tratara algún tema relacionado con la filosofía de la religión o
alguna materia en torno al ecumenismo.\footcite[cf.~][p.~7]{wisemanlects}

El 27 de octubre de 1975, para la quinta edición de las conferencias, Anscombe
presentó una lección titulada simplemente ``Faith''. Allí planteaba la
siguiente cuestión: \citalitlar{Quiero decir qué puede ser entendido sobre la fe
  por alguien que no la tenga; alguien, incluso, que no necesariamente crea que
  Dios existe, pero que sea capaz de pensar cuidadosa y honestamente sobre ella.
  Bertrand Russell llamó a la fe ``certeza sin prueba''. Esto parece correcto.
  Ambrose Bierce tiene una definición en su \emph{Devil's Dictionary}: ``La
  actitud de la mente de uno que cree sin evidencia a uno que habla sin
  conocimiento cosas sin parangón''. ¿Qué deberíamos pensar de
  esto?\footcite[p.~115]{faith}}

\subsection{Descripción del Concepto `Fe'}
Hubo una época en la que se vivió gran entusiasmo por la racionalidad de la
fe.

El carácter racional de la fe estaba sujeto a los llamados preambulos y el
paso de estos a la fe. Anscombe entiende que éstos son construcciones ideales.
Al menos parte de ellos, sería más apropiado llamarles
presuposiciones.


%SECCIÓN 4: WHAT IS IT TO BELIEVE SOMEONE?
%\section{What is it to Believe Someone?}
Creer a alguien no es sólo un 


\chapter{La Concepción de G.\,E.\,M.\,Anscombe sobre el Testimonio}

\section{Acerca del Creer y Su Estructura}
\subsection{Un Peculiar Patrón de Argumento.}
Peter Geach dedica un breve apartado a Anscombe en su ``Autobiografía
Filosófica''. Ambos se dedicaban a la filosofía y era común que cuestionaran a
uno sobre el pensamiento del otro, sin embargo no era raro que no supieran cómo
contestar. Los dos tenían distintos intereses en sus investigaciones y tambíen
un estilo diferente al acercarse a los problemas filosóficos. Geach lo describe
así: \citalitlar{Como una filósofa madura, Elizabeth me parece ser una pensadora
  más intrépida que yo: es ella quien tiene ideas audaces y que a primera vista
  resultan meramente alocadas, a lo que en ocasiones he reaccionado con inicial
  indignación. (Cfr. sus escritos \emph{The Intentionality of Sensation} y
  \emph{The First Person}) Usualmente llego a ver cómo estas audaces ideas son
  más justificables de lo que originalmente
  suponía\footnote{\cite[11]{geach1991philaut}: <<As a mature philosopher,
    Elizabeth strikes me as a more adventurous thinker than I am: it is she who
    gets bold and at first sight merely zany ideas, to which I sometimes reacted
    with initial outrage. (Cfr. her papers `The Intentionality of Sensation' and
    `The First Person') Usually I come to think these bold ideas are more
    defensible than I had originally supposed.>>}.} El mismo libro que recoge
estas memorias de Geach contiene un breve artículo de Elizabeth titulado
\emph{On a Queer Pattern of Argument}\footnote{\cite{anscombe1991aqp} En
  adelante la referencia al artículo será como aparece en:
  \cite{anscombe2015logic:qpa}} que ejemplifica adecudamente las palabras antes
referidas sobre su esposa.

En esta ocasión la consideración intrépida consitirá en indagar sobre la validez
de principios lógicos familiares aplicándolos a diversos ejemplos de
argumentaciones. El extraño patrón de argumento que da título a la investigación
queda expresado de este modo:
  \begin{adjustwidth}{1.2cm}{}
    1.\hspace{.459cm}Si $p$, entonces $q$.\\
    2.\hspace{.459cm}Si $r$, entonces no (si $p$ entonces $q$).\\
    3.\hspace{.459cm}Si no $p$ entonces $r$.\\
    $\therefore$\hspace{.459cm}$p$ y $q$.
  \end{adjustwidth}

  Se obtiene `no $r$' de las primeras dos premisas y entonces `$p$' de `no $r$'
  y la tercera premisa; con la primera premisa nuevamente y `$p$' obtenemos la
  conclusión.{\footnote{\cite[299]{anscombe2015logic:qpa} <<We get `not $r$'
      from the first two premises and then `$p$' from `not $r$' and the third;
      with the first one again this gives us the conclusion>>.}} Hecha esta
  descripción, Anscombe entonces invita a considerar el siguiente argumento
  construido según el patrón anterior:
  \begin{adjustwidth}{1.2cm}{}
    1.\hspace{.459cm}Si ese árbol cae, entonces interrumpirá el paso por el camino
    durante mucho tiempo.\\
    2.\hspace{.459cm}Eso no es verdad si hay una máquina para remover árboles
    funcionando.\\
    3.\hspace{.459cm}Si el árbol no cae, habrá una máquina para remover árboles
    funcionando.\\
    $\therefore$\hspace{.459cm}El árbol caerá e interrumpirá el paso por el camino
    durante mucho tiempo.
  \end{adjustwidth}

  ¿Qué resultado se obtiene si se intenta formar un juicio razonable o
  conocimiento desde este argumento? <<Si ese árbol cae entonces interrumpirá el
  camino y si hay una maquina para remover árboles funcionando entonces no será
  verdad que si el árbol cae entonces interrumpira el camino.>> (`Si $p$
  entonces $q$ y si $r$ entonces no [si $p$ entonces $q$]'). De esta conjunción
  se sigue `no habrá una maquina para remover árboles funcionando' (`no $r$'),
  pero ¿se podría considerar esta deducción un juicio razonable?. La segunda
  premisa se lee como arrojando duda sobre la primera, y la tercera premisa
  expresa la pertinencia de la segunda. Descartar la duda y afirmar la primera
  sugiere que ya se cree la primera premisa antes de evaluar la segunda. Pero en
  ese caso el argumento mismo no explicaría los fundamentos para la conclusión.
  Aún cuando se estuviera asintiendo a las otras dos premisas porque ya se cree
  la primera, estas trabajan junto a un hipotético para sostener la
  creencia\autocite[Cf.~][300]{anscombe2015logic:qpa}.

  Anscombe entonces propone: \citalitinterlin{Si todo esto es correcto, tenemos
    aquí un caso bastante interesante de una serie de proposiciones que implican
    una conclusión pero no son fundamentos posibles para llegar a esa
    conclusión}\footnote{\cite[300]{anscombe2015logic:qpa} <<If all this is
    right, we have here a rather interesting case of a set of propositions which
    entail a conclusion but are impossible grounds for coming to that
    conclusion>>}. El argumento no necesita que se juzgue como verdadera la
  conclusión o parte de ella para considerar verdadera alguna de las premisas,
  pero sí reclama que parte de la conclusión sea fundamento para aceptar la
  combinación de modo que se pueda formar conocimiento o un juicio
  razonable\footnote{\cite[Cf.~][301]{anscombe2015logic:qpa}}.

  En este caso `$q$' no se sigue necesariamente de `$p$' y así, al no ser una
  verdad necesaria, sólo se puede aceptar la conjunción de las primeras dos
  premisas si se está independientemente seguro de que `no $r$'. No es común que
  estemos en la situación de pensar que `si $p$ entonces $q$' y que sólo por eso
  esté claro que `si $r$ entonces no (si $p$ entonces $q$)' y entonces poder
  deducir razonablemente de esto que `no $r$'. Es más común que al juzgar la
  conjunción de las primeras dos premisas, el antecedente de la segunda pierda
  fuerza. El punto de la segunda premisa es arrojar duda sobre la primera; la
  conjunción de la segunda premisa y la tercera refuerzan la pertinencia de la
  segunda. Sin embargo la segunda premisa sólo tendrá la fuerza de poner en duda
  la primera premisa ---y no al revés--- si, además de ser verdadera y
  pertinente, resulta imposible de descartar porque resulta necesario tomar en
  serio su antecedente `si $r$'\autocite[Cf.~][301]{anscombe2015logic:qpa}.

  Anscombe acuña la expresión `revocabilidad
  esencial'\footnote{\cite[Cf.~][301]{anscombe2015logic:qpa}: <<Then we have
    perhaps discovered the special character of (theoretical) hypotheticals
    whose consquents don't follow logically from their antecedents. We might
    call this character `essential defeasibility'>>.} para denominar al carácter
  especial de hipotéticos teoréticos cuyos consecuentes no se siguen lógicamente
  de sus antecedentes. En este caso esta característica es la que hace que
  incluso cuando `no $r$' se sigue de `si $p$ entonces $q$ y si $r$, entonces no
  (si $p$ entonces $q$)', no sería razonable deducir `no $r$' de esa conjunción.
  Elizabeth además observa que hay un gran número de juicios que son así. Al
  hacer una afirmación categórica con la seguridad apropiada, frecuentemente se
  descarta inmediatamente lo que la falsificaría sólo porque se sabe que ésta es
  verdadera. Sin embargo existe toda una clase de juicios como el que se ha
  analizado que al ser hechos no se descarta implicitamente como falso todo lo
  que los falsificaría\autocite[Cf.~][302]{anscombe2015logic:qpa}.

\subsection{¿Qué es creer a alguien?}
\subsubsection{Cuestión preliminar}
En el análisis anterior Anscombe ha descrito un escenario en el que combinar
varias premisas como conocimiento o juicio razonable resulta problemático a la
hora de justificar el fundamento de la conclusión apoyándose sólo en las
premisas y su relación lógica.

En su investigación titulada \emph{What is it to believe someone?} Anscombe
comienza describiendo otro escenario basado en el mismo argumento, situándose así
en una situación que plantea la misma dificultad; también en el creer a alguien
el fundamento para la combinación de las premisas en un juicio razonable parece
estar más allá de las mismas premisas y sus relaciones. En esta ocasión cada
premisa aparece atribuida a una persona distinta y la conclusión a un cuarto
personaje. El pequeño relato aparece como sigue: \citalitlar{Había tres hombres,
  $A$, $B$ y $C$, hablando en cierta aldea. $A$ dijo: ``Si ese árbol cae,
  interrumpirá el paso por el camino durante mucho tiempo.'' ``No será así si
  hay alguna máquina para remover árboles funcionando'', dijo $B$. $C$ destacó:
  ``\emph{Habrá} una, si el árbol no cae.'' El famoso sofista Eutidemo, un
  extraño en el lugar, estaba escuchando. Inmediatamente dijo: ``Les creo a
  todos. Así que infiero que el árbol caerá e interrumpirá el paso por el
  camino.'' \footnote{\cite[1]{anscombe2008faith:tobelieve} <<There were three
    men, $A$, $B$ and $C$, talking in a certain village. $A$ said ``If that tree
    falls down, it'll block the road for a long time.'' ``That's not so if
    there's a tree-clearing machine working'', said $B$. $C$ remarked ``There
    \emph{will} be one, if the tree doesn't fall down.'' The famous sophist
    Euthydemus, a stranger in the place, was listening. He immediately said ``I
    believe you all. So I infer that the tree will fall and the road will be
    blocked.''>>}}

¿En qué está mal Eutidemo? Si se evalúa la lógica del argumento antes expuesto
no aparece ninguna contradicción, sin embargo hay algo extraño en la afirmación
``les creo a todos''. Si la lógica del argumento parece permitir que la
inferencia de Eutidemo sea posible, ¿por qué suena tan extraña la posibilidad de
que les crea a todos y juzgue esa conclusión?

\subsubsection{Naturaleza de la Investigación}
Es útil recordar aquí en términos generales el modo en el que Anscombe actua en
una investigación filosófica. Wittgenstein inicialmente describió el análisis
del lenguaje bajo la concepción de que la lógica conforma el orden que está
debajo y que sostiene todo lenguaje posible. El trabajo del filósofo es analizar
el lenguaje para sacar al descubierto el orden lógico que está debajo del
lenguaje ordinario y que es la forma de la realidad. Wittgenstein abandonó esta
concepción; en Investigaciones Filosóficas exclama: \citalitlar{Cuanto más de
  cerca examinamos el lenguaje actual, más crece el conflicto entre éste y
  nuestro requisito. (Pues la pureza cristalina de la lógica no era, por
  supuesto, algo que yo hubiera \emph{descubierto}: era un requisito.) El
  conflicto se hace intolerable; el requisito llega ahora a estar en peligro de
  tornarse vacuo. --- Nos hemos situado en hielo resbaladizo donde no hay
  fricción, y así, en cierto sentido, las condiciones son ideales; pero también,
  justo por eso, no somos capaces de caminar. Queremos caminar: así que
  necesitamos \emph{fricción}. ¡De vuelta al terreno
  escarpado!\footnote{\cite[\S107]{wittgenstein1953phiinv}: <<The more closely
    we examine actual language, the greater becomes the conflict between it and
    our requirement. (For the crystalline purity of logic was, of course, not
    something I had \emph{discovered}: it was a requirement.) The conflict
    becomes intolerable; the requirement is in danger of becoming vacuous. ---
    We have got on to slippery ice where there is no friction, and so, in a
    certain sense, the conditions are ideal; but also, just because of that, we
    are unable to walk. We want to walk: so we need \emph{friction}. Back to the
    rough ground!>>}.}

Los nombres, las proposiciones, el lenguaje, no tienen una forma esencial para
ser puesta al descubierto por el análisis, sino que son familias de estructuras
que están a plena vista y que pueden ser clarificadas por medio de la
descripción\autocite[Cf.~][12]{bakerhacker2009understanding}. Wittgenstein le
\citalitinterlin{da la vuelta a la
  busqueda}\autocite[\S108]{wittgenstein1953phiinv}, y trata a la lógica no como
lo que está debajo del lenguaje para ser descubierto, sino como
\citalitinterlin{una cuadrícula que imponemos sobre los argumentos para probar y
  demostrar su validez}\footnote{\cite[12]{bakerhacker2009understanding}: <<a
  grid we impose upon arguments to test and demonstrate their validity>>}.

Descartada esta concepción sublime, Wittgenstein describe los problemas
filosóficos como formas de malentendidos o falta de entendimiento que pueden ser
disueltos por medio de descripciones de los usos de las palabras. La tarea de la
filosofía es la \citalitinterlin{clarificación gramatical que disuelve la
  perplejidad conceptual y ofrece una visión amplia o representación estudiable
  de un segmento de la gramática de nuestro
  lenguaje}\footnote{\cite[12]{bakerhacker2009understanding}: <<grammatical
  clarification that dissolves conceptual puzzlement and gives an overview of or
  surveyable representation of a segment of the grammar of our language>>}. Esta
metodología, por tanto, no pretende ofrecer teorías explicativas fruto de la
deducción o la hipótesis; tampoco pretende ofrecer tesis dogmáticas o
esencialistas. Más bien busca describir usos familiares de las palabras y
ordenarlas de tal manera que los patrones de su uso sean
estudiables\autocite[Cf.~][12]{bakerhacker2009understanding}. La metodología de
Elizabeth está basada en esto.

\subsubsection{Investigación Gramática de `creer a $x$ que $p$'.}
Anscombe pone el interés de su investigación en la forma de la expresión `creer
a $x$ que $p$'\autocite[Cf.~][2]{anscombe2008faith:tobelieve}. Su análisis se va
desenvolviendo a lo largo de la descripción de los usos de la expresión.

\citalitinterlin{Si me dijeras `Napoleón perdió la batalla de Waterloo' y te
  digo `te creo' sería una
  broma}\footnote{\cite[4]{anscombe2008faith:tobelieve}: <<If you tell me
  `Napoleon lost the battle of Waterloo' and I say `I believe you' that is a
  joke.>>}. A primer golpe `creer a $x$ que $p$' parece que significa
simplemente creer lo que alguien me dice, o creer que lo que me dice es
verdadero. Sin embargo esto no es suficiente. Puede ser que ya crea lo que
alguien me venga a decir. Puede ser que la comunicación suscite que forme mi
propio juicio acerca de la verdad comunicada, pero aquí no podría decir que
estoy creyendo al que comunica o que estoy contando con él para mi creer que
$p$.

¿Entonces creer a alguien es creer algo apoyado en el hecho de que lo ha dicho?
\citalitinterlin{Puede que se le pregunte a un testigo `¿Por qué pensó que aquel
  hombre se estaba muriendo?' y que éste responda `Porque el doctor me lo dijo'
  [\ldots] `no me hice ninguna opinión propia --- yo sólo creí al
  doctor'}\footnote{\cite[4]{anscombe2008faith:tobelieve}: <<A witness might be
  asked `Why did you think the man was dying?' and reply `Because the doctor
  told me'. If asked further what his own judgement was, he may reply `I had no
  opinion of my own --- I just believed the doctor'.>>}. Este puede ser un
ejemplo de contar con $x$ para la verdad de $p$. Esto, sin embargo, tampoco
parece ser suficiente. Puedo imaginar el caso en el que esté convencido de que
alguien a la vez cree lo opuesto a la verdad de $p$ y quiera mentirme. Según
este cálculo podría decir que creo en lo que ha dicho por el hecho de que me lo
ha dicho, pero no estaría diciendo que le creo a él.

¿Qué se puede decir del <<les creo a todos>> de Eutidemo en la cuestión
preliminar? Anscombe juzga que la exclamación no expresa simplemente una opinión
apresurada o excesiva credulidad, sino más bien suena a
locura\autocite[5]{anscombe2008faith:tobelieve}. Eutidemo no puede estar
diciendo la verdad cuando dice que les cree a todos. La expresión de $C$ da
pertinencia a lo que dice $B$, y la manera natural de entender lo que dice $B$
es como arrojando duda sobre lo que $A$ ha dicho. ¿Se puede pensar que $A$
todavía cree lo que ha dicho inicialmente? ¿Eutidemo puede creer a $A$ sin saber
cuál es su reacción a lo que $B$ y $C$ han dicho? Entonces Anscombe concluye,
\citalitinterlin{Para creer a $N$ uno debe creer que $N$ mismo cree lo que está
  diciendo}\footnote{\cite[5]{anscombe2008faith:tobelieve}: <<To believe $N$ one
  must believe that $N$ himself believes what he is saying>>.} Creer a $N$ sin
saber si $N$ cree lo que dice le suena a Elizabeth como una locura.

En este punto queda expuesta a la luz una segunda creencia involucrada en el
creer a $x$ que $p$. Anscombe fija su atención en esto. Creer a $x$ que $p$
conlleva otras creencias, éstas son presuposiciones implicadas en llegar a
plantearse si creer o no. En primer lugar, si se cree a alguien, tiene que ser
el caso que se cree que una comunicación es de
alguien\autocite[Cf.~][6]{anscombe2008faith:tobelieve}. Esta presuposición no
parece tan problemática si se piensa en las ocasiones en las que creemos a
alguien que es percibido. Sin embargo tiene más profundidad si se considera que
con frecuencia recibimos la comunicación sin que esté presente el que habla,
como cuando leemos un libro\autocite[Cf.~][5]{anscombe2008faith:tobelieve}.

Se puede imaginar aquí una situación problemática. Supongamos que alguien recibe
una carta en la que el autor no es el comunicador ostensible o aparente, es
decir, quien firma la carta no es quien la ha escrito. ¿Se puede decir que el
que recibe la carta cree o descree al autor o al comunicador ostensible? Creer
al autor, afirma Anscombe, conlleva un tipo de juicio y especulación que no son
mediaciones ordinarias en el creer a
alguien\autocite[Cf.~][7]{anscombe2008faith:tobelieve}. Para decir que creo al
autor tendría que discernir que la comunicación que viene bajo otro nombre es
realmente de esta otra persona que además me quiere decir esto.

Respecto de la posibilidad de decir que se cree al comunicador ostensible
Anscombe distingue entre un comunicador ostensible que exista o no. Ante una
comunicación que viene de parte de un comunicador aparente que no existe,
alguien puede responder diciendo que cree o descree al comunicador aparente,
pero la decisión de decir esto ---dice Anscombe--- \citalitinterlin{es una
  decisión de dar a estos verbos un uso `intencional', como el verbo `ir
  tras'}\footnote{\cite[7]{anscombe2008faith:tobelieve}: <<is a decision to give
  those verbs an `intentional' use like the verb `to look for'>> Ver:
  \cite{anscombe1981metaphysics:intsens}. Anscombe propone que un verbo es usado
  intencionalmente cuando tiene como objeto directo un `objeto intencional'
  (`objeto' no en el sentido material, sino de finalidad).}. Esto lo ilustra
añadiendo: \citalitlar{Y así uno podría hablar de alguien como creyendo al dios
  (Apolo, digamos), cuando consultó el oráculo del dios -- sin que por esto uno
  estuviera implicando que uno mismo cree en la existencia del dios. Todo lo que
  queremos es que necesitamos saber lo que es llamado que el dios le diga
  algo\footnote{\cite[7]{anscombe2008faith:tobelieve}: <<And so we might speak
    of someone as believing the god (Apollo, say), when he consulted the oracle
    of the god -- without thereby implying that one believed in the existence of
    the gos oneself. All we want is that we should know what is called the god's
    telling him something>>}.} `Creer' usado aquí intencionalmente viene a decir
que se busca o se desea creer a $x$ (Apolo en este caso) cuando se recibe
aquello que alguien entiende como una comunicación suya.

En el caso de que el comunicador ostensivo sí exista, la noción de creerle
manifiesta una cierta oscilación. Una tercera persona podría decir que `aquel,
pensando que $N$ dijo esto, le creyo', o el comunicador aparente puede decir
`veo que pensaste que fui yo quien dijo esto y me creiste', sin embargo, si el
que ha recibido la comunicación dijera `naturalmente te creí', el comunicador
aparente podría contestar `ya que no lo he dicho yo, no me estabas creyendo a
mi'\autocite[Cf.~][8]{anscombe2008faith:tobelieve}.

Estas consideraciones llevan a Anscombe a distinguir entre el que habla en una
comunicación y el productor inmediato de la
comunicación\autocite[Cf.~][8]{anscombe2008faith:tobelieve}. Éste puede ser
cualquiera que pase hacia adelante alguna comunicación, un maestro o mensajero,
o un interprete o traductor; éste es \citalitinterlin{el productor inmediato de
  aquello que se entiende, o incluye una reclamación interna de ser entendido
  como una comunicación de $NN$}\footnote{\cite[8]{anscombe2008faith:tobelieve}:
  <<we can speak of the immediate producer of what is taken, or makes an
  internal claim to be taken, as a communication from $NN$>>}. Si digo que creo
a un intérprete estoy afirmando que creo lo que ha dicho su principal, y mi
contar con el intérprete consiste en la creencia de que ha reproducido lo que
aquel ha dicho. En este sentido el intérprete no le falta rectitud si dice algo
que no es verdadero pero no ha representado falsamente lo que ha dicho su
principal. Por el contrario, al maestro sí le faltaría rectitud si lo que dice
no es verdadero. Cuando se cree al maestro, aún en el caso que no sea de ninguna
manera autoridad original de lo que comunica, se le cree a él sobre lo que
transmite. Para Anscombe no es necesario que cuando se cree a alguien se le
trate como una autoridad
original\autocite[Cf.~][5]{anscombe2008faith:tobelieve}. En esto el ejemplo del
maestro como distinto del intérprete es ilustrativo. Un maestro puede conocer lo
que enseña porque lo ha recibido de alguna tradición de información y al
transmitir lo que enseña se le está creyendo a él.

Asoma aquí otro aspecto relacionado con esta presuposición. Al creer que una
comunicación es de alguien se cree a una persona que puede tener distintos
grados de autoridad sobre lo que dice. El maestro del que se ha hablado antes
podría afirmar <<Leonardo da Vinci dibujó diseños para una máquina voladora>> y
en esto no es para nada una autoridad
original\autocite[Cf.~][6]{anscombe2008faith:tobelieve}. Conoce esto porque lo
ha escuchado, incluso si ha visto los diseños. Aún cuando los hubiera
descubierto él mismo, tendría que haber contado con alguna información recibida
de que esos diseños que ve son de Leonardo. En este caso sí seria una autoridad
original en notar que estos diseños que ha escuchado que son de Leonardo son de
máquinas voladoras. Anscombe explica la distinción diciendo:
\citalitlar{[Alguien] es \emph{una} autoridad original en aquello que él mismo
  ha hecho y visto y oido: digo \emph{una} autoridad original porque sólo quiero
  decir que él mismo sí contribuye algo, es algún tipo de testigo por ejemplo,
  en lugar de alguien que sólo transmite información recibida. Pero su informe
  de aquello de lo que es testigo es con frecuencia [\ldots] fuertemente
  influenciado o más bien casi del todo formado por la información que \emph{él}
  ha recibido\footnote{\cite[5]{anscombe2008faith:tobelieve}: <<He is \emph{an}
    original authority on what he himself has done and seen and heard: I say
    \emph{an} original authority because I only mean that he does himself
    contribute something, e.g. is in some sort a witness, as oposed to one who
    only transmits information received. But his account of what he is a witness
    to is very often [\ldots] heavily affected or ratherl all but completely
    formed by what information \emph{he} had received.>>}.} Además de ser
\emph{una} autoridad original sobre algún hecho, una persona puede ser una
autoridad \emph{totalmente} original. Si la distinción entre alguien que no es
una autoridad original y alguien que sí lo es ha sido descrita como la
contribución de algo propio que junto con la información recibida permite
construir un informe, lo particular de una autoridad totalmente original es que
no se apoya en ninguna información recibida para construir su informe de los
hechos. Anscombe no entiende el lenguaje como información recibida. Pone como
ejemplo de informe de una autoridad totalmente original a alguien que dice `esta
mañana comí una manzana' y dice: \citalitlar{si él está en la situación usual
  entre nosotros, el sabe lo que una manzana es --- es decir, puede reconocer
  una. Así que aún cuando se le ha `enseñado el concepto' al aprender a usar el
  lenguaje en la vida ordinaria, no cuento esto como un caso de depender en
  información recibida.\footnote{\cite[6]{anscombe2008faith:tobelieve}: <<if he
    is in the situation usual among us, he knows what an apple is --- i.e. can
    recognise one. So though he was `taught the concept' in learning to use
    language in everyday life, I do not count that as a case of reliance on
    information received.>>}}

Hasta aquí se ha visto que el creer a $x$ que $p$ implica otras creencias que
son presuposiciones a la pregunta sobre si se cree o se descree a alguien y se
ha descrito lo que tiene que ver con la creencia de que una comunicación viene
de alguien. Anscombe examina otras presuposiciones más. También tiene que ser el
caso que creamos que por la comunicación, la persona que habla quiere decir
\emph{esto}. En situaciones ordinarias no es difícil distinguir si alguien está
diciendo o escribiendo algún lenguaje. Sin embargo, aún cuando el que habla use
palabras que puedo `hacer mías' y creer simplemente las palabras que dice, aquí
queda espacio para decir que hay una creencia adicional de que se ha dicho `tal
cosa' en la comunicación. Elaboramos en aquello que hemos creido y usamos otras
palabras distintas, nuestras creencias no están atadas a palabras específicas.
También podríamos pensar que alguien diga que cree \emph{esto} porque cree a $x$
y que se le cuestione su creencia preguntando `¿qué tomaste como $x$ dicicéndote
eso?'.

Otra presuposición más sería que se cree que la comunicación está
\emph{dirigida} a alguien, aunque sea `quien lea esto' o `a quien pueda
interesar'. Esta creencia se podría problematizar pensando en algún caso que
alguien reciba una comunicación con otro destinatario, ¿estaría creyendo al que
se comunica?. Asncombe opina que en un sentido extendido o reducido y considera
que el tema parece de poca
importancia\autocite[Cf.~][7]{anscombe2008faith:tobelieve}.

Una persona a quien se dirige una comunicación puede \emph{fallar en creerla} si
no nota la comunicación, o si notándola no la interpreta como lenguaje, o si
notándola como lenguaje no la toma como dirigida hacia ella; o puede que crea
todo esto, pero lo interprete incorrectamente, o puede que lo interprete bien
pero no crea que viene realmente de $N$. En este tipo de casos la persona no ha
descreido, sino que no ha llegado a estar en la situación de plantearse esa
pregunta. Para poder llegar a preguntar si alguien cree a $x$ que $p$ habría que
excluir o asumir como excluidos todos los casos en los que estas otras
presuposiciones no se han cumplido. Es así que Anscombe concluye:
\citalitlar{Supongamos que todas la presuposiciones están dadas. $A$ está
  entonces en la situación ---una muy común--- donde surge la pregunta sobre si
  creer o dudar (suspender el juicio ante) $NN$. Sin confusión por todas las
  preguntas que surgen por las presuposiciones, podemos ver que creer a alguien
  (en el caso particular) es confiar en él para la verdad -- en el caso
  particular. \footnote{\cite[9]{anscombe2008faith:tobelieve}: <<Let us suppose
    that all the presuppositions are in. $A$ is then in the situation ---a very
    normal one--- where the question arises of believing or doubting (suspending
    judgement in face of) $NN$. Unconfused by all the questions that arise
    because of the presuppositions, we can see that believing someone (in the
    particular case) is trusting him for the truth -- in the particular
    case.>>}.}
Que $A$ crea a $N$ que $p$ implica que $A$ cree que en una comunicación, que puede
venir de un productor inmediato, $N$ es el que habla y lo que dice es $p$ y esta
comunicación está dirigida hacia $A$; entonces $A$, creyendo que $N$ cree que
$p$, confia en $N$ sobre la verdad de $p$.

\subsection{`Creer', Conocimiento y Testimonio}
El análisis de Elizabeth ofrece la posibilidad de hacer una descripción general
de lo que significa `creer a un testigo que $p$'. Anscombe ha hecho la
distinción entre alguien que simplemente transmite información y alguien que
puede ser considerado algún tipo de testigo. Un testigo es un ejemplo de
autoridad original y alguien es una autoridad original acerca de lo que él mismo
ha hecho y visto y oído. Un testigo que es una autoridad original aporta algo de
lo que él mismo ha hecho y visto y oído y lo considera junto a información que
ha recibido para comunicar su informe de algún hecho. Cuando el testigo no
cuenta con información recibida, sino que habla sólo de lo que aporta él mismo,
es una autoridad totalmente original. Aunque es una descripción del testigo
todavía muy amplia, permite afirmar que cuando $A$ cree a un testigo que $p$,
$A$ cree que en una comunicación, que puede venir de un productor inmediato, es
esta autoridad original el que habla y que dice $p$ y tiene a $A$ como
destinatario; entonces $A$, creyendo que esta autoridad original cree lo que
dice, confia en el testigo sobre la verdad de $p$.

La investigación de Anscombe sobre el creer incluye otra descripción amplia
relacionada con el testimonio en la justificación o preámbulo de su análisis.
Propone lo que sigue: \citalitlar{Hemos de reconocer al testimonio como el que
  nos da nuestro mundo más grande en no menor grado, o incluso en un grado
  mayor, que la relación de causa y efecto; y creerlo es bastante distinto en
  estructura que el creer en causas y efectos. Tampoco es lo que el testimonio
  nos da una parte completamente desprendible, como el fleco de grasa en un
  pedazo de filete. Es más bien como los flequillos y rayas de grasa que están
  distribuidos a través de la buena carne; aunque hay nudos de pura grasa
  también\footnote{\cite[3]{anscombe2008faith:tobelieve}:<<We must acknowledge
    testimony as giving us our larger world in no smaller degree, or even in a
    greater degree, than the relation of cause and effect; and believing it is
    quite dissimilar in structure from belief in causes and effects. Nor is what
    testimony gives us entirely a detachable part, like the thick fringe of fat
    on a chunk of steak. It is more like the flecks and streaks of fat that are
    distributed through good meat; though there are lumps of pure fat as
    well>>}.} Para Anscombe no es problemático entender el creer como parte de
la teoría del conocimiento y considera incluso que la mayor parte de nuestro
conocimiento de la realidad está apoyado en la creencia que tenemos en las cosas
que se nos han enseñado o dicho\autocite[Cf.~][3]{anscombe2008faith:tobelieve}.
Al respecto, Elizabeth critica la teoría de Hume en la que propone que nuestro
acceso a una idea del mundo más allá de nuestras experiencias personales es la
idea de causa-y-efecto. Para él, así como creemos en las causas al percibir sus
efectos porque hemos descubierto que causa y efecto van siempre unidos, creemos
en la verdad del testimonio porque percibimos el testimonio y hemos descubierto
que verdad y testimonio van siempre unidos. Para Anscombe la postura es
absurda\autocite[Cf.~][3]{anscombe2008faith:tobelieve}.

La descripción que hace Anscombe aquí del papel que juega el testimonio en
nuestro acceso a una idea

\section{Sobre la Primacía de la Verdad}
\subsection{¿Qué es tener la verdad?}
Elizabeth Anscombe visitó muchas veces la Universidad de Navarra junto con Peter
Geach. Allí impartió algunos seminarios y participó de las Reuniones
Filosóficas.\footcite[Cf.~][15]{torralbaynubiola2005fayeh} En una de sus
visitas, en octubre de 1983, ofreció dos lecciones tituladas: ``Verdad'' y ``La
unidad de la verdad''. Las dos investigaciones estan apoyadas en algunas
reflexiones de San Anselmo cuyos argumentos sirven a Anscombe para explorar
modos de hablar de aquello de lo que decimos que tiene verdad. Anscombe dio
inicio a su ponencia planteando la cuestión como sigue: \citalitlar{Hay verdad
  en muchas cosas. Mirando a mi título [`Truth'] me quedo algo sobrecogida por
  él, pues lo que salta de la página hacia mi es uno de los nombres de Dios. `He
  amado la verdad' me dijo una vez un profesor moribundo, después de hablarme de
  la dificultad que sentía sobre la idea de amar a Dios. Sin embargo: `He amado
  la verdad'. Y luego, temiendo que yo no malentendiera su afirmación: `No me
  refiero, cuando digo eso, que \emph{tenga} la verdad'.} \citalitlar{Tener la
  verdad, estar en la verdad---¿qué es esto?
  \footnote{\cite[71]{anscombe2011plato:truth}: <<There is truth in many things.
    Looking at my title [`Truth'] I am somewhat awed by it, for what leaps out
    of the page at me is one of the names of God. `I have loved the truth' a
    dying teacher once said to me, after speaking of the difficulty he felt
    about the idea of loving God. But:`I have loved the truth'. And then,
    fearing lest I misconstrue his statement: `I do not mean, when I say that,
    that I \emph{have} the truth'. To have the truth, to stand in the truth --
    what are these?>>}.}

%Y qué quiso decir Nuestro Señor al llamarse a \emph{sí mismo} la verdad? `No hay
%tal cosa como la verdad, sólo hay verdades', decía mi suegro a la primera esposa
%de Bertrand Russell. Russell fue su maestro; la influencia se ve con facilidad.}

%\citalitlar{¿Pero cuáles son las cosas que tienen verdad en ellas? ¿Tiene la
%  creación? ¿tienen las acciones? A qué se refería Aristóteles cuando dijo que
%  el bien de la razón práctica era `verdad de acuerdo con el recto deseo'? ¿Las
%  cosas hechas por los hombres tienen verdad en ellas? ¿Qué, de nuevo, quiso
%  decir Aristóteles cuando afirmó que el arte o la habilidad es una disposición
%  productiva con un logos verdadero? Mas allá todavía: Qué fuerza tiene contar
%  la verdad entre los `trascendentales', esas cosas que `atraviesan' todas las
%  categorías y todas las formas especiales de las cosas; y que no pertenecen
%  cada uno a una categoría, como el color: amarillo; o el area: un acre; o el
%  animal: un caballo.}

\subsection{La primacia de la verdad sobre la falsedad}
Este cuestionamiento lleva a Anscombe a indagar en una materia en la que
Wittgenstein y San Anselmo ---dice--- son `hermanos intelectuales': ¿cuál es la
primacía de la verdad sobre la
falsedad?\autocite[Cf.~][73]{anscombe2011plato:truth}.

San Anselmo queda prendado de esta pregunta como consecuencia de su indagación
en el capítulo segundo del \emph{De Veritate}: ¿qué es la verdad de la
enunciación?\footnote{\cite[Cf.~][493]{anselm1952obras:deveritate} Para las
  citas del texto de San Anselmo se ha empleado la traducción de
  \cite{anselm1952obras} donde `\emph{enuntiatio}' se traduce como
  `enunciación', Anscombe lo traducirá como `\emph{proposition}'. `Enunciación'
  y `proposición' se usarán aquí indistintamente.}. Anselmo elige indagar en las
enunciaciones o proposiciones como aquellas clases de las cuales más
naturalmente se puede pensar que contienen los posibles portadores del predicado
`verdadero'. Así lo expresa cuando dice \citalitinterlin{Busquemos, pues, en
  primer lugar qué es la verdad de la enunciación, puesto que es ésta la que
  calificamos con más frecuencia de verdadera o
  falsa}\autocite[493]{anselm1952obras:deveritate}.

Wittgenstein recorre una ruta análoga en los apartados que conforman el \S4.06
del Tractatus. Argumenta que \citalitinterlin{Una proposición puede ser
  verdadera o falsa sólo en virtud de ser una imagen de la
  realidad}\footnote{\cite[\S4.06]{wittgenstein1922tractatus}:<<Propositions can
  be true or false only by being pictures of the reality.>>}. Y advierte que
\citalitlar{No debe ser pasado por alto que una proposición tiene un sentido que
  es independiente de los hechos: de otra manera uno podría fácilmente suponer
  que verdadero y falso son relaciones igualmente justificadas entre los signos
  y aquello que
  significan\footnote{\cite[\S4.061]{wittgenstein1922tractatus}:<<If one does
    not observe that propositions have a sense independent of the facts, one can
    easily believe that true and false are two relations between signs and
    things signified with equal rights.>>}.}

Elizabeth realiza su investigación adentrándose en la misma cuestión trabajada
por ambos autores. El primer movimiento que hace en su análisis es indagar en la
distinción entre significado y verdad. Según se ha visto, la distinción es
familiar en las elucidaciones del Tractatus: \citalitinterlin{La proposición
  tiene un sentido que es independiente de los hechos}
\footnote{\cite[\S~4.061]{wittgenstein1922tractatus}: <<propositions have a
  sense independent of the fact>>} San Anselmo también lo considera. Una
proposición no pierde su significado cuando no es verdadera. Si el significado
(\emph{significatio}) de una proposición fuera su verdad, ésta
\citalitinterlin{semper esset vera}\autocite[492]{anselm1952obras:deveritate},
siempre sería verdadera. Sin embargo el significado de una proposición
\citalitinterlin{manent \ldots et cum est quod enunciat, et cum non
  est}\autocite[492]{anselm1952obras:deveritate}, permanece lo mismo cuando lo
que se afirma es el caso que es y cuando no lo es.

Significado y verdad en una proposición son distintos. Entonces, ¿qué es la
verdad de una proposición?. Se podría querer responder que es la
\citalitinterlin{res enunciata}, es decir, la realidad correspondiente, lo que
la proposición verdadera dice. Esta respuesta nos llevaría a confusión. ``La
verdad de una proposición es este hecho que es su significado''. Si esto es así,
entonces cuando deja de ser verdadera también pierde su significado, pues el
hecho que era su signifcado ya no es. Además, si la desaparición del hecho es la
desaparición del significado y la verdad, ¿no será entonces que el hecho es la
misma cosa que el significado y la
verdad?\autocite[Cf.~][72]{anscombe2011plato:truth}. Sin embargo no es así, el
hecho es lo que la hace verdadera: lo que la proposición verdadera dice, la
\emph{res enunciata} es la causa de la verdad de una proposición y no su verdad:
\citalitinterlin{non eius veritas, sed causa veritatis eius dicenda
  est}\autocite[492]{anselm1952obras:deveritate}.

La distinción abre otra línea de consideraciones. El hecho o la \emph{res
  enunciata} por la proposición verdadera es la causa de la verdad del
enunciado. La proposición tiene significado independientemente de si es
verdadera o falsa. En este sentido, una proposición con significado puede
guardar relación de verdad o de falsedad con los hechos. Una proposición falsa
no carece de toda relación con el hecho, sino que contiene una descripción del
hecho que hace a la proposición contraria
verdadera\autocite[Cf.~][73]{anscombe2011plato:truth}. Podríamos pensar,
entonces, que la proposición verdadera y la proposición falsa pueden
intercambiar roles.

Wittgenstein sugiere esto cuando afirma que el hecho de \citalitinterlin{que los
  signos ``$p$'' y ``${\sim}p$'' (``no $p$'') pueden intercambiar roles es
  importante, pues muestra que ``$\sim$'' (``no'') no corresponde con nada en la
  realidad}\footnote{\cite[\S4.0621]{wittgenstein1922tractatus}: <<That,
  however, the signs ``$p$'' and ``${\sim}p$'' can say the same thing is important,
  for it shows that the sign ``$\sim$'' corresponds to nothing in reality.>>}. Más
aún ``$p$'' y ``${\sim}p$'' son opuestos en significado pero a ambos enunciados
corresponde una sola realidad; esto es el hecho, la \emph{res enunciata} por el
enunciado verdadero. Esto permitiría sostener que verdadero y falso son tipos de
relaciones entre el signo y la cosa significada que están igualmente
justificadas. ``$p$'' y ``${\sim}p$'' significan la misma realidad, cualquiera
de las dos posibilidades que resulte ser la realidad correspondería con
ambas\autocite[Cf.~][73]{anscombe2011plato:truth}. La única distinción que
Wittgenstein se reserva entre ambas proposiciones es que una significa
falsamente lo que la otra significa verdaderamente. Sin embargo esta distinción
puede quedar disuelta con facilidad si se considera que `significa
verdaderamente' o `significa falsamente' no son descripciones de los sentidos de
las proposiciones verdaderas o falsas. Se puede entender el sentido de ``estoy
sentado'' o ``no estoy sentado'' sin conocer cuál enunciado se corresponde con
la realidad o cuál de ambas expresiones está significando verdaderamente y cuál
falsamente. En cuanto a la relación entre signo y significado ambas
proposiciones no tienen diferencia\autocite[Cf.~][74]{anscombe2011plato:truth}.

En San Anselmo esta noción de relaciones igualmente justificadas aparece bajo la
forma de una pregunta planteada por el discípulo en el diálogo con su maestro.
Dice: \citalitlar{enséñame a responder a aquel que me dijese que aun cuando el
  discurso exprese la existencia de lo que no existe, significa lo que debe,
  porque ha podido significar igualmente la existencia de lo que es y de lo que
  no es. En efecto, si no significara también la existencia de lo que no existe,
  no lo significaría. Por lo cual, aun cuando dice ser lo que no es, significa
  lo que debe. Pero si, al significar lo que debe, es recta y verdadera, como
  has demostrado, el discurso es verdadero aun cuando enuncia la existencia de
  lo que no existe\autocite[495]{anselm1952obras:deveritate}.} Las dos
relaciones son expresadas como una paridad: \citalitinterlin{pariter accepit
  significare esse, et quod est, et quod non
  est}\autocite[494]{anselm1952obras:deveritate}. Esta paridad es esencial ya
que si la proposición no significara lo que significa igualmente cuando lo que
significa es y también cuando tal cosa no es, no sería capaz de significar del
todo.

A propósito de esta paridad, Wittgenstein plantea: \citalitinterlin{¿Acaso no
  podríamos hacernos entender usando proposiciones falsas tal como hemos hecho
  hasta ahora por medio de las verdaderas, siempre y cuando sepamos que están
  significadas falsamente?}\footnote{\cite[\S4.062]{wittgenstein1922tractatus}:
  <<Can we not make ourselves understood by means of false propositions as
  hitherto with true ones, so long as we know that they are meant to be
  false?>>}. Anscombe compara este posible modo de actuar a una táctica de Santa
Juana de Arco. La Santa empleaba un código en las comunicaciones con sus
generales subordinados que consistía en que las cartas que ella marcaba con una
cruz contenían proposiciones que debían ser interpretadas en el sentido
contrario\autocite[Cf.~][73]{anscombe2011plato:truth}. El código es posible.

Hasta aquí Anscombe ha insitido en los argumentos de San Anselmo y de
Wittgenstein que apoyan la idea de que las proposiciones falsas y verdaderas
tienen igualdad de relación con la realidad significada. Wittgenstein ha
advertido del supuesto de entender ambas relaciones como igualmente
justificadas, sin embargo lo que ha propuesto hasta ahora parece apoyar esta
idea. La paridad propuesta ha resultado esencial para el significado, el sentido
o \emph{significatio} del tipo de proposiciones que pueden ser verdaderas o
falsas. La pregunta ahora es ¿qué, entonces, \emph{es} desigual entre ellas?
¿Cuál es la primacia de la verdad?

La respuesta de Wittgenstein a esta pregunta llegará a ser: no se puede
describir a alguien como comunicándose con proposiciones falsas entendidas como
significadas falsamente ya que se tornan en proposiciones verdaderas al ser
afirmadas\autocite[Cf.~][75]{anscombe2011plato:truth}. Esta es su respuesta a la
pregunta ¿podemos darnos a entender con proposiciones falsas?:
\citalitinterlin{¡No! Pues una proposición es verdadera si las cosas son así
  como estamos usándola para decir que son, y entonces si usamos ``$p$'' para
  decir que ${\sim}p$, y las cosas son como queremos decir que son, entonces
  ``$p$'' es vedadero en nuestro nuevo modo de tomarlo y no
  falso}\footnote{\cite[\S4.062]{wittgenstein1922tractatus}: <<No! For a
  proposition is true, if what we assert by means of it is the case; and if by
  ``$p$'' we mean ${\sim}p$, and what we mean is the case, then ``$p$'' in the
  new conception is true and not false.>>}. En la táctica antes descrita, Santa
Juana de Arco no mentía con su código y, si no estaba en error acerca de los
hechos, sus oraciones eran verdaderas y no
falsas\autocite[Cf.~][75]{anscombe2011plato:truth}.

Para Anscombe, esta descripción de la primacía de la verdad no parece explicar
cómo rechazar que verdadero y falso tengan relaciones igualmente justificadas
¿Acaso este tipo de imposibilidad general contiene toda la sustancia de las
`relaciones no igualmente justificadas'? Se puede aceptar que verdadero y falso
no son relaciones igualmente justificadas porque lo falso no podría hacerse
cargo del rol de lo verdadero en las afirmaciones y en el pensamiento. Sin
embargo, podemos mentir\ldots\, o equivocarnos. La imposibilidad general de
intercambiar los roles de verdadero y falso propuesta por Wittgenstein no
considera ni el error ni la mentira. Esta imposibilidad general puede ofrecer
una cierta primacia de la verdad dentro de la teoría del significado, pero ¿se
podría apoyar en esto el decir que la proposición verdadera tiene una relación
mas \emph{justificada} con la realidad que la
falsa?\autocite[Cf.~][75]{anscombe2011plato:truth}.

En San Anselmo, por su parte, se puede encontrar una propuesta sobre la primacía
de la verdad dentro de su definición de lo que la verdad es. Su punto de partida
ha sido la pregunta: \citalitinterlin{¿Cuál es el fin de la
  afirmación?}\autocite[495]{anselm1952obras:deveritate} El diálogo se
desarrolla de este modo: \citalitlar{\emph{Maestro.}---¿Cuál te parece ser aquí
  la verdad?\\
  \emph{Discípulo.}---No sé más que, cuando significa existir lo que existe
  realmente, está en ella la verdad y es verdadera.\\
  \emph{M.}---¿Cuál es el fin de la afirmación?\\
  \emph{D.}---Expresar lo que es.\\
  \emph{M.}---¿Debe, pues, hacerlo?\\
  \emph{D.}---Ciertamente.\\
  \emph{M.}---Por consiguiente, cuando expresa la existencia de lo que existe,
  expresa lo que debe.\\
  \emph{D.}---Es evidente.\\
  \emph{M.}---Y cuando expresa lo que debe, expresa con exactitud.\\
  \emph{D.}---Así es.\\
  \emph{M.}---Pero cuando expresa con rectitud, ¿su significación es exacta?\\
  \emph{D.}---Sin duda ninguna.\\
  \emph{M.}---Cuando expresa la existencia de lo que es, ¿la significación es recta?\\
  \emph{D.}---Es una conclusión que se impone.\\
  \emph{M.}---Igualmente, cuando significa la existencia de lo que existe, su
  significado es verdadero.\\
  \emph{D.}---Ciertamente es a la vez verdadera y recta cuando expresa la
  existencia de lo que es.\\
  \emph{M.}---¿Entonces es una misma y única cosa para ella el ser recta y
  verdadera, es decir, manifestar la existencia de lo que es?\\
  \emph{D.}---Es una sola y misma cosa.\\
  \emph{M.}---Por consiguiente, para ella, la verdad no es otra cosa que la
  rectitud.\\
  \emph{D.}---Sí; veo con claridad que la verdad no es más que esta rectitud.\\
  \emph{M.}---Lo mismo hay que decir cuando la enunciación expresa la no
  existencia de lo que existe\autocite[495]{anselm1952obras:deveritate}.}

El discípulo ha visto que la verdad del enunciado no es la \emph{res enunciata}
por una proposición verdadera, tampoco está en la significación, o en cualquier
cosa perteneciente a la definición, sino que \citalitinterlin{Nihil aliud scio
  nisi quia cum significat esse qous est, tunc est in ea veritas et est
  vera}\autocite[492]{anselm1952obras:deveritate}. Cuando una afirmación hace
aquello para lo que es, la significación (\emph{significatio}) está hecha
rectamente. Esta rectitud es lo que la verdad es. Es aquí que el discípulo
presenta la objeción antes expuesta: `Cuando una expresión significa que es algo
que no es, ¿se puede decir que está significando lo que debe?'. La respuesta del
maestro será: \citalitinterlin{veritatem tamen et rectitudinem habet, quia facit
  quod debet}\autocite[494]{anselm1952obras:deveritate}. Una expresión falsa
hace lo que debe en significar aquello que le ha sido dado significar, hace
aquello para lo que la expresión es. Sin embargo, teniendo este modo de ser
verdadera, no solemos llamarla verdadera pues habitualmente decimos que la
expresión es verdadera y correcta sólo cuando significa que es aquello que es y
no cuando significa que es aquello que no es, pues tiene mayor deber de hacer
aquello para lo que se le ha dado significar que para lo que no se le ha dado.
Es sorprendente que el maestro no rechace la descripción del discípulo, más aún
que la reitere. La objeción presentada no supone un impedimento para sostener
esta descripción de la verdad. El maestro retiene su explicación apoyada en que
la verdad de un enunciado es que hace lo que
debe\autocite[Cf.~][76]{anscombe2011plato:truth}.

¿En qué consiste entonces la primacía de la verdad? La proposición verdadera
hace lo que debe de dos maneras: significa justo aquello que se le ha dado
significar ---independientemente de si es el caso que es o no--- y significa
aquello para lo que se le ha dado esa significación, esto es, afirmar como que
es el caso lo que \emph{es} el caso. Calificamos de justa y verdadera la
proposición en virtud de ese hacer doblemente lo que debe, es decir, por su
rectitud y verdad.\autocite[Cf.~][497]{anselm1952obras:deveritate}.

Una observación adicional de Anselmo puede ser relacionada con la pregunta de
Wittgenstein: `¿Podríamos darnos a entender por medio de proposiciones falsas?'.
\citalitinterlin{[la enunciación] no ha sido hecha para expresar que una cosa
  existe cuando no existe o que no existe cuando sí existe, porque fue imposible
  hacer que expresase solamente la existencia cuando ésta existe, o la no
  existencia cuando no existe}\autocite[497]{anselm1952obras:deveritate}. A la
proposición no se le podía dar significar que algo es solamente cuando eso que
significa da el caso que es o su no ser sólo cuando es el caso que no es,
solamente por eso puede significar lo contrario de lo que existe, aunque no ha
sido hecha para eso\autocite[Cf~.][76]{anscombe2011plato:truth}. La observación
se acerca a la respuesta de Wittgenstein. En este sentido, lo falso sólo es
posible porque lo verdadero (en este tipo de proposiciones) no puede ser la
única posibilidad.

La descripción de la verdad que Anselmo comienza aquí le llevará por medio de
consideraciones sobre la verdad en el pensamiento, la voluntad, la acción y el
ser de las cosas a su conocida definición de la verdad como \emph{veritas est
  rectitudo sola mente perceptibilis}\autocite[522]{anselm1952obras:deveritate}.

\subsection{Solución de Anscombe}
Anscombe no llega a proponer una respuesta suya a la cuestión planteada en
\emph{Truth}. Culmina constatando cómo San Anselmo y Wittgenstein indican una
cierta primacia de la verdad en la materia del significado apoyados en distintas
razones. Sin embargo en \emph{Truth, Sense and
  Assertion}\autocite{anscombe2015logic:tsa} quedan recogidas sus notas para una
lección ofrecida en \emph{Johns Hopkins University} en abril de
1987\autocite[Cf.~][264 n.~1]{anscombe2015logic:tsa} en donde continúa su
análisis y ofrece una solución propia.

La pregunta fundamental que plantea Anscombe en este análisis es:
\citalitinterlin{¿Es la enunciación lo mismo que la
  significación?}\footnote{\cite[271]{anscombe2015logic:tsa}:<<Is enuntiation
  the same as signification?>>}. El sentido de un enunciado es el mismo cuando
éste es verdadero o falso, pero ¿se puede decir lo mismo de la enunciación?. La
proposición verdadera tiene una \emph{res enuntiata}, ¿hay algo enunciado cuando
una proposición es falsa?.

Curiosamente, Elizabeth hecha mano de los sofistas para formular su respuesta.
Trae a la memoria cómo para el sofista todo lo que opina cualquier persona es
verdad, lo que viene al pensamiento es como la percepción, es el modo en que las
cosas se presentan a cada uno. Desde esta idea, el sofista inventa el argumento
de que \citalitinterlin{Aquel que piensa lo que es falso piensa lo que no es;
  pero lo que no existe no es nada; así que el que piensa lo que es falso no
  está pensando nada, pero si no está pensando nada, no está
  pensando}\footnote{\cite[264]{anscombe2015logic:tsa}: <<`He who thinks what is
  false thinks what is not; but what is not isn't anything; so he who thinks
  what is false isn't thinking \emph{anything}, but if he isn't thinking
  anything, he isn't thinking.'>>}. Anscombe propone entonces lo que considera
\citalitinterlin{el último pedazo, la piedra angular del arco que representa las
  relaciones entre verdad, sentido y
  aserción}\footnote{\cite[271]{anscombe2015logic:tsa}: <<the last bit, the
  keystone of the arch representing the relations of truth, sense and
  assertion>>}, dice:\citalitlar{Se llega a donde los Sofistas estaban en lo
  correcto en mi formulación presente: la proposición falsa, mientras que sí
  \emph{dice algo}, no es el caso, cuando es creída, que \emph{enuncie} a sus
  creyentes cosa alguna. Así: aquel que piensa lo que es falso piensa lo que no
  es; piensa algo que le dice nada; pero esto no significa que piense nada, es
  decir, que no esté pensando en nada
  \footnote{\cite[271]{anscombe2015logic:tsa}<<Where the Sophists were right is
    reached in my present formulation: the false proposition, while it does
    \emph{say something}, does not, being believed, \emph{tell} its believers
    anything. So: he who thinks what is false thinks what is not; he thinks
    something which tells him nothing; but that does not mean he thinks nothing,
    i.e. does not think anything.>>}.}

Según Anscombe una proposición verdadera refleja la existencia de lo que sí es,
mientras que la situación analoga en la proposición falsa es que refleja la
existencia de aquello que no es; ambos, la existencia reflejada y aquello que no
es, son nada\autocite[271]{anscombe2015logic:tsa}. En ese sentido, la proposición
falsa, aunque dice o expresa un signo, no transmite o informa nada, puesto que
lo que refleja no es.

Elizabeth establece una distinción adicional. Una aserción no sólo tiene como
objeto la proposición afirmada, sino que también tiene un sujeto personal. La
persona usa la proposición para afirmar lo que la proposición significa. La
proposición cumple con la tarea de significar siendo falsa o cierta, la persona
que la usa para afirmar, en este sentido, tiene un deber mayor de emplearla para
significar la existencia de lo que sí
es\autocite[Cf.~][267]{anscombe2015logic:tsa}. Hecha esta distinción, se puede
decir que una persona enuncie una falsedad, pero esta proposición, si es creida,
no informa a su creyente. El pensamiento que se construya desde esa creencia
dice algo que no informa de nada\autocite[Cf.~][271]{anscombe2015logic:tsa}. Una
paradoja, por otra parte, no sólo no informa o eununcia, sino que no dice o
expresa nada\autocite[Cf.~][271]{anscombe2015logic:tsa}.

La cuestión planteada al final de la investigación sobre creer a alguien quedó
formulada diciendo: dada la posibilidad de adquirir la misma creencia `$p$' de
alguien que habla rectamente y es veraz o de alguien que habla equivocadamente y
miente, ¿por qué hay una indisposición a llamar al segundo caso creer al que
habla?.

Después de examinar la descripción que Anscombe ha hecho sobre la verdad se
puede añadir aquí que la proposición misma, aún teniendo significación, no
enuncia nada (carece de \emph{res enuntiata}) cuando es falsa; en ese sentido
significa como debe (tiene esa rectitud) pero no hace aquello para lo que se le
ha dado el significar. Cuando una persona hace una afirmación, usa la
proposición para significar y en esto tiene rectitud, sin embargo tiene un deber
mayor de emplear la proposición para el fin que se le ha dado el significar, es
decir, reflejar la existencia de lo que sí es. Cuando una persona se equivoca y
miente emplea una proposición que hace lo que debe al significar, y por tanto se
puede `calcular' su opuesto para llegar a la verdad; pero no es recta en su
deber de emplear la proposición para el fin que se le ha dado el significar. Es
esta rectitud en el uso de la proposición lo que generaría la disposición a
decir que se cree a alguien, y no sólo lo que dice.

Después de este recorrido examinaremos el segundo punto derivado anteriormente
desde la investigación sobre el creer: la descripción de `fe' como `creer a
Dios'.

\section{Fe como creer a Dios}
\subsection{Que se puede entender de la fe sin tenerla}
En Oscott College, el seminario de la Archidiócesis de Birmingham, se comenzaron
a celebrar las conferencias llamadas Wiseman Lectures en 1971. Para estas
lecciones ofrecidas anualmente en memoria de Nicholas Wiseman se invitaba un
ponente que tratara algún tema relacionado con la filosofía de la religión o
alguna materia en torno al ecumenismo.\footcite[cf.~][p.~7]{wisemanlects}

El 27 de octubre de 1975, para la quinta edición de las conferencias, Anscombe
presentó una lección titulada simplemente ``Faith''. Allí planteaba la
siguiente cuestión: \citalitlar{Quiero decir qué puede ser entendido sobre la fe
  por alguien que no la tenga; alguien, incluso, que no necesariamente crea que
  Dios existe, pero que sea capaz de pensar cuidadosa y honestamente sobre ella.
  Bertrand Russell llamó a la fe ``certeza sin prueba''. Esto parece correcto.
  Ambrose Bierce tiene una definición en su \emph{Devil's Dictionary}: ``La
  actitud de la mente de uno que cree sin evidencia a uno que habla sin
  conocimiento cosas sin parangón''. ¿Qué deberíamos pensar de
  esto?\footcite[p.~115]{faith}}

\subsection{Descripción del Concepto `Fe'}
Hubo una época en la que se vivió gran entusiasmo por la racionalidad de la
fe.

El carácter racional de la fe estaba sujeto a los llamados preambulos y el
paso de estos a la fe. Anscombe entiende que éstos son construcciones ideales.
Al menos parte de ellos, sería más apropiado llamarles
presuposiciones.

\section{Cuestión sobre la Estructura del Testimonio}

\section{Profecías y Milagros}

\section{Sentido del Misterio y Racionabilidad}
\end{document}
