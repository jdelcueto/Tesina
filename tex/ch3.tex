\documentclass[./main.tex]{subfiles}
\begin{document}
\setcounter{chapter}{2}

% \setcounter{chapter}{2}
%\chapter{Desarrollo filosófico de Elizabeth Anscombe}

%ESTE CAPÍTULO SE OFRECE COMO UN TESTIMONIO DE LA VIDA, EL PENSAMIENTO Y LA
%FILOSOFÍA DE ANSCOMBE
%¿Cómo Elizabeth Anscombe hizo filosofía?
%Peter Geach tenía estas palabras que
%decir sobre su esposa:\citalitlar{Como una filósofa madura, Elizabeth me parece
%    ser una pensadora más intrépida que yo: es ella quien tiene ideas audaces y
%    que a primera vista resultan meramente alocadas, a lo que en ocasiones he
%    reaccionado con inicial indignación. (Cfr. sus escritos \emph{The
%        Intentionality of Sensation} y \emph{The Fisrt Person}) Usualmente llego
%    a ver como estas audaces ideas son más justificables de lo que originalmente
%    suponía. \autocite[p.~11]{philaut}}

%    Bernard Williams, cuya relación intelectual con Anscombe no estuvo libre de
%    fricciones, hablando sobre su experiencia en Oxford en los años 50,
%    dijo\autocite[p.~228]{teichmann}: \citalitlar{Otra persona que tuvo un tipo
%    de influencia sobre mi ---¡aunque me alegra decir que pienso que no me ha
%    influido de otros modos!--- fue Elizabeth Anscombe. Una cosa que hacía, que
%    sacó de Wittgenstein, era que imprimía sobre uno que el ser ingenioso no
%    era suficiente. La filosofía de Oxford, y esto todavía es cierto hasta
%    cierto punto, tenía una gran tendencia a ser ingeniosa. Era muy erística:
%    había
%    mucho intercambio dialéctico competitivo, y mucho demostrar que los otros
%    estaban equivocados. Yo era muy bueno en todo eso. Pero Elizabeth transmitía
%    un fuerte sentido de seriedad al tema, y cómo éste era difícil en modos para
%    los que ser ingenioso simplemente no era suficiente.\autocite[Bernard
%    Williams en entrevista con el Harvard Review of philosophy, 2004]{ref}}

%Roger Teichmann propone también: \citalitlar{Hay otra razón para la falta de
%    aparente sistematicidad en los escritos de Anscombe, y esto es que su
%    propósito al escribir era típicamente llegar a algún sitiio en sus propios
%    pensamientos sobre algún tema; usalmente dedica poco o nigún tiempo en
%    proveer un trasfondo, o en justificar su principales `presupuestos',
%    prefiriendo empezar \emph{in media res}.\autocite[p.~1]{teichmann}}

%Anscombe protestó públicamente la guerra y también la legalización del aborto.

%Y añade: \citalitlar{Un modo en el que Anscombe se diferencia considerablemente
%    de Wittgenstein es en su actitud hacia los males político y sociales.
%    [...]Ella era en totalmente una de esas personas que se ganan el epíteto de
%    `ser francos', y sus amigos más indulgentes tendrían que admitir que sus
%    modos podían en ocasiones ser inquietantemente
%    bruscos.\autocite[p.4]{teichmann}}

%Describiremos su quehacer filosófico en dos pasos: atendiendo el impacto que
%Wittgenstein causa en la filosofía con el Tractatus y luego atendiendo las
%cuestiones filosóficas que Anscombe confronta.

%SECCIÓN 1: ANSCOMBE Y WITTGENSTEIN
%%SECCIÓN 1: TRASFONDO FILOSÓFICO DE ELIZABETH
\section{La filosofía de Wittgenstein como trasfondo}

\subsection{El método de Wittgenstein}

Anscombe escuchó alguna vez a Wittgenstein explicar en sus clases cómo pretendía
ofrecer ejemplos de `ejercicios para cinco dedos', como los que se emplean para
el piano. Eran ejercicios para pensar. <<Soy como un maestro de piano>>
--decía-- <<intento enseñar un estilo de pensar, una técnica, no una materia>>.
Lo que se escuchaba en sus lecciones, sin embargo, no eran piezas musicales,
sino más bien las prácticas donde el pianista afina los movimientos que van
dirigidos a construir su
concierto.\autocite[cf.~][p.~357]{KlaggeNordman2003pubnpriv}

En cierta ocasión, Wittgenstein recibió a Anscombe con una pregunta: <<¿Por qué
la gente dice que era natural pensar que el sol giraba alrededor de la tierra en
lugar de que la tierra rotaba en su eje?>> Elizabeth contestó: <<Supongo que
porque se veía como si el sol girara alrededor de la tierra.>> <<Bueno\ldots>>,
añadió Wittgenstein, <<¿cómo se hubiera visto si se hubiera \emph{visto} como si
la tierra rotara en su propio eje?>> Anscombe reaccionó extendiendo las manos
delante de ella con las palmas hacia arriba y, levantándolas desde sus rodillas
con un movimiento circular, se inclinó hacia atrás asumiendo una expresión de
mareo. <<¡Exactamente!>> exclamó
Wittgenstein.\autocite[cf.~][p.~151]{anscombe1959iwt}

Anscombe se percató del problema; la pregunta de Wittgenstein había puesto en
evidencia que hasta aquél momento no había ofrecido ningún significado relevante
para su expresión \emph{``se veía como si''} en su respuesta \emph{``se veía
  como si el sol girara alrededor de la tierra''}. En ocasiones como ésta
Elizabeth descubría cómo no podía ofrecer ningún significado que no fuera
respaldado por una concepción ingenua, y ésta podía ser destruida fácilmente por
una pregunta. \citalitinterlin{<<La concepción ingenua es realmente descuido en
  el pensamiento, pero puede necesitar el poder de un Copérnico para
  cuestionarla efectivamente.>>\autocite[p.~151]{anscombe1959iwt}}

¿Qué tipo de problema es éste? ¿Qué falta cuando una expresión carece de
significado?

\subsection{El arte de hacer filosofía}

\ifdraft{\subsubsection{Vida salvaje luchando por emerger abiertamente}}{}
Wittgenstein pensaba que
% Within all great art there is a WILD animal: tamed.
\citalitinterlin{ dentro de todo buen arte hay un animal salvaje
  domado}\autocite[p.~43e]{wittgenstein1998cnv}. Su talante artístico, sin
embargo, no manifestaba esta primitiva vitalidad; o como él mismo decía:
% In my artistic activities I have merely good manners
\citalitinterlin{ en mis actividades artísticas tengo meramente buenos
  modales.}\autocite[p.~29e]{wittgenstein1998cnv}

Ejemplo de estos ``buenos modales'' fue el diseño que realizó para la casa de su
hermana Margaret en Viena, terminada en 1928.
% my house for Gretl is the product of a sensitive ear, good manners, the
% expression of great understanding... wild life striving to erupt in the open
% is lacking... health is lacking (Kierkergaard)
Trabajó como arquitecto de la casa con exhaustiva minuciosidad y el producto
manifestaba gran entendimiento, ``buen oido'', pero le escaseaba ``salud'',
pensaba él.\autocite[p.~43e]{wittgenstein1998cnv}

% Even in music... feeling, he showed above all great understanding, rather than
% manifesting wild life... When he played music with others... his interest was
% in getting it right... When he played, he was not expressing himself... but
% the thoughts... of others. He was probably right to regard himself not as
% creative but as reproductive ...It was only in philosophy that his creativity
% could really be awakened. Only then, as Russell had long ago noticed, does one
% see in him 'wild life striving to erupt in the open''

También en la música, arte por la que tenía la mayor afición, era llamativa su
recia exactitud. Cuando tocaba con otros ponía su mayor interés en lograr una
expresión exacta y correcta, recreando música y pensamientos ajenos, más que
expresándose a sí mismo. Perseguía reproducir más que
crear.\autocite[loc.˜]{monk1991duty}

Esta fuerza creativa ausente en su rigurosa actitud hacia la actividad artística
estallaba, sin embargo, en su actividad filosófica. Aquella cualidad que él
encontraba característica del buen arte, esa ``vida salvaje luchando por emerger
abiertamente'',\autocite[cf.˜][loc.˜]{monk1991duty} quedaba expresada en su
quehacer filosífico.

\ifdraft{\subsubsection{Filosofía emergente}}{}
La filosofía nació así en Ludwig. Como una fuerza violenta. Se hallaba
estudiando ingeniería en Manchester y se interesó por los fundamentos de las
matemáticas. Este interés no tardó en convertirse en el deseo de elaborar un
trabajo filosófico. Su hermana Hermine le describe así en sus memorias de la
familia Wittgenstein
\footnote{Hermine Wittgenstein escribió la historia y memorias de su familia
  ``Familienerinnerungen'' durante la segunda Guerra Mundial.}:

\citalitlar{Fue repentinamente agarrado por la filosofía ---es decir, por la
  reflexión en problemas filosóficos--- tan violentamente y tan en contra de su
  voluntad que sufrió severamente por la doble y conflictiva llamada interior y
  se veía a sí mismo como roto en dos. Una de muchas transformaciones por las
  que pasaría en su vida había venido sobre él y le estremeció hasta lo más
  profundo. Estaba concentrado en escribir un trabajo filosófico y finalmente
  determinó mostrar el plan de su obra al Profesor Frege en Jena, quien había
  discutido preguntas similares. [\ldots] Frege alentó a Ludwig en su búsqueda
  filosófica y le aconsejó que fuera a Cambridge como alumno del Profesor
  Russell, cosa que Ludwig ciertamente hizo.\autocite[p. 73]{mcguinness}}

La investigación filosófica comenzada en aquel momento se convirtió en la tarea
del resto de su vida. Sus incipientes ideas filosóficas pasarían por diversas
transformaciones, pero expresaban ya desde el principio una preocupación por los
problemas fundamentales. Por las reglas del juego, se podría decir.

\ifdraft{\subsubsection{La Naturaleza de los problemas Filosóficos}}{}
Entre esas cuestiones fundamentales se halla una de las constantes importantes
en su pensamiento. Ésta es su definición de la naturaleza de los problemas
filosóficos. Para Wittgenstein las cuestiones de la filosofía no son
problemáticas por ser erróneas, sino por no tener
significado.\autocite[cf.~][4.003]{wittgenstein1922tractatus}

Una proposición sin significado que no es puesta al descubierto como tal atrapa
al filósofo dentro de una confusión del lenguaje que no le permite acceder a la
realidad. Salir de la confusión no consiste en refutar una doctrina y plantear
una teoría alternativa, sino en examinar las operaciones hechas con las palabras
para llegar a manejar una visión clara del empleo de nuestras expresiones. La
filosofía no es un cuerpo doctrinal, sino una
actividad\autocite[cf.~][4.112]{wittgenstein1922tractatus}y una
terapia\autocite[cf.~][\S133]{wittgenstein1953phiinv}.

La actitud terapéutica adoptada por Wittgenstein en su atención de las
confusiones filosóficas fue su respuesta más definitiva a la naturaleza de estos
problemas. Para ello halló los más eficaces remedios en sus investigaciones
sobre el significado y el sentido del lenguaje.

Ordinariamente tomamos parte en esta actividad humana que es el lenguaje.
Jugamos el juego del lenguaje. ---¿Jugarlo es entenderlo?--- A la vista de
Wittgenstein saltaban extraños problemas sobre las reglas de este juego;
entonces no podía evitar escudriñarlas al
detalle.\autocite[cf.~][loc.7099]{monk1991duty} En este análisis del lenguaje está la
raíz de sus ideas sobre el sentido, el significado y la verdad.

Durante su vida sostuvo dos grandes descripciones del significado. Originalmente
describió el lenguaje como una imagen que representa el posible estado de las
cosas en el mundo. En una segunda etapa se distanció de esta analogía para
describir al lenguaje como una herramienta cuyo significado consiste en la suma
de las múltiples semejanzas familiares que aparecen en los distintos usos para
los cuales el lenguaje es empleado en la actividad humana. Dentro de la primera
descripción una expresión sin significado es una cuyos elementos no componen una
representación del posible estado de las cosas. Dentro de la segunda descripción
una expresión sin significado resulta del empleo de una expresión propia de un
``juego del lenguaje'' fuera de su contexto.

\subsection{Dos Cortes en la Filosofía}
 Estas dos etapas del pensamiento de Wittgenstein son representadas por dos
 importantes tratados. El \emph{'Tractatus Logico\=/Philosophicus'}, publicado en
 1921, recoge sus esfuerzos por elaborar un gran tratado filosófico comenzados en
 1911 y culminados durante la Primera Guerra Mundial. El segundo,
 \emph{'Philosophische Untersuchungen'}, o \emph{'Investigaciones Filosóficas'},
 traducido por Anscombe y publicado posthumamente en 1953, fue elaborado a partir
 de múltiples manuscritos desarrollados por Wittgenstein desde su regreso a
 Cambridge en 1929 hasta su muerte en 1951.

 \citalitinterlin{Wittgenstein es extraordinario entre los filósofos por haber
   generado dos épocas, o cortes\footnote{Anscombe toma el termino 'corte' de
     Boguslaw Wolniewicz, filósofo polaco y amigo.}, en la historia de la
   filosofía.}\autocite[p.~181]{anscombe2011plato:twocuts} 
 Con estas palabras Anscombe comenzaría su discurso inaugural para el Sexto
 Simposio Internacional de Wittgenstein unos treinta años después de la
 publicación de las \emph{'Investigaciones Filosóficas'}. Y explica:
 \citalitinterlin{un filósofo hace un corte si genera un cambio en el modo en que
   la filosofía es hecha: la filosofía tras el corte no puede ser la misma de
   antes.}\autocite[p.~181]{anscombe2011plato:twocuts}

 Estos cambios de época generados por la influencia de Wittgenstein vinieron
 caracterizados por el esfuerzo de comprender cada libro tras su publicación,
 tarea complicada en ambos casos por la dificultad intrínseca de los tratados,
 ofuscada a su vez por los prejuicios filosóficos proyectados a cada obra por sus
 lectores. La presunción, por ejemplo, de que \emph{'Investigaciones
   Filosóficas'} presenta una teoría del lenguaje ---quizás sobre cómo los
 sonidos se tornan en discursos significativos--- nos dejaría situados lejos de
 las preguntas que genuinamente ocupan a
 Wittgenstein.\autocite[cf.~][p.~183]{anscombe2011plato:twocuts} Ahora bien, la comprensión
 adecuada de su pensamiento y método trae consigo, a juicio de Anscombe, cierto
 efecto curativo.

\ifdraft{\subsubsection{Ver el mundo claramente}}{}
Quedar 'curados' es quedar liberados de la trampa de ciertas inclinaciones que
impiden llegar a concepciones verdaderas. El trabajo de Wittgenstein busca tener
este efecto en la filosofía. ¿Lo logra?

Elizabeth analiza uno de estos esfuerzos. Es una aflicción extendida entre los
filósofos la excesiva dependencia en explicaciones o conexiones necesarias. ¿Han
podido quedar curados los que han estudiado a Wittgenstein? Y añade:
\citalitlar{La filosofía profesional es en gran medida una gran fábrica para la
  manufactura de necesidades---sólo las necesidades nos dan paz mental. No es de
  extrañarse que Wittgenstein despierte cierto odio entre nosotros. Amenaza
  privarnos de nuestro empleo en la fábrica.\autocite[p~.184]{anscombe2011plato:twocuts}}

La dependencia en estas explicaciones que \emph{`deben de ser'} para justificar
nuestras proposiciones nos impide tener una concepción clara del panorama de la
realidad. Anscombe lo ilustra de este modo:
\citalitlar{La descripción detallada de la distribución de manchas de color en
  un canvas no nos revela la imagen que está en él, sin embargo, si dices:
  ``Pero la imagen es \emph{también}. \emph{¿En qué consiste?} \emph{debe de}
  haber ahí algo más además de pintura en un canvas''--estarías embarcandote en
  una busqueda ilusoria. El vasto número de cosas que conocemos y hacemos y que
  indagamos son como la imagen en el canvas. Las realidades acerca de nuestro
  conocer, nuestro hacer y nuestro indagar son enormemente interesantes; pero
  necesidades de tipo absolutamente \emph{a priori} no pueden ser encontradas
  para justificar nuestras aserciones.\autocite[p.~185]{anscombe2011plato:twocuts}}

En contraste con este uso engañoso de la necesidad hay un uso inocuo de ese
\emph{`deber de'} que ocurre en regiones más especializadas. Un ejemplo
notable es el modo en el que hacemos cuentas en una serie, o el modo en el que
calculamos el valor de una variable $\mathcal{Y}$ dado un cierto valor para
$\mathcal{X}$ en una fórmula. Podríamos decir que la serie está determinada ya
de antemano por la fórmula, al calcularla sólo ponemos en tinta, por así
decirlo, la parte de la serie que estamos computando. Aquí no estamos
exactamente manufacturando una necesidad, sino más bien
\citalitinterlin{tratando de formular el ideal de una necesidad que está siendo
  imitada por los cálculos cuando son de resultados que son `determinados', en
  ese sentido inofensivo de necesidad \autocite[p.~185]{anscombe2011plato:twocuts}}.

Pues bien, para Wittgenstein la pregunta sobre la manera adecuada de continuar
una serie es la misma pregunta sobre cómo usar la palabra `rojo'. Así como la
serie tiene una cierta determinación por su formula, la palabra tiene una cierta
determinación por su uso. En este sentido, conocer el significado de una palabra
consiste en comprender ese \emph{`deber de'} que determina su futura aplicación.

Este camino en la busqueda del significado de las proposiciones puede ser
ocasión de otra inclinación:
\citalitinterlin{Aquí no estamos tan tentados de inventar o manufacturar
  necesidades, sino de descansar conformes con las que creemos haber
  comprendido.\autocite[p.~185]{anscombe2011plato:twocuts}}

Esta podría ser nuestra actitud respecto de nuestro uso de las proposiciones
hasta que alguien nos interrumpe con una pregunta sobre la necesidad de estar en
lo correcto cuando usamos una palabra de cierto modo. Esta pregunta sería
esceptica sólo para aquel que asumiera que sus presunciones son
irrefragablemente correctas y la base del significado y la
verdad.\autocite[cfr.~][p.~186]{anscombe2011plato:twocuts}

El impacto de Wittgenstein en la filosofía es para Anscombe una ruta que permite
llegar a concepciones verdaderas. Nos permite ver la pintura con claridad.
Siguiendo la anterior ilustración:

\citalitlar{Es un impedimento para llegar a mirar la imagen, si estás aferrado a
  la convicción de que debes una de dos; extraer la imagen desde la descripción
  del color de cada mancha de pintura en una fina cuadrícula extendida sobre
  esta, o que debes tener una teoría de lo que la imagen es aparte de lo que esa
  descripción describe. Si renuncias a ambas inclinaciones podrás llegar a mirar
  a la pintura y haciéndolo podrías encontrarte lleno de asombro. O, como
  Wittgenstein una vez lo dijera, puedes encontrarte a tí mismo `caminando en
  una montaña de maravillas'}

Según Anscombe el método general adecuado de discutir los problemas filosóficos
propuesto por Wittgenstein consiste en mostrar que la persona no ha provisto
significado (o referencia) para ciertos signos en sus expresiones.\autocite[cf.
p. 151]{anscombe1959iwt} Creía que el camino que lleva a formular estos problemas está
frecuentemente trazado por la mala comprensión de la lógica de nuestro lenguaje.

Cada obra de Wittgenstein representa su esfuerzo de superar estas confusiones
y propone un método para remediarlas. Su primera propuesta plantea que el modo
de aclarar las confusiones de los problemas filosóficos consiste en
identificar en el lenguaje el límite de lo que expresa pensamiento; lo que
queda al otro lado de esta frontera sería simplemente sinsentido. En otras
palabras: \citalitinterlin{
  % What can be said at all
  Lo que puede ser dicho en absoluto puede ser dicho claramente; y de lo que uno
  no puede hablar, de eso, uno debe guardar silencio}.
\autocite[prefacio]{wittgenstein1922tractatus}

Con esta expresión Wittgenstein resumió el sentido del \emph{`Tractatus
Logico\=/Philosophicus'}.

\subsection{El gran tratado de Wittgenstein}

\ifdraft{\subsubsection{De Manchester a Cambridge}}{}
Los primeros esfuerzos de Wittgenstein por escribir una obra sobre filosofía
habían comenzado en 1911. En otoño de ese año en lugar de continuar sus estudios
de ingeniería en Manchester, determinó irse a Cambridge donde Bertrand Russell
ofrecía sus lecciones.

Asistió a un término de lecciones con Russell y al finalizar no estaba seguro de
abandonar la ingeniería por la filosofía, se cuestionaba si verdaderamente tenía
talento para ella. Consultó a su nuevo profesor al respecto y éste le pidió que
escribiera algo para ayudarle a hacer un juicio.

En enero de 1912 Wittgenstein regresó a Cambridge con un manuscrito que
demostraba auténtica agudeza filosófica. Convencido de su gran capacidad,
Russell alentó a Ludwig a continuar dedicándose a la filosofía. Este apoyo fue
crucial para Wittgenstein, hecho puesto de manifiesto por el gran empeño con el
que trabajó en sus estudios aquel curso. Al finalizar el termino Russell alegaba
que Ludwig había aprendido todo lo que él podía enseñarle.\autocite[cap. 3 loc
865]{monk1991duty}

\ifdraft{\subsubsection{A Noruega a Resolver los problemas de la lógica}}{}
Después de una temporada en Cambridge llena de eventos y desarrollos
Wittgenstein anunció en septiembre de 1913 sus planes de retirarse para
dedicarse exclusivamente a trabajar en resolver los problemas fundamentales de
la lógica. Su idea era irse a Noruega, a algún lugar apartado, ya que pensaba
que en Cambridge las interrupciones obstaculizarían su trabajo.\autocite[cap. 4
loc 1844]{monk1991duty}

\ifdraft{\subsubsection{La Gran Guerra}}{} 

El trabajo en Noruega fue escabroso. En el verano de 1914 interrumpió su tarea
para tomar un receso en Viena.\autocite[cap. 5 loc 2154]{monk1991duty} Había planificado
regresar a Noruega después del verano, sin embargo la tensión entre las
potencias europeas, agravada desde el atentado de Sarajevo a finales de junio de
aquel año, detonó en el estallido de la Gran Guerra. El 7 de agosto de 1914
Wittgenstein se enlistaba como voluntario al servicio militar. Sería en las
trincheras donde continuría su tratado filosófico.

El 22 de octubre de 1915 Wittgenstein escribió a Russell desde el taller de
artillería en Sokal, al norte de Lemberg, con lo que sería una primera versión
de su libro.\autocite[cf. p.84]{wittgenstein2012letters} 

En 1918 se le otorgó a Wittgenstein un largo periodo de excedencia entre julio y
septiembre. En ese tiempo pudo terminar su libro. Culminado el trabajo, ofreció
una copia a Frege y le llevó otra copia a Paul Engelmann. También intentó su
publicación, y todavía estaba esperando respuesta de la editorial cuando tuvo
que regresar al frente en Italia. En octubre le llegaron noticias de que la
publicación había sido rechazada. Al final del mes fue hecho prisionero de
guerra. Estuvo en un campamento en Como y en enero fue trasladado a Cassino. El
13 de marzo, escribió a Russell\autocite[cf. p.268]{mcguinness}: 
\citalitlar{He escrito un libro llamado ``Logisch-Philosophische Abhandlung''
  que contiene todo mi trabajo de los últimos seis años. Creo que finalmente he
  resuelto todos nuestros problemas. Esto puede sonar arrogante, pero no puedo
  evitar creerlo. Terminé el libro en agosto de 1918 y dos meses más tarde fui
  hecho 'Prigioniere'.\autocite[p.89]{wittgenstein2012letters}}

\ifdraft{\subsubsection{Aire de Misticismo}}{}
En junio de aquel año logró enviar el manuscrito del libro a Russell por medio
de John Maynard Keynes quien intervino con las autoridades italianas para
permitir el envío seguro del texto\autocite[p.90 y 91]{wittgenstein2012letters}. El 26
de agosto de 1919 fue oficialmente liberado de sus funciones
militares\autocite[p.277]{mcguinness} y en diciembre finalmente pudo encontrarse
con Russell en la Haya. De aquel encuentro Russell escribe:
\citalitlar{Había sentido un sabor a misticismo en su libro, pero me quedé
    asombrado cuando vi que se ha convertido en un completo místico. Lee a gente
    como Kierkergaard y Angelus Silesius, y ha contemplado seriamente el
    convertirse en un monje. Todo comenzó con ``Las variedades de la experiencia
    religiosa'' de William James y creció durante el invierno que pasó solo en
    Noruega antes de la guerra cuando casi se había vuelto loco. Luego, durante
    la guerra, algo curioso ocurrió. Estuvo de servicio en el pueblo de Tarnov
    en Galicia, y se encontró con una librería que parecía contener solamente
    postales. Sin embargo, entró y encontró que tenían un sólo libro: Los
    Evangelios abreviados de Tolstoy. Compró el libro simplemente porque no
    había otro. Lo leyó y releyó y desde entonces lo llevaba siempre consigo,
    estando bajo fuego y en todo momento. Aunque en su conjunto le gusta menos
    Tolstoy que Dostoeweski. Ha penetrado profundamente en místicos modos de
    pensar y sentir, aunque pienso que lo que le gusta del misticismo es su
    poder para hacerle dejar de pensar. No creo que realmente se haga monje, es
    una idea, no una intención. Su intención es ser profesor. Repartió todo su
    dinero entre sus hermanos y hermanas, pues encuentra que las posesiones
    terrenales son una carga. \autocite[p. 112]{wittgenstein2012letters}}

\ifdraft{\subsubsection{En busca de una experiencia religiosa}}{}
Cuando Wittgenstein se enlistó en el ejercito para la guerra en 1914 tenía
motivaciones más complejas que la defensa de su patria.\autocite[loc2276]{monk1991duty}
Sentía que, de algún modo, la experiencia de encarar la muerte le haría mejor
persona. Había leído sobre el valor espiritual de confrontarse con la muerte en
``Las variedades de la experiencia religiosa'':
\citalitlar{No importa cuales sean las fragilidades de un hombre, si estuviera
    dispuesto a encarar la muerte, y más aún si la padece heroicamente, en el
    servicio que éste haya escogido, este hecho le consagra para
    siempre.\autocite[loc 2295]{monk1991duty}}

Wittgenstein esperaba esta experiencia religiosa de la guerra.
\citalitinterlin{Quizás}, escribía en su diario, \citalitinterlin{La cercanía de
    la muerte traerá luz a la vida. Dios me ilumine.}\autocite[loc2295]{monk1991duty}
La guerra había coincidido con esta época en la que el deseo de convertirse en
una persona diferente era más fuerte aún que su deseo de resolver los problemas
fundamentales de la lógica.\autocite[loc2305]{monk1991duty}

\ifdraft{\subsubsection{La Principal Contienda}}{}
Esta transformación sorprendió a Russell en aquel encuentro en la Haya, pero
además fue motivo de confusión en la tarea de entender el Tractatus. Cuando
Russell recibió el manuscrito en agosto escribió a Wittgenstein cuestionando
algunos puntos difíciles del texto. En su carta observaba: 
\citalitlar{Estoy convencido de que estás en lo correcto en tu principal
    contienda, que las proposiciones lógicas son tautologías, las cuales no son
    verdad en el mismo modo que las proposiciones
    sustanciales.\autocite[p.96]{wittgenstein2012letters}}

Esta interpretación del texto se ajusta bien a la importancia que había tenido
esta cuestión en las discusiones entre Russell y Wittgenstein. Así lo expresaba
Russell en ``Introducción a la Filosofía Matemática'' publicado en mayo de aquel
año: 
\citalitlar{
  % The importance of “tautology” for a definition of
  % mathematics was pointed out to me by my former pupil Ludwig
  % Wittgenstein, who was working on the problem. I do not know whether he
  % has solved it, or even whether he is alive or dead.
    La importancia de la ``tautología'' para una definición de las
    matemáticas me fue señalada por mi ex-alumno Ludwig Wittgenstein, quien
    estaba trabajando en el problema. No sé si lo ha resuelto, o siquera si está
    vivo o muerto.\autocite[p.205 n\,1]{russell1919intromathphi}} 

Sin embargo para el Tractatus la cuestión sobre las proposiciones lógicas como
tautologías no es ya el tema principal, sino que enfatiza otra cuestión, así
corrige Wittgenstein en su respuesta a la carta de Russell:
\citalitlar{Ahora me temo que realmente no has captado mi principal contienda,
    para lo cual todo el asunto de las proposiciones lógicas es sólo corolario.
    El punto principal es la teoría sobre lo que puede ser expresado por
    proposiciones ---es decir, por el lenguaje--- (y, lo que viene a ser lo mismo,
    aquello que puede ser pensado) y lo que no puede ser expresado por medio de
    proposiciones, sino solamente mostrado; lo cual, creo, es el problema
    cardinal de la filosofía\ldots \autocite[p. 98]{wittgenstein2012letters}}

Esta respuesta de Wittgenstein no solo pone de manifiesto su cambio de enfoque,
sino que ofrece una clave para introducirse en su obra. 

%CUARTA CUESTIÓN: LA ``DOCTRINA'' DEL TRACTATUS
%1. La filosofía como actividad
%2. El pensamiento como representación
%3. Los polos de verdad y falsedad de las proposiciones
%4. La diferencia ente decir y mostrar

\subsection{Las elucidaciones del Tractatus}
% Este párrafo resume los cuatro puntos del Tractatus que se desglosarán en los
% próximos párrafos
Desde las proposiciones principales del Tractatus queda claro que el tema
central del libro es la conexión entre el lenguaje, o el pensamiento, y la
realidad.  
% 1.Filosofía como actividad
En este nexo es donde la actividad filosófica ha de buscar esclarecer el
pensamiento.
% 2.El pensamiento como representación
La tesis básica sobre esta relación consiste en que las proposiciones, o su
equivalente en la mente, son imágenes de los hechos.
% 3.Las proposiciones como proyecciones con polos de verdad-falsedad
La proposición es la misma imagen tanto si es cierta como si es falsa, es decir,
es la misma imagen sin importar que lo que se corresponde a ésta es el caso que
es cierto o no. El mundo es la totalidad de los hechos, a saber, de lo
equivalente en la realidad a las proposiciones verdaderas.
% 4.La distinción entre el decir y el mostrar
Sólo las situaciones que pueden ser plasmadas en imágenes pueden ser afirmadas
en proposiciones. Adicionalmente hay mucho que es inexpresable, lo cual no
debemos intentar enunciar, sino más bien contemplar sin palabras.\autocite[cf.
p.19]{anscombe1959iwt}

\subsubsection{La filosofía como actividad}

La filosofía es la actividad que tiene como objeto la clarificación lógica
de los pensamientos.\autocite[4.112 p. 52]{wittgenstein1922tractatus} El problema de muchas de
las proposiciones y preguntas que se han escrito acerca de asuntos filosóficos
no es que sean falsas, sino carentes de significado. Wittgenstein continúa: 
\citalitlar{4.003~En consecuencia no podemos dar respuesta a preguntas de este
    tipo, sino exponer su falta de sentido. Muchas cuestiones y proposiciones de
    los filósofos resultan del hecho de que no entendemos la lógica de nuestro
    lenguaje. (Son del mismo genero que la pregunta sobre si lo Bueno es más o
    menos idéntico a lo Bello). Y así no hay que sorprenderse ante el hecho de
    que los problemas más profundos realmente no son problemas.\autocite[4.003
    p. 45]{wittgenstein1922tractatus}} 

Es así que el precipitado de la reflexión filosófica que el Tractatus recoge no
pretende componer un cuerpo doctrinal articulado por proposiciones filosóficas,
sino más bien ofrecer `elucidaciones' que sirven como etapas escalonadas y
transitorias que al ser superadas conducen a ver el mundo correctamente. Este
esfuerzo hace de pensamientos opacos e indistintos unos claros y con límites
bien definidos.\autocite[cf. 4.112 y 6.54]{wittgenstein1922tractatus} 
La posibilidad de llegar a una visión clara del mundo es fruto de la posibilidad
de lograr aclarar la lógica del lenguaje. El lenguaje, a su vez, está compuesto
de la totalidad de las proposiciones, y éstas, cuando tienen sentido,
representan el pensamiento.\autocite[cf. 4 y 4.001]{wittgenstein1922tractatus} 
Sin embargo, el mismo lenguaje que puede expresar el pensamiento lo disfraza:

\citalitlar{4.002~El lenguaje disfraza el pensamiento; de tal manera que de la
    forma externa de sus ropajes uno no puede inferir la forma del pensamiento
    que estos revisten, porque la forma externa de la vestimenta esta elaborada
    con un propósito bastante distinto al de favorecer que la forma del cuerpo
    sea conocida.}

El intento de llegar desde el lenguaje al pensamiento por medio de las
proposiciones con significado es el esfuerzo por conocer una imagen de la
realidad. El pensamiento es la imagen lógica de los hechos, en él se contiene la
posibilidad del estado de las cosas que son pensadas y la totalidad de los
pensamientos verdaderos es una imagen del mundo.\autocite[cf.][3 y
3.001]{wittgenstein1922tractatus}

\subsubsection{El pensamiento como representación}

El pensamiento es representación de la realidad por la identidad existente entre
la posibilidad de la estructura de una proposición y la posibilidad de la
estructura un hecho:

\citalitlar{Los objetos ---que son simples--- se combinan en situaciones
    elementales. El modo en el que se sujetan juntos en una situación tal es su
    estructura. Forma es la posibilidad de esa estructura. No todas las
    estructuras posibles son actuales: una que es actual es un `hecho
    elemental'. Nosotros formamos imágenes de los hechos, de hechos posibles
    ciertamente, pero algunos de ellos son actuales también. Una imagen consiste
    en sus elementos combinados en un modo específico. Al estar así presentan a
    los objetos denominados por ellos como combinados específicamente en ese
    mismo modo. La combinación de los elementos de la imagen ---la combinación
    siendo presentada--- se llama su estructura y su posibilidad se llama la
    forma de representación de la imagen.   
    Esta `forma de representación' es la posibilidad de que las cosas están
    combinadas como lo están los elementos de la imagen.
    \footnote{\cite[cf.][p.~171]{simplicity}; \cite[n.~2.15]{wittgenstein1922tractatus}}}

La representación y los hechos tienen en común la forma lógica:
\citalitlar{2.18~Lo que toda representación, de una forma cualquiera, debe tener
    en común con la realidad, de manera que pueda representarla ---cierta o
    falsamente--- de algún modo, es su forma lógica, esto es, la forma de la
    realidad.\autocite[p.34]{wittgenstein1922tractatus}}

\subsubsection{Las proposiciones como proyecciones con polos de verdad-falsedad}
    La imagen de la realidad se convierte en proposición en el momento en que
    nosotros correlacionamos sus elementos con las cosas
    actuales.\autocite[cf.~][p.\,73]{anscombe1959iwt}
    La condición de posibilidad de entablar dicha correlación es la relación interna
    entre los elementos de la imagen en una estructura con
    sentido.\autocite[cf.~][p.~68]{anscombe1959iwt}
    De este modo:
    \citalitlar{5.4733~Frege dice: Toda proposición legítimamente construida tiene
        que tener un sentido; y yo digo: Toda proposición posible está legítimamente
        construida, y si ésta no tiene sentido es sólo porque no hemos dado
        significado a alguna de sus partes constitutivas. (Incluso cuando pensemos
        que lo hemos hecho.)\autocite[p.~78]{wittgenstein1922tractatus}}

    La proposición expresa el pensamiento perceptiblemente por medio de signos.
    Usamos los signos de las proposiciones como proyecciones del estado de las cosas
    y las proposiciones son el signo proposicional en su relación proyectiva con el
    mundo. A la proposición le corresponde todo lo que le corresponde a la
    proyección, pero no lo que es proyectado, de tal modo, que la proposición no
    contiene aún su sentido, sino la posibilidad de expresarlo; la forma de su
    sentido, pero no su contenido.\autocite[cf.~][3.1,3.11-3.13]{wittgenstein1922tractatus} 

    La proposición no `contiene su sentido' porque la correlación la hacemos nosotros,
    al `pensar su sentido'. Hacemos esto cuando usamos los elementos de la
    proposición para representar los objetos cuya posible configuración estamos 
    reproduciendo en la disposición de los elementos de la proposición. Esto es lo
    que significa que la proposición sea llamada una imagen de la
    realidad.\autocite[cf.~][p.69]{anscombe1959iwt}  

    Toda proposición-imagen tiene dos acepciones. Puede ser una descripción de
    la existencia de una configuración de objetos o puede ser una descripción de la
    no-existencia de una configuración de objetos.\autocite[cf.~][p.~72]{anscombe1959iwt} 
    %Es una peculiaridad de la proyección el que de ésta y del método de proyección
    %se puede decir qué es lo que se está proyectando, sin que sea necesario que tal
    %cosa exista físicamente.\autocite[cf.~][p.~72]{anscombe1959iwt} 
    %La idea de la proyección es peculiarmente apta para explicar el carácter de una
    %proposición como teniendo sentido independientemente de los hechos, como
    %inteligible aún antes de que se sepa que es cierta; como algo que concierne lo
    %que se puede cuestionar sobre si es verdad, y saber lo que se pregunta antes de
    %conocer la respuesta.\autocite[cf.~][p.~73]{anscombe1959iwt}
    Esta doble acepción es el resultado de que la proposición-imagen puede ser una
    proyección hecha en sentido positivo o negativo.\autocite[cf.~][p.~74]{anscombe1959iwt} Esto
    queda ilustrado en una analogía:

    \citalitlar{4.463~La proposición, la imagen, el modelo, son en el sentido
        negativo como un cuerpo solido, que restringe el libre movimiento de otro:
        en el sentido positivo, son como un espacio limitado por una sustancia
        sólida, en la cual un cuerpo puede ser colocado.\autocite[p.~63]{wittgenstein1922tractatus}}

    De este modo toda proposición-imagen tiene dos polos; de verdad y de falsedad.
    Las tautologías y las contradicciones, por su parte, no son imagenes de la
    realidad ya que no representan ningún posible estado de las cosas. Así continúa
    la ilustración anterior:

    \citalitlar{4.463~Una tautología deja abierto para la realidad el total infinito
        del espacio lógico; una contradicción llena el total del espacio lógico no
        dejando ningún punto de él para la realidad. Así pues ninguna de las dos
        puede determinar la realidad de ningún modo.\autocite[p.~78]{wittgenstein1922tractatus}}

    La verdad de las proposiciones es posible, de las tautologías es cierta y de las
    contradicciones imposible. La tautología y la contradicción son los casos límite
    de la combinación de signos ---específicamente--- su
    disolución.\autocite[cf.~][4.464 y 4.466]{wittgenstein1922tractatus} Las tautologías son
    proposiciones sin sentido (carecen de polos de verdad y falsedad), su negación son
    las contradicciones. Los intentos de decir lo que sólo puede ser mostrado
    resultan en esto, en formaciones de palabras que carecen de sentido, es decir,
    son formaciones que parecen oraciones, cuyos componentes resultan no tener
    significado en esa forma de oración.\autocite[cf.~][p.~163~\S2]{anscombe1959iwt}.

\subsubsection{La distinción entre el decir y el mostrar}
      La conexión entre las tautologías y aquello que no se puede decir, sino
      mostrar, es que éstas ---siendo proposiciones lógicas sin sentido--- muestran
      la 'lógica del mundo'.\autocite[cf.~][p.~163~\S3]{anscombe1959iwt}. Esta 'lógica del
      mundo' o 'de los hechos' es la que más prominentemente aparece en el Tractatus
      entre las cosas que no pueden ser dichas, sino mostradas. Esta lógica no solo
      se muestra en las tautologías, sino en todas las proposiciones. Queda exhibida
      en las proposiciones diciendo aquello que pueden decir.

      La forma lógica no puede expresarse desde el lenguaje, pues es la forma del
      lenguaje mismo, se hace manifiesta en éste, no es representativa de los objetos
      y tampoco puede ser representada por signos, tiene que ser mostrada:
      \citalitlar{4.0312~La posibilidad de las proposiciones se basa en el principio de
          la representación de los objetos por medio de signos. Mi pensamiento
          fundamental es que las ``constantes lógicas'' no son representativas. Que la
          lógica de los hechos no puede ser representada.\autocite[p.~48]{wittgenstein1922tractatus}}

      La lógica es, por tanto, trascendental, no en el sentido de que las
      proposiciones sobre lógica afirmen verdades trascendentales, sino en que todas
      las proposiciones muestran algo que permea todo lo decible, pero es en sí mismo
      indecible.\autocite[cf.~][p.~166 \S2]{anscombe1959iwt}

      Otra cuestión notoria entre aquello que no puede ser dicho, sino mostrado es la
      cuestión acerca de la verdad del solipsismo. Los limites del mundo son los
      límites de la lógica, lo que no podemos pensar, no podemos pensarlo, y por tanto
      tampoco decirlo. Los límites de mi lenguaje significan los límites de mi
      mundo.\autocite[cf~.][5.6~y~5.61]{wittgenstein1922tractatus} De este modo:
      \citalitlar{5.62~[\ldots]Lo que el solipsismo \emph{significa}, es ciertamente
          correcto, sólo que no puede ser \emph{dicho}, pero se muestra a sí
          mismo. Que el mundo es \emph{mi} mundo, se muestra a sí mismo en el hecho
          de que los limites del lenguaje (de \emph{aquel} lenguaje que yo
          entiendo) significan los límites de mi
          mundo.\autocite[cf~.][p.~89]{wittgenstein1922tractatus}} 

      Así como la lógica del mundo y la verdad del solipsismo quedan mostradas,
      también, las verdades éticas y religiosas, aunque no expresables, se manifiestan
      a sí mismas en la vida. 

      Existe, por tanto lo inexpresable que se muestra a sí mismo, esto es lo
      místico.\autocite[cf.~][6.522]{wittgenstein1922tractatus}

      De la voluntad como sujeto de la ética no podemos
      hablar\autocite[cf.~][6.423]{wittgenstein1922tractatus}. El mundo es independiente de nuestra
      voluntad ya que no hay conexión lógica entre ésta y los hechos.
      La voluntad y la acción como fenómenos, por tanto, interesan sólo a la
      psicología.\autocite[cf.~][p.171 \S3]{anscombe1959iwt}

      El significado del mundo tiene que estar fuera del
      mundo\autocite[cf.~][6.41]{wittgenstein1922tractatus} y Dios no se revela \emph{en} el
      mundo\autocite[cf.~][6.432]{wittgenstein1922tractatus}. 
      Esto se sigue de la teoría de la representación; una proposición y su negación
      son ambas posibles, cuál es verdad es accidental.\autocite[cf.~][p.170 \S4]{anscombe1959iwt}
      Si hay un valor que valga la pena para el mundo tiene que estar fuera de lo que
      es el caso que es; lo que hace que el mundo tenga un valor no-accidental tiene
      que estar fuera de lo accidental, tiene que estar fuera del
      mundo.\autocite[cf.~][6.41]{wittgenstein1922tractatus} 

      Finalmente, aplicar el límite de lo que puede ser expresado a la actividad
      filosófica significa que:
      \citalitlar{6.53~El método correcto para la filosofía sería este. No decir nada
          excepto lo que pueda ser dicho, esto es, proposiciones de la ciencia
          natural, es decir, algo que no tiene nada que ver con la filosofía: y luego
          siempre, cuando alguien quiera decir algo metafísico, demostrarle que no ha
          logrado dar significado a ciertos signos en sus proposiciones. Este método
          sería insatisfactorio para la otra persona ---no tendría la impresión de que
          le estuviéramos enseñando filosofía--- pero este método sería el único
          estrictamente correcto.\autocite[p. 107--108]{wittgenstein1922tractatus}}

        La frase usada para describir la obra: \citalitinterlin{de lo que no podemos
          hablar, de eso hemos de guardar silencio}, pertende expresar tanto una
        verdad logico-filosófica como un precepto ético. El sinsentido que resulta de
        tratar de decir lo que sólo puede ser mostrado no sólo es lógicamente
        insostenible, sino éticamente indeseable.\autocite[cf.~][p.~156]{monk1991duty}
        Wittgenstein explicó esta finalidad ética de su obra en una carta a Ludwig von
        Ficker de este modo: \citalitlar{[\ldots] el punto del libro es ético. Hubo un
          tiempo en que quise ofrecer en el prefacio algunas palabras que ya no están
          ahí, éstas, sin embargo, quiero escribirtelas ahora porque pueden ser clave
          para ti: quise escribir que mi trabajo consiste en dos partes: en la que
          está aquí, y en todo lo que \emph{no} he escrito. Y precisamente esta
          segunda parte es la importante. Pues lo ético es delimitado desde dentro,
          por así decirlo, por mi libro; y estoy convencido de que,
          \emph{estrictamente} hablando, éste SÓLO puede ser delimitado de este modo.
          En resumen, pienso que: todo de lo que \emph{muchos} están
          \emph{mascullando} hoy en día, lo he definido en mi libro al mantenerme en
          silencio sobre ello.\autocite[p.~22-23]{howtoread}}

\subsection{Del \emph{Tractatus} a \emph{Investigaciones Filosóficas}}
Aún como prisionero en Cassino, Wittgenstein había decidido que a su regreso a
Viena se prepararía para ser profesor de escuela
elemental\autocite[cf.~][p.~158]{monk1991duty}. Fue liberado en agosto de 1919
y, según su propósito, se enlistó en el \emph{Lehrerbildungsanhalt} para recibir
formación en enseñanza. En septiembre de 1920 estaría en el pequeño pueblo de
Trattenbach en Noruega como profesor de escuela elemental. A lo largo de aquel
año intentó sin éxito la publicación del Tractatus y tuvo que dejar la tarea en
manos de Russell al partir hacia Noruega. En 1922 el libro de Wittgenstein sería
finalmente publicado.

En 1929 Wittgenstein regresó a la tarea filosófica. Presentó el \emph{Tractatus
  Logico\=/Philosophicus} como su tesis doctoral en Cambridge y recibió un
fellowship de cinco años en ``Trinity College''. Comenzó sus lecciones en el
periodo Lent de 1930. Terminó su fellowship en el curso 1935-1936 y tomó un
receso. Regresó a ofrecer lecciones en Cambridge en 1938. El 11 de febrero de
1939 fue nombrado a la cátedra de filosofía en Cambridge tras el retiro de
G.\,E.\,Moore. Permanecería en esta labor hasta su retiro en 1947.

Cuando Wittgenstein regresó a la filosofía en 1929 encontró grandes defectos en
las tesis lógicas y metafísicas del Tractatus. Esto le llevó a abandonar
principios relacionados con la idea central de su teoría de la imagen. Rechazó
la noción de los objetos simples como significados de los nombres simples, la
concepción de los hechos y las ideas como compartiendo la forma lógica o la
propuesta de que toda inferencia lógica depende de una composición de función de
verdad\autocite[cf.][p.~44]{bakerhacker2014rules}.

Una idea que no abandonó inicialmente, sino que reforzó, fue la del lenguaje
como un cálculo de reglas. En el \emph{Tractatus} había propuesto que cualquier
lenguaje posible tiene como base la estructura de un cálculo lógico--sintáctico
conectado a la realidad por nombres lógicamente apropiados cuyos significados
son objetos simples que constituyen la sustancia del mundo. Su argumentación
ahora es que cualquier lenguaje posible es un calculo autónomo de reglas y el
significado es otorgado a los signos primitivos indefinibles, en parte, por
medio de definiciones ostensivas. Las muestras empleadas en la definición
ostensiva son ellas mismas parte de los medios de representación. Según esto el
significado de una expresión no es un objeto en la realidad, sino que consiste
en la totalidad de las reglas que determinan su uso dentro del cálculo del
lenguaje. El significado de una palabra es su lugar en la gramática, su rol en
el cálculo\autocite[cf.~][p.44]{bakerhacker2014rules}.

En 1931 empezaría a proponer que el hablar un lenguaje es un sistema
multifacético de actividades gobernadas por reglas, abandonando la idea de que
hay un sistema de reglas que rigen un cálculo que está debajo y sostiene todo
discurso significativo. Entonces fue dejando de hablar del cálculo del lenguaje
y empezó a usar el calcular como una analogía para describir el uso del
lenguaje. La operación de hacer un cálculo y seguir las reglas que éste sugiere
guarda relación con el modo en el que operamos cuando usamos el lenguaje y
seguimos las reglas que éste nos presenta.

Subsecuentemente abandonaría incluso la analogía del cálculo. En 1930 había
empezado a comparar el lenguaje con un juego de ajedrez al reflexionar en el
debate entre Frege y formalistas matemáticos como Heine, Thomae y
Weyl.\autocite[cf.~][p.134]{bakerhacker2014rules} En 1931 empezó a preferir esta
analogía a la del cálculo. Al igual que al hacer un cálculo, al jugar un juego
se siguen reglas que gobiernan las operaciones realizadas dentro de éste. Las
palabras son como piezas de ajedrez, las explicaciones de los significados de
las palabras son como las reglas del ajedrez y los significados de las palabras
son como el potencial de movimiento y captura de las piezas de ajedrez. La
analogía del ajedrez para hablar del lenguaje resultó fructífera precisamente
porque se trata de un juego. El uso de las expresiones es involucrarse en un
juego de lenguaje.

Fue así como Wittgenstein fue cambiando su atención hacia los usos de las
expresiones en las prácticas humanas y su investigación empezó a girar en torno
al hablar como una actividad integrada en la vida humana, entretejida con otra
multitud de acciones, actividades, relaciones y respuestas.

Wittgenstein llegará a sostener, como queda atestiguado en \emph{Investigaciones
  Filosóficas} \S90, que la filosofía es una investigación gramática en la que
los problemas filosóficos son resueltos por medio de la descripción del uso de
las palabras, clarificando la gramática de las expresiones y tabulando reglas.
Con Moore, se podría objetar que gramática es el tipo de cosas que se enseña a
los niños en la escuela, por ejemplo: <<no se dice ``tres hombres \emph{estaba}
en el campo'', sino ``tres hombres \emph{estaban} en el campo''>> ---eso es
gramática. Y ¿qué tiene que ver eso con filosofía? A lo que Wittgenstein
contestaría: efectivamente este ejemplo no tiene nada que ver con filosofía, ya
que en él todo está claro. Pero qué tal si dijéramos ``Dios el Padre, Dios el
Hijo y Dios el Espíritu Santo''; ¿\emph{estaban} en el campo o \emph{estaba} en
el campo?\autocite[cf.~][55]{bakerhacker2014rules}

%Esta metodología resultante de la evolución en la filosofía de Wittgenstein será
%en la que tomaría parte Elizabeth Anscombe cuando llegó a sus lecciones en 1942.

\subsection{El nuevo método de Wittgenstein}
En sus reflexiones sobre los fundamentos de las matemáticas entre 1937 y 1938,
Wittgenstein plantea la siguiente pregunta: \citalitinterlin{¿Cómo sé que al
  calcular la serie $+2$ debo escribir `$20004$, $20006$' y no `$20004$,
  $20008$'?}

La pregunta tiene que ver con el modo en el que actuamos según una regla. Al
calcular esta serie se ha ofrecido $+2$ como norma para el cálculo. Ahora la
pregunta es cómo se sabe qué hacer con ese conocimiento previo cuando llega el
momento de ponerlo en acto. Si se ha comprendido la guia inicial se tendrá
certeza sobre qué hacer después de $20004$, y esta certeza no implica que
$20006$ haya quedado determinado de antemano, sino que en que ante cualquier
número ofrecido se tiene la capacidad de ofrecer el siguiente. Entonces
continua:
\citalitlar{<<¿Pero entonces en qué consiste la peculiar inexorabilidad de las
  matemáticas?>> ---¿No será acaso la inexorabilidad con la que dos sigue a uno
  y tres a dos un buen ejemplo? ---Pero presuntamente esto significa: se sigue
  así en la \emph{serie de números cardinales}; pues en una serie distinta se
  seguiría de un modo distinto. Pero ¿acaso esta serie no está definida
  precisamente por esta secuencia? ---<<¿Hay que suponer que esto significa que
  cualquier modo en el que una persona cuente es igualmente correcto, y que
  cualquiera puede contar en el orden que quiera?>> ---Probablemente no lo
  llamaríamos `contar' si todo el mundo dijera los números uno después de otro
  \emph{de cualquier manera}; pero por supuesto esto no se trata simplemente de
  un problema sobre el nombre que se usa. Pues lo que llamamos `contar' es una
  parte importante de las actividades de nuestras vidas. Contar y calcular no
  son --por ejemplo-- un simple pasatiempo. Contar (y eso significa: contar
  \emph{así}) es una técnica que es empleada diariamente en las operaciones más
  variadas de nuestras vidas. Y por eso es que aprendemos a contar como lo
  hacemos: con prácticas interminables, con despiadada exactitud; por eso es que
  es inexorablemente insistido que hemos de decir `dos' después de `uno', `tres'
  después de `dos' y así sucesivamente. ---<<Pero entonces este contar es sólo
  un uso; ¿acaso no hay alguna verdad que se corresponda con esta secuencia?>>
  La \emph{verdad} es que contar ha demostrado que paga. ---<<Entonces quieres
  decir que `ser verdad' significa: ser utilizable (o útil)?>> ---No, no eso;
  pero que no puede ser dicho de la serie de números naturales --y tampoco de
  nuestro lenguaje-- que es verdad, pero: que es utilizable, y, sobre todo que
  \emph{se usa de hecho}.\autocite[p.~37 \S4]{wittgenstein1956remmath}}

A la pregunta sobre cómo continuar la serie, Wittgenstein añade la observación:
\citalitinterlin{la pregunta <<¿cómo sé que este color es `rojo'?>> es similar.}
La cuestión planteada no solo tiene que ver con el modo en el que vamos según
una serie, sino con las operaciones que hacemos con las palabras. Tambíen con
las palabras hay una comprensión inicial de su uso que luego se aplica en cada
caso. ¿Cómo sé que en esta ocasión estoy empleando una expresión según la regla
que es su uso?

En \emph{Investigaciones Filosóficas} \S380 encontramos:
\citalitlar{¿Cómo reconozco que esto es rojo? ---``Veo que es \emph{esto}; y
  entonces sé que eso es lo que esto es llamado'' ¿Esto? ---¡¿Qué?! ¿Qué tipo de
  respuesta a esta pregunta tiene sentido? (Sigues girando hacia una explicación
  ostensiva interna.) No podría aplicar ninguna regla a una transición
  \emph{privada} desde lo que es visto a las palabras. Aquí las reglas realmente
  quedarían suspendidas en el aire; pues la institución para su aplicación esta
  ausente.}

Y añade en \S381: \citalitinterlin{¿Cómo reconozco que este color es rojo?
  ---Una respuesta sería: <<He aprendido [castellano]>>.} Ir según una regla es
ir según una costumbre, un uso, una institución; \citalitinterlin{entender una
  oración significa entender un lenguaje, entender un lenguaje significa dominar
  una técnica\autocite[p.~87 \S9]{wittgenstein1953phiinv}.} La gramática de la
expresión `seguir una regla' supone la existencia de una prática, una
regularidad, un comportamiento normativo. Sólo cuando esta red de
comportamientos está en juego se puede hablar de que existe una
regla\autocite[cf.~][p.~14]{bakerhacker2009understanding}. No es posible que
haya una sola persona que en una sola ocasión `siguió una regla', esta
consideración no es correspondiente con la gramática de la
expresión\autocite[cf.~][p.~87 \S1 199]{wittgenstein1953phiinv}.

Cuando Elizabeth Anscombe participó de estas discusiones en las clases con
Wittgenstein encontró una ruta para sus propias indagaciones filosóficas.
\citalitinterlin{En cierto punto Wittgenstein estaba discutiendo en sus clases
  la interpretación del letrero (sign-post), y estalló en mi que el modo en que
  vas según éste es la interpretación
  final.\autocite[p.~viii]{anscombe1981metaphysicsintro}} Un letrero es una
expresión de una regla ante la que hemos sido entrenados a reaccionar de un modo
particular. Pensar que se está siguiendo una regla no es seguir una regla, y por
eso no es posible seguir una regla `privadamente' \autocite[cf.~][p.87 /S1
202]{wittgenstein1953phiinv}. La interpretación definitiva de una expresión de
una regla es cómo se actua ante ella.

Durante sus estudios en Oxford Anscombe había rechazado con fuerza un realismo
representativo lockeano que insistía que los colores como ella los veía no son
parte del mundo externo. Como reacción contraria tendía a identificar estas
sensaciones con \emph{esto} (this), como si `azul' o `amarillo' fueran artículos
que `están ahí'. Esta noción también le parecía equivocada, pero no lograba
librarse de ella\autocite[cf.][210]{diamond2004crisscross}: \citalitlar{En otra
  ocasión salí con: <<Pero todavía quiero decir: ``Azul esta ahí''>>. Manos más
  veteranas sonrieron o rieron, pero Wittgenstein las detuvo tomándolo en serio,
  diciendo: <<Déjame pensar qué medicina necesitas\ldots>> <<Supón que tenemos
  la palabra \emph{`painy'}, como una palabra para la propiedad de ciertas
  superficies>>. La `medicina' fue efectiva\ldots}
  % y la historia ilustra la habilidad de Wittgenstein para comprender el
  % pensamiento que se le estaba siendo ofrecido en objeción.
\citalitlar{
  % Uno podría protestar, desde luego, que precisamente ésto es equivocado en la
  % asimilación que hace Locke de las cualidades secundarias al dolor: puedes
  % esbozar el funcionamiento de ``dolor'' como una palabra para una cualidad
  % secundaria, pero no puedes hacer la operación inversa. Pero la `medicina' no
  % implicaba que podrías.
  [\ldots] Si \emph{`painy'} fuera una palabra posible para una cualidad
  secundaria, ¿no podría el mismo motivo conducirme a decir: \emph{`painy'} está
  aquí que lo que me condujo a decir `azul' está aquí?
  % Mi expresión no significaba que `azul' es el nombre de esta sensación que
  % estoy teniendo, ni cambié a ese pensamiento.
  \autocite[p.~viii]{andcombe1981metaphysicsintro}}

¿Qué cambió en la comprensión del lenguaje para Anscombe?

\subsection{Investigaciones Filosóficas}
% Al igual que con la introducción al análisis presentado para el Tractatus
% resumimos en este parrafo los puntos que se trataran sobre Investigaciones
% Filosóficas.
Las primeras lineas del prefacio de \emph{Investigaciones Filosóficas} leen:
\citalitinterlin{Los pensamientos que publico en lo que sigue son el precipitado
  de investigaciones filosóficas que me han ocupado durante los últimos
  dieciseis años.} El prefacio fue escrito en 1945. 

Qué vamos a ver?

Estructura general según baker and hacker:

1-27a Explicación preliminar de concepcion agustiniana del lenguaje

27b-64 malentendidos acerca de los nombres y el uso de los nombres bajo la
concepción agustiniana

65-88 investigación sobre concepción de nombres simples ligados a objetos
simples que son los constituyentes últimos de la realidad

89-108 crítica de los principios metodológicos más profundos que guiaron el
tractatus y repudio de una concepción sublime de la filosofía y la investigación
lógica que lo informó

109-133 bosquejo de la nueva concepción de la filosofía y de sus métodos

133-142 transición desde la discusión de doble faz de la filosofía y la
subsecuente investigación sobre el comprender

143-184 contra una idea de que comprender es un estado que implica que la
aplicación está comprendida previo a su uso, esto para aclarar el status
categorial de comprender

185-242 complementa la secuencia de comentarios anterior y clarifica la relación
entre entender una expresión, el significado o uso de esta y la explicación de
lo que significa, que es una regla para su uso

243-315
incorpora los argumentos sobre el lenguaje privado

316-362 on thinking

363-397 on imagination

398-427 mundo subjetivo de sensación experiencia y imaginación, el yo y auto
referencia y conceptos de conciencia y auto conciencia

428-65 el malentendido de que el significado de los signos, su habilidad para
representar lo que representan depende de procesos mentales de pensar

466-490 discusión breve sobre el problema de la justificación del razonamiento
inductivo

491-570 examen de significado y otros problemas relacionados

571-693 conceptos psicológicos




   Entre las primeras inquetudes filosóficas de Elizabeth estaban las preguntas:
   <<¿Qué conozco?>>, <<¿Cómo conozco?>>, <<¿Qué veo verdaderamente?>>. Sus
   incipientes reflexiones en torno a estas cuestiones le llevaron a formular sus
   propias explicaciones:

   \citalitlar{ Como una adolescente cautivada por algunos problemas filosóficos,
     entre ellos ¿Qué conozco? ¿Y cómo?, y sin saber siquiera que este tipo de
     investigación se llama `filosofía', y sin haber escuchado nunca las palabras
     `definición ostensiva', formulé una explicación como esta: Yo sabía lo que
     algunas palabras significan por definición verbal, hasta que llegaba a algunas
     que representaban cosas a las que yo podía apuntar. Las cualidades sensibles
     eran fáciles, pero me preocupaba mucho por `gatos' y `tazas'. Cuando escuché
     más tarde la palabra `definición ostensiva' respondí inmediatamente a ella
     como que expresaba una idea familiar; yo misma había estado dándome
     definiciones ostensivas hacía un año o dos a modo de ilustrar mi teoría del
     conocimiento; si hubiera entrado en conversación con alguien al respecto (que
     no recuerdo haber hecho) hubiera señalado cosas o las hubiera mencionado como
     objetos familiares de mi experiencia.\autocite[p.~244]{POD}}

   Su reflexión sobre la precepción fue pasando por varias étapas:

   \citalitlar{ Estaba segura de que veía objetos, como paquetes de cigarrillos o
     tazas o\ldots cualquier cosa más o menos sustancial servía. Pero pienso que
     estaba concentrada en artefactos, como otros productos de la vida urbana, y
     los primeros ejemplos mas naturales que me llamaron la atención fueron
     `madera' y el cielo. Éste último me golpeó en el centro porque estaba diciendo
     dogmáticamente que uno debe conocer la categoría de objeto del que uno está
     hablando -- si era un color o un tipo de cosa, por ejemplo; \emph{eso}
     pertenecía a la lógica del termino que uno estaba usando. No podía ser una
     cuestión de descubrimiento empírico que algo perteneciera a una categoría
     distinta. El cielo me detuvo.}

   \citalitlar{Por años gastaría el tiempo, en cafés, por ejemplo, mirando
     fijamente objetos y diciéndome: <<Veo un paquete. Pero ¿qué veo realmente?
     ¿Cómo puedo decir que veo aquí algo mas que una extensión amarilla?>>
     \autocite[p.~viii]{anscombe1981metaphysicsintro}}


   \citalitlar{Aún mientras hacía \emph{Honour Mods}, y por tanto antes de entrar
     en mi curso de estudios de grado en filosofía, asístí a las lecciones de
     H.~H.~Price en percepción y fenomenalismo. Las encontré intensamente
     interesantes. Ciertamente, de toda la gente que escuché en Oxford, él fue
     quien inspiró mi respeto; el único que encontré que merecía la pena escuchar.
     Esto no era porque estuviera de acuerdo con él, en efecto, solía sentarme
     rasgando mi vestido a tiras porque quería rebatir tanto de lo que él decía.
     Aún así, me parecía que lo que decía era absolutamente sobre lo que había que
     hablar. El único libro suyo que encontre muy bueno fue \emph{Hume's Theory of
       the External World} lo leí de un golpe desde la primera oración a la última.
     [\ldots] Fue él quien despertó mi intenso interés por el capítulo de Hume
     ``Del escepticismo con respecto a los sentidos''.}






   Las lecciones con Wittgenstein eran directas y con franqueza. Esta metodología
   carente de cualquier parafernalia era inquietante para algunos, pero fue
   tremendamente liberadora para Elizabeth. La metodología terapéutica empleada por
   Wittgenstein fue exitosa donde otros métodos más teoréticos habían fallado en
   liberarla de confusiones filosóficas.\autocite[loc 9853 Chapter 4, Section 24,
   \S5]{monk1991duty}

   Una confusión significativa que Anscombe tuvo que combatir fue en torno a la
   percepción.

   Siempre odié el fenomenalismo y me sentía atrapada por él. Yo no podía ver cómo
   salir de él, pero no lo creía. No era suficiente señalar las dificultades sobre
   él, las cosas que Russell econtraba incorrectas con él, por ejemplo. La fuerza,
   el nervio central de éste permanecía vivo y rabiaba terriblemente. Fue sólo en
   las lecciones con Wittgenstein en 1944 que vi el nervio siendo extraido, el
   pensamiento central "Tengo esto, y defino `amarillo' (digamos) como esto''
   siendo efectivamente atacado.



   se había sentido atrapada por el fenomenalismo porque había respondido
   fuertemente en contra de un realismo representativo Lockeano que insistía que
   los colores como ella los veía no eran genuinamente parte del mundo externo.

   Pero, encontrandose insistiendo que azul (este azul), o amarillo (esto), están
   allí, allí fuera, ella estaba en un camino que llevaba, o parecía llevar, en una
   dirección en la que ella no quería seguir, hacia una lectura del mundo como él
   mismo hecho de estos artículos del los que ella estaba consciente de este modo,
   un mundo construido de los 'esto's: hecho de el amarillo del que ella era
   consciente al fijarse en el paquete de cigarillos frente a ella, y de otras
   cosas como esta.

   Nosotros debemos entonces imaginarnosla, sentada en las lecciones de
   Wittgenstein, escuchando la discusión de las definiciones ostensivas que podemos
   pensar que nos damos a nosotros mismos.


%SECCIÓN 2: TRUTH
%%SECCIÓN 1: 
\section{Actividad Filosófica de Elizabeth Anscombe}

\subsection{Los primeros arduos esfuerzos}

\ifdraft{\subsubsection{De Wittgenstein a Anscombe}}{} 
En el 1929 Wittgenstein presentó el Tractatus Logico\=/Philosophicus como su
tesis doctoral en Cambridge. Ese mismo año fue designado como profesor en
``Trinity College'', allí estaría hasta 1936.

\ifdraft{\subsubsection{Causalidad reflexiones iniciales de Anscombe}}{}

Por aquella época de mediados de los 30 la joven Gertrude Elizabeth Margaret
Anscombe, andaba buscando un buen argumento que demostrara que todo lo que
existe tiene que tener una causa. ¿Por qué cuando algo ocurre estamos seguros de
que tiene una causa? Nadie sabía darle una respuesta.\autocite[cf.~][p.~vii
]{anscombe1981metaphysicsintro} Así, sin darse cuenta, se iniciaba en la ardua
tarea de la filosofía. Rigurosa y enérgica desde el principio.

El origen de su peculiar curiosidad por la causalidad se hallaba en una obra
llamada `Teología Natural' escrita por un jesuita del siglo XIX. Había llegado a
este libro motivada por su conversión a la Iglesia
Católica.\autocite[cf.~][p.~vii]{anscombe1981metaphysicsintro} El tratado le
resultó problemático en dos cuestiones.

La primera fue la doctrina de la \emph{`scientia media'}, según la cual Dios
tiene conocimiento, por ejemplo, de lo que alguien podría haber hecho si no
hubiera muerto cuando murió. A Elizabeth le parecía que lo que hubiera ocurrido
si lo que pasó no hubiera pasado simplemente no existe; no hay qué conocer. Y no
podía creer esto. Anscombe tuvo la oportunidad de discutir esta preocupación con
Richard Kehoe durante su preparación religiosa en su primer año en Oxford. La
dificultad para creer aquella doctrina le parecía un límite para aceptar la fe
católica. Richard le aclaró que no hacía falta que creyera en eso. Con el tiempo
entendió que se trataba de una discusión de escuela, en la que los jesuítas y
dominicos entablaron una ardua disputa y que la postura que ella había adoptado
era la qu había sido defendida por los
dominicos.\autocite[cf.~][p.~vii]{anscombe1981metaphysicsintro}

La segunda cuestión problematica la encontró en un argumento sobre la existencia
de la `Causa Primera'. El tratado ofrecía como preliminar al argumento una
demostración de un `principio de causalidad' según el cual todo cuanto existe
tiene que tener una causa. Anscombe notó, escasamente escondido en una premisa,
un presupuesto de la conclusión del propio argumento. Aquel \emph{petitio
  principii} le pareció un simple descuido y resolvió, por tanto, escribir una
versión mejorada de la demostración. Durante los siguientes dos o tres años
produjo unas cinco versiones que le parecían satisfactorias, sin embargo
eventualmente descubría que contenían la misma falacia, cada vez disimulada más
astutamente. Todo este esfuerzo lo realizó sin ninguna enseñanza formal en
filosofía, incluso su último intento de argumento lo hizo antes de estudiar
`Greats'.\autocite[cf.~][p.~vii]{anscombe1981metaphysicsintro}

\ifdraft{\subsubsection{Oxford: La Percepción y el fenomenalismo de Price}}{}

Sus lecturas en torno a su conversión fueron motivo de más reflexiones. Esta
vez, como fruto de \emph{The Nature of Belief} de Martin D'Arcy, se interesó por
el tema de la percepción. Durante años ocupaba su tiempo, en cafeterías, por
ejemplo, mirando fijamente objetos, diciendose a sí misma: <<Veo un paquete.
¿Pero qué veo realmente? ¿Cómo puedo decir que veo algo más que una extensión
amarilla?>>\autocite[cf.~][p.~viii]{anscombe1981metaphysicsintro}

Al principio su impresión era que lo que veía eran objetos:
\citalitinterlin{Estaba segura de que veía objetos, como paquetes de cigarrillos
  o tazas o\ldots~cualquier cosa más o menos sustancial
  servía.}\autocite[p.~viii]{anscombe1981metaphysicsintro} Además creía que
debemos de conocer la categoría de un objeto cuando hablamos de él, eso
corresponde a la lógica del término usado para hablar del objeto y no de algún
descubrimiento empírico. Estas ideas, sin embargo, las había desarrollado
fijándose en artefactos urbanos. Los ejemplos de percepción de la naturaleza que
más la impactaron fueron `madera' y el cielo. Este último le hizo retractarse de
su creencia sobre el conocimiento lógico de la categoría de los
objetos.\autocite[cf.~][p.~viii]{anscombe1981metaphysicsintro}

Sus indagaciones sobre la percepción, así como le ocurrió con la causalidad,
fueron previas al periodo de `Greats' donde estudiaría formalmente la filosofía.
Ya desde `Mods' asistía a las lecciones de H.H.~Price sobre percepción y
fenomenalismo. De todos los que escuchó en Oxford fue quién le inspiró mayor
respeto, no porque estuviera de acuerdo con lo que decía, sino porque hablaba de
lo que había que hablar. El único libro suyo que le pareció realmente bueno fue
\emph{Hume's Theory of the External World} y lo leyó sin interrupción de
principio a fin. Fue Price quien despertó en ella un intenso interés por el
capítulo de Hume sobre ``Del escepticismo con respecto a los sentidos''. Aunque
le parecía que Price tendía a suavizar a Hume, el hecho de que escribiera sobre
él le parecia que era escribir sobre las cosas mismas que merecía la pena
discutir. Asncombe, sin embargo, odiaba el fenomenalismo y se sentía atrapada
por él, pero no sabía salir de él, o rebatirlo. La postura escéptica tampoco la
convencía como para adoptarla y no la dejaba satisfecha. Esta insatisfacción no
haría más que crecer en sus años en Oxford.
\autocites[cf.~][p.~viii]{anscombe1981metaphysicsintro}
[~y~][p.~26]{torralba2005accion}

\ifdraft{\subsubsection{En Cambrdige con Wittgenstein}}{}

  En las lecciones con Wittgenstein en Cambridge fue que el pensamiento central
  <<Tengo \emph{esto}, y defino `amarillo' como \emph{esto}>> fue efectivamente
  atacado. Anscombe misma lo narra usando dos ejemplos:

  \citalitlar{En cierto punto Wittgenstein estaba discutiendo en sus clases la
    interpretación del letrero (sign-post), y estalló en mi que el modo en que
    vas según éste es la interpretación
    final.\autocite[p.~viii]{andcombe1981metaphysicsintro}}

En \emph{Investigaciones Filosóficas} \S198 

toda interpretación queda sostenida en el aire junto con lo que interpreta, y no
puede darle a ésto ningún apoyo. Las interpretaciones por sí solas no determinan
el significado.

  Aquí Elizabeth se refiere a \autocite[p.~86~\S198]{PI}

  \citalitlar{En otra ocasión salí con: <<Pero todavía quiero decir: ``Azul esta
    ahí''>>. [\ldots] [Wittgenstein] dijo: <<Déjame pensar qué medicina
    necesitas\ldots>> <<Supón que tenemos la palabra \emph{`painy'}, como una
    palabra para la propiedad de ciertas superficies>>. La `medicina' fue efectiva
    [\ldots] Si \emph{`painy'} fuera una palabra posible para una cualidad
    secundaria, ¿no podría el mismo motivo conducirme a decir: \emph{`painy'} esta
    aquí que lo que me condujo a decir azul está aquí? Mi expresión no significaba
    que `azul' es el nombre de esta sensación que estoy teniendo, ni cambié a ese
    pensamiento.\autocite[p.~viii]{andcombe1981metaphysicsintro}}

  The issue's significance can be seen by considering how the argument is
  embedded in the structure of Philosophical Investigations. Immediately
  prior to the introduction of the argument (§§241f), Wittgenstein suggests
  that the existence of the rules governing the use of language and making
  communication possible depends on agreement in human behaviour—such as the
  uniformity in normal human reaction which makes it possible to train most
  children to look at something by pointing at it. (Unlike cats, which react
  in a seemingly random variety of ways to pointing.) One function of the
  private language argument is to show that not only actual languages but
  the very possibility of language and concept formation depends on the
  possibility of such agreement.

  Another, related, function is to oppose the idea that metaphysical
  absolutes are within our reach, that we can find at least part of the
  world as it really is in the sense that any other way of conceiving that
  part must be wrong (cf. Philosophical Investigations p. 230). Philosophers
  are especially tempted to suppose that numbers and sensations are examples
  of such absolutes, self-identifying objects which themselves force upon us
  the rules for the use of their names. Wittgenstein discusses numbers in
  earlier sections on rules (185–242). Some of his points have analogues in
  his discussion of sensations, for there is a common underlying confusion
  about how the act of meaning determines the future application of a
  formula or name. In the case of numbers, one temptation is to confuse the
  mathematical sense of ‘determine’ in which, say, the formula y = 2x
  determines the numerical value of y for a given value of x (in contrast
  with y > 2x, which does not) with a causal sense in which a certain
  training in mathematics determines that normal people will always write
  the same value for y given both the first formula and a value for x—in
  contrast with creatures for which such training might produce a variety of
  outcomes (cf. §189). This confusion produces the illusion that the result
  of an actual properly conducted calculation is the inevitable outcome of
  the mathematical determining, as though the formula's meaning itself were
  shaping the course of events.

  In the case of sensations, the parallel temptation is to suppose that they
  are self-intimating. Itching, for example, seems like this: one just feels
  what it is directly; if one then gives the sensation a name, the rules for
  that name's subsequent use are already determined by the sensation itself.
  Wittgenstein tries to show that this impression is illusory, that even
  itching derives its identity only from a sharable practice of expression,
  reaction and use of language. If itching were a metaphysical absolute,
  forcing its identity upon me in the way described, then the possibility of
  such a shared practice would be irrelevant to the concept of itching: the
  nature of itching would be revealed to me in a single mental act of naming
  it (the kind of mental act which Russell called ‘acquaintance’); all
  subsequent facts concerning the use of the name would be irrelevant to how
  that name was meant; and the name could be private. The private language
  argument is intended to show that such subsequent facts could not be
  irrelevant, that no names could be private, and that the notion of having
  the true identity of a sensation revealed in a single act of acquaintance
  is a confusion.




    \begin{revision}
       ``For a large class of cases of the employment of the word ‘meaning’—though not
       for all—this way can be explained in this way: the meaning of a word is its use
       in the language'' (PI 43). This basic statement is what underlies the change of
       perspective most typical of the later phase of Wittgenstein's thought: a change
       from a conception of meaning as representation to a view which looks to use as
       the crux of the investigation. 
       \end{revision}

      \begin{revision}
      Philosophical Investigations:
      --Undertake an investigation, leading, not to the construction of new and
      surprising theories or explanations, but the examination of our life with
      language. This is a grammatical investigation PI~\S90 
      --The ideas of explanation and discovery are misleading and inappropiate when
      applied to questions like: what is meaning?
      --We feel as if we had to repair a spider web with our fingers PI~\S106
      --PI~\S129
      --By putting details together in the right way or by using a new analogy or
      comparison to prompt us to see our practice of using language in a new light, we
      find that we achieve the understanding that we thought would only come with the
      construction of an explanatory account. RFGB, p.30
      --Philosopher's questions must be treated like an illness is treated. PI~\S133
      and \S255.
      --The aim of grammatical investigations is perspicious representation PI~\S122
      --Meaning is use.
      --The question of a philosopher is: how do I go about this?
      \end{revision}


      \begin{revision}
      What marks the transition from early to later Wittgenstein can be summed up as
      the total rejection of dogmatism, i.e., as the working out of all the
      consequences of this rejection. The move from the realm of logic to that of
      ordinary language as the center of the philosopher's attention; from an emphasis
      on definition and analysis to ‘family resemblance’ and ‘language-games’; and
      from systematic philosophical writing to an aphoristic style—all have to do with
      this transition towards anti-dogmatism in its extreme. It is in the
      Philosophical Investigations that the working out of the transitions comes to
      culmination. Other writings of the same period, though, manifest the same
      anti-dogmatic stance, as it is applied, e.g., to the philosophy of mathematics
      or to philosophical psychology.
      \end{revision}




      2. La metodología terapéutica y franca de Wittgenstein fue liberadora
      \begin{revision}



      El método terapeútico de Wittgenstein tuvo éxito en liberarla de confusiones
      filosóficas donde otras metodologíás mas teoréticas habían fallado. En sus
      estudios en St. Hugh's escuchaba a Price/ldots
      \end{revision}


      \begin{revision}
      Este modo de criticar una proposición desvelando que no expresa un pensamiento
      verdadero ilustra los principios propuestos en el \emph{Tractatus} y recuerda
      una de sus tesis más conocidas: 
      En el prefacio de las Investigaciones Filosóficas, con fecha de enero de 1945
      Wittgenstein dice que los pensamientos que publica en el libro son el
      precipitado de invetigaciones filosóficas que le han ocupado durante los pasados
      16 años. En enero 1929 Wittgenstein estaba regresando a Cambridge.
      \end{revision}


      \begin{revision}
      En ocasiones como esta la
      discusión con Wittgenstein llevaba a Anscombe a afirmaciones para las cuales no
      podía ofrecer mejor significado que los sugeridos por concepciones ingenuas. Una
      concepción así no es otra cosa que ausencia de pensamiento, pero su falta de
      significado no es evidente, sino que requiere de la fuerza de un `Copérnico'
      para ponerla en cuestión efectivamente.\autocite[cf. 151]{IWT} 
      \end{revision}

Anscombe conoció a Wittgenstein en los años culminantes de su pensamiento
     filosófico. 
     Al comienzo de sus lecciones en 1944 Wittgenstein escribía a su amigo Rush Rhees:
     \citalitinterlin{
         \ldots mis clases no han ido tan mal. Thouless esta asistiendo, y una mujer, 
         'Mrs so and so'
         que se llama a sí misma 
         'Miss Anscombe',
         que ciertamente es inteligente, aunque no del calibre de Kreisel.
         \autocite[p.~371]{cambridgeletters}
     }
     Un año mas tarde escribía a Norman Malcolm:
     \citalitinterlin{
         \ldots mi clase ahora es bastante grande, 19 personas. \ldots Smythies esta
         viniendo, y una mujer que es muy buena, es decir, más que solamente
         inteligente\ldots 
         \autocite[p.~388]{cambridgeletters}
     }
     Aquellos años no sólo creció en Wittgenstein la apreciación de la capacidad de
     Anscombe, sino que se afianzó entre ellos una estrecha amistad. 

     La influencia de Wittgenstein fue decisiva para el desarrollo filosófico de
     Elizabeth. Las lecciones con Wittgenstein eran directas y con franqueza. Esta
     metodología carente de cualquier parafernalia era inquietante para algunos,
     inspiradora para otros, pero fue tremendamente liberadora para
     ella.\autocite[loc 9853 Chapter 4, Section 24, \S5]{monk} Esta libertad
     quedaba demostrada en que Anscombe no se contentaba con repetir lo que decía
     Wittgenstein, sino que pensaba por sí misma; en esto precisamente era más fiel
     al espíritu de la filosofía que había aprendido de él. Sobre esta relación,
     Phillipa Foot, amiga de ambos, cuenta que durante mucho tiempo sostuvo
     objeciones a las afirmaciones de Wittgenstein, eventualmente, un comentario de
     Norman Malcom la hizo pensar que podía haber valor en lo que Wittgenstein decía.
     Cuestionó entonces a Anscombe: 
     ``¿Por qué no me dijiste?'', ella le contestó: ``Porque es importante que uno
     tenga sus resistencias''. Anscombe evidentemente pensaba ---continúa Foot: 
     \citalitlar{
         que un largo periodo de vigorosa objeción era la mejor manera de entender a
         Wittgenstein. Aun cuando era su amiga cercana y albacea literaria, y una de
         los primeros en reconocer su grandeza, nada podía ser más lejano de su
         carácter y modo de pensamiento que el discipulado.\autocite[p.~4]{teichmann}
     }

     Peter geach que dice que les ayudó que estudiaron otros filósofos antes de
     Wittgenstein.

\pnote{introducir algunos contrastes y relaciones entre
       Anscombe y Wittgenstein para explicar la incursión en la vida/pensamiento
       de W.}


%SECCIÓN 3: FAITH
%\subsection{Faith (1975)}

En \emph{Oscott College}, el seminario de la Archidiócesis de Birmingham, se comenzaron a celebrar las conferencias llamadas \emph{Wiseman Lectures} en 1971. Para estas lecciones ofrecidas anualmente en memoria de Nicholas Wiseman se invitaba un ponente que tratara algún tema relacionado con la filosofía de la religión o alguna materia en torno al ecumenismo\footnote{\cite[Cf.][7]{wisemanlects}.}.

El 27 de octubre de 1975, para la quinta edición de las conferencias, Anscombe presentó una lección titulada simplemente \emph{Faith}. Allí planteaba la siguiente cuestión: \blockquote[{\Cite[115]{anscombe1981erp:faith}}: \enquote{I want to say what might be understood about faith by someone who did not have it; someone, even, who does not necessarily believe that God exists, but who is able to think carefully and truthfully about it. Bertrand Russell called faith `certainty without proof'. That seems correct. Ambrose Bierce has a definition in his Devil's Dictionary: `The attitude of mind of one who believes without evidence one who tells without knowledge things without parallel.' What should we think of this?}]{Quiero decir qué es lo que puede ser entendido sobre la fe por alguien que no la tenga; alguien, incluso, que no necesariamente crea que Dios existe, pero que sea capaz de pensar cuidadosa y honestamente sobre ella. Bertrand Russell llamó a la fe `certeza sin prueba'. Esto parece correcto. Ambrose Bierce tiene una definición en su \emph{Devil's Dictionary}: `La actitud de la mente de uno que cree sin evidencia a uno que habla sin conocimiento cosas sin parangón'. ¿Qué deberíamos pensar de esto?}

El objetivo de Elizabeth, hablar de la fe para quien no tiene esa experiencia, determina un enfoque específico a su investigación. La descripción del fenómeno de la fe tiene que ser realizada razonablemente, de modo que pueda ser considerada por alguien \enquote*{que sea capaz de pensar cuidadosa y honestamente} sobre ella. Su estrategia consiste aquí de nuevo en una descripción de usos familiares de la palabra analizada que son articulados de tal manera que los patrones de estos usos sean estudiables\footnote{\cite[Cf.][12]{bakerhacker2009understanding}: \enquote{There is no room in philosophy for explanatory (hypothetico-deductive) theory, on the model of science, or for dogmatic (essentialist) thesis, on the model of metaphysics. Its task is grammatical clarifiaction that dissolves conceptual puzzlement and gives an overview or surveyable representation of a segment of the grammar of our language \textelp{} It describes the familiar uses of words and arranges them so that the patterns of their use become surveyable, and our entanglement in the web of grammar becomes perspicuous}.}. Se enfoca en un modo antiguo de usar la palabra `fe' en el que se le empleaba para decir \enquote*{creer a alguien que $p$}. `Fe humana' era creer a una persona humana, `fe divina' era creer a Dios\footnote{\cite[Cf.][2]{anscombe2008faith:tobelieve}: \enquote{At one time there was the following way of speaking: faith was distinguished as human and divine. Human faith was believing a mere human being; divine faith was believing God}.}. Así por ejemplo: \enquote*{Abrám creyó a Dios (\textgreek{ἐπίστευσεν τῷ Θεῷ}) y esto se le contó como justicia} (Gn 15,6). De tal modo que es llamado \enquote*{padre de la fe} (Cf.~Rm 4 y Ga 3,7). La pregunta \enquote*{¿qué es creer a alguien?} queda situada en el centro de este análisis\footnote{\cite[Cf.][116]{anscombe1981erp:faith}: \enquote{It is clear that the topic I introduced of \emph{believing somebody} is in the middle of our target}.}. Anscombe emplea esta noción para indagar sobre la estructura del creer que está relacionada con la dinámica de la fe. Creer a alguien implica ciertas presuposiciones, al hablar de la fe como \enquote*{creer a Dios que $p$} se le atribuye la misma implicación. La cuestión acerca de lo que es creer a alguien resultará de suficiente interés a Anscombe como para dedicarle su propio artículo y en esta investigación, sin duda, juega un papel importante.

Para exponer el desarrollo del análisis que Elizabeth recorre en su discusión podemos atender a tres movimientos principales realizados en su argumentación. Primero se fija en el carácter racional de la fe y recuerda una cierta apologética en la que se le atribuyó este carácter a los llamados preámbulos y el paso de estos a la fe misma; y establece que la designación correcta de estos `preámbulos de la fe', al menos para parte de ellos, es más bien `presuposiciones'. En segundo lugar describe cuáles son las presuposiciones implicadas en creer a una persona humana cuando esta comunica algo. En tercer lugar examina el fenómeno particular del creer cuando la comunicación viene de Dios.

Elizabeth nos introduce a su reflexión recordando una época en la que la racionalidad de la fe estuvo en el foco de cierta discusión teológica: \blockquote[{\Cite[113]{anscombe1981erp:faith}}: \enquote{There was in a preceding time a professed enthusiasm for rationality, perhaps inspired by the teaching of Vatican I against fideism, certainly carried along by the promotion of neo-thomist studies \textelp{} the word was that the Catholic Christian faith was \emph{rational}, and a problem, to those able to feel it as a problem, was how it was \emph{gratuitous}\,---\,a special gift of grace. Why would it \emph{essentially} need the promptings of grace to follow a process of reasoning?}]{Hubo en una época pasada un profuso entusiasmo por la racionalidad, quizás inspirado por la enseñanza del Vaticano~I contra el fideísmo, ciertamente sostenidos por la promoción de estudios neo-tomistas [\ldots] la noticia era que la fe Cristiana Católica era \emph{racional}, y el problema, para aquellos capaces de sentirlo como tal, era cómo era \emph{gratuita}\,---\,un don especial de la gracia. ¿Por qué tendría que ser \emph{esencialmente} necesaria la ayuda de la gracia para seguir un proceso de razonamiento?} 

Según la descripción de Anscombe este proceso de razonamiento consistía en una especie de cadena de demostraciones; se afirmaba a Dios, y luego la divinidad de Jesús, y después la institución de la Iglesia por él con el Papa a la cabeza con la encomienda de enseñar. Cada demostración permitía justificar la certeza de la verdad de las enseñanzas de la Iglesia\footnote{\cite[Cf.][113]{anscombe1981erp:faith}: \enquote{It was as if we were assured there was a chain of proof. First God. Then, the divinity of Jesus Christ. Then, \emph{his} establishment of a church with a Pope at the head of it and with a teaching commission from him. This body was readily identifiable. Hence you could demonstrate the truth of what the Church taught. Faith, indeed, is not the same thing as knowledge --- but that could be accounted for by the \emph{extrinsic} character of the proofs of the \emph{de fide} doctrines. \textelp{} For matters which were strictly `of faith' intrinsic proofs were not possible, and that was why faith contrasted with `knowledge'}.}. Elizabeth argumenta que esta breve descripción representa una postura quizás más `extravagante'\footnote{\cite[Cf.][113]{anscombe1981erp:faith}: \enquote{This is a picture of the more extravagant form of this teaching. A more sober variation would relate to the Church that our Lord established. In this variant one wouldn't identify the church by its having the Pope, but otherwise; and one would discover that it had a Pope and that that was all right}.}, y otras variantes más sobrias enfatizaban más la figura de la Iglesia, o la divinidad de Jesús\footnote{\cite[Cf.][113-114]{anscombe1981erp:faith}: \enquote{A yet more sober variant would have avoided trading on the cultural inheritance for which the name of Jesus was so holy that it was easy to go straight from the belief in God to belief in Jesus as God's Son}.}. A juicio de Anscombe esta actitud más sobria o crítica ante aquellos que pretendían defender la razonabilidad de la fe como una quasi demostrabilidad sirvió en beneficio de la veracidad y la honestidad\footnote{\cite[Cf.][114]{anscombe1981erp:faith}: \enquote{The `sober variants' would have a disadvantage for the propagandists of the rationality (near demonstrability) of faith --- though a great advantage in respect of honesty and truthfulness}.}. Ciertamente estas opiniones presentaban problemas. Era obvio que identificar la Iglesia católica que conocemos con la Iglesia que Cristo instituyó no era tarea fácil y necesitaba conocimiento y técnica\footnote{\cite[Cf.][114]{anscombe1981erp:faith}: \enquote{The disadvantage was that no one could suppose it quite easy for anyone to see that what Jesus established was matched by the Catholic Church that we know. \textelp{} it was \emph{obvious} that learning and skill would be required to make the identification}.}. Entonces ¿qué carácter tiene la certeza atribuida a la fe? \blockquote[{\Cite[114]{anscombe1981erp:faith}}: \enquote{The so-called preambles of faith could not possibly have the sort of certainty that \emph{it} had. And if less, then where was the vaunted rationality?}]{Los llamados preámbulos de la fe no podrían tener el tipo de certeza que \emph{esta} tiene. Y si es menos, entonces ¿dónde está la racionalidad proclamada?}. 

Otro problema tenía que ver con la fe de los doctos y los sencillos, ¿aquellos que no conocen estos argumentos tienen un tipo de fe inferior a los doctos? Por otra parte, los que han estudiado ¿realmente conocen todas estas cosas? Ser racional en tener fe implicaba sostener la creencia de que el conocimiento estaba ahí para argumentar y demostrar la verdad de Dios, de Cristo y de la Iglesia, quizá repartido entre algunos expertos o al menos de manera teorética. Todo esto hacía problemáticas estas opiniones\footnote{\cite[Cf.][114]{anscombe1981erp:faith}: \enquote{the implication was that the knowledge was there somehow, perhaps scattered through different learned heads, perhaps merely theoretically and abstractly available. In the belief that this was so, one was being rational in having faith. But then it had to be acknowledged that all this was problematic --- and so adherence to faith was really a matter of hanging on, and both its being a \emph{gift} and its \emph{voluntariness} would \emph{at this point} be stressed}.}.

Anscombe describe brevemente estas discusiones y este modo de hacer apologética que fue empleado en el pasado y ya no se usa en las discusiones de su época. Esto, dice, \blockquote[{\Cite[114]{anscombe1981erp:faith}}: \enquote{not necessarily because better thoughts about faith are now common; there is a vacuum where these ideas once were prominent}.]{no necesariamente porque sean ahora más comunes pensamientos mejores sobre la fe; hay un vacío en donde estas ideas antes fueron prominentes}. Sin embargo opina que no hay que lamentar que estas opiniones hayan pasado, y añade: \blockquote[{\Cite[114]{anscombe1981erp:faith}}: \enquote{They attached the character of `rationality' entirely to what were called the preambles and to the passage from the preambles to faith itself. But both these preambles and that passage were in fact an `ideal' construction \textelp{} `fanciful', indeed dreamed up according to prejudices: prejudices, that is, about what it is to be reasonable in holding a belief}.]{Estas atribuían el carácter de `racionalidad' por entero a lo que se llamaron los preámbulos y al paso de estos preámbulos a la fe misma. Pero tanto estos preámbulos como ese paso eran realmente una construcción `ideal' \textelp{} `imaginaria', ciertamente soñada de acuerdo a prejuicios: esto es, prejuicios sobre qué es lo que es ser razonable en sostener una creencia}.

De acuerdo al objetivo trazado al inicio de su discusión, Anscombe busca presentar una descripción del carácter racional de la fe libre de estos prejuicios. En el centro de su propuesta está la comprensión de `fe' como `creer a $x$ que $p$' y, partiendo de esto, el valor de los presupuestos involucrados en creer una comunicación. Comienza, entonces, proponiendo un ejemplo: \blockquote[{\Cite[114]{anscombe1981erp:faith}}: \enquote{You receive a letter from someone you know, let's call him Jones. In it, he tells you that his wife has died. You believe him. That is, you now believe that his wife has died because you believe \emph{him}. Let us call this just what it used to be called, ``human faith''. That sense of ``faith'' still occurs on our language. ``Why'', someone may be asked, ``do you believe such-and-such?'', and he may reply ``I just took it on faith\,---\,so-and-so told me''}.]{Recibes una carta de alguien que conoces, llamémosle Jones. En ella te dice que su esposa ha muerto. Tu le crees. Esto es, ahora crees que su esposa ha muerto porque le crees a él. Llamemos a esto justo como solía ser llamado, ``fe humana''. Este sentido de ``fe'' todavía ocurre en nuestro lenguaje. ``Por qué'', se le puede preguntar a alguien, ``crees esto y aquello?'', y podría responder ``Lo tomé en buena fe\,---\,fulano me dijo''}. 

Al especificar este uso de `fe', Elizabeth busca justificar que la designación más adecuada para los llamados `preámbulos' de la fe, al menos para parte de ellos, es `presuposiciones'. En el ejemplo propuesto hay tres creencias implicadas en haberle creído a Jones, estas \blockquote[{\Cite[114]{anscombe1981erp:faith}}: \enquote{three convictions or assumptions are, logically, pressupositions that \emph{you} have if your belief that Jones' wife has died is a case of your believing Jones}.]{tres convicciones o supuestos son, lógicamente, presuposiciones que \emph{tú} tienes si tu creencia de que la esposa de Jones ha muerto es un caso de que crees a Jones}.

Al creerlo presupones primero que tu amigo Jones existe, segundo, que la carta viene verdaderamente de él, y tercero, que esto que crees es verdaderamente lo que la carta dice. Estas son presuposiciones tuyas, el que puedas llegar a creer la comunicación de la carta no presupone estas tres cosas de hecho, sino que tú crees estas tres cosas.

Ahora bien, `fe' en la tradición en la que ese concepto se origina se refiere a `fe divina' y significa `creer a Dios'. Según esta acepción la fe es absolutamente cierta, puesto que es creer a Dios y, si las presuposiciones son ciertas, conlleva creer sobre los mejores fundamentos a uno que habla con conocimiento perfecto. Lo problemático aquí sería en qué consiste creer a Dios.
%, pero antes de indagar más sobre esto, Anscombe estudia con más detalle las presuposiciones relacionadas con creer a una persona humana.

Después de discutir cómo puede atribuírsele a la fe algún carácter de racionalidad y haberse decidido por valorar las convicciones implicadas en la certeza que depositamos en lo que creemos porque creemos a alguien,
%Anscombe ahora nos adentra en el análisis de estas presuposiciones y la utilidad que puedan tener para comprender el fenómeno de la fe.
Anscombe se plantea algunas preguntas relacionadas con estas presuposiciones que discutiremos más adelante: (III, \S\ref{subsec:presups}, p.~\pageref{subsec:presups}). Aquí solo destacamos dos elementos adicionales sobre ellas discutidos por Anscombe.
%¿Qué es creer a alguien? Anscombe vuelve a su ejemplo. Creer a Jones, que su esposa ha muerto, ¿significa que el hecho de que Jones me diga esto es la \emph{causa} de mi creencia? o ¿significa que el hecho de que se comunique es mi \emph{evidencia} para creer en la muerte de su esposa? ¿Esto sería creer a Jones? No del todo. Puesto que podría ser que la comunicación llama mi atención sobre la cuestión, pero llego a la creencia por mi propio juicio. O puedo tomar lo que me están diciendo y pensar que la persona que me habla me está engañando y a la misma vez está equivocada en lo que me dice, entonces podría decir que creo lo que me dice porque me lo ha dicho, pero no estaría creyendo a la persona. Entonces ¿creer a alguien significa creer que la persona cree lo que me está diciendo? Ordinariamente asumimos esto, pero incluso puede imaginarse el caso en el que alguien me dice algo que cree, pero yo sé que en el origen de su creencia hay una falsedad y por tanto creo lo contrario de lo que esta persona cree y me dice, entonces tampoco estaría creyendole a ella. Sin embargo, en el caso de creer a un maestro, un profesor de historia por ejemplo, sería suficiente para creerle \emph{a él} que creas lo que dice porque lo ha dicho y piensas que no está mintiendo y piensas que lo que él cree es verdadero.
%Estas dificultades no aparecen si se puede establecer con certeza que la persona conoce lo que dice y no miente, sin embargo el tema de creer a alguien no es asunto sencillo. Hay, además, otras preguntas relacionadas con las presuposiciones involucradas en creer a alguien. Al creer lo que dice la comunicación presupones que Jones existe, que escribió la carta y que esta dice lo que has llegado a creer. Pero estos son tus presupuestos y no son condiciones de hecho. ¿Qué se puede decir del caso en el que de hecho no existe la persona que se cree que es quien se comunica? ¿Se puede decir que se está creyendo a Jones si es el caso que de hecho no existe? Si insistiéramos en decir que no se está creyendo en la persona que no existe, afirma Anscombe, \blockquote[{\Cite[117]{anscombe1981erp:faith}}: \enquote{you will deprive yourself of the best way of describing his situation: ``he believed this non-existent person''}]{te estarías privando de la mejor manera de describir esta situación: ``le creyó a esta persona no existente''}. De un antiguo que creyó en el oráculo del dios Apolo, por ejemplo, se puede decir efectivamente que creyó en Apolo\,---\,que no existe. Lo mismo se podría decir del caso en el que de hecho existe la persona, pero esta comunicación que se cree que viene de ella no proviene de ella de hecho.
%Dos elementos adicionales son destacados por Anscombe acerca de las presuposiciones.

Primero comenta que \blockquote[{\Cite[117]{anscombe1981erp:faith}}: \enquote{the presuppositions of faith are not themselves part of the content of what in a narrow sense is believed by faith}.]{las presuposiciones de la fe no son ellas mismas parte del contenido de lo que en un sentido estricto es creido por la fe}. En segundo lugar explica que hay también una \blockquote[{\Cite[118]{anscombe1981erp:faith}}: \enquote{difference between presuppositions of believing $N$ and believing such-and-such as coming from $N$. ``Pre-suppositions'' don't have to be temporarily prior beliefs}.]{diferencia entre las presuposiciones de creer a $N$ y creer esto o aquello como viniendo de $N$. Las ``pre-suposiciones'' no tienen que ser creencias temporalmente previas}. Elizabeth ilustra esto imaginando el caso en el que la carta dijera que viene de alguien: \enquote*{Esta es una carta de tu viejo amigo Jones}, y al leerla se ponga en duda esta afirmación, o incluso no se ponga en duda sino que se lea acríticamente, sin pensar en ello, entonces se cree lo que dice la carta, pero no se está contando con la credibilidad de Jones como garantía de que la carta viene de él, se tiene en cuenta lo que la carta dice, incluido el que viene de él, pero no se le está creyendo a él y en este sentido las presuposiciones y el contenido de lo que es la fe propiamente son distintos. 

Otra ilustración puede ser el caso en el que no se tiene un conocimiento previo de la persona que se comunica: \enquote*{Esto es de parte de un amigo desconocido\,---\,llámame $N$}. Imaginemos un prisionero que recibe una comunicación de esta naturaleza y en ella se le ofrecen ayudas para sus necesidades, no sabe si son genuinas, pero responde a la comunicación y recibe las ayudas prometidas. Este prisionero recibe otras comunicaciones que parecen ser de la misma persona y estas contienen nueva información. Al creer esta información el prisionero cree a $N$, pero su creencia en que $N$ existe y que las cartas vienen de él no son creer algo apoyándose en que $N$ lo ha dicho. Es en este sentido en el que \blockquote[{\Cite[118]{anscombe1981erp:faith}}: \enquote{the beliefs which \emph{are} cases of believing $N$ and the belief that $N$ exists are logically different}.]{las creencias que \emph{son} casos de creer a $N$ y la creencia de que $N$ existe son lógicamente diferentes}.

En todos estos ejemplos Anscombe ha recurrido a comunicaciones entre personas humanas. ¿Qué se puede decir del caso en que la comunicación viene de Dios? \blockquote[{\Cite[118]{anscombe1981erp:faith}}: \enquote{Suarez said that in every revelation God reveals that he reveals}.]{Suarez dijo que en cada revelación Dios revela que Él revela} y esto es como decir \blockquote[{\Cite[118]{anscombe1981erp:faith}}: \enquote{in every bit of information $N$ is also claiming (implicitly or explicitly, it doesn't matter which) that he is giving the prisioner information}.]{en cada pedazo de información $N$ está también declarando (implícita o explícitamente, no importa como) que está dando información al prisionero}. Y aquí hay una dificultad central en el asunto de la fe: \blockquote[{\Cite[118]{anscombe1981erp:faith}}: \enquote{In all other cases we have been considering, it can be made clear \emph{what} it is for someone to believe someone. But what can it mean ``to believe God''? Could a learned clever man inform me on the authority of his learning, that the evidence is that God has spoken? No. The only posssible use of a learned clever man is as a \emph{causa removens prohibens}. There are gross obstacles in the received opinion of my time and in its characteristic ways of thinking, and someone learned and clever may be able to dissolve these}.]{En todos los otros casos que hemos estado considerando, puede ser aclarado \emph{qué} es que alguien crea a alguien. Pero ¿qué puede significar ``creer a Dios''? ¿Podría un hombre docto e inteligente informarme sobre la autoridad de su conocimiento, que la evidencia es que Dios ha hablado? No. El único uso posible para un hombre docto e inteligente es a modo de \emph{causa removens prohibens}. Hay grandes obstáculos en la opinion aceptada en mi época y en sus característicos modos de pensar, y alguien con inteligencia y conocimiento podría ser capaz de disolverlos}.

Con esto llegamos al tercer esfuerzo de Anscombe por arrojar luz sobre este tema. ¿Qué estamos creyendo cuando creemos que Dios ha hablado? Para hablar sobre esto Elizabeth recurre a una noción rabínica llamada \emph{Bath Qol} o la `hija de la voz': \blockquote[{\Cite[118-119]{anscombe1981erp:faith}}: \enquote{You hear a sentence as you stand in a crowd\,---\,a few words out of what someone is saying perhaps: it leaps at you, it `speaks to your condition'. Thus there was a man standing in a crowd and he heard a woman saying ``Why are you wasting your time?'' He had been dithering about, putting off the question of becoming a Catholic. The voice struck him to the heart and he acted in obedience to it. Now, he did not have to suppose, nor did he suppose, that that remark was not made in the course of some exchange between the woman and her companion, which had nothing to do with him. But he believed that God had spoken to him in that voice. The same thing happened to St Augustine, hearing the child's cry, ``Tolle lege''}.]{Escuchas una oración mientras que estás en medio de una muchedumbre\,---\,algunas palabras de entre lo que alguien está diciendo: saltan hacia ti, `hablan a tu condición'. Así había un hombre que entre la muchedumbre escuchó una mujer que estaba diciendo ``¿Por qué estas desperdiciando tu tiempo?'' Había estado vacilando, ignorando la cuestión de hacerse católico. La voz le golpeó en el corazón y actuó en obediencia a ella. Ahora, él no tenía que suponer, ni de hecho supuso, que este comentario no fuera hecho en el curso de alguna conversación entre la mujer y su acompañante, la cuál no tenía nada que ver con él. Lo mismo ocurrió a San Agustín, al escuchar el grito del niño ``Tolle lege''}.

Ahora bien, todavía hace falta una aclaración adicional respecto de lo que significa decir que se cree que Dios habla. En los ejemplos anteriores estaba claro qué significa para alguien que \enquote*{cree a $X$} que \enquote*{$X$ está hablando}. Incluso en el caso de que no exista. Pero no es claro que Dios sea el que hable. Aquí, entender deidad como el objeto de adoración no es útil puesto que habría que definir adoración como el honor ofrecido a una deidad. En este sentido por `Dios' Anscombe no entiende el objeto de esta o aquella adoración; `Dios' no es un nombre propio, sino una `descripción definitiva' en el sentido técnico. Es decir es equivalente a `el uno y único dios verdadero'. Un ateo cree que Dios está entre los dioses que no son dioses, pero podría entender la identidad de `Dios' con `el uno y único dios'. En este sentido decir que Dios es el dios de Israel es decir que lo que Israel ha adorado como dios es `el uno y único dios verdadero'. Esto podría ser afirmado o negado por alguien incluso que considerara que esa expresión es vacía o no se refiere a nada.

Con esto, Anscombe llega a una descripción conclusiva: \blockquote[{\Cite[119-120]{anscombe1981erp:faith}}: \enquote{And so we can say this: the supposition that someone has faith is the supposition that he believes that something ---it may be a voice, it may be something he has been taught--- comes as a word from God. Faith is then the belief he accords to that word}.]{Y entonces podemos decir esto: la suposición de que alguien tiene fe es la suposición de que cree que algo ---puede ser una voz, puede ser algo que ha aprendido--- viene como una palabra de Dios. Fe es entonces la creencia que otorga a esa palabra}. Esto puede ser entendido por alguien que no tiene fe, sea que su actitud ante este fenómeno sea de reverencia, indiferencia u hostilidad. Esto además puede ser dicho en términos generales sobre el fenómeno de la fe. En el caso específico del que cree en Cristo: \blockquote[{\Cite[120]{anscombe1981erp:faith}}: \enquote{the Christian adds that such a belief is sometimes the truth, and that the consequent belief is only then what \emph{he} means by faith}.]{el cristiano añade que esta creencia es en ocasiones la verdad, y esta creencia consecuente es solo lo que \emph{él} entiende por fe}.

\vspace{2.83334em}
\vspace{1.41667em}
La premisa de que Anscombe entiende la fe como `saber testimonial' es una de las claves principales de nuestro estudio. La ruta de Elizabeth en este artículo ilustra 
el modo en que puede encontrarse una descripción del carácter testimonial de la revelación dentro de su obra. En esta reflexión ella parte de una descripción de la fe y sus presupuestos y nos conduce de tal modo que llegamos a preguntarnos sobre qué significa decir que se cree que Dios ha hablado. El camino para hablar de la revelación parte del análisis de la naturaleza de la creencia que es la fe como correlato de la comunicación divina (Cf. III, \S\ref{subsec:fecorrel}, p.~\pageref{subsec:fecorrel}).


%SECCIÓN 4: WHAT IS IT TO BELIEVE SOMEONE?
%\section{What is it to Believe Someone?}
\textsc{This text is in small caps}
Creer a alguien no es sólo un 


\chapter{La Concepción de G.\,E.\,M.\,Anscombe sobre el Testimonio}

\section{Acerca del Creer y Su Estructura}
\subsection{Un Peculiar Patrón de Argumento.}
Peter Geach dedica un breve apartado a Anscombe en su ``Autobiografía
Filosófica''. Ambos se dedicaban a la filosofía y era común que cuestionaran a
uno sobre el pensamiento del otro, sin embargo no era raro que no supieran cómo
contestar. Los dos tenían distintos intereses en sus investigaciones y tambíen
un estilo diferente al acercarse a los problemas filosóficos. Geach lo describe
así: \citalitlar{Como una filósofa madura, Elizabeth me parece ser una pensadora
  más intrépida que yo: es ella quien tiene ideas audaces y que a primera vista
  resultan meramente alocadas, a lo que en ocasiones he reaccionado con inicial
  indignación. (Cfr. sus escritos \emph{The Intentionality of Sensation} y
  \emph{The First Person}) Usualmente llego a pensar que estas audaces ideas son
  más justificables de lo que originalmente
  suponía\footnote{\cite[11]{geach1991philaut}: <<As a mature philosopher,
    Elizabeth strikes me as a more adventurous thinker than I am: it is she who
    gets bold and at first sight merely zany ideas, to which I sometimes reacted
    with initial outrage. (Cfr. her papers `The Intentionality of Sensation' and
    `The First Person') Usually I come to think these bold ideas are more
    defensible than I had originally supposed.>>}.} El mismo libro que recoge
estas memorias de Geach contiene un breve artículo de Elizabeth titulado
\emph{On a Queer Pattern of Argument}\footnote{\cite{anscombe1991aqp} En
  adelante la referencia al artículo será como aparece en:
  \cite{anscombe2015logic:qpa}} que ejemplifica adecudamente las palabras antes
referidas sobre su esposa.

En esta ocasión la consideración intrépida consitirá en indagar sobre la validez
de principios lógicos familiares aplicándolos a diversos ejemplos de
argumentaciones. El extraño patrón de argumento que da título a la investigación
queda expresado de este modo:
  \begin{adjustwidth}{1.2cm}{}
    1.\hspace{.459cm}Si $p$, entonces $q$.\\
    2.\hspace{.459cm}Si $r$, entonces no (si $p$ entonces $q$).\\
    3.\hspace{.459cm}Si no $p$ entonces $r$.\\
    $\therefore$\hspace{.459cm}$p$ y $q$.
  \end{adjustwidth}

  Se obtiene `no $r$' de las primeras dos premisas y entonces `$p$' de `no $r$'
  y la tercera premisa; con la primera premisa nuevamente y `$p$' obtenemos la
  conclusión.{\footnote{\cite[299]{anscombe2015logic:qpa} <<We get `not $r$'
      from the first two premises and then `$p$' from `not $r$' and the third;
      with the first one again this gives us the conclusion>>.}} Hecha esta
  descripción, Anscombe entonces invita a considerar el siguiente argumento
  construido según el patrón anterior:
  \begin{adjustwidth}{1.2cm}{}
    1.\hspace{.459cm}Si ese árbol cae, entonces interrumpirá el paso por el camino
    durante mucho tiempo.\\
    2.\hspace{.459cm}Eso no es verdad si hay una máquina para remover árboles
    funcionando.\\
    3.\hspace{.459cm}Si el árbol no cae, habrá una máquina para remover árboles
    funcionando.\\
    $\therefore$\hspace{.459cm}El árbol caerá e interrumpirá el paso por el camino
    durante mucho tiempo.
  \end{adjustwidth}

  ¿Qué resultado se obtiene si se intenta formar un juicio razonable o
  conocimiento desde este argumento? <<Si ese árbol cae entonces interrumpirá el
  camino y si hay una maquina para remover árboles funcionando entonces no será
  verdad que si el árbol cae entonces interrumpira el camino.>> (`Si $p$
  entonces $q$ y si $r$ entonces no [si $p$ entonces $q$]'). De esta conjunción
  se sigue `no habrá una maquina para remover árboles funcionando' (`no $r$'),
  pero ¿se podría considerar esta deducción un juicio razonable?. La segunda
  premisa se lee como arrojando duda sobre la primera, y la tercera premisa
  expresa la pertinencia de la segunda. Descartar la duda y afirmar la primera
  sugiere que ya se cree la primera premisa antes de evaluar la segunda. Pero en
  ese caso el argumento mismo no explicaría los fundamentos para la conclusión.
  Aún cuando se estuviera asintiendo a las otras dos premisas porque ya se cree
  la primera, estas trabajan junto a un hipotético para sostener la
  creencia\autocite[Cf.~][300]{anscombe2015logic:qpa}.

  Anscombe entonces propone: \citalitinterlin{Si todo esto es correcto, tenemos
    aquí un caso bastante interesante de una serie de proposiciones que implican
    una conclusión pero no son fundamentos posibles para llegar a esa
    conclusión}\footnote{\cite[300]{anscombe2015logic:qpa} <<If all this is
    right, we have here a rather interesting case of a set of propositions which
    entail a conclusion but are impossible grounds for coming to that
    conclusion>>}. El argumento no necesita que se juzgue como verdadera la
  conclusión o parte de ella para considerar verdadera alguna de las premisas,
  pero sí reclama que parte de la conclusión sea fundamento para aceptar la
  combinación de las premisas de modo que se pueda formar conocimiento o un
  juicio razonable\footnote{\cite[Cf.~][301]{anscombe2015logic:qpa}}.

  En este caso `$q$' no se sigue necesariamente de `$p$' y así, al no ser una
  verdad necesaria, sólo se puede aceptar la conjunción de las primeras dos
  premisas si se está independientemente seguro de que `no $r$'. No es común que
  estemos en la situación de pensar que `si $p$ entonces $q$' y que sólo por eso
  esté claro que `si $r$ entonces no (si $p$ entonces $q$)' y entonces poder
  deducir razonablemente de esto que `no $r$'. Es más común que al juzgar la
  conjunción de las primeras dos premisas, el antecedente de la segunda pierda
  fuerza. El punto de la segunda premisa es arrojar duda sobre la primera; la
  conjunción de la segunda premisa y la tercera refuerzan la pertinencia de la
  segunda. Sin embargo la segunda premisa sólo tendrá la fuerza de poner en duda
  la primera premisa ---y no al revés--- si, además de ser verdadera y
  pertinente, resulta imposible de descartar porque resulta necesario tomar en
  serio su antecedente `si $r$'\autocite[Cf.~][301]{anscombe2015logic:qpa}.

  Anscombe acuña la expresión `revocabilidad
  esencial'\footnote{\cite[Cf.~][301]{anscombe2015logic:qpa}: <<Then we have
    perhaps discovered the special character of (theoretical) hypotheticals
    whose consquents don't follow logically from their antecedents. We might
    call this character `essential defeasibility'>>.} para denominar al carácter
  especial de hipotéticos teoréticos cuyos consecuentes no se siguen lógicamente
  de sus antecedentes. En este caso esta característica es la que hace que
  incluso cuando `no $r$' se sigue de `si $p$ entonces $q$ y si $r$, entonces no
  (si $p$ entonces $q$)', no sería razonable deducir `no $r$' de esa conjunción.
  Elizabeth además observa que hay un gran número de juicios que son así. Al
  hacer una afirmación categórica con la seguridad apropiada, frecuentemente se
  descarta inmediatamente lo que la falsificaría sólo porque se sabe que ésta es
  verdadera. Sin embargo existe toda una clase de juicios como el que se ha
  analizado que al ser hechos no se descarta implicitamente como falso todo lo
  que los falsificaría\autocite[Cf.~][302]{anscombe2015logic:qpa}.

\subsection{¿Qué es creer a alguien?}

\subsubsection{Cuestión preliminar}
En el análisis anterior Anscombe ha descrito un escenario en el que combinar
varias premisas como conocimiento o juicio razonable resulta problemático a la
hora de justificar el fundamento de la conclusión apoyándose sólo en las
premisas y su relación lógica.

En su investigación titulada \emph{What is it to believe someone?} Anscombe
comienza describiendo otro escenario basado en el mismo argumento, proponiendo así
una situación que plantea la misma dificultad; también en el creer a alguien
el fundamento para la combinación de las premisas en un juicio razonable parece
estar más allá de las mismas premisas y sus relaciones. En esta ocasión cada
premisa aparece atribuida a una persona distinta y la conclusión a un cuarto
personaje. El pequeño relato aparece como sigue: \citalitlar{Había tres hombres,
  $A$, $B$ y $C$, hablando en cierta aldea. $A$ dijo: ``Si ese árbol cae,
  interrumpirá el paso por el camino durante mucho tiempo.'' ``No será así si
  hay alguna máquina para remover árboles funcionando'', dijo $B$. $C$ destacó:
  ``\emph{Habrá} una, si el árbol no cae.'' El famoso sofista Eutidemo, un
  extraño en el lugar, estaba escuchando. Inmediatamente dijo: ``Les creo a
  todos. Así que infiero que el árbol caerá e interrumpirá el paso por el
  camino.'' \footnote{\cite[1]{anscombe2008faith:tobelieve} <<There were three
    men, $A$, $B$ and $C$, talking in a certain village. $A$ said ``If that tree
    falls down, it'll block the road for a long time.'' ``That's not so if
    there's a tree-clearing machine working'', said $B$. $C$ remarked ``There
    \emph{will} be one, if the tree doesn't fall down.'' The famous sophist
    Euthydemus, a stranger in the place, was listening. He immediately said ``I
    believe you all. So I infer that the tree will fall and the road will be
    blocked.''>>}}

¿En qué está mal Eutidemo? Si se evalúa la lógica del argumento antes expuesto
no aparece ninguna contradicción, sin embargo hay algo extraño en la afirmación
``les creo a todos''. Si la lógica del argumento parece permitir que la
inferencia de Eutidemo sea posible, ¿por qué suena tan extraña la posibilidad de
que les crea a todos y juzgue esa conclusión?

\subsubsection{Naturaleza de la Investigación}
Es útil recordar aquí en términos generales el modo en el que Anscombe actua en
una investigación filosófica. Wittgenstein inicialmente describió el análisis
del lenguaje bajo la concepción de que la lógica conforma el orden que está
debajo y que sostiene todo lenguaje posible. El trabajo del filósofo es analizar
el lenguaje para sacar al descubierto el orden lógico que está debajo del
lenguaje ordinario y que es la forma de la realidad. Wittgenstein abandonó esta
concepción; en Investigaciones Filosóficas exclama: \citalitlar{Cuanto más de
  cerca examinamos el lenguaje actual, más crece el conflicto entre éste y
  nuestro requisito. (Pues la pureza cristalina de la lógica no era, por
  supuesto, algo que yo hubiera \emph{descubierto}: era un requisito.) El
  conflicto se hace intolerable; el requisito llega ahora a estar en peligro de
  tornarse vacuo. --- Nos hemos situado en hielo resbaladizo donde no hay
  fricción, y así, en cierto sentido, las condiciones son ideales; pero también,
  justo por eso, no somos capaces de caminar. Queremos caminar: así que
  necesitamos \emph{fricción}. ¡De vuelta al terreno
  escarpado!\footnote{\cite[\S107]{wittgenstein1953phiinv}: <<The more closely
    we examine actual language, the greater becomes the conflict between it and
    our requirement. (For the crystalline purity of logic was, of course, not
    something I had \emph{discovered}: it was a requirement.) The conflict
    becomes intolerable; the requirement is in danger of becoming vacuous. ---
    We have got on to slippery ice where there is no friction, and so, in a
    certain sense, the conditions are ideal; but also, just because of that, we
    are unable to walk. We want to walk: so we need \emph{friction}. Back to the
    rough ground!>>}.}

Los nombres, las proposiciones, el lenguaje, no tienen una forma esencial para
ser puesta al descubierto por el análisis, sino que son familias de estructuras
que están a plena vista y que pueden ser clarificadas por medio de la
descripción\autocite[Cf.~][12]{bakerhacker2009understanding}. Wittgenstein le
\citalitinterlin{da la vuelta a la
  busqueda}\autocite[\S108]{wittgenstein1953phiinv}, y trata a la lógica no como
lo que está debajo del lenguaje para ser descubierto, sino como
\citalitinterlin{una cuadrícula que imponemos sobre los argumentos para probar y
  demostrar su validez}\footnote{\cite[12]{bakerhacker2009understanding}: <<a
  grid we impose upon arguments to test and demonstrate their validity>>}.

Descartada la concepción sublime de la tarea filosófica, Wittgenstein describe
los problemas filosóficos como formas de malentendidos o falta de entendimiento
que pueden ser disueltos por medio de descripciones de los usos de las palabras.
La tarea de la filosofía es la \citalitinterlin{clarificación gramatical que
  disuelve la perplejidad conceptual y ofrece una visión amplia o representación
  estudiable de un segmento de la gramática de nuestro
  lenguaje}\footnote{\cite[12]{bakerhacker2009understanding}: <<grammatical
  clarification that dissolves conceptual puzzlement and gives an overview of or
  surveyable representation of a segment of the grammar of our language>>}. Esta
metodología, por tanto, no pretende ofrecer teorías explicativas fruto de la
deducción o la hipótesis; tampoco pretende ofrecer tesis dogmáticas o
esencialistas. Más bien busca describir usos familiares de las palabras y
ordenarlas de tal manera que los patrones de su uso sean
estudiables\autocite[Cf.~][12]{bakerhacker2009understanding}. La metodología de
Elizabeth está basada en esto.

\subsubsection{Investigación Gramática de `creer a $x$ que $p$'.}
Anscombe pone el interés de su investigación en la forma de la expresión `creer
a $x$ que $p$'\autocite[Cf.~][2]{anscombe2008faith:tobelieve}. Su análisis se va
desenvolviendo a lo largo de la descripción de los usos de la expresión.

\citalitinterlin{Si me dijeras `Napoleón perdió la batalla de Waterloo' y te
  digo `te creo' sería una
  broma}\footnote{\cite[4]{anscombe2008faith:tobelieve}: <<If you tell me
  `Napoleon lost the battle of Waterloo' and I say `I believe you' that is a
  joke.>>}. A primer golpe `creer a $x$ que $p$' parece que significa
simplemente creer lo que alguien me dice, o creer que lo que me dice es
verdadero. Sin embargo esto no es suficiente. Puede ser que ya crea lo que
alguien me venga a decir. Puede ser que la comunicación suscite que forme mi
propio juicio acerca de la verdad comunicada, pero aquí no podría decir que
estoy creyendo al que comunica o que estoy contando con él para mi creer que
$p$.

¿Entonces creer a alguien es creer algo apoyado en el hecho de que lo ha dicho?
\citalitinterlin{Puede que se le pregunte a un testigo `¿Por qué pensó que aquel
  hombre se estaba muriendo?' y que éste responda `Porque el doctor me lo dijo'
  [\ldots] `no me hice ninguna opinión propia --- yo sólo creí al
  doctor'}\footnote{\cite[4]{anscombe2008faith:tobelieve}: <<A witness might be
  asked `Why did you think the man was dying?' and reply `Because the doctor
  told me'. If asked further what his own judgement was, he may reply `I had no
  opinion of my own --- I just believed the doctor'.>>}. Éste puede ser un
ejemplo de contar con $x$ para la verdad de $p$. Esto, sin embargo, tampoco
parece ser suficiente. Puedo imaginar el caso en el que esté convencido de que
alguien a la vez cree lo opuesto a la verdad de $p$ y quiera mentirme. Según
este cálculo podría decir que creo en lo que ha dicho por el hecho de que me lo
ha dicho, pero no estaría diciendo que le creo a él.

¿Qué se puede decir del <<les creo a todos>> de Eutidemo en la cuestión
preliminar? Anscombe juzga que la exclamación no expresa simplemente una opinión
apresurada o excesiva credulidad, sino más bien suena a
locura\autocite[5]{anscombe2008faith:tobelieve}. Eutidemo no puede estar
diciendo la verdad cuando dice que les cree a todos. La expresión de $C$ da
pertinencia a lo que dice $B$, y la manera natural de entender lo que dice $B$
es como arrojando duda sobre lo que $A$ ha dicho. ¿Se puede pensar que $A$
todavía cree lo que ha dicho inicialmente? ¿Eutidemo puede creer a $A$ sin saber
cuál es su reacción a lo que $B$ y $C$ han dicho? Entonces Anscombe concluye,
\citalitinterlin{Para creer a $N$ uno debe creer que $N$ mismo cree lo que está
  diciendo}\footnote{\cite[5]{anscombe2008faith:tobelieve}: <<To believe $N$ one
  must believe that $N$ himself believes what he is saying>>.} Creer a $N$ sin
saber si $N$ cree lo que dice le suena a Elizabeth como una locura.

En este punto queda expuesta a la luz una segunda creencia involucrada en el
creer a $x$ que $p$. Anscombe fija su atención en esto. Creer a $x$ que $p$
conlleva otras creencias, éstas son presuposiciones implicadas en llegar a
plantearse si creer o no. En primer lugar, si se cree a alguien, tiene que ser
el caso que se cree que una comunicación es de
alguien\autocite[Cf.~][6]{anscombe2008faith:tobelieve}. Esta presuposición no
parece tan problemática si se piensa en las ocasiones en las que creemos a
alguien que es percibido. Sin embargo tiene más profundidad si se considera que
con frecuencia recibimos la comunicación sin que esté presente el que habla,
como cuando leemos un libro\autocite[Cf.~][5]{anscombe2008faith:tobelieve}.

Se puede imaginar aquí una situación problemática. Supongamos que alguien recibe
una carta en la que el autor no es el comunicador ostensible o aparente, es
decir, quien firma la carta no es quien la ha escrito. ¿Se puede decir que el
que recibe la carta cree o descree al autor o al comunicador ostensible? Creer
al autor, afirma Anscombe, conlleva un tipo de juicio y especulación que no son
mediaciones ordinarias en el creer a
alguien\autocite[Cf.~][7]{anscombe2008faith:tobelieve}. Para decir que creo al
autor tendría que discernir que la comunicación que viene bajo otro nombre es
realmente de esta otra persona que además me quiere decir esto.

Respecto de la posibilidad de decir que se cree al comunicador ostensible
Anscombe distingue entre un comunicador ostensible que exista o no. Ante una
comunicación que viene de parte de un comunicador aparente que no existe,
alguien puede responder diciendo que cree o descree al comunicador aparente,
pero la decisión de decir esto ---dice Anscombe--- \citalitinterlin{es una
  decisión de dar a estos verbos un uso `intencional', como el verbo `ir
  tras'}\footnote{\cite[7]{anscombe2008faith:tobelieve}: <<is a decision to give
  those verbs an `intentional' use like the verb `to look for'>> Ver:
  \cite{anscombe1981metaphysics:intsens}. Anscombe propone que un verbo es usado
  intencionalmente cuando tiene como objeto directo un `objeto intencional'
  (`objeto' no en el sentido material, sino de finalidad).}. Esto lo ilustra
añadiendo: \citalitlar{Y así uno podría hablar de alguien como creyendo al dios
  (Apolo, digamos), cuando consultó el oráculo del dios -- sin que por esto uno
  estuviera implicando que uno mismo cree en la existencia del dios. Todo lo que
  queremos es que necesitamos saber lo que es llamado que el dios le diga
  algo\footnote{\cite[7]{anscombe2008faith:tobelieve}: <<And so we might speak
    of someone as believing the god (Apollo, say), when he consulted the oracle
    of the god -- without thereby implying that one believed in the existence of
    the gos oneself. All we want is that we should know what is called the god's
    telling him something>>}.} `Creer' usado aquí intencionalmente viene a decir
que se busca o se desea creer a $x$ (Apolo en este caso) cuando se recibe
aquello que alguien entiende como una comunicación suya.

En el caso de que el comunicador ostensivo sí exista, la noción de creerle
manifiesta una cierta oscilación. Una tercera persona podría decir que `aquel,
pensando que $N$ dijo esto, le creyó', o el comunicador aparente puede decir
`veo que pensaste que fui yo quien dijo esto y me creiste', sin embargo, si el
que ha recibido la comunicación dijera `naturalmente te creí', el comunicador
aparente podría contestar `ya que no lo he dicho yo, no me estabas creyendo a
mi'\autocite[Cf.~][8]{anscombe2008faith:tobelieve}.

Estas consideraciones llevan a Anscombe a distinguir entre el que habla en una
comunicación y el productor inmediato de la
comunicación\autocite[Cf.~][8]{anscombe2008faith:tobelieve}. Éste puede ser
cualquiera que pase hacia adelante alguna comunicación, un maestro o mensajero,
o un interprete o traductor; éste es \citalitinterlin{el productor inmediato de
  aquello que se entiende, o incluye una reclamación interna de ser entendido
  como una comunicación de $NN$}\footnote{\cite[8]{anscombe2008faith:tobelieve}:
  <<we can speak of the immediate producer of what is taken, or makes an
  internal claim to be taken, as a communication from $NN$>>}. Si digo que creo
a un intérprete estoy afirmando que creo lo que ha dicho su principal, y mi
contar con el intérprete consiste en la creencia de que ha reproducido lo que
aquel ha dicho. En este sentido el intérprete no le falta rectitud si dice algo
que no es verdadero pero no ha representado falsamente lo que ha dicho su
principal. Por el contrario, al maestro sí le faltaría rectitud si lo que dice
no es verdadero. Cuando se cree al maestro, aún en el caso que no sea de ninguna
manera autoridad original de lo que comunica, se le cree a él sobre lo que
transmite. Para Anscombe no es necesario que cuando se cree a alguien se le
trate como una autoridad
original\autocite[Cf.~][5]{anscombe2008faith:tobelieve}. En esto el ejemplo del
maestro como distinto del intérprete es ilustrativo. Un maestro puede conocer lo
que enseña porque lo ha recibido de alguna tradición de información y al
transmitir lo que enseña se le está creyendo a él.

Asoma aquí otro aspecto relacionado con esta presuposición. Al creer que una
comunicación es de alguien se cree a una persona que puede tener distintos
grados de autoridad sobre lo que dice. El maestro del que se ha hablado antes
podría afirmar <<Leonardo da Vinci dibujó diseños para una máquina voladora>> y
en esto no es para nada una autoridad
original\autocite[Cf.~][6]{anscombe2008faith:tobelieve}. Conoce esto porque lo
ha escuchado, incluso si ha visto los diseños. Aún cuando los hubiera
descubierto él mismo, tendría que haber contado con alguna información recibida
de que esos diseños que ve son de Leonardo. En este caso sí seria una autoridad
original en notar que estos diseños que ha escuchado que son de Leonardo son de
máquinas voladoras. Anscombe explica la distinción diciendo:
\citalitlar{[Alguien] es \emph{una} autoridad original en aquello que él mismo
  ha hecho y visto y oido: digo \emph{una} autoridad original porque sólo quiero
  decir que él mismo sí contribuye algo, es algún tipo de testigo por ejemplo,
  en lugar de alguien que sólo transmite información recibida. Pero su informe
  de aquello de lo que es testigo es con frecuencia [\ldots] fuertemente
  influenciado o más bien casi del todo formado por la información que \emph{él}
  ha recibido\footnote{\cite[5]{anscombe2008faith:tobelieve}: <<He is \emph{an}
    original authority on what he himself has done and seen and heard: I say
    \emph{an} original authority because I only mean that he does himself
    contribute something, e.g. is in some sort a witness, as oposed to one who
    only transmits information received. But his account of what he is a witness
    to is very often [\ldots] heavily affected or ratherl all but completely
    formed by what information \emph{he} had received.>>}.} Además de ser
\emph{una} autoridad original sobre algún hecho, una persona puede ser una
autoridad \emph{totalmente} original. Si la distinción entre alguien que no es
una autoridad original y alguien que sí lo es ha sido descrita como la
contribución de algo propio que junto con la información recibida permite
construir un informe, lo particular de una autoridad totalmente original es que
no se apoya en ninguna información recibida para construir su informe de los
hechos. Anscombe no entiende el lenguaje como información recibida. Pone como
ejemplo de informe de una autoridad totalmente original a alguien que dice `esta
mañana comí una manzana' y dice: \citalitlar{si él está en la situación usual
  entre nosotros, sabe lo que una manzana es --- es decir, puede reconocer
  una. Así que aún cuando se le ha `enseñado el concepto' al aprender a usar el
  lenguaje en la vida ordinaria, no cuento esto como un caso de depender en
  información recibida.\footnote{\cite[6]{anscombe2008faith:tobelieve}: <<if he
    is in the situation usual among us, he knows what an apple is --- i.e. can
    recognise one. So though he was `taught the concept' in learning to use
    language in everyday life, I do not count that as a case of reliance on
    information received.>>}}

Hasta aquí se ha visto que el creer a $x$ que $p$ implica otras creencias que
son presuposiciones a la pregunta sobre si se cree o se descree a alguien y se
ha descrito lo que tiene que ver con la creencia de que una comunicación viene
de alguien. Anscombe examina otras presuposiciones más. También tiene que ser el
caso que creamos que por la comunicación, la persona que habla quiere decir
\emph{esto}. En situaciones ordinarias no es difícil distinguir si alguien está
diciendo o escribiendo algún lenguaje. Sin embargo, aún cuando el que habla use
palabras que puedo `hacer mías' y creer simplemente las palabras que dice, aquí
queda espacio para decir que hay una creencia adicional de que se ha dicho `tal
cosa' en la comunicación. Elaboramos en aquello que hemos creido y usamos otras
palabras distintas, nuestras creencias no están atadas a palabras específicas.
También podríamos pensar que alguien diga que cree \emph{esto} porque cree a $x$
y que se le cuestione su creencia preguntando `¿qué tomaste como $x$ dicicéndote
eso?'\autocite[Cf.~][8]{anscombe2008faith:tobelieve}.

Otra presuposición más sería que se cree que la comunicación está
\emph{dirigida} a alguien, aunque sea `a quien lea esto' o `a quien pueda
interesar'. Esta creencia se podría problematizar pensando en algún caso que
alguien reciba una comunicación con otro destinatario, ¿estaría creyendo al que
se comunica?. Asncombe opina que en un sentido extendido o reducido y considera
que el tema parece de poca
importancia\autocite[Cf.~][7]{anscombe2008faith:tobelieve}.

Una persona a quien se dirige una comunicación puede \emph{fallar en creerla} si
no nota la comunicación, o si notándola no la interpreta como lenguaje, o si
notándola como lenguaje no la toma como dirigida hacia ella; o puede que crea
todo esto, pero lo interprete incorrectamente, o puede que lo interprete bien
pero no crea que viene realmente de $N$. En este tipo de casos la persona no ha
descreido, sino que no ha llegado a estar en la situación de plantearse esa
pregunta. Para poder llegar a preguntar si alguien cree a $x$ que $p$ habría que
excluir o asumir como excluidos todos los casos en los que estas otras
presuposiciones no se han cumplido. Es así que Anscombe concluye:
\citalitlar{Supongamos que todas la presuposiciones están dadas. $A$ está
  entonces en la situación ---una muy común--- donde surge la pregunta sobre si
  creer o dudar (suspender el juicio ante) $NN$. Sin confusión por todas las
  preguntas que surgen por las presuposiciones, podemos ver que creer a alguien
  (en el caso particular) es confiar en él para la verdad -- en el caso
  particular. \footnote{\cite[9]{anscombe2008faith:tobelieve}: <<Let us suppose
    that all the presuppositions are in. $A$ is then in the situation ---a very
    normal one--- where the question arises of believing or doubting (suspending
    judgement in face of) $NN$. Unconfused by all the questions that arise
    because of the presuppositions, we can see that believing someone (in the
    particular case) is trusting him for the truth -- in the particular
    case.>>}.}
Que $A$ crea a $N$ que $p$ implica que $A$ cree que en una comunicación, que puede
venir de un productor inmediato, $N$ es el que habla y lo que dice es $p$ y esta
comunicación está dirigida hacia $A$; entonces $A$, creyendo que $N$ cree que
$p$, confia en $N$ sobre la verdad de $p$.

\subsection{Valoraciones Preliminares}
Hasta aquí hemos recorrido con Elizabeth una descripción de un tipo de juicio
cuyo fundamento se encuentra más allá de la relación lógica de sus premisas.
Formar un juicio razonable a partir de la creencia depositada en el informe de
alguien acerca de algún hecho no sólo tiene como fundamento una valoración de la
lógica de su argumentación, sino que implica, dadas las presuposiciones, confiar
en el que habla, creyendo además que cree lo que dice.

El análisis de Anscombe también ofrece la posibilidad de hacer una descripción
general de lo que significa `creer a un testigo que $p$'. Anscombe ha hecho la
distinción entre alguien que simplemente transmite información y alguien que
puede ser considerado algún tipo de testigo. Un testigo es un ejemplo de
autoridad original y alguien es una autoridad original acerca de lo que él mismo
ha hecho y visto y oído. Un testigo que es una autoridad original aporta algo de
lo que él mismo ha hecho y visto y oído y lo considera junto a información que
ha recibido para comunicar su informe de algún hecho. Cuando el testigo no
cuenta con información recibida, sino que habla sólo de lo que aporta él mismo,
es una autoridad totalmente original. Aunque esta descripción del testigo es muy
amplia, permite afirmar que cuando $A$ cree a un testigo que $p$, $A$ cree que
en una comunicación, que puede venir de un productor inmediato, es esta
autoridad original el que habla y que dice $p$ y tiene a $A$ como destinatario;
entonces $A$, creyendo que esta autoridad original cree lo que dice, confia en
el testigo sobre la verdad de $p$.

La investigación sobre el creer abre además varias rutas de análisis en torno al
tema del testimonio. Dos de estas conexiones aparecen en la investigación de
Anscombe a modo de preámbulo y la tercera queda planteada como una cuestión
abierta al final.

\subsubsection{Acceso al mundo más allá de la experiencia}
¿Cómo accedemos a una idea del mundo más allá de nuestra experiencia personal?
Una de las cuestiones que Anscombe plantea como preámbulo a su análisis sobre el
`creer a alguien' tiene que ver con esta pregunta.

Hume diría que el puente que permite nuestro contacto con la realidad más allá
de nuestra experiencia es la relación
causa-y-efecto\autocite[Cf.~][3]{anscombe2008faith:tobelieve}. Inferimos las
causas desde sus efectos porque estamos acostumbrados a ver que causa y efecto
van juntas. Estas causas inferidas las verificamos en la percepción inmediata de
nuestra memoria o nuestros sentidos, o por medio de la inferencia de otras
causas verificadas del mismo
modo\autocite[Cf.~][88]{anscombe1981parmenides:humeandjulius}. Hume entonces
propone que la relación entre el testimonio y la verdad es de la misma clase,
inferimos la verdad del testimonio porque estamos acostumbrados a que vayan
juntas\autocite[Cf.~][3]{anscombe2008faith:tobelieve}.

Anscombe tacha de absurda esta visión del rol del testimonio en el conocimiento
humano\autocite[Cf.~][3]{anscombe2008faith:tobelieve} y le parece que
\citalitinterlin{el misterio es cómo Hume la pudo haber llegado a
  sostener}\footnote{\cite[Cf.~][3]{anscombe2008faith:tobelieve}: <<the mystery
  is how Hume could ever have entertained it.>>}. Entonces explica:
\citalitlar{Hemos de reconocer al testimonio como el que nos da nuestro mundo
  más grande en no menor grado, o incluso en un grado mayor, que la relación de
  causa y efecto; y creerlo es bastante distinto en estructura que el creer en
  causas y efectos. Tampoco es lo que el testimonio nos da una parte
  completamente desprendible, como el borde de grasa en un pedazo de filete. Es
  más bien como las manchas y rayas de grasa que están distribuidas a través
  de la buena carne; aunque hay nudos de pura grasa
  también\footnote{\cite[3]{anscombe2008faith:tobelieve}:<<We must acknowledge
    testimony as giving us our larger world in no smaller degree, or even in a
    greater degree, than the relation of cause and effect; and believing it is
    quite dissimilar in structure from belief in causes and effects. Nor is what
    testimony gives us entirely a detachable part, like the thick fringe of fat
    on a chunk of steak. It is more like the flecks and streaks of fat that are
    distributed through good meat; though there are lumps of pure fat as
    well>>}.} Elizabeth considera que la mayor parte de nuestro conocimiento de
la realidad está apoyado en la creencia que tenemos en las cosas que se nos han
enseñado o dicho\autocite[Cf.~][3]{anscombe2008faith:tobelieve}. Para ella, la
investigación acerca de `creer a alguien' no sólo es del interés de la teología
o de la filosofía de la religión, sino de enorme importancia para la teoría del
conocimiento\autocite[Cf.~][3]{anscombe2008faith:tobelieve}.

La ruta de análisis que Anscombe abre con esta propuesta consiste en una
descripción más adecuada de la `estructura del creer en el testimonio' como una
estructura distinta de la relación causa y efecto. Aquí la descripción vista
sobre el `creer a algiuen' ha ofrecido ya pistas valiosas. Sin embargo, Anscombe
aborda el tema en otras discusiones y es necesario tenerlas en cuenta para hacer
una descripción más completa.

\subsubsection{`Creer a alguien' como `Fe'}
Una segunda cuestión que aparece como preámbulo en la investigación de Anscombe
es planteada así: \citalitinterlin{Si las palabras siempre guardaran sus
  antiguos valores, podría haber llamado mi tema `Fe'. Este corto término ha
  sido usado en el pasado justo con este significado, el de creer a
  alguien}\footnote{\cite{anscombe2008faith:tobelieve}: <<If words always kept
  their old values, I might have called my subject `Faith'. That short term has
  in the past been used in just this meaning, of believing someone.>>}. Este uso
de la expresión sera útil para Anscombe en su análisis del uso de la palabra
`fe'. Su descripción estará enfocada en `fe' como `creer a Dios que $p$'. Esta
segunda ruta será explorada más adelante.

\subsubsection{Creer a quien habla rectamente}
Al final de la investigación, Anscombe propone una cuestión que se queda
abierta. Tiene que ver con uno de los ejemplos relacionados a creer que la
comunicación viene de alguien. Allí proponia imaginar el caso en el que
estuvieramos convencidos de que alguien viene a decirnos lo que cree que es
falso, pero a la misma vez sabemos que lo que cree es lo contrario a la verdad.
Al decir lo que cree que es falso estaría afirmando la verdad. En ese caso,
afirmaba Anscombe, podría decir que creo en lo que dice y además creo porque lo
dice, pero no le creo a él. Se podría preguntar ¿cuál es la diferencia entre
llegar a la creencia de $p$ porque alguien que está en lo correcto y es veraz me
lo ha dicho, y llegar a la misma creencia porque me lo ha dicho alguien que está
equivocado y miente? Ambos casos parecen implicar un cálculo, en uno se calcula
que está en lo correcto y es veraz y en el otro se calcula que está equivocado y
miente. ¿Por qué estamos dispuestos a decir que creemos al que habla sólo en el
caso en que esté en lo correcto y sea veraz? ¿Acaso no llevan ambos casos a la
misma creencia que $p$?

Aquí Anscombe percibe que hay más que decir sobre la prioridad que damos a la
rectitud y la veracidad en el creer lo que se nos dice sobre la realidad. De las
tres rutas descritas, recorreremos ésta primero.

\section{Sobre la Primacía y Unidad de la Verdad}
%% SECCIÓN 1: La verdad
\section{Verdad y Significado}

\subsection{¿Qué es tener la verdad?}
Elizabeth Anscombe visitó muchas veces la Universidad de Navarra junto con Peter
Geach. Allí impartió algunos seminarios y participó de las Reuniones
Filosóficas.\footcite[cf.~][p.~15]{fa&esphom} En una de sus visitas, en octubre
de 1983, ofreció dos lecciones tituladas: ``Verdad'' y ``La unidad de la
verdad''. Las dos investigaciones estan apoyadas en algunas reflexiones de San
Anselmo cuyos argumentos sirven a Anscombe para explorar modos de hablar de
aquello de lo que decimos que tiene verdad. Anscombe dio inicio a su ponencia
planteando la cuestión como sigue: \citalitlar{Hay verdad en muchas cosas.
  Mirando a mi título `Truth' me quedo algo sobrecogida por él, pues lo que
  salta de la página hacia mi es uno de los nombres de Dios. <<He amado la
  verdad>> me dijo una vez un profesor moribundo, después de hablarme de la
  dificultad que sentía sobre la idea de amar a Dios. Sin embargo: <<He amado la
  verdad>>. Y luego, temiendo que yo no malentendiera su afirmación: <<No me
  refiero, cuando digo eso, que \emph{tenga} la verdad>>} \citalitlar{Tener la
  verdad, estar en la verdad---¿qué es esto? Y qué quiso decir Nuestro Señor al
  llamarse a \emph{sí mismo} la verdad? <<No hay tal cosa como la verdad, sólo
  hay verdades>>, decía mi suegro a la primera esposa de Bertrand Russell.
  Russell fue su maestro; la influencia se ve con facilidad.} \citalitlar{¿Pero
  cuáles son las cosas que tienen verdad en ellas? ¿Tiene la creación? ¿tienen
  las acciones? A qué se refería Aristóteles cuando dijo que el bien de la razón
  práctica era `verdad de acuerdo con el recto deseo'? ¿Las cosas hechas por los
  hombres tienen verdad en ellas? ¿Qué, de nuevo, quiso decir Aristóteles cuando
  afirmó que el arte o la habilidad es una disposición productiva con un logos
  verdadero? Mas allá todavía: Qué fuerza tiene contar la verdad entre los
  `trascendentales', esas cosas que `atraviesan' todas las categorías y todas
  las formas especiales de las cosas; y que no pertenecen cada uno a una
  categoría, como el color: amarillo; o el area: un acre; o el animal: un
  caballo.\footcite[~71]{anscombe2011plato:truth}}

\subsection{La primacia de la verdad sobre la falsedad}
Estos cuestionamientos llevan a Anscombe a indagar en una materia en la que
Wittgenstein y San Anselmo --dice-- son `hermanos intelectuales': ¿cuál es la
primacía de la verdad sobre la falsedad?.

San Anselmo queda prendado de esta pregunta como consecuencia de su indagación
en el capítulo segundo del \emph{De Veritate}: ¿qué es la verdad de una
proposición o declaración?. Ha elegido indagar en las proposiciones o las
declaraciones como aquellas clases de las cuales más naturalmente se puede
pensar que contienen los posibles portadores del predicado `verdadero'. Así lo
expresa cuando dice \citalitinterlin{Busquemos primero qué es la verdad en una
  proposición, dado que con frecuencia llamamos a éstas verdaderas o
  falsas.}\autocite{De Veritate c. 2}

Wittgesntein recorre la ruta analoga en los apartados que conforman el \S4.06
del Tractatus. Argumenta que \citalitinterlin{Una proposición puede ser
  verdadera o falsa sólo en virtud de ser una imagen de la
  realidad}\autocite[\S4.06]{wittgenstein1922tractatus}. Y advierte que
\citalitinterlin{No debe ser pasado por alto que una proposición tiene un
  sentido que es independiente de los hechos: de otra manera uno podría
  fácilmente suponer que verdadero y falso son relaciones de igualdad entre los
  signos y aquello que significan}\autocite[\S4.061]{wittgenstein1922tractatus}.

Elizabeth realiza su invetigación adentrándose en la misma cuestión trabajada
por ambos autores. El primer movimiento que hace en su análisis es indagar en la
distinción entre significado y verdad. Según se ha visto, la distinción es
familiar en las elucidaciones del Tractatus: \citalitinterlin{La proposición
  tiene un sentido que es independiente de los hechos}
\autocite[\S~4.061]{wittgenstein1922tractatus} San Anselmo también lo considera.
Una proposición no pierde su significado cuando no es verdadera. Si el
significado (\emph{significatio}) de una proposición fuera su verdad, ésta
\citalitinterlin{semper esset vera}, siempre sería verdadera. Sin embargo el
significado de una proposición \citalitinterlin{manent \ldots et cum est quod
  enunciat, et cum non est}, permanece lo mismo cuando lo que se afirma es el
caso que es y cuando no lo es.

Significado y verdad en una proposición son distintos. Entonces, ¿qué es la
verdad de una proposición?. Se podría querer responder que es la
\citalitinterlin{res enunciata}, es decir, la realidad correspondiente, lo que
la proposición verdadera dice. Esta respuesta nos llevaría a confusión. ``La
verdad de una proposición es este hecho que significa''. Si esto es así,
entonces cuando deja de ser verdadera también pierde su significado, pues el
hecho que era su signifcado ya no es. Además, si la desaparición del hecho es la
desaparición del significado y la verdad, ¿no será entonces que el hecho es la
misma cosa que el significado y la verdad? Sin embargo no es así, el hecho es lo
que la hace verdadera: lo que la proposición verdadera dice, la \emph{res
  enunciata} es la causa de la verdad de una proposición y no su verdad:
\citalitinterlin{non eius veritas, sed causa veritatis eius dicenda est}

La distinción abre otra línea de consideraciones. El hecho o la \emph{res
  enunciata} por la proposición verdadera es la causa de la verdad del
enunciado. La proposición tiene significado independientemente de si es
verdadera o falsa. En este sentido, una proposición con significado puede
guardar relación de verdad o de falsedad con los hechos. Una proposición falsa
no carece de toda relación con el hecho, sino que contiene una descripción del
hecho que hace a la proposición contraria verdadera. Podríamos pensar, entonces,
que la proposición verdadera y la proposición falsa pueden intercambiar roles.

Wittgenstein sugiere esto cuando afirma que el hecho de que `\emph{p}' y
`$\sim$\emph{p}' pueden intercambiar roles es importante pues muestra que `no'
no representa nada en la realidad. Más aún `\emph{p}' y `$\sim$\emph{p}' son
opuestos en significado pero a ambos enunciados corresponde una sola realidad;
esto es el hecho, la \emph{res enunciata} por el enunciado verdadero. Esto
permitiría sostener que verdadero y falso son tipos de relaciones entre el signo
y la cosa significada que están igualmente justificadas. `\emph{p}' y
`$\sim$\emph{p}' significan la misma realidad, cualquiera de las dos
posibilidades que resulte ser la realidad correspondería con ambas. La única
distinción que Wittgenstein se reserva entre ambas proposiciones es que una
significa falsamente lo que la otra significa verdaderamente. Sin embargo esta
distinción puede quedar disuelta con facilidad si se considera que `significa
verdaderamente' o `significa falsamente' no son descripciones de los sentidos de
las proposiciones verdaderas o falsas. Se puede entender el sentido de ``estoy
sentado'' o ``no estoy sentado'' sin conocer cuál enunciado se corresponde con
la realidad o cuál de ambas expresiones está significando verdaderamente y cuál
falsamente. En cuanto a la relación entre signo y significado ambas
proposiciones no tienen diferencia.

En San Anselmo esta noción de relaciones igualmente justificadas aparece bajo la
forma de una pregunta planteada por el discípulo en el diálogo con su maestro.
Dice: \citalitlar{Dime qué he de responder si alguien dice que incluso cuando
  una expresión significa que es algo que no es, está significando lo que debe.
  Puesto que se le ha dado igualmente el significar como que es tanto lo que es
  como lo que no es. Pues si no se le hubiera dado el significar como siendo
  incluso lo que no es, no lo significaría. Así que incluso cuando significa que
  es lo que no es, está significando lo que debe. Pero si es correcto y
  verdadero en significar lo que debe, como has mostrado, entonces la expresión
  es verdadera incluso cuando dice que es algo que no es.\autocite{deveritate}}
Las dos relaciones son expresadas como una paridad: \citalitinterlin{pariter
  accepit significare esse, et quod est, et quod non est}. Esta paridad es
esencial ya que si la proposición no significara lo que significa igualmente
cuando lo que significa es y también cuando tal cosa no es, no sería capaz de
significar del todo.

A propósito de esta paridad, Wittgenstein plantea: \citalitinterlin{¿Acaso no
  podríamos hacernos entender usando proposiciones falsas tal como hemos hecho
  hasta ahora por medio de las verdaderas---siempre y cuando sepamos que están
  significadas falsamente?\footcite[\S4.062]{wittgenstein1922tractatus}}
Anscombe compara este posible modo de actuar a una táctica de Santa Juana de
Arco. La Santa empleaba un código en las comunicaciones con sus generales
subordinados que consistía en que las cartas que ella marcaba con una cruz
contenían proposiciones que debían ser interpretadas en el sentido contrario. El
código es posible.

Hasta aquí Anscombe ha insitido en los argumentos de San Anselmo y de
Wittgenstein que apoyan la idea de que las proposiciones falsas y verdaderas
tienen igualdad de relación con la realidad significada. Wittgenstein ha
advertido del supuesto de entender ambas relaciones como igualmente
justificadas, sin embargo lo que ha propuesto hasta ahora parece apoyar esta
idea. La paridad propuesta ha resultado esencial para el significado, el sentido
o \emph{significatio} del tipo de proposiciones que pueden ser verdaderas o
falsas. La pregunta ahora es ¿qué, entonces, \emph{es} desigual entre ellas?
¿Cuál es la primacia de la verdad?

La respuesta de Wittgenstein a esta pregunta llegará a ser: no se puede
describir a alguien como comunicándose con proposiciones falsas entendidas como
significadas falsamente ya que se tornan en proposiciones verdaderas al ser
afirmadas. Esta es su respuesta a la pregunta ¿podemos darnos a entender con
proposiciones falsas?: \citalitinterlin{¡No! Pues una proposición es verdadera
  si las cosas son así como estamos usándola para decir que son, y entonces si
  usamos `\emph{p}' para decir que $\sim$\emph{p} y las cosas son como queremos
  decir que son, entonces `\emph{p}' es vedadero en nuestro nuevo modo de
  tomarlo y no falso.\autocite[\S4.062]{wittgenstein1922tractatus}} En la
táctica antes descrita, Santa Juana de Arco no mentía con su código y, si no
estaba en error acerca de los hechos, sus oraciones eran verdaderas y no falsas.

Para Anscombe, esta descripción de la primacía de la verdad no parece explicar
cómo rechazar que verdadero y falso tengan relaciones igualmente justificadas
¿Acaso este tipo de imposibilidad general contiene toda la sustancia de las
`relaciones no igualmente justificadas'? Se puede aceptar que verdadero y falso
no son relaciones igualmente justificadas porque lo falso no podría hacerse
cargo del rol de lo verdadero en las afirmaciones y en el pensamiento. Sin
embargo, podemos mentir\ldots\, o equivocarnos. La imposibilidad general de
intercambiar los roles de verdadero y falso propuesta por Wittgenstein no
considera ni el error ni la mentira. Esta imposibilidad general puede ofrecer una
cierta primacia de la verdad dentro de la teoría del significado, pero ¿se
podría apoyar en esto el decir que la proposición verdadera tiene una relación
mas \emph{justificada} con la realidad que la falsa?

En San Anselmo, por su parte, se puede encontrar una propuesta sobre la primacía
de la verdad dentro de su definición de lo que la verdad es. Su punto de partida
ha sido la pregunta: ¿\emph{Para qué} es un enunciado? El diálogo se desarrolla
de este modo: \citalitlar{\emph{Maestro.} ¿Qué te parece que es la verdad en
  el enunciado mismo?\\
  \emph{Discípulo.} No sé más que esto: cuando significa
  ser lo que es, entonces es verdadero y hay verdad en él.\\
  \emph{M.} ¿Para qué se hace una afirmación?\\
  \emph{D.} Para significar que lo que es, es.\\
  \emph{M.} Luego, debe significarlo.\\
  \emph{D.} Es cierto.\\
  \emph{M.} Cuando significa que lo que es, es, significa lo que debe.\\
  \emph{D.} Es manifiesto.\\
  \emph{M.} Y cuando significa lo que debe, significa rectamente.\\
  \emph{D.} Así es.\\
  \emph{M.} Cuando significa rectamente, la significación es recta.\\
  \emph{D.} No hay duda.\\
  \emph{M.} Luego, cuando significa que lo que es, es, la significación es recta.\\
  \emph{D.} Eso se sigue.\\
  \emph{M.} También cuando significa que lo que es,
  es, la significación es verdadera.\\
  \emph{D.} Verdaderamente, cuando significa que
  lo que es, es, es recta y verdadera.\\
  \emph{M.} Para ella es lo mismo ser recta y ser verdadera, es decir significar
  que lo que es, es.\\
  \emph{D.} Es lo mismo, en verdad.\\
  \emph{M.} Por lo tanto, para ella, la verdad no es otra cosa que la rectitud.\\
  \emph{D.} Ahora veo claramente que la verdad es esa rectitud.\\
  \emph{M.} E igual sucede cuando el enunciado significa que lo que no es, no
  es.}

El discípulo ha visto que la verdad del enunciado no es la \emph{res enunciata}
por una proposición verdadera, tampoco está en la significación, o en cualquier
cosa perteneciente a la definición, sino que \citalitinterlin{Nihil aliud scio
  nisi quia cum significat esse qous est, tunc est in ea veritas et est vera}.
Cuando una afirmación hace aquello para lo que es, la significación
(\emph{significatio}) está hecha rectamente. Esta rectitud es lo que la verdad
es. Es aquí que el discípulo presenta la objeción antes expuesta: `Cuando una
expresión significa que es algo que no es, ¿se puede decir que está significando
lo que debe?'. La respuesta del maestro será: \citalitinterlin{Veritatem et
  rectitudinem habet, quia facit quod debet}. Una expresión falsa hace lo que
debe en significar aquello que le ha sido dado significar, hace aquello para lo
que la expresión es. Sin embargo, teniendo este modo de ser verdadera, no
solemos llamarla verdadera pues habitualmente decimos que la expresión es
verdadera y correcta sólo cuando significa que es aquello que es y no cuando
significa que es aquello que no es, pues tiene mayor deber de hacer aquello para
lo que se le ha dado significar que para lo que no se le ha dado. Es
sorprendente que el maestro no rechace la descripción del discípulo, más aún que
la reitere. La objeción presentada no supone un impedimento para sostener esta
descripción de la verdad. El maestro retiene su explicación apoyada en que la
verdad de un enunciado es que hace lo que debe.

¿En qué consiste entonces la primacía de la verdad? La proposición verdadera
hace lo que debe de dos maneras: significa justo aquello que se le ha dado
significar --independientemente de si es el caso que es o no-- y significa
aquello para lo que se le ha dado esa significación, esto es, afirmar como que
es el caso lo que \emph{es} el caso.

Una observación adicional de Anselmo puede ser relacionada con la pregunta de
Wittgenstein: `¿Podríamos darnos a entender por medio de proposiciones falsas?'.
A la proposición no se le da el significar como siendo aquello que no es [o no
siendo aquello que sí es], excepto porque no se le podía dar significar que algo
es solamente cuando eso que significa da el caso que es o su no ser sólo cuando
es el caso que no es. La observación se acerca a la respuesta de Wittgenstein.
En este sentido, lo falso sólo es posible porque lo verdadero (en este tipo de
proposiciones) no puede ser la única posibilidad.

La descripción de la verdad que Anselmo comienza aquí le llevará por medio de
consideraciones sobre la verdad en el pensamiento, la voluntad, la acción y el
ser de las cosas a su conocida definición de la verdad como \emph{rectitudo sola
  mente perceptibilis}.

\section{Fe como creer a Dios}
%\subsection{Faith (1975)}

En \emph{Oscott College}, el seminario de la Archidiócesis de Birmingham, se comenzaron a celebrar las conferencias llamadas \emph{Wiseman Lectures} en 1971. Para estas lecciones ofrecidas anualmente en memoria de Nicholas Wiseman se invitaba un ponente que tratara algún tema relacionado con la filosofía de la religión o alguna materia en torno al ecumenismo\footnote{\cite[Cf.][7]{wisemanlects}.}.

El 27 de octubre de 1975, para la quinta edición de las conferencias, Anscombe presentó una lección titulada simplemente \emph{Faith}. Allí planteaba la siguiente cuestión: \blockquote[{\Cite[115]{anscombe1981erp:faith}}: \enquote{I want to say what might be understood about faith by someone who did not have it; someone, even, who does not necessarily believe that God exists, but who is able to think carefully and truthfully about it. Bertrand Russell called faith `certainty without proof'. That seems correct. Ambrose Bierce has a definition in his Devil's Dictionary: `The attitude of mind of one who believes without evidence one who tells without knowledge things without parallel.' What should we think of this?}]{Quiero decir qué es lo que puede ser entendido sobre la fe por alguien que no la tenga; alguien, incluso, que no necesariamente crea que Dios existe, pero que sea capaz de pensar cuidadosa y honestamente sobre ella. Bertrand Russell llamó a la fe `certeza sin prueba'. Esto parece correcto. Ambrose Bierce tiene una definición en su \emph{Devil's Dictionary}: `La actitud de la mente de uno que cree sin evidencia a uno que habla sin conocimiento cosas sin parangón'. ¿Qué deberíamos pensar de esto?}

El objetivo de Elizabeth, hablar de la fe para quien no tiene esa experiencia, determina un enfoque específico a su investigación. La descripción del fenómeno de la fe tiene que ser realizada razonablemente, de modo que pueda ser considerada por alguien \enquote*{que sea capaz de pensar cuidadosa y honestamente} sobre ella. Su estrategia consiste aquí de nuevo en una descripción de usos familiares de la palabra analizada que son articulados de tal manera que los patrones de estos usos sean estudiables\footnote{\cite[Cf.][12]{bakerhacker2009understanding}: \enquote{There is no room in philosophy for explanatory (hypothetico-deductive) theory, on the model of science, or for dogmatic (essentialist) thesis, on the model of metaphysics. Its task is grammatical clarifiaction that dissolves conceptual puzzlement and gives an overview or surveyable representation of a segment of the grammar of our language \textelp{} It describes the familiar uses of words and arranges them so that the patterns of their use become surveyable, and our entanglement in the web of grammar becomes perspicuous}.}. Se enfoca en un modo antiguo de usar la palabra `fe' en el que se le empleaba para decir \enquote*{creer a alguien que $p$}. `Fe humana' era creer a una persona humana, `fe divina' era creer a Dios\footnote{\cite[Cf.][2]{anscombe2008faith:tobelieve}: \enquote{At one time there was the following way of speaking: faith was distinguished as human and divine. Human faith was believing a mere human being; divine faith was believing God}.}. Así por ejemplo: \enquote*{Abrám creyó a Dios (\textgreek{ἐπίστευσεν τῷ Θεῷ}) y esto se le contó como justicia} (Gn 15,6). De tal modo que es llamado \enquote*{padre de la fe} (Cf.~Rm 4 y Ga 3,7). La pregunta \enquote*{¿qué es creer a alguien?} queda situada en el centro de este análisis\footnote{\cite[Cf.][116]{anscombe1981erp:faith}: \enquote{It is clear that the topic I introduced of \emph{believing somebody} is in the middle of our target}.}. Anscombe emplea esta noción para indagar sobre la estructura del creer que está relacionada con la dinámica de la fe. Creer a alguien implica ciertas presuposiciones, al hablar de la fe como \enquote*{creer a Dios que $p$} se le atribuye la misma implicación. La cuestión acerca de lo que es creer a alguien resultará de suficiente interés a Anscombe como para dedicarle su propio artículo y en esta investigación, sin duda, juega un papel importante.

Para exponer el desarrollo del análisis que Elizabeth recorre en su discusión podemos atender a tres movimientos principales realizados en su argumentación. Primero se fija en el carácter racional de la fe y recuerda una cierta apologética en la que se le atribuyó este carácter a los llamados preámbulos y el paso de estos a la fe misma; y establece que la designación correcta de estos `preámbulos de la fe', al menos para parte de ellos, es más bien `presuposiciones'. En segundo lugar describe cuáles son las presuposiciones implicadas en creer a una persona humana cuando esta comunica algo. En tercer lugar examina el fenómeno particular del creer cuando la comunicación viene de Dios.

Elizabeth nos introduce a su reflexión recordando una época en la que la racionalidad de la fe estuvo en el foco de cierta discusión teológica: \blockquote[{\Cite[113]{anscombe1981erp:faith}}: \enquote{There was in a preceding time a professed enthusiasm for rationality, perhaps inspired by the teaching of Vatican I against fideism, certainly carried along by the promotion of neo-thomist studies \textelp{} the word was that the Catholic Christian faith was \emph{rational}, and a problem, to those able to feel it as a problem, was how it was \emph{gratuitous}\,---\,a special gift of grace. Why would it \emph{essentially} need the promptings of grace to follow a process of reasoning?}]{Hubo en una época pasada un profuso entusiasmo por la racionalidad, quizás inspirado por la enseñanza del Vaticano~I contra el fideísmo, ciertamente sostenidos por la promoción de estudios neo-tomistas [\ldots] la noticia era que la fe Cristiana Católica era \emph{racional}, y el problema, para aquellos capaces de sentirlo como tal, era cómo era \emph{gratuita}\,---\,un don especial de la gracia. ¿Por qué tendría que ser \emph{esencialmente} necesaria la ayuda de la gracia para seguir un proceso de razonamiento?} 

Según la descripción de Anscombe este proceso de razonamiento consistía en una especie de cadena de demostraciones; se afirmaba a Dios, y luego la divinidad de Jesús, y después la institución de la Iglesia por él con el Papa a la cabeza con la encomienda de enseñar. Cada demostración permitía justificar la certeza de la verdad de las enseñanzas de la Iglesia\footnote{\cite[Cf.][113]{anscombe1981erp:faith}: \enquote{It was as if we were assured there was a chain of proof. First God. Then, the divinity of Jesus Christ. Then, \emph{his} establishment of a church with a Pope at the head of it and with a teaching commission from him. This body was readily identifiable. Hence you could demonstrate the truth of what the Church taught. Faith, indeed, is not the same thing as knowledge --- but that could be accounted for by the \emph{extrinsic} character of the proofs of the \emph{de fide} doctrines. \textelp{} For matters which were strictly `of faith' intrinsic proofs were not possible, and that was why faith contrasted with `knowledge'}.}. Elizabeth argumenta que esta breve descripción representa una postura quizás más `extravagante'\footnote{\cite[Cf.][113]{anscombe1981erp:faith}: \enquote{This is a picture of the more extravagant form of this teaching. A more sober variation would relate to the Church that our Lord established. In this variant one wouldn't identify the church by its having the Pope, but otherwise; and one would discover that it had a Pope and that that was all right}.}, y otras variantes más sobrias enfatizaban más la figura de la Iglesia, o la divinidad de Jesús\footnote{\cite[Cf.][113-114]{anscombe1981erp:faith}: \enquote{A yet more sober variant would have avoided trading on the cultural inheritance for which the name of Jesus was so holy that it was easy to go straight from the belief in God to belief in Jesus as God's Son}.}. A juicio de Anscombe esta actitud más sobria o crítica ante aquellos que pretendían defender la razonabilidad de la fe como una quasi demostrabilidad sirvió en beneficio de la veracidad y la honestidad\footnote{\cite[Cf.][114]{anscombe1981erp:faith}: \enquote{The `sober variants' would have a disadvantage for the propagandists of the rationality (near demonstrability) of faith --- though a great advantage in respect of honesty and truthfulness}.}. Ciertamente estas opiniones presentaban problemas. Era obvio que identificar la Iglesia católica que conocemos con la Iglesia que Cristo instituyó no era tarea fácil y necesitaba conocimiento y técnica\footnote{\cite[Cf.][114]{anscombe1981erp:faith}: \enquote{The disadvantage was that no one could suppose it quite easy for anyone to see that what Jesus established was matched by the Catholic Church that we know. \textelp{} it was \emph{obvious} that learning and skill would be required to make the identification}.}. Entonces ¿qué carácter tiene la certeza atribuida a la fe? \blockquote[{\Cite[114]{anscombe1981erp:faith}}: \enquote{The so-called preambles of faith could not possibly have the sort of certainty that \emph{it} had. And if less, then where was the vaunted rationality?}]{Los llamados preámbulos de la fe no podrían tener el tipo de certeza que \emph{esta} tiene. Y si es menos, entonces ¿dónde está la racionalidad proclamada?}. 

Otro problema tenía que ver con la fe de los doctos y los sencillos, ¿aquellos que no conocen estos argumentos tienen un tipo de fe inferior a los doctos? Por otra parte, los que han estudiado ¿realmente conocen todas estas cosas? Ser racional en tener fe implicaba sostener la creencia de que el conocimiento estaba ahí para argumentar y demostrar la verdad de Dios, de Cristo y de la Iglesia, quizá repartido entre algunos expertos o al menos de manera teorética. Todo esto hacía problemáticas estas opiniones\footnote{\cite[Cf.][114]{anscombe1981erp:faith}: \enquote{the implication was that the knowledge was there somehow, perhaps scattered through different learned heads, perhaps merely theoretically and abstractly available. In the belief that this was so, one was being rational in having faith. But then it had to be acknowledged that all this was problematic --- and so adherence to faith was really a matter of hanging on, and both its being a \emph{gift} and its \emph{voluntariness} would \emph{at this point} be stressed}.}.

Anscombe describe brevemente estas discusiones y este modo de hacer apologética que fue empleado en el pasado y ya no se usa en las discusiones de su época. Esto, dice, \blockquote[{\Cite[114]{anscombe1981erp:faith}}: \enquote{not necessarily because better thoughts about faith are now common; there is a vacuum where these ideas once were prominent}.]{no necesariamente porque sean ahora más comunes pensamientos mejores sobre la fe; hay un vacío en donde estas ideas antes fueron prominentes}. Sin embargo opina que no hay que lamentar que estas opiniones hayan pasado, y añade: \blockquote[{\Cite[114]{anscombe1981erp:faith}}: \enquote{They attached the character of `rationality' entirely to what were called the preambles and to the passage from the preambles to faith itself. But both these preambles and that passage were in fact an `ideal' construction \textelp{} `fanciful', indeed dreamed up according to prejudices: prejudices, that is, about what it is to be reasonable in holding a belief}.]{Estas atribuían el carácter de `racionalidad' por entero a lo que se llamaron los preámbulos y al paso de estos preámbulos a la fe misma. Pero tanto estos preámbulos como ese paso eran realmente una construcción `ideal' \textelp{} `imaginaria', ciertamente soñada de acuerdo a prejuicios: esto es, prejuicios sobre qué es lo que es ser razonable en sostener una creencia}.

De acuerdo al objetivo trazado al inicio de su discusión, Anscombe busca presentar una descripción del carácter racional de la fe libre de estos prejuicios. En el centro de su propuesta está la comprensión de `fe' como `creer a $x$ que $p$' y, partiendo de esto, el valor de los presupuestos involucrados en creer una comunicación. Comienza, entonces, proponiendo un ejemplo: \blockquote[{\Cite[114]{anscombe1981erp:faith}}: \enquote{You receive a letter from someone you know, let's call him Jones. In it, he tells you that his wife has died. You believe him. That is, you now believe that his wife has died because you believe \emph{him}. Let us call this just what it used to be called, ``human faith''. That sense of ``faith'' still occurs on our language. ``Why'', someone may be asked, ``do you believe such-and-such?'', and he may reply ``I just took it on faith\,---\,so-and-so told me''}.]{Recibes una carta de alguien que conoces, llamémosle Jones. En ella te dice que su esposa ha muerto. Tu le crees. Esto es, ahora crees que su esposa ha muerto porque le crees a él. Llamemos a esto justo como solía ser llamado, ``fe humana''. Este sentido de ``fe'' todavía ocurre en nuestro lenguaje. ``Por qué'', se le puede preguntar a alguien, ``crees esto y aquello?'', y podría responder ``Lo tomé en buena fe\,---\,fulano me dijo''}. 

Al especificar este uso de `fe', Elizabeth busca justificar que la designación más adecuada para los llamados `preámbulos' de la fe, al menos para parte de ellos, es `presuposiciones'. En el ejemplo propuesto hay tres creencias implicadas en haberle creído a Jones, estas \blockquote[{\Cite[114]{anscombe1981erp:faith}}: \enquote{three convictions or assumptions are, logically, pressupositions that \emph{you} have if your belief that Jones' wife has died is a case of your believing Jones}.]{tres convicciones o supuestos son, lógicamente, presuposiciones que \emph{tú} tienes si tu creencia de que la esposa de Jones ha muerto es un caso de que crees a Jones}.

Al creerlo presupones primero que tu amigo Jones existe, segundo, que la carta viene verdaderamente de él, y tercero, que esto que crees es verdaderamente lo que la carta dice. Estas son presuposiciones tuyas, el que puedas llegar a creer la comunicación de la carta no presupone estas tres cosas de hecho, sino que tú crees estas tres cosas.

Ahora bien, `fe' en la tradición en la que ese concepto se origina se refiere a `fe divina' y significa `creer a Dios'. Según esta acepción la fe es absolutamente cierta, puesto que es creer a Dios y, si las presuposiciones son ciertas, conlleva creer sobre los mejores fundamentos a uno que habla con conocimiento perfecto. Lo problemático aquí sería en qué consiste creer a Dios.
%, pero antes de indagar más sobre esto, Anscombe estudia con más detalle las presuposiciones relacionadas con creer a una persona humana.

Después de discutir cómo puede atribuírsele a la fe algún carácter de racionalidad y haberse decidido por valorar las convicciones implicadas en la certeza que depositamos en lo que creemos porque creemos a alguien,
%Anscombe ahora nos adentra en el análisis de estas presuposiciones y la utilidad que puedan tener para comprender el fenómeno de la fe.
Anscombe se plantea algunas preguntas relacionadas con estas presuposiciones que discutiremos más adelante: (III, \S\ref{subsec:presups}, p.~\pageref{subsec:presups}). Aquí solo destacamos dos elementos adicionales sobre ellas discutidos por Anscombe.
%¿Qué es creer a alguien? Anscombe vuelve a su ejemplo. Creer a Jones, que su esposa ha muerto, ¿significa que el hecho de que Jones me diga esto es la \emph{causa} de mi creencia? o ¿significa que el hecho de que se comunique es mi \emph{evidencia} para creer en la muerte de su esposa? ¿Esto sería creer a Jones? No del todo. Puesto que podría ser que la comunicación llama mi atención sobre la cuestión, pero llego a la creencia por mi propio juicio. O puedo tomar lo que me están diciendo y pensar que la persona que me habla me está engañando y a la misma vez está equivocada en lo que me dice, entonces podría decir que creo lo que me dice porque me lo ha dicho, pero no estaría creyendo a la persona. Entonces ¿creer a alguien significa creer que la persona cree lo que me está diciendo? Ordinariamente asumimos esto, pero incluso puede imaginarse el caso en el que alguien me dice algo que cree, pero yo sé que en el origen de su creencia hay una falsedad y por tanto creo lo contrario de lo que esta persona cree y me dice, entonces tampoco estaría creyendole a ella. Sin embargo, en el caso de creer a un maestro, un profesor de historia por ejemplo, sería suficiente para creerle \emph{a él} que creas lo que dice porque lo ha dicho y piensas que no está mintiendo y piensas que lo que él cree es verdadero.
%Estas dificultades no aparecen si se puede establecer con certeza que la persona conoce lo que dice y no miente, sin embargo el tema de creer a alguien no es asunto sencillo. Hay, además, otras preguntas relacionadas con las presuposiciones involucradas en creer a alguien. Al creer lo que dice la comunicación presupones que Jones existe, que escribió la carta y que esta dice lo que has llegado a creer. Pero estos son tus presupuestos y no son condiciones de hecho. ¿Qué se puede decir del caso en el que de hecho no existe la persona que se cree que es quien se comunica? ¿Se puede decir que se está creyendo a Jones si es el caso que de hecho no existe? Si insistiéramos en decir que no se está creyendo en la persona que no existe, afirma Anscombe, \blockquote[{\Cite[117]{anscombe1981erp:faith}}: \enquote{you will deprive yourself of the best way of describing his situation: ``he believed this non-existent person''}]{te estarías privando de la mejor manera de describir esta situación: ``le creyó a esta persona no existente''}. De un antiguo que creyó en el oráculo del dios Apolo, por ejemplo, se puede decir efectivamente que creyó en Apolo\,---\,que no existe. Lo mismo se podría decir del caso en el que de hecho existe la persona, pero esta comunicación que se cree que viene de ella no proviene de ella de hecho.
%Dos elementos adicionales son destacados por Anscombe acerca de las presuposiciones.

Primero comenta que \blockquote[{\Cite[117]{anscombe1981erp:faith}}: \enquote{the presuppositions of faith are not themselves part of the content of what in a narrow sense is believed by faith}.]{las presuposiciones de la fe no son ellas mismas parte del contenido de lo que en un sentido estricto es creido por la fe}. En segundo lugar explica que hay también una \blockquote[{\Cite[118]{anscombe1981erp:faith}}: \enquote{difference between presuppositions of believing $N$ and believing such-and-such as coming from $N$. ``Pre-suppositions'' don't have to be temporarily prior beliefs}.]{diferencia entre las presuposiciones de creer a $N$ y creer esto o aquello como viniendo de $N$. Las ``pre-suposiciones'' no tienen que ser creencias temporalmente previas}. Elizabeth ilustra esto imaginando el caso en el que la carta dijera que viene de alguien: \enquote*{Esta es una carta de tu viejo amigo Jones}, y al leerla se ponga en duda esta afirmación, o incluso no se ponga en duda sino que se lea acríticamente, sin pensar en ello, entonces se cree lo que dice la carta, pero no se está contando con la credibilidad de Jones como garantía de que la carta viene de él, se tiene en cuenta lo que la carta dice, incluido el que viene de él, pero no se le está creyendo a él y en este sentido las presuposiciones y el contenido de lo que es la fe propiamente son distintos. 

Otra ilustración puede ser el caso en el que no se tiene un conocimiento previo de la persona que se comunica: \enquote*{Esto es de parte de un amigo desconocido\,---\,llámame $N$}. Imaginemos un prisionero que recibe una comunicación de esta naturaleza y en ella se le ofrecen ayudas para sus necesidades, no sabe si son genuinas, pero responde a la comunicación y recibe las ayudas prometidas. Este prisionero recibe otras comunicaciones que parecen ser de la misma persona y estas contienen nueva información. Al creer esta información el prisionero cree a $N$, pero su creencia en que $N$ existe y que las cartas vienen de él no son creer algo apoyándose en que $N$ lo ha dicho. Es en este sentido en el que \blockquote[{\Cite[118]{anscombe1981erp:faith}}: \enquote{the beliefs which \emph{are} cases of believing $N$ and the belief that $N$ exists are logically different}.]{las creencias que \emph{son} casos de creer a $N$ y la creencia de que $N$ existe son lógicamente diferentes}.

En todos estos ejemplos Anscombe ha recurrido a comunicaciones entre personas humanas. ¿Qué se puede decir del caso en que la comunicación viene de Dios? \blockquote[{\Cite[118]{anscombe1981erp:faith}}: \enquote{Suarez said that in every revelation God reveals that he reveals}.]{Suarez dijo que en cada revelación Dios revela que Él revela} y esto es como decir \blockquote[{\Cite[118]{anscombe1981erp:faith}}: \enquote{in every bit of information $N$ is also claiming (implicitly or explicitly, it doesn't matter which) that he is giving the prisioner information}.]{en cada pedazo de información $N$ está también declarando (implícita o explícitamente, no importa como) que está dando información al prisionero}. Y aquí hay una dificultad central en el asunto de la fe: \blockquote[{\Cite[118]{anscombe1981erp:faith}}: \enquote{In all other cases we have been considering, it can be made clear \emph{what} it is for someone to believe someone. But what can it mean ``to believe God''? Could a learned clever man inform me on the authority of his learning, that the evidence is that God has spoken? No. The only posssible use of a learned clever man is as a \emph{causa removens prohibens}. There are gross obstacles in the received opinion of my time and in its characteristic ways of thinking, and someone learned and clever may be able to dissolve these}.]{En todos los otros casos que hemos estado considerando, puede ser aclarado \emph{qué} es que alguien crea a alguien. Pero ¿qué puede significar ``creer a Dios''? ¿Podría un hombre docto e inteligente informarme sobre la autoridad de su conocimiento, que la evidencia es que Dios ha hablado? No. El único uso posible para un hombre docto e inteligente es a modo de \emph{causa removens prohibens}. Hay grandes obstáculos en la opinion aceptada en mi época y en sus característicos modos de pensar, y alguien con inteligencia y conocimiento podría ser capaz de disolverlos}.

Con esto llegamos al tercer esfuerzo de Anscombe por arrojar luz sobre este tema. ¿Qué estamos creyendo cuando creemos que Dios ha hablado? Para hablar sobre esto Elizabeth recurre a una noción rabínica llamada \emph{Bath Qol} o la `hija de la voz': \blockquote[{\Cite[118-119]{anscombe1981erp:faith}}: \enquote{You hear a sentence as you stand in a crowd\,---\,a few words out of what someone is saying perhaps: it leaps at you, it `speaks to your condition'. Thus there was a man standing in a crowd and he heard a woman saying ``Why are you wasting your time?'' He had been dithering about, putting off the question of becoming a Catholic. The voice struck him to the heart and he acted in obedience to it. Now, he did not have to suppose, nor did he suppose, that that remark was not made in the course of some exchange between the woman and her companion, which had nothing to do with him. But he believed that God had spoken to him in that voice. The same thing happened to St Augustine, hearing the child's cry, ``Tolle lege''}.]{Escuchas una oración mientras que estás en medio de una muchedumbre\,---\,algunas palabras de entre lo que alguien está diciendo: saltan hacia ti, `hablan a tu condición'. Así había un hombre que entre la muchedumbre escuchó una mujer que estaba diciendo ``¿Por qué estas desperdiciando tu tiempo?'' Había estado vacilando, ignorando la cuestión de hacerse católico. La voz le golpeó en el corazón y actuó en obediencia a ella. Ahora, él no tenía que suponer, ni de hecho supuso, que este comentario no fuera hecho en el curso de alguna conversación entre la mujer y su acompañante, la cuál no tenía nada que ver con él. Lo mismo ocurrió a San Agustín, al escuchar el grito del niño ``Tolle lege''}.

Ahora bien, todavía hace falta una aclaración adicional respecto de lo que significa decir que se cree que Dios habla. En los ejemplos anteriores estaba claro qué significa para alguien que \enquote*{cree a $X$} que \enquote*{$X$ está hablando}. Incluso en el caso de que no exista. Pero no es claro que Dios sea el que hable. Aquí, entender deidad como el objeto de adoración no es útil puesto que habría que definir adoración como el honor ofrecido a una deidad. En este sentido por `Dios' Anscombe no entiende el objeto de esta o aquella adoración; `Dios' no es un nombre propio, sino una `descripción definitiva' en el sentido técnico. Es decir es equivalente a `el uno y único dios verdadero'. Un ateo cree que Dios está entre los dioses que no son dioses, pero podría entender la identidad de `Dios' con `el uno y único dios'. En este sentido decir que Dios es el dios de Israel es decir que lo que Israel ha adorado como dios es `el uno y único dios verdadero'. Esto podría ser afirmado o negado por alguien incluso que considerara que esa expresión es vacía o no se refiere a nada.

Con esto, Anscombe llega a una descripción conclusiva: \blockquote[{\Cite[119-120]{anscombe1981erp:faith}}: \enquote{And so we can say this: the supposition that someone has faith is the supposition that he believes that something ---it may be a voice, it may be something he has been taught--- comes as a word from God. Faith is then the belief he accords to that word}.]{Y entonces podemos decir esto: la suposición de que alguien tiene fe es la suposición de que cree que algo ---puede ser una voz, puede ser algo que ha aprendido--- viene como una palabra de Dios. Fe es entonces la creencia que otorga a esa palabra}. Esto puede ser entendido por alguien que no tiene fe, sea que su actitud ante este fenómeno sea de reverencia, indiferencia u hostilidad. Esto además puede ser dicho en términos generales sobre el fenómeno de la fe. En el caso específico del que cree en Cristo: \blockquote[{\Cite[120]{anscombe1981erp:faith}}: \enquote{the Christian adds that such a belief is sometimes the truth, and that the consequent belief is only then what \emph{he} means by faith}.]{el cristiano añade que esta creencia es en ocasiones la verdad, y esta creencia consecuente es solo lo que \emph{él} entiende por fe}.

\vspace{2.83334em}
\vspace{1.41667em}
La premisa de que Anscombe entiende la fe como `saber testimonial' es una de las claves principales de nuestro estudio. La ruta de Elizabeth en este artículo ilustra 
el modo en que puede encontrarse una descripción del carácter testimonial de la revelación dentro de su obra. En esta reflexión ella parte de una descripción de la fe y sus presupuestos y nos conduce de tal modo que llegamos a preguntarnos sobre qué significa decir que se cree que Dios ha hablado. El camino para hablar de la revelación parte del análisis de la naturaleza de la creencia que es la fe como correlato de la comunicación divina (Cf. III, \S\ref{subsec:fecorrel}, p.~\pageref{subsec:fecorrel}).

\section{Cuestión sobre la Estructura del Testimonio}

\section{Profecías y Milagros}

\section{Sentido del Misterio y Racionabilidad}
\end{document}
