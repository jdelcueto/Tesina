\documentclass[../main.tex]{subfiles}
\begin{document}



\chapter{La Categoría del Testimonio en el Pensamiento de Elizabeth Anscombe}


Este es el capítulo central del trabajo. El recorrido que haremos a lo largo del pensamiento de Anscombe está orientado según las siguientes cuestiones:

\noindent- La legitimidad de la pregunta filosófica y religiosa sobre la verdad.\\
- Aclarar la naturaleza original del testimonio cristiano.\\
- Demostrar que el testimonio es un modo adecuado de conocer y de transmitir la verdad\\
- Examen de la relevancia del testimonio en nuestras vidas\\
- Lugar del testimonio en el panorama epistemológico y la tradición del debate sobre ese lugar\\
- Qué se puede decir para defender la extensa relevancia que se ha afirmado sobre nuestra confianza en el testimonio

Anscombe no crea un sistema o teoría general en el acercamiento a los problemas filosóficos, más bien toma cada caso en sus propios méritos. Esta evasión de sistema complica la tarea de estudiar su pensamiento. Como respuesta a esta dificultad es útil la metodología empleada por Teichmann en el estudio de la filosofía de Anscombe que él mismo describe en tres momentos: ``read thoroughly'', ``bring out the manifold connections between her thoughts on different topics'' y ``engage with what Anscombe says''.

Añado a pie de página los articulos de Anscombe a los que se hará referencia en cada apartado.


\section{Wittgenstein y Anscombe: La Razonabilidad de la Fe}
\footnote{
Ludwig Wittgenstein, 
Wittgenstein on Rules and Private Language, 
Wittgenstein, Frege and Ramsey, 
Wittgenstein: Whose Philosopher?, 
Wittgenstein's 'two cuts' in the history of philosophy, 
Consequences of the Picture Theory, 
On the form of Wittgsenstein's writing, 
Was Wittgenstein a conventionalist?, 
The Simplicity of the Tractatus, 
An Introduction to Wittgenstein's Tractatus
}

El primer apartado recorre la biógrafía de Anscombe; su desarrollo como filósofa y creyente. Estudia también su relación con Wittgenstein. En este camino examinamos el tema de la razonabilidad de la fe. Cómo Anscombe responde a este aspecto de la fe y cuáles cuestiones Wittgenstein plantea sobre este asunto. Al final de este apartado abrimos la pregunta sobre la verdad.

El primer interés filosófico de Anscombe fue en el tema de la causalidad.
El segundo la percepción.

For years, I would spend time, in cafés, for example, staring at objects saying to myself: "I see a packet. But what do I really see? How can I say that I see here anything more than a yellow expanse?" ...I always hated phenomenalism and felt trapped by it. I couldn't see my way out of it but I didn't believe it. It was no good pointing to difficulties about it, things which Russell found wrong with it, for example. The strength, the central nerve of it remained alive and raged achingly. It was only in Wittgenstein's classes in 1944 that I saw the nerve being extracted, the central thought I have got this, and I define "yellow" (say) as this being effectively attacked.[viii - ix, M\&PM]

\subsubsection{'Why should one tell the truth if it's to one's advantage to tell a lie?'}
<<¿Por qué uno debería de decir la verdad si es para beneficio de uno decir una mentira?>> De pie en un portal de su casa, con ocho o nueve años de edad, Wittgenstein no encontraba una objeción a esta consideración. Esta experiencia, si no fue decisiva para el futuro modo de vida del filósofo, es en cualquier caso característica de su naturaleza en esa época. 

\footnote{When I was 8 or 9 I had an experience which if not decisive for my future way of life was at any rate characteristic of my nature at the time. How it happened, I do not know: I only see myself standing in a doorway in our house and thinking 'Why should one tell the truth if it's to one's advantage to tell a lie?' I could see nothing against it. [...] my lies had the aim of making me appear agreeable in the eyes of others. They were simply lies out of cowardice. (Wittgenstein: A Life : Young Ludwig, 1889-1921, Volume 1 By Brian McGuinness p. 48)}

\subsubsection{'wild life striving to erupt into the open'}

\subsubsection{Tractatus: Connection between language, or thought, and reality}

In the Tractatus truth recieves a good deal more attention than meaning

Nature of philosophy
Accordingly, “the word ‘philosophy’ must mean something which stands above or below, but not beside the natural sciences” (TLP 4.111). Not surprisingly, then, “most of the propositions and questions to be found in philosophical works are not false but nonsensical” (TLP 4.003). Is, then, philosophy doomed to be nonsense (unsinnig), or, at best, senseless (sinnlos) when it does logic, but, in any case, meaningless? What is left for the philosopher to do, if traditional, or even revolutionary, propositions of metaphysics, epistemology, aesthetics, and ethics cannot be formulated in a sensical manner? The reply to these two questions is found in Wittgenstein's characterization of philosophy: philosophy is not a theory, or a doctrine, but rather an activity. It is an activity of clarification (of thoughts), and more so, of critique (of language). Described by Wittgenstein, it should be the philosopher's routine activity: to react or respond to the traditional philosophers' musings by showing them where they go wrong, using the tools provided by logical analysis. In other words, by showing them that (some of) their propositions are nonsense.

“All propositions are of equal value” (TLP 6.4)—that could also be the fundamental thought of the book. For it employs a measure of the value of propositions that is done by logic and the notion of limits. It is here, however, with the constraints on the value of propositions, that the tension in the Tractatus is most strongly felt. It becomes clear that the notions used by the Tractatus—the logical-philosophical notions—do not belong to the world and hence cannot be used to express anything meaningful. Since language, thought and the world, are all isomorphic, any attempt to say in logic (i.e., in language) “this and this there is in the world, that there is not” is doomed to be a failure, since it would mean that logic has got outside the limits of the world, i.e. of itself. That is to say, the Tractatus has gone over its own limits, and stands in danger of being nonsensical.

The “solution” to this tension is found in Wittgenstein's final remarks, where he uses the metaphor of the ladder to express the function of the Tractatus. It is to be used in order to climb on it, in order to “see the world rightly”; but thereafter it must be recognized as nonsense and be thrown away. Hence: “whereof one cannot speak, thereof one must be silent” (7).

\subsubsection{Philosophical Investigations: 'I'll teach you differences'}

meaning gets more attetntion than truth

par 599. In philosophy we do not draw conclusions. ``But it must be like this!'' is not a philosophical proposition. Philosophy only states what everyone admits.

nature of philosophy
In his later writings Wittgenstein holds, as he did in the Tractatus, that philosophers do not—or should not—supply a theory, neither do they provide explanations. “Philosophy just puts everything before us, and neither explains nor deduces anything.—Since everything lies open to view there is nothing to explain” (PI 126). The anti-theoretical stance is reminiscent of the early Wittgenstein, but there are manifest differences. Although the Tractatus precludes philosophical theories, it does construct a systematic edifice which results in the general form of the proposition, all the while relying on strict formal logic; the Investigations points out the therapeutic non-dogmatic nature of philosophy, verily instructing philosophers in the ways of therapy. “The work of the philosopher consists in marshalling reminders for a particular purpose” (PI 127). Working with reminders and series of examples, different problems are solved. Unlike the Tractatus which advanced one philosophical method, in the Investigations “there is not a single philosophical method, though there are indeed methods, different therapies, as it were” (PI 133d). This is directly related to Wittgenstein's eschewal of the logical form or of any a-priori generalization that can be discovered or made in philosophy. Trying to advance such general theses is a temptation which lures philosophers; but the real task of philosophy is both to make us aware of the temptation and to show us how to overcome it. Consequently “a philosophical problem has the form: ‘I don't know my way about.’” (PI 123), and hence the aim of philosophy is “to show the fly the way out of the fly-bottle” (PI 309).

The style of the Investigations is strikingly different from that of the Tractatus. Instead of strictly numbered sections which are organized hierarchically in programmatic order, the Investigations fragmentarily voices aphorisms about language-games, family resemblance, forms of life, “sometimes jumping, in a sudden change, from one area to another” (PI Preface). This variation in style is of course essential and is “connected with the very nature of the investigation” (PI Preface). As a matter of fact, Wittgenstein was acutely aware of the contrast between the two stages of his thought, suggesting publication of both texts together in order to make the contrast obvious and clear.

Still, it is precisely via the subject of the nature of philosophy that the fundamental continuity between these two stages, rather than the discrepancy between them, is to be found. In both cases philosophy serves, first, as critique of language. It is through analyzing language's illusive power that the philosopher can expose the traps of meaningless philosophical formulations. This means that what was formerly thought of as a philosophical problem may now dissolve “and this simply means that the philosophical problems should completely disappear” (PI 133). Two implications of this diagnosis, easily traced back in the Tractatus, are to be recognized. One is the inherent dialogical character of philosophy, which is a responsive activity: difficulties and torments are encountered which are then to be dissipated by philosophical therapy. In the Tractatus, this took the shape of advice: “The correct method in philosophy would really be the following: to say nothing except what can be said, i.e. propositions of natural science … and then whenever someone else wanted to say something metaphysical, to demonstrate to him that he had failed to give a meaning to certain signs in his propositions” (TLP 6.53) The second, more far- reaching, “discovery” in the Investigations “is the one that enables me to break off philosophizing when I want to” (PI 133). This has been taken to revert back to the ladder metaphor and the injunction to silence in the Tractatus.

\subsubsection{'I have loved the truth'}

'I do not mean, when I say that, that I have the truth'.

\subsubsection {Differences in Anscombe}
In Anscombe's writing, the two topics of meaning and truth, insofar as they can be separated, seem to enjoy roguhly equal status, although her manner of with each is not the same.

A. Almost always invokes meaning in the course of dealing with a topic not belonging as such to philosophy of language.
By contrast A. treats truth much more as a topic in its own right. 

For A. in indicative sentences sensefulness is associated with bivalence. W. and Russell is in the same side of the fence. For them 'having a sense' was one and the same thing with being true or false. A. says that W. remained on this side of the fence his whole life.(IWT 58, 59) (TEICH192)

``It was left to the moderns to deduce what could be from what could hold of thought, as we see Hume to have done. This trend is still strong. But the ancientys had the better approach, arguing only that a thought was impossible because the thing was impossible, or as the Tractatus puts i, 'an impossible thought is an impossible thought''. (FPW,p .xi) (TEICH 193)

A. does not swallow the whole of the picture theory of propositions. But she sees what is probably the most illuminating thing about W.'s comparison of propositions and pictures; namely, this janus-faced aspect of a proposition, an aspect that can be expressed in various ways... in her lecture ``la verdad'' A. raises the question having to do with the primacy of truth over falsehood. What is the inequality of truth and falsehood?
Anselm solution to this is to ascribe a purpose to the assertion, that of saying what is tha case. What is to use a proposition to say what is the case? Could we adopt the rule of using propositional signs to say what is not the case?
Can we not make ourselves understood with false propositions just as we have done up till now with true ones? So long as it is known that they are false. No! For a proposition is true if we use it to say things stand in a certain way, and they do; and if by 'p' we mean not-p  and things stand as we mean that they do, then, construed in the new way, 'p' is true and not false.(TRACTATUS 4.062)

A. asks: Does the general impossibility [of exchanging the roles of true and false] contain the whole substance of the ``not equally justified relations''? A. takes W. to have said that truth and falsehood do not bear equally justified relations to the things depicted. 

How does truth and not falsehood bear a 'justified relation' to the thing signified?
Teichmann thinks the answer can be found in A.'s explanation of practical necessity. It has two strands: an account of the nature of stopping/forcing modals; an account of the aristotelian necessity of our going in for the practice within which those modals have force.

\section{La pregunta sobre la Verdad}
\footnote{
Truth: Anselm and Wittgenstein, 
Truth: Anselm or Thomas?, 
Anselm and the Unity of Truth, 
A theory of Language?, 
Necessity and Truth, 
Thought and Action in Aristotle: What is Practical Truth?, 
Practical Truth
}

Atendemos en el segundo apartado la pregunta sobre la verdad en los escritos de Anscombe. Este tema nos conducirá a la cuestión sobre la verdad de la fe.

\section{Fe, verdad y testimonio}
\footnote{
Faith, 
What is to believe someone?, 
A Reply to Mr. C. S. Lewis's Argument that “Naturalism” is Self- Refuting, 
Has Mankind One Soul: An Angel Distributed among many Bodies?, 
Human Essence, 
La esencia Humana, 
Plato, Soul and 'the Unity of Apperception', 
Why Anselm's Proof in the Proslogion in not an onthological argument, 
On the Hatred of God, 
On Attachment to Things and Obedience to God, 
On being on Good Faith, 
On Humanae Vitae, 
Philosophers and Economists: Two Philosphers' Objections to Usury, 
Retractation, 
Sin: the McGivney Lectures, 
The Inmortality of the Soul, 
Two Moral Theologians, 
You Can Have Sex without Children: Christianity and the New Offer, 
Morality, 
Modern Moral Philosophy
}

Anscombe estudia el tema de la fe en ``Faith'' donde nos introducirá al tema del testimonio. Afirma: ``the supposition that someone has faith is the supposition that he believes that something --it may be a voice, it may be something he has been thaught-- comes as a word from God. Faith is then the belief he accords to that word.'' La relación entre fe y testimonio queda remarcada en ``What is It to Believe Someone?''. Estos dos escritos nos ofrecen el vínculo entre la pregunta sobre la verdad, la fe y el testimonio. Al final de este apartado quedará abierta la pregunta sobre el valor epistemológico del testimonio.

\section{La tradición sobre el valor epistemológico del testimonio}
\footnote{
Hume and Julius Caesar, 
Hume on causality: introductory, 
The Reality of the Past, 
Causality and Determination, 
Causality and Extensionality, 
“Whatever has a beginning of existence must have a cause”: Hume's Argument Exposed, 
Times, Beginnings and Causes, 
Before and After, 
The Causation of Action, 
Chisolm on Action, 
Action, Intention and 'Double Effect', 
Part Three: Causality and time
Aristotle and the Sea Battle: De Interpretatione, Chapter IX, 
Prophecy and Miracles, 
Hume on Miracles, 
Modern Moral Philosophy, 
Good and Bad Human Action
}

Realizamos el estudio sobre el lugar epistemológico del testimonio y la tradición de ese lugar desde las aportaciones de Hume. En ``Hume and Julius Caesar'' Anscombe plantea la postura de Hume sobre el conocimiento por testimonio en el conocimiento de la historia y argumenta sobre ella. ``Prophecy and Miracles'' y ``Hume on Miracles'' nos permitiran considerar también el valor del testimonio de narraciones extraordinarias. 

\section{El testimonio en el lenguaje epistémico y creyente}
\footnote{
On Wisdom, 
Knowledge and Certainty, 
Knowledge and Reverence for Human Life, 
'The General Form of Proposition', 
Comments on Professor R. L. Gregory's Paper on Perception, 
On Brute Facts, 
Will and Emotion, 
Memory, 'Experience' and Causation, 
Understanding Proofs: Meno, 85d9 – 86c2, 
Subjunctive Conditionals, 
What is it to Believe Someone?, 
The Intentionality of Sensation, 
Substance, 
The Subjectivity of Sensation, 
Events in the mind, 
On Sensations of Position, 
Intention, 
Pretending, 
Practical Inference
What is it to Believe Someone?
Authority in Morals, 
On the Source of the Authority of the State, 
The Moral Enviroment of the Child, 
On Promising and its justice, and Whether it Need be Respected in Foro Interno, 
Rules, Rights and Promises, 
The Two Kinds of error in action
}

Los temas tratados en los anterirores apartados nos han dejado con algunos terminos epistémicos relacionados con el lenguaje sobre el testimonio como autoridad, creer y confiar. En este apartado los examinamos con más detalle y comparamos el testimonio con otros terminos relacionados con el conocimiento como son la percepción, la memoria y los sentidos. Se trata de considerar y valorar el testimonio como parte del lenguaje epistémico y como parte del lenguaje sobre la fe. Al final de este apartado abrimos la pregunta sobre el misterio.

\section{Sentido, sinsentido y misterio}
\footnote{
`Mysticism' and Solipsism, 
Analytical Philosophy and the Sipirituality of Man, 
On Transubstantiation, 
Parmenides, Mystery and Contradiction, 
The Question of Linguistic Idealism, 
Paganism, Superstition and Philosophy, 
On Piety, or: Plato's Euthyphro.
}

En este último apratado examinamos el misterio, el sentido y el sinsentido. Anscombe afirma en ``The Question of Linguistic Idealism'': ``In the Catholic faith, certain beliefs (such as the Trinity, the Incarnation, the Eucharist) are called 'mysteries'; this means at the very least that it is neither possible to demonstrate them nor possible to show once and for all that they are not contradictory and absurd. On the other hand contradiction and absurdity is not embraced; <<this can be disproved, but I still believe it>> is not an attitude of faith at all.''(QLI, 122) Anscombe se pregunta cómo se puede distinguir entre ``nonsense'' y ``mystery''. Su respuesta tiene que ver con nuestro tema del testimonio y en su escrito ``On transubstantiation'' encontramos un buen lugar para culminar el recorrido por su pensamiento.

\end{document}