\documentclass[../main.tex]{subfiles}
\begin{document}



\chapter*{La Categoría del Testimonio en el Pensamiento de Elizabeth Anscombe}


En una ocasión Wittgenstein recibió a Anscombe con la pregunta: <<¿Por qué la gente dice que era natural pensar que el sol giraba alrededor de la tierra en lugar de que la tierra giraba en su eje?>> Elizabeth contestó: <<Supongo que porque se veía como si el sol girara alrededor de la tierra.>> <<Bueno\ldots>>, añadió Wittgenstein, <<¿cómo se hubiera visto si se hubiera \emph{visto} como si la tierra girara en su propio eje?>> A esta pregunta Anscombe reaccionó extendiendo las manos delante de ella con las palmas hacia arriba y, levantándolas desde sus rodillas con un movimiento circular, se inclinó hacia atrás asumiendo una expresión de mareo. <<¡Exactamente!>> exclamó Wittgenstein.\footcite[151]{IWT}

Anscombe estaba familiarizada con este método de análisis de las proposiciones; Wittgenstein buscaba mostrar que ella no había provisto significado o referencia para ciertos signos en su afirmación y éste era el propósito de su segunda pregunta. Al cuestionar ``¿cómo se hubiera visto como si la tierra girara en su propio eje?'' sale a relucir que hasta aquél momento Anscombe no había ofrecido ningún significado relevante para su expresión ``se veía como si'' en su respuesta ``se veía como si el sol girara alrededor de la tierra''. 

Este modo de criticar una proposición desvelando que no expresa un pensamiento verdadero ilustra los principios propuestos en el \emph{Tractatus} y recuerda una de sus tesis más conocidas: 

6.53 \emph{``El método correcto para la filosofía sería este. No decir nada excepto lo que pueda ser dicho, esto es, proposiciones de la ciencia natural, es decir, algo que no tiene nada que ver con la filosofía: y luego siempre, cuando alguien quiera decir algo metafísico, demostrarle que no ha logrado dar significado a ciertos signos en sus proposiciones. Este método sería insatisfactorio para la otra persona --no tendría la impresión de que le estuviéramos enseñando filosofía-- pero este método sería el único estrictamente correcto.'' \footcite[p. 107--108]{tractatus}}

Wittgenstein causó una diferencia en el modo de hacer filosofía con el Tractatus. Anscombe remite la atestación de un filósofo austriaco que describía el efecto cataclísmico diciendo que profesores largamente consolidados se deshacían de sus viejos libros; la tarea consistía ahora en hacer filosofía en el modo indicado por el Tractatus.\footcite[p.181]{twocuts}   a donde la describe como una actividad y no una teoría cuyo objeto es la clarificación lógica de los pensamientos.\footcite[4.112 p. 52]{tractatus}
La mayoría de las proposiciones y de las preguntas que se pueden encontrar en las obras filosóficas no son falsas, sino carentes de significado. Consecuentemente, no podemos dar ninguna respuesta a preguntas de este tipo, sino sólo hacer notar que no tienen significado. La mayoría de las proposiciones y de las preguntas de los filósofos obedecen a nuestro fracaso a la hora de entender la lógica de nuestro lenguaje. (Pertenecen a la misma clase de preguntas que la pregunta de si el bien es más o menos idéntico que la belleza.) Y no es sorprendente que los problemas más profundos no sean de hecho ningún problema en absoluto.

Un pensamiento es una proposición con sentido.

\end{document}