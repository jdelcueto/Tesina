\documentclass[../main.tex]{subfiles}
\begin{document}



\chapter{La Categoría del Testimonio en el Pensamiento de Elizabeth Anscombe}


Este es el capítulo central del trabajo. El recorrido que haremos a lo largo del pensamiento de Anscombe está orientado según las siguientes cuestiones:

\noindent- La legitimidad de la pregunta filosófica y religiosa sobre la verdad.\\
- Aclarar la naturaleza original del testimonio cristiano.\\
- Demostrar que el testimonio es un modo adecuado de conocer y de transmitir la verdad\\
- Examen de la relevancia del testimonio en nuestras vidas\\
- Lugar del testimonio en el panorama epistemológico y la tradición del debate sobre ese lugar\\
- Qué se puede decir para defender la extensa relevancia que se ha afirmado sobre nuestra confianza en el testimonio

Anscombe no crea un sistema o teoría general en el acercamiento a los problemas filosóficos, más bien toma cada caso en sus propios méritos. Esta evasión de sistema complica la tarea de estudiar su pensamiento. Como respuesta a esta dificultad es útil la metodología empleada por Teichmann en el estudio de la filosofía de Anscombe que él mismo describe en tres momentos: ``read thoroughly'', ``bring out the manifold connections between her thoughts on different topics'' y ``engage with what Anscombe says''.

Añado a pie de página los articulos de Anscombe a los que se hará referencia en cada apartado.


\section{Wittgenstein y Anscombe: La Razonabilidad de la Fe}
\footnote{
Ludwig Wittgenstein, 
Wittgenstein on Rules and Private Language, 
Wittgenstein, Frege and Ramsey, 
Wittgenstein: Whose Philosopher?, 
Wittgenstein's 'two cuts' in the history of philosophy, 
Consequences of the Picture Theory, 
On the form of Wittgsenstein's writing, 
Was Wittgenstein a conventionalist?, 
The Simplicity of the Tractatus, 
An Introduction to Wittgenstein's Tractatus
}

El primer apartado recorre la biógrafía de Anscombe; su desarrollo como filósofa y creyente. Estudia también su relación con Wittgenstein. En este camino examinamos el tema de la razonabilidad de la fe. Cómo Anscombe responde a este aspecto de la fe y cuáles cuestiones Wittgenstein plantea sobre este asunto. Al final de este apartado abrimos la pregunta sobre la verdad.

El primer interés filosófico de Anscombe fue en el tema de la causalidad.
El segundo la percepción.

For years, I would spend time, in cafés, for example, staring at objects saying to myself: "I see a packet. But what do I really see? How can I say that I see here anything more than a yellow expanse?" ...I always hated phenomenalism and felt trapped by it. I couldn't see my way out of it but I didn't believe it. It was no good pointing to difficulties about it, things which Russell found wrong with it, for example. The strength, the central nerve of it remained alive and raged achingly. It was only in Wittgenstein's classes in 1944 that I saw the nerve being extracted, the central thought I have got this, and I define "yellow" (say) as this being effectively attacked.[viii - ix, M&PM]

\section{La pregunta sobre la Verdad}
\footnote{
Truth: Anselm and Wittgenstein, 
Truth: Anselm or Thomas?, 
Anselm and the Unity of Truth, 
A theory of Language?, 
Necessity and Truth, 
Thought and Action in Aristotle: What is Practical Truth?, 
Practical Truth
}

Atendemos en el segundo apartado la pregunta sobre la verdad en los escritos de Anscombe. Este tema nos conducirá a la cuestión sobre la verdad de la fe.

\section{Fe, verdad y testimonio}
\footnote{
Faith, 
What is to believe someone?, 
A Reply to Mr. C. S. Lewis's Argument that “Naturalism” is Self- Refuting, 
Has Mankind One Soul: An Angel Distributed among many Bodies?, 
Human Essence, 
La esencia Humana, 
Plato, Soul and 'the Unity of Apperception', 
Why Anselm's Proof in the Proslogion in not an onthological argument, 
On the Hatred of God, 
On Attachment to Things and Obedience to God, 
On being on Good Faith, 
On Humanae Vitae, 
Philosophers and Economists: Two Philosphers' Objections to Usury, 
Retractation, 
Sin: the McGivney Lectures, 
The Inmortality of the Soul, 
Two Moral Theologians, 
You Can Have Sex without Children: Christianity and the New Offer, 
Morality, 
Modern Moral Philosophy
}

Anscombe estudia el tema de la fe en ``Faith'' donde nos introducirá al tema del testimonio. Afirma: ``the supposition that someone has faith is the supposition that he believes that something --it may be a voice, it may be something he has been thaught-- comes as a word from God. Faith is then the belief he accords to that word.'' La relación entre fe y testimonio queda remarcada en ``What is It to Believe Someone?''. Estos dos escritos nos ofrecen el vínculo entre la pregunta sobre la verdad, la fe y el testimonio. Al final de este apartado quedará abierta la pregunta sobre el valor epistemológico del testimonio.

\section{La tradición sobre el valor epistemológico del testimonio}
\footnote{
Hume and Julius Caesar, 
Hume on causality: introductory, 
The Reality of the Past, 
Causality and Determination, 
Causality and Extensionality, 
“Whatever has a beginning of existence must have a cause”: Hume's Argument Exposed, 
Times, Beginnings and Causes, 
Before and After, 
The Causation of Action, 
Chisolm on Action, 
Action, Intention and 'Double Effect', 
Part Three: Causality and time
Aristotle and the Sea Battle: De Interpretatione, Chapter IX, 
Prophecy and Miracles, 
Hume on Miracles, 
Modern Moral Philosophy, 
Good and Bad Human Action
}

Realizamos el estudio sobre el lugar epistemológico del testimonio y la tradición de ese lugar desde las aportaciones de Hume. En ``Hume and Julius Caesar'' Anscombe plantea la postura de Hume sobre el conocimiento por testimonio en el conocimiento de la historia y argumenta sobre ella. ``Prophecy and Miracles'' y ``Hume on Miracles'' nos permitiran considerar también el valor del testimonio de narraciones extraordinarias. 

\section{El testimonio en el lenguaje epistémico y creyente}
\footnote{
On Wisdom, 
Knowledge and Certainty, 
Knowledge and Reverence for Human Life, 
'The General Form of Proposition', 
Comments on Professor R. L. Gregory's Paper on Perception, 
On Brute Facts, 
Will and Emotion, 
Memory, 'Experience' and Causation, 
Understanding Proofs: Meno, 85d9 – 86c2, 
Subjunctive Conditionals, 
What is it to Believe Someone?, 
The Intentionality of Sensation, 
Substance, 
The Subjectivity of Sensation, 
Events in the mind, 
On Sensations of Position, 
Intention, 
Pretending, 
Practical Inference
What is it to Believe Someone?
Authority in Morals, 
On the Source of the Authority of the State, 
The Moral Enviroment of the Child, 
On Promising and its justice, and Whether it Need be Respected in Foro Interno, 
Rules, Rights and Promises, 
The Two Kinds of error in action
}

Los temas tratados en los anterirores apartados nos han dejado con algunos terminos epistémicos relacionados con el lenguaje sobre el testimonio como autoridad, creer y confiar. En este apartado los examinamos con más detalle y comparamos el testimonio con otros terminos relacionados con el conocimiento como son la percepción, la memoria y los sentidos. Se trata de considerar y valorar el testimonio como parte del lenguaje epistémico y como parte del lenguaje sobre la fe. Al final de este apartado abrimos la pregunta sobre el misterio.

\section{Sentido, sinsentido y misterio}
\footnote{
`Mysticism' and Solipsism, 
Analytical Philosophy and the Sipirituality of Man, 
On Transubstantiation, 
Parmenides, Mystery and Contradiction, 
The Question of Linguistic Idealism, 
Paganism, Superstition and Philosophy, 
On Piety, or: Plato's Euthyphro.
}

En este último apratado examinamos el misterio, el sentido y el sinsentido. Anscombe afirma en ``The Question of Linguistic Idealism'': ``In the Catholic faith, certain beliefs (such as the Trinity, the Incarnation, the Eucharist) are called 'mysteries'; this means at the very least that it is neither possible to demonstrate them nor possible to show once and for all that they are not contradictory and absurd. On the other hand contradiction and absurdity is not embraced; <<this can be disproved, but I still believe it>> is not an attitude of faith at all.''(QLI, 122) Anscombe se pregunta cómo se puede distinguir entre ``nonsense'' y ``mystery''. Su respuesta tiene que ver con nuestro tema del testimonio y en su escrito ``On transubstantiation'' encontramos un buen lugar para culminar el recorrido por su pensamiento.

\end{document}