\documentclass[../main.tex]{subfiles}
\begin{document}
\setcounter{chapter}{0}

\chapter{Introducción General al Problema del Testimonio}
% Este capítulo se compone de tres apartados:

% El primer apartado es una introducción a todo el trabajo y se
% plantea cuál es la naturaleza de la pregunta que la investigación
% pretende responder
% LA NATURALEZA DE LA PREGUNTA SOBRE EL TESTIMONIO
%\input{./ch1/natura_quaestio.tex}

% El segundo apartado examina el uso que se hace en la escritura de la categoría
% del testimonio. Fundamentalmente enfatiza el uso del testimonio como
% estructura de la Revelación misma
% LA CATEGORÍA DEL TESTIMONIO EN LA SAGRADA ESCRITURA
%\section{La Categoría del Testimonio en la Sagrada Escritura}

\subsection{La Revelación descrita con estrucura testimonial}

La Iglesia de hoy, como María, conserva el Evangelio meditándolo en su
corazón.\footnote{Cf.~Lc 2,19} Así está presente en el centro de la comunidad
creyente el anuncio de Cristo vivo como fundamento de su esperanza en cada etapa
de la historia. Este motivo de esperanza conservado es también compartido y
expresado, según la enseñanza del apóstol:
\blockquote[1Pe 3,15]{glorificad a Cristo en vuestros corazones, dispuestos
  siempre a dar explicación a todo el que os pida una razón de vuestra
  esperanza}.

Este Evangelio atesorado como fundamento en el centro de la vida de la comunidad
eclesial, así como Buena Nueva proclamada y transmitida en el tiempo y en el
mundo puede ser comprendido como tres testimonios que son uno: \enquote{palabra
  vivida en el Espíritu}.\footnote{\cite[Cf.~][110]{latourelle1975et}: Car c'est
  L'Esprit qui posse l'Eglise à poursuivre son oeuvre d'évangelisation; c'est
  l'Esprit qui inspire la foi, la nourrit et l'approfondit. C'est l'Esprit qui
  relie entre eux ces trois témoignages qui n'en font qu'un: celui de la
  parole-vécue-dans-l'Esprit. Par son témoignage, l'Esprit intériorise le
  témoignage extérieur de la Bonne Nouvelle du salut en Jésus-Christ et le porte
  à l'accomplissement de la foi, qui est la réponse d'amour de l'humanité à
  l'appel d'amour du Père par le Christ. Ver también \cite[582]{ninot2009tf}
  donde este triple testimonio sirve para orientar la reflexión sobre el
  testimonio como vía empírica de la credibilidad de la Iglesia.}

La Evangelización puede ser entendida en este sentido como testimonio de la
\enquote{palabra de vida}\footnote{1Jn 1,1} que los apóstoles anuncian como
testigos de lo que han contemplado y palpado\footnote{ibíd.}. Es también el
testimonio de los cristianos que, acogiendo esta palabra, la viven, poniendo por
obra lo que ella enseña. Es además testimonio del Espíritu Santo que interioriza
el testimonio externo de la Buena Noticia y lo lleva al cumplimiento de la fe en
cada persona.\autocite[Cf.~][110]{latourelle1975et} Es el Espíritu el que
santifica y fecunda la acción de los cristianos, es tambíen el que impulsa y
sostiene la acción de la Iglesia; es el Espíritu el que inspira la fe, la nutre
y la profundiza.\autocite[Cf.~][110]{latourelle1975et}

Este dinamísmo fundamental que puede encontrarse vivo hoy en la comunidad de la
Iglesia ha actuado en ella desde su origen y le ha acompañado en cada época.
Según esto es posible valorar lo que se transmite en la tradición eclesial como
la perpetuación de la actividad de Cristo y los apóstoles, que es a su vez
proyección del testimonio divino.\footnote{\cite[Cf.~][573]{ninot2009tf}:
  \enquote{el testimonio divino se proyecta luego en el apostólico y se perpetúa
    en el testimonio eclesial. Por eso, el testimonio es revelación en la
    actividad de Cristo y de los apóstoles y es transmisión de la revelación en
    la tradición eclesial.}}

En la actividad de Cristo el testimonio divino queda proyectado como
interpelación a la libertad realizada por la identidad propia de Jesús:
\blockquote[Jn 4,10]{Si conocieras el don de Dios y quién es el que te dice
  \enquote{dame de beber} le pedirías tu, y él te daría agua viva};
\blockquote{\enquote{¿Crees tú en el Hijo del hombre?}\textelp{} \enquote{¿Y
    quién es, Señor, para que crea en él?}\textelp{} \enquote{Lo estás viendo:
    el que te está hablando, ese es}}.\footnote{Jn 9, 35--37} En la actividad
apostólica, el testimonio divino sigue interpelando la libertad humana como
manifestación de Jesús Resucitado. Los apóstoles actuan como testigos de los
acontecimientos de la Pascua de Jesús y su valor
salvífico\autocite[Cf.][576]{ninot2009tf} y este testimonio es descrito como
acción del Espíritu que impulsa la tarea apostólica y que da nueva vida a los
que acogen el anuncio de la Buena Noticia.

Puede encontrarse un ejemplo de esto en el testimonio de Felipe. El apóstol sale
más allá de Jerusalén hacia Samaria, y todavía llega más lejos, al compartir la
Buena Noticia de Jesús con un extranjero Etíope:

\blockquote[Hch 8, 29--39]{El Espíritu dijo a Felipe: \enquote{Acércate y pégate
    a la carroza}. Felipe se acercó corriendo, le oyó leer el profeta Isaías, y
  le preguntó: \enquote{¿Entiendes lo que estás leyendo?}. Contestó: \enquote{¿Y
    cómo voy a entenderlo si nadie me guía?}. E invitó a Felipe a subir y a
  sentarse con él. El pasaje de la Escritura que estaba leyendo era este:
  \emph{Como cordero fue llevado al matadero, como oveja muda ante el
    esquilador, así no abre su boca. En su humillación no se le hizo justicia.
    ¿Quién podrá contar su descendencia? Pues su vida ha sido arrancada de la
    tierra.} El eunuco preguntó a Felipe: \enquote{Por favor, ¿de quién dice
    esto el profeta?; ¿de él mismo o de otro?}. Felipe se puso a hablarle y,
  tomando pie de este pasaje, le anunció la Buena Nueva de Jesús. Continuando el
  camino, llegaron a un sitio donde había agua, y dijo el eunuco: \enquote{Mira,
    agua. ¿Qué dificultad hay en que me bautice?}. Mandó parar la carroza,
  bajaron los dos al agua, Felipe y el eunuco, y lo bautizó. Cuando salieron del
  agua, el Espíritu del Señor arrebató a Felipe. El eunuco no volvió a verlo, y
  siguió su camino lleno de alegría.}

Además de ser ejemplo de la actividad apostólica, este relato puede servir como
síntesis del modo en que la categoría del testimonio está presente en la
Escritura.

El testimonio comienza con la iniciativa de Dios mismo que impulsa tanto la
palabra profética del Antiguo Testamento como el anuncio apostólico del Nuevo
Testamento. Esta iniciativa de Dios tiende hacia el testimonio de la Palabra
definitiva del Padre que es Cristo resucitado. En aquellos que creen en el
testimonio de Dios se engendra alegría y vida nueva. En palabras de R.
Latourelle:

\blockquote[{\cite[1531]{latourelle2000testimonio}}]{En el trato de las tres
  personas divinas con los hombres existe un intercambio de testimonios que
  tiene la finalidad de proponer la revelación y de alimentar la fe. Son tres
  los que revelan o dan testimonio, y esos tres son más que uno. Cristo da
  testimonio del Padre, mientras que el Padre y el Espíritu dan testimonio del
  Hijo. Los apóstoles a su vez dan testimonio de lo que han visto y oído del
  verbo de la vida. Pero su testimonio no es la comunicación de una ideología,
  de un descubrimiento científico, de una técnica inédita, sino la proclamación
  de la salvación prometida y finalmente realizada.}

De este modo el anuncio del apóstol Felipe sirve aquí como un ejemplo específico
del testimonio, que ilustra sin embargo, una noción que
\blockquote[{\cite[109]{prades2015testimonio}}]{atraviesa toda la Escritura y se
  corresponde con la estructura misma de la revelación}. El testimonio está
presente a lo largo de la Escritura junto a otras categorías como pueden ser la
de \enquote{alianza}, \enquote{palabra}, \enquote{paternidad} o
\enquote{filiación}, como parte del
\blockquote[{\cite[1523]{latourelle2000testimonio}}]{grupo de analogías
  empleadas por la Escritura para introducir al hombre en las riquezas del
  misterio divino}.

Esta clave servirá para dar enfoque a un examen sobre la categoría del
testimonio en la Escritura. ¿Qué nos dice el Antiguo y el Nuevo Testamento de la
revelación como acto testimonial de Dios? Esta pregunta supone que la revelación
comparte los rasgos de la actividad humana que es el testimonio, sin embargo,
como Latourelle adiverte:
\blockquote[{\cite[1526]{latourelle2000testimonio}}]{globalmente se puede decir
  que el testimonio bíblico asume pero al mismo tiempo exalta hasta sublimarlos,
  los rasgos del testimonio humano}.

Cabe añadir una última consideración. La revelación de Dios entendida como acto
testimonial suyo tiene como expresión definitiva el misterio pascual de
Cristo.\footnote{\cite[128]{prades2015testimonio}: el misterio pascual al cual
  tiende toda la existencia terrena de Cristo, constituye el acto testimonial
  por excelencia de Dios.} Este misterio ocupa el lugar principal en el
testimonio bíblico:
\blockquote[{\cite[404]{ninot2009tf}}]{la Resurrección como \enquote{final} de
  la unicidad del acontecimiento de Jesucristo, encarnado, muerto y resucitado,
  subraya específicamente la definitividad de la existencia humana salvada por
  Dios en la carne de Jesús de Nazaret, ya que la autocomunicación de Dios ha
  alcanzado su palabra última en la Resurrección de Jesucristo, y por eso es
  prenda de la resurrección de todos los hombres.}
Como tal, parece justo tratar el testimonio que es el misterio pascual en su
propio apartado. Y será éste precisamente el punto de partida para esta
descripción de la categoría del testimonio en la Escritura.

\subsection{El testimonio en el misterio y anuncio pascual}

\enquote{Cristo ha resucitado}\footnote{Cf.~1Tes 4,15; 1Cor 15,12--20; Rom 6,4}
es la confesión que está en el núcleo del más primitivo anuncio del
evangelio.\autocite[Cf.][403]{ninot2009tf} Creer en esta noticia conlleva acoger
la manifestación más plena de la Revelación y la motivación más definitiva para
creer. En este sentido:
\blockquote[{\cite[405]{ninot2009tf}}]{La Resurrección de Jesús mirada desde la
  perspectiva de la teología fundamental presupone un estatuto epistemológico
  peculiar, puesto que es el punto culminante y objeto de la Revelación y, a su
  vez, es su acreditación suprema y máximo motivo de credibilidad, tal como
  recuerda el texto citado de Pablo \enquote{si Cristo no ha resucitado, nuestra
    predicación es vana y vana es nuestra fe} (1 Cor 15,14).}

Este misterio pascual no aparece como hecho desconectado del conjunto de la vida
y misión de Jesús, sino que hacia él tienden sus obras y palabras desde el
comienzo. Cristo pasó por el mundo haciendo el bien, como testimonio de la
bondad de Dios, y esta acción va orientada a ese punto culminante que es su
pasión, muerte y resurrección;
\blockquote[{\cite[127]{prades2015testimonio}}]{el testimonio que Jesús va
  ofreciendo durante su vida pública le va a reclamar una entrega definitiva a
  favor de los que lo han acogido y frente a la resistencia que ha generado en
  quienes le rechazan.}

A lo largo de este camino Jesús manifiesta su confianza en el Padre:
\blockquote[Jn 11,41b--42a]{Padre, te doy gracias porque me has escuchado; yo sé
  que tu me escuchas siempre}; esta relación queda afirmada plenamente ante la
pasión como confianza puesta en su voluntad: \blockquote[Lc 22,42]{Padre
  \textelp{} que no se haga mi voluntad, sino la tuya}. De este modo en el
misterio pascual queda atestiguada la plena unidad de Cristo con el Padre, en la
mayor confianza imaginable.\autocite[Cf.~][127]{prades2015testimonio}

A lo largo de su misión, Cristo dió testimonio del amor del Padre \blockquote[Jn
13,1]{habiendo amado a los suyos que estaban en el mundo\ldots}. En el misterio
pascual, donde \blockquote[ibíd.]{los amó hasta el extremo}, queda confirmado
definitivamente como testigo del Padre. Con su entrega ofrece el testimonio
pleno del amor salvador del Padre: \blockquote[Jn 3,16]{Porque tanto amó Dios al
  mundo, que entregó a su Unigénito, para que todo el que cree en él no perezca,
  sino que tenga vida eterna}.

A lo largo de su vida, Cristo también es testigo de la necesidad del camino
salvífico que es libre e irrevocable decisión trinitaria de redimir a los
hombres\autocite[Cf.~][128]{prades2015testimonio}: \blockquote[Lc 2, 49]{¿No
  sabíais que yo debía estar en las cosas de mi Padre?}; \blockquote[Mc 8,31]{El
  hijo del hombre tiene que padecer mucho, ser reprobado por los ancianos, sumos
  sacerdotes y escribas, ser ejecutado y resucitar a los tres días.} Este
testimonio de la voluntad divina es comprendido por los discípulos por la luz
del Resucitado; \blockquote[Lc 24,45--47a]{les abrió el entendimiento para
  comprender las Escrituras\ldots \enquote{así está escrito: el Mesías padecerá,
    resucitaráde entre los muertos al tercer día y en su nombre se proclamará la
    conversión}}.

La intencionalidad de este testimonio que Jesús ofrece a lo largo de su vida
hasta llegar al acto testimonial definitivo de Dios al mundo que es el misterio
pascual aparece con claridad en la respuesta de Cristo a Pilato antes de la
Pasión: \blockquote[Jn 18,37]{Yo para esto he nacido y para esto he venido al
  mundo: para dar testimonio de la verdad. Todo el que es de la verdad escucha
  mi voz.} En su vida pública y en su misión Cristo ha actuado como profeta que
anuncia la verdad; da a conocer al Padre, a quien nadie ha visto nunca, pero que
el Hijo sí conoce.\footnote{Cf.~ Jn 1,18; Ver también
  \cite[28]{ratzinger2007jdenaz}: En Jesús se cumple la promesa del nuevo
  profeta. En Él se ha hecho plenamente realidad lo que en Moisés era sólo
  imperfecto: Él vive ante el rostro de Dios no sólo como amigo, sino como Hijo;
  vive en la más íntima unidad con el Padre.} En el misterio pascual Jesús se
manifiesta como verdadero profeta, acreditado por el hecho mismo de la
Resurrección donde se ha realizado en él mismo lo que ha revelado y
prometido.\autocite[128]{prades2015testimonio}

La resurrección de Cristo no sólo acredita su propio testimonio, sino que
sostiene el testimonio apostólico. Si Cristo no ha resucitado sería vana
cualquier argumentación, sin embargo, Jesús es \enquote{el Viviente}, estuvo
muerto, pero vive por los siglos de los siglos.\footnote{Ap 1,17--18}

Los apóstoles son testigos de la vida de Cristo, de sus palabras y acciones,
muerte y resurrección. De tal modo, son testigos en continuidad con el
testimonio de Cristo. El testimonio apostólico es un anuncio de estos hechos que
ellos conocen y cuyo valor han reconocido por la fe. Así Pedro proclama estas
cosas el día de Pentecostés: \blockquote[Hch 2,32]{A este Jesús lo resucitó
  Dios, de lo cual todos nosotros somos testigos}. El apóstol es testigo en la
fe sobre un acontecimiento enraizado en la historia.\autocite[Cf.~][402;
406]{ninot2009tf}

Así mismo es presentado el testimonio de Pedro en casa de Cornelio donde el
centurión y todos lo que lo acompañaban esperaban reunidos para escuchar lo que
el Señor quisiera comunicarles por medio del apóstol. Pedro, comprendiendo que
la verdad de Dios no hace acepción de personas, narra los hechos que él bien
conoce:
\blockquote[Hch 10,37--41]{Vosotros conocéis lo que sucedió en toda Judea,
  comenzando por Galilea, después del bautismo que predicó Juan. Me refiero a
  Jesús de Nazaret, ungido por Dios con la fuerza del Espíritu Santo, que pasó
  haciendo el bien y curando a todos los oprimidos por el diablo, porque Dios
  estaba con él. Nosotros somos testigos de todo lo que hizo en la tierra de los
  judíos y en Jerusalén. A este lo mataron, colgándolo de un madero. Pero Dios
  lo resucitó al tercer día y le concedió la gracia de manifestarse, no a todo
  el pueblo, sino a los testigos designados por Dios: a nosotros, que hemos
  comido y bebido con él después de su resurrección de entre los muertos.}
Este testimonio de los hechos queda enlazado con un testimonio de fe sobre el
sentido profundo de lo que Pedro conoce, Jesús, a quien los apóstoles y el
pueblo vieron y escucharon, es ahora juez de vivos y muertos:
\blockquote[Hch 10,42-43]{Nos encargó predicar al pueblo, dando solemne
  testimonio de que Dios lo ha constituido juez de vivos y muertos. De él dan
  testimonio todos los profetas: que todos los que creen en él reciben, por su
  nombre, el perdón de los pecados.}

El apóstol entiende estos hechos y su alcance religioso y salvífico
interpretándolos en continuidad con la voluntad de Dios manifestada en su acción
en favor del pueblo judío a quién habló por medio de los profetas; voluntad
hecha manifiesta en definitva en \blockquote[Hch 2,22]{Jesús el Nazareno, varón
  acreditado por Dios ante vosotros con los milagros, prodigios y signos que
  Dios realizó por medio de él, como vosotros mismos sabéis}.

Este anuncio es experiencia del Resucitado que comió y bebió con ellos; él mismo
se apareció a los que él quiso dando testimonio de su resurrección.
\blockquote[{\cite[129]{prades2015testimonio}}]{Cristo glorificado manifiesta su
  verdad a los que él quiere y esta manifestación es simultaneamente testimonio
  de su identidad y testimonio de que él es la Vida (1Jn 5,11)}

El misterio divino que se manifiesta en la Pascua de Jesús no deja de expresarse
en el anuncio pascual realizado por los apóstoles. Ellos son testigos de un
hecho enraizado en la historia, que tiene un alcance religioso y salvífico y que
es interpretado desde la voluntad de Dios manifestada en los hechos y palabras
de Cristo. Sin las obras que Jesús realizó, el testimonio apostólico se
derrumba, no existe.\autocite[Cf.][1529]{latourelle2000testimonio} Sin la vida y
obra, muerte y resurrección de Jesús \blockquote[1Cor 15,15]{resultamos unos
  falsos testigos de Dios, porque hemos dado testimonio contra él, diciendo que
  ha resucitado a Cristo, a quien no ha resucitado}.

En Cristo, testigo acreditado por su Resurrección, encuentra su cumplimiento la
promesa hecha al pueblo de Israel: \blockquote[Dt 18,15;
{\cite[Cf.~][24ss]{ratzinger2007jdenaz}}]{El Señor, tu Dios, te suscitará de
  entre los tuyos, de entre tus hermanos, un profeta como yo. A él lo
  escucharéis}. Así como el misterio pascual y su anuncio no están desconectados
de la vida de Cristo, tampoco lo están de la acción salvadora de Dios en el AT.
Como veremos, el misterio divino se manifiesta a un pueblo que también está
llamado a dar testimonio, reconociendo desde la confianza en Dios el valor
salvífico de los sucesos de su historia.

\subsection{La acción testimonial de Dios en el Antiguo Testamento}

En el Antiguo Testamento encontramos ese \enquote{intercambio de testimonios}
que existe en el trato de las tres personas divinas con los
hombres.\autocite[Cf.][1531]{latourelle2000testimonio} También aquí la acción
testimonial divina se despliega de diversos modos. En la vida del pueblo de la
alianza YHWH da testimonio de sí a través de la creación, la ley y, de modo
eminente, en personas elegidas y enviadas por
él.\autocite[Cf.][114s]{prades2015testimonio} Esta manifestiación divina implica
como testigo al mismo pueblo, hacia quien ha sido dirigida la voz del Señor.

La literatura sapiencial recoge la profundización en la experiencia de Dios que
ha tenido el pueblo de Israel. En ella se describe el acceso posible al
conocimiento de Dios a partir de los bienes visibles o de sus obras:
\blockquote[Sab 13,1--5]{Son necios por naturaleza todos los hombres que han
  ignorado a Dios y no han sido capaces de conocer al que es a partir de los
  bienes visibles, ni de reconocer al artífice fijándose en sus obras, sino que
  tuvieron por dioses al fuego, al viento, al aire ligero, a la bóveda
  estrellada, al agua impetuosa y a los luceros del cielo, regidores del mundo.
  Si, cautivados por su hermosura, los creyeron dioses, sepan cuánto los
  aventaja su Señor, pues los creó el mismo autor de la belleza. Y si los
  asombró su poder y energía, calculen cuánto más poderoso es quien los hizo,
  pues por la grandeza y hermosura de las criaturas se descubre por analogía a
  su creador.}

El Dios que puede ser reconocido por analogía en el asombro y belleza de las
ciraturas es un Dios personal que concede sabiduría al piadoso:
\blockquote[Eclo 43,32--3]{Aún quedan misterios mucho más grandes: tan solo
  hemos visto algo de sus obras. Porque el Señor lo ha hecho todo y a los
  piadosos les ha dado la sabiduría.}
Esta sabiduria es justicia y raíz de inmortalidad:
\blockquote[Sab 15,1--3]{Pero tú, Dios nuestro, eres bueno y fiel, eres paciente
  y todo lo gobiernas con misericordia. Aunque pequemos, somos tuyos y
  reconocemos tu poder, pero no pecaremos, sabiendo que te pertenecemos.
  Conocerte a ti es justicia perfecta y reconocer tu poder es la raíz de la
  inmortalidad.}
En este sentido la misma creación es acto testimonial de Dios donde se comunica
su misterio y la vida que Él ofrece.

YHWH también aparece en el Antiguo Testamento como testigo de los mandamientos
contenidos en la Ley.\autocite[Cf.][115]{prades2015testimonio} Ésta queda
grabada en las \enquote{tablas del testimonio} y confiadas a Moisés:
\blockquote[Ex 31,18]{Cuando acabó de hablar con Moisés en la montaña del Sinaí,
  le dio las dos tablas del Testimonio, tablas de piedra escritas por el dedo de
  Dios.}
Este testimonio se enfrenta a un pueblo con el corazón extraviado:
\blockquote[Ex 32,19]{Al acercarse al campamento y ver el becerro y las danzas,
  Moisés, encendido en ira, tiró las tablas y las rompió al pie de la montaña.}
Sin embargo Dios no se detiene ante la dureza del pueblo. Las tablas del
testimonio son reconstruidas:
\blockquote[Ex 34,1.27]{El Señor dijo a Moisés: \enquote{Labra dos tablas de
    piedra como las primeras y yo escribiré en ellas las palabras que había en
    las primeras tablas que tú rompiste.} \textelp{} \enquote{Escribe estas
    palabras: de acuerdo con estas palabras concierto alianza contigo y con
    Israel}.}
Moisés, que conoció el nombre del Señor (Ex 3,13s), y habló con Él como un amigo
(Ex 33,11), aparece ante el pueblo como testigo del único Dios, y de su lealtad
con el pueblo. Pertenece a aquellos que el Señor elige como testigos suyos en
cada etapa de la historia del pueblo de Israel como testimonio suyo y de su
fidelidad.

Este es el modo eminente en que el AT describe el testimonio que Dios dirige al
pueblo. Los profetas y ungidos por YHWH son testigos del Señor y de su
compromiso con el pueblo. La vida totalmente comprometida del profeta expresa
tanto a Dios, absoluto que comunica, como su lealtad:
\blockquote[{\cite[116s]{prades2015testimonio}}]{es Dios quien da testimonio de
  sí mismo y de sus obras y designios a través de las personas elegidas, que se
  comprometen en su integridad como testigos de YHWH incluso hasta la muerte si
  el testimonio les lleva a ello. Por eso, la autoridad del testimonio no
  descansa en los testigos, sino en el mismo YHWH, que es quien los escoge y
  envía.}
En tanto que testigos, la acción de estos escogidos puede ser descrita según los
rasgos que tiene la actividad humana de dar testimonio, sin embargo la noción de
testigo que aplica a estos elegidos de Dios va más allá de la que encontraríamos
en el lenguaje ordinario. La vida del profeta queda comprometida con un
testimonio que no le pertenece, sino que
\blockquote[{\cite[118]{prades2015testimonio}}]{procede de una iniciativa
  absoluta, en cuanto a su origen y en cuanto a su contenido}
puesto que viene de Dios y es testimonio de sí. Aquí la categoría de testimonio
significa mas allá de su uso ordinario en la actividad humana y adquiere un
sentido religioso como dimensión totalmente
nueva\autocite[Cf.][118]{prades2015testimonio}.

El testimonio de YHWH que el profeta proclama con su actividad y el compromiso
de su vida implica al pueblo y le hace testigo:
\blockquote[{Is 43,8--12}]{Saca afuera a un pueblo que tiene ojos, pero está
  ciego, que tiene oídos, pero está sordo. Que todas las naciones se congreguen
  y todos los pueblos se reúnan. ¿Quién de entre ellos podría anunciar esto, o
  proclamar los hechos antiguos? Que presenten sus testigos para justificarse,
  que los oigan y digan: es verdad. Vosotros sois mis testigos --—oráculo del
  Señor--—, y también mi siervo, al que yo escogí, para que sepáis y creáis y
  comprendáis que yo soy Dios. Antes de mí no había sido formado ningún dios, ni
  lo habrá después. Yo, yo soy el Señor, fuera de mí no hay salvador. Yo lo
  anuncié y os salvé; lo anuncié y no hubo entre vosotros dios extranjero.
  Vosotros sois mis testigos --—oráculo del Señor--—: yo soy Dios.}
El siervo es testigo que el Señor ha escogido para que el pueblo sepa, crea y
comprenda que YHWH es el único Dios verdadero. Al compartir este saber de Dios
con el pueblo, éstos también están llamados a ser testigos. Ninguna otra nación
podría anunciar como ellos lo que YHWH ha hecho para proveer, liberar, salvar.

Así como el profeta, el pueblo es escogido y enviado por YHWH y por medio de él
el Señor da testimonio de sí mismo y se propone como quien da sentido y
consistencia a toda la realidad humana. Este testimonio tiene importancia social
puesto que está llamado a ser proclamado, y esta proclamación implica el
compromiso de los actos y la vida del testigo, es decir, del profeta y todo el
pueblo.\autocite[Cf.][1526s]{latourelle2000testimonio}

El testimonio de Dios a través de personas escogidas por Él en el AT queda
constituido por la narración de hechos que acontecen en la historia, estos
hechos son interpretados en su valor absoluto y carácter redentor, y son
confesados como actuación de Dios en la vida
humana.\autocite[Cf.][119]{prades2015testimonio} Esto vuelve a ponernos en
conexión con la figura de Cristo como profeta acreditado por su Resurrección y
los apóstoles como verdaderos testigos de un hecho enraizado en la historia,
confesado desde la fe e interpretado desde la acción de Dios en Jesús. Esta
sintonía anticipa lo que se verá a continuación sobre el testimonio en el Nuevo
Testamento. En él la acción testimonial de Dios se describe en continuidad con
la tradición veterotestamentaria y llegará a su manifestación plena en el
misterio pascual.


% El tercer apartado repasa algunos textos del magisterio reciente. El foco es
% el rol que juega la iglesia en la dinámica de la revelación como signo
% sacramental. Esto para unir estructura testimonial y motivo de credibilidad.
% LA CATEGORÍA DEL TESTIMONIO EN EL MAGISTERIO
\input{./ch1/magisterium.tex}

% El cuarto apartado problematiza la categoría del testimonio empleando desafíos
% propuestos por la filosofía moderna y contemporanea.
% LA CATEGORÍA DEL TESTIMONIO COMO PROBLEMA
\section{La Categoría del Testimonio como Problema}

Hasta aquí se ha querido ofrecer una descripción general del modo en que se
puede encontrar la categoría de testimonio empleada en la Escritura, el
Magisterio y la vida de la Iglesia. En términos generales, al hablar del
testimonio en estos contextos en donde se le encuentra como \enquote{\emph{cosa
    familiar y conocida}}, se ha querido destacar el uso que se le da a esta
categoría como analogía empleada para hablar de la acción divina en la
Revelación.

Ahora nos permitimos tratar al testimonio como algo que hay que esclarecer, algo
que se encuentra presente en la actividad humana y sobre lo que se plantean
preguntas, de modo que hay que \enquote{\emph{traer a la mente}} una explicación
adecuada.

Se pueden destacar varios objetivos al preguntarse sobre el testimonio. Desde el
punto de vista teológico el hecho mismo de que esta categoría sea empleada en la
Escritura sirve ya como justificación para estudiar mejor el fenómeno del
testimonio, como dice Latourelle:
\blockquote[{\cite[1523]{latourelle2000testimonio}}]{Si la revelación misma se
  apoya en la experiencia humana del testimonio para expresar una de las
  relaciones fundamentales que unen al hombre con Dios, la reflexión teológica
  se encuentra entonces autorizada a explorar los datos de esta experiencia.}
Sin embargo el interés por la categoría del testimonio en la investigación
teológica más reciente claramente está motivado por la presencia de esta noción
en las reflexiones del Concilio Vaticano~II y el magisterio post-conciliar:
\blockquote[{\cite[81]{prades2015testimonio}}]{La teología ha ido revalorizando
  el testimonio, que había quedado relegado a un segundo plano en otros momentos
  de la historia de la teología, hasta alcanzar una difusión realmente masiva en
  los años posteriores al Concilio.}
El testimonio es un tema privilegiado en el Concilio y se le encuentra presente
como \enquote{\emph{leitmotiv}} en las constituciones y
decretos.\footnote{Cf.~\cite[1523]{latourelle2000testimonio}} Vaticano~II
potencia así este termino que ya se encontraba presente en las reflexiones del
Vaticano~I:
\blockquote[{\cite[572]{ninot2009tf}}]{Desde hace aproximadamente un siglo, la
  categoría testimonio se ha introducido de forma progresiva en el vocabulario
  eclesial. La concentración y personalización operada por el Concilio
  Vaticano~II conlleva la potenciación de un término nuevo como es el
  testimonio. \textelp{} lo que el Vaticano I pretendía al tratar el signo de la
  Iglesia, que también era visto como ``un testimonio'' [DH 3013], se encuentra
  en la categoría testimonio, que con el Vaticano~II irrumpe masivamente.}

Tras el entusiasmo inicial por el testimonio en ámbitos pastorales y teológicos
se ha ido advirtiendo en algunos textos magisteriales y teológicos el aviso de
cierto peligro de ambigüedad o abuso en el uso de esta
categoría:\footnote{Cf.~\cite[83]{prades2015testimonio}}
\blockquote[{\cite[84]{prades2015testimonio}}]{se ha hecho notar que el
  testimonio podía verse limitado a la manifestación de una especie de seriedad
  con lo humano, ya fuera en términos de reivindicación social o de autenticidad
  existencial, con la inevitable prevalencia del sujeto ---individual o
  colectivo--- pero sin llegar a remitir a la verdad de Cristo. \textelp{}

  Se trataría del riesgo de una reducción experiencialista del testimonio, donde
  lo más importante sería su carácter social-existencial y no tanto la efectiva
  verdad teologal transmitida. Se ha criticado consecuentemente la reducción del
  testimonio ---y de la misma teología--- a puro relato autobiográfico.

  Si se recupera la profundidad implicada en el testimonio se contribuirá a
  salir del subjetivismo ---antiguo y moderno---, con su carga correspondiente
  de individualismo, tan contrario a la verdadera naturaleza social del hombre y
  al carácter a la vez personal y comunitario de la salvación cristiana.}
Atendiendo a estos datos, una investigación teológica sobre el testimonio tiene
el interés de profundizar en una categoría valiosa en el ámbito teológico y
pastoral de modo que sea empleada y formulada adecuadamente.

Este interés interno de la discusión teológica está enmarcado en un contexto
histórico actual del que también se derivan motivaciones para una valoración de
la categoría del testimonio. Dos rasgos que cabe destacar de este momento
presente son:
\blockquote[{\cite[75]{prades2015testimonio}} Un análisis detallado del contexto
presente se encuentra en {\cite[3--77]{prades2015testimonio}}]{la tensión entre
  multiculturalismo y globalización como indicio de la dificultad para combinar
  positivamente el carácter individual y comunitario de la vida humana, y la
  discusión sobre el papel público de la religión, donde la tesis dominante de
  la <<edad secular>> se ve contrapesada por la irrupción de un nuevo paradigma
  que se denomina <<postsecular>>.}

En este contexto, el preguntarse sobre el testimonio tiene como objetivo un
adecuado modo de entender la presencia pública de los cristianos en las
sociedades plurales de occidente donde resulta problemática la comprensión del
ser humano en su relación con Dios a través de la
realidad.\footnote{Cf.~\cite[75]{prades2015testimonio}} Es importante que en
este contexto la cuestión de la presencia del cristianismo en la sociedad no
tiene como solución adecuada una
\enquote*{<<autorrelativización>>}\footnote{Cf.~\cite[75;\,40--44]{prades2015testimonio}}
de sí mismo; igualmente:
\blockquote[{\cite[75; Cf.~33--40]{prades2015testimonio}}]{no podemos presuponer
  el reconocimiento de su carácter universal por parte de los interlocutores ni
  podemos pretender alcanzarlo por una mera comparación de argumentos racionales
  que desnaturalice el carácter libre y singular de la revelación personal de
  Dios en Jesucristo.}
El análisis de la categoría del testimonio viene a responder a la necesidad de
recuperar una concepción de la razón y de la verdad más rica y más amplia;
\blockquote[{\cite[76]{prades2015testimonio}}]{Es imprescindible repensar el
  nexo entre razón, afectos y libertad en la relación del hombre con lo real. Si
  se recupera esa visión amplia e integral de razón y de realidad se puede
  entonces mostrar convincentemente la credibilidad de la fe como asentimiento a
  una revelación personal en la historia.}

Teniendo en cuenta estas motivaciones desarrollaremos los elementos que componen
las cuestiones problemáticas del testimonio que serán estudiadas en el
pensamiento de Anscombe. Para ello recuperamos la pregunta formulada al inicio
de éste capítulo, que si ampliamos un poco queda: \enquote{¿qué es conocer una
  verdad para la vida por el testimonio de la revelación divina?}.
Desde esta pregunta se pueden distinguir ya dos cuestiones: ¿qué implica conocer
una verdad por medio del testimonio? y ¿qué valor puede tener para la vida un
testimonio de la revelación divina? ---o incluso--- ¿qué puede ser valorado como
un testimonio de la revelación divina? Para examinar mejor estas cuestiones
acudiremos a reflexiones filosóficas de la época moderna y contemporánea.

Aún cuando el testimonio ocupa un lugar vital en nuestro contacto con el mundo,
no siempre ha gozado del interés de la investigación filosófica. Más
recientemente, sin embargo, su importancia ha sido mejor apreciada y así lo
refleja la variedad literatura que puede encontrarse en la filosofía
contemporánea.\footnote{Cf.~\cite[1]{lackeysosa2006eptest}: Despite the vital
  role that testimony occupies in our epistemic lives, traditional
  epistemological theories focused primarily on other sources, such as sense
  perception, memory, and reason, with relatively little attention devoted
  specifically to testimony. In recent years, however, the epistemic
  significance of testimony has been more fully appreciated, and the current
  literature has benefited from the publication of a considerable amount of
  interesting and innovative work in this area.} Uno de estos estudios es el que
se encuentra en la obra \emph{Testimony} de C.\,A.\,J.~Coady y su informe sobre
los inicios de sus discusiones sobre el testimonio confirman el creciente
aprecio que ha ganado el estudio de esta categoría:
\blockquote[{\cite[vii]{coady1992test}}: When I began reading papers on the
subject, my audiences mostly reacted with incomprehension, or the sort of
disbelief evoked by denials of the merest common sense. Gradually, the climate
of thought has changed and there is now more sympathy for the view that
testimony is a prominent and underexplored epistemological landscape, although
what sort of feature it is and how largely it looms are still naturally matters
for disagreement.]{Cuando comencé a ofrecer lecciones sobre este tema, las
  audiencias mayormente reaccionaban con incomprensión, o el tipo de
  incredulidad evocada por rechazos del más básico sentido común. Gradualmente,
  el clima del pensamiento ha cambiado y ahora hay más simpatía para el punto de
  vista de que el testimonio es un campo epistemológico prominente y poco
  explorado, aunque en qué tipo de rasgo consiste y con cuánta magnitud se
  impone son todavía cuestiones en debate.}
Otro dato ofrecido por Coady sirve para orientar el planteamiento de la cuestión
sobre el testimonio:
Nos interesa otro dato ofrecido por Coady porque nos da una pista para nuestro
propio estudio
\blockquote[{\cite[vii]{coady1992test}}: I first began thinking about the
epistemological status of testimony in the 1960s when writing a thesis at Oxford
on issues in the theory of perception. \textelp{} I recall being intrigued by
some remarks of Elizabeth Anscombe on the topic during her lectures on the
empiricists \textelp{}]{Empecé por primera vez a pensar sobre la situación
  epistemológica del testimonio en los años 60 cuando estuve escribiendo una
  tesis en Oxford sobre problemas en la teoría de la percepción. \textelp{}
  Recuerdo haber quedado intrigado por algunas afirmaciones de Elizabeth
  Anscombe sobre el tema durante sus lecciones sobre los empiristas \textelp{}}

Esta época mas prolija en discusiones no es, sin embargo, el origen de algunas
posturas propuestas en torno al testimonio; éste lo encontramos más bien en la
época moderna. Recurriremos, por tanto, a algunas aportaciones y desafíos
ofrecidos por la filosofía moderna y contemporánea para expandir nuestra
anterior pregunta y formular las cuestiones principales que servirán luego para
navegar en el pensamiento de Elizabeth Anscombe.

\subsection{¿Cuál es el valor epistemológico del testimonio?}
Corresponde a la epistemología la tarea de estudiar la naturaleza del conocer y
su justificación. ¿Cuáles son los componentes del conocimiento? ¿sus fuentes o
condiciones? ¿sus límites?\footnote{Cf.~\cite[3]{moser2002ep}: Epistemology,
  characterized broadly, is an account of knowledge. Within the discipline of
  philosophy, epistemology is the study of the nature of knowledge and
  justification: in particular, the study of (a) the defining components, (b)
  the substantive conditions or sources, and (c) the limits of knowledge and
  justification.} La pregunta sobre el valor epistemológico del testimonio
consiste en juzgar el lugar que éste ocupa en una descripción del conocimiento;
¿qué se puede decir del testimonio como estrategia para adquirir la verdad y
evitar el error?\footnote{Cf.~\cite[14]{moser2002ep}: Any standard or strategy
  worthy of the title ``epistemic'' must have as its fundamental goal the
  acquisition of truth and the avoidance of error.}

Podemos recurrir al análisis tradicional empleado para hablar del conocimiento
proposicional y entenderlo como \enquote{creencia verdadera
  justificada}.\footnote{\cite[4]{moser2002ep}: Ever since Plato's Theaetetus,
  epipstemologists have tried to identify the essential, defining components of
  propositional knowledge. These components will yield an analysis of
  propositional knowledge. An influential traditional view, inspired by Plato
  and Kant among others, is that propositional knowledge has three individually
  necessary and jointly sufficient components: justification, truth, and belief.
  On this view, propositional knowledge is, by definition, justified true
  belief. This tripartite definition has come to be called ``the standard
  analysis''.} Según esta composición tripartita la pregunta sobre el valor
epistemológico del testimonio se puede plantear diciendo: \enquote{dada una
  comunicación que cualifique como testimonio y que sea al caso que la creencia
  formada desde esta comunicación está basada enteramente en el testimonio
  recibido,\footnote{Cf.~\cite[4]{lackeysosa2006eptest}: Even if an expression
    of thought qualifies as testimony and the resulting belief formed is
    entirely testimonially based for the hearer, however, there is the further
    question of how precisely such a belief successfully counts as justified
    belief or an instance of knowledge.} ¿cómo adquirimos efectivamente una
  creencia verdadera justificada sobre la base de lo que alguien nos ha
  dicho?},\footnote{Cf.~\cite[2]{lackeysosa2006eptest}: how we successfully
  acquire justified belief or knowledge on the basis of what other people tell
  us. This, rather than what testimony is, is often taken to be the issue of
  central import from an epistemological point of view.} es decir,
\enquote{¿cómo, precisamente, una creencia como esta puede ser contada
  satisfactoriamente como creencia justificada o una instancia de conocimiento?}
\footnote{Cf.~\cite[4]{lackeysosa2006eptest}: how precisely such a belief
  successfully counts as justified belief or an instance of knowledge}

Las respuestas a esta pregunta central sobre la epistemología del testimonio se
han situado en dos posturas que se han denominado \enquote{reduccionista} y
\enquote{no-reduccionista}.\footnote{Cf.~\cite[4]{lackeysosa2006eptest}: Indeed,
  this is the question at the center of the epistemology of testimony, and the
  current philosophical literature contains two central options for answering
  it: non-reductionism and reductionism.} Las raíces históricas de la primera
postura se le suelen atribuir a Hume y de la segunda a Thomas Reid.

De acuerdo a los no-reduccionistas el testimonio es simplemente una fuente de
justificación como lo sería la percepción de los sentidos, la memoria o la
inferencia. Según esto, siempre que no haya una justificación contraria
suficientemente relevante, el que escucha tiene justificación verdadera para
creer las proposiciones del testimonio del que
habla.\footnote{Cf.~\cite[4]{lackeysosa2006eptest}: According to
  non-reductionists ---whose historical roots are standardly traced back to
  Reid--- testimony is just as basic a source of justification (warrant,
  entitlement, knowledge, etc.) as sense perception, memory, inference, and the
  like. Accordingly, so long as there are no relevant defeaters, hearers can
  justifiedly accept the assertions of speakers merely on the basis of a
  speaker's testimony.}

Hume, por su parte, \blockquote[{\cite[79]{coady1992test}}: is one of the few
philosophers who has offered anything like a sustained account of testimony and
if any view has a claim to the title of `the received view' it is his]{es uno de
  los pocos filósofos que ha ofrecido algo así como una descripción sostenida
  acerca del testimonio y si alguna perspectiva puede reclamar el título de `el
  punto de vista adoptado' es la suya}. En la base de su valoración del
testimonio está su estima de la relación de causa y efecto como fundamento de
cualquier razonamiento concerniente a cuestiones de hecho.

Distinto a las relaciones de ideas, la evidencia de la veracidad de una cuestión
de hecho no se demuestra a priori, sino que ha de ser descubierta en la
experiencia. Ahora bien, ¿cuál es la naturaleza de la evidencia de aquellas
cuestiones de hecho que están más allá de la percepción de nuestros sentidos o
de las impresiones de nuestra memoria?\footnote{Cf.~\cite[\S4,1.
  15]{hume1777enquiry}: Matters of fact, which are the second objects of human
  reason, are not ascertained in the same manner; nor is our evidence of their
  truth, however great, of a like nature with the foregoing (relations of ideas)
  \textelp{} The contrary of every matter of fact is still possible \textelp{}
  We should, in vain, therefore attempt to demonstrate its falsehood. Were it
  demonstratively false, it would imply a contradiction, and could never be
  distinctly conceived by the mind \textelp{} what is the nature of that
  evidence which assures us of any real existence and matter of fact, beyond the
  present testimony of our senses, or the records of our memory.} Nuestros
razonamientos relacionados con algún hecho se componen de inferencias realizadas
a partir del conocimiento que tenemos de que a una causa se sigue su
efecto.\footnote{Cf.~\cite[\S4,1. 16]{hume1777enquiry}: All our reasonings
  concerning fact are of the same nature; and here it is constantly supposed
  that there is a connection between the present fact and that which is inferred
  from it. Were there nothing to bind them together, the inference would be
  entirely precarious.} Este conocimiento de la relación causa y efecto, a su
vez, no consiste en un razonamiento a priori, \blockquote[{\cite[\S4,1.
  17]{hume1777enquiry}}: that the knowledge of this relation is not, in any
instance, attained by reasonings a priori, but arises entirely from experience,
when we find that any particular objects are constantly conjoined with each
other.]{sino que surge completamente de la experiencia, cuando descubrimos que
  cualesquiera objetos particulares están constantemente unidos entre sí}. Es
así que \blockquote[{\cite[\S4,1. 16]{hume1777enquiry}}: By means of that
relation alone, we can go beyond the evidence of our memory and senses.]{tan
  solo por medio de esta relación, podemos ir más allá de nuestra memoria y
  sentidos}.

Esta misma línea de razonamiento es la que se sigue en la descripción acerca del
testimonio y su valor: \blockquote[{\cite[\S10,1. 74]{hume1777enquiry}}: there
is no species of reasoning more common, more useful, and even necessary to human
life, than that which is derived from the testimony of men, and the reports of
eye witnesses and spectators. This species of reasoning, perhaps, one may deny
to be founded on the relation of cause and effect. I shall not dispute about a
word. It will be sufficient to observe, that our assurance in any argument of
this kind, is derived from no other principle than our observation of the
veracity of human testimony, and of the usual conformity of facts to the reports
of witnesses. It being a general maxim, that no objects have any discoverable
connection together, and that all the inferences which we can draw from one to
another, are founded merely on our experience of their constant and regular
conjunction; it is evident, that we ought not to make an exception to this maxim
in favour of human testimony, whose connection with any event seems, in itself,
as little necessary as any other. Were not the memory tenacious to a certain
degree; had not men commonly an inclination to truth and a principle of probity;
were they not sensible to shame, when detected in a falsehood; were not these, I
say, discovered by experience to be qualities inherent in human nature, we
should never repose the least confidence in human testimony. A man delirious, or
noted for falsehood and villany, has no manner of authority with us.]{no hay un
  tipo de razonamiento más común, más útil, e incluso necesario para la vida
  humana, que aquel que se deriva del testimonio de los hombres, y los informes
  de testigos oculares y espectadores. Quizá uno pueda negar que esta clase de
  razonamiento esté fundada en la relación de causa y efecto. No discutiré por
  una palabra. Será suficiente observar, que nuestra confianza en un argumento
  de este tipo, no se deriva de otro principio que el de nuestra observación de
  la veracidad del testimonio humano, y la correspondencia habitual de los
  hechos con los informes de los testigos. Siendo esto una máxima general, que
  ningún caso de objetos tienen alguna conexión entre sí que pueda ser
  descubierta, y que todas las inferencias que podamos sacar de uno por el otro,
  son fundadas meramente en nuestra experiencia de su constante y regular
  conjunción; es evidente, que no deberíamos hacer una excepción a esta máxima
  en favor del testimonio humano, cuya conexión con cualquier evento parece, en
  sí misma, tan poco necesaria como cualquier otra. Si la memoria no fuera tenaz
  en cierto grado; si no tuvieran los hombres comúnmente una inclinación a la
  verdad y un principio de honradez; si no fueran sensibles a la vergüenza,
  cuando son descubiertos en la mentira; digo yo, si éstas no fueran cualidades
  que la experiencia descubre como inherentes a la naturaleza humana, jamas
  tendríamos la menor confianza en el testimonio humano. Un hombre delirante, o
  notorio por mentiroso o villano, no tiene ninguna clase de autoridad entre
  nosotros.}

Así como nuestra habitual experiencia de la relación de causa y efecto nos
permite hacer inferencias acerca de las cuestiones de hecho que están más allá
de nuestros sentidos, la conformidad que usualmente experimentamos entre los
hechos y el informe que un testigo nos da de ellos nos permite inferir su
veracidad. Según el análisis ofrecido por C.\,A.\,J.~Coady, la teoría de Hume:
\blockquote[{\cite[79]{coady1992test}}: constitutes a reduction of testimony as
a form of evidence or support to the status of a species (one might almost say,
a mutation) of inductive inference. And, again, in so far as inductive inference
is reduced by Hume to a species of observation and consequences attendant upon
observations, then in a like fashion testimony meets the same fate.]{constituye
  una reducción del testimonio como una forma de evidencia o fundamento al
  estatuto de una especie (uno podría casi decir, una mutación) de inferencia
  inductiva. Y, una vez más, en tanto que la inferencia inductiva queda reducida
  por Hume a una especie de observación y consecuencias relacionadas con las
  observaciones, en consecuencia igualmente el testimonio corre la misma suerte}
La valoración epistemológica del testimonio y la perspectiva ofrecida por Hume
nos deja así con un primer desafío:
\blockquote[{\cite[294]{prades2015testimonio}}]{en la vida social cabe aceptar
  un conocimiento por testimonio a condición de que su grado de certeza se
  limite a la probabilidad, y a condición de que pueda ser siempre reconducido a
  una verificación por conocimiento directo}.

Estas consideraciones añaden algunos elementos a nuestra cuestión inicial.
Conocer una verdad para la vida desde el testimonio implica que pueda obtenerse
una creencia verdadera justificada basada en lo que una persona ha comunicado.
La visión de Hume es que la evidencia que puede ofrecer un testimonio para
justificar una creencia no es mayor que la probabilidad y esta evidencia está
basada en la inferencia que nos permite la habitual experiencia de que el
testimonio comunicado y la verdad de los hechos suelen ir unidos. Más adelante
veremos qué tiene que decir Anscombe ante este desafío. Todavía podemos plantear
una segunda cuestión; esta vez relacionada con la segunda parte de nuestra
pregunta original.

\subsection{¿Tiene fuerza un testimonio histórico del Absoluto?}
El contexto de la reflexión de Hume sobre el testimonio es precisamente el de la
  creencia en los milagros. La preocupación de Hume es que el \enquote{hombre
  sabio} pueda verificar sus creencias de modo que no sea víctima de
\enquote{engaños supersticiosos}. Para esto, estima, que ha encontrado un
argumento que servirá para distinguir superstición de
verdad.\footnote{\cite[\S10,1. 73]{hume1777enquiry}: I flatter myself, that I
  have discovered an argument of a like nature, which, if just, will, with the
  wise and learned, be an everlasting check to all kinds of superstitious
  delusion, and consequently will be useful as long as the world endures.} Dice:

\blockquote[{\cite[\S10,1. 73]{hume1777enquiry}}: in our reasonings concerning
matter of fact, there are all imaginable degrees of assurance, from the highest
certainty to the lowest species of moral evidence. A wise man, therefore,
proportions his belief to the evidence]{en nuestros razonamientos concernientes
  a cuestiones de hecho, se dan todos los grados imaginables de seguridad, desde la
  certeza más alta hasta las especies más bajas de evidencia moral. Un hombre
  sabio, por tanto, adecua su creencia a la evidencia}.

Entonces sugiere un criterio que permite ajustar las creencias
a la evidencia:

\blockquote[{\cite[\S10,1. 77]{hume1777enquiry}}: `That no testimony is
sufficient to establish a miracle, unless the testimony be of such a kind, that
its falsehood would be more miraculous than the fact which it endeavours to
establish; and, even in that case, there is a mutual destruction of arguments;
and the superior only gives us an assurance suitable to that degree of force
which remains after deducting the inferior.']{`Que ningún testimonio es
  suficiente para establecer un milagro, excepto si el testimonio es de tal
  tipo, que su falsedad sea más milagrosa que el hecho que se esfuerza por
  establecer; e, incluso en este caso, hay una mutua destrucción de argumentos;
  y el superior sólo nos da certeza apropiada al grado de fuerza que permanece
  después de restar el inferior.'}

Esto tiene como consecuencia que lo razonable sea abandonar la razonabilidad de
las verdades cristianas, comprendiendo que solo pueden ser contempladas desde la
fe. Empleando su criterio ofrece una valoración de la revelación de la escritura
como sigue:

\blockquote[{\cite[\S10,1. 89]{hume1777enquiry}}: I am the better pleased with
the method of reasoning here delivered, as I think it may serve to confound
those dangerous friends, or disguised enemies to the Christian religion, who
have undertaken to defend it by the principles of human reason. Our most holy
religion is founded on faith, not on reason; and it is a sure method of exposing
it, to put it to such a trial as it is by no means fitted to endure. To make
this more evident, let us examine those miracles related in Scripture; and, not
to lose ourselves in too wide a field, let us confine ourselves to such as we
find in the Pentateuch, which we shall examine according to the principles of
these pretended Christians, not as the word or testimony of God himself, but as
the production of a mere human writer and historian. Here then we are first to
consider a book, presented to us by a barbarous and ignorant people, written in
an age when they were still more barbarous, and in all probability long after
the facts which it relates, corroborated by no concurring testimony, and
resembling those fabulous accounts which every nation gives of its origin. Upon
reading this book, we find it full of prodigies and miracles. It gives an
account of a state of the world and of human nature entirely different from the
present: of our fall from that state; of the age of man extended to near a
thousand years; of the destruction of the world by a deluge; of the arbitrary
choice of one people, as the favourites of heaven, and that people the
countrymen of the author; of their deliverance from
bondage by prodigies the most astonishing imaginable.\\
I desire any one to lay his hand upon his heart, and, after a serious
consideration, declare, whether he thinks that the falsehood of such a book,
supported by such a testimony, would be more extraordinary and miraculous than
all the miracles it relates; which is, however, necessary to make it be received
according to the measures of probability above established.]{Estoy más
  satisfecho con el método de razonar aquí expuesto, pues pienso que puede
  servir para confundir esos amigos peligrosos, o los enemigos disfrazados de la
  religión Cristiana, que se han propuesto defenderla con los principios de la
  razón humana. Nuestra más sagrada religión se funda en la fe, no en la razón;
  y es un modo seguro de exponerla, el someterla a una prueba que de ningún modo
  está capacitada para soportar. Para hacer esto más evidente examinemos los
  milagros relatados en la escritura y, para no perdernos en un campo demasiado
  amplio, limitémonos a los que encontramos en el Pentatéuco, que examinaremos
  de acuerdo con los principios de aquellos supuestos Cristianos, no como la
  palabra o testimonio de Dios mismo, sino como la producción de un mero
  escritor e historiador humano. Aquí entonces hemos de considerar primero un
  libro que un pueblo bárbaro e ignorante nos presenta, escrito en una edad aún
  más bárbara y, con toda probabilidad, mucho después de los hechos que relata,
  no corroborado por testimonio concurrente alguno, y asemejándose a las
  narraciones fabulosas que toda nación da de su origen. Al leer este libro, lo
  encontramos lleno de prodigios y milagros. Ofrece un relato del estado del
  mundo y de la naturaleza humana totalmente distinto al presente: de nuestra
  pérdida de aquella condición; de la edad del hombre que alcanza a casi mil
  años; de la destrucción del mundo por un diluvio; de la elección arbitraria de
  un pueblo como el favorito del cielo y que dicho pueblo lo componen los
  compatriotas del autor; de su liberación de la servidumbre por los prodigios
  más asombrosos que se puede uno imaginar.

  Invito a cualquiera a que ponga su mano sobre el corazón, y, tras seria
  consideración, declare, si piensa que la falsedad de tal libro, apoyado por
  tal testimonio, sería más extraordinaria y milagrosa que todos los milagros
  que narra; lo cual, sin embargo, es necesario para que sea aceptado de acuerdo
  con las medidas de probabilidad arriba establecidas.}

¿Se puede afirmar que sería más \enquote{milagrosa} la falsedad de los milagros
que atestigua la escritura? La posibilidad de recibir este testimonio como
evidencia de alguna verdad descansaría sobre esta condición y una persona
razonable debería medir la probabilidad de veracidad de estos relatos teniendo
en cuenta que el estado de las cosas que describe es distinto al que
experimentamos en el presente.

En una línea similar de pensamiento encontramos las reflexiones de
G.\,E.~Lessing. Dos cuestiones expresadas en \emph{On the proof of the spirit
  and of power} merecen ser destacadas:

\blockquote[The problem is that reports of fulfilled prophecies are not
fullfiled prophecies; that reports of miracles are not miracles. These, the
prophecies fulfilled before my eyes, the miracles that occur before my eyes, are
immediate in their effect. But those---the reports of fulfilled prophecies and
miracles, have to work through a medium which takes away all their force]{El
  problema es que las noticias de profecías cumplidas no son profecías
  cumplidas; que las noticias de milagros no son milagros. Estas, las profecías
  cumplidas ante mis ojos, los milagros que ocurren ante mis ojos, son
  inmediatos en su efecto. Pero esas---las noticias de profecías y milagros,
  tienen que pasar trabajosamente por un medio que les arrebata toda su fuerza}

Lo que debería tener la fuerza para justificar la credibilidad queda debilitado
por su medio de transmisión, entonces

\blockquote[the problem is that this proof of the spirit and of power no longer
has any spirit or power, but has sunk to the level of human testimonies of
spirit and power]{el problema es que esta prueba en espíritu y fuerza ya no
  tiene ningún espíritu ni fuerza, sino que se ha hundido al nivel de
  testimonios humanos de espíritu y de fuerza}.

Tal como lo plantea Lessing y teniendo en cuenta el criterio propuesto por Hume,
el testimonio, en tanto que dinamismo humano, no tiene fuerza suficiente para
justificar razonablemente creencias sobre Dios como verdadero conocimiento.

Esta objeción nos lleva a la siguiente:

\blockquote[the reports which we have of these prophecies and miracles are as
reliable as historical truths can ever be \textelp{} But if they are as reliable
as this, why are they treated as if they were infinitely more reliable?
\textelp{} If no historical truth can be demonstrated, then nothing can be
demonstrated by means of historical truths. That is: \emph{accidental truths of
  history can never become proof of necessary truths of reason.}]{las noticias
  que tenemos de estas profecías y milagros son tan fiables como lo puedan
  llegar a ser las verdades históricas \textelp{} Pero si son tan fiables como
  éstas, ¿por qué son tratadas como si fueran infinitamente más fiables?
  \textelp{} Si ninguna verdad histórica puede ser demostrada, entonces nada
  puede ser demostrado por medio de verdades históricas. Esto es: \emph{verdades
    contingentes de la historia nunca pueden llegar a ser demostración de
    verdades de razón necesarias}}

El punto que Lessing señala es infranqueable para su propio intento de
comprometerse con la verdad que la creencia cristiana pretende comunicar. La
singularidad de la persona y obra de Jesús como manifestación de la realidad de
Dios pierde para él toda su fuerza, puesto que no puede estimar estas verdades
históricas como fundamento para una verdad necesaria como lo es la verdad de
Dios.

Esto nos deja con una segunda problemática:

\blockquote[{\cite[294]{prades2015testimonio}}]{no se puede tener conocimiento
  directo de milagros y profecías \textelp{} no se puede aceptar una
  comunicación divina que no sea inmediatamente dirigida al individuo}.

Este desafío viene a poner en cuestión que un hecho histórico de la vida
personal o colectiva pueda ser estimado como testimonio del absoluto. La
revelación de Dios por medio de testigos no es un fenómeno que tenga
justificación razonable para su veracidad, y por tanto sólo puede ser acogida
por una fe desconectada de la razón.

\subsection{¿Tiene carácter veritativo el lenguaje teológico?}
Una problemática adicional está representada en la crítica al lenguaje religioso
planteada por el Círculo de Viena. A\,J.~Ayer lo expresa como sigue:
\blockquote[{\cite[155]{dominguez2009at}}]{Si la existencia de tal dios fuese
  probable, la proposición de que existiera sería una hipótesis empírica. Y, en
  ese caso, sería posible deducir de ella, y de otras hipótesis científicas,
  ciertas proposiciones experienciales que no fuesen deducibles de esas otras
  hipótesis solas. Pero, en realidadm esto no es posible. [\ldots] Porque decir
  que ``Dios existe'' es realizar una expresión metafísica que no pude ser ni
  verdadera ni falsa. Y, según el mismo criterio, ninguna oración que pretenda
  describir la naturaleza de un Dios trascendente puede poseer ninguna
  significación literal.}

La intención del Círculo es la unificación de la ciencia y como herramienta para
este trabajo, empleó el análisis del lenguaje. Un análisis de la expresión
linguística empleada en el conocimiento científico permite distinguir las
afirmaciones que pueden tener valor veritativo de las que no contienen esta
posibilidad y, por tanto, no son lenguaje significativo. A. Flew, por ejemplo,
considera que la Teología no es un lenguaje significativo poruqe no es posible
su falsabilidad. De este modo:
\blockquote[{\cite[155]{dominguez2009at}}]{La crítica del Círculo de Viena no se
  suma al ``Dios ha muerto'' de Nietzsche, sino que va aún más allá: lo que ha
  muerto es la misma palabra: ``Dios''. Nos encontramos ante lo que podemos
  considerar una nueva y refinada especie de ateísmo: el ateísmo semántico. Esta
  forma de ateísmo se sustenta en un equivocismo hermenéutico. No cabe comparar,
  arguyen los equivocistas, los nombres de supuestas realidades trascendentes
  con los de las realidades empíricas.}


%\section{Naturaleza de la pregunta sobre el testimonio}
  Es una experiencia familiar en nuestras comunidades reunirnos en torno a la
  Sagrada Escritura y compartir la Palabra buscando en ella luz para nuestro
  presente. Una escena evangélica en torno a la cual muchos se han reunido a
  escuchar al Señor es la narración de Mateo del comienzo de la misión pública de
  Jesús y la llamada de los primeros discípulos:

  \citalitlar{Al enterarse Jesús de que habían arrestado a Juan se retiró a
    Galilea. Dejando Nazaret se estableció en Cafarnaún, junto al mar, en el
    territorio de Zabulón y Neftalí, para que se cumpliera lo dicho por medio del
    profeta Isaías:\\
    <<Tierra de Zabulón y tierra de Neftalí, camino del mar, al otro lado del
    Jordán, Galilea de los gentiles. El pueblo que habitaba en tinieblas vio una
    luz grande; a los que habitaban en tierra y sombras de
    muerte, una luz les brilló>>.\\
    Desde entonces comenzó Jesús a predicar diciendo: <<Convertíos, porque está
    cerca el reino de los cielos>>.\\
    Paseando junto al mar de Galilea vio a dos hermanos, a Simón, llamado Pedro, y
    a Andrés, que estaban echando la red en el mar, pues eran pescadores. Les
    dijo: <<Venid en pos de mí y os haré pescadores de hombres>>. Inmediatamente
    dejaron las redes y lo siguieron. Y pasando adelante vio a otros dos hermanos,
    a Santiago, hijo de Zebedeo, y a Juan, su hermano, que estaban en la barca
    repasando las redes con Zebedeo, su padre, y los llamó. Inmediatamente dejaron
    la barca y a su padre y lo siguieron.\footnote{Mt~4,12--22}}

  No sería difícil ahora visualizar una variedad de escenarios en los que este
  texto pueda ser discutido en nuestro contexto eclesial. Es proclamado, por
  ejemplo, en el ciclo A el III Domingo del Tiempo Ordinario. Es así que puede
  escucharse en las reflexiones del Papa Francisco en el Ángelus en la Plaza de
  San Pedro, donde destaca el hecho de que la misión de Jesús comience en una zona
  periférica:
  \citalitlar{Es una tierra de frontera, una zona de tránsito donde se encuentran
    personas diversas por raza, cultura y religión. La Galilea se convierte así en
    el lugar simbólico para la apertura del Evangelio a todos los pueblos. Desde
    este punto de vista, Galilea se asemeja al mundo de hoy: presencia simultánea
    de diversas culturas, necesidad de confrontación y necesidad de encuentro.
    También nosotros estamos inmersos cada día en una <<Galilea de los gentiles>>,
    y en este tipo de contexto podemos asustarnos y ceder a la tentación de
    construir recintos para estar más seguros, más protegidos. Pero Jesús nos
    enseña que la Buena Noticia, que Él trae, no está reservada a una parte de la
    humanidad, sino que se ha de comunicar a todos. Es un feliz anuncio destinado
    a quienes lo esperan, pero también a quienes tal vez ya no esperan nada y no
    tienen ni siquiera la fuerza de buscar y pedir.\footnote{PAPA FRANCISCO
      ÁNGELUS Plaza de San Pedro Domingo 26 de enero de 2014}}

  Tambíen el Papa Benedicto XVI ofreció su comentario y se fijó en la fuerza de esa
  noticia que Cristo comenzaba a anunciar:
  \citalitlar{El término ``evangelio'', en tiempos de Jesús, lo usaban los
    emperadores romanos para sus proclamas. Independientemente de su contenido, se
    definían ``buenas nuevas'', es decir, anuncios de salvación, porque el
    emperador era considerado el señor del mundo, y sus edictos, buenos presagios.
    Por eso, aplicar esta palabra a la predicación de Jesús asumió un sentido
    fuertemente crítico, como para decir: Dios, no el emperador, es el Señor del
    mundo, y el
    verdadero Evangelio es el de Jesucristo.\\
    La ``buena nueva'' que Jesús proclama se resume en estas palabras: ``El reino
    de Dios —-o reino de los cielos-— está cerca''. ¿Qué significa esta expresión?
    Ciertamente, no indica un reino terreno, delimitado en el espacio y en el
    tiempo; anuncia que Dios es quien reina, que Dios es el Señor,
    y que su señorío está presente, es actual, se está realizando.\\
    Por tanto, la novedad del mensaje de Cristo es que en él Dios se ha hecho
    cercano, que ya reina en medio de nosotros, como lo demuestran los milagros y
    las curaciones que realiza.\footnote{BENEDICTO XVI ÁNGELUS Plaza de San Pedro
      Domingo 27 de enero de 2008}}

  No sólo en San Pedro, sino que también podría encontrarse este texto en la
  celebración de la eucaristía domincal resonando en las comunidades y parroquias;
  en las homilias, oraciones, reflexiones o cánticos, invitando a la conversión y
  haciendo nueva la invitación de Jesús: <<Convertíos, porque está cerca el reino
  de los cielos>>. Quizás tambíen se le oiga entre algún grupo juvenil donde
  Simón, Andrés, Santiago y Juan sean tratados como modelos de vocación a la vida
  consagrada o al apostolado, atendiendo con entusiasmo cómo lo dejaron todo en el
  momento para seguir a Jesús. Seguramente algún joven reconociendo aquella
  llamada: <<Venid en pos de mí y os haré pescadores de hombres>> sonando como voz
  dentro de sí.

  El texto de la Escritura es tratado en estos contextos como testimonio de la
  vida de Jesucristo y de la vida de aquellos que le llaman maestro y que
  participan de su misión. No son, sin embargo, tratados como historias del
  pasado, sino como palabras para el presente. Es hoy que la Buena Noticia no está
  reservada a una parte de la humanidad, sino que ha de comunicarse a todos como
  insiste el Papa Francisco. Es hoy que Dios se hace cercano en Cristo para reinar
  en medio de nosotros como enseñó Benedicto XVI. Es hoy que Jesús nos invita a la
  conversión y a ir en pos de él.

  Es sobre esta costumbre de la Iglesia que ha de formularse ahora una pregunta.
  Resultará apropiado apelar aquí a otra costumbre de la Iglesia y buscar luz para
  esto en las Confesiones de San Agustín. Pensando en Dios y pensando en el
  tiempo, Agustín queda inquieto por una serie de preguntas: \citalitlar{¿Qué es,
    pues, el tiempo? ¿Quién podrá explicar esto fácil y brevemente? ¿Quién podrá
    comprenderlo con el pensamiento, para hablar luego de él? Y, sin embargo, ¿qué
    cosa más familiar y conocida mentamos en nuestras conversaciones que el
    tiempo? Y cuando hablamos de él, sabemos sin duda qué es, como sabemos o
    entendemos lo que es cuando lo oímos pronunciar a otro. ¿Qué es, pues, el
    tiempo? Si nadie me lo pregunta, lo sé; pero si quiero explicárselo al que me
    lo pregunta, no lo sé.\footnote{De las confesiones xi.14 (n. 17)}}

  Agustín expresa su extrañeza de que un concepto empleado ordinariamente se
  torne tan desconocido cuando llega la hora de explicarlo. ``¿Qué es el
  tiempo?'' o ``¿qué es conocer?'', ``¿la libertad?'' y ``¿qué es la fe?'' son
  preguntas de este tipo; distintas, por ejemplo, a ``¿cuál es el peso exacto de
  este objeto?'' o ``¿quién será la próxima persona en entrar por esa
  puerta?''.\footnote{cf. Wittgenstein BT. p.304} Preguntar ``¿qué es conocer una
  verdad para la vida por el testimonio de la Escritura?'' sería, como la pregunta
  agustiniana sobre el tiempo, una pregunta sobre la naturaleza o esencia de
  este fenómeno. Un concepto familiar en la vida de la Iglesia como el
  testimonio queda enmarcado como problema cuando nos acercamos a él queriendo
  comprender su esencia.

  Para continuar explorando la naturaleza de la pregunta sobre el testimonio
  resultará útil recurrir aquí al modo en que el psicólogo William James formula
  algunas preguntas sobre la Escritura al comienzo de sus conferencias sobre la
  \emph{religion natural}. Apelando a la literatura de lógica de su época a
  comienzos del siglo XX distingue dos niveles de investigación sobre cualquier
  tema: aquellas preguntas que se resuelven por medio de prposiciones
  \emph{existenciales}, como ``¿qué constitución, qué origen, qué historia tiene
  esto?'' o ``¿cómo se ha realizado esto?''; en segundo lugar las preguntas que se
  responden con proposiciones de \emph{valor} como ``¿cuál es la importancia,
  sentido o significado actual de esto?''. A este segundo juicio James lo denomina
  \emph{juicio espiritual}. Aplicando esta distinción a la Biblia se cuestiona:

  \citalitlar{ <<¿Bajo qué condiciones biográficas los escritores sagrados aportan
    sus diferentes contribuciones al volumen sacro?>>, <<¿Cúal era exactamente el
    contenido intelectual de sus declaraciones en cada caso particular?>>. Por
    supuesto, éstas son preguntas sobre hechos históricos y no vemos cómo las
    respuestas pueden resolver, de súbito, la última pregunta: <<¿De qué modo este
    libro, que nace de la forma descrita, puede ser una guía para nuestra vida y
    una revelación?>>. Para contestar habríamos de poseer alguna teoría general
    que nos mostrara con qué peculiaridades ha de contar una cosa para adquirir
    valor en lo que concierne a la revelación; y, en ella misma, tal teoría sería
    lo que antes hemos denominado un juicio espiritual.\footnote{William James
      Variedades de la Experiencia Religiosa p. 27} }

  Desde esta perspectiva la pregunta sobre cómo el testimonio de la escritura
  puede ser una guía para nuestra vida es una investigación sobre la importancia,
  sentido o significado que ésta tiene actualmente. La respuesta emitida en
  conclusión sería un juicio de valor sobre el fenómeno del testimonio. James
  propone que sería necesaria una teoría general que explicara qué características
  ha de tener alguna cosa para que merezca ser valorada como revelación. Así
  planteado, la pregunta sobre el testimonio sería atendida adecuadamente por
  medio de una investigación que indagara dentro de este fenómeno para descubrir
  los elementos que le otorgan el valor adecuado como para ser considerado guía
  para nuestra vida o una revelación. La explicación de dichos elementos
  configurarían una teoría que nos permitiría juzgar un testimonio concreto como
  valioso, o no, como guía o revelación para nuestras vidas.

  La ruta sugerida por este modo de conducir esta investigación, sin embargo, nos
  dejaría lejos del modo en que Elizabeth Anscombe se plantea un problema
  filosófico. En el trasfondo de su metodología filosófica está la propuesta por
  Ludwig Wittgenstein. Aunque se verá con más detalle qué implica esto, es
  necesario anticipar ahora algo acerca del modo en que ambos se encaminan a la
  hora de atender una investigación filosófica.

  En \emph{Investigaciones Filosóficas} \S89 Wittgenstein hace referencia al
  texto antes citado de las Confesiones para describir la peculiaridad de las
  preguntas filosóficas:
  \citalitlar{ Augustine says in \emph{Confessions} XI. 14, ``quid est ergo
    tempus? si nemo ex me quaerat scio; si quaerenti explicare velim nescio''.
    --This could not be said about a question of natural science (``What is the
    specific gravity of hydrogen'', for instance). Something that one knows when
    nobody asks one but no longer knows when one is asked to explain it, is
    something that has to be \emph{called to mind}. (And it is obviously
    something which, for some reason, it is difficult to call to mind.)}

  Para Wittgenstein es de gran importancia atender el paso que damos para
  resolver la perplejidad causada por el reclamo de explicar un fenómeno. El
  deseo de aclararlo nos puede impulsar a buscar una explicación dentro del
  fenómeno mismo, o cómo él diría: \citalitinterlin{We feel as if we had to see
    right into phenomena}.\footnote{\S90} Esta predisposición nos puede conducir
  a ignorar la amplitud del modo en que el lenguaje sobre esto es empleado en la
  actividad humana y a enfocarnos sólo en un elemento particular del lenguaje
  sobre este fenómeno y tomarlo como un ejemplo paradigmático para construir un
  modelo abstrayendo explicaciones y generalizaciones sobre él. Esta manera de
  indagar, le parece a Wittgenstein, nos hunde cada vez más profundamente en un
  estado de frustración y confusión filosófica de modo que llegamos a imaginar
  que para alcanzar claridad \citalitinterlin{we have to describe extreme
    subtleties, which again we are quite unable to describe with the means at
    our disposal. We feel as if we had to repair a torn spider's web with our
    fingers.}\footnote{\S106}

  La alternativa que Wittgenstein propone es una investigación que no esté
  dirigida hacia dentro del fenómeno, sino \citalitinterlin{as one might say,
    towards the \emph{`possibilities'} of phenomena. What that means is that we
    call to mind the \emph{kinds of statement} that we make about phenomena}. A
  este esfuerzo le denomina ``investigación gramática''. La describe de este modo:
  \citalitlar{ Our inquiry is therefore a grammatical one. And this inquiry sheds
    light on our problem by clearing misunderstandings away. Misunderstandings
    concerning the use of words, brought about, among other things, by certain
    analogies between the forms of expression in different regions of our
    language. -- Some of them can be removed by substituting one form of
    expression for another; this may be called `analysing' our forms of
    expression, for sometimes this procedure resembles taking things
    apart.\footnote{\S90}} El modo de salir de nuestra perplejidad, por tanto,
  consiste en prestar cuidadosa atención al uso que hacemos de hecho con las
  palabras y la aplicación que empleamos de las expresiones. Esto está al
  descubierto en nuestro uso del lenguaje de modo que la dificultad para
  \emph{traer a la mente} aquello que aclare un fenómeno no está en descubrir algo
  oculto en éste, sino en aprender a valorar lo que tenemos ante nuestra vista:
  \citalitinterlin{The aspects of things that are most important for us are hidden
    because of their simplicity and familiarity. (One is unable to notice
    something -- because it is always before one's eyes.)}\footnote{\S129} La
  descripción de los hechos concernientes al uso del lenguaje en nuestra actividad
  humana ordinaria componen los pasos del tipo de investigación sugerido por
  Wittgenstein. Hay cierta insatisfacción en este modo de proceder, como él mismo
  afirma: \citalitlar{Where does this investigation get its importance from, given
    that it seems only to destroy everything interesting: that is, all that is
    great and important? (As it were, all the buildings, leaving behind only bits
    of stone and rubble.) But what we are destroying are only houses of cards, and
    we are
    clearing up the ground of language on which they stood.\\
    The results of philosophy are the discovery of some piece of plain nonsense
    and the bumps that the understanding has got running up against the limit of
    language. They -- these bumps -- make us see the value of that discovery.}

  Anscombe, al igual que Wittgenstein, no se limita a emplear un sólo método para
  hacer filosofía, como afirma el mismo Wittgenstein: \citalitinterlin{There is
    not a single philosophical method, though there are indeed methods, different
    therapies as it were}.\footnote{\S133} Sin embargo si atendemos a su modo de
  hacer filosofía podemos encontrarla empleando lenguajes o juegos de lenguaje
  imaginarios para arrojar luz sobre modos actuales de usar el lenguaje o esquemas
  conceptuales; del mismo modo su trabajo esta lleno de ejemplos donde la
  encontramos examinando con detenimiento el uso que de hecho hacemos del
  lenguaje.\footnote{cf. teichmann p. 228-229} Es visible en ella ese
  \citalitinterlin{modo característicamente Wittgensteniano de rebatir la
    tendencia del filósofo de explicar alguna cuestión filosóficamente enigmática
    inventando una entidad o evento que la causa, así como los físicos inventan
    partículas como el gravitón}.\footnote{There is however a somehow
    chracteristically Wittgenstenian way of countering the philosopher's tendency
    to explain a philosophically puzzling thing by inventing an entity or event
    which causes it, as physicists invent particles like the graviton. From plato
    to witt intro xix}

  Según el título de este trabajo ha prometido, el análisis sobre el testimonio
  que será expuesto es el que se encuentra desarrollado en el pensamiento de
  Elizabeth Anscombe. La pregunta planteada al inicio: ¿qué es conocer una verdad
  para la vida por el testimonio de la Escritura?, entendida como investigación
  filosófica, será examinada en las descripiciones que Anscombe realiza sobre el
  modo de usar el lenguaje sobre el creer, la confianza, la verdad, la fe y otros
  fenómenos relacionados con el conocer por testimonio. Nuestro título adiverte
  además que ésta es una investigación en perspectiva teólogica, cabe
  inmendiatamente añadir algo breve al respecto.

  ¿Qué es teología?, se preguntaba Joseph Ratzinger en su alocución en el 75
  aniversario del nacimiento del cardenal Hermann Volk en 1978, e introducía
  suscintamente su respuesta a esa pregunta tan grande diciendo:

  \citalitlar{Cuando se intenta decir algo sobre esta materia, precisamente como
    tributo al cardenal Volk y a su pensamiento, se asocian, poco menos que
    automáticamente, dos ideas. Me viene a las mientes, por un lado, su divisa (y
    título de uno de sus libros): \emph{Dios todo en todos}, y el programa
    espiritual contenido en ella; por otra parte, se aviva el recuerdo de lo que
    ya antes se ha insinuado: un modo de interrogar total y absolutamente
    filosófico, que no se detiene en reales o supuestas comprobaciones históricas,
    en diagnósticos sociológicos o en técnicas pastorales, sino que se lanza
    implacablemente a la busqueda de los fundamentos.\\
    Según esto, cabría formular ya dos tesis que pueden servirnos de hilo
    conductor para nuestro interrogante sobre la esencia de la teología:\\
    1. La teología se refiere a Dios.\\
    2. El pensamiento teológico está vinculado al modo de cuestionar filosófico
    como a su método fundamental.\footnote{teoría de los principios teológicos, p
      380}}
  Esta investigación sobre el testimonio como parte de la vida de la Iglesia será
  realizada atendiendo al modo de cuestionar filosófico realizado por Elizabeth
  Anscombe como método, examinando esta experiencia en referencia a Dios, es
  decir, como vivencia de su ser y de su obrar.

  Hasta aquí simplemente se ha descrito un modo de andar a través de la discusión
  acerca de la categoría del testimonio atendiendo el hecho de que tanto la
  temática como la figura de Anscombe otorgan a este camino peculiaridades que hay
  que tener en cuenta. Siendo concientes de estas particularidades podríamos ahora
  ampliar más el horizonte respecto de dos cuestiones brevemente expuestas
  anteriormente. En primer lugar es necesario ampliar la descripción hecha hasta
  aquí del fenómeno del testimonio en la vida de la Iglesia, ya que aunque nos
  resulte familiar relacionarlo con el testimonio de la Sagrada Escritura, tanto
  en el Magisterio de la Iglesia como en la propia Escritura se haya presente la
  categoría del testimonio con una riqueza que merece la pena explorar. En segundo
  lugar habría que detallar todavía mejor lo problemático del testimonio, sobre
  todo cuando se considera su importancia en la transmisión de la fe y el anuncio
  del Evangelio en el mundo.

\section{La categoría del Testimonio en la Sagrada Escritura}
La Iglesia de hoy, como María, conserva el Evangelio meditándolo en su
corazón.\footnote{Lc 2,19} Así está presente en el centro de la comunidad
creyente el anuncio de Cristo vivo como fundamento de su esperanza en cada etapa
de la historia. Este motivo de esperanza conservado es también compartido y
expresado, según la enseñanza del apóstol:\citalitinterlin{glorificad a Cristo
  en vuestros corazones, dispuestos siempre a dar explicación a todo el que os
  pida una razón de vuestra esperanza}.\footnote{1Pe 3, 15} Este Evangelio
atesorado como fundamento en el centro de la vida de la comunidad eclesial, así
como Buena Nueva proclamada y transmitida en el tiempo y en el mundo puede ser
comprendido como tres testimonios que son uno:<<palabra vivida en el
Espíritu>>\footnote{cf. Porque es el Espíritu el que impulsa a la Iglesia a
  perseguir son obras de evangelización; es el Espíritu quien santifica y
  fecunda el testimonio de su vida; y es el Espíritu el que inspira la fe, la
  nutre y la profundiza. Es el Espíritu quien alivia entre estos tres
  testimonios que son uno: el de la palabra vivida en el Espíritu. A través del
  testimonio, el Espíritu internaliza el testimonio externo de la Buena Nueva de
  la salvación en Jesucristo y lo lleva al cumplimiento de la fe, que es la
  respuesta del amor del verdadero amor de la humanidad a través del Padre.
  Cristo; Latourelle Evangelisation et temoignage ninot 582}.

La Evangelización puede ser entendida en este sentido como testimonio de la
<<palabra de vida>>\footnote{1Jn 1,1} que los apóstoles anuncian como testigos
de lo que han contemplado y palpado\footnote{1Jn 1,1}. Es también el testimonio
del modo de vivir de los cristianos que, acogiendo esta palabra, la viven,
poniendo por obra lo que ella enseña. Es además testimonio del Espíritu Santo
que internaliza el testimonio externo de la Buena Noticia y lo lleva al
cumplimiento de la fe en cada persona.\footnote{cf. latourelle, ninot 582} Es el
Espíritu el que santifica y fecunda la acción de los cristianos, es tambíen el
que impulsa y sostiene la acción de la Iglesia; es el Espíritu el que inspira la
fe, la nutre y la profundiza.\footnote{latourelle evangelisation et temoignage}

Este dinamísmo fundamental que puede encontrarse vivo hoy en la comunidad de la
Iglesia ha actuado en ella desde su origen y le ha acompañado en cada época.
Según esto es posible valorar lo que se transmite en la tradición eclesial como
la perpetuación de la actividad de Cristo y los apóstoles, que es a su vez
proyección del testimonio divino.\footnote{ el testimonio divino se proyecta
  luego en el apostólico y se perpetúa en el testimonio eclesial. Por eso, el
  testimonio es revelación en la actividad de Cristo y de los apóstoles y es
  transmisión de la revelación en la tradición eclesial. ninot 573}

En la actividad de Cristo el testimonio divino queda proyectado como
interpelación a la libertad realizada por la identidad propia de Jesús:
\citalitinterlin{Si conocieras el don de Dios y quién es el que te dice ``dame
  de beber'' le pedirías tu, y él te daría agua viva}\footnote{Jn 4, 10};
\citalitinterlin{``¿Crees tú en el Hijo del hombre?''\ldots ``¿Y quién es,
  Señor, para que crea en él?''\ldots ``Lo estás viendo: el que te está
  hablando, ese es''}\footnote{Jn 9, 35--37}. En la actividad apostólica, el
testimonio divino sigue interpelando la libertad humana como manifestación de
Jesús Resucitado. Los apóstoles actuan como testigos de los acontecimientos de
la Pascua de Jesús y su valor salvífico\footnote{cf. ninot 576} y este
testimonio es descrito como acción del Espíritu que impulsa la tarea apostólica
y que da nueva vida a los que acogen el anuncio de la Buena Noticia.

Puede encontrarse un ejemplo de esto en el testimonio de Felipe. El apóstol sale
más allá de Jerusalén hacia Samaria, y todavía llega más lejos, al compartir la
Buena Noticia de Jesús con un extranjero Etíope: \citalitlar{El Espíritu dijo a
  Felipe: <<Acércate y pégate a la carroza>>. Felipe se acercó corriendo, le oyó
  leer el profeta Isaías, y le preguntó: <<¿Entiendes lo que estás leyendo?>>.
  Contestó: <<¿Y cómo voy a entenderlo si nadie me guía?>>. E invitó a Felipe a
  subir y a sentarse con él. El pasaje de la Escritura que estaba leyendo era
  este: \emph{Como cordero fue llevado al matadero, como oveja muda ante el
    esquilador, así no abre su boca. En su humillación no se le hizo justicia.
    ¿Quién podrá contar su descendencia? Pues su vida ha sido arrancada de la
    tierra.} El eunuco preguntó a Felipe: <<Por favor, ¿de quién dice esto el
  profeta?; ¿de él mismo o de otro?>>. Felipe se puso a hablarle y, tomando pie
  de este pasaje, le anunció la Buena Nueva de Jesús. Continuando el camino,
  llegaron a un sitio donde había agua, y dijo el eunuco: «Mira, agua. ¿Qué
  dificultad hay en que me bautice?». Mandó parar la carroza, bajaron los dos al
  agua, Felipe y el eunuco, y lo bautizó. Cuando salieron del agua, el Espíritu
  del Señor arrebató a Felipe. El eunuco no volvió a verlo, y siguió su camino
  lleno de alegría. \footnote{Hch 8, 29--39}} Además de ser ejemplo de la
actividad apostólica, este relato puede servir como síntesis del modo en que la
categoría del testimonio está presente en la Escritura.

El testimonio comienza con la iniciativa de Dios mismo que impulsa tanto la
palabra profética del Antiguo Testamento como el anuncio apostólico del Nuevo
Testamento. Esta iniciativa de Dios tiende hacia el testimonio de la Palabra
definitiva del Padre que es Cristo resucitado. En aquellos que creen en el
testimonio de Dios se engendra alegría y vida nueva. En palabras de R.
Latourelle:
\citalitlar{En el trato de las tres personas divinas con los hombres existe un
  intercambio de testimonios que tiene la finalidad de proponer la revelación y
  de alimentar la fe. Son tres los que revelan o dan testimonio, y esos tres son
  más que uno. Cristo da testimonio del Padre, mientras que el Padre y el
  Espíritu dan testimonio del Hijo. Los apóstoles a su vez dan testimonio de lo
  que han visto y oído del verbo de la vida. Pero su testimonio no es la
  comunicación de una ideología, de un descubrimiento científico, de una técnica
  inédita, sino la proclamación de la salvación prometida y finalmente
  realizada.\footnote{diccion testimonio p.1531}}
De este modo el anuncio del apóstol Felipe sirve aquí como un ejemplo específico
del testimonio, que ilustra sin embargo, una noción
que\citalitinterlin{atraviesa toda la Escritura y se corresponde con la
  estructura misma de la revelación.}\footnote{la noción de testimonio atraviesa
  la Escritura y se coresponde con la estructura misma de la Revelación: <<la
  Escritura describe la revelación como una economía del testimonio>>. prades
  109} El testimonio está presente a lo largo de la Escritura junto a otras
categorías como pueden ser la de `alianza', `palabra', `paternidad' o
`filiación', como parte del \citalitinterlin{grupo de analogías empleadas por la
  Escritura para introducir al hombre en las riquezas del misterio
  divino}.\footnote{latourelle p. 1523}

Esta clave servirá para dar enfoque a un examen sobre la categoría del
testimonio en la Escritura. ¿Qué nos dice el Antiguo y el Nuevo Testamento de la
revelación como acto testimonial de Dios? Esta pregunta supone que la revelación
comparte los rasgos de la actividad humana que es el testimonio, sin embargo,
como Latourelle adiverte: \citalitinterlin{globalmente se puede decir que el
  testimonio bíblico asume pero al mismo tiempo exalta hasta sublimarlos, los
  rasgos del testimonio humano.}\footnote{cf. latourelle 1526 Globalmente se
  puede decir que el testimonio bíblico asume pero al mismo tiempo exalta hasta
  sublimarlos, los rasgos del testimonio humano. latourelle 1526}

Cabe añadir una última consideración. La revelación de Dios entendida como acto
testimonial suyo tiene como expresión definitiva el misterio pascual de
Cristo.\footnote{cf. el misterio pascual al cual tiende toda la existencia
  terrena de Cristo, constituye el acto testimonial por excelencia de Dios
  prades 128} Este misterio ocupa el lugar principal en el testimonio bíblico:
\citalitlar{la Resurrección como ``final'' de la unicidad del acontecimiento de
  Jesucristo, encarnado, muerto y resucitado, subraya específicamente la
  definitividad de la existencia humana salvada por Dios en la carne de Jesús de
  Nazaret, ya que la autocomunicación de Dios ha alcanzado su palabra última en
  la Resurrección de Jesucristo, y por eso es prenda de la resurrección de todos
  los hombres.\footnote{ninot 404}}
Como tal, parece justo tratar el testimonio que es el misterio pascual en su
propio apartado. Y será éste precisamente el punto de partida para esta
descripción de la categoría del testimonio en la Escritura.

\subsection{El testimonio en el misterio y anuncio pascual}

\ifdraft{\subsubsection{Estatuto espistemológico especial}}{}
<<Cristo ha resucitado>>\footnote{Cf. 1 Tes 4,15; 1Cor 15,12--20; Rom 6,4} es la
confesión que está en el núcleo del más primitivo anuncio del
evangelio.\footnote{ninot 403} Creer en esta noticia conlleva acoger la
manifestación más plena de la Revelación y la motivación más definitiva para
creer. En este sentido:
\citalitlar{La Resurrección de Jesús mirada desde la perspectiva de la teología
  fundamental presupone un estatuto epistemológico peculiar, puesto que es el
  punto culminante y objeto de la Revelación y, a su vez, es su acreditación
  suprema y máximo motivo de credibilidad, tal como recuerda el texto citado de
  Pablo ``si Cristo no ha resucitado, nuestra predicación es vana y vana es
  nuestra fe'' (1 Cor 15,14).\footnote{ninot 405}}

Este misterio pascual no aparece como hecho desconectado del conjunto de la vida
y misión de Jesús, sino que hacia él tienden sus obras y palabras desde el
comienzo. Cristo pasó por el mundo haciendo el bien, como testimonio de la
bondad de Dios, y esta acción va orientada a ese punto culminante que es su
pasión, muerte y resurrección; \citalitinterlin{el testimonio que Jesús va
  ofreciendo durante su vida pública le va a reclamar una entrega definitiva a
  favor de los que o han acogido y frente a la resistencia que ha generado en
  quienes le rechazan.}\footnote{prades 127}

\ifdraft{\subsubsection{Es testimonio y motivo de credibilidad definitvo de
    confianza absoluta en el Padre}}{}

A lo largo de este camino Jesús manifiesta su confianza en el Padre:
\citalitinterlin{Padre, te doy gracias porque me has escuchado; yo sé que tu me
  escuchas siempre}\footnote{Jn 11, 41b-42a}; esta relación queda afirmada
plenamente ante la pasión como confianza puesta en la voluntad del Padre:
\citalitinterlin{Padre\ldots que no se haga mi voluntad, sino la
  tuya}\footnote{Lc 22,42}. De este modo en el misterio pascual queda
atestiguada la plena unidad de Cristo con el Padre, en la mayor confianza
imaginable.\footnote{prades 127}

\ifdraft{\subsubsection{Es testimonio y motivo de credibilidad definitvo del
    Amor de Dios}}{}

\ifdraft{\subsubsection{Es testimonio y motivo de credibilidad definitvo del
    proyecto de salvación deseado libremente por Dios}}{}

Si Cristo no ha resucitado
sería vana cualquier argumentación, sin embargo, Jesús es <<el Viviente>>,
estuvo muerto, pero vive por los siglos de los siglos.\footnote{Ap 1, 17--18}
Así Pedro da testimonio de ésto el día de Pentecostés: \citalitinterlin{A este
  Jesús lo resucitó Dios, de lo cual todos nosotros somos
  testigos}.\footnote{Hch 2, 32} El apóstol es testigo en la fe sobre un
acontecimiento enraizado en la historia.\footnote{ninot 402 y 406 enraizado}

Así mismo es presentado el testimonio de Pedro en casa de Cornelio donde el
centurión y todos lo que lo acompañaban esperaban reunidos para escuchar lo que
el Señor quisiera comunicarles por medio del apóstol. Pedro, comprendiendo que
la verdad de Dios no hace acepción de personas, narra los hechos que él bien
conoce: \citalitlar{<<Vosotros conocéis lo que sucedió en toda Judea, comenzando
  por Galilea, después del bautismo que predicó Juan. Me refiero a Jesús de
  Nazaret, ungido por Dios con la fuerza del Espíritu Santo, que pasó haciendo
  el bien y curando a todos los oprimidos por el diablo, porque Dios estaba con
  él. Nosotros somos testigos de todo lo que hizo en la tierra de los judíos y
  en Jerusalén. A este lo mataron, colgándolo de un madero. Pero Dios lo
  resucitó al tercer día y le concedió la gracia de manifestarse, no a todo el
  pueblo, sino a los testigos designados por Dios: a nosotros, que hemos comido
  y bebido con él después de su resurrección de entre los
  muertos.>>\footnote{Hch 10, 37--41}} Este testimonio de los hechos queda
enlazado con un testimonio de fe sobre el sentido profundo de lo que Pedro
conoce, Jesús, a quien los apóstoles y el pueblo vieron y escucharon es ahora
juez de vivos y muertos:
\citalitlar{<<Nos encargó predicar al pueblo, dando solemne testimonio de que
  Dios lo ha constituido juez de vivos y muertos. De él dan testimonio todos los
  profetas: que todos los que creen en él reciben, por su nombre, el perdón de
  los pecados.>>\footnote{Hch 10, 42-43}}

El apóstol entiende estos hechos y su alcance religioso y salvífico
interpretándolos en continuidad con la voluntad de Dios manifestada en su acción
en favor del pueblo judío a quién habló por medio de los profetas; voluntad
hecha manifiesta en definitva en \citalitinterlin{Jesús el Nazareno, varón
  acreditado por Dios ante vosotros con los milagros, prodigios y signos que
  Dios realizó por medio de él, como vosotros mismos sabéis}.\footnote{Hch 2,22}

La categoría del testimonio en el anuncio pascual es sobre un hecho enraizado en
la historia, que tiene un alcance religioso y salvífico y que es interpretado
desde la voluntad de Dios manifestada en los hechos y palabras de Cristo. Sin
las obras que Jesús realizó, el testimonio apostólico se derrumba, no
existe.\footnote{Latourelle 1529} Sin la vida y obra, muerte y resurrección de
Jesús \citalitinterlin{resultamos unos falsos testigos de Dios, porque hemos dado
  testimonio contra él, diciendo que ha resucitado a Cristo, a quien no ha
  resucitado}.\footnote{1 Cor 15, 15} Sin embargo, Jesús es testigo verdadero y
acreditado, en su resurreción se realiza en acto lo que ha
prometido\footnote{prades 128}, de modo que el testimonio apostólico queda
acreditado.

\subsection{El testimonio en el Antiguo Testamento}

  I. en el AT se nos describe la revelación como acto testimonial de Dios en estos términos:
  prades 114-115:

  Yahvé da testimonio de sí
  en las obras de la creación
  en la ley que son testimonio
  hombres escogidos como moisés

  II. en el AT la revelación es acto testimonial de Dios ante todo a través de los
  profetas

  latourelle: en el AT el testigo es ante todo el profeta

  el testigo es también el pueblo de Israel

Is 43, 8--12

  la autoridad del testigo no viene de él, sino de su vocación privilegiada y de
  su envío. prades 117

  lo que escinde este nuevo sentido del testimonio de todos sus usos en el
  lenguaje ordinario es que el testimonio no pertenece al testigo. Este procede de
  una iniciativa absoluta, en cuanto a su origen y en cuanto a su contenido.
  prades 118 - ricoeur

  en el AT el testimonio en su sentido más denso y sublime, es el de Dios mismo a
  través de personas escogidas por Él, que reminten continuamente a hechos
  acontecidos en la historia y a la interpretación que los acompaña, para
  reconocer con ello la presencia y actuación de Dios en la historia humana.

  Hay peligro de falsos profetas y también de corazones sordos. El pueblo es
  infiel y testarudo y actua irracionalmente rechazando las múltiples pruebas de
  la predilección divina.

\subsection{El testimonio en el Nuevo Testamento}
  El testimonio exterior va acompañado de un testimonio interior del Espíritu que
  hace al hombre capaz de abrirse al evangelio y de adherirse a él por la fe. 1530


!!!Aquí terminar otra vez con mateo y hablando de el universal concreto



\end{document}
