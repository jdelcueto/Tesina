\documentclass[../main.tex]{subfiles}
\begin{document}
\setcounter{chapter}{0}

\chapter{INTRODUCCIÓN GENERAL AL PROBLEMA DEL TESTIMONIO}
% Este capítulo se compone de tres apartados:

% El primer apartado es una introducción a todo el trabajo y se
% plantea cuál es la naturaleza de la pregunta que la investigación
% pretende responder
% LA NATURALEZA DE LA PREGUNTA SOBRE EL TESTIMONIO
%\section{Naturaleza de la pregunta sobre el testimonio}

Es una experiencia familiar en nuestras comunidades reunirnos en torno a la
Sagrada Escritura y compartir la Palabra buscando en ella luz para nuestro
presente. Podemos recurrir a este típico escenario para apoyar el primer paso de
nuestra investigación sobre el testimonio. Imaginemos un domingo,
específicamente un tercer domingo del Tiempo Ordinario. En el ciclo A, el
Evangelio que se proclama ese día es este:

\blockquote[Mt~4,12--23]{Al enterarse Jesús de que habían arrestado a Juan se
  retiró a Galilea. Dejando Nazaret se estableció en Cafarnaún, junto al mar, en
  el territorio de Zabulón y Neftalí, para que se cumpliera lo dicho por medio
  del profeta Isaías:

  \enquote{Tierra de Zabulón y tierra de Neftalí, camino del mar, al otro lado
    del Jordán, Galilea de los gentiles. El pueblo que habitaba en tinieblas vio
    una luz grande; a los que habitaban en tierra y sombras de muerte, una luz
    les brilló}.

  Desde entonces comenzó Jesús a predicar diciendo: \enquote{Convertíos, porque
    está cerca el reino de los cielos}.

  Paseando junto al mar de Galilea vio a dos hermanos, a Simón, llamado Pedro, y
  a Andrés, que estaban echando la red en el mar, pues eran pescadores. Les
  dijo: \enquote{Venid en pos de mí y os haré pescadores de hombres}.
  Inmediatamente dejaron las redes y lo siguieron. Y pasando adelante vio a
  otros dos hermanos, a Santiago, hijo de Zebedeo, y a Juan, su hermano, que
  estaban en la barca repasando las redes con Zebedeo, su padre, y los llamó.
  Inmediatamente dejaron la barca y a su padre y lo siguieron. Jesús recorría
  toda Galilea enseñando en sus sinagogas, proclamando el evangelio del reino y
  curando toda enfermedad y toda dolencia en el pueblo.}

No sería difícil ahora visualizar una variedad de situaciones en las que este
texto pueda ser discutido en nuestro contexto eclesial. En enero de 2014 el Papa
Francisco lo reflexionaba en el Ángelus en la Plaza de San Pedro y destacaba que
la misión de Jesús comenzara en una zona periférica:

\blockquote[{\cite{francisco2014angelus}}]{Es una tierra de frontera, una zona
  de tránsito donde se encuentran personas diversas por raza, cultura y
  religión. La Galilea se convierte así en el lugar simbólico para la apertura
  del Evangelio a todos los pueblos. Desde este punto de vista, Galilea se
  asemeja al mundo de hoy: presencia simultánea de diversas culturas, necesidad
  de confrontación y necesidad de encuentro. También nosotros estamos inmersos
  cada día en una \enquote{Galilea de los gentiles}, y en este tipo de contexto
  podemos asustarnos y ceder a la tentación de construir recintos para estar más
  seguros, más protegidos. Pero Jesús nos enseña que la Buena Noticia, que Él
  trae, no está reservada a una parte de la humanidad, sino que se ha de
  comunicar a todos. Es un feliz anuncio destinado a quienes lo esperan, pero
  también a quienes tal vez ya no esperan nada y no tienen ni siquiera la fuerza
  de buscar y pedir.}

El Papa Benedicto XVI también había ofrecido su comentario unos años antes. En
su caso el acento del relato lo encontró en la fuerza singular de esa
\enquote{buena nueva} que Cristo comenzaba a anunciar:

\blockquote[{\cite{benedicto2008angelus}}]{El término \enquote{evangelio}, en
  tiempos de Jesús, lo usaban los emperadores romanos para sus proclamas.
  Independientemente de su contenido, se definían \enquote{buenas nuevas}, es
  decir, anuncios de salvación, porque el emperador era considerado el señor del
  mundo, y sus edictos, buenos presagios. Por eso, aplicar esta
  palabra a la predicación de Jesús asumió un sentido fuertemente crítico, como
  para decir: Dios, no el emperador, es el Señor del mundo, y el verdadero
  Evangelio es el de Jesucristo.

  La \enquote{buena nueva} que Jesús proclama se resume en estas palabras:
  \enquote{El reino de Dios ---o reino de los cielos--- está cerca}. ¿Qué
  significa esta expresión? Ciertamente, no indica un reino terreno, delimitado
  en el espacio y en el tiempo; anuncia que Dios es quien reina, que Dios es el
  Señor, y que su señorío está presente, es actual, se está realizando.

  Por tanto, la novedad del mensaje de Cristo es que en él Dios se ha hecho
  cercano, que ya reina en medio de nosotros, como lo demuestran los milagros y
  las curaciones que realiza.}

Ciertamente este texto no se le encontraría solamente en San Pedro, sino que
estaría presente en la celebración de la eucaristía domincal resonando en las
comunidades y parroquias alrededor del mundo; en las homilias, oraciones,
reflexiones o cánticos, invitando a la conversión y haciendo nueva la invitación
de Jesús: \enquote{Convertíos, porque está cerca el reino de los cielos}. Quizás
tambíen se le oiga entre algún grupo juvenil donde Simón, Andrés, Santiago y
Juan sean tratados como modelos de vocación a la vida consagrada o al
apostolado, atendiendo con entusiasmo cómo lo dejaron todo en el momento para
seguir a Jesús. Seguramente algún joven reconociendo aquella llamada:
\enquote{Venid en pos de mí y os haré pescadores de hombres} sonando como voz
dentro de sí.

El texto de la Escritura es tratado en estos contextos como testimonio de la
vida de Jesucristo y de la vida de aquellos que le llaman maestro y que
participan de su misión. No son, sin embargo, tratados como historias del
pasado, sino como palabras para el presente. Es hoy que la Buena Noticia no está
reservada a una parte de la humanidad, sino que ha de comunicarse a todos como
insiste el Papa Francisco. Es hoy que Dios se hace cercano en Cristo para reinar
en medio de nosotros como enseñó Benedicto XVI. Es hoy que Jesús nos invita a la
conversión y a ir en pos de él.

Es sobre esta costumbre de la Iglesia que quisieramos formular una pregunta que
ponga en marcha nuestra investigación. Para esto nos servirá emplear otra
costumbre eclesial y acudir al pensamiento de San Agustín para encontrar algo de
luz. En el capítulo XI de las \emph{Confesiones} nos lo encontramos inquieto
---como siempre--- esta vez pensando en Dios y pensando en el tiempo, asaltado
por una serie de preguntas:

\blockquote[{\cite[XI.14 n.17]{confesiones}}]{¿Qué es, pues, el tiempo? ¿Quién
  podrá explicar esto fácil y brevemente? ¿Quién podrá comprenderlo con el
  pensamiento, para hablar luego de él? Y, sin embargo, ¿qué cosa más familiar y
  conocida mentamos en nuestras conversaciones que el tiempo? Y cuando hablamos
  de él, sabemos sin duda qué es, como sabemos o entendemos lo que es cuando lo
  oímos pronunciar a otro. ¿Qué es, pues, el tiempo? Si nadie me lo pregunta, lo
  sé; pero si quiero explicárselo al que me lo pregunta, no lo sé.}

Agustín expresa su extrañeza de que un concepto empleado ordinariamente se torne
tan desconocido cuando llega la hora de explicarlo. \enquote{¿Qué es el tiempo?}
o \enquote{¿qué es conocer?}, \enquote{¿la libertad?} y \enquote{¿qué es la fe?}
son preguntas de este tipo; distintas, por ejemplo, a \enquote{¿cuál es el peso
  exacto de este objeto?} o \enquote{¿quién será la próxima persona en entrar
  por esa puerta?}.\autocite[Cf.~][304]{wittgenstein2005bt} Preguntar
\enquote{¿qué es conocer una verdad para la vida por el testimonio de la
  Escritura?} sería, como la pregunta agustiniana sobre el tiempo, una pregunta
sobre la naturaleza o esencia de este fenómeno. Un concepto familiar en la vida
de la Iglesia como el testimonio queda enmarcado como problema cuando nos
acercamos a él queriendo comprender su esencia.

Esto ya nos da una pista sobre el modo en que nos cuestionaremos acerca del
testimonio. El siguiente elemento que servirá de clave para el estudio lo
obtenemos si precisamos un poco cómo Elizabeth Anscombe se conduce a través de
cuestiones filosóficas como las planteadas anteriormente. Así como telón de
fondo podemos desplegar otro cierto modo de proceder como el que se encuentra en
la investigación realizada a inicios del siglo XX por el psicólogo William
James. Esto nos servirá para contrastar.

Al comienzo de sus conferencias sobre \emph{religión natural} dedica una
exposición breve para explicar algo del método de su estudio sobre las
tendencias religiosas de las personas. Se apoya sobre la literatura de lógica de
su época para distinguir dos niveles de investigación sobre cualquier tema:
aquellas preguntas que se resuelven por medio de proposiciones
\emph{existenciales}, como \enquote{¿qué constitución, qué origen, qué historia
  tiene esto?} o \enquote{¿cómo se ha realizado esto?}. En otro nivel están las
preguntas que se responden con proposiciones de \emph{valor} como \enquote{¿cuál
  es la importancia, sentido o significado actual de esto?}. A este segundo
juicio James lo denomina \emph{juicio espiritual}. El enfoque de sus
conferencias sobre la religión será el existencial, pero no deja de ser
interesante su apreciación de lo que sería un juicio espiritual aplicado a la
Escritura:

\blockquote[{\cite[27]{james2002variedades}}]{\enquote{¿Bajo qué condiciones
    biográficas los escritores sagrados aportan sus diferentes contribuciones al
    volumen sacro?}, \enquote{¿Cúal era exactamente el contenido intelectual de
    sus declaraciones en cada caso particular?}. Por supuesto, éstas son
  preguntas sobre hechos históricos y no vemos cómo las respuestas pueden
  resolver, de súbito, la última pregunta: \enquote{¿De qué modo este libro, que
    nace de la forma descrita, puede ser una guía para nuestra vida y una
    revelación?}. Para contestar habríamos de poseer alguna teoría general que
  nos mostrara con qué peculiaridades ha de contar una cosa para adquirir valor
  en lo que concierne a la revelación; y, en ella misma, tal teoría sería lo que
  antes hemos denominado un juicio espiritual.}

Desde esta perspectiva la pregunta sobre cómo el testimonio de la Escritura
puede ser una guía para nuestra vida es una investigación sobre la importancia,
sentido o significado que éste pueda tener de hecho. La respuesta emitida en
conclusión sería un juicio de valor sobre este fenómeno testimonial. James
propone que sería necesaria una teoría general que explicara qué características
debería de tener alguna cosa para que merezca ser valorada como revelación. Así
planteado, la pregunta sobre el testimonio de la Escritura sería atendida
adecuadamente por medio de una investigación que indagara dentro de este
fenómeno para descubrir los elementos que le otorgan el valor adecuado como para
ser considerado como revelación o estimado como guía para nuestra vida. La
explicación de dichos elementos configurarían una teoría que nos permitiría
juzgar este testimonio concreto como valioso, o no, como revelación y guía para
nuestras vidas.

Si traemos al frente ahora la metodología de Anscombe y la comparamos con la
propuesta de William James se aprecian bien algunas distinciones características
de su filosofía que de tener en cuenta nos evitarán confusiones en la travesía a
lo largo de su obra y pensamiento. En efecto:
\blockquote[{\cite[1]{teichmann2008ans}}: Part of the difficulty in reading
Anscombe is in finding your bearings, and this has to do with her eschewal of
System. A system or theory often makes things easier for the reader. Once you
have grasped N's theory, you can frequently infer what N would have to say on
some point by simply `applying' the theory. But it can often be hard to predict
in advance what Anscombe will say about some given thing. She is infuriatingly
prone to take each case on its merits.]{Parte de la dificultad en leer a
  Anscombe está en encontrar nuestro rumbo, y esto tiene que ver con su evasión
  de Sistema. Un sistema o teoría a menudo hace las cosas más fáciles para el
  lector. Una vez que haz captado la teoría de N, con frecuencia puedes inferir
  qué N tendría que decir sobre algún punto al simplemente \enquote*{aplicar} la
  teoría. Pero frecuentemente puede ser difícil predecir de antemano qué
  Anscombe dirá acerca de alguna cosa dada. Tiene la exasperante tendencia a
  tomar cada caso en sus propios méritos.}
No quiere decir esto que Anscombe carezca de rigor o sistematicidad en sus
escritos, sin embargo suele adentrarse \enquote{in media res} en las discusiones
con la intención de llegar a algún sitio por la fuerza de sus propias
reflexiones sin detenerse a dar mucha explicación de sus presupuestos o del
trasfondo de su discusión.\autocite[Cf.~][1]{teichmann2008ans} Sin embargo en
esta característica de su método hay una cuestión de fondo que tiene que ver con
la influencia de Wittgenstein:
\blockquote[{\cite[1]{teichmann2008ans}}: There is a familiar philosophical, or
meta-philosophical, issue here, to do with the pointfulness or otherwise of
constructing generalizations. Wittgenstein considered prefacing the text of the
Philosophical Investigations with the epigraph `I'll teach you differences', and
Anscombe certainly shared Wittgenstein's belief that glossing over differences
was one of the main sources of error in philosophy. ]{Hay una familiar
  filosófica, o meta-filosófica, cuestión aquí, concerniente a la utilidad o no
  de construir generalizaciones. Wittgenstein consideró prologar el texto de
  \emph{Investigaciones Filosóficas} con el epígrafe \enquote*{Te enseñaré
    diferencias}, y Anscombe ciertamente compartía la creencia de Wittgenstein
  de que pasar por encima de las diferencias era una de las principales fuentes
  de error en la filosofía.}

Efectivamente esta preocupación por el modo específico de confrontar un problema
filosófico ocupa un lugar importante en \emph{Investigaciones Filosóficas}. En
el \S89 se encuentra una referencia al texto antes citado de las
\emph{Confesiones} para describir la peculiaridad de las preguntas filosóficas:
\blockquote[{\cite[\S89]{wittgenstein1953phiinv}}: Augustine says in
\emph{Confessions} XI. 14, \enquote{quid est ergo tempus? si nemo ex me quaerat
  scio; si quaerenti explicare velim nescio}. --This could not be said about a
question of natural science (\enquote{What is the specific gravity of hydrogen},
for instance). Something that one knows when nobody asks one but no longer knows
when one is asked to explain it, is something that has to be \emph{called to
  mind}. (And it is obviously something which, for some reason, it is difficult
to call to mind.)]{Agustín dice en \emph{Confesiones} XI. 14, \enquote{quid est
    ergo tempus? si nemo ex me quaerat scio; si quaerenti explicare velim
    nescio}. ---Esto no podría ser dicho de una pregunta propia de la ciencia
  natural (\enquote{Cuál es la gravedad específica del hidrógeno}, por ejemplo).
  Algo que uno conoce cuando nadie le pregunta pero que no conoce ya cuando
  alguien pide que lo explique, es algo que tiene que \emph{ser traído a la
    mente}. (Y esto es obviamente algo que, por algún motivo, es dificil de
  traer a la mente.)}
Para Ludwig es de gran importancia atender el paso que damos para resolver
la perplejidad causada por el reclamo de explicar un fenómeno. El deseo de
aclararlo nos puede impulsar a buscar una explicación dentro del fenómeno mismo,
o como él diría:
\blockquote[{\cite[\S90]{wittgenstein1953phiinv}}: We feel as if we had to see
right into phenomena.]{Nos sentimos como si tuviéramos que mirar directamente
hacia dentro de los fenómenos}.
Esta predisposición nos puede conducir a ignorar la amplitud del modo en que el
lenguaje es empleado en la actividad humana para hablar de lo que se investiga y
a enfocarnos sólo en un elemento particular del lenguaje sobre este fenómeno y
tomarlo como un ejemplo paradigmático para construir un modelo abstrayendo
explicaciones y generalizaciones sobre él. Esta manera de indagar, le parece a
Wittgenstein, nos hunde cada vez más profundamente en un estado de frustración y
confusión filosófica de modo que llegamos a imaginar que para alcanzar claridad:
\blockquote[{\cite[\S106]{wittgenstein1953phiinv}}: we have to describe extreme
subtleties, which again we are quite unable to describe with the means at our
disposal. We feel as if we had to repair a torn spider's web with our fingers.
]{tenemos que describir sutilezas extremas, las cuales una vez más somos
  bastante incapaces de describir con los medios que tenemos a nuestra
  disposición. Sentimos como si tuvieramos que reparar una telaraña rota usando
  nuestros dedos.}

La alternativa que Wittgenstein propone es una investigación que no esté
dirigida hacia dentro del fenómeno, sino
\blockquote[{\cite[\S90]{wittgenstein1953phiinv}}: as one might say, towards the
\emph{`possibilities'} of phenomena. What that means is that we call to mind the
\emph{kinds of statement} that we make about phenomena.]{como se podría decir,
  hacia \enquote{\emph{posibilidades}} de fenómenos. Lo que eso significa es que
  traemos a la mente los \emph{tipos de afirmaciones} que hacemos acerca de los
  fenómenos.}
Este tipo de investigación la denomina \enquote{gramatical} y la describe
diciendo:
\blockquote[{\cite[\S90]{wittgenstein1953phiinv}}: Our inquiry is therefore a
grammatical one. And this inquiry sheds light on our problem by clearing
misunderstandings away. Misunderstandings concerning the use of words, brought
about, among other things, by certain analogies between the forms of expression
in different regions of our language. --- Some of them can be removed by
substituting one form of expression for another; this may be called
\enquote{analysing} our forms of expression, for sometimes this procedure
resembles taking things apart.]{Por tanto nuestra investigación es una
  gramatical. Y esta investigación arroja luz sobre nuestro problema al despejar
  los malentendidos. Malentendidos concernientes al uso de las palabras,
  suscitados, entre otras cosas, por ciertas analogías entre las formas de
  expresión en diferentes regiones de nuestro lenguaje. --- Algunos de éstos
  pueden ser eliminados por medio de sustituir una forma de expresión por otra;
  esto puede ser llamado \enquote{analizar} nuestras formas de expresión, puesto
  que a veces este procedimiento se parece a desarmar algo.}

El modo de salir de nuestra perplejidad, por tanto, consiste en prestar
cuidadosa atención al uso que hacemos de hecho con las palabras y la aplicación
que empleamos de las expresiones. Esto está al descubierto en nuestro uso del
lenguaje de modo que la dificultad para \emph{traer a la mente} aquello que
aclare un fenómeno no está en descubrir algo oculto en éste, sino en aprender a
valorar lo que tenemos ante nuestra vista:
\blockquote[{\cite[\S129]{wittgenstein1953phiinv}}: The aspects of things that
are most important for us are hidden because of their simplicity and
familiarity. (One is unable to notice something --- because it is always before
one's eyes.)]{Los aspectos de las cosas que son más importantes para nosotros
  están escondidos por su simplicidad y familiaridad. (Uno es incapaz de notar
  algo --- porque lo tiene siempre ante sus ojos.)}
La descripción de los hechos concernientes al uso del lenguaje en nuestra
actividad humana ordinaria componen los pasos del tipo de investigación sugerido
por Wittgenstein. Hay cierta insatisfacción en este modo de proceder, como él
mismo afirma:
\blockquote[{\cite[\S118]{wittgenstein1953phiinv}}: Where does this
investigation get its importance from, given that it seems only to destroy
everything interesting: that is, all that is great and important? (As it were,
all the buildings, leaving behind only bits of stone and rubble.) But what we
are destroying are only houses of cards, and we are clearing up the ground of
language on which they stood.]{¿De dónde esta investigación adquiere su
  importancia, dado que parece solo destruir todo lo interesante: esto es, todo
  lo que es grandioso e importante? (Por así decirlo, todos los edificios,
  dejando solamente pedazos de piedra y escombros.) Pero lo que estamos
  destruyendo son solo casas de naipes, y estamos despejando el terreno del
  lenguaje donde estaban erigidas.}


estas cosas no solamente se encuentran en W
sino que como se ha dicho están en anscombe


Anscombe, al igual que Wittgenstein, no se limita a emplear un sólo método para
hacer filosofía, como afirma el mismo Wittgenstein:

\blockquote[{\cite[\S133]{wittgenstein1953phiinv}}: There is not a single
philosophical method, though there are indeed methods, different therapies as it
were]{No hay un solo método filosófico, aunque ciertamente hay métodos,
  diferentes terapias por así decirlo}


Sin embargo si atendemos a su modo de hacer filosofía podemos encontrarla
empleando lenguajes o juegos de lenguaje imaginarios para arrojar luz sobre
modos actuales de usar el lenguaje o esquemas conceptuales; del mismo modo su
trabajo esta lleno de ejemplos donde la encontramos examinando con detenimiento
el uso que de hecho hacemos del
lenguaje.\autocite[Cf.~][228--229]{teichmann2008ans} Es visible en ella ese

\blockquote[{\cite[xix]{anscombe2011plato}}: There is however a somehow
chracteristically Wittgenstenian way of countering the philosopher's tendency to
explain a philosophically puzzling thing by inventing an entity or event which
causes it, as physicists invent particles like the graviton.]{modo
  característicamente Wittgensteniano de rebatir la tendencia del filósofo de
  explicar alguna cuestión filosóficamente enigmática inventando una entidad o
  evento que la causa, así como los físicos inventan partículas como el
  gravitón.}

Según el título de este trabajo ha prometido, el análisis sobre el testimonio
que será expuesto es el que se encuentra desarrollado en el pensamiento de
Elizabeth Anscombe. La pregunta planteada al inicio: ¿qué es conocer una verdad
para la vida por el testimonio de la Escritura?, entendida como investigación
filosófica, será examinada en las descripiciones que Anscombe realiza sobre el
modo de usar el lenguaje sobre el creer, la confianza, la verdad, la fe y otros
fenómenos relacionados con el conocer por testimonio. Nuestro título adiverte
además que ésta es una investigación en perspectiva teólogica, cabe
inmendiatamente añadir algo breve al respecto.

¿Qué es teología?, se preguntaba Joseph Ratzinger en su alocución en el 75
aniversario del nacimiento del cardenal Hermann Volk en 1978, e introducía
suscintamente su respuesta a esa pregunta tan grande diciendo:

\citalitlar{Cuando se intenta decir algo sobre esta materia, precisamente como
  tributo al cardenal Volk y a su pensamiento, se asocian, poco menos que
  automáticamente, dos ideas. Me viene a las mientes, por un lado, su divisa (y
  título de uno de sus libros): \emph{Dios todo en todos}, y el programa
  espiritual contenido en ella; por otra parte, se aviva el recuerdo de lo que
  ya antes se ha insinuado: un modo de interrogar total y absolutamente
  filosófico, que no se detiene en reales o supuestas comprobaciones históricas,
  en diagnósticos sociológicos o en técnicas pastorales, sino que se lanza
  implacablemente a la busqueda de los fundamentos.\\
  Según esto, cabría formular ya dos tesis que pueden servirnos de hilo
  conductor para nuestro interrogante sobre la esencia de la teología:\\
  1. La teología se refiere a Dios.\\
  2. El pensamiento teológico está vinculado al modo de cuestionar filosófico
  como a su método fundamental.\footnote{teoría de los principios teológicos, p
    380}} Esta investigación sobre el testimonio como parte de la vida de la
Iglesia será realizada atendiendo al modo de cuestionar filosófico realizado por
Elizabeth Anscombe como método, examinando esta experiencia en referencia a
Dios, es decir, como vivencia de su ser y de su obrar.

Hasta aquí simplemente se ha descrito un modo de andar a través de la discusión
acerca de la categoría del testimonio atendiendo el hecho de que tanto la
temática como la figura de Anscombe otorgan a este camino peculiaridades que hay
que tener en cuenta. Siendo concientes de estas particularidades podríamos ahora
ampliar más el horizonte respecto de dos cuestiones brevemente expuestas
anteriormente. En primer lugar es necesario ampliar la descripción hecha hasta
aquí del fenómeno del testimonio en la vida de la Iglesia, ya que aunque nos
resulte familiar relacionarlo con el testimonio de la Sagrada Escritura, tanto
en el Magisterio de la Iglesia como en la propia Escritura se haya presente la
categoría del testimonio con una riqueza que merece la pena explorar. En segundo
lugar habría que detallar todavía mejor lo problemático del testimonio, sobre
todo cuando se considera su importancia en la transmisión de la fe y el anuncio
del Evangelio en el mundo.


% El segundo apartado examina el uso que se hace en la escritura de la categoría
% del testimonio. Fundamentalmente enfatiza el uso del testimonio como
% estructura de la Revelación misma
% LA CATEGORÍA DEL TESTIMONIO EN LA SAGRADA ESCRITURA
%\section{El testimonio en la Escritura: la Revelación como acción testimonial de Dios}

\subsection{La Revelación descrita con estrucura testimonial}

La Iglesia de hoy, como María, conserva el Evangelio meditándolo en su corazón (Cf. Lc 2,19). Así está presente en el centro de la comunidad creyente el anuncio de Cristo vivo como fundamento de su esperanza en cada etapa de la historia. Este motivo de esperanza conservado es también compartido y expresado, según la enseñanza del apóstol: \blockquote[][\,(1Pe 3,15)]{glorificad a Cristo en vuestros corazones, dispuestos siempre a dar explicación a todo el que os pida una razón de vuestra esperanza}.

Este Evangelio atesorado como fundamento en el centro de la vida de la comunidad eclesial, así como Buena Nueva proclamada y transmitida en el tiempo y en el mundo puede ser comprendido como tres testimonios que son uno: \enquote*{Palabra vivida en el Espíritu}\footnote{\Cite[Cf.][110]{latourelle1975et}: \enquote{Car c'est L'Esprit qui posse l'Eglise à poursuivre son oeuvre d'évangelisation; c'est l'Esprit qui inspire la foi, la nourrit et l'approfondit. C'est l'Esprit qui relie entre eux ces trois témoignages qui n'en font qu'un: celui de la parole-vécue-dans-l'Esprit. Par son témoignage, l'Esprit intériorise le témoignage extérieur de la Bonne Nouvelle du salut en Jésus-Christ et le porte à l'accomplissement de la foi, qui est la réponse d'amour de l'humanité à l'appel d'amour du Père par le Christ.} Ver también \Cite[582]{ninot2009tf}, donde este triple testimonio sirve para orientar la reflexión sobre el testimonio como vía empírica de la credibilidad de la Iglesia.}.

La Evangelización puede ser entendida en este sentido como testimonio de la `palabra de vida' (1Jn 1,1) que los apóstoles anuncian como testigos de lo que han contemplado y palpado. Es también el testimonio de los cristianos que, acogiendo esta palabra, la viven, poniendo por obra lo que ella enseña. Es además testimonio del Espíritu Santo que interioriza el testimonio externo de la Buena Noticia y lo lleva al cumplimiento de la fe en cada persona\footcite[Cf.][110]{latourelle1975et}. Es el Espíritu el que santifica y fecunda la acción de los cristianos, es tambíen el que impulsa y sostiene la acción de la Iglesia; es el Espíritu el que inspira la fe, la nutre y la profundiza\footcite[Cf.][110]{latourelle1975et}.

Este dinamismo fundamental que puede encontrarse vivo hoy en la comunidad de la Iglesia ha actuado en ella desde su origen y le ha acompañado en cada época. Según esto es posible valorar lo que se transmite en la tradición eclesial como la perpetuación de la actividad de Cristo y los apóstoles, que es a su vez proyección del testimonio divino\footnote{\Cite[Cf.][573]{ninot2009tf}: \enquote{el testimonio divino se proyecta luego en el apostólico y se perpetúa en el testimonio eclesial. Por eso, el testimonio es revelación en la actividad de Cristo y de los apóstoles y es transmisión de la revelación en la tradición eclesial}.}.

En la actividad de Cristo el testimonio divino queda proyectado como interpelación a la libertad realizada por la identidad propia de Jesús: \blockquote[][\,(Jn 4,10)]{Si conocieras el don de Dios y quién es el que te dice <<dame de beber>> le pedirías tu, y él te daría agua viva}; \blockquote{``¿Crees tú en el Hijo del hombre?''\textelp{} ``¿Y quién es, Señor, para que crea en él?''\textelp{} ``Lo estás viendo: el que te está hablando, ese es''} (Jn 9,35-37). En la actividad apostólica, el testimonio divino sigue interpelando la libertad humana como manifestación de Jesús Resucitado. Los apóstoles actúan como testigos de los acontecimientos de la Pascua de Jesús y su valor salvífico\footcite[Cf.][576]{ninot2009tf} y este testimonio es descrito como acción del Espíritu que impulsa la tarea apostólica y que da nueva vida a los que acogen el anuncio de la Buena Noticia.

Puede encontrarse un ejemplo en el testimonio de Felipe. El apóstol sale más allá de Jerusalén hacia Samaria, y todavía llega más lejos, al compartir la Buena Noticia de Jesús con un extranjero etíope: \blockquote[][\,(Hch 8, 29-39)]{El Espíritu dijo a Felipe: <<Acércate y pégate a la carroza>>. Felipe se acercó corriendo, le oyó leer el profeta Isaías, y le preguntó: <<¿Entiendes lo que estás leyendo?>>. Contestó: <<¿Y cómo voy a entenderlo si nadie me guía?>>. E invitó a Felipe a subir y a sentarse con él. El pasaje de la Escritura que estaba leyendo era este: \emph{Como cordero fue llevado al matadero, como oveja muda ante el esquilador, así no abre su boca. En su humillación no se le hizo justicia. ¿Quién podrá contar su descendencia? Pues su vida ha sido arrancada de la tierra.} El eunuco preguntó a Felipe: <<Por favor, ¿de quién dice esto el profeta?; ¿de él mismo o de otro?>>. Felipe se puso a hablarle y, tomando pie de este pasaje, le anunció la Buena Nueva de Jesús. Continuando el camino, llegaron a un sitio donde había agua, y dijo el eunuco: <<Mira, agua. ¿Qué dificultad hay en que me bautice?>>. Mandó parar la carroza, bajaron los dos al agua, Felipe y el eunuco, y lo bautizó. Cuando salieron del agua, el Espíritu del Señor arrebató a Felipe. El eunuco no volvió a verlo, y siguió su camino lleno de alegría}. Además de ser ejemplo de la actividad apostólica, este relato puede servir como síntesis del modo en que la categoría del testimonio está presente en la Escritura.

El testimonio comienza con la iniciativa de Dios mismo que impulsa tanto la palabra profética del Antiguo Testamento como el anuncio apostólico del Nuevo Testamento. Esta iniciativa de Dios tiende hacia el testimonio de la Palabra definitiva del Padre que es Cristo resucitado. En aquellos que creen en el testimonio de Dios se engendra alegría y vida nueva. En palabras de R. Latourelle: \blockquote[{\Cite[1530]{latourelle2000testimonio}}.]{En el trato de las tres personas divinas con los hombres existe un intercambio de testimonios que tiene la finalidad de proponer la revelación y de alimentar la fe. Son tres los que revelan o dan testimonio, y esos tres no son más que uno. Cristo da testimonio del Padre, mientras que el Padre y el Espíritu dan testimonio del Hijo. Los apóstoles a su vez dan testimonio de lo que han visto y oído del verbo de la vida. Pero su testimonio no es la comunicación de una ideología, de un descubrimiento científico, de una técnica inédita, sino la proclamación de la salvación prometida y finalmente realizada}.

De este modo el anuncio del apóstol Felipe sirve aquí como un ejemplo específico del testimonio, que ilustra una noción que \blockquote[{\Cite[109]{prades2015testimonio}}.]{atraviesa toda la Escritura y se corresponde con la estructura misma de la revelación}. El testimonio está presente a lo largo de la Escritura junto a otras categorías como pueden ser la de `alianza', `palabra', `paternidad' o `filiación', como parte del \blockquote[{\Cite[1523]{latourelle2000testimonio}}.]{grupo de analogías empleadas por la Escritura para introducir al hombre en las riquezas del misterio divino}.

Esta clave servirá para dar enfoque a un examen sobre la categoría del testimonio en la Escritura. ¿Qué nos dice el Antiguo y el Nuevo Testamento de la revelación como acto testimonial de Dios? Esta pregunta supone que la revelación comparte los rasgos de la actividad humana que es el testimonio, sin embargo, como Latourelle adiverte: \blockquote[{\Cite[1526]{latourelle2000testimonio}}.]{globalmente se puede decir que el testimonio bíblico asume, pero al mismo tiempo exalta hasta sublimarlos, los rasgos del testimonio humano}.

Cabe añadir una última consideración. La revelación de Dios entendida como acto testimonial suyo tiene como expresión definitiva el misterio pascual de Cristo\footnote{\Cite[128]{prades2015testimonio}: \enquote{el misterio pascual al cual tiende toda la existencia terrena de Cristo, constituye el acto testimonial por excelencia de Dios}.}. Este misterio ocupa el lugar principal en el testimonio bíblico: \blockquote[{\Cite[404]{ninot2009tf}}.]{la Resurrección como ``final'' de la unicidad del acontecimiento de Jesucristo, encarnado, muerto y resucitado, subraya específicamente la definitividad de la existencia humana salvada por Dios en la carne de Jesús de Nazaret, ya que la autocomunicación de Dios ha alcanzado su palabra última en la Resurrección de Jesucristo, y por eso es prenda de la resurrección de todos los hombres}. Como tal, parece justo tratar el testimonio que es el misterio pascual en un apartado propio. Y será este precisamente el punto de partida para la descripción de la categoría del testimonio en la Escritura.

\subsection{El testimonio en el misterio y anuncio pascual}

\enquote*{Cristo ha resucitado} (Cf. 1Tes 4,15; 1Cor 15,12-20; Rom 6,4) es la confesión que está en el núcleo del anuncio más primitivo del evangelio\footcite[Cf.][403]{ninot2009tf}. Creer en esta noticia conlleva acoger la manifestación más plena de la Revelación y la motivación más definitiva para creer. En este sentido: \blockquote[{\Cite[405]{ninot2009tf}}.]{La Resurrección de Jesús mirada desde la perspectiva de la teología fundamental presupone un estatuto epistemológico peculiar, puesto que es el punto culminante y objeto de la Revelación y, a su vez, es su acreditación suprema y máximo motivo de credibilidad, tal como recuerda el texto citado de Pablo ``si Cristo no ha resucitado, nuestra predicación es vana y vana es nuestra fe'' (1 Cor 15,14)}.

El misterio pascual no aparece como desconectado del conjunto de la vida y misión de Jesús, sino que hacia él tienden sus obras y palabras desde el comienzo. Cristo pasó por el mundo haciendo el bien, como testimonio de la bondad de Dios, y esta acción va orientada a ese punto culminante que es su pasión, muerte y resurrección; \blockquote[{\Cite[127]{prades2015testimonio}}.]{el testimonio que Jesús va ofreciendo durante su vida pública le va a reclamar una entrega definitiva a favor de los que lo han acogido y frente a la resistencia que ha generado en quienes le rechazan}.

A lo largo de este camino Jesús manifiesta su confianza en el Padre: \blockquote[][\,(Jn 11,41b-42a)]{Padre, te doy gracias porque me has escuchado; yo sé que tu me escuchas siempre}; esta relación queda afirmada plenamente ante la pasión como confianza puesta en su voluntad: \blockquote[][\,(Lc 22,42)]{Padre \textelp{} que no se haga mi voluntad, sino la tuya}. De este modo en el misterio pascual queda atestiguada la plena unidad de Cristo con el Padre, en la mayor confianza imaginable\footcite[Cf.][127]{prades2015testimonio}.

A lo largo de su misión, Cristo dio testimonio del amor del Padre \blockquote[][\,(Jn 13,1)]{habiendo amado a los suyos que estaban en el mundo\ldots}. En el misterio pascual, donde \blockquote[][\,(ibíd.)]{los amó hasta el extremo}, queda confirmado definitivamente como testigo del Padre. Con su entrega ofrece el testimonio pleno del amor salvador del Padre: \blockquote[][\,(Jn 3,16)]{Porque tanto amó Dios al mundo, que entregó a su Unigénito, para que todo el que cree en él no perezca, sino que tenga vida eterna}.

A lo largo de su vida, Cristo también es testigo de la necesidad del camino salvífico ofrecido como decisión trinitaria libre e irrevocable de redimir a los hombres\footcite[Cf.][128]{prades2015testimonio}. \blockquote[][\,(Mc 8,31)]{El hijo del hombre tiene que padecer mucho, ser reprobado por los ancianos, sumos sacerdotes y escribas, ser ejecutado y resucitar a los tres días}. Este testimonio de la voluntad divina es comprendido por los discípulos a la luz del Resucitado; \blockquote[][\,(Cf. Lc 24,45-47a)]{les abrió el entendimiento para comprender las Escrituras \textelp{} ``así está escrito: el Mesías padecerá, resucitará de entre los muertos al tercer día y en su nombre se proclamará la conversión''}.

La intencionalidad de este testimonio que Jesús ofrece a lo largo de su vida hasta llegar al acto testimonial definitivo de Dios al mundo que es el misterio pascual aparece con claridad en la respuesta de Cristo a Pilato antes de la Pasión: \blockquote[][\,(Jn 18,37)]{Yo para esto he nacido y para esto he venido al mundo: para dar testimonio de la verdad. Todo el que es de la verdad escucha mi voz}. En su vida pública y en su misión Cristo ha actuado como profeta que anuncia la verdad; da a conocer al Padre, a quien nadie ha visto nunca, pero que el Hijo sí conoce\footnote{Cf. Jn 1,18; Ver también \Cite[28]{ratzinger2007jdenaz}: \enquote{En Jesús se cumple la promesa del nuevo profeta. En Él se ha hecho plenamente realidad lo que en Moisés era sólo imperfecto: Él vive ante el rostro de Dios no sólo como amigo, sino como Hijo; vive en la más íntima unidad con el Padre.}}. En el misterio pascual Jesús se manifiesta como verdadero profeta, acreditado por el hecho mismo de la Resurrección donde se ha realizado en él mismo lo que ha revelado y prometido\footcite[128]{prades2015testimonio}.

La resurrección de Cristo no sólo acredita su propio testimonio, sino que sostiene el testimonio apostólico. Si Cristo no ha resucitado sería vana cualquier argumentación, sin embargo, Jesús es `el Viviente', estuvo muerto, pero vive por los siglos de los siglos (Cf. Ap 1,17-18).

Los apóstoles son testigos de la vida de Cristo, de sus palabras y acciones, muerte y resurrección. De tal modo, son testigos en continuidad con el testimonio de Cristo. El testimonio apostólico es un anuncio de estos hechos que ellos conocen y cuyo valor han reconocido por la fe. Así Pedro proclama estas cosas el día de Pentecostés: \blockquote[][\,(Hch 2,32)]{A este Jesús lo resucitó Dios, de lo cual todos nosotros somos testigos}. El apóstol es testigo en la fe sobre un acontecimiento enraizado en la historia\footcite[Cf.][402; 406]{ninot2009tf}.

Así mismo es presentado el testimonio de Pedro en casa de Cornelio donde el centurión y todos lo que lo acompañaban esperaban reunidos para escuchar lo que el Señor quisiera comunicarles por medio del apóstol. Pedro, comprendiendo que la verdad de Dios no hace acepción de personas, narra los hechos que él bien conoce: \blockquote[][\,(Hch 10,37-41)]{Vosotros conocéis lo que sucedió en toda Judea, comenzando por Galilea, después del bautismo que predicó Juan. Me refiero a Jesús de Nazaret, ungido por Dios con la fuerza del Espíritu Santo, que pasó haciendo el bien y curando a todos los oprimidos por el diablo, porque Dios estaba con él. Nosotros somos testigos de todo lo que hizo en la tierra de los judíos y en Jerusalén. A este lo mataron, colgándolo de un madero. Pero Dios lo resucitó al tercer día y le concedió la gracia de manifestarse, no a todo el pueblo, sino a los testigos designados por Dios: a nosotros, que hemos comido y bebido con él después de su resurrección de entre los muertos}. Este testimonio de los hechos es iluminado en su sentido profundo porque Pedro conoce a Jesús a quien los apóstoles y el pueblo vieron y escucharon, y que es ahora juez de vivos y muertos: \blockquote[][\,(Hch 10,42-43)]{Nos encargó predicar al pueblo, dando solemne testimonio de que Dios lo ha constituido juez de vivos y muertos. De él dan testimonio todos los profetas: que todos los que creen en él reciben, por su nombre, el perdón de los pecados}.

El apóstol entiende estos hechos y su alcance religioso y salvífico interpretándolos en continuidad con la voluntad de Dios manifestada en su acción en favor del pueblo judío a quién habló por medio de los profetas; voluntad hecha manifiesta en \blockquote[][\,(Hch 2,22)]{Jesús el Nazareno, varón acreditado por Dios ante vosotros con los milagros, prodigios y signos que Dios realizó por medio de él, como vosotros mismos sabéis}.

Este anuncio es experiencia del Resucitado que comió y bebió con ellos; él mismo se apareció a los que él quiso dando testimonio de su resurrección. \blockquote[{\Cite[129]{prades2015testimonio}}.]{Cristo glorificado manifiesta su verdad a los que él quiere y esta manifestación es simultáneamente testimonio de su identidad y testimonio de que él es la Vida (1Jn 5,11)}.

El misterio divino que se manifiesta en la Pascua de Jesús no deja de expresarse en el anuncio pascual realizado por los apóstoles. Ellos son testigos de un hecho enraizado en la historia, que tiene un alcance religioso y salvífico y que es interpretado desde la voluntad de Dios manifestada en los hechos y palabras de Cristo. Sin las obras que Jesús realizó, el testimonio apostólico se derrumba, no existe\footcite[Cf.][1529]{latourelle2000testimonio}. Sin la vida y obra, muerte y resurrección de Jesús \blockquote[][\,(1Cor 15,15)]{resultamos unos falsos testigos de Dios, porque hemos dado testimonio contra él, diciendo que ha resucitado a Cristo, a quien no ha resucitado}.

En Cristo, testigo acreditado por su Resurrección, encuentra su cumplimiento la promesa hecha al pueblo de Israel: \blockquote[Dt 18,15 y Hch 3,22; {\Cite[Cf.][24-29]{ratzinger2007jdenaz}}.]{El Señor, tu Dios, te suscitará de entre los tuyos, de entre tus hermanos, un profeta como yo. A él lo escucharéis}. Así como el misterio pascual y su anuncio no están desconectados de la vida de Cristo, tampoco lo están de la acción salvadora de Dios en el AT. Como veremos, el misterio divino se manifiesta a un pueblo que también está llamado a dar testimonio, reconociendo desde la confianza en Dios el valor salvífico de los sucesos de su historia.

\subsection{La acción testimonial de Dios en el Antiguo Testamento}

En el Antiguo Testamento la Revelación también puede ser comprendida como el `intercambio de testimonios' que existe en el trato de Dios con los hombres\footcite[Cf.][1530]{latourelle2000testimonio}. También aquí la acción testimonial divina se despliega de diversos modos. En la vida del pueblo de la alianza YHWH da testimonio de sí a través de la creación, la ley y, de modo eminente, en personas elegidas y enviadas por él\footcite[Cf.][114-115]{prades2015testimonio}. Esta manifestación divina implica como testigo al mismo pueblo, hacia quien ha sido dirigida la voz del Señor.

La literatura sapiencial recoge la profundización en la experiencia de Dios que ha tenido el pueblo de Israel. En ella se describe el acceso posible al conocimiento de Dios a partir de los bienes visibles o de sus obras: \blockquote[][\,(Sab 13,1-5)]{Son necios por naturaleza todos los hombres que han ignorado a Dios y no han sido capaces de conocer al que es a partir de los bienes visibles, ni de reconocer al artífice fijándose en sus obras, sino que tuvieron por dioses al fuego, al viento, al aire ligero, a la bóveda estrellada, al agua impetuosa y a los luceros del cielo, regidores del mundo. Si, cautivados por su hermosura, los creyeron dioses, sepan cuánto los aventaja su Señor, pues los creó el mismo autor de la belleza. Y si los asombró su poder y energía, calculen cuánto más poderoso es quien los hizo, pues por la grandeza y hermosura de las criaturas se descubre por analogía a su creador}.

El Dios que puede ser reconocido por analogía en el asombro y belleza de las criaturas es un Dios personal que concede sabiduría al piadoso: \blockquote[][\,(Eclo 43,32-3)]{Aún quedan misterios mucho más grandes: tan solo hemos visto algo de sus obras. Porque el Señor lo ha hecho todo y a los piadosos les ha dado la sabiduría}. Esta sabiduría es justicia y raíz de inmortalidad: \blockquote[][\,(Sab 15,1-3)]{Pero tú, Dios nuestro, eres bueno y fiel, eres paciente y todo lo gobiernas con misericordia. Aunque pequemos, somos tuyos y reconocemos tu poder, pero no pecaremos, sabiendo que te pertenecemos. Conocerte a ti es justicia perfecta y reconocer tu poder es la raíz de la inmortalidad}. En este sentido la misma creación es acto testimonial de Dios donde se comunica su misterio y la vida que Él ofrece.

YHWH también aparece en el Antiguo Testamento como testigo de los mandamientos contenidos en la Ley\footcite[Cf.][115]{prades2015testimonio}. Esta queda grabada en las `tablas del testimonio' y confiadas a Moisés: \blockquote[][\,(Ex 31,18)]{Cuando acabó de hablar con Moisés en la montaña del Sinaí, le dio las dos tablas del Testimonio, tablas de piedra escritas por el dedo de Dios}. Este testimonio se enfrenta a un pueblo con el corazón extraviado: \blockquote[][\,(Ex 32,19)]{Al acercarse al campamento y ver el becerro y las danzas, Moisés, encendido en ira, tiró las tablas y las rompió al pie de la montaña}. Sin embargo Dios no se detiene ante la dureza del pueblo. Las tablas del testimonio son reconstruidas: \blockquote[][\,(Ex 34,1.27)]{El Señor dijo a Moisés: <<Labra dos tablas de piedra como las primeras y yo escribiré en ellas las palabras que había en las primeras tablas que tú rompiste.>> \textelp{} <<Escribe estas palabras: de acuerdo con estas palabras concierto alianza contigo y con Israel>>}. Moisés, que conoció el nombre misterioso del Señor (Ex 3,13s), y habló con Él como un amigo (Ex 33,11), aparece ante el pueblo como testigo del único Dios, y de su lealtad con el pueblo. Pertenece a aquellos que el Señor elige como testigos suyos en cada etapa de la historia del pueblo de Israel como testimonio suyo y de su fidelidad.

Este es el modo eminente en que el AT describe el testimonio que Dios dirige al pueblo. Los profetas y ungidos por YHWH son testigos del Señor y de su compromiso con el pueblo. La vida totalmente comprometida del profeta expresa tanto a Dios, absoluto que comunica, como su lealtad: \blockquote[{\Cite[116-117]{prades2015testimonio}}.]{es Dios quien da testimonio de sí mismo y de sus obras y designios a través de las personas elegidas, que se comprometen en su integridad como testigos de YHWH incluso hasta la muerte si el testimonio les lleva a ello. Por eso, la autoridad del testimonio no descansa en los testigos, sino en el mismo YHWH, que es quien los escoge y envía}. En tanto que testigos, la acción de estos escogidos puede ser descrita según los rasgos que tiene la actividad humana de dar testimonio, sin embargo la noción de testigo que se aplica a estos elegidos de Dios va más allá de la que encontraríamos en el lenguaje ordinario. La vida del profeta queda comprometida con un testimonio que no le pertenece, sino que \blockquote[{\Cite[118]{prades2015testimonio}}.]{procede de una iniciativa absoluta, en cuanto a su origen y en cuanto a su contenido} puesto que viene de Dios y es testimonio de sí mismo. Aquí la categoría de testimonio significa mas allá de su uso ordinario en la actividad humana y adquiere un sentido religioso como dimensión totalmente nueva\footcite[Cf.][118]{prades2015testimonio}.

El testimonio de YHWH que el profeta proclama con su actividad y el compromiso de su vida implica al pueblo y le hace testigo: \blockquote[][\,(Is 43,8-12)]{Saca afuera a un pueblo que tiene ojos, pero está ciego, que tiene oídos, pero está sordo. Que todas las naciones se congreguen y todos los pueblos se reúnan. ¿Quién de entre ellos podría anunciar esto, o proclamar los hechos antiguos? Que presenten sus testigos para justificarse, que los oigan y digan: es verdad. Vosotros sois mis testigos ---oráculo del Señor---, y también mi siervo, al que yo escogí, para que sepáis y creáis y comprendáis que yo soy Dios. Antes de mí no había sido formado ningún dios, ni lo habrá después. Yo, yo soy el Señor, fuera de mí no hay salvador. Yo lo anuncié y os salvé; lo anuncié y no hubo entre vosotros dios extranjero. Vosotros sois mis testigos ---oráculo del Señor---: yo soy Dios}. El siervo es testigo al que el Señor ha escogido para que el pueblo sepa, crea y comprenda que YHWH es el único Dios verdadero. Al compartir este saber de Dios con el pueblo, estos también están llamados a ser testigos. Ninguna otra nación podría anunciar como ellos lo que YHWH ha hecho para proveer, liberar, salvar.

Así como el profeta, el pueblo es escogido y enviado por YHWH y por medio de él el Señor da testimonio de sí mismo y se propone como quien da sentido y consistencia a toda la realidad humana. Este testimonio tiene importancia social puesto que está llamado a ser proclamado, y esta proclamación implica el compromiso de los actos y la vida del testigo, es decir, del profeta y todo el pueblo\footcite[Cf.][1526-1527]{latourelle2000testimonio}.

El testimonio de Dios a través de personas escogidas por Él en el AT queda constituido por la narración de hechos que acontecen en la historia, estos hechos son interpretados en su valor absoluto y carácter redentor, y son confesados como actuación de Dios en la vida humana\footcite[Cf.][119]{prades2015testimonio}. Esto vuelve a ponernos en conexión con la figura de Cristo como profeta acreditado por su Resurrección y los apóstoles como verdaderos testigos de un hecho enraizado en la historia, confesado desde la fe e interpretado desde la acción de Dios en Jesús. Esta sintonía anticipa lo que se verá a continuación sobre el testimonio en el Nuevo Testamento. En él la acción testimonial de Dios se describe en continuidad con la tradición veterotestamentaria y llegará a su manifestación plena en el misterio pascual.

\subsection{La acción testimonial de Dios en el Nuevo Testamento}

El Evangelio de Mateo enseña que cuando Jesús llegó a Cafarnaún a comenzar su predicación se cumplieron las promesas que Dios había hecho por medio de los profetas. El Reino de los cielos se desvelaba en su cercanía. Allí la vida de los primeros discípulos cambió. El testimonio de Cristo no es sobre cualquier anuncio o cualquier hecho, sino que tiene un valor absoluto. Jesús de Nazaret \blockquote[{\Cite[126]{prades2015testimonio}}.]{no se limita a proponer una cierta inspiración espiritual o un cierto sentido ético para el obrar de la persona o del pueblo, sino que pretende ser radicalmente <<testimonio de la verdad>> (Jn 18,37) de alcance universal}.

Jesús es testimonio de carácter singular, en quien se da a conocer el momento de la plenitud de la salvación, presencia del hombre nuevo y `paradigma universal de humanidad'\footcite[Cf.][279; 290-291]{ninot2009tf}. Este valor universal de la verdad que se comunica en Jesús se desarrolla y se manifiesta en sus acciones concretas: comiendo con los pecadores o sanando a los enfermos es donde se muestra \blockquote[][\,(Cf. Jn 14,6)]{el camino, la verdad y la vida} para todos.

Este testimonio de Cristo, su vida, actos y palabras, fue sometido al juicio de sus contemporáneos. Asombrados porque no enseña como los demás y por las signos que realiza, se cuestionan sobre su autoridad y poder. Entonces Jesús también tiene que ofrecer testimonio de su credibilidad. La respuesta a este juicio del pueblo se halla en su ministerio en sintonía con las Escrituras: \blockquote[][\,(Lc 4,21)]{Hoy se ha cumplido esta Escritura que acabáis de oír}; donde el pueblo puede encontrar la vida y el sentido que buscan: \blockquote[][\,(Jn 5,39-40)]{estudiáis las Escrituras pensando encontrar en ellas vida eterna; pues ellas están dando testimonio de mi, ¡y no queréis venir a mí para tener vida!}. El testimonio de credibilidad de Jesús ante el pueblo se encuentra también en sus obras, que son las obras del Padre y son confirmación y realización de sus enseñanzas: \blockquote[][\,(Jn 10,38)]{Si no hago las obras de mi Padre, no me creáis, pero si las hago, aunque no me creáis a mí, creed a las obras, para que comprendáis y sepáis que el Padre está en mí y yo en el Padre}.

El singular testimonio de Cristo es comunicación de la verdad con valor universal. El testimonio de Cristo es también su actividad e identidad que hacen creíble lo que comunica. De este modo entre lo que Jesús testimonia y la credibilidad que suscita su testimonio hay una circularidad constante: \blockquote[{\Cite[124]{prades2015testimonio}}.]{La pretensión única que encerraba su testimonio resultaba tan exorbitante que hubiera sido inaceptable para los hombres si no fuera porque sus obras, sus palabras y, en rigor, su presencia misma, lo hacían profundamente razonable en su singularidad}.

Acoger el testimonio de Jesús es escuchar la Escritura y creer en las obras del Padre. Sin embargo la palabra de Cristo choca con el odio de aquellos que son hostiles a la verdad y que, rechazando su testimonio, se juzgan a sí mismos\footnote{\Cite[Cf.][1530]{latourelle2000testimonio}.
%: \enquote{Pero la palabra de Cristo choca con la contestación y el odio. Enfrentados con Cristo, los judíos, que representan al conjunto del mundo hostil a la verdad, rechazan su testimonio y se juzgan a sí mismos.}
}. \blockquote[][\,(Jn 15,22-24)]{Si yo no hubiera venido y no les hubiera hablado, no tendrían pecado, pero ahora no tienen excusas de su pecado. El que me odia a mí, odia también a mi Padre. Si yo no hubiera hecho en medio de ellos obras que ningún otro ha hecho, no tendrían pecado, pero ahora las han visto y me han odiado a mí y a mi Padre}.

Jesús es \blockquote[][\,(Jn 1,5)]{la luz que brilla en la tiniebla y la tiniebla no la recibió}. Jesús es el \blockquote[][\,(Jn 1,18)]{unigénito, que está en el seno del Padre, es quien lo ha dado a conocer}. Este testimonio es manifestación de la comunión trinitaria. Cristo revela al Padre y comunica al Espíritu, y su identidad de Hijo es manifestada como acción del Padre y del Espíritu: \blockquote[][\,(Mt 4,16-17)]{Apenas se bautizó Jesús, salió del agua; se abrieron los cielos y vio que el Espíritu de Dios bajaba como una paloma y se posaba sobre él. Y vino una voz de los cielos que decía: ``Este es mi Hijo amado, en quien me complazco''}.

La acción testimonial de Dios que se describe en el Nuevo Testamento está concentrada en la persona de Cristo, y en su relación manifiesta con el Padre y el Espíritu se expresa el testimonio de la Trinidad misma: \blockquote[{\Cite[410]{latourelle1999rev}}. Ver también: {\Cite[131]{prades2015testimonio}}.]{la Escritura describe la actividad reveladora de la trinidad en forma de testimonios mutuos. El Hijo es el testigo del Padre, y como tal se da a conocer a los apóstoles. A su vez, el Padre da también testimonio de que Cristo es el Hijo, por la atracción que produce en las almas, por las obras que da al Hijo para que las realice y sobre todo por la resurrección, testimonio decisivo del Padre en favor del Hijo. El Hijo da testimonio del Espíritu porque promete enviarlo como educador, consolador, santificador. Y el Espíritu viene y da testimonio del hijo porque le recuerda, le da a conocer, descubre la plenitud de sentido de sus palabras, lo insinúa en las almas}. Esta actividad reveladora de la trinidad introduce al ser humano en la comunión trinitaria. Dios trino se comunica al ser humano actuando en su interior, atrayendo, inspirando; también se comunica externamente por las obras que realiza. Esta participación en la comunión divina viene bien expresada en la finalidad del testimonio apostólico: \blockquote[][\,(1Jn 1,3)]{Eso que hemos visto y oído os lo anunciamos, para que estéis en comunión con nosotros y nuestra comunión es con el Padre y con su Hijo Jesucristo}.

Jesús es el fundamento, testigo fiel y veraz para todo tiempo y lugar\footnote{Cf. Ap 1,15; 3,14. Ver también: \Cite[132]{prades2015testimonio}.}. Creer su testimonio es acoger al absoluto en la historia, esta confianza la hace posible el Espíritu: \blockquote[{\Cite{latourelle2000testimonio}}.]{Cristo es, por tanto, el testigo absoluto, el que lleva en sí mismo la garantía de su testimonio. El hombre, sin embargo, no sería capaz de acoger por la fe este testimonio del absoluto, manifestado en la carne y el lenguaje de Jesús, sin una atracción interior (Jn 6,44), que es un don del Padre y un testimonio del Espíritu (1Jn 5,9-10)}.

Aquellos que creen en Cristo no sólo encuentran una respuesta a su búsqueda de vida y de sentido, sino que \blockquote[][\,(Jn 7,38)]{de sus entrañas manarán ríos de agua viva}. Y esto Jesús lo dice \blockquote[][\,(Jn 7,39)]{refiriéndose al Espíritu que habían de recibir los que creyeran en él}. Esta promesa del Espíritu acontece en Pentecostés y sin ese testimonio postpascual del Espíritu quedaría incompleta la comunicación de Dios en el misterio Pascual\footcite[Cf.][135]{prades2015testimonio}. El envío y la acción del Espíritu prometido completa la acción testimonial de Dios: \blockquote[{\Cite[134-135]{prades2015testimonio}}.]{Al haber <<acompañado>> al Hijo en la tierra de una manera singular desde el momento de su unción en el Jordán, que dispone al Hijo ---concebido por obra del Espíritu Santo--- para la misión en la carne, el Espíritu Santo vuelve al Padre portando en sí todo el misterio redentor del Hijo. De este modo, cuando el Resucitado lo envía a la Iglesia, el Espíritu vuelve como Testigo de la verdad completa, que incluye la perfecta glorificación de la carne del Hijo como plenitud de lo humano}.

El Espíritu enviado por Cristo lleva a la verdad plena a los apóstoles: \blockquote[][\,(Jn 16,13)]{cuando venga él, el Espíritu de la verdad, os guiará hasta la verdad plena. Pues no hablará por cuenta propia, sino que hablará de lo que oye y os comunicará lo que está por venir}. Este testimonio del Espíritu completa también el testimonio de los apóstoles: \blockquote[][\,(Jn 15,26-27)]{Cuando venga el Paráclito, que os enviaré desde el Padre, el Espíritu de la verdad, que procede del Padre, él dará testimonio de mí; y también vosotros daréis testimonio, porque desde el principio estáis conmigo}. Ellos han estado desde el principio con Cristo, así son testigos que pueden narrar lo que han visto y oído; su testimonio queda perfeccionado por el Espíritu que les introduce en el misterio del Hijo encarnado y les permite interpretar y comprender la verdad del Hijo, y por este, la del Padre\footcite[Cf.][139]{prades2015testimonio}.

Los que han compartido con Jesús desde el principio son testigos del Evangelio, pero el Resucitado sigue eligiendo apóstoles y en virtud de la acción del Espíritu estos son testigos del mismo misterio\footcite[Cf.][576]{ninot2009tf}. Así Matías no sólo es \blockquote[][\,(Hch 1,21)]{uno de los que nos acompañaron todo el tiempo que convivió con nosotros el Señor Jesús}, sino que es elegido por el Resucitado (Cf. Hch 1,24-26). Igualmente Pablo es constituido testigo por la llamada del Resucitado, así puede decir \blockquote[][\,(1Cor 2,1)]{Yo mismo hermanos cuando vine a vosotros a anunciaros el testimonio de Dios\ldots}. De este modo la transmisión viva del testimonio cristiano esta constituida por un momento fundacional en la convivencia con Jesús y un momento continuante como dos aspectos históricos inseparables.\footcite[Cf.][148]{prades2015testimonio} Este momento continuante está compuesto por los que han sido testigos oculares, y por los que no lo han sido: \blockquote[{\Cite[148]{prades2015testimonio}}.]{unos y otros son elegidos, llamados y enviados por Cristo, el Cristo histórico los primeros y el Cristo glorioso los segundos}. Aquel que recibe este testimonio y cree en él encuentra la vida nueva. ``¿Qué dificultad hay en que me bautice?'', decide aquel hombre que recibió el testimonio de Felipe y ``siguió su camino lleno de alegría'' después de haber encontrado a Dios. Considerar la revelación divina como acción testimonial de Dios conduce en definitiva a estimar la revelación misma como forma de amor y libertad de Dios que interpela el amor y libertad humano. En tanto que comunicación libre y amorosa, el testimonio de Dios atiende la naturaleza humana de su beneficiario; en tanto que don divino queda desvelado su origen y meta más allá de lo humano\footcite[Cf.][152]{prades2015testimonio}.


% El tercer apartado repasa algunos textos del magisterio reciente. El foco es
% el rol que juega la iglesia en la dinámica de la revelación como signo
% sacramental. Esto para unir estructura testimonial y motivo de credibilidad.
% LA CATEGORÍA DEL TESTIMONIO EN EL MAGISTERIO
%\section{Iglesia como signo sacramental el Testimonio en el Magisterio Reciente}

Nuestro recorrido comenzó al inicio de este capítulo tomando como punto de
partida a la Iglesia como signo visible. La vida de la comunidad eclesial, sus
costumbres y actitudes, son presencia histórica y realidad perceptible. La
Iglesia puede ser reconocida hoy actuando según su costumbre de reunirse en
torno a la Palabra de Dios para celebrarla y conocer la verdad para su vida. Lo
que se vive hoy y se ha transmitido en la tradición eclesial lo hemos valorado
como perpetuación de la actividad de Cristo y de los apóstoles y, por tanto,
como proyección del testimonio divino. En este sentido hemos considerado la
presencia de la Revelación divina en el corazón y anuncio de la Iglesia como
triple testimonio usando la expresión de Latourelle: \enquote{palabra vivida en
  el Espíritu}. Esta reflexión ha querido servir para describir la naturaleza de
la Revelación como experiencia familiar en la vida de la Iglesia. La noción de
la categoría del testimonio que atraviesa la escritura ha servido para valorar
la naturaleza de la Revelación según su estructura testimonial.

Así como la categoría del testimonio ha servido para decir algo sobre la
Revelación en la Escritura, ahora se pretende decir algo sobre lo que la
categoría del testimonio puede aportar para comprender la identidad de la
Iglesia y su misión en el mundo y cómo ésta forma parte del dinamismo de la
Revelación divina.

Con Latourelle se ha dicho que el testimonio es una de esas categorías que la
escritura emplea como analogía para introducirnos al misterio divino. El
Concilio nos regala otra analogía que va de la mano con la categoría del
testimonio en la comprensión de la Iglesia y su misión:
\blockquote[LG 8]{la sociedad provista de sus órganos jerárquicos y el Cuerpo
  místico de Cristo, la asamblea visible y la comunidad espiritual, la Iglesia
  terrestre y la Iglesia enriquecida con los bienes celestiales, no deben ser
  consideradas como dos cosas distintas, sino que más bien forman una realidad
  compleja que está integrada de un elemento humano y otro divino. Por eso se la
  compara, por una notable analogía, al misterio del Verbo encarnado, pues así
  como la naturaleza asumida sirve al Verbo divino como de instrumento vivo de
  salvación unido indisolublemente a Él, de modo semejante la articulación
  social de la Iglesia sirve al Espíritu Santo}

La visibilidad de la Iglesia está al servicio del Espíritu Santo de modo que su
naturaleza humana sirve a la presencia divina como instrumento vivo de
salvación. La presencia de la articulación social de la Iglesia actua de manera
análoga a la presencia de Cristo. Según esto se puede decir que
\blockquote[{\cite[566]{ninot2009tf}}]{la eclesiología se resuelve en la
  Cristología y por eso el \enquote{lugar} de la Iglesia en el acto de creer
  será \enquote{análogo} al de Cristo}. Esta relación con Cristo y el Espíritu
otorgan a la Iglesia valor sacramental:
\blockquote[LG 48]{Porque Cristo, levantado sobre la tierra, atrajo hacia sí a
  todos (cf. Jn 12, 32 gr.); habiendo resucitado de entre los muertos (Rm 6, 9),
  envió sobre los discípulos a su Espíritu vivificador, y por El hizo a su
  Cuerpo, que es la Iglesia, sacramento universal de salvación; estando sentado
  a la derecha del Padre, actúa sin cesar en el mundo para conducir a los
  hombres a la Iglesia y, por medio de ella, unirlos a sí más estrechamente y
  para hacerlos partícipes de su vida gloriosa alimentándolos con su cuerpo y
  sangre. Así que la restauración prometida que esperamos, ya comenzó en Cristo,
  es impulsada con la misión del Espíritu Santo y por Él continúa en la Iglesia,
  en la cual por la fe somos instruidos también acerca del sentido de nuestra
  vida temporal, mientras que con la esperanza de los bienes futuros llevamos a
  cabo la obra que el Padre nos encomendó en el mundo y labramos nuestra
  salvación (cf. Flp 2, 12).}
Esta Iglesia que es sacramento es mediación de acción salvadora de Dios;
comunica los dones de la gracia y manifiesta el misterio de Dios:
\blockquote[GS 45]{Todo el bien que el Pueblo de Dios puede dar a la familia
  humana al tiempo de su peregrinación en la tierra, deriva del hecho de que la
  Iglesia es "sacramento universal de salvación", que manifiesta y al mismo
  tiempo realiza el misterio del amor de Dios al hombre.}

La Iglesia en el mundo es así uno de los signos contenidos en la Revelación que
ayudan a la razón que busca la comprensión del misterio. El signo sacramental
que es la Iglesia permite atestiguar desde la fe el misterio de Dios que en ella
se expresa del mismo modo que ocurre con la Eucaristía o la presencia de Cristo
encarnado:
\citalitlar{Podemos fijarnos, en cierto modo, en el horizonte sacramental de la
  Revelación y, en particular, en el signo eucarístico donde la unidad
  inseparable entre la realidad y su significado permite captar la profundidad
  del misterio. Cristo en la Eucaristía está verdaderamente presente y vivo, y
  actúa con su Espíritu, pero como acertadamente decía Santo Tomás, <<lo que no
  comprendes y no ves, lo atestigua una fe viva, fuera de todo el orden de la
  naturaleza. Lo que aparece es un signo: esconde en el misterio realidades
  sublimes>>. A este respecto escribe el filósofo Pascal: <<Como Jesucristo
  permaneció desconocido entre los hombres, del mismo modo su verdad permanece,
  entre las opiniones comunes, sin diferencia exterior. Así queda la Eucaristía
  entre el pan común>>.\footnote{FR 13}}
El misterio sublime que aparece en un signo puede ser atestiguado por la fe
viva. El asentimiento al signo sacramental por la fe supone el reconocimiento de
que viene de Dios y por tanto es creer a quien es garante de su propia verdad.
Este asentimiento implica a la persona por completo:
\citalitlar{Desde la fe el hombre da su asentimiento a ese testimonio divino.
  Ello quiere decir que reconoce plena e integralmente la verdad de lo revelado,
  porque Dios mismo es su garante. Esta verdad, ofrecida al hombre y que él no
  puede exigir, se inserta en el horizonte de la comunicación interpersonal e
  impulsa a la razón a abrirse a la misma y a acoger su sentido profundo. Por
  esto el acto con el que uno confía en Dios siempre ha sido considerado por la
  Iglesia como un momento de elección fundamental, en la cual está implicada
  toda la persona. Inteligencia y voluntad desarrollan al máximo su naturaleza
  espiritual para permitir que el sujeto cumpla un acto en el cual la libertad
  personal se vive de modo pleno.\footnote{FR 13}}
La acogida del misterio divino comunicado en el signo sacramental es así un acto
de libertad plena que no sólo permite reconocer el misterio de Dios, sino que
nos desvela nuestra vocación de comunión con Él, que es nuestro sentido más
auténtico:
\citalitlar{El conocimiento de fe, en definitiva, no anula el misterio; sólo lo
  hace más evidente y lo manifiesta como hecho esencial para la vida del hombre:
  Cristo, el Señor, <<en la misma revelación del misterio del Padre y de su
  amor, manifiesta plenamente el hombre al propio hombre y le descubre la
  grandeza de su vocación>>, que es participar en el misterio de la vida
  trinitaria de Dios.\footnote{FR 13}}

La Iglesia es signo sacramental unido inseparablemente al misterio divino que
comunica, de modo análogo a la unión del Verbo divino y la naturaleza asumida
por Él. El conocimiento de la fe abre la razón humana a la verdad revelada como
comunicación interpersonal de Dios realizada por medio de este signo sacramental
que es la Iglesia. Este acto de confianza es movimiento de la libertad como
asentimiento y elección de Dios que se revela y acogida de su llamada a
participar de la comunión trinitaria. Aquí sacramento y testimonio son
categorías que interactuan para describir el acceso al misterio divino que se
comunica a través de signos. Esta Iglesia que es signo sacramental es signo
creíble por el testimonio de la vida comprometida con el misterio de amor que
significa: \citalitlar{La misión primera y fundamental que recibimos de los
  santos Misterios que celebramos es la de dar testimonio con nuestra vida. El
  asombro por el don que Dios nos ha hecho en Cristo infunde en nuestra vida un
  dinamismo nuevo, comprometiéndonos a ser testigos de su amor. Nos convertimos
  en testigos cuando, por nuestras acciones, palabras y modo de ser, aparece
  Otro y se comunica. Se puede decir que el testimonio es el medio con el que la
  verdad del amor de Dios llega al hombre en la historia, invitándolo a acoger
  libremente esta novedad radical. En el testimonio Dios, por así decir, se
  expone al riesgo de la libertad del hombre. Jesús mismo es el testigo fiel y
  veraz (cf. Ap 1,5; 3,14); vino para dar testimonio de la verdad (cf. Jn
  18,37). Con estas reflexiones deseo recordar un concepto muy querido por los
  primeros cristianos, pero que también nos afecta a nosotros, cristianos de
  hoy: el testimonio hasta el don de sí mismos, hasta el martirio, ha sido
  considerado siempre en la historia de la Iglesia como la cumbre del nuevo
  culto espiritual: <<Ofreced vuestros cuerpos>> (Rm 12,1). Se puede recordar,
  por ejemplo, el relato del martirio de san Policarpo de Esmirna, discípulo de
  san Juan: todo el acontecimiento dramático es descrito como una liturgia, más
  aún como si el mártir mismo se convirtiera en Eucaristía. Pensemos también en
  la conciencia eucarística que san Ignacio de Antioquía expresa ante su
  martirio: él se considera <<trigo de Dios>> y desea llegar a ser en el
  martirio <<pan puro de Cristo>>. El cristiano que ofrece su vida en el
  martirio entra en plena comunión con la Pascua de Jesucristo y así se
  convierte con Él en Eucaristía. Tampoco faltan hoy en la Iglesia mártires en
  los que se manifiesta de modo supremo el amor de Dios. Sin embargo, aun cuando
  no se requiera la prueba del martirio, sabemos que el culto agradable a Dios
  implica también interiormente esta disponibilidad, y se manifiesta en el
  testimonio alegre y convencido ante el mundo de una vida cristiana coherente
  allí donde el Señor nos llama a anunciarlo.\footnote{SCa 85}}
El testimonio hasta el don de nosotros mismos se convierte en signo sacramental,
el cristiano que ofrece su vida por completo como testigo entra en comunión con
la Pascua y se convierte con Cristo en Eucaristía. La vida entregada, este signo
sacramental, es el medio adecuado para comunicar la comunión con Dios:
\citalitlar{En efecto, la fe necesita un ámbito en el que se pueda testimoniar y
  comunicar, un ámbito adecuado y proporcionado a lo que se comunica. Para
  transmitir un contenido meramente doctrinal, una idea, quizás sería suficiente
  un libro, o la reproducción de un mensaje oral. Pero lo que se comunica en la
  Iglesia, lo que se transmite en su Tradición viva, es la luz nueva que nace
  del encuentro con el Dios vivo, una luz que toca la persona en su centro, en
  el corazón, implicando su mente, su voluntad y su afectividad, abriéndola a
  relaciones vivas en la comunión con Dios y con los otros. Para transmitir esta
  riqueza hay un medio particular, que pone en juego a toda la persona, cuerpo,
  espíritu, interioridad y relaciones. Este medio son los sacramentos,
  celebrados en la liturgia de la Iglesia. En ellos se comunica una memoria
  encarnada, ligada a los tiempos y lugares de la vida, asociada a todos los
  sentidos; implican a la persona, como miembro de un sujeto vivo, de un tejido
  de relaciones comunitarias. Por eso, si bien, por una parte, los sacramentos
  son sacramentos de la fe, también se debe decir que la fe tiene una estructura
  sacramental. El despertar de la fe pasa por el despertar de un nuevo sentido
  sacramental de la vida del hombre y de la existencia cristiana, en el que lo
  visible y material está abierto al misterio de lo eterno. \footnote{LF 40}}
Al celebrar los sacramentos con fe viva, la comunidad eclesial se deja implicar
por completo por la luz del Dios vivo que se comunica y el memorial que se
encarna. Despertar a la fe en los sacramentos es también despertar al sentido
sacramental que tiene la propia vida cristiana. Así como en los sacramentos los
signos visibles comunican la luz de Dios, también la propia existencia del
cristiano puede arrojar esa luz.

Este valor sacramental de la vida del cristiano y de la comunidad eclesial hace
de su propia existencia un testimonio kerygmático:
\citalitlar{La Buena Nueva debe ser proclamada en primer lugar, mediante el
  testimonio. Supongamos un cristiano o un grupo de cristianos que, dentro de la
  comunidad humana donde viven, manifiestan su capacidad de comprensión y de
  aceptación, su comunión de vida y de destino con los demás, su solidaridad en
  los esfuerzos de todos en cuanto existe de noble y bueno. Supongamos además
  que irradian de manera sencilla y espontánea su fe en los valores que van más
  allá de los valores corrientes, y su esperanza en algo que no se ve ni osarían
  soñar. A través de este testimonio sin palabras, estos cristianos hacen
  plantearse, a quienes contemplan su vida, interrogantes irresistibles: ¿Por
  qué son así? ¿Por qué viven de esa manera? ¿Qué es o quién es el que los
  inspira? ¿Por qué están con nosotros? Pues bien, este testimonio constituye ya
  de por sí una proclamación silenciosa, pero también muy clara y eficaz, de la
  Buena Nueva. Hay en ello un gesto inicial de evangelización. Son posiblemente
  las primeras preguntas que se plantearán muchos no cristianos, bien se trate
  de personas a las que Cristo no había sido nunca anunciado, de bautizados no
  practicantes, de gentes que viven en una sociedad cristiana pero según
  principios no cristianos, bien se trate de gentes que buscan, no sin
  sufrimiento, algo o a Alguien que ellos adivinan pero sin poder darle un
  nombre. Surgirán otros interrogantes, más profundos y más comprometedores,
  provocados por este testimonio que comporta presencia, participación,
  solidaridad y que es un elemento esencial, en general al primero absolutamente
  en la evangelización.\footnote{EN 21}}
La acción testimonial de Dios que se manifiesta en Cristo y en los sacramentos
instituidos por Él está analogamente presente en la vida comprometida del
cirstiano. El testimonio humano es respuesta de fe de aquellos que han
reconocido a Dios en los signos que le encarnan y que corresponden con palabras
y obras que quieren significar la vida nueva que viene del Señor. En esta
correspondencia se unden las raíces de la misión de proclamar la Buena Nueva.

El testimonio es así acción propia de todo bautizado que ha quedado unido a
Cristo y a la Iglesia.\autocite[Cf.][188]{prades2015testimonio} Toda la Iglesia
tiene la misión de dar testimonio; los obispos ofrecen al mundo el rostro de la
Iglesia con su trato y trabajo pastoral\footnote{GS 43}, los presbíteros,
creciendo en el amor por el desempeño de su oficio, han de ser un vivo
testimonio de Dios\footnote{LG 41}, los fieles han de dar testimonio de verdad
como testigos de la resurrección\footnote{LG 28 y LG 38}, los religiosos deben
ofrecer un testimonio sostenido por la integridad de la fe, por la caridad y el
amor a la cruz y la esperanza de la gloria futura\footnote{PC 25}, los
profesores han de dar testimonio tanto con su vida como con su
doctrina\footnote{GE 8}, los misioneros han de ofrecer testimonio con una vida
enteramente evangélica, con paciencia, longanimidad, suavidad, caridad sincera,
y si es necesario hasta con la propia sangre.\footnote{AG 24}

El signo que es la vida de los cristianos y, por tanto la Iglesia, esta llamado
a purificarse y crecer. La contradicción entre la fe y la vida de los cristianos
puede constituir un motivo de tropiezo, en lugar de dar a conocer la luz de
Dios. El testimonio de la vida entregada, aún cuando ha sido estimado según su
valor sacramental, es un signo imperfecto que debe ser madurado con una actitud
vigilante:
\citalitlar{Aunque la Iglesia, por la virtud del Espíritu Santo, se ha mantenido
  como esposa fiel de su Señor y nunca ha cesado de ser signo de salvación en el
  mundo, sabe, sin embargo, muy bien que no siempre, a lo largo de su prolongada
  historia, fueron todos sus miembros, clérigos o laicos, fieles al espíritu de
  Dios. Sabe también la Iglesia que aún hoy día es mucha la distancia que se da
  entre el mensaje que ella anuncia y la fragilidad humana de los mensajeros a
  quienes está confiado el Evangelio. Dejando a un lado el juicio de la historia
  sobre estas deficiencias, debemos, sin embargo, tener conciencia de ellas y
  combatirlas con máxima energía para que no dañen a la difusión del Evangelio.
  De igual manera comprende la Iglesia cuánto le queda aún por madurar, por su
  experiencia de siglos, en la relación que debe mantener con el mundo. Dirigida
  por el Espíritu Santo, la Iglesia, como madre, no cesa de ``exhortar a sus
  hijos a la purificación y a la renovación para que brille con mayor claridad
  la señal de Cristo en el rostro de la Iglesia''\footnote{GS 34}}
La vida de la Iglesia está marcada por esa llamada a este enriquecimiento
constante. Como afirma DV 8: \citalitinterlin{la Iglesia, en el decurso de los
  siglos, tiende constantemente a la plenitud de la verdad divina, hasta que en
  ella se cumplan las palabras de Dios.}

La categoría del testimonio ha servido para acercarnos a algunos textos
magisteriales y describir la vida de la Iglesia como signo sacramental. A modo
de conclusión son luminosas las palabras de K. Wojtyła:
\citalitlar{El significado del testimonio en la doctrina del Vaticano II es
  explícitamente analógico, puesto que el Concilio habla del testimonio de Dios
  y del hombre, que, de diversa manera, corresponde al divino, y a una respuesta
  multiforme a la revelación. En todo caso, sin embargo, la respuesta es
  testimonio y el testimonio, respuesta. \footnote{Para una discusión más amplia
    de la lectura de Wojtyła véase \cite[194--197]{prades2015testimonio}}}

Este recorrido a través de algunos modos de emplear la categoría del testimonio
en la Escritura y la doctrina magisterial ha servido para describir los
dinamísmos de la Revelación como acción libre y amorosa del Padre encarnada en
en la naturaleza humana asumida por el Verbo y sostenida por la acción interior
del Espíritu. Esta acción de la libertad divina ha encontrado la correspondencia
de la libertad humana que acoge la invitación al amor y se compromete por
completo a la comunión con Dios. Este intercambio testimonial comunica el amor
divino.



% Los tres apartados anteriores quedaron resumidos en este apartado nuevo:
% Carácter testimonial de la Revelación
Teniendo en cuenta que la categoría de testimonio representa un amplio campo de estudio, resulta necesario dedicar los primeros pasos de esta discusión a detallar brevemente el ámbito y alcance de la investigación. En este capítulo introductorio se proponen en términos simples algunas nociones sobre el enfoque y motivaciones de este estudio en cuanto que investigación en perspectiva teológico-fundamental. Tambíen se destacan algunas peculiaridades generales propias del método de la filosofía analítica empleado por Anscombe. Finalmente se detallan propiamente tres cuestiones fundamentales que orientarán el análisis de la categoría de testimonio dentro de la obra de Elizabeth Anscombe.

\section{Estudio del testimonio como clave de comprensión de la Revelación}

Podemos comenzar considerando una ilustración simple. Es una experiencia familiar en nuestras comunidades reunirnos en torno a la Sagrada Escritura y compartir la Palabra buscando en ella luz para nuestro presente. Podemos imaginar un domingo, por ejemplo el tercer domingo del Tiempo Ordinario. En el ciclo A, el Evangelio que se proclama ese día es este:
\blockquote[][\,(Mt~4,12-23)]{Al enterarse Jesús de que habían arrestado a Juan se retiró a Galilea. Dejando Nazaret se estableció en Cafarnaún, junto al mar, en el territorio de Zabulón y Neftalí, para que se cumpliera lo dicho por medio del profeta Isaías: <<Tierra de Zabulón y tierra de Neftalí, camino del mar, al otro lado del Jordán, Galilea de los gentiles. El pueblo que habitaba en tinieblas vio una luz grande; a los que habitaban en tierra y sombras de muerte, una luz les brilló>>. Desde entonces comenzó Jesús a predicar diciendo: <<Convertíos, porque está cerca el reino de los cielos>>. Paseando junto al mar de Galilea vio a dos hermanos, a Simón, llamado Pedro, y a Andrés, que estaban echando la red en el mar, pues eran pescadores. Les dijo: <<Venid en pos de mí y os haré pescadores de hombres>>. Inmediatamente dejaron las redes y lo siguieron. Y pasando adelante vio a otros dos hermanos, a Santiago, hijo de Zebedeo, y a Juan, su hermano, que estaban en la barca repasando las redes con Zebedeo, su padre, y los llamó. Inmediatamente dejaron la barca y a su padre y lo siguieron. Jesús recorría toda Galilea enseñando en sus sinagogas, proclamando el evangelio del reino y curando toda enfermedad y toda dolencia en el pueblo}.

No sería difícil ahora visualizar una variedad de situaciones en las que este texto pueda ser discutido en nuestro contexto eclesial. En enero de 2014 el Papa Francisco lo reflexionaba en el Ángelus en la Plaza de San Pedro y destacaba que la misión de Jesús comenzara en una zona periférica:
\blockquote[{\Cite{francisco2014angelus}}.]{Es una tierra de frontera, una zona de tránsito donde se encuentran personas diversas por raza, cultura y religión. La Galilea se convierte así en el lugar simbólico para la apertura del Evangelio a todos los pueblos. Desde este punto de vista, Galilea se asemeja al mundo de hoy: presencia simultánea de diversas culturas, necesidad de confrontación y necesidad de encuentro. También nosotros estamos inmersos cada día en una <<Galilea de los gentiles>>, y en este tipo de contexto podemos asustarnos y ceder a la tentación de construir recintos para estar más seguros, más protegidos. Pero Jesús nos enseña que la Buena Noticia, que Él trae, no está reservada a una parte de la humanidad, sino que se ha de comunicar a todos. Es un feliz anuncio destinado a quienes lo esperan, pero también a quienes tal vez ya no esperan nada y no tienen ni siquiera la fuerza de buscar y pedir}.

El Papa Benedicto XVI también había ofrecido su comentario unos años antes. En su caso el acento del relato lo encontró en la fuerza singular de esa `buena nueva' que Cristo comenzaba a anunciar:
\blockquote[{\Cite{benedicto2008angelus}}.]{El término <<evangelio>>, en tiempos de Jesús, lo usaban los emperadores romanos para sus proclamas. Independientemente de su contenido, se definían <<buenas nuevas>>, es decir, anuncios de salvación, porque el emperador era considerado el señor del mundo, y sus edictos, buenos presagios. Por eso, aplicar esta palabra a la predicación de Jesús asumió un sentido fuertemente crítico, como para decir: Dios, no el emperador, es el Señor del mundo, y el verdadero Evangelio es el de Jesucristo.

La <<buena nueva>> que Jesús proclama se resume en estas palabras: <<El reino de Dios ---o reino de los cielos--- está cerca>>. ¿Qué significa esta expresión? Ciertamente, no indica un reino terreno, delimitado en el espacio y en el tiempo; anuncia que Dios es quien reina, que Dios es el Señor, y que su señorío está presente, es actual, se está realizando.

Por tanto, la novedad del mensaje de Cristo es que en él Dios se ha hecho cercano, que ya reina en medio de nosotros, como lo demuestran los milagros y las curaciones que realiza}.

Ciertamente este texto no se proclamaría solamente en San Pedro, sino que estaría presente en la celebración de la eucaristía dominical resonando en las comunidades y parroquias alrededor del mundo; en las homilías, oraciones, reflexiones o cánticos, invitando a la conversión y haciendo nueva la invitación de Jesús: \enquote*{Convertíos, porque está cerca el reino de los cielos}. Quizás también se le oiga entre algún grupo juvenil donde Simón, Andrés, Santiago y Juan sean tratados como modelos de vocación a la vida consagrada o al apostolado, atendiendo con entusiasmo cómo lo dejaron todo en el momento para seguir a Jesús. Seguramente algún joven reconocería aquella llamada: \enquote*{Venid en pos de mí y os haré pescadores de hombres} sonando como voz dentro de sí.

Este ejemplo que acabamos de describir sirve de clave para nuestro enfoque. Podemos identificar en el relato de Mateo una síntesis de la dinámica de la Revelación Divina; Dios, por amor, se ha comunicado a sí mismo en Cristo, Verbo encarnado, y nos ha hablado como amigos y nos ha invitado a la comunión con Él (Cf. DV 2). Esta comunicación del Absoluto en la historia se nos describe en el texto evangélico como la llegada de la luz prometida en el anuncio profético, presencia cercana del Reino de Dios, mirada comprensiva y llamada al seguimiento, acción sanadora y palabras que anuncian el Reino. Todo esto realizado en Jesús.

Si atendemos desde esta perspectiva a las reflexiones del Papa Benedicto y del Papa Francisco podemos encontrar en ellas una descripción de esta dinámica de la Revelación como hecho que continúa en nuestro presente: ``Galilea se asemeja al mundo de hoy\ldots'' y el Reino de Dios ``está presente, es actual, se está realizando''. Dios, por amor, sigue comunicándose por medio de Cristo y esta Buena Nueva es anuncio de salvación destinado a todos. Tanto el texto de la Escritura como las palabras de los Pontífices describen esta obra de Dios que implica a la Iglesia: ``también nosotros estamos inmersos cada día en una `Galilea de los Gentiles'\ldots'' y, según la enseñanza de Jesus, esta Buena Noticia ``se ha de comunicar a todos''. Esto también es visible en el modo en el que la Palabra de Dios se celebra, se proclama y se acoge en la Iglesia en todo el mundo.

Incluso en esta descripción general y básica, no es difícil identificar el carácter testimonial que tiene la Revelación. Jesús se comunica como testigo definitivo de Dios y la Iglesia comparte esta misión. En este sentido el análisis de la categoría de testimonio consiste en un modo de acercarnos al hecho de la Revelación. La motivación para este acercamiento específico nos lo da en primer lugar la propia enseñanza de la Escritura y el Magisterio donde el fenómeno de la Revelación divina se ha descrito en clave testimonial. Es desde esta premisa que parte el deseo de analizar el caracter testimonial de la Revelación, o la Revelación en tanto que testimonio divino, para comprender adecuadamente el ser y actuar de Dios y también la misión, vocación e identidad de la Iglesia que es testigo y ha de ser testigo para el mundo de hoy\footnote{En términos generales este acercamiento está orientado por la descripción de la identidad y articulación de la Teología Fundamental propuesta por Salvador Pié-Ninot. \Cite[Cf.][74-85]{ninot2009tf}.}.

Para un exámen detallado de la categoría de testimonio como clave de comprensión de la Revelación y su presencia en la Escritura y el Magisterio referimos a algunos estudios especializados como son J. \textsc{Prades}, \emph{Dar Testimonio}; S. \textsc{Pié-Ninot}, \emph{La Teología Fundamental} y R. \textsc{Latourelle}, \emph{Teología de la Revelación}, <<Évangélisation et témoignage>> en \emph{Evangelisation} y <<Testimonio>> en \emph{Diccionario de Teología Fundamental}. A continuación nos limitamos a destacar algunas claves generales que consideramos relevantes para nuestro estudio.

\section{La Revelación descrita con estrucura testimonial en la Escritura y el Magisterio}

La Iglesia de hoy, como María, conserva el Evangelio meditándolo en su corazón (Cf. Lc 2,19). Así está presente en el centro de la comunidad creyente el anuncio de Cristo vivo como fundamento de su esperanza en cada etapa de la historia. Este motivo de esperanza conservado es también compartido y expresado, según la enseñanza del apóstol: \blockquote[][\,(1Pe 3,15)]{glorificad a Cristo en vuestros corazones, dispuestos siempre a dar explicación a todo el que os pida una razón de vuestra esperanza}.

Este Evangelio atesorado como fundamento en el centro de la vida de la comunidad eclesial, así como Buena Nueva proclamada y transmitida en el tiempo y en el mundo puede ser comprendido como tres testimonios que son uno: \enquote*{Palabra vivida en el Espíritu}\footnote{\Cite[Cf.][110]{latourelle1975et}: \enquote{Car c'est L'Esprit qui posse l'Eglise à poursuivre son oeuvre d'évangelisation; c'est l'Esprit qui inspire la foi, la nourrit et l'approfondit. C'est l'Esprit qui relie entre eux ces trois témoignages qui n'en font qu'un: celui de la parole-vécue-dans-l'Esprit. Par son témoignage, l'Esprit intériorise le témoignage extérieur de la Bonne Nouvelle du salut en Jésus-Christ et le porte à l'accomplissement de la foi, qui est la réponse d'amour de l'humanité à l'appel d'amour du Père par le Christ.} Ver también \Cite[582]{ninot2009tf}, donde este triple testimonio sirve para orientar la reflexión sobre el testimonio como vía empírica de la credibilidad de la Iglesia.}.

La Evangelización puede ser entendida en este sentido como testimonio de la `palabra de vida' (1Jn 1,1) que los apóstoles anuncian como testigos de lo que han contemplado y palpado. Es también el testimonio de los cristianos que, acogiendo esta palabra, la viven, poniendo por obra lo que ella enseña. Es además testimonio del Espíritu Santo que interioriza el testimonio externo de la Buena Noticia y lo lleva al cumplimiento de la fe en cada persona\footcite[Cf.][110]{latourelle1975et}. Es el Espíritu el que santifica y fecunda la acción de los cristianos, es tambíen el que impulsa y sostiene la acción de la Iglesia; es el Espíritu el que inspira la fe, la nutre y la profundiza\footcite[Cf.][110]{latourelle1975et}.

Este dinamismo fundamental que puede encontrarse vivo hoy en la comunidad de la Iglesia ha actuado en ella desde su origen y le ha acompañado en cada época. Según esto es posible valorar lo que se transmite en la tradición eclesial como la perpetuación de la actividad de Cristo y los apóstoles, que es a su vez proyección del testimonio divino\footnote{\Cite[Cf.][573]{ninot2009tf}: \enquote{el testimonio divino se proyecta luego en el apostólico y se perpetúa en el testimonio eclesial. Por eso, el testimonio es revelación en la actividad de Cristo y de los apóstoles y es transmisión de la revelación en la tradición eclesial}.}.

En la actividad de Cristo el testimonio divino queda proyectado como interpelación a la libertad realizada por la identidad propia de Jesús: \blockquote[][\,(Jn 4,10)]{Si conocieras el don de Dios y quién es el que te dice <<dame de beber>> le pedirías tu, y él te daría agua viva}; \blockquote{``¿Crees tú en el Hijo del hombre?''\textelp{} ``¿Y quién es, Señor, para que crea en él?''\textelp{} ``Lo estás viendo: el que te está hablando, ese es''} (Jn 9,35-37). En la actividad apostólica, el testimonio divino sigue interpelando la libertad humana como manifestación de Jesús Resucitado. Los apóstoles actúan como testigos de los acontecimientos de la Pascua de Jesús y su valor salvífico\footcite[Cf.][576]{ninot2009tf} y este testimonio es descrito como acción del Espíritu que impulsa la tarea apostólica y que da nueva vida a los que acogen el anuncio de la Buena Noticia.

Un ejemplo que ilustra toda esta dinámica puede encontrarse en el relato de la tarea evangelizadora de Felipe. El apóstol sale más allá de Jerusalén hacia Samaria, y todavía llega más lejos, al compartir la Buena Noticia de Jesús con un extranjero etíope: \blockquote[][\,(Hch 8, 29-39)]{El Espíritu dijo a Felipe: <<Acércate y pégate a la carroza>>. Felipe se acercó corriendo, le oyó leer el profeta Isaías, y le preguntó: <<¿Entiendes lo que estás leyendo?>>. Contestó: <<¿Y cómo voy a entenderlo si nadie me guía?>>. E invitó a Felipe a subir y a sentarse con él. El pasaje de la Escritura que estaba leyendo era este: \emph{Como cordero fue llevado al matadero, como oveja muda ante el esquilador, así no abre su boca. En su humillación no se le hizo justicia. ¿Quién podrá contar su descendencia? Pues su vida ha sido arrancada de la tierra.} El eunuco preguntó a Felipe: <<Por favor, ¿de quién dice esto el profeta?; ¿de él mismo o de otro?>>. Felipe se puso a hablarle y, tomando pie de este pasaje, le anunció la Buena Nueva de Jesús. Continuando el camino, llegaron a un sitio donde había agua, y dijo el eunuco: <<Mira, agua. ¿Qué dificultad hay en que me bautice?>>. Mandó parar la carroza, bajaron los dos al agua, Felipe y el eunuco, y lo bautizó. Cuando salieron del agua, el Espíritu del Señor arrebató a Felipe. El eunuco no volvió a verlo, y siguió su camino lleno de alegría}. Además de ser ejemplo de la actividad apostólica, este relato puede servir como síntesis del modo en que la categoría del testimonio está presente en la Escritura.

El testimonio comienza con la iniciativa de Dios mismo que impulsa tanto la palabra profética del Antiguo Testamento como el anuncio apostólico del Nuevo Testamento. Esta iniciativa de Dios tiende hacia el testimonio de la Palabra definitiva del Padre que es Cristo resucitado. En aquellos que creen en el testimonio de Dios se engendra alegría y vida nueva. En palabras de R. Latourelle: \blockquote[{\Cite[1530]{latourelle2000testimonio}}.]{En el trato de las tres personas divinas con los hombres existe un intercambio de testimonios que tiene la finalidad de proponer la revelación y de alimentar la fe. Son tres los que revelan o dan testimonio, y esos tres no son más que uno. Cristo da testimonio del Padre, mientras que el Padre y el Espíritu dan testimonio del Hijo. Los apóstoles a su vez dan testimonio de lo que han visto y oído del verbo de la vida. Pero su testimonio no es la comunicación de una ideología, de un descubrimiento científico, de una técnica inédita, sino la proclamación de la salvación prometida y finalmente realizada}.

De este modo el anuncio del apóstol Felipe sirve aquí como un ejemplo específico del testimonio, que ilustra una noción que \blockquote[{\Cite[109]{prades2015testimonio}}.]{atraviesa toda la Escritura y se corresponde con la estructura misma de la revelación}. El testimonio está presente a lo largo de la Escritura junto a otras categorías como pueden ser la de `alianza', `palabra', `paternidad' o `filiación', como parte del \blockquote[{\Cite[1523]{latourelle2000testimonio}}.]{grupo de analogías empleadas por la Escritura para introducir al hombre en las riquezas del misterio divino}.

La categoría del testimonio ha servido para acercarnos a algunos textos magisteriales y describir la vida de la Iglesia como signo sacramental. Las luminosas palabras de K. Wojtyła pueden servirnos aquí para concluir: \blockquote[Para una discusión más amplia de la lectura de Wojtyła véase {\cite[194-197]{prades2015testimonio}}.]{El significado del testimonio en la doctrina del Vaticano~II es explícitamente analógico, puesto que el Concilio habla del testimonio de Dios y del hombre, que, de diversa manera, corresponde al divino, y a una respuesta multiforme a la revelación. En todo caso, sin embargo, la respuesta es testimonio y el testimonio, respuesta}.

Este recorrido a través de algunos modos de emplear la categoría del testimonio en la Escritura y la doctrina magisterial ha servido para describir los dinamismos de la Revelación como acción libre y amorosa del Padre encarnada en la naturaleza humana asumida por el Verbo y sostenida por la acción interior del Espíritu. Esta acción de la libertad divina ha encontrado la correspondencia de la libertad humana que acoge la invitación al amor y se compromete por completo a la comunión con Dios. Este intercambio testimonial comunica el amor divino.

%Es sobre esta práctica de la Iglesia que quisiéramos formular una pregunta que ponga en marcha nuestra investigación. Para esto nos servirá acudir al pensamiento de San Agustín y encontrar algo de luz. En el capítulo XI de las \emph{Confesiones} nos lo encontramos inquieto ---como siempre--- esta vez pensando en Dios y pensando en el tiempo, asaltado por una serie de preguntas:

¿Cómo podemos caracterizar el modo en que la pregunta sobre el fenómeno de la Revelación divina puede ser atendida dentro del contexto de la filosofía analítica?

\blockquote[Confesiones XI.14 n.17.]{¿Qué es, pues, el tiempo? ¿Quién podrá explicar esto fácil y brevemente? ¿Quién podrá comprenderlo con el pensamiento, para hablar luego de él? Y, sin embargo, ¿qué cosa más familiar y conocida mentamos en nuestras conversaciones que el tiempo? Y cuando hablamos de él, sabemos sin duda qué es, como sabemos o entendemos lo que es cuando lo oímos pronunciar a otro. ¿Qué es, pues, el tiempo? Si nadie me lo pregunta, lo sé; pero si quiero explicárselo al que me lo pregunta, no lo sé}.

Agustín expresa su extrañeza de que un concepto empleado ordinariamente se torne tan desconocido cuando llega la hora de explicarlo. \enquote*{¿Qué es el tiempo?} o \enquote*{¿qué es conocer?}, \enquote*{¿la libertad?} y \enquote*{¿qué es la fe?} son preguntas de este tipo; distintas, por ejemplo, a \enquote*{¿cuál es el peso exacto de este objeto?} o \enquote*{¿quién será la próxima persona en entrar por esa puerta?}\footnote{\Cite[Cf.][304]{wittgenstein2005bt}: \enquote{(Questions of different kinds occupy us. For instance, ``What is the specific weight of this body'', ``Will the weather stay nice today'', ``Who will come through the door next'', etc. But among our questions there are those of a special kind. Here we have a different experience. These questions seem to be more fundamental than the others. And now I say: When we have this experience, we have arrived at the limits of language.)}.}. Preguntar \enquote*{¿qué es conocer una verdad para la vida por el testimonio de la Revelación divina?} sería, como la pregunta agustiniana sobre el tiempo, una pregunta sobre la naturaleza o esencia de este fenómeno. Un concepto familiar en la vida de la Iglesia como el testimonio queda enmarcado como problema cuando nos acercamos a él queriendo comprender su esencia.

%Esto ya nos da una pista sobre el modo en que nos cuestionaremos acerca del testimonio. El siguiente elemento que servirá de clave para el estudio lo obtenemos si precisamos un poco cómo Elizabeth Anscombe se conduce a través de cuestiones filosóficas como las planteadas anteriormente. Así, como telón de fondo, podemos desplegar otro modo de proceder como el que se encuentra en la investigación realizada a inicios del siglo XX por el psicólogo William James. Nos servirá para contrastar.

%Al comienzo de sus conferencias sobre \emph{religión natural} James dedica una exposición breve para explicar algo del método de su estudio sobre las tendencias religiosas de las personas. Se apoya sobre la literatura de la lógica de su época para distinguir dos niveles de investigación sobre cualquier tema: aquellas preguntas que se resuelven por medio de proposiciones \emph{existenciales}, como \enquote*{¿qué constitución, qué origen, qué historia tiene esto?} o \enquote*{¿cómo se ha realizado esto?}. En otro nivel están las preguntas que se responden con proposiciones de \emph{valor} como \enquote*{¿cuál es la importancia, sentido o significado actual de esto?}. A este segundo juicio James lo denomina \emph{juicio espiritual}. El enfoque de sus conferencias sobre la religión será el existencial, pero no deja de ser interesante su apreciación de lo que sería un juicio espiritual aplicado a la Escritura:

%\blockquote[{\cite[27]{james2002variedades}}]{\enquote{¿Bajo qué condiciones biográficas los escritores sagrados aportan sus diferentes contribuciones al volumen sacro?}, \enquote{¿Cúal era exactamente el contenido intelectual de sus declaraciones en cada caso particular?}. Por supuesto, éstas son preguntas sobre hechos históricos y no vemos cómo las respuestas pueden resolver, de súbito, la última pregunta: \enquote{¿De qué modo este libro, que nace de la forma descrita, puede ser una guía para nuestra vida y una revelación?}. Para contestar habríamos de poseer alguna teoría general que nos mostrara con qué peculiaridades ha de contar una cosa para adquirir valor en lo que concierne a la revelación; y, en ella misma, tal teoría sería lo que antes hemos denominado un juicio espiritual}.

%Desde esta perspectiva la pregunta sobre cómo el testimonio de la Escritura puede ser una guía para nuestra vida es una investigación sobre la importancia, sentido o significado que éste pueda tener de hecho. La respuesta emitida sería un juicio de valor sobre este fenómeno testimonial. James propone que sería necesaria una teoría general que explicara qué características debería tener alguna cosa para que merezca ser valorada como revelación. Así planteada, la pregunta sobre el testimonio de la Escritura sería atendida adecuadamente por medio de una investigación que indagara dentro de este fenómeno para descubrir los elementos que le otorgan el valor adecuado como para ser considerado como revelación o estimado como guía para nuestra vida. La explicación de dichos elementos configurarían una teoría que nos permitiría juzgar este testimonio concreto como valioso o no, como revelación y guía para nuestras vidas.

%Si examinamos ahora la metodología de Anscombe y la comparamos con la propuesta de William James se aprecian algunas distinciones características de su filosofía que nos evitarán confusiones en la travesía a lo largo de su obra y pensamiento. En efecto: \blockquote[{\cite[1]{teichmann2008ans}}: Part of the difficulty in reading Anscombe is in finding your bearings, and this has to do with her eschewal of System. A system or theory often makes things easier for the reader. Once you have grasped N's theory, you can frequently infer what N would have to say on some point by simply `applying' the theory. But it can often be hard to predict in advance what Anscombe will say about some given thing. She is infuriatingly prone to take each case on its merits.]{Parte de la dificultad en leer a Anscombe está en encontrar nuestro rumbo, y esto tiene que ver con su evasión de Sistema. Un sistema o teoría a menudo hace las cosas más fáciles para el lector. Una vez que haz captado la teoría de $N$, con frecuencia puedes inferir qué tendría que decir $N$ sobre algún punto al simplemente \enquote*{aplicar} la teoría. Pero frecuentemente puede ser difícil predecir de antemano qué dirá Anscombe acerca de alguna cosa dada. Tiene la exasperante tendencia a tomar cada caso en sus propios méritos.} No quiere decir esto que Anscombe carezca de rigor o sistematicidad en sus escritos, sin embargo suele adentrarse \enquote{in medias res} en las discusiones con la intención de llegar a algún sitio por la fuerza de sus propias reflexiones sin detenerse a dar mucha explicación de sus presupuestos o del trasfondo de su discusión.\footnote{\cite[Cf.][1]{teichmann2008ans}: \textelp{} there is another reason for the lack of apparent systematicity in Anscombe's writings, and that is that her purpose in writing was typically to get somewhere in her own thoughts on some topic; she usually spends little or no time in providing a background, or in justifying her main `assumptions', preferring to begin \emph{in medias res}.} Sin embargo en esta característica de su método hay una cuestión de fondo que tiene que ver con la influencia de Wittgenstein: \blockquote[{\cite[1]{teichmann2008ans}}: There is a familiar philosophical, or meta-philosophical, issue here, to do with the pointfulness or otherwise of constructing generalizations. Wittgenstein considered prefacing the text of the Philosophical Investigations with the epigraph `I'll teach you differences', and Anscombe certainly shared Wittgenstein's belief that glossing over differences was one of the main sources of error in philosophy.]{Hay aquí una cuestión familiar filosófica, o meta-filosófica, concerniente a la utilidad o no de construir generalizaciones. Wittgenstein consideró prologar el texto de \emph{Investigaciones Filosóficas} con el epígrafe \enquote*{Te enseñaré las diferencias}, y Anscombe ciertamente compartía la creencia de Wittgenstein de que pasar por encima de las diferencias era una de las principales fuentes de error en la filosofía}.

Esta preocupación por el modo específico de afrontar un problema filosófico ocupa un lugar importante en \emph{Investigaciones Filosóficas} De Ludwig Wittgenstein. En la \S89 se encuentra una referencia al texto antes citado de las \emph{Confesiones} para describir la peculiaridad de las preguntas filosóficas: \blockquote[{\Cite[\S89]{wittgenstein1953phiinv}}: \enquote{Augustine says in \emph{Confessions} XI. 14, ``quid est ergo tempus? si nemo ex me quaerat scio; si quaerenti explicare velim nescio''. ---This could not be said about a question of natural science (``What is the specific gravity of hydrogen'', for instance). Something that one knows when nobody asks one but no longer knows when one is asked to explain it, is something that has to be \emph{called to mind}. (And it is obviously something which, for some reason, it is difficult to call to mind.)}.]{Agustín dice en \emph{Confesiones} XI. 14, ``quid est ergo tempus? si nemo ex me quaerat scio; si quaerenti explicare velim nescio''. ---Esto no podría ser dicho de una pregunta propia de la ciencia natural (``Cuál es la gravedad específica del hidrógeno'', por ejemplo). Algo que uno conoce cuando nadie le pregunta pero que no conoce ya cuando alguien pide que lo explique, es algo que tiene que \emph{ser traído a la mente}. (Y esto es obviamente algo que, por algún motivo, es difícil de traer a la mente.)} Para Ludwig es de gran importancia atender el paso que damos para resolver la perplejidad causada por el reclamo de explicar un fenómeno. El deseo de aclararlo nos puede impulsar a buscar una explicación dentro del fenómeno mismo, o como él diría: \blockquote[{\Cite[\S90]{wittgenstein1953phiinv}}: \enquote{We feel as if we had to see right into phenomena}.]{Nos sentimos como si tuviéramos que mirar directamente hacia dentro de los fenómenos}.

%Esta predisposición nos puede conducir a ignorar la amplitud del modo en que el lenguaje es empleado en la actividad humana para hablar de lo que se investiga y a enfocarnos sólo en un elemento particular del lenguaje sobre este fenómeno y tomarlo como un ejemplo paradigmático para construir un modelo abstrayendo explicaciones y generalizaciones sobre él. Esta manera de indagar, le parece a Wittgenstein, nos hunde cada vez más profundamente en un estado de frustración y confusión filosófica de modo que llegamos a imaginar que para alcanzar claridad: \blockquote[{\cite[\S106]{wittgenstein1953phiinv}}: we have to describe extreme subtleties, which again we are quite unable to describe with the means at our disposal. We feel as if we had to repair a torn spider's web with our fingers. ]{tenemos que describir sutilezas extremas, las cuales una vez más somos bastante incapaces de describir con los medios que tenemos a nuestra disposición. Sentimos como si tuvieramos que reparar una telaraña rota usando nuestros dedos.}

La alternativa que Wittgenstein propone es una investigación que no esté dirigida hacia dentro del fenómeno, sino \blockquote[{\Cite[\S90]{wittgenstein1953phiinv}}: \enquote{as one might say, towards the \emph{`possibilities'} of phenomena. What that means is that we call to mind the \emph{kinds of statement} that we make about phenomena}.]{como se podría decir, hacia `\emph{posibilidades}' de fenómenos. Lo que eso significa es que traemos a la mente los \emph{tipos de afirmaciones} que hacemos acerca de los fenómenos}. Este tipo de investigación la denomina `gramatical' y la describe diciendo: \blockquote[{\Cite[\S90]{wittgenstein1953phiinv}}: \enquote{Our inquiry is therefore a grammatical one. And this inquiry sheds light on our problem by clearing misunderstandings away. Misunderstandings concerning the use of words, brought about, among other things, by certain analogies between the forms of expression in different regions of our language.\,---\,Some of them can be removed by substituting one form of expression for another; this may be called `analysing' our forms of expression, for sometimes this procedure resembles taking things apart}.]{Por tanto nuestra investigación es gramatical. Y esta investigación arroja luz sobre nuestro problema al despejar los malentendidos. Malentendidos concernientes al uso de las palabras, suscitados, entre otras cosas, por ciertas analogías entre las formas de expresión en diferentes regiones de nuestro lenguaje.\,---\,Algunos de estos pueden ser eliminados si se sustituye una forma de expresión por otra; esto puede ser llamado `analizar' nuestras formas de expresión, puesto que a veces este procedimiento se parece a desarmar algo}.

El modo de salir de nuestra perplejidad, por tanto, consiste en prestar cuidadosa atención al uso que hacemos de hecho de las palabras y la aplicación que asignamos a las expresiones. Esto queda al descubierto en nuestro uso del lenguaje de modo que la dificultad para \emph{traer a la mente} aquello que aclare un fenómeno no está en descubrir algo oculto en este, sino en aprender a valorar lo que tenemos ante nuestra vista: \blockquote[{\Cite[\S129]{wittgenstein1953phiinv}}: \enquote{The aspects of things that are most important for us are hidden because of their simplicity and familiarity. (One is unable to notice something --- because it is always before one's eyes.)}.]{Los aspectos de las cosas que son más importantes para nosotros están escondidos por su simplicidad y familiaridad. (Uno es incapaz de notar algo --- porque lo tiene siempre ante sus ojos.)} La descripción de los hechos concernientes al uso del lenguaje en nuestra actividad humana ordinaria componen los pasos del tipo de investigación sugerido por Wittgenstein. Hay cierta insatisfacción en este modo de proceder, como él mismo afirma: \blockquote[{\Cite[\S118]{wittgenstein1953phiinv}}: \enquote{Where does this investigation get its importance from, given that it seems only to destroy everything interesting: that is, all that is great and important? (As it were, all the buildings, leaving behind only bits of stone and rubble.) But what we are destroying are only houses of cards, and we are clearing up the ground of language on which they stood}.]{¿De dónde adquiere su importancia esta investigación, dado que parece solo destruir todo lo interesante: esto es, todo lo que es grandioso e importante? (Por así decirlo, todos los edificios, dejando solamente pedazos de piedra y escombros.) Pero lo que estamos destruyendo son solo casas de naipes, y estamos despejando el terreno del lenguaje donde estaban erigidas}.

La estrategia de Elizabeth Anscombe comparte estas actitudes explicadas por Wittgenstein. A la hora de atender una pregunta filosófica lo que Anscombe nos invita a \emph{traer a la mente} no son elementos ocultos en el fenómeno que se estudia, sino los tipos de afirmaciones ---que están claramente ante nuestra vista--- empleados para expresar aquello que se está indagando. Al describir estas expresiones se aclara el uso del lenguaje y se disipa el problema filosófico. Elizabeth adopta, por tanto, ese: \blockquote[{\Cite[xix]{anscombe2011plato}}: \enquote{There is however a somehow characteristically Wittgenstenian way of countering the philosopher's tendency to explain a philosophically puzzling thing by inventing an entity or event which causes it, as physicists invent particles like the graviton}.]{modo característicamente Wittgensteniano de rebatir la tendencia del filósofo de explicar alguna cuestión filosóficamente enigmática inventando una entidad o evento que la causa, así como los físicos inventan partículas como el gravitón}.

Ciertamente Anscombe no se limita exclusivamente a un solo método. El mismo Wittgenstein diría que \blockquote[{\Cite[\S133]{wittgenstein1953phiinv}}: \enquote{There is not a single philosophical method, though there are indeed methods, different therapies as it were}.]{No hay un solo método filosófico, aunque ciertamente hay métodos, diferentes terapias por así decirlo}. Sin embargo cabe destacar esta estrategia porque la emplea con frecuencia. En escritos importantes de su obra podemos encontrarla empleando lenguajes o juegos de lenguaje imaginarios para arrojar luz sobre modos actuales de usar el lenguaje o esquemas conceptuales; del mismo modo su trabajo esta lleno de ejemplos donde la encontramos examinando con detenimiento el uso que de hecho hacemos del lenguaje\footnote{\Cite[Cf.][228-229]{teichmann2008ans}: \enquote{Another way which we can learn from Anscombe is by seeing \emph{how} she does philosophy, and understanding why she does it the way she does. Here is the point where it might be useful to consider whether Anscombe can be called a `linguistic philosopher', and if so, in what sense. A distinction worth making straight away is that between (a) philosophers who direct our attention to what we actually say, and to features of our actual language (or group of languages), and (b) philosophers who ask us to think about possible, as well as actual, languages and language-games. The first group of philosophers might be called ordinary-language philosophers. Anscombe quite clearly belongs to (b), not to (a); examples of her imaginary languages include the language containing the self-referential `A', the language containing the verb to REMBER \textelp{}, the language containing the verb to blip, analogous to `promise' \textelp{}, and the language containing the primitive past-tense report `red' \textelp{}. The purpose of presenting these imaginary languages is of course to cast light on our actual languages and conceptual schemes}.}.
%El título de este trabajo señala que el análisis sobre el testimonio que será expuesto es el que se encuentra desarrollado en el pensamiento de Elizabeth Anscombe.

La pregunta planteada al inicio: ¿qué es conocer una verdad para la vida por el testimonio de la Escritura?, entendida como investigación filosófica, será examinada a partir de las descripciones que Anscombe realiza sobre el modo de usar el lenguaje sobre el creer, la confianza, la verdad, la fe y otros fenómenos relacionados con el conocer por testimonio.
%El modo en que Anscombe trata el lenguaje actual y posible en su metodología, no solo influirá en el capítulo dedicado a su obra, sino que orienta el desarrollo general de este estudio sobre la categoría del testimonio.

%Como nuestro título también señala, el estudio dirigido al pensamiento y obra de Anscombe se realiza en perspectiva teológica.
Sobre la relación que pueda haber en una investigación teológica enfocada en el pensamiento filosófico de una autora como Elizabeth es iluminadora la manera en que Joseph Ratzinger responde a la pregunta \enquote*{¿Qué es teología?} en su \emph{Teoría de los Principios Teológicos}. Inspirado en la vida y obra del Cardenal Hermann Volk y en su divisa \emph{Dios todo en todos}, habla de la teología como un programa espiritual y como un modo de interrogar dirigido hacia los fundamentos. Entonces sugiere dos tesis que considera que no son contradictorias: \enquote{La teología se refiere a Dios} y  \enquote{El pensamiento teológico está vinculado al modo de cuestionar filosófico como a su método fundamental}\footnote{\Cite[Cf.][380]{ratzinger2005teoria}.
%: \enquote{Me viene a las mientes, por un lado, \textelp*{la expresión}: \emph{Dios todo en todos}, y el programa espiritual contenido \textelp*{ahí}; por otra parte, \textelp{} un modo de interrogar total y absolutamente filosófico, que no se detiene en reales o supuestas comprobaciones históricas, en diagnósticos sociológicos o en técnicas pastorales, sino que se lanza implacablemente a la búsqueda de los fundamentos. Según esto, cabría formular ya dos tesis que pueden servirnos de hilo conductor para nuestro interrogante sobre la esencia de la teología: 1. La teología se refiere a Dios. 2. El pensamiento teológico está vinculado al modo de cuestionar filosófico como a su método fundamental. Podría parecer que estas tesis son contradictorias si, por un lado, se entiende por filosofía un pensamiento que, en virtud de su propia naturaleza, prescinde ---y debe prescindir--- de la revelación y si, por otro lado, se sustenta la opinión de que sólo se puede llegar al conocimiento de Dios por el camino de la revelación y en consecuencia, el problema de Dios no es, estrictamente hablando, un tema de la razón en cuanto tal. Estoy convencido de que esta postura \textelp{} a largo plazo desembocará irremediablemente en la paralización por un igual de la filosofía y de la teología.}
}.
Esta investigación sobre el testimonio como expresión de la vida de la Iglesia será realizada atendiendo al modo de cuestionar filosófico realizado por Elizabeth Anscombe como método, examinando esta experiencia en referencia a Dios, es decir, como vivencia de su ser y de su obrar. Tras estas consideraciones metodológicas generales, en los apartados siguientes, estudiaremos el testimonio dentro del contexto de la Sagrada Escritura y del Magisterio, para luego plantearnos algunas líneas de investigación al examinar el testimonio como objeto de estudio teológico.

%Hasta aquí simplemente se ha descrito un modo de andar a través de la discusión acerca de la categoría del testimonio atendiendo el hecho de que tanto la temática como la figura de Anscombe otorgan a este camino peculiaridades que hay que tener en cuenta. Siendo conscientes de estas particularidades podríamos ahora ampliar el horizonte respecto de dos cuestiones brevemente ya expuestas antes. En primer lugar es necesario ampliar la descripción hecha hasta aquí del fenómeno del testimonio en la vida de la Iglesia, ya que aunque nos resulte familiar relacionarlo con el testimonio de la Sagrada Escritura esta categoría se halla presente con una riqueza más grande y diversa tanto en la vida eclesial, como en el Magisterio de la Iglesia, como en la propia Escritura. En segundo lugar habría que detallar todavía mejor el aspecto problemático del testimonio, sobre todo cuando se considera su importancia en la transmisión de la fe y el anuncio del Evangelio en el mundo.



% El cuarto apartado problematiza la categoría del testimonio empleando desafíos
% propuestos por la filosofía moderna y contemporanea.
% LA CATEGORÍA DEL TESTIMONIO COMO PROBLEMA
\section{La categoría de testimonio como objeto de estudio teológico}

Se pueden destacar diversas motivaciones para preguntarse sobre el testimonio. Desde el punto de vista teológico el hecho mismo de que esta categoría sea empleada en la Escritura sirve ya como justificación para estudiar mejor el fenómeno del testimonio dentro de la actividad humana, así argumenta R. Latourelle: \blockquote[{\Cite[1523]{latourelle2000testimonio}}. Varios estudios especializados en perspectiva teológico-fundamental han servido como marco de referencia general: \emph{Dar Testimonio} de J. Prades, \emph{La Teología Fundamental} de S. Pié-Ninot y \emph{Teología de la Revelación}, <<Évangélisation et témoignage>> en \emph{Evangelisation} y <<Testimonio>> en \emph{Diccionario de Teología Fundamental} de R. Latourelle. Remitimos a estos trabajos para una visión más amplia del testimonio como clave para el análisis de la revelación y su pretensión de verdad como propuesta sensata de credibilidad y la presencia de esta categoría teológica en la Escritura y el Magisterio.]{Si la revelación misma se apoya en la experiencia humana del testimonio para expresar una de las relaciones fundamentales que unen al hombre con Dios, la reflexión teológica se encuentra entonces autorizada a explorar los datos de esta experiencia}. Sin embargo el interés por la categoría de testimonio en la investigación teológica más reciente claramente está motivado por la presencia de esta noción en las reflexiones del Concilio Vaticano~II y el magisterio post-conciliar: \blockquote[{\Cite[81]{prades2015testimonio}}.]{La teología ha ido revalorizando el testimonio, que había quedado relegado a un segundo plano en otros momentos de la historia de la teología, hasta alcanzar una difusión realmente masiva en los años posteriores al Concilio}. El testimonio es un tema privilegiado en el Concilio y se le encuentra presente como `\emph{leitmotiv}' en las constituciones y decretos\footcite[Cf.][1523]{latourelle2000testimonio}. Vaticano~II potencia así este termino que ya se encontraba presente en las reflexiones del Vaticano~I: \blockquote[{\Cite[572]{ninot2009tf}}.]{Desde hace aproximadamente un siglo, la categoría testimonio se ha introducido de forma progresiva en el vocabulario eclesial. La concentración y personalización operada por el Concilio Vaticano~II conlleva la potenciación de un término nuevo como es el testimonio. \textelp{} lo que el Vaticano I pretendía al tratar el signo de la Iglesia, que también era visto como ``un testimonio'' [DH 3013], se encuentra en la categoría testimonio, que con el Vaticano~II irrumpe masivamente}.

Tras este entusiasmo inicial por el testimonio en ámbitos pastorales y teológicos se ha ido advirtiendo en algunos textos magisteriales y teológicos el aviso de cierto peligro de ambigüedad o abuso en el uso de esta categoría\footcite[Cf.][83]{prades2015testimonio}: \blockquote[{\Cite[84]{prades2015testimonio}}.]{se ha hecho notar que el testimonio podía verse limitado a la manifestación de una especie de seriedad con lo humano, ya fuera en términos de reivindicación social o de autenticidad existencial, con la inevitable prevalencia del sujeto ---individual o colectivo--- pero sin llegar a remitir a la verdad de Cristo. \textelp{}
Se trataría del riesgo de una reducción experiencialista del testimonio, donde lo más importante sería su carácter social-existencial y no tanto la efectiva verdad teologal transmitida. Se ha criticado consecuentemente la reducción del testimonio ---y de la misma teología--- a puro relato autobiográfico.

Si se recupera la profundidad implicada en el testimonio se contribuirá a salir del subjetivismo ---antiguo y moderno---, con su carga correspondiente de individualismo, tan contrario a la verdadera naturaleza social del hombre y al carácter a la vez personal y comunitario de la salvación cristiana}. Atendiendo a estos datos, una investigación teológica sobre el testimonio tiene el interés de profundizar en una categoría valiosa en el ámbito teológico y pastoral de modo que sea empleada y formulada adecuadamente.

Este interés interno de la discusión teológica está enmarcado en un contexto histórico del que también se derivan motivaciones para una valoración de la categoría del testimonio. Dos rasgos que cabe destacar de este momento presente son: \blockquote[{\Cite[75]{prades2015testimonio}}. Un análisis detallado del contexto presente se encuentra en {\cite[3-77]{prades2015testimonio}}.]{la tensión entre multiculturalismo y globalización como indicio de la dificultad para combinar positivamente el carácter individual y comunitario de la vida humana, y la discusión sobre el papel público de la religión, donde la tesis dominante de la <<edad secular>> se ve contrapesada por la irrupción de un nuevo paradigma que se denomina <<postsecular>>.}

En este contexto, el preguntarse sobre el testimonio tiene como objetivo un adecuado modo de entender la presencia pública de los cristianos en las sociedades plurales de occidente donde resulta problemática la comprensión del ser humano en su relación con Dios a través de la realidad\footcite[Cf.][75]{prades2015testimonio}. Es importante aclarar que en este contexto la cuestión de la presencia del cristianismo en la sociedad no tiene como solución adecuada una `autorrelativización'\footcite[Cf.][75;\,40-44]{prades2015testimonio} de sí mismo; igualmente: \blockquote[{\Cite[75; Cf. 33-40]{prades2015testimonio}}.]{no podemos presuponer el reconocimiento de su carácter universal por parte de los interlocutores ni podemos pretender alcanzarlo por una mera comparación de argumentos racionales que desnaturalice el carácter libre y singular de la revelación personal de Dios en Jesucristo}. Ante esto, el análisis de la categoría del testimonio viene a responder a la necesidad de recuperar una concepción de la razón y de la verdad más rica y más amplia; \blockquote[{\Cite[76]{prades2015testimonio}}.]{Es imprescindible repensar el nexo entre razón, afectos y libertad en la relación del hombre con lo real. Si se recupera esa visión amplia e integral de razón y de realidad se puede entonces mostrar convincentemente la credibilidad de la fe como asentimiento a una revelación personal en la historia}.
\label{subsec:amplia}

Será útil destacar sintéticamente algunas nociones fundamentales que estas referencias ofrecen a nuestro enfoque. La primera se deriva de una comprensión de la Iglesia: \blockquote[][\,(LG 8)]{Cristo, el único Mediador, instituyó y mantiene continuamente en la tierra a su Iglesia santa, comunidad de fe, esperanza y caridad, como un todo visible, comunicando mediante ella la verdad y la gracia a todos. Mas la sociedad provista de sus órganos jerárquicos y el Cuerpo místico de Cristo, la asamblea visible y la comunidad espiritual, la Iglesia terrestre y la Iglesia enriquecida con los bienes celestiales, no deben ser consideradas como dos cosas distintas, sino que más bien forman una realidad compleja que está integrada de un elemento humano y otro divino. Por eso se la compara, por una notable analogía, al misterio del Verbo encarnado, pues así como la naturaleza asumida sirve al Verbo divino como de instrumento vivo de salvación unido indisolublemente a El, de modo semejante la articulación social de la Iglesia sirve al Espíritu Santo, que la vivifica, para el acrecentamiento de su cuerpo (cf. Ef 4,16)}.
Esta descripción de la Iglesia como un complejo compuesto por un elemento humano y otro divino que puede ser comprendido teniendo como referente el misterio del Verbo encarnado sirve para comprender también el dinamismo de la revelación\footnote{\Cite[Cf.][480]{ninot2009tf}.}. La comunicación divina también tiene como referente central la presencia de Cristo en el mundo y se puede entender desde él como un hecho que está compuesto por un elemento humano y uno divino.

Esta composición se puede ver con claridad cuando se describe la revelación en clave testimonial: \blockquote[][\,(SCa 85)]{La misión primera y fundamental que recibimos de los santos Misterios que celebramos es la de dar testimonio con nuestra vida. El asombro por el don que Dios nos ha hecho en Cristo infunde en nuestra vida un dinamismo nuevo, comprometiéndonos a ser testigos de su amor. Nos convertimos en testigos cuando, por nuestras acciones, palabras y modo de ser, aparece Otro y se comunica. Se puede decir que el testimonio es el medio con el que la verdad del amor de Dios llega al hombre en la historia, invitándolo a acoger libremente esta novedad radical. En el testimonio Dios, por así decir, se expone al riesgo de la libertad del hombre. Jesús mismo es el testigo fiel y veraz (cf. Ap 1,5; 3,14); vino para dar testimonio de la verdad (cf. Jn 18,37)}.
%Con estas reflexiones deseo recordar un concepto muy querido por los primeros cristianos, pero que también nos afecta a nosotros, cristianos de hoy: el testimonio hasta el don de sí mismos, hasta el martirio, ha sido considerado siempre en la historia de la Iglesia como la cumbre del nuevo culto espiritual: <<Ofreced vuestros cuerpos>> (Rm 12,1). Se puede recordar, por ejemplo, el relato del martirio de san Policarpo de Esmirna, discípulo de san Juan: todo el acontecimiento dramático es descrito como una liturgia, más aún como si el mártir mismo se convirtiera en Eucaristía. Pensemos también en la conciencia eucarística que san Ignacio de Antioquía expresa ante su martirio: él se considera <<trigo de Dios>> y desea llegar a ser en el martirio <<pan puro de Cristo>>. El cristiano que ofrece su vida en el martirio entra en plena comunión con la Pascua de Jesucristo y así se convierte con Él en Eucaristía. Tampoco faltan hoy en la Iglesia mártires en los que se manifiesta de modo supremo el amor de Dios. Sin embargo, aun cuando no se requiera la prueba del martirio, sabemos que el culto agradable a Dios implica también interiormente esta disponibilidad, y se manifiesta en el testimonio alegre y convencido ante el mundo de una vida cristiana coherente allí donde el Señor nos llama a anunciarlo}

La vida entregada del cristiano, en correspondencia al don que Dios hace de sí en Cristo en  los misterios que celebramos, constituye el \blockquote[{\Cite[Cf.][399]{prades2015testimonio}. Una clave sintética importante relacionada con esta noción se encuentra, en la valoración hecha por K. Wojtyła del testimonio en los documentos conciliares: <<el testimonio consiste en creer y profesar la fe, es decir, acoger el testimonio del mismo Dios y, al tiempo, responder a aquel con el propio testimonio>>, véase: \Cite[194-197]{prades2015testimonio}. Otra categoría importante es la de \enquote*{culto razonable} (Rom 12,1) propuesta por J. Prades como compendio del testimonio cristiano, véase: \Cite[405-430]{prades2015testimonio}. También es relevante la expresión \enquote*{Palabra-vivida-en-el-Espíritu} empleada por Latourelle para describir las diversas dimensiones de la dinámica de la revelación y su interrelación, véase: \Cite[Cf.][110]{latourelle1975et} y también \Cite[582]{ninot2009tf}.}]{signo histórico de la verdad de Dios trino en la historia}. Su entrega es acción testimonial que corresponde con la naturaleza de la revelación y su transmisión:
\blockquote[{\Cite[419-420]{prades2015testimonio}}.]{Se trata, efectivamente, de la verdad revelada en su índole propia, que se transmite y comunica mediante la humanidad del Verbo encarnado, con hechos y palabras intrínsecamente ligados (DV 2). Fue así en la primera proclamación evangélica y sigue siendo así por analogía en la tradición apostólica que acontece mediante la totalidad del cuerpo eclesial, en íntima relación de Palabra, sacramento, carismas y ministerio (DV 7-8; LG cap. I).}

La obra de Elizabeth Anscombe ofrece amplias oportunidades para estudiar el nexo de razón, afectos y libertad como uno que, de acuerdo a la perspectiva de la filosofía analítica, se encuentra vivo en la actividad humana del lenguaje. Sus discusiones han determinado la ruta de esta investigación. Nuestro estudio del caracter testimonial de la revelación divina dentro de la obra de Anscombe tendrá en el centro una descripción de la naturaleza especial de la creencia que es la fe, entendida como correlato de la revelación\footnote{\Cite[Cf.][185]{conesa1994cc}. Este punto se desarrollará más adelante en: (\S\ref{subsec:fecorrel}, p.~\pageref{subsec:fecorrel}).}. Esto significa que el análisis que haremos desde su obra consistirá en un estudio de la fe entendida como respuesta correspondiente a la revelación divina. En este sentido, así como la categoría de testimonio ha servido para hablar de lo que llamamos `revelación divina' como una `realidad compleja' que puede entenderse teniendo como referente el misterio del Verbo encarnado, la misma categoría será útil para caracterizar la naturaleza especial de esa creencia que llamamos `fe' entendiéndola como `saber testimonial'. Las investigaciones de Anscombe que examinaremos consisten en esfuerzos por describir y dilucidar la naturaleza de esa disposición particular que llamamos `fe' o las justificaciones sobre las cuales juzgamos que una realidad puede ser tenida como comunicación de Dios. En ese sentido, consistirán en un acercamiento al hecho de la revelación y su credibilidad desde el análisis de su correlato que es el hecho de la fe y su justificación. 

\label{subsec:aptitud}
En términos generales podemos decir que el análisis de la fe en la obra de Anscombe consistirá en un examen de esa disposición particular que se tiene sobre la verdad de una comunicación recibida justificada por la certeza que merece la confianza en lo que se ha juzgado como testimonio recto. Si recurrimos al lenguaje de \emph{Fides et Ratio} podemos decir también que las investigaciones de Anscombe nos ofrecen nociones valiosas para describir \enquote{la posibilidad de discernir la revelación divina de otros fenómenos, en el reconocimiento de su credibilidad} y \blockquote[{Cf. FR 67 y \Cite[198-201]{ninot2009tf}}.]{la aptitud del lenguaje humano para hablar de forma significativa y verdadera incluso de lo que supera toda experiencia humana}. Desde estas ideas generales quisieramos establecer cuestiones más específicas, es por esto que a continuación explicaremos tres cuestionamientos fundamentales que representan las áreas de nuestro análisis posterior del testimonio como objeto de estudio teológico dentro de la obra de Elizabeth Anscombe.

\subsection{¿Cuál es el valor epistemológico del testimonio?}

Corresponde a la epistemología la tarea de estudiar la naturaleza del conocer y su justificación. ¿Cuáles son los componentes del conocimiento? ¿sus fuentes o condiciones? ¿sus límites? La pregunta sobre el valor epistemológico del testimonio consiste en juzgar el lugar que este ocupa en una descripción del conocimiento; ¿qué se puede decir del testimonio como estrategia para adquirir la verdad y evitar el error?

Podemos recurrir al análisis tradicional empleado para hablar del conocimiento proposicional y entenderlo como \enquote*{creencia verdadera justificada}\footnote{\Cite[4]{moser2002ep}: \enquote{Ever since Plato's Theaetetus, epistemologists have tried to identify the essential, defining components of propositional knowledge. These components will yield an analysis of propositional knowledge. An influential traditional view, inspired by Plato and Kant among others, is that propositional knowledge has three individually necessary and jointly sufficient components: justification, truth, and belief. On this view, propositional knowledge is, by definition, justified true belief. This tripartite definition has come to be called ``the standard analysis''}.}. Según esta composición tripartita la pregunta sobre el valor epistemológico del testimonio se puede plantear diciendo: \enquote*{dada una comunicación que cualifique como testimonio y que sea el caso que la creencia formada desde esta comunicación está basada enteramente en el testimonio recibido}\footnote{\cite[Cf.][4]{lackeysosa2006eptest}: \enquote{Even if an expression of thought qualifies as testimony and the resulting belief formed is entirely testimonially based for the hearer, however, there is the further question of how precisely such a belief successfully counts as justified belief or an instance of knowledge}.}, \enquote*{¿cómo adquirimos efectivamente una creencia verdadera justificada sobre la base de lo que alguien nos ha dicho?}\footnote{\cite[Cf.][2]{lackeysosa2006eptest}: \enquote{how we successfully acquire justified belief or knowledge on the basis of what other people tell us. This, rather than what testimony is, is often taken to be the issue of central import from an epistemological point of view}.}, es decir, \enquote*{¿cómo, precisamente, una creencia como esta puede ser contada satisfactoriamente como creencia justificada o una instancia de conocimiento?}\footnote{\cite[Cf.][4]{lackeysosa2006eptest}: \enquote{how precisely such a belief successfully counts as justified belief or an instance of knowledge}.}

Las respuestas a esta pregunta central sobre la epistemología del testimonio se han situado en dos posturas que se han denominado `reduccionista' y `no-reduccionista'\footnote{\cite[Cf.][4]{lackeysosa2006eptest}: \enquote{Indeed, this is the question at the center of the epistemology of testimony, and the current philosophical literature contains two central options for answering it: non-reductionism and reductionism}.}. Las raíces históricas de la primera postura se le suelen atribuir a Hume y de la segunda a Thomas Reid.

De acuerdo a los no-reduccionistas el testimonio es simplemente una fuente de justificación como lo sería la percepción de los sentidos, la memoria o la inferencia. Según esto, siempre que no haya una justificación contraria suficientemente relevante, el que escucha tiene justificación verdadera para creer las proposiciones del testimonio del que habla\footnote{\cite[Cf.][4]{lackeysosa2006eptest}: \enquote{According to non-reductionists ---whose historical roots are standardly traced back to Reid--- testimony is just as basic a source of justification (warrant, entitlement, knowledge, etc.) as sense perception, memory, inference, and the like. Accordingly, so long as there are no relevant defeaters, hearers can justifiedly accept the assertions of speakers merely on the basis of a speaker's testimony}.}.

Hume, por su parte, \blockquote[{\Cite[79]{coady1992test}}: \enquote{is one of the few philosophers who has offered anything like a sustained account of testimony and if any view has a claim to the title of `the received view' it is his}.]{es uno de los pocos filósofos que ha ofrecido algo así como una descripción sostenida acerca del testimonio y si alguna perspectiva puede reclamar el título de `el punto de vista común' es la suya}. En la base de su valoración del testimonio está su estima de la relación de causa y efecto como fundamento de cualquier razonamiento concerniente a cuestiones de hecho.

Hume explica en la \S4,1 del \emph{Enquiry} que los objetos de nuestros razonamientos son relaciones de ideas o cuestiones de hecho. Mientras que las primeras pueden ser demostradas \emph{a priori}, las segundas dependen de las evidencias de nuestras experiencias presentes ante nuestros sentidos o memoria. Según esta concepción, la posibilidad de conocer algo más allá de nuestra experiencia es fruto de la inferencia que podemos hacer desde las relaciones que habitualmente experimentamos entre los hechos y las cosas\footnote{\cite[Cf.][\S4, 1. 15-20]{hume1777enquiry}.}.

%Distinto a las relaciones de ideas, la evidencia de la veracidad de una cuestión de hecho no se demuestra a priori, sino que ha de ser descubierta en la experiencia. Ahora bien, ¿cuál es la naturaleza de la evidencia de aquellas cuestiones de hecho que están más allá de la percepción de nuestros sentidos o de las impresiones de nuestra memoria?\footnote{Cf.~\cite[\S4,1. 15]{hume1777enquiry}: Matters of fact, which are the second objects of human reason, are not ascertained in the same manner; nor is our evidence of their truth, however great, of a like nature with the foregoing (relations of ideas) \textelp{} The contrary of every matter of fact is still possible \textelp{} We should, in vain, therefore attempt to demonstrate its falsehood. Were it demonstratively false, it would imply a contradiction, and could never be distinctly conceived by the mind \textelp{} what is the nature of that evidence which assures us of any real existence and matter of fact, beyond the present testimony of our senses, or the records of our memory.} Nuestros razonamientos relacionados con algún hecho se componen de inferencias realizadas a partir del conocimiento que tenemos de que a una causa se sigue su efecto.\footnote{Cf.~\cite[\S4,1. 16]{hume1777enquiry}: All our reasonings concerning fact are of the same nature; and here it is constantly supposed that there is a connection between the present fact and that which is inferred from it. Were there nothing to bind them together, the inference would be entirely precarious.} Este conocimiento de la relación causa y efecto, a su vez, no consiste en un razonamiento a priori, \blockquote[{\cite[\S4,1. 17]{hume1777enquiry}}: that the knowledge of this relation is not, in any instance, attained by reasonings a priori, but arises entirely from experience, when we find that any particular objects are constantly conjoined with each other.]{sino que surge completamente de la experiencia, cuando descubrimos que cualesquiera objetos particulares están constantemente unidos entre sí}. Es así que \blockquote[{\cite[\S4,1. 16]{hume1777enquiry}}: By means of that relation alone, we can go beyond the evidence of our memory and senses.]{tan solo por medio de esta relación, podemos ir más allá de nuestra memoria y sentidos}.

Esta misma línea de razonamiento es la que se sigue en la descripción acerca del testimonio y su valor. Así lo sostiene uno de los grandes especialistas en la epistemología del testimonio, C.\,A.\,J. Coady, del que tomo\footnote{\Cite[Cf.][7]{coady1992test}} esta larga cita de Hume: \blockquote[{
%}]
\Cite[\S10, 1. 74]{hume1777enquiry}: \enquote{there is no species of reasoning more common, more useful, and even necessary to human life, than that which is derived from the testimony of men, and the reports of eye witnesses and spectators. This species of reasoning, perhaps, one may deny to be founded on the relation of cause and effect. I shall not dispute about a word. It will be sufficient to observe, that our assurance in any argument of this kind, is derived from no other principle than our observation of the veracity of human testimony, and of the usual conformity of facts to the reports of witnesses. It being a general maxim, that no objects have any discoverable connection together, and that all the inferences which we can draw from one to another, are founded merely on our experience of their constant and regular conjunction; it is evident, that we ought not to make an exception to this maxim in favour of human testimony, whose connection with any event seems, in itself, as little necessary as any other. Were not the memory tenacious to a certain degree; had not men commonly an inclination to truth and a principle of probity; were they not sensible to shame, when detected in a falsehood; were not these, I say, discovered by experience to be qualities inherent in human nature, we should never repose the least confidence in human testimony. A man delirious, or noted for falsehood and villany, has no manner of authority with us}. Traducción al español de todos los textos del \emph{Enquiry} tomada de \Cite{hume1777enquiryes}}.]
%{no hay un tipo de razonamiento más común, más útil, e incluso necesario para la vida humana, que aquel que se deriva del testimonio de los hombres, y los informes de testigos oculares y espectadores. Quizá uno pueda negar que esta clase de razonamiento esté fundada en la relación de causa y efecto. No discutiré por una palabra. Será suficiente observar, que nuestra confianza en un argumento de este tipo, no se deriva de otro principio que el de nuestra observación de la veracidad del testimonio humano, y la correspondencia habitual de los hechos con los informes de los testigos. Siendo esto una máxima general, que ningún caso de objetos tienen alguna conexión entre sí que pueda ser descubierta, y que todas las inferencias que podamos sacar de uno por el otro, son fundadas meramente en nuestra experiencia de su constante y regular conjunción; es evidente, que no deberíamos hacer una excepción a esta máxima en favor del testimonio humano, cuya conexión con cualquier evento parece, en sí misma, tan poco necesaria como cualquier otra. Si la memoria no fuera tenaz en cierto grado; si no tuvieran los hombres comúnmente una inclinación a la verdad y un principio de honradez; si no fueran sensibles a la vergüenza, cuando son descubiertos en la mentira; digo yo, si éstas no fueran cualidades que la experiencia descubre como inherentes a la naturaleza humana, jamas tendríamos la menor confianza en el testimonio humano. Un hombre delirante, o notorio por mentiroso o villano, no tiene ninguna clase de autoridad entre nosotros.}
{no hay un tipo de razonamiento más común, más útil o incluso más necesario para la vida humana que el derivado de los testimonios de los hombres y los informes de los testigos presenciales y de los espectadores. Quizá uno pueda negar que esta clase de razonamiento esté fundado en la relación causa-efecto. No discutiré sobre la palabra. Bastará con apuntar que nuestra seguridad, en cualquier argumento de esta clase, no deriva de ningún otro principio que la observación de la veracidad del testimonio humano y de la habitual conformidad de los hechos con los informes de los testigos. Siendo un principio general que ningún objeto tiene una conexión con otro que pueda descubrirse, y que todas las inferencias que podemos sacar del uno al otro están meramente fundadas en nuestra experiencia de regularidad y constancia de su conjunción, es evidente que no debemos hacer una excepción de este principio en el caso del testimonio humano, cuya conexión con otro suceso cualquiera parece en sí misma tan poco necesaria como cualquier otra conexión. Si la mente no fuera en cierto grado tenaz, si los hombres no tuvieran comúnmente una inclinación a la verdad y conciencia moral, si no sintieran vergüenza cuando se les coge mintiendo, si estas no fueran cualidades que la \emph{experiencia} descubre como inherentes a la naturaleza humana, jamás tendríamos la menor confianza en el testimonio humano. Un hombre que delira o que es conocido por su falsedad y vileza no tiene ninguna clase de autoridad entre nosotros}.

Así como nuestra habitual experiencia de la relación de causa y efecto nos permite hacer inferencias acerca de las cuestiones de hecho que están más allá de nuestros sentidos, la conformidad que usualmente experimentamos entre los hechos y el informe que un testigo nos da de ellos nos permite inferir su veracidad. Según el análisis ofrecido por Coady, la teoría de Hume: \blockquote[{\Cite[79]{coady1992test}}: \enquote{constitutes a reduction of testimony as a form of evidence or support to the status of a species (one might almost say, a mutation) of inductive inference. And, again, in so far as inductive inference is reduced by Hume to a species of observation and consequences attendant upon observations, then in a like fashion testimony meets the same fate}.]{constituye una reducción del testimonio como una forma de evidencia o fundamento al estatuto de una especie (uno podría casi decir, una mutación) de inferencia inductiva. Y, una vez más, en tanto que la inferencia inductiva queda reducida por Hume a una especie de observación y consecuencias relacionadas con las observaciones, en un modo similar, el testimonio corre la misma suerte} La valoración epistemológica del testimonio y la perspectiva ofrecida por Hume nos deja así con un primer desafío: \blockquote[{\Cite[294]{prades2015testimonio}}.]{en la vida social cabe aceptar un conocimiento por testimonio a condición de que su grado de certeza se limite a la probabilidad, y a condición de que pueda ser siempre reconducido a una verificación por conocimiento directo}.

Será interesante hacer notar aquí que el desafío expresado por Hume en la época moderna no deja de ser un reto en la época contemporánea. El mismo Coady lo constata cuando narra la acogida del tema del testimonio en los ámbitos en donde plantea la discusión: \blockquote[{\Cite[vii]{coady1992test}}: \enquote{When I began reading papers on the subject, my audiences mostly reacted with incomprehension, or the sort of disbelief evoked by denials of the merest common sense. Gradually, the climate of thought has changed and there is now more sympathy for the view that testimony is a prominent and underexplored epistemological landscape, although what sort of feature it is and how largely it looms are still naturally matters for disagreement}.]{Cuando comencé a ofrecer lecciones sobre este tema, las audiencias mayormente reaccionaban con incomprensión, o el tipo de incredulidad evocada por rechazos del más básico sentido común. Gradualmente, el clima del pensamiento ha cambiado y ahora hay más simpatía para el punto de vista de que el testimonio es un campo epistemológico prominente y poco explorado, aunque en qué tipo de rasgo consiste y con cuánta magnitud se impone son todavía cuestiones en debate}. De igual interés es también aquí la apreciación de Coady sobre las discusiones de Anscombe que le movieron a estudiar el testimonio: \blockquote[{\Cite[vii]{coady1992test}}: \enquote{I first began thinking about the epistemological status of testimony in the 1960s when writing a thesis at Oxford on issues in the theory of perception. \textelp{} I recall being intrigued by some remarks of Elizabeth Anscombe on the topic during her lectures on the empiricists}.]{Empecé por primera vez a pensar sobre la situación epistemológica del testimonio en los años 60 cuando escribía una tesis en Oxford sobre problemas en la teoría de la percepción. \textelp{} Recuerdo haber quedado intrigado por algunas afirmaciones de Elizabeth Anscombe sobre el tema durante sus lecciones sobre los empiristas}

Estas consideraciones añaden algunos elementos a nuestra cuestión inicial. Conocer una verdad para la vida desde el testimonio implica que pueda obtenerse una creencia verdadera justificada basada en lo que una persona ha comunicado. La visión de Hume es que la evidencia que puede ofrecer un testimonio para justificar una creencia no es mayor que la probabilidad y esta evidencia está basada en la inferencia que nos permite la habitual experiencia de que el testimonio comunicado y la verdad de los hechos suelen ir unidos. Más adelante veremos qué tiene que decir Anscombe ante este desafío.

\subsection{¿Hay justificación para valorar un hecho histórico como atestación divina?}

El contexto de la reflexión de Hume sobre el testimonio es precisamente el de la creencia en los milagros. La preocupación de Hume es que el `hombre sabio' pueda verificar sus creencias de modo que no sea víctima de `engaños supersticiosos'. Para esto, estima que ha encontrado un argumento que servirá para distinguir la superstición de la verdad\footnote{\cite[Cf.][\S10,1. 73]{hume1777enquiry}.
%: I flatter myself, that I have discovered an argument of a like nature, which, if just, will, with the wise and learned, be an everlasting check to all kinds of superstitious delusion, and consequently will be useful as long as the world endures.
}. Dice el filósofo escocés: \blockquote[{\Cite[\S10,1. 73-74]{hume1777enquiry}: \enquote{in our reasonings concerning matter of fact, there are all imaginable degrees of assurance, from the highest certainty to the lowest species of moral evidence. A wise man, therefore, proportions his belief to the evidence}}.]
%]
%{en nuestros razonamientos concernientes a cuestiones de hecho, se dan todos los grados imaginables de seguridad, desde la certeza más alta hasta las especies más bajas de evidencia moral. Un hombre sabio, por tanto, adecúa su creencia a la evidencia}.
{en nuestros razonamientos acerca de las cuestiones de hecho se dan todos los grados imaginables de seguridad, desde la máxima certeza hasta la clase más baja de certeza moral. Por tanto, un hombre sabio adecúa su creencia a la evidencia}. Entonces sugiere un criterio que permite ajustar las creencias a la evidencia: \blockquote[{\Cite[\S10,1. 77]{hume1777enquiry}: \enquote{That no testimony is sufficient to establish a miracle, unless the testimony be of such a kind, that its falsehood would be more miraculous than the fact which it endeavours to establish; and, even in that case, there is a mutual destruction of arguments; and the superior only gives us an assurance suitable to that degree of force which remains after deducting the inferior}}.]
%]{`Que ningún testimonio es suficiente para establecer un milagro, excepto si el testimonio es de tal tipo, que su falsedad sea más milagrosa que el hecho que se esfuerza por establecer; e, incluso en este caso, hay una mutua destrucción de argumentos; y el superior sólo nos da certeza apropiada al grado de fuerza que permanece después de restar el inferior.'}
{<<que ningún testimonio es suficiente para establecer un milagro, a no ser que el testimonio sea tal que su falsedad fuera más milagrosa que el hecho que intenta establecer; e incluso en este caso hay una destrucción mutua de argumentos, y el superior solo nos da una seguridad adecuada al grado de fuerza que queda después de deducir el inferior>>}.
\label{subsec:humarg}
Esto tiene como consecuencia que lo razonable sea abandonar la razonabilidad de las verdades cristianas, comprendiendo que solo pueden ser sostenidas por la fe. Argumenta que examinar si hay algún fundamento razonable para lo que creemos de la religión cristiana es \enquote{someterla a una prueba que no está capacitada para soportar} y, respecto de los hechos extraordinarios que la Escritura narra, hace la siguiente exhortación: \blockquote[{\Cite[\S10,2. 90]{hume1777enquiry}: \enquote{I desire any one to lay his hand upon his heart, and, after a serious consideration, declare, whether he thinks that the falsehood of such a book, supported by such a testimony, would be more extraordinary and miraculous than all the miracles it relates; which is, however, necessary to make it be received according to the measures of probability above established}}.]{Invito a cualquiera a que ponga su mano sobre el corazón y, tras seria consideración, declare si piensa que la falsedad de tal libro, apoyado por tal testimonio, sería más extraordinaria y milagrosa que todos los milagros que narra; lo cual sin embargo es necesario para que sea aceptado, de acuerdo con las medidas de probabilidad arriba establecidas}.
%: I am the better pleased with the method of reasoning here delivered, as I think it may serve to confound those dangerous friends, or disguised enemies to the Christian religion, who have undertaken to defend it by the principles of human reason. Our most holy religion is founded on faith, not on reason; and it is a sure method of exposing it, to put it to such a trial as it is by no means fitted to endure. To make this more evident, let us examine those miracles related in Scripture; and, not to lose ourselves in too wide a field, let us confine ourselves to such as we find in the Pentateuch, which we shall examine according to the principles of these pretended Christians, not as the word or testimony of God himself, but as the production of a mere human writer and historian. Here then we are first to consider a book, presented to us by a barbarous and ignorant people, written in an age when they were still more barbarous, and in all probability long after the facts which it relates, corroborated by no concurring testimony, and resembling those fabulous accounts which every nation gives of its origin. Upon reading this book, we find it full of prodigies and miracles. It gives an account of a state of the world and of human nature entirely different from the present: of our fall from that state; of the age of man extended to near a thousand years; of the destruction of the world by a deluge; of the arbitrary choice of one people, as the favourites of heaven, and that people the countrymen of the author; of their deliverance from bondage by prodigies the most astonishing imaginable. I desire any one to lay his hand upon his heart, and, after a serious consideration, declare, whether he thinks that the falsehood of such a book, supported by such a testimony, would be more extraordinary and miraculous than all the miracles it relates; which is, however, necessary to make it be received according to the measures of probability above established.]{Estoy más satisfecho con el método de razonar aquí expuesto, pues pienso que puede servir para confundir esos amigos peligrosos, o enemigos disfrazados de la religión Cristiana, que se han propuesto defenderla con los principios de la razón humana. Nuestra más sagrada religión se funda en la fe, no en la razón; y es un modo seguro de exponerla, el someterla a una prueba que de ningún modo está capacitada para soportar. Para hacer esto más evidente examinemos los milagros relatados en la escritura y, para no perdernos en un campo demasiado amplio, limitémonos a los que encontramos en el Pentatéuco, que examinaremos de acuerdo con los principios de aquellos supuestos Cristianos, no como la palabra o testimonio de Dios mismo, sino como la producción de un mero escritor e historiador humano. Aquí entonces hemos de considerar primero un libro que un pueblo bárbaro e ignorante nos presenta, escrito en una edad aún más bárbara y, con toda probabilidad, mucho después de los hechos que relata, no corroborado por testimonio concurrente alguno, y asemejándose a las narraciones fabulosas que toda nación da de su origen. Al leer este libro, lo encontramos lleno de prodigios y milagros. Ofrece un relato del estado del mundo y de la naturaleza humana totalmente distinto al presente: de nuestra pérdida de aquella condición; de la edad del hombre que alcanza a casi mil años; de la destrucción del mundo por un diluvio; de la elección arbitraria de un pueblo como el favorito del cielo y que dicho pueblo lo componen los compatriotas del autor; de su liberación de la servidumbre por los prodigios más asombrosos que se puede uno imaginar. Invito a cualquiera a que ponga su mano sobre el corazón, y, tras seria consideración, declare, si piensa que la falsedad de tal libro, apoyado por tal testimonio, sería más extraordinaria y milagrosa que todos los milagros que narra; lo cual, sin embargo, es necesario para que sea aceptado de acuerdo con las medidas de probabilidad arriba establecidas.}

¿Se puede afirmar que sería más `milagrosa' la falsedad de los milagros que atestigua la escritura? La posibilidad de recibir este testimonio como evidencia de alguna verdad descansaría sobre esta condición y una persona razonable debería medir la probabilidad de veracidad de estos relatos teniendo en cuenta que el estado de las cosas que describe es distinto al que experimentamos en el presente.

En una línea similar de pensamiento encontramos las reflexiones de G.\,E.~Lessing. Dos cuestiones expresadas en \emph{Sobre la demostración en Espíritu y Fuerza} merecen ser destacadas:
\blockquote[{\Cite[446]{lessing1982escritos:demo}}.]{Porque las noticias de profecías cumplidas no son profecías cumplidas, porque las noticias de milagros no son milagros. Las profecías que se cumplen ante mis ojos, los milagros que suceden ante mis ojos, influyen \emph{directamente}. Pero las noticias de profecías y milagros cumplidos, han de influir \emph{mediante} algo que les quita toda la fuerza}.

Lo que debería tener la fuerza para justificar la credibilidad queda debilitado por su medio de transmisión, entonces el problema es que \blockquote[{\Cite[446]{lessing1982escritos:demo}}.]{esa prueba en espíritu y fuerza ya no tiene ahora ni espíritu ni fuerza, sino que ha descendido a la categoría de testimonio humano sobre el espíritu y la fuerza}.

Tal como lo plantea Lessing y teniendo en cuenta el criterio propuesto por Hume, el testimonio, en tanto que dinamismo humano, no tiene fuerza suficiente para justificar razonablemente creencias sobre Dios como verdadero conocimiento. Esta objeción nos lleva a la siguiente: \blockquote[{\Cite[446]{lessing1982escritos:demo}}.]{las noticias de aquellas profecías y milagros son tan atendibles como puedan serlo en todo caso las verdades históricas \textelp{} Pero si \emph{sólo} pueden ser tan atendibles, ¿por qué al mismo tiempo se las hace de hecho infinitamente más atendibles? \textelp{} Si no puede demostrarse ninguna verdad histórica, tampoco podrá demostrarse nada \emph{por medio} de verdades históricas. Es decir: \emph{Las verdades históricas, como contingentes que son, no pueden servir de prueba de las verdades de razón como necesarias que son}}.

El punto que Lessing señala es infranqueable para él y para su intento de comprometerse con la verdad que la creencia cristiana pretende comunicar. La singularidad de la persona y obra de Jesús como manifestación de la realidad de Dios pierde para él toda su fuerza, puesto que no puede estimar estas verdades históricas como fundamento para una verdad necesaria como lo es la verdad de Dios. Esto nos deja con un problema adicional: \blockquote[{\Cite[294]{prades2015testimonio}}.]{no se puede tener conocimiento directo de milagros y profecías \textelp{} no se puede aceptar una comunicación divina que no sea inmediatamente dirigida al individuo}.

Este desafío viene a poner en cuestión que un hecho histórico de la vida personal o colectiva pueda ser estimado como testimonio del absoluto. La revelación de Dios por medio de testigos no es un fenómeno que tenga justificación razonable para su veracidad, y por tanto solo puede ser acogida por una fe desconectada de la razón.

\subsection{¿Tiene carácter `veritativo' el lenguaje teológico?}
\label{subsec:viena}
Un tercer punto de nuestra problemática está representado en la crítica al lenguaje religioso planteada por el Círculo de Viena. Este fenómeno cultural fue una corriente de renovación del positivismo y empirismo sostenido por el interés de univocidad semántica en los términos empleados por las ciencias, la búsqueda de rigor lógico-sintáctico en los sistemas científicos y un frenético intento de verificación empírica como justificación de las proposiciones `veritativas'\footnote{\Cite[Cf.][152]{dominguez2009at}. La expresión `veritativo' ha sido tomada de P. Dominguez que la emplea en \cite[155]{dominguez2009at}. F.Conesa utiliza la expresión análoga `valor cognoscitivo' en \Cite{conesa1994cc}.}. Desde la perspectiva de esta corriente, los discursos metafísicos, entre ellos la teología, eran considerados como una forma de especulación incontrolada.

En su \emph{Introduction to Wittgenstein's Tractatus}, Anscombe describe de modo general la actitud del Círculo como aplicación de una de las afirmaciones principales de esta obra: \blockquote[{\Cite[150]{anscombe1959iwt}}: \enquote{Probably the best known thesis of the \emph{Tractatus} is that `metaphysical' statements are nonsensical, and that the only sayable things are propositions of natural sciences (6.53). Now natural science is surely the sphere of the empirically discoverable; and the `empirically discoverable' is the same as `what can be verified by the senses'. The passage therefore suggests the following quick and easy way of dealing with `metaphysical' propositions: what sense-observations would verify and falsify them? If none, then they are senseless. This was the method of criticism adopted by the Vienna Circle and in this country by Professor A.J.Ayer}.]{Probablemente la tesis más conocida del \emph{Tractatus} es que las afirmaciones `metafísicas' no tienen sentido, y que las únicas cosas que pueden afirmarse son las proposiciones de las ciencias naturales (6.53). Ahora ciencia natural es ciertamente el ámbito de lo que puede ser descubierto empíricamente; y `lo que puede ser descubierto empíricamente' es lo mismo que `lo que puede ser verificado por los sentidos'. El pasaje entonces sugiere el siguiente modo fácil y rápido para lidiar con las proposiciones `metafísicas': ¿qué observaciones sensoriales las verificarían o falsificarían? Si no hay ninguna, entonces son sin-sentido. Este fue el método adoptado por el Círculo de Viena y en este país por el Profesor A.J.Ayer}.

Las expresiones de A.\,J. Ayer manifiestan la aplicación del método antes sugerido de modo que no solo no es posible demostrar la existencia de un Dios trascendente, sino incluso resulta imposible demostrar su probabilidad: \blockquote[{\Cite[Cf.][155]{dominguez2009at}}.]{Si la existencia de tal dios fuese probable, la proposición de que existiera sería una hipótesis empírica. Y, en ese caso, sería posible deducir de ella, y de otras hipótesis científicas, ciertas proposiciones experienciales que no fuesen deducibles de esas otras hipótesis solas. Pero, en realidad esto no es posible. \textelp{} Porque decir que ``Dios existe'' es realizar una expresión metafísica que no puede ser ni verdadera ni falsa. Y, según el mismo criterio, ninguna oración que pretenda describir la naturaleza de un Dios trascendente puede poseer ninguna significación literal}. Esta crítica, entonces, no se limita a cuestionar la justificación que pueda tener la creencia en Dios o las afirmaciones religiosas, sino que pone en duda la posibilidad de emplear este lenguaje como uno cuyas proposiciones comunican algún conocimiento: \blockquote[{\Cite[155]{dominguez2009at}}.]{La crítica del Círculo de Viena no se suma al ``Dios ha muerto'' de Nietzsche, sino que va aún más allá: lo que ha muerto es la misma palabra: ``Dios''. Nos encontramos ante lo que podemos considerar una nueva y refinada especie de ateísmo: el ateísmo semántico. Esta forma de ateísmo se sustenta en un equivocismo hermenéutico. No cabe comparar, arguyen los equivocistas, los nombres de supuestas realidades trascendentes con los de las realidades empíricas}.

Anscombe advierte, sin embargo que \blockquote[{\Cite[150]{anscombe1959iwt}}: \enquote{There are certain difficulties about ascribing this doctrine to the \emph{Tractatus}. There is nothing about sensible verification there}.]{Hay ciertas dificultades para adscribir esta doctrina al \emph{Tractatus}. No hay nada sobre verificación sensible ahí}. Ciertamente, a juicio de Anscombe, la metodología creada por el Círculo de Viena no se corresponde con la tesis del \emph{Tractatus}. Tampoco va en sintonía con los objetivos de Wittgenstein en su esfuerzo por purificar la metodología filosófica\footnote{\cite[Cf.][152]{anscombe1959iwt}: \enquote{`Psychology is no more akin to philosophy than any other natural science. Theory of knowledge is the philosophy of psychology' (4.1121). In this passage Wittgenstein is trying to break the dictatorial control over the rest of philosophy that had long been exercised by what is called theory of knowledge---that is, by the philosophy of sensation, perception, imagination, and, generally, of `experience'. He did not succeed. He and Frege avoided making theory of knowledge the cardinal theory of philosophy simply by cutting it dead; by doing none, and concentrating on the philosophy of logic. But the influence of the \emph{Tractatus} produced logical positivism, whose main doctrine is `verificationism'; and in that doctrine theory of knowledge once more reigned supreme, and a prominent position was given to the test for significance by asking for the observations that would verify a statement}.}.

%Dice Anscombe: \blockquote[{\cite[152]{anscombe1959iwt}}: \enquote{`Psychology is no more akin to philosophy than any other natural science. Theory of knowledge is the philosophy of psychology' (4.1121). In this passage Wittgenstein is trying to break the dictatorial control over the rest of philosophy that had long been exercised by what is called theory of knowledge---that is, by the philosophy of sensation, perception, imagination, and, generally, of `experience'. He did not succeed. He and Frege avoided making theory of knowledge the cardinal theory of philosophy simply by cutting it dead; by doing none, and concentrating on the philosophy of logic. But the influence of the \emph{Tractatus} produced logical positivism, whose main doctrine is `verificationism'; and in that doctrine theory of knowledge once more reigned supreme, and a prominent position was given to the test for significance by asking for the observations that would verify a statement.}]{`La psicología no es más semejante a la filosofía que cualquier otra ciencia natural. La teoría del conocimiento es filosofía de la psicología' (4.1121). En este pasaje Wittgenstein esta tratando de romper el control dictatorial sobre el resto de la filosofía que por largo tiempo ha sido ejercido por lo que se llama teoría del conocimiento\,---\,esto es, por la filosofía de la sensación, percepción, imaginación, y, en general, de la experiencia. No tuvo éxito. Él y Frege evitaron hacer de la teoría del conocimiento la teoría cardinal de la filosofía simplemente al no alimentarla; al no hacer ninguna, y concentrándose en la filosofía de la lógica. Sin embargo la influencia del \emph{Tractatus} produjo el positivismo lógico, cuya doctrina principal es el `verificacionismo'; y en esa doctrina la teoría del conocimiento una vez más reinó, y se le dio una posición prominente a la prueba sobre la significación requiriendo observaciones que pudieran verificar una afirmación}.

La influencia del Círculo de Viena, sin embargo, fue notable y las posturas de las reflexiones sucesivas fueron diversas. A. Flew propuso que dado que el lenguaje teológico no es falseable, tampoco es susceptible de afirmar alguna verdad o conocimiento proposicional\footcite[Cf.][27-30]{conesa1994cc}. R.\,M. Hare consideró el lenguaje religioso como evocativo, más que informativo\footcite[Cf.][35-36]{conesa1994cc}. Van Buren consideró artificial la posibilidad de un antagonismo entre la Ciencia y la Teología puesto que: \blockquote[{\Cite[156]{dominguez2009at}}.]{el lenguaje de la Ciencia y el de la Teología pertenecen a dos ámbitos tan distintos entre sí ---equívocos--- que al carecer de una semántica común, hasta la rivalidad resultaría artificial. Poniendo un ejemplo analógico: igual que no es posible oponer ``voltios'' a ``sentimientos'', no es posible hacer entrar en conflicto la Ciencia con la Metafísica. ¿Es en verdad esto sostenible?}

Los desafíos que representan las discusiones del Círculo de Viena vienen a ofrecernos la pregunta \enquote*{¿es cognoscitivo el lenguaje religioso?}. Esto no es una pregunta sobre si es significativo como lo pudiera ser el lenguaje poético o mítico, sino específicamente si es susceptible de ser verdadero o falso. ¿Existe un conocimiento religioso? ¿Cuál es su valor?\footcite[Cf.][23]{conesa1994cc}. La pregunta se dirige específicamente hacia el lenguaje del testimonio. ¿Puede significar algo? ¿Puede comunicar un conocimiento? Un ejemplo propuesto por Anscombe tiene que ver con la ocasión de enseñar a un niño sobre la transubstanciación, para ello es útil señalar lo que ocurre y decir cómo está haciéndose presente Jesús y cómo hemos de reaccionar. Al hacer esto \blockquote[{\Cite[21]{conesa1994cc}}.]{le está enseñando una técnica, a la vez que le abre a un modo de relación con Dios y le enseña parte del mensaje revelado. Estos modos de conocimiento no solo están vinculados, sino también en una íntima relación: el saber proposicional conduce a conocer, este a saber obrar, y viceversa}.

Investigaremos respuestas y discusiones en torno a estas cuestiones problemáticas de la categoría del testimonio en el trabajo de Elizabeth Anscombe, pero antes de entrar en este análisis resultará útil hacer un recorrido general por su vida, obra y pensamiento.


%\section{Naturaleza de la pregunta sobre el testimonio}
  Es una experiencia familiar en nuestras comunidades reunirnos en torno a la
  Sagrada Escritura y compartir la Palabra buscando en ella luz para nuestro
  presente. Una escena evangélica en torno a la cual muchos se han reunido a
  escuchar al Señor es la narración de Mateo del comienzo de la misión pública de
  Jesús y la llamada de los primeros discípulos:

  \citalitlar{Al enterarse Jesús de que habían arrestado a Juan se retiró a
    Galilea. Dejando Nazaret se estableció en Cafarnaún, junto al mar, en el
    territorio de Zabulón y Neftalí, para que se cumpliera lo dicho por medio del
    profeta Isaías:\\
    <<Tierra de Zabulón y tierra de Neftalí, camino del mar, al otro lado del
    Jordán, Galilea de los gentiles. El pueblo que habitaba en tinieblas vio una
    luz grande; a los que habitaban en tierra y sombras de
    muerte, una luz les brilló>>.\\
    Desde entonces comenzó Jesús a predicar diciendo: <<Convertíos, porque está
    cerca el reino de los cielos>>.\\
    Paseando junto al mar de Galilea vio a dos hermanos, a Simón, llamado Pedro, y
    a Andrés, que estaban echando la red en el mar, pues eran pescadores. Les
    dijo: <<Venid en pos de mí y os haré pescadores de hombres>>. Inmediatamente
    dejaron las redes y lo siguieron. Y pasando adelante vio a otros dos hermanos,
    a Santiago, hijo de Zebedeo, y a Juan, su hermano, que estaban en la barca
    repasando las redes con Zebedeo, su padre, y los llamó. Inmediatamente dejaron
    la barca y a su padre y lo siguieron.\footnote{Mt~4,12--22}}

  No sería difícil ahora visualizar una variedad de escenarios en los que este
  texto pueda ser discutido en nuestro contexto eclesial. Es proclamado, por
  ejemplo, en el ciclo A el III Domingo del Tiempo Ordinario. Es así que puede
  escucharse en las reflexiones del Papa Francisco en el Ángelus en la Plaza de
  San Pedro, donde destaca el hecho de que la misión de Jesús comience en una zona
  periférica:
  \citalitlar{Es una tierra de frontera, una zona de tránsito donde se
    encuentran personas diversas por raza, cultura y religión. La Galilea se
    convierte así en el lugar simbólico para la apertura del Evangelio a todos
    los pueblos. Desde este punto de vista, Galilea se asemeja al mundo de hoy:
    presencia simultánea de diversas culturas, necesidad de confrontación y
    necesidad de encuentro. También nosotros estamos inmersos cada día en una
    <<Galilea de los gentiles>>, y en este tipo de contexto podemos asustarnos y
    ceder a la tentación de construir recintos para estar más seguros, más
    protegidos. Pero Jesús nos enseña que la Buena Noticia, que Él trae, no está
    reservada a una parte de la humanidad, sino que se ha de comunicar a todos.
    Es un feliz anuncio destinado a quienes lo esperan, pero también a quienes
    tal vez ya no esperan nada y no tienen ni siquiera la fuerza de buscar y
    pedir.\autocite{francisco2014angelus}}

  Tambíen el Papa Benedicto XVI ofreció su comentario y se fijó en la fuerza de esa
  noticia que Cristo comenzaba a anunciar:
  \citalitlar{El término ``evangelio'', en tiempos de Jesús, lo usaban los
    emperadores romanos para sus proclamas. Independientemente de su contenido, se
    definían ``buenas nuevas'', es decir, anuncios de salvación, porque el
    emperador era considerado el señor del mundo, y sus edictos, buenos presagios.
    Por eso, aplicar esta palabra a la predicación de Jesús asumió un sentido
    fuertemente crítico, como para decir: Dios, no el emperador, es el Señor del
    mundo, y el
    verdadero Evangelio es el de Jesucristo.\\
    La ``buena nueva'' que Jesús proclama se resume en estas palabras: ``El reino
    de Dios —--o reino de los cielos-—- está cerca''. ¿Qué significa esta expresión?
    Ciertamente, no indica un reino terreno, delimitado en el espacio y en el
    tiempo; anuncia que Dios es quien reina, que Dios es el Señor,
    y que su señorío está presente, es actual, se está realizando.\\
    Por tanto, la novedad del mensaje de Cristo es que en él Dios se ha hecho
    cercano, que ya reina en medio de nosotros, como lo demuestran los milagros y
    las curaciones que realiza.\autocite{benedicto2008angelus}}

  No sólo en San Pedro, sino que también podría encontrarse este texto en la
  celebración de la eucaristía domincal resonando en las comunidades y parroquias;
  en las homilias, oraciones, reflexiones o cánticos, invitando a la conversión y
  haciendo nueva la invitación de Jesús: <<Convertíos, porque está cerca el reino
  de los cielos>>. Quizás tambíen se le oiga entre algún grupo juvenil donde
  Simón, Andrés, Santiago y Juan sean tratados como modelos de vocación a la vida
  consagrada o al apostolado, atendiendo con entusiasmo cómo lo dejaron todo en el
  momento para seguir a Jesús. Seguramente algún joven reconociendo aquella
  llamada: <<Venid en pos de mí y os haré pescadores de hombres>> sonando como voz
  dentro de sí.

  El texto de la Escritura es tratado en estos contextos como testimonio de la
  vida de Jesucristo y de la vida de aquellos que le llaman maestro y que
  participan de su misión. No son, sin embargo, tratados como historias del
  pasado, sino como palabras para el presente. Es hoy que la Buena Noticia no está
  reservada a una parte de la humanidad, sino que ha de comunicarse a todos como
  insiste el Papa Francisco. Es hoy que Dios se hace cercano en Cristo para reinar
  en medio de nosotros como enseñó Benedicto XVI. Es hoy que Jesús nos invita a la
  conversión y a ir en pos de él.

  Es sobre esta costumbre de la Iglesia que ha de formularse ahora una pregunta.
  Resultará apropiado apelar aquí a otra costumbre de la Iglesia y buscar luz para
  esto en las Confesiones de San Agustín. Pensando en Dios y pensando en el
  tiempo, Agustín queda inquieto por una serie de preguntas: \citalitlar{¿Qué es,
    pues, el tiempo? ¿Quién podrá explicar esto fácil y brevemente? ¿Quién podrá
    comprenderlo con el pensamiento, para hablar luego de él? Y, sin embargo, ¿qué
    cosa más familiar y conocida mentamos en nuestras conversaciones que el
    tiempo? Y cuando hablamos de él, sabemos sin duda qué es, como sabemos o
    entendemos lo que es cuando lo oímos pronunciar a otro. ¿Qué es, pues, el
    tiempo? Si nadie me lo pregunta, lo sé; pero si quiero explicárselo al que me
    lo pregunta, no lo sé.\footnote{De las confesiones xi.14 (n. 17)}}

  Agustín expresa su extrañeza de que un concepto empleado ordinariamente se
  torne tan desconocido cuando llega la hora de explicarlo. ``¿Qué es el
  tiempo?'' o ``¿qué es conocer?'', ``¿la libertad?'' y ``¿qué es la fe?'' son
  preguntas de este tipo; distintas, por ejemplo, a ``¿cuál es el peso exacto de
  este objeto?'' o ``¿quién será la próxima persona en entrar por esa
  puerta?''.\footnote{cf. Wittgenstein BT. p.304} Preguntar ``¿qué es conocer una
  verdad para la vida por el testimonio de la Escritura?'' sería, como la pregunta
  agustiniana sobre el tiempo, una pregunta sobre la naturaleza o esencia de
  este fenómeno. Un concepto familiar en la vida de la Iglesia como el
  testimonio queda enmarcado como problema cuando nos acercamos a él queriendo
  comprender su esencia.

  Para continuar explorando la naturaleza de la pregunta sobre el testimonio
  resultará útil recurrir aquí al modo en que el psicólogo William James formula
  algunas preguntas sobre la Escritura al comienzo de sus conferencias sobre la
  \emph{religion natural}. Apelando a la literatura de lógica de su época a
  comienzos del siglo XX distingue dos niveles de investigación sobre cualquier
  tema: aquellas preguntas que se resuelven por medio de prposiciones
  \emph{existenciales}, como ``¿qué constitución, qué origen, qué historia tiene
  esto?'' o ``¿cómo se ha realizado esto?''; en segundo lugar las preguntas que se
  responden con proposiciones de \emph{valor} como ``¿cuál es la importancia,
  sentido o significado actual de esto?''. A este segundo juicio James lo denomina
  \emph{juicio espiritual}. Aplicando esta distinción a la Biblia se cuestiona:

  \citalitlar{ <<¿Bajo qué condiciones biográficas los escritores sagrados aportan
    sus diferentes contribuciones al volumen sacro?>>, <<¿Cúal era exactamente el
    contenido intelectual de sus declaraciones en cada caso particular?>>. Por
    supuesto, éstas son preguntas sobre hechos históricos y no vemos cómo las
    respuestas pueden resolver, de súbito, la última pregunta: <<¿De qué modo este
    libro, que nace de la forma descrita, puede ser una guía para nuestra vida y
    una revelación?>>. Para contestar habríamos de poseer alguna teoría general
    que nos mostrara con qué peculiaridades ha de contar una cosa para adquirir
    valor en lo que concierne a la revelación; y, en ella misma, tal teoría sería
    lo que antes hemos denominado un juicio espiritual.\footnote{William James
      Variedades de la Experiencia Religiosa p. 27} }

  Desde esta perspectiva la pregunta sobre cómo el testimonio de la escritura
  puede ser una guía para nuestra vida es una investigación sobre la importancia,
  sentido o significado que ésta tiene actualmente. La respuesta emitida en
  conclusión sería un juicio de valor sobre el fenómeno del testimonio. James
  propone que sería necesaria una teoría general que explicara qué características
  ha de tener alguna cosa para que merezca ser valorada como revelación. Así
  planteado, la pregunta sobre el testimonio sería atendida adecuadamente por
  medio de una investigación que indagara dentro de este fenómeno para descubrir
  los elementos que le otorgan el valor adecuado como para ser considerado guía
  para nuestra vida o una revelación. La explicación de dichos elementos
  configurarían una teoría que nos permitiría juzgar un testimonio concreto como
  valioso, o no, como guía o revelación para nuestras vidas.

  La ruta sugerida por este modo de conducir la investigación, sin embargo, nos
  dejaría apartados de la manera en que Elizabeth Anscombe se plantea un problema
  filosófico. En el trasfondo de su metodología filosófica está la propuesta por
  Ludwig Wittgenstein. Aunque se verá con más detalle qué implica esto, es
  necesario anticipar ahora algo acerca del modo en que ambos se encaminan a la
  hora de atender una investigación filosófica.

  En \emph{Investigaciones Filosóficas} \S89 Wittgenstein hace referencia al
  texto antes citado de las Confesiones para describir la peculiaridad de las
  preguntas filosóficas:
  \citalitlar{ Augustine says in \emph{Confessions} XI. 14, ``quid est ergo
    tempus? si nemo ex me quaerat scio; si quaerenti explicare velim nescio''.
    --This could not be said about a question of natural science (``What is the
    specific gravity of hydrogen'', for instance). Something that one knows when
    nobody asks one but no longer knows when one is asked to explain it, is
    something that has to be \emph{called to mind}. (And it is obviously
    something which, for some reason, it is difficult to call to mind.)}

  Para Wittgenstein es de gran importancia atender el paso que damos para
  resolver la perplejidad causada por el reclamo de explicar un fenómeno. El
  deseo de aclararlo nos puede impulsar a buscar una explicación dentro del
  fenómeno mismo, o como él diría: \citalitinterlin{We feel as if we had to see
    right into phenomena}.\footnote{\S90} Esta predisposición nos puede conducir
  a ignorar la amplitud del modo en que el lenguaje sobre esto es empleado en la
  actividad humana y a enfocarnos sólo en un elemento particular del lenguaje
  sobre este fenómeno y tomarlo como un ejemplo paradigmático para construir un
  modelo abstrayendo explicaciones y generalizaciones sobre él. Esta manera de
  indagar, le parece a Wittgenstein, nos hunde cada vez más profundamente en un
  estado de frustración y confusión filosófica de modo que llegamos a imaginar
  que para alcanzar claridad \citalitinterlin{we have to describe extreme
    subtleties, which again we are quite unable to describe with the means at
    our disposal. We feel as if we had to repair a torn spider's web with our
    fingers.}\footnote{\S106}

  La alternativa que Wittgenstein propone es una investigación que no esté
  dirigida hacia dentro del fenómeno, sino \citalitinterlin{as one might say,
    towards the \emph{`possibilities'} of phenomena. What that means is that we
    call to mind the \emph{kinds of statement} that we make about phenomena}. A
  este esfuerzo le denomina ``investigación gramática''. La describe de este modo:
  \citalitlar{ Our inquiry is therefore a grammatical one. And this inquiry sheds
    light on our problem by clearing misunderstandings away. Misunderstandings
    concerning the use of words, brought about, among other things, by certain
    analogies between the forms of expression in different regions of our
    language. -- Some of them can be removed by substituting one form of
    expression for another; this may be called `analysing' our forms of
    expression, for sometimes this procedure resembles taking things
    apart.\footnote{\S90}} El modo de salir de nuestra perplejidad, por tanto,
  consiste en prestar cuidadosa atención al uso que hacemos de hecho con las
  palabras y la aplicación que empleamos de las expresiones. Esto está al
  descubierto en nuestro uso del lenguaje de modo que la dificultad para
  \emph{traer a la mente} aquello que aclare un fenómeno no está en descubrir algo
  oculto en éste, sino en aprender a valorar lo que tenemos ante nuestra vista:
  \citalitinterlin{The aspects of things that are most important for us are hidden
    because of their simplicity and familiarity. (One is unable to notice
    something -- because it is always before one's eyes.)}\footnote{\S129} La
  descripción de los hechos concernientes al uso del lenguaje en nuestra actividad
  humana ordinaria componen los pasos del tipo de investigación sugerido por
  Wittgenstein. Hay cierta insatisfacción en este modo de proceder, como él mismo
  afirma: \citalitlar{Where does this investigation get its importance from, given
    that it seems only to destroy everything interesting: that is, all that is
    great and important? (As it were, all the buildings, leaving behind only bits
    of stone and rubble.) But what we are destroying are only houses of cards, and
    we are
    clearing up the ground of language on which they stood.\\
    The results of philosophy are the discovery of some piece of plain nonsense
    and the bumps that the understanding has got running up against the limit of
    language. They -- these bumps -- make us see the value of that discovery.}

  Anscombe, al igual que Wittgenstein, no se limita a emplear un sólo método para
  hacer filosofía, como afirma el mismo Wittgenstein: \citalitinterlin{There is
    not a single philosophical method, though there are indeed methods, different
    therapies as it were}.\footnote{\S133} Sin embargo si atendemos a su modo de
  hacer filosofía podemos encontrarla empleando lenguajes o juegos de lenguaje
  imaginarios para arrojar luz sobre modos actuales de usar el lenguaje o esquemas
  conceptuales; del mismo modo su trabajo esta lleno de ejemplos donde la
  encontramos examinando con detenimiento el uso que de hecho hacemos del
  lenguaje.\footnote{cf. teichmann p. 228-229} Es visible en ella ese
  \citalitinterlin{modo característicamente Wittgensteniano de rebatir la
    tendencia del filósofo de explicar alguna cuestión filosóficamente enigmática
    inventando una entidad o evento que la causa, así como los físicos inventan
    partículas como el gravitón}.\footnote{There is however a somehow
    chracteristically Wittgenstenian way of countering the philosopher's tendency
    to explain a philosophically puzzling thing by inventing an entity or event
    which causes it, as physicists invent particles like the graviton. From plato
    to witt intro xix}

  Según el título de este trabajo ha prometido, el análisis sobre el testimonio
  que será expuesto es el que se encuentra desarrollado en el pensamiento de
  Elizabeth Anscombe. La pregunta planteada al inicio: ¿qué es conocer una verdad
  para la vida por el testimonio de la Escritura?, entendida como investigación
  filosófica, será examinada en las descripiciones que Anscombe realiza sobre el
  modo de usar el lenguaje sobre el creer, la confianza, la verdad, la fe y otros
  fenómenos relacionados con el conocer por testimonio. Nuestro título adiverte
  además que ésta es una investigación en perspectiva teólogica, cabe
  inmendiatamente añadir algo breve al respecto.

  ¿Qué es teología?, se preguntaba Joseph Ratzinger en su alocución en el 75
  aniversario del nacimiento del cardenal Hermann Volk en 1978, e introducía
  suscintamente su respuesta a esa pregunta tan grande diciendo:

  \citalitlar{Cuando se intenta decir algo sobre esta materia, precisamente como
    tributo al cardenal Volk y a su pensamiento, se asocian, poco menos que
    automáticamente, dos ideas. Me viene a las mientes, por un lado, su divisa (y
    título de uno de sus libros): \emph{Dios todo en todos}, y el programa
    espiritual contenido en ella; por otra parte, se aviva el recuerdo de lo que
    ya antes se ha insinuado: un modo de interrogar total y absolutamente
    filosófico, que no se detiene en reales o supuestas comprobaciones históricas,
    en diagnósticos sociológicos o en técnicas pastorales, sino que se lanza
    implacablemente a la busqueda de los fundamentos.\\
    Según esto, cabría formular ya dos tesis que pueden servirnos de hilo
    conductor para nuestro interrogante sobre la esencia de la teología:\\
    1. La teología se refiere a Dios.\\
    2. El pensamiento teológico está vinculado al modo de cuestionar filosófico
    como a su método fundamental.\footnote{teoría de los principios teológicos, p
      380}}
  Esta investigación sobre el testimonio como parte de la vida de la Iglesia será
  realizada atendiendo al modo de cuestionar filosófico realizado por Elizabeth
  Anscombe como método, examinando esta experiencia en referencia a Dios, es
  decir, como vivencia de su ser y de su obrar.

  Hasta aquí simplemente se ha descrito un modo de andar a través de la discusión
  acerca de la categoría del testimonio atendiendo el hecho de que tanto la
  temática como la figura de Anscombe otorgan a este camino peculiaridades que hay
  que tener en cuenta. Siendo concientes de estas particularidades podríamos ahora
  ampliar más el horizonte respecto de dos cuestiones brevemente expuestas
  anteriormente. En primer lugar es necesario ampliar la descripción hecha hasta
  aquí del fenómeno del testimonio en la vida de la Iglesia, ya que aunque nos
  resulte familiar relacionarlo con el testimonio de la Sagrada Escritura, tanto
  en el Magisterio de la Iglesia como en la propia Escritura se haya presente la
  categoría del testimonio con una riqueza que merece la pena explorar. En segundo
  lugar habría que detallar todavía mejor lo problemático del testimonio, sobre
  todo cuando se considera su importancia en la transmisión de la fe y el anuncio
  del Evangelio en el mundo.

\section{La Categoría del Testimonio en la Sagrada Escritura}
La Iglesia de hoy, como María, conserva el Evangelio meditándolo en su
corazón.\footnote{Lc 2,19} Así está presente en el centro de la comunidad
creyente el anuncio de Cristo vivo como fundamento de su esperanza en cada etapa
de la historia. Este motivo de esperanza conservado es también compartido y
expresado, según la enseñanza del apóstol:\citalitinterlin{glorificad a Cristo
  en vuestros corazones, dispuestos siempre a dar explicación a todo el que os
  pida una razón de vuestra esperanza}.\footnote{1Pe 3, 15} Este Evangelio
atesorado como fundamento en el centro de la vida de la comunidad eclesial, así
como Buena Nueva proclamada y transmitida en el tiempo y en el mundo puede ser
comprendido como tres testimonios que son uno:<<palabra vivida en el
Espíritu>>\footnote{cf. Porque es el Espíritu el que impulsa a la Iglesia a
  perseguir son obras de evangelización; es el Espíritu quien santifica y
  fecunda el testimonio de su vida; y es el Espíritu el que inspira la fe, la
  nutre y la profundiza. Es el Espíritu quien alivia entre estos tres
  testimonios que son uno: el de la palabra vivida en el Espíritu. A través del
  testimonio, el Espíritu internaliza el testimonio externo de la Buena Nueva de
  la salvación en Jesucristo y lo lleva al cumplimiento de la fe, que es la
  respuesta del amor del verdadero amor de la humanidad a través del Padre.
  Cristo; Latourelle Evangelisation et temoignage ninot 582}.

La Evangelización puede ser entendida en este sentido como testimonio de la
<<palabra de vida>>\footnote{1Jn 1,1} que los apóstoles anuncian como testigos
de lo que han contemplado y palpado\footnote{1Jn 1,1}. Es también el testimonio
de los cristianos que, acogiendo esta palabra, la viven,
poniendo por obra lo que ella enseña. Es además testimonio del Espíritu Santo
que internaliza el testimonio externo de la Buena Noticia y lo lleva al
cumplimiento de la fe en cada persona.\footnote{cf. latourelle, ninot 582} Es el
Espíritu el que santifica y fecunda la acción de los cristianos, es tambíen el
que impulsa y sostiene la acción de la Iglesia; es el Espíritu el que inspira la
fe, la nutre y la profundiza.\footnote{latourelle evangelisation et temoignage}

Este dinamísmo fundamental que puede encontrarse vivo hoy en la comunidad de la
Iglesia ha actuado en ella desde su origen y le ha acompañado en cada época.
Según esto es posible valorar lo que se transmite en la tradición eclesial como
la perpetuación de la actividad de Cristo y los apóstoles, que es a su vez
proyección del testimonio divino.\footnote{ el testimonio divino se proyecta
  luego en el apostólico y se perpetúa en el testimonio eclesial. Por eso, el
  testimonio es revelación en la actividad de Cristo y de los apóstoles y es
  transmisión de la revelación en la tradición eclesial. ninot 573}

En la actividad de Cristo el testimonio divino queda proyectado como
interpelación a la libertad realizada por la identidad propia de Jesús:
\citalitinterlin{Si conocieras el don de Dios y quién es el que te dice ``dame
  de beber'' le pedirías tu, y él te daría agua viva}\footnote{Jn 4, 10};
\citalitinterlin{``¿Crees tú en el Hijo del hombre?''\ldots ``¿Y quién es,
  Señor, para que crea en él?''\ldots ``Lo estás viendo: el que te está
  hablando, ese es''}\footnote{Jn 9, 35--37}. En la actividad apostólica, el
testimonio divino sigue interpelando la libertad humana como manifestación de
Jesús Resucitado. Los apóstoles actuan como testigos de los acontecimientos de
la Pascua de Jesús y su valor salvífico\autocite[Cf.][576]{ninot2009tf} y este
testimonio es descrito como acción del Espíritu que impulsa la tarea apostólica
y que da nueva vida a los que acogen el anuncio de la Buena Noticia.

Puede encontrarse un ejemplo de esto en el testimonio de Felipe. El apóstol sale
más allá de Jerusalén hacia Samaria, y todavía llega más lejos, al compartir la
Buena Noticia de Jesús con un extranjero Etíope: \citalitlar{El Espíritu dijo a
  Felipe: <<Acércate y pégate a la carroza>>. Felipe se acercó corriendo, le oyó
  leer el profeta Isaías, y le preguntó: <<¿Entiendes lo que estás leyendo?>>.
  Contestó: <<¿Y cómo voy a entenderlo si nadie me guía?>>. E invitó a Felipe a
  subir y a sentarse con él. El pasaje de la Escritura que estaba leyendo era
  este: \emph{Como cordero fue llevado al matadero, como oveja muda ante el
    esquilador, así no abre su boca. En su humillación no se le hizo justicia.
    ¿Quién podrá contar su descendencia? Pues su vida ha sido arrancada de la
    tierra.} El eunuco preguntó a Felipe: <<Por favor, ¿de quién dice esto el
  profeta?; ¿de él mismo o de otro?>>. Felipe se puso a hablarle y, tomando pie
  de este pasaje, le anunció la Buena Nueva de Jesús. Continuando el camino,
  llegaron a un sitio donde había agua, y dijo el eunuco: «Mira, agua. ¿Qué
  dificultad hay en que me bautice?». Mandó parar la carroza, bajaron los dos al
  agua, Felipe y el eunuco, y lo bautizó. Cuando salieron del agua, el Espíritu
  del Señor arrebató a Felipe. El eunuco no volvió a verlo, y siguió su camino
  lleno de alegría. \footnote{Hch 8, 29--39}} Además de ser ejemplo de la
actividad apostólica, este relato puede servir como síntesis del modo en que la
categoría del testimonio está presente en la Escritura.

El testimonio comienza con la iniciativa de Dios mismo que impulsa tanto la
palabra profética del Antiguo Testamento como el anuncio apostólico del Nuevo
Testamento. Esta iniciativa de Dios tiende hacia el testimonio de la Palabra
definitiva del Padre que es Cristo resucitado. En aquellos que creen en el
testimonio de Dios se engendra alegría y vida nueva. En palabras de R.
Latourelle:
\citalitlar{En el trato de las tres personas divinas con los hombres existe un
  intercambio de testimonios que tiene la finalidad de proponer la revelación y
  de alimentar la fe. Son tres los que revelan o dan testimonio, y esos tres son
  más que uno. Cristo da testimonio del Padre, mientras que el Padre y el
  Espíritu dan testimonio del Hijo. Los apóstoles a su vez dan testimonio de lo
  que han visto y oído del verbo de la vida. Pero su testimonio no es la
  comunicación de una ideología, de un descubrimiento científico, de una técnica
  inédita, sino la proclamación de la salvación prometida y finalmente
  realizada.\autocite[1531]{latourelle2000testimonio}}
De este modo el anuncio del apóstol Felipe sirve aquí como un ejemplo específico
del testimonio, que ilustra sin embargo, una noción
que\citalitinterlin{atraviesa toda la Escritura y se corresponde con la
  estructura misma de la revelación.}\footnote{la noción de testimonio atraviesa
  la Escritura y se coresponde con la estructura misma de la Revelación: <<la
  Escritura describe la revelación como una economía del testimonio>>.
  \autocite[109]{prades2015testimonio}} El testimonio está presente a lo largo
de la Escritura junto a otras categorías como pueden ser la de `alianza',
`palabra', `paternidad' o `filiación', como parte del \citalitinterlin{grupo de
  analogías empleadas por la Escritura para introducir al hombre en las riquezas
  del misterio divino}.\footnote{latourelle p. 1523}

Esta clave servirá para dar enfoque a un examen sobre la categoría del
testimonio en la Escritura. ¿Qué nos dice el Antiguo y el Nuevo Testamento de la
revelación como acto testimonial de Dios? Esta pregunta supone que la revelación
comparte los rasgos de la actividad humana que es el testimonio, sin embargo,
como Latourelle adiverte: \citalitinterlin{globalmente se puede decir que el
  testimonio bíblico asume pero al mismo tiempo exalta hasta sublimarlos, los
  rasgos del testimonio humano.}\footnote{cf. latourelle 1526 Globalmente se
  puede decir que el testimonio bíblico asume pero al mismo tiempo exalta hasta
  sublimarlos, los rasgos del testimonio humano. latourelle 1526}

Cabe añadir una última consideración. La revelación de Dios entendida como acto
testimonial suyo tiene como expresión definitiva el misterio pascual de
Cristo.\footnote{cf. el misterio pascual al cual tiende toda la existencia
  terrena de Cristo, constituye el acto testimonial por excelencia de Dios
  prades 128} Este misterio ocupa el lugar principal en el testimonio bíblico:
\citalitlar{la Resurrección como ``final'' de la unicidad del acontecimiento de
  Jesucristo, encarnado, muerto y resucitado, subraya específicamente la
  definitividad de la existencia humana salvada por Dios en la carne de Jesús de
  Nazaret, ya que la autocomunicación de Dios ha alcanzado su palabra última en
  la Resurrección de Jesucristo, y por eso es prenda de la resurrección de todos
  los hombres.\footnote{ninot 404}}
Como tal, parece justo tratar el testimonio que es el misterio pascual en su
propio apartado. Y será éste precisamente el punto de partida para esta
descripción de la categoría del testimonio en la Escritura.

\subsection{El testimonio en el misterio y anuncio pascual}

<<Cristo ha resucitado>>\footnote{Cf.~1Tes 4,15; 1Cor 15,12--20; Rom 6,4} es la
confesión que está en el núcleo del más primitivo anuncio del
evangelio.\autocite[Cf.][403]{ninot2009tf} Creer en esta noticia conlleva acoger la
manifestación más plena de la Revelación y la motivación más definitiva para
creer. En este sentido:
\citalitlar{La Resurrección de Jesús mirada desde la perspectiva de la teología
  fundamental presupone un estatuto epistemológico peculiar, puesto que es el
  punto culminante y objeto de la Revelación y, a su vez, es su acreditación
  suprema y máximo motivo de credibilidad, tal como recuerda el texto citado de
  Pablo ``si Cristo no ha resucitado, nuestra predicación es vana y vana es
  nuestra fe'' (1 Cor 15,14).\autocite[405]{ninot2009tf}}

Este misterio pascual no aparece como hecho desconectado del conjunto de la vida
y misión de Jesús, sino que hacia él tienden sus obras y palabras desde el
comienzo. Cristo pasó por el mundo haciendo el bien, como testimonio de la
bondad de Dios, y esta acción va orientada a ese punto culminante que es su
pasión, muerte y resurrección; \citalitinterlin{el testimonio que Jesús va
  ofreciendo durante su vida pública le va a reclamar una entrega definitiva a
  favor de los que lo han acogido y frente a la resistencia que ha generado en
  quienes le rechazan.}\autocite[127]{prades2015testimonio}

A lo largo de este camino Jesús manifiesta su confianza en el Padre:
\citalitinterlin{Padre, te doy gracias porque me has escuchado; yo sé que tu me
  escuchas siempre}\footnote{Jn 11, 41b-42a}; esta relación queda afirmada
plenamente ante la pasión como confianza puesta en su voluntad:
\citalitinterlin{Padre\ldots que no se haga mi voluntad, sino la
  tuya}\footnote{Lc 22,42}. De este modo en el misterio pascual queda
atestiguada la plena unidad de Cristo con el Padre, en la mayor confianza
imaginable.\footnote{prades 127}

A lo largo de su misión, Cristo dió testimonio del amor del Padre
\citalitinterlin{habiendo amado a los suyos que estaban en el
  mundo\ldots}\footnote{Jn 13,1}. En el misterio pascual, donde
\citalitinterlin{los amó hasta el extremo}\footnote{Jn 13, 1}, queda confirmado
definitivamente como testigo del Padre. Con su entrega ofrece el testimonio
pleno del amor salvador del Padre: \citalitinterlin{Porque tanto amó Dios al
  mundo, que entregó a su Unigénito, para que todo el que cree en él no perezca,
  sino que tenga vida eterna.}\footnote{Jn 3,16}

A lo largo de su vida, Cristo también es testigo de la necesidad del camino
salvífico que es libre e irrevocable decisión trinitaria de redimir a los
hombres\footnote{prades 128}: \citalitinterlin{¿No sabíais que yo debía estar en
  las cosas de mi Padre?}\footnote{Lc 2, 49}; \citalitinterlin{El hijo del
  hombre tiene que padecer mucho, ser reprobado por los ancianos, sumos
  sacerdotes y escribas, ser ejecutado y resucitar a los tres días.}\footnote{Mc
  8, 31} Este testimonio de la voluntad divina es comprendido por los discípulos
por la luz del Resucitado; \citalitinterlin{les abrió el entendimiento para
  comprender las Escrituras\ldots ``así está escrito: el Mesías padecerá,
  resucitaráde entre los muertos al tercer día y en su nombre se proclamará la
  conversión''}.\footnote{Lc 24, 45-47a}

La intencionalidad de este testimonio que Jesús ofrece a lo largo de su vida
hasta llegar al acto testimonial definitivo de Dios al mundo que es el misterio
pascual aparece con claridad en la respuesta de Cristo a Pilato antes de la
Pasión: \citalitinterlin{Yo para esto he nacido y para esto he venido al mundo:
  para dar testimonio de la verdad. Todo el que es de la verdad escucha mi
  voz.}\footnote{Jn 18,37} En su vida pública y en su misión Cristo ha actuado
como profeta que anuncia la verdad; da a conocer al Padre, a quien nadie ha
visto nunca, pero que el Hijo sí conoce.\footnote{cf. Jn 1,18 vease también
  Jesús de Nazaret 24} En el misterio pascual Jesús se manifiesta como verdadero
profeta, acreditado por el hecho mismo de la Resurrección donde se ha realizado
en él mismo lo que ha revelado y prometido. \footnote{prades 128}

La resurrección de Cristo no sólo acredita su propio testimonio, sino que
sostiene el testimonio apostólico. Si Cristo no ha resucitado sería vana
cualquier argumentación, sin embargo, Jesús es <<el Viviente>>, estuvo muerto,
pero vive por los siglos de los siglos.\footnote{Ap 1, 17--18}

Los apóstoles son testigos de la vida de Cristo, de sus palabras y acciones,
muerte y resurrección. De tal modo, son testigos en continuidad con el testimonio
de Cristo. El testimonio apostólico es un anuncio de estos hechos que ellos
conocen y cuyo valor han reconocido por la fe. Así Pedro proclama estas cosas el
día de Pentecostés: \citalitinterlin{A este Jesús lo resucitó Dios, de lo cual
  todos nosotros somos testigos}.\footnote{Hch 2, 32} El apóstol es testigo en
la fe sobre un acontecimiento enraizado en la historia.\footnote{ninot 402 y 406
  enraizado}

Así mismo es presentado el testimonio de Pedro en casa de Cornelio donde el
centurión y todos lo que lo acompañaban esperaban reunidos para escuchar lo que
el Señor quisiera comunicarles por medio del apóstol. Pedro, comprendiendo que
la verdad de Dios no hace acepción de personas, narra los hechos que él bien
conoce: \citalitlar{<<Vosotros conocéis lo que sucedió en toda Judea, comenzando
  por Galilea, después del bautismo que predicó Juan. Me refiero a Jesús de
  Nazaret, ungido por Dios con la fuerza del Espíritu Santo, que pasó haciendo
  el bien y curando a todos los oprimidos por el diablo, porque Dios estaba con
  él. Nosotros somos testigos de todo lo que hizo en la tierra de los judíos y
  en Jerusalén. A este lo mataron, colgándolo de un madero. Pero Dios lo
  resucitó al tercer día y le concedió la gracia de manifestarse, no a todo el
  pueblo, sino a los testigos designados por Dios: a nosotros, que hemos comido
  y bebido con él después de su resurrección de entre los
  muertos.>>\footnote{Hch 10,37--41}} Este testimonio de los hechos queda
enlazado con un testimonio de fe sobre el sentido profundo de lo que Pedro
conoce, Jesús, a quien los apóstoles y el pueblo vieron y escucharon, es ahora
juez de vivos y muertos:
\citalitlar{<<Nos encargó predicar al pueblo, dando solemne testimonio de que
  Dios lo ha constituido juez de vivos y muertos. De él dan testimonio todos los
  profetas: que todos los que creen en él reciben, por su nombre, el perdón de
  los pecados.>>\footnote{Hch 10,42--43}}

El apóstol entiende estos hechos y su alcance religioso y salvífico
interpretándolos en continuidad con la voluntad de Dios manifestada en su acción
en favor del pueblo judío a quién habló por medio de los profetas; voluntad
hecha manifiesta en definitva en \citalitinterlin{Jesús el Nazareno, varón
  acreditado por Dios ante vosotros con los milagros, prodigios y signos que
  Dios realizó por medio de él, como vosotros mismos sabéis}.\footnote{Hch 2,22}

Este anuncio es experiencia del Resucitado que comió y bebió con ellos;
él mismo se apareció a los que él quiso dando testimonio de su
resurrección. \citalitinterlin{Cristo glorificado manifiesta su verdad a los que
  él quiere y esta manifestación es simultaneamente testimonio de su identidad y
  testimonio de que él es la Vida (1Jn 5,11)}\autocite[129]{prades2015testimonio}

El misterio divino que se manifiesta en la Pascua de Jesús no deja de expresarse
en el anuncio pascual realizado por los apóstoles. Ellos son testigos de un
hecho enraizado en la historia, que tiene un alcance religioso y salvífico y que
es interpretado desde la voluntad de Dios manifestada en los hechos y palabras
de Cristo. Sin las obras que Jesús realizó, el testimonio apostólico se
derrumba, no existe.\autocite[Cf.][1529]{latourelle2000testimonio} Sin la vida y
obra, muerte y resurrección de Jesús \citalitinterlin{resultamos unos falsos
  testigos de Dios, porque hemos dado testimonio contra él, diciendo que ha
  resucitado a Cristo, a quien no ha resucitado}.\footnote{1Cor 15,15}

En Cristo, testigo acreditado por su Resurrección, encuentra su cumplimiento
la promesa hecha al pueblo de Israel: \citalitinterlin{El Señor, tu Dios, te
  suscitará de entre los tuyos, de entre tus hermanos, un profeta como yo. A
  él lo escucharéis}.\footnote{Dt 18, 15 véase intro Jesús de Nazaret} Así
como el misterio pascual y su anuncio no están desconectados de la vida de
Cristo, tampoco lo están de la acción salvadora de Dios en el AT. Como veremos,
el misterio divino se manifiesta a un pueblo que también está llamado a dar
testimonio, reconociendo desde la confianza en Dios el valor salvífico de los
sucesos de su historia.



\subsection{La acción testimonial de Dios en el Antiguo Testamento}

En el AT encontramos ese <<intercambio de testimonios>> que existe en el trato
de las tres personas divinas con los
hombres.\autocite[Cf.][1531]{latourelle2000testimonio} También aquí la acción
testimonial divina se despliega de diversos modos. En la vida del pueblo de la
alianza YHWH da testimonio de sí a través de la creación, la ley y, de modo
eminente, en personas elegidas y enviadas por
él.\autocite[Cf.][114s]{prades2015testimonio} Esta manifestiación divina implica
como testigo al mismo pueblo, hacia quien ha sido dirigida la voz del Señor.

La literatura sapiencial recoge la profundización en la experiencia de
Dios que ha tenido el pueblo de Israel. En ella se describe el acceso posible al
conocimiento de Dios a partir de los bienes visibles o de sus obras:
\citalitlar{Son necios por naturaleza todos los hombres que han ignorado a Dios
  y no han sido capaces de conocer al que es a partir de los bienes visibles, ni
  de reconocer al artífice fijándose en sus obras, sino que tuvieron por dioses
  al fuego, al viento, al aire ligero, a la bóveda estrellada, al agua impetuosa
  y a los luceros del cielo, regidores del mundo. Si, cautivados por su
  hermosura, los creyeron dioses, sepan cuánto los aventaja su Señor, pues los
  creó el mismo autor de la belleza. Y si los asombró su poder y energía,
  calculen cuánto más poderoso es quien los hizo, pues por la grandeza y
  hermosura de las criaturas se descubre por analogía a su creador.\footnote{Sab
    13,1--5}}
El Dios que puede ser reconocido por analogía en el asombro y belleza de las
ciraturas es un Dios personal que concede sabiduría al
piadoso:\citalitinterlin{Aún quedan misterios mucho más grandes: tan solo hemos
  visto algo de sus obras. Porque el Señor lo ha hecho todo y a los piadosos les
  ha dado la sabiduría.}\footnote{Eclo 43,32--33} Esta sabiduria es justicia y
raíz de inmortalidad:
\citalitlar{Pero tú, Dios nuestro, eres bueno y fiel, eres paciente y todo lo
  gobiernas con misericordia. Aunque pequemos, somos tuyos y reconocemos tu
  poder, pero no pecaremos, sabiendo que te pertenecemos. Conocerte a ti es
  justicia perfecta y reconocer tu poder es la raíz de la
  inmortalidad.\footnote{Sab 15,1--3}}
En este sentido la misma creación es acto testimonial de Dios donde se comunica
su misterio y la vida que Él ofrece.

YHWH también aparece en el AT como testigo de los mandamientos contenidos en la
Ley.\autocite[Cf.][115]{prades2015testimonio} Ésta queda grabada en las ``tablas
del testimonio'' y confiadas a Moisés:\citalitinterlin{Cuando acabó de hablar
  con Moisés en la montaña del Sinaí, le dio las dos tablas del Testimonio,
  tablas de piedra escritas por el dedo de Dios.}\footnote{Ex 31,18} Este
testimonio se enfrenta a un pueblo con el corazón extraviado:\citalitinterlin{Al
  acercarse al campamento y ver el becerro y las danzas, Moisés, encendido en
  ira, tiró las tablas y las rompió al pie de la montaña.}\footnote{Ex 32,19}
Sin embargo Dios no se detiene ante la dureza del pueblo. Las tablas del
testimonio son reconstruidas:
\citalitlar{El Señor dijo a Moisés: <<Labra dos tablas de piedra como las
  primeras y yo escribiré en ellas las palabras que había en las primeras tablas
  que tú rompiste.>>\ldots~<<Escribe estas palabras: de acuerdo con estas
  palabras concierto alianza contigo y con Israel>>.\footnote{Ex 34,1.27}}
Moisés, que conoció el nombre del Señor (Ex 3,13s), y habló con Él como un amigo
(Ex 33,11), aparece ante el pueblo como testigo del único Dios, y de su lealtad
con el pueblo. Pertenece a
aquellos que el Señor elige como testigos suyos en cada etapa de la historia del
pueblo de Israel como testimonio suyo y de su fidelidad.

Este es el modo eminente en que el AT describe el testimonio que Dios dirige al
pueblo. Los profetas y ungidos por YHWH son testigos del Señor y de su
compromiso con el pueblo. La vida totalmente comprometida del profeta expresa
tanto a Dios, absoluto que comunica, como su lealtad:
\citalitlar{es Dios quien da testimonio de sí mismo y de sus obras y designios a
  través de las personas elegidas, que se comprometen en su integridad como
  testigos de YHWH incluso hasta la muerte si el testimonio les lleva a ello.
  Por eso, la autoridad del testimonio no descansa en los testigos, sino en el
  mismo YHWH, que es quien los escoge y
  envía.\autocite[116s]{prades2015testimonio}}
En tanto que testigos, la acción de estos escogidos puede ser descrita según los
rasgos que tiene la actividad humana de dar testimonio, sin embargo la noción de
testigo que aplica a estos elegidos de Dios va más allá de la que encontraríamos
en el lenguaje ordinario. La vida del profeta queda comprometida con un
testimonio que no le pertenece, sino que \citalitinterlin{procede de una
  iniciativa absoluta, en cuanto a su origen y en cuanto a su
  contenido}\autocite[118]{prades2015testimonio} puesto que viene de Dios y es
testimonio de sí. Aquí la categoría de testimonio significa mas allá de su uso
ordinario en la actividad humana y adquiere un sentido religioso como dimensión
totalmente nueva\autocite[Cf.][118]{prades2015testimonio}.

El testimonio de YHWH que el profeta proclama con su actividad y el compromiso
de su vida implica al pueblo y le hace testigo:
\citalitlar{Saca afuera a un pueblo que tiene ojos, pero está ciego, que tiene
  oídos, pero está sordo. Que todas las naciones se congreguen y todos los
  pueblos se reúnan. ¿Quién de entre ellos podría anunciar esto, o proclamar los
  hechos antiguos? Que presenten sus testigos para justificarse, que los oigan y
  digan: es verdad. Vosotros sois mis testigos --—oráculo del Señor--—, y
  también mi siervo, al que yo escogí, para que sepáis y creáis y comprendáis
  que yo soy Dios. Antes de mí no había sido formado ningún dios, ni lo habrá
  después. Yo, yo soy el Señor, fuera de mí no hay salvador. Yo lo anuncié y os
  salvé; lo anuncié y no hubo entre vosotros dios extranjero. Vosotros sois mis
  testigos --—oráculo del Señor--—: yo soy Dios.\footnote{Is 43,8--12}}
El siervo es testigo que el Señor ha escogido para que el pueblo sepa, crea y
comprenda que YHWH es el único Dios verdadero. Al compartir este saber de Dios
con el pueblo, éstos también están llamados a ser testigos. Ninguna otra nación
podría anunciar como ellos lo que YHWH ha hecho para proveer, liberar, salvar.

Así como el profeta, el pueblo es escogido y enviado por YHWH y por medio de él
el Señor da testimonio de sí mismo y se propone como quien da sentido y
consistencia a toda la realidad humana. Este testimonio tiene importancia social
puesto que está llamado a ser proclamado, y esta proclamación implica el
compromiso de los actos y la vida del testigo, es decir, del profeta y todo el
pueblo.\autocite[Cf.][1526s]{latourelle2000testimonio}

El testimonio de Dios a través de personas escogidas por Él en el AT queda
constituido por la narración de hechos que acontecen en la historia, estos
hechos son interpretados en su valor absoluto y carácter redentor, y son
confesados como actuación de Dios en la vida
humana.\autocite[Cf.][119]{prades2015testimonio} Esto vuelve a ponernos en
conexión con la figura de Cristo como profeta acreditado por su Resurrección y
los apóstoles como verdaderos testigos de un hecho enraizado en la historia,
confesado desde la fe e interpretado desde la acción de Dios en Jesús. Esta
sintonía anticipa lo que se verá a continuación sobre el testimonio en el Nuevo
Testamento. En él la acción testimonial de Dios se describe en continuidad con
la tradición veterotestamentaria y llegará a su manifestación plena en el
misterio pascual.

\subsection{La acción testimonial de Dios en el Nuevo Testamento}
El Evangelio de Mateo enseña que el día que Jesús llegó a Cafarnaún a comenzar
su predicación se cumplieron las promesas que Dios había hecho por medio de los
profetas. Ese día el Reino de los cielos quedó desvelado en su cercanía. Allí la
vida de los primeros discípulos cambió al punto y definitivamente. El testimonio
de Cristo no es cualquier anuncio o cualquier hecho, sino que tiene un valor
absoluto. Jesús de Nazaret \citalitinterlin{no se limita a proponer una cierta
  inspiración espiritual o un cierto sentido ético para el obrar de la persona o
  del pueblo, sino que pretende ser radicalmente ``testimonio de la verdad'' (Jn
  18,37) de alcance universal.}\autocite[126]{prades2015testimonio} Jesús es
testimonio de carácter singular,\autocite[Cf.][279]{ninot2009tf} en quien se da
a conocer el momento de la plenitud de la
salvación,\autocite[Cf.][290]{ninot2009tf} presencia del hombre nuevo y
``paradigma universal de humanidad''\autocite[Cf.][291]{ninot2009tf}. Este valor
universal de la verdad que se comunica en Jesús se desarrolla y se manifiesta en
sus acciones concretas: comiendo con los pecadores o sanando a los enfermos es
donde se muestra \citalitinterlin{el camino, la verdad y la
  vida}\footnote{Cf.~Jn 14,6} para todos.

Este testimonio de Cristo, su vida, actos y palabras, fue sometido al juicio de
sus contemporaneos. Asombrados porque no enseña como los demás y por las signos
que realiza, se cuestionan sobre su autoridad y poder. Entonces Jesús también
tiene que ofrecer testimonio de su credibilidad. La respuesta a este juicio del
pueblo se halla en su ministerio en sintonía con las Escrituras:
\citalitinterlin{Hoy se ha cumplido esta Escritura que acabáis de
  oir}\footnote{Lc 4,21}; donde el pueblo puede encontrar la vida y el sentido
que buscan: \citalitinterlin{estudiáis las Escrituras pensando encontrar en
  ellas vida eterna; pues ellas están dando testimonio de mi, ¡y no queréis
  venir a mí para tener vida!}\footnote{Jn 5, 39--40}. El testimonio de
credibilidad de Jesús ante el pueblo se encuentra también en sus obras, que son
las obras del Padre y son confirmación y realización de sus enseñanzas:
\citalitinterlin{Si no hago las obras de mi Padre, no me creáis, pero si las
  hago, aunque no me creáis a mí, creed a las obras, para que comprendáis y
  sepáis que el Padre está en mí y yo en el Padre}\footnote{Jn 10,38}.

El singular testimonio de Cristo es comunicación de la verdad con valor
universal. El testimonio de Cristo es también su actividad e identidad que hacen
creíble lo que comunica. De este modo entre lo que Jesús testimonia y la
credibilidad que suscita su testimonio hay una circularidad constante:
\citalitlar{La pretensión única que encerraba su testimonio resultaba tan
  exorbitante que hubiera sido inaceptable para los hombres si no fuera porque
  sus obras, sus palabras y, en rigor, su presencia misma, lo hacían
  profundamente razonable en su
  singularidad.\autocite[124]{prades2015testimonio}}

Acoger el testimonio de Jesús es escuchar la Escritura y creer en las obras del
Padre. Sin embargo la palabra de Cristo choca con el odio de aquellos que son
hostiles a la verdad y que, rechazando su testimonio, se juzgan a sí
mismos.\footnote{Pero la palabra de Cristo choca con la contestación y el odio.
  Enfrentados con Cristo, los judíos, que representan al conjunto del mundo
  hostil a la verdad, rechazan su testimonio y se juzgan a sí
  mismos.\autocite[1530]{latourelle2000testimonio}} \citalitlar{Si yo no hubiera
  venido y no les hubiera hablado, no tendrían pecado, pero ahora no tienen
  excusas de su pecado. El que me odia a mí, odia también a mi Padre. Si yo no
  hubiera hecho en medio de ellos obras que ningún otro ha hecho, no tendrían
  pecado, pero ahora las han visto y me han odiado a mí y a mi Padre\footnote{Jn
    15,22--24}}

Jesús es \citalitinterlin{la luz que brilla en la tiniebla y la tiniebla no la
  recibió}\footnote{Jn 1,5}. Jesús es el \citalitinterlin{unigénito, que está en
  el seno del Padre, es quien lo ha dado a conocer}\footnote{Jn 1,18}. Este
testimonio es manifestación de la comunión trinitaria. Cristo revela al Padre y
comunica al Espíritu, y su identidad de Hijo es manifestada como acción del
Padre y del Espíritu: \citalitinterlin{Apenas se bautizó Jesús, salió del agua;
  se abrieron los cielos y vio que el Espíritu de Dios bajaba como una paloma y
  se posaba sobre él. Y vino una voz de los cielos que decía: <<Este es mi Hijo
  amado, en quien me complazco>>.}\footnote{Mt 4,16--17}

La acción testimonial de Dios que se describe en el Nuevo Testamento está
concentrada en la persona de Cristo y en su relación manifiesta con el Padre y
el Espíritu se expresa el testimonio de la Trinidad misma:
\citalitlar{la Escritura describe la actividad reveladora de la trinidad en
  forma de testimonios mutuos. El Hijo es el testigo del padre, y como tal se da
  a conocer a los apóstoles. A su vez, el Padre da también testimonio de que
  Cristo es el Hijo, por la atracción que produce en las almas, por las obras
  que da al Hijo para que las realice y sobre todo por la resurrección,
  testimonio decisivo del Padre en favor del Hijo. El Hijo da testimonio del
  Espíritu porque promete enviarlo como educador, consolador, santificador. Y el
  Espíritu viene y da testimonio del hijo porque le recuerda, le da a conocer,
  descubre la plenitud de sentido de sus palabras, lo insinua en las
  almas.\footnote{de Latourelle, Teología de la Revelación 410
    en\autocite[131]{prades2015testimonio}}}
Esta actividad reveladora de la trinidad introduce al ser humano en la comunión
trinitaria. Dios trino se comunica al ser humano actuando en su interior,
atrayendo, inspirando; también se comunica externamente por las obras que
realiza. Esta participación en la comunión divina viene bien expresada en la
finalidad del testimonio apostólico: \citalitinterlin{Eso que hemos visto y oído
  os lo anunciamos, para que estéis en comunión con nosotros y nuestra comunión
  es con el Padre y con su Hijo Jesucristo.}\footnote{1Jn 1,3}

Jesús es el fundamento, testigo fiel y veraz para todo tiempo y
lugar.\autocite[Cf.][132]{prades2015testimonio} Creer su testimonio es acoger al
absoluto en la historia, esta confianza la hace posible el Espíritu:
\citalitlar{Cristo es, por tanto, el testigo absoluto, el que lleva en sí mismo
  la garantía de su testimonio. El hombre, sin embargo, no sería capaz de acoger
  por la fe este testimonio del absoluto, manifestado en la carne y el lenguaje
  de Jesús, sin una atracción interior (Jn 6,44), que es un don del Padre y un
  testimonio del Espíritu (1Jn 5,9--10).\autocite{latourelle2000testimonio}}
Aquellos que creen en Cristo no sólo encuentran una respuesta a su busqueda de
vida y sentido, sino que \citalitinterlin{de sus entrañas manarán ríos de agua
  viva}.\footnote{Jn 7,38} Y esto Jesús lo dice \citalitinterlin{refiriéndose al
  Espíritu que habían de recibir los que creyeran en él}.\footnote{Jn 7,39} Esta
promesa del Espíritu acontece en Pentecostés y sin ese testimonio postpascual
del Espíritu quedaría incompleta la comunicación de Dios en el misterio
Pascual.\autocite[Cf.][135]{prades2015testimonio} El envío y la acción del
Espíritu prometido completa la acción testimonial de Dios:
\citalitlar{Al haber ``acompañado'' al Hijo en la tierra de una manera singular
  desde el momento de su unción en el Jordán, que dispone al Hijo ---concebido
  por obra del Espíritu Santo--- para la misión en la carne, el Espíritu Santo
  vueve al Padre portando en sí todo el misterio redentor del Hijo. De este
  modo, cuando el Resucitado lo envía a la Iglesia, el Espíritu vuelve como
  Testigo de la verdad completa, que incluye la perfecta glorificación de la
  carne del Hijo como plenitud de lo
  humano.\autocite[134s]{prades2015testimonio}}

El Espíritu enviado por Cristo lleva a la verdad plena a los apóstoles:
\citalitinterlin{cuando venga él, el Espíritu de la verdad, os guiará hasta la
  verdad plena. Pues no hablará por cuenta propia, sino que hablará de lo que
  oye y os comunicará lo que está por venir}.\footnote{Jn 16,13} Este testimonio
del Espíritu completa tambíen el testimonio de los apóstoles:
\citalitinterlin{Cuando venga el Paráclito, que os enviaré desde el Padre, el
  Espíritu de la verdad, que procede del Padre, él dará testimonio de mí; y
  también vosotros daréis testimonio, porque desde el principio estáis
  conmigo}.\footnote{Jn 15,26--27} Ellos han estado desde el principio con
Cristo, así son testigos que pueden narrar lo que han visto y oído; su testimono
queda perfeccionado por el Espíritu que les introduce en el misterio del Hijo
encarnado y les permite interpretar y comprender la verdad del Hijo, y por éste,
la del Padre.\autocite[Cf.][139]{prades2015testimonio}

Los que han compartido con Jesús desde el principio son testigos del Evangelio,
pero el Resucitado sigue eligiendo apóstoles y en virtud de la acción del
Espíritu éstos son testigos del mismo misterio.\autocite[Cf.][576]{ninot2009tf}
Así Matías no sólo es \citalitinterlin{uno de los que nos acompañaron todo el
  tiempo que convivió con nosotros el Señor Jesús}\footnote{Hch 1,21}, sino que
es elegido por el Resucitado.\footnote{Cf.~Hch 1,24--26} Igualmente Pablo es
constituido testigo por la llamada del Resucitado, asi puede decir
\citalitinterlin{Yo mismo hermanos cuando vine a vosotros anunciaros el
  testimonio de Dios\ldots}\footnote{1\,Cor 2,1}. De este modo la transmisión
viva del testimonio cristiano esta constituida por un momento fundacional en la
convivencia con Jesús y un momento continuante como dos aspectos históricos
inseparables.\autocite[Cf.][148]{prades2015testimonio} Este momento continuante
esta compuesto por los que han sido testigos oculares, como por los que no:
\citalitinterlin{unos y otrs son elegidos, llamados y enviados por Cristo, el
  Cristo histórico los primeros y el Cristo glorioso los
  segundos}.\autocite[148]{prades2015testimonio}

Aquel que recibe este testimonio y cree en él encuentra la vida nueva. <<¿Qué
dificultad hay en que me bautice?>>, decide aquel hombre que recibió el
testimonio de Felipe y <<siguió su camino lleno de alegría>> después de haber
encontrado a Dios. Considerar la revelación divina como acción testimonial de
Dios conduce en definitiva a estimar la revelación misma como forma de amor y
libertad de Dios que interpela el amor y libertad humano. En tanto que
comunicación libre y amorosa, el testimonio de Dios atiende la naturaleza humana
de su beneficiario; en tanto que don divino queda desvelado su origen y meta más
allá de lo humano.\autocite[Cf.][152]{prades2015testimonio}

\section{Iglesia como signo sacramental el Testimonio en el Magisterio Reciente}
Nuestro recorrido comenzó al inicio de este capítulo tomando como punto de
partida a la Iglesia como signo visibile. La vida de la comunidad eclesial, sus
costumbres y actitudes, son presencia histórica y realidad perceptible. La
Iglesia puede ser reconocida hoy actuando según su costumbre de reunirse en
torno a la Palabra de Dios para celebrarla y conocer la verdad para su vida. Lo
que se vive hoy y se ha transmitido en la tradición eclesial lo hemos valorado
como perpetuación de la actividad de Cristo y de los apóstoles y, por tanto,
como proyección del testimonio divino. En este sentido hemos considerado la
presencia de la Revelación divina en el corazón y anuncio de la Iglesia como
triple testimonio usando la expresión de Latourelle: <<palabra vivida en el
Espíritu>>. Esta reflexión ha querido servir para describir la naturaleza de la
Revelación como experiencia familiar en la vida de la Iglesia. La noción de la
categoría del testimonio que atraviesa la escritura ha servido para valorar la
naturaleza de la Revelación según su estructura testimonial.

Así como la categoría del testimonio ha servido para decir algo sobre la
Revelación en la Escritura, ahora se pretende decir algo sobre lo que la
categoría del testimonio puede aportar para comprender la identidad de la
Iglesia y su misión en el mundo y cómo ésta forma parte del dinamismo de la
Revelación divina.

Con Latourelle se ha dicho que el testimonio es una de esas categorías que la
escritura emplea como analogía para introducirnos al misterio divino. El
Concilio nos regala otra analogía que va de la mano con la categoría del
testimonio en la comprensión de la Iglesia y su misión:
\citalitlar{la sociedad provista de sus órganos jerárquicos y el Cuerpo místico de
Cristo, la asamblea visible y la comunidad espiritual, la Iglesia terrestre y la
Iglesia enriquecida con los bienes celestiales, no deben ser consideradas como
dos cosas distintas, sino que más bien forman una realidad compleja que está
integrada de un elemento humano y otro divino. Por eso se la compara, por una
notable analogía, al misterio del Verbo encarnado, pues así como la naturaleza
asumida sirve al Verbo divino como de instrumento vivo de salvación unido
indisolublemente a Él, de modo semejante la articulación social de la Iglesia
sirve al Espíritu Santo\footnote{LG 8}}

La visibilidad de la Iglesia está al servicio del Espíritu Santo de modo que su
naturaleza humana sirve a la presencia divina como instrumento vivo de
salvación. La presencia de la articulación social de la Iglesia actua de manera
análoga a la presencia de Cristo. Según esto se puede decir que
\citalitinterlin{la eclesiología se resuelve en la Cristología y por eso el
  ``lugar'' de la Iglesia en el acto de creer será ``análogo'' al de
  Cristo}.\autocite[566]{ninot2009tf} Esta relación con Cristo y el Espíritu
otorgan a la Iglesia valor sacramental:
\citalitlar{Porque Cristo, levantado sobre la tierra, atrajo hacia sí a todos
  (cf. Jn 12, 32 gr.); habiendo resucitado de entre los muertos (Rm 6, 9), envió
  sobre los discípulos a su Espíritu vivificador, y por El hizo a su Cuerpo, que
  es la Iglesia, sacramento universal de salvación; estando sentado a la derecha
  del Padre, actúa sin cesar en el mundo para conducir a los hombres a la
  Iglesia y, por medio de ella, unirlos a sí más estrechamente y para hacerlos
  partícipes de su vida gloriosa alimentándolos con su cuerpo y sangre. Así que
  la restauración prometida que esperamos, ya comenzó en Cristo, es impulsada
  con la misión del Espíritu Santo y por Él continúa en la Iglesia, en la cual
  por la fe somos instruidos también acerca del sentido de nuestra vida
  temporal, mientras que con la esperanza de los bienes futuros llevamos a cabo
  la obra que el Padre nos encomendó en el mundo y labramos nuestra salvación
  (cf. Flp 2, 12).\footnote{LG 48}}
Esta Iglesia que es sacramento es mediación de acción salvadora de Dios;
comunica los dones de la gracia y manifiesta el misterio de Dios:
\citalitlar{Todo el bien que el Pueblo de Dios puede dar a la familia humana al
  tiempo de su peregrinación en la tierra, deriva del hecho de que la Iglesia es
  "sacramento universal de salvación", que manifiesta y al mismo tiempo realiza
  el misterio del amor de Dios al hombre.\footnote{GS 45}}

La Iglesia en el mundo es así uno de los signos contenidos en la Revelación que
ayudan a la razón que busca la comprensión del misterio. El signo sacramental
que es la Iglesia permite atestiguar desde la fe el misterio de Dios que en ella
se expresa del mismo modo que ocurre con la Eucaristía o la presencia de Cristo
encarnado:
\citalitlar{Podemos fijarnos, en cierto modo, en el horizonte sacramental de la
  Revelación y, en particular, en el signo eucarístico donde la unidad
  inseparable entre la realidad y su significado permite captar la profundidad
  del misterio. Cristo en la Eucaristía está verdaderamente presente y vivo, y
  actúa con su Espíritu, pero como acertadamente decía Santo Tomás, <<lo que no
  comprendes y no ves, lo atestigua una fe viva, fuera de todo el orden de la
  naturaleza. Lo que aparece es un signo: esconde en el misterio realidades
  sublimes>>. A este respecto escribe el filósofo Pascal: <<Como Jesucristo
  permaneció desconocido entre los hombres, del mismo modo su verdad permanece,
  entre las opiniones comunes, sin diferencia exterior. Así queda la Eucaristía
  entre el pan común>>.\footnote{FR 13}}
El misterio sublime que aparece en un signo puede ser atestiguado por la fe
viva. El asentimiento al signo sacramental por la fe supone el reconocimiento de
que viene de Dios y por tanto es creer a quien es garante de su propia verdad.
Este asentimiento implica a la persona por completo:
\citalitlar{Desde la fe el hombre da su asentimiento a ese testimonio divino.
  Ello quiere decir que reconoce plena e integralmente la verdad de lo revelado,
  porque Dios mismo es su garante. Esta verdad, ofrecida al hombre y que él no
  puede exigir, se inserta en el horizonte de la comunicación interpersonal e
  impulsa a la razón a abrirse a la misma y a acoger su sentido profundo. Por
  esto el acto con el que uno confía en Dios siempre ha sido considerado por la
  Iglesia como un momento de elección fundamental, en la cual está implicada
  toda la persona. Inteligencia y voluntad desarrollan al máximo su naturaleza
  espiritual para permitir que el sujeto cumpla un acto en el cual la libertad
  personal se vive de modo pleno.\footnote{FR 13}}
La acogida del misterio divino comunicado en el signo sacramental es así un acto
de libertad plena que no sólo permite reconocer el misterio de Dios, sino que
nos desvela nuestra vocación de comunión con Él, que es nuestro sentido más
auténtico:
\citalitlar{El conocimiento de fe, en definitiva, no anula el misterio; sólo lo
  hace más evidente y lo manifiesta como hecho esencial para la vida del hombre:
  Cristo, el Señor, <<en la misma revelación del misterio del Padre y de su
  amor, manifiesta plenamente el hombre al propio hombre y le descubre la
  grandeza de su vocación>>, que es participar en el misterio de la vida
  trinitaria de Dios.\footnote{FR 13}}

La Iglesia es signo sacramental unido inseparablemente al misterio divino que
comunica, de modo análogo a la unión del Verbo divino y la naturaleza asumida
por Él. El conocimiento de la fe abre la razón humana a la verdad revelada como
comunicación interpersonal de Dios realizada por medio de este signo sacramental
que es la Iglesia. Este acto de confianza es movimiento de la libertad como
asentimiento y elección de Dios que se revela y acogida de su llamada a
participar de la comunión trinitaria. Aquí sacramento y testimonio son
categorías que interactuan para describir el acceso al misterio divino que se
comunica a través de signos. Esta Iglesia que es signo sacramental es signo
creíble por el testimonio de la vida comprometida con el misterio de amor que
significa: \citalitlar{La misión primera y fundamental que recibimos de los
  santos Misterios que celebramos es la de dar testimonio con nuestra vida. El
  asombro por el don que Dios nos ha hecho en Cristo infunde en nuestra vida un
  dinamismo nuevo, comprometiéndonos a ser testigos de su amor. Nos convertimos
  en testigos cuando, por nuestras acciones, palabras y modo de ser, aparece
  Otro y se comunica. Se puede decir que el testimonio es el medio con el que la
  verdad del amor de Dios llega al hombre en la historia, invitándolo a acoger
  libremente esta novedad radical. En el testimonio Dios, por así decir, se
  expone al riesgo de la libertad del hombre. Jesús mismo es el testigo fiel y
  veraz (cf. Ap 1,5; 3,14); vino para dar testimonio de la verdad (cf. Jn
  18,37). Con estas reflexiones deseo recordar un concepto muy querido por los
  primeros cristianos, pero que también nos afecta a nosotros, cristianos de
  hoy: el testimonio hasta el don de sí mismos, hasta el martirio, ha sido
  considerado siempre en la historia de la Iglesia como la cumbre del nuevo
  culto espiritual: <<Ofreced vuestros cuerpos>> (Rm 12,1). Se puede recordar,
  por ejemplo, el relato del martirio de san Policarpo de Esmirna, discípulo de
  san Juan: todo el acontecimiento dramático es descrito como una liturgia, más
  aún como si el mártir mismo se convirtiera en Eucaristía. Pensemos también en
  la conciencia eucarística que san Ignacio de Antioquía expresa ante su
  martirio: él se considera <<trigo de Dios>> y desea llegar a ser en el
  martirio <<pan puro de Cristo>>. El cristiano que ofrece su vida en el
  martirio entra en plena comunión con la Pascua de Jesucristo y así se
  convierte con Él en Eucaristía. Tampoco faltan hoy en la Iglesia mártires en
  los que se manifiesta de modo supremo el amor de Dios. Sin embargo, aun cuando
  no se requiera la prueba del martirio, sabemos que el culto agradable a Dios
  implica también interiormente esta disponibilidad, y se manifiesta en el
  testimonio alegre y convencido ante el mundo de una vida cristiana coherente
  allí donde el Señor nos llama a anunciarlo.\footnote{SCa 85}}
El testimonio hasta el don de nosotros mismos se convierte en signo sacramental,
el cristiano que ofrece su vida por completo como testigo entra en comunión con
la Pascua y se convierte con Cristo en Eucaristía. La vida entregada, este signo
sacramental, es el medio adecuado para comunicar la comunión con Dios:
\citalitlar{En efecto, la fe necesita un ámbito en el que se pueda testimoniar y
  comunicar, un ámbito adecuado y proporcionado a lo que se comunica. Para
  transmitir un contenido meramente doctrinal, una idea, quizás sería suficiente
  un libro, o la reproducción de un mensaje oral. Pero lo que se comunica en la
  Iglesia, lo que se transmite en su Tradición viva, es la luz nueva que nace
  del encuentro con el Dios vivo, una luz que toca la persona en su centro, en
  el corazón, implicando su mente, su voluntad y su afectividad, abriéndola a
  relaciones vivas en la comunión con Dios y con los otros. Para transmitir esta
  riqueza hay un medio particular, que pone en juego a toda la persona, cuerpo,
  espíritu, interioridad y relaciones. Este medio son los sacramentos,
  celebrados en la liturgia de la Iglesia. En ellos se comunica una memoria
  encarnada, ligada a los tiempos y lugares de la vida, asociada a todos los
  sentidos; implican a la persona, como miembro de un sujeto vivo, de un tejido
  de relaciones comunitarias. Por eso, si bien, por una parte, los sacramentos
  son sacramentos de la fe, también se debe decir que la fe tiene una estructura
  sacramental. El despertar de la fe pasa por el despertar de un nuevo sentido
  sacramental de la vida del hombre y de la existencia cristiana, en el que lo
  visible y material está abierto al misterio de lo eterno. \footnote{LF 40}}
Al celebrar los sacramentos con fe viva, la comunidad eclesial se deja implicar
por completo por la luz del Dios vivo que se comunica y el memorial que se
encarna. Despertar a la fe en los sacramentos es también despertar al sentido
sacramental que tiene la propia vida cristiana. Así como en los sacramentos los
signos visibles comunican la luz de Dios, también la propia existencia del
cristiano puede arrojar esa luz.

Este valor sacramental de la vida del cristiano y de la comunidad eclesial hace
de su propia existencia un testimonio kerygmático:
\citalitlar{La Buena Nueva debe ser proclamada en primer lugar, mediante el
  testimonio. Supongamos un cristiano o un grupo de cristianos que, dentro de la
  comunidad humana donde viven, manifiestan su capacidad de comprensión y de
  aceptación, su comunión de vida y de destino con los demás, su solidaridad en
  los esfuerzos de todos en cuanto existe de noble y bueno. Supongamos además
  que irradian de manera sencilla y espontánea su fe en los valores que van más
  allá de los valores corrientes, y su esperanza en algo que no se ve ni osarían
  soñar. A través de este testimonio sin palabras, estos cristianos hacen
  plantearse, a quienes contemplan su vida, interrogantes irresistibles: ¿Por
  qué son así? ¿Por qué viven de esa manera? ¿Qué es o quién es el que los
  inspira? ¿Por qué están con nosotros? Pues bien, este testimonio constituye ya
  de por sí una proclamación silenciosa, pero también muy clara y eficaz, de la
  Buena Nueva. Hay en ello un gesto inicial de evangelización. Son posiblemente
  las primeras preguntas que se plantearán muchos no cristianos, bien se trate
  de personas a las que Cristo no había sido nunca anunciado, de bautizados no
  practicantes, de gentes que viven en una sociedad cristiana pero según
  principios no cristianos, bien se trate de gentes que buscan, no sin
  sufrimiento, algo o a Alguien que ellos adivinan pero sin poder darle un
  nombre. Surgirán otros interrogantes, más profundos y más comprometedores,
  provocados por este testimonio que comporta presencia, participación,
  solidaridad y que es un elemento esencial, en general al primero absolutamente
  en la evangelización.\footnote{EN 21}}
La acción testimonial de Dios que se manifiesta en Cristo y en los sacramentos
instituidos por Él está analogamente presente en la vida comprometida del
cirstiano. El testimonio humano es respuesta de fe de aquellos que han
reconocido a Dios en los signos que le encarnan y que corresponden con palabras
y obras que quieren significar la vida nueva que viene del Señor. En esta
correspondencia se unden las raíces de la misión de proclamar la Buena Nueva.

El testimonio es así acción propia de todo bautizado que ha quedado unido a
Cristo y a la Iglesia.\autocite[Cf.][188]{prades2015testimonio} Toda la Iglesia
tiene la misión de dar testimonio; los obispos ofrecen al mundo el rostro de la
Iglesia con su trato y trabajo pastoral\footnote{GS 43}, los presbíteros,
creciendo en el amor por el desempeño de su oficio, han de ser un vivo
testimonio de Dios\footnote{LG 41}, los fieles han de dar testimonio de verdad
como testigos de la resurrección\footnote{LG 28 y LG 38}, los religiosos deben
ofrecer un testimonio sostenido por la integridad de la fe, por la caridad y el
amor a la cruz y la esperanza de la gloria futura\footnote{PC 25}, los
profesores han de dar testimonio tanto con su vida como con su
doctrina\footnote{GE 8}, los misioneros han de ofrecer testimonio con una vida
enteramente evangélica, con paciencia, longanimidad, suavidad, caridad sincera,
y si es necesario hasta con la propia sangre.\footnote{AG 24}

El signo que es la vida de los cristianos y, por tanto la Iglesia, esta llamado
a purificarse y crecer. La contradicción entre la fe y la vida de los cristianos
puede constituir un motivo de tropiezo, en lugar de dar a conocer la luz de
Dios. El testimonio de la vida entregada, aún cuando ha sido estimado según su
valor sacramental, es un signo imperfecto que debe ser madurado con una actitud
vigilante:
\citalitlar{Aunque la Iglesia, por la virtud del Espíritu Santo, se ha mantenido
  como esposa fiel de su Señor y nunca ha cesado de ser signo de salvación en el
  mundo, sabe, sin embargo, muy bien que no siempre, a lo largo de su prolongada
  historia, fueron todos sus miembros, clérigos o laicos, fieles al espíritu de
  Dios. Sabe también la Iglesia que aún hoy día es mucha la distancia que se da
  entre el mensaje que ella anuncia y la fragilidad humana de los mensajeros a
  quienes está confiado el Evangelio. Dejando a un lado el juicio de la historia
  sobre estas deficiencias, debemos, sin embargo, tener conciencia de ellas y
  combatirlas con máxima energía para que no dañen a la difusión del Evangelio.
  De igual manera comprende la Iglesia cuánto le queda aún por madurar, por su
  experiencia de siglos, en la relación que debe mantener con el mundo. Dirigida
  por el Espíritu Santo, la Iglesia, como madre, no cesa de ``exhortar a sus
  hijos a la purificación y a la renovación para que brille con mayor claridad
  la señal de Cristo en el rostro de la Iglesia''\footnote{GS 34}}
La vida de la Iglesia está marcada por esa llamada a este enriquecimiento
constante. Como afirma DV 8: \citalitinterlin{la Iglesia, en el decurso de los
  siglos, tiende constantemente a la plenitud de la verdad divina, hasta que en
  ella se cumplan las palabras de Dios.}

La categoría del testimonio ha servido para acercarnos a algunos textos
magisteriales y describir la vida de la Iglesia como signo sacramental. A modo
de conclusión son luminosas las palabras de K. Wojtyła:
\citalitlar{El significado del testimonio en la doctrina del Vaticano II es
  explícitamente analógico, puesto que el Concilio habla del testimonio de Dios
  y del hombre, que, de diversa manera, corresponde al divino, y a una respuesta
  multiforme a la revelación. En todo caso, sin embargo, la respuesta es
  testimonio y el testimonio, respuesta. \footnote{Para una discusión más amplia
    de la lectura de Wojtyła véase \cite[194--197]{prades2015testimonio}}}

Este recorrido a través de algunos modos de emplear la categoría del testimonio
en la Escritura y la doctrina magisterial ha servido para describir los
dinamísmos de la Revelación como acción libre y amorosa del Padre encarnada en
en la naturaleza humana asumida por el Verbo y sostenida por la acción interior
del Espíritu. Esta acción de la libertad divina ha encontrado la correspondencia
de la libertad humana que acoge la invitación al amor y se compromete por
completo a la comunión con Dios. Este intercambio testimonial comunica el amor
divino.

\section{La Categoría del Testimonio como problema}

Hasta ahora se ha empleado la categoría del testimonio sin problematizarla. Es
decir, hemos tratado el testimonio como <<cosa familiar y conocida>> empleada
ordinariamente en nuestras conversaciones. Es aquí donde nos permitimos tratar
al testimonio como algo que hay que esclarecer, algo sobre lo que se plantean
preguntas, de modo que hay que traer a la mente una explicación adecuada. En
palabras de Latourelle:
\citalitlar{Si la revelación misma se apoya en la experiencia humana del
  testimonio para expresar una de las relaciones fundamentales que unen al
  hombre con Dios, la reflexión teológica se encuentra entonces autorizada a
  explorar los datos de esta
  experiencia.\autocite[1523]{latourelle2000testimonio}}

Hasta el momento sólo hemos formulado una pregunta, que ahora podemos ampliar:
¿qué es conocer una verdad para la vida por el testimonio de la revelación
divina?

Si atendemos al modo en que se ha reflexionado sobre el testimonio en la
filosofía moderna y contemporanea encontramos desafíos y aportaciones que
La filosofía moderna y contemporanea

Desde aquí examinaremos qué otros cuestionamientos se pueden plantear
relacionados con este fenómeno. Es posible agrupar algunos desafíos y
aportaciones de la filosofía moderna y contemporánea en varías preguntas
principales. Éstas serviran luego para navegar en el pensamiento de Elizabeth
Anscombe.

\subsection{¿Cuál es el valor epistemológico del testimonio?}
Corresponde a la epistemología la tarea de estudiar la naturaleza del conocer y
su justificación. ¿Cuáles son los componentes del conocimiento? ¿sus fuentes o
condiciones? ¿sus límites?\footnote{Cf. Oxford Epistemology, characterized
  broadly, is an account of knowledge. Within the discipline of philosophy,
  epistemology is the study of the nature of knowledge and justification: in
  particular, the study of (a) the defining components, (b) the substantive
  conditions or sources, and (c) the limits of knowledge and justification.} La
pregunta sobre el valor epistemológico del testimonio consiste en juzgar el
lugar que éste ocupa en una descripición del conocimiento; ¿qué se puede decir
del testimonio como estrategia para adquirir la verdad y evitar el
error?\footnote{Cf. Any standard or strategy worthy of the title ``epistemic''
  must have as its fundamental goal the acquisition of truth and the avoidance
  of error.}

Un análisis tradicional empleado para hablar del conocimiento proposicional es
el que le describe como `creencia verdadera justificada'.\footnote{Ever since
  Plato's Theaetetus, epipstemologists have tried to identify the essential,
  defining components of propositional knowledge. These components will yield an
  analysis of propositional knowledge. An influential traditional view, inspired
  by Plato and Kant among others, is that propositional knowledge has three
  individually necessary and jointly sufficient components: justification,
  truth, and belief. On this view, propositional knowledge is, by definition,
  justified true belief. This tripartite definition has come to be called ``the
  standard analysis''.} Según esta composición tripartita la pregunta sobre el
valor epistemológico del testimonio se puede plantear diciendo: dada una
comunicación que cualifique como testimonio y que sea al caso que la creencia
formada desde esta comunicación está basada enteramente en el testimonio
recibido,\footnote{Even if an expression of thought qualifies as testimony and
  the resulting belief formed is entirely testimonially based for the hearer,
  however, there is the further question of how precisely such a belief
  successfully counts as justified belief or an instance of knowledge.} ¿cómo
adquirimos efectivamente una creencia verdadera justificada sobre la base de lo
que alguien nos ha dicho?\footnote{how we successfully acquire justified belief
  or knowledge on the basis of what other people tell us. This, rather than what
  testimony is, is often taken to be the issue of central import from an
  epistemological point of view.} Las respuestas a esta pregunta central sobre
la epistemología del testimonio se han situado en lo que se ha llamado la
postura reduccionista y la no-reduccionista.\footnote{Indeed, this is the
  question at the center of the epistemology of testimony, and the current
  philosophical literature contains two central options for answering it:
  non-reductionism and reductionism.} Las raices históricas de la primera
postura se le suelen atribuir a Hume y de la segunda a Thomas Reid.

De acuerdo a los no-reduccionistas el testimonio es simplemente una fuente de
justificación como lo sería la percepción de los sentidos, la memoria o la
inferencia. Según esto, siempre que no haya una justificación contraria
suficientemente relevante, el que escucha tiene justificación verdadera para
creer las proposiciones del testimonio del que habla.\footnote{According to
  non-reductionists-whose historical roots are standardly traced back to
  Reid-testimony is just as basic a source of justification (warrant,
  entitlement, knowledge, etc.) as sense perception, memory, inference, and the
  like. Accordingly, so long as there are no relevant defeaters, hearers can
  justifiedly accept the assertions of speakers merely on the basis of a
  speaker's testimony.}

Las reflexiones de Anscombe le llevarán a dialogar más con las preocupaciones de
Hume. Conviene anticipar algo sobre sus propuestas. ¿Qué lugar ocupa el
testimonio en la descripción que hace sobre el conocimiento?

En primer lugar cabe destacar que le considera un tema poco desarrollado en la
filosofía antigua y moderna:
\citalitlar{It may, therefore, be a subject worthy of curiosity, to enquire what
  is the nature of that evidence which assures us of any real existence and
  matter of fact, beyond the present testimony of our senses, or the records of
  our memory. This part of philosophy, it is observable, has been little
  cultivated, either by the ancients or moderns}



To apply these principles to a particular instance, we may observe, that there
is no species of reasoning more common, more useful, and even necessary to human
life, than that which is derived from the testimony of men, and the reports of
eye witnesses and spectators. This species of reasoning, perhaps, one may deny
to be founded on the relation of cause and effect. I shall not dispute about a
word. It will be sufficient to observe, that our assurance in any argument of
this kind, is derived from no other principle than our observation of the
veracity of human testimony, and of the usual conformity of facts to the reports
of witnesses. It being a general maxim, that no objects have any discoverable
connection together, and that all the inferences which we can draw from one to
another, are founded merely on our experience of their constant and regular
conjunction; it is evident, that we ought not to make an exception to this maxim
in favour of human testimony, whose connection with any event seems, in itself,
as little necessary as any other. Were not the memory tenacious to a certain
degree; had not men commonly an inclination to truth and a principle of probity;
were they not sensible to shame, when detected in a falsehood; were not these, I
say, discovered by experience to be qualities inherent in human nature, we
should never repose the least confidence in human testimony. A man delirious, or
noted for falsehood and villany, has no manner of authority with us.


en la vida social cabe aceptar un conocimiento por testimonio a condición de que
su grado de certeza se limite a la probabilidad

\subsection{Puede haber un hecho histórico que signifique una realidad absoluta?}
no se puede aceptar una comunicación divina que no sea inmediatamente dirigida
al individuo

\subsection{Lessing es lo mismo relatos de milagros etc...}
no se puede tener conocimiento directo de milagros
o profecías

I am the better pleased with the method of reasoning here delivered, as I think
it may serve to confound those dangerous friends, or disguised enemies to the
Christian religion, who have undertaken to defend it by the principles of human
reason. Our most holy religion is founded on faith, not on reason; and it is a
sure method of exposing it, to put it to such a trial as it is by no means
fitted to endure. To make this more evident, let us examine those miracles
related in Scrip ture; and, not to lose ourselves in too wide a field, let us
con fine ourselves to such as we find in the Pentateuch, which we shall examine
according to the principles of these pretended Christians, not as the word or
testimony of God himself, but as the production of a mere human writer and
historian. Here then we are first to consider a book, presented to us by a
barbarous and ignorant people, written in an age when they were still more
barbarous, and in all probability long after the facts which it relates,
corroborated by no concurring testimony, and resembling those fabulous accounts
which every nation gives of its origin. Upon reading this book, we find it full
of prodigies and miracles. It gives an account of a state of the world and of
human nature entirely different from the present: of our fall from that state;
of the age of man extended to near a thousand years; of the destruction of the
world by a deluge; of the arbitrary choice of one people, as the favourites of
heaven, and that people the countrymen of the author; of their deliverance from
bondage by prodigies the most astonishing imaginable.

I desire any one to lay his hand upon his heart, and, after a serious
consideration, declare, whether he thinks that the falsehood of such a book,
supported by such a testimony, would be more extraordinary and miraculous than
all the miracles it relates; which is, however, necessary to make it be received
according to the measures of probability above established.

\subsection{El lenguaje religioso es capaz de transmitir alguna verdad?}
Ciruclo de viena, falsabilidad como cuarto atributo de el conocimiento
proposicional. cita de analogía teológica.



\end{document}
