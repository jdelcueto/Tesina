\section{Análisis Diacrónico}

\subsection{Prophecy and Miracles}

El texto de \emph{Prophecy and Miracles} ha sido publicado en \emph{Faith in a
  Hard Ground} de la colección de St. Andrew's. Con mucha probabilidad es el
texto de una ponencia ofrecida en 1957 en el \emph{Philosophical Enquiry Group}
que solía reunirse en el Centro de Conferencias de los Dominicos en \emph{Spode
  House, Staffordshire}.

Anscombe especifica las coordenadas de su discusión en torno a tres documentos:
\begin{enumerate}
\item La constitución dogmática \enquote{Dei Filius}, específicamente el
  capítulo tercero: \blockquote[Ut nihilominus fidei nostrae obsequium rationi
  consentaneum esset, voluit Deus cum internis Spiritus Sancti auxiliis externa
  iungi revelationis suae argumenta, facta scilicet divina, atque imprimis
  miracula et prophetias, quae cum Dei omnipotentiam et infinitam scientiam
  luculenter commonstrent, divinae revelationis signa sunt certissima et omnium
  intelligentiae accommodata. Quare tum Moyses et Prophetae, tum ipse maxime
  Christus Dominus multa et manifestissima miracula et prophetias ediderunt; et
  de Apostolis legimus: Illi autem profecti praedicaverunt ubique, domino
  cooperante, et sermonem confirmante, sequentibus signis (Mare. XVI, 20). Et
  rursum scriptum est: Habemus firmiorem propheticum sermonem, cui bene facitis
  attendentes quasi lucernae lucenti in caliginoso loco (2 Petr. I, 19).]{Sin
    embargo, para que el obsequio de nuestra fe sea de acuerdo a la razón, quiso
    Dios que a la asistencia interna del Espíritu Santo estén unidos argumentos
    externos de su revelación, esto es, hechos divinos y, ante todo, milagros y
    profecías, que, mostrando claramente la omnipotencia y conocimiento infinito
    de Dios, son signos ciertísimos de la revelación y son adecuados al
    entendimiento de todos. Por eso Moisés y los profetas, y especialmente el
    mismo Cristo Nuestro Señor, obraron muchos milagros absolutamente claros y
    pronunciaron profecías; y de los apóstoles leemos: <<Salieron a predicar por
    todas partes, colaborando el Señor con ellos y confirmando la Palabra con
    las señales que la acompañaban>>. Y nuevamente está escrito: <<Tenemos una
    palabra profética más firme, a la cual hacéis bien en prestar atención, como
    a lámparas que iluminan en lugar oscuro>>.}
\item La advertencia del Deuteronomio: \blockquote{Todo lo que yo os mando, lo
    debéis observar y cumplir; no añadirás ni suprimirás nada. Si surge en medio
    de ti un profeta o un visionario soñador y te propone: \enquote{Vamos en pos
      de otros dioses ---que no conoces--- y sirvámoslos}, aunque te anuncie una
    señal o un prodigio y se cumpla la señal o el prodigio, no has de escuchar
    las palabras de ese profeta o visionario soñador. (Dt 13, 1--4a)}
\item \emph{On the proof of the spirit and of Power} de Lessing. Del cual
  considera varios puntos, pero se enfoca en su argumento central:
  \blockquote[Who denies it ---I do not--- that the reports of those miracles
  and prophecies are just as trustworthy as any historical truth can be? ---But
  now: if they are only so trustworthy, why are they so used as suddenly to make
  them infinitely more trustworthy? How? By building quite different things, and
  more things, on them, than one is entitled to build on historically evidenced
  truths. If no historical truth can be demonstrated, then neither can anything
  be demonstrated by historical truths. That is: accidental historical truths
  can never become the proof of necessary truths of reason.]{¿Quién lo niega
    ---no lo hago yo--- que los informes de esos milagros y profecías son tan
    dignos de confianza como puede ser cualquier verdad histórica? ---Pero
    ahora: si solo son tan merecedores de confianza, ¿por qué de repente son
    empleados como si fueran infinitamente confiables? ¿Cómo? Al construir cosas
    bastante distintas, y más cosas, sobre ellos, de las que se está en
    autoridad de construir sobre verdades de evidencia histórica. Si ninguna
    verdad historica puede ser demonstrada, entonces tampoco ninguna otra cosa
    puede ser demonstrada por medio de verdades históricas. Esto es: verdades
    contingentes en tanto que históricas nunca pueden llegar a ser prueba de
    verdades de razón en tanto que necesarias.}
\end{enumerate}

Anscombe comienza su discusión desenmarañando algunos puntos de los argumentos.
Lessing destaca que todos creemos que hubo una persona llamada Alejandro, que
conquistó casi toda Asia en corto tiempo, entonces lanza el reto:
\enquote{¿Quién, en consecuencia de esta creencia, estaría dispuesto a abjurar
  permanentemente de todo conocimiento que pueda entrar en conflicto con ella?}.
Entonces invita a considerar, que después de todo, sería posible que la creencia
en estas grandes conquistas podrían estar fundadas simplemente en los poemas de
Quérilo que acompañó a Alejandro.\footnote{Cf.~ \enquote{We all believe that
    there was such a person as Alexander, who conquered almost all Asia in a
    short time. But who would stake something of great and permanent importance,
    whose loss was irreplaceable, on this belief? Who, in consequence of this
    belief, would forswear for ever all knowledge that should concflict with it?
    Certainly I should not. I have at present nothing to object against
    Alexander and his conquests; but it would after all be possible that they
    were founded on a mere poem of Choerilus who accompanied Alexander, just as
    the ten year siege of Troy is founded on nothing but Homer.}} Esta última
propuesta resulta llamativa para Anscombe. Parece una alusión al hecho de que
conocemos de Cristo por una fuente o tradición \enquote{única}. Sin embargo
Anscombe piensa más bien que viene a apoyar la afirmación de que las verdades
históricas no pueden ser fundamentos de verdades necesarias. Una verdad
metafísica o una verdad matemática no puede seguirse de un hecho histórico, éste
tendría que contar con el mismo grado de certeza que estas verdades de razón;
pero una verdad histórica es muy incierta, como lo serían las conquistas de
Alejandro, si solo supieramos de ellas por los poemas de Quérilo. Ahora bien, a
juicio de Anscombe, esta premisa no merece gran atención. El supuesto de que
cualquier cosa creíble sobre Dios tiene que ser una verdad necesaria de razón le
parece una derivación de las nociones de Leibniz de Dios como un ser necesario.
En adición a esto, es una premisa derivada de la idea de que las verdades de la
religión son de tal naturaleza que la razón humana podría haber llegado a
pensarlas por sí misma.

Anscombe encuentra valor en la premisa acerca de no afirmar certezas más allá de
las que las verdades históricas nos dan la autoridad de justificar. La
constitución del Vaticano I habla de los milagros y profecias cumplidas como
solidos argumentos externos. ¿Puede una verdad histórica contar con certeza
suficiente para representar un solido argumento externo? No es el papel de estas
manifestaciones ser una demostración que reemplace el rol del Espíritu en la
fundamentación de la fe. Entonces parece que verdades historicas que no puedan
ser estimadas más que como probabilidades podrían jugas ese papel. ¿Se podría
conceder que la fe no necesita de argumentos externos ciertos para ser abrazada?
¿Podrían emplearse errores históricos y argumentaciones equivocadas como una
escalera que se usa para llegar a la fe y luego se descarta? Para Anscombe sería
un error pensar que una \enquote{escalera} como esta podría acercarnos
adecuadamente a la fe. Aunque se descarte la idea de Lessing de que toda
creencia sobre Dios tiene que ser una verdad necesaria, hay algo de valor en la
idea de que una fe cierta no se puede afirmar simplemente en argumentos externos
con fundamentos inciertos.

Es importante también tener en cuenta que la posición de Lessing ante el
Cristianismo es incompatible con las creencias Cristianas. Emplea una analogía
que ilustra su actitud: \blockquote[Suppose that a very useful mathematical
truth had been reached by the discoverer through an obvious fallacy. \textelp{}
Should I deny this truth? Should I refuse to use this truth? Would I be on that
account an Ungrateful reviler of the discoverer, if I were unwilling to prove
from his insight in other respects, indeed did not consider it capable of proof,
that the fallacy through which he stumbled upon the truth could not be a
fallacy?]{Supongamos que una muy útil verdad matemática haya sido alcanzada por
  su descubridor por medio de una falacia evidente. \textelp{} ¿Debería de negar
  esta verdad? ¿Sería en esta cuenta un desagradecido agraviador del que ha
  realizado el descubrimiento, si fuera renuente a apoyarme en su inteligencia
  en otras consideraciones, ciertamente si no considerara capaz de demostración,
  que la falacia a través de la cual tropezó sobre la verdad podría no ser una
  falacia?} Su interés en Cristo es en la enseñanza que este maestro pueda
ofrecer. Adicionalmente, su opinión es que lo que puede ser afirmado sobre Dios,
no solo no pueden ser proposiciones que derivan su justificación desde
afirmaciones históricas, sino que además no podrían ser afirmaciones
incompatibles con lo que podría ser razonable en estimar como históricamente
posible. Según esto, hace distinción entre la religión Cristiana y la religión
de Cristo, atribuyendo lo obscuro o confuso de la enseñanza de Cristo
transmitida a la religión Crisitana, y lo claro y útil a la religión de Cristo.

Una aclaración adicional que Anscombe destaca es que, a su juicio, Lessing
exagera la certidumbre (desde un punto de vista externo) que podría tener
Orígenes de los milagros y profecías cumplidas. Tanto en su tiempo como en el
nuestro los milagros serían hechos completamente extraordinarios y serían
estimados por los escépticos con tanta incredulidad como ahora, mientras que los
Católicos los aceptan.

Hechas estas consideraciones, Anscombe estudia el argumento central establecido
por Lessing. Su impresión es que esta cuestión viene a reducirse a que sea
razonable decir: \enquote{Pero estas cosas \emph{pueden} no ser verdad, ¿cómo
  puedo emplearlas para apoyar el Cristianismo?}. En esto está en conflicto con
la afirmación de Vaticano I.

\subsection{Hume and Julius Caesar}

Los artículos \emph{Hume and Julius Caesar} y \emph{``Whatever has a beginning
  of existence must have a cause'': Hume’s Argument Exposed} de Anscombe, fueron
publicados en la revista académica \emph{Analysis} en octubre de 1973 y abril de
1974 respectivamente. Ambos están relacionados por el tema de la causalidad en
Hume. En el trasfondo de los dos artículos está otro documento no publicado
hasta 2011 con el título \emph{Hume on causality: introductory}.

Anscombe again and again found in Hume a starting point for her discussions; and
we must not be misled bye her frequent dissent from his views into thinking of
her as `anti-Humean'. Indeed, in her treatment of the topic of causation
Anscombe can even be seen as continuing Hume's work---as out-Huming Hume.
teichmann 177

  Una de las actitudes características de Anscombe es su tendencia a quedar
  atraída por preguntas que representan cuestiones profundas, incluso en
  discusiones cuyos argumentos, método o conclusiones no le parecen tan
  interesantes.

  Un autor que suele tener este efecto en ella es Hume. En \emph{Modern Moral
    Philosophy} dice:

  \blockquote[{\cite[172]{anscombe1981mmph}}: The features of Hume’s philosophy
  which I have mentioned, like many other features of it, would incline me to
  think that Hume was a mere ---brilliant--— sophist; and his procedures are
  certainly sophistical. But I am forced, not to reverse, but to add to this
  judgement by a peculiarity of Hume’s philosophizing: namely that, although he
  reaches his conclusions --—with which he is in love--— by sophistical methods,
  his considerations constantly open up very deep and important problems. It is
  often the case that in the act of exhibiting the sophistry one finds oneself
  noticing matters which deserve a lot of exploring: the obvious stands in need of
  investigation as a result of the points that Hume pretends to have made.]{Las
    características de la filosofía de Hume que he mencionado, como muchas otras
    de sus características, me hacen inclinarme a pensar que Hume era un simple
    ---brillante--- sofista; y sus procedimientos son ciertamente sofísticos. Sin
    embargo me veo forzada, no a retractarme, sino a añadir a este juicio por la
    peculiaridad del filosofar de Hume: a saber, que aunque llega a sus
    conclusiones ---con las que está enamorado--- por métodos sofísticos, sus
    consideraciones constantemente abren problemas bien profundos e importantes.
    Frecuentemente es el caso que en el acto de exhibir la sofística uno se
    encuentra a sí mismo notando temas que merecen mucha exploración: lo obvio
    queda necesitado de investigación como resultado de los puntos que Hume
    pretende haber hecho.}

En el artículo \emph{Hume and Julius Caesar} la discusión que capta el interés
de Anscombe se encuentra en la sección IV de la tercera parte del \emph{Treatise
  of Human Nature} sobre el tema de la justificación de nuestro creer en
cuestiones que están más allá de nuestra experiencia y memoria. Anscombe cita el
texto de Hume como sigue:

\blockquote[{\cite[86]{anscombe1981hjc}}When we infer effects from causes, we
must establish the existence of these causes\ldots either by an immediate
perception of our memory or senses, or by an inference from other causes; which
causes we must ascertain in the same manner either by a present impression, or
by an inference from their causes and so on, until we arrive at some object
which we see or remember. 'Tis impossible for us to carry on our inferences
\emph{in infinitum}, and the only thing that can stop them, is an impression of
the memory or senses, beyond which there is no room for doubt or enquiry.
(Selby-Bigge's edition, pp. 82--3)]{Cuando inferimos efectos partiendo de causas
  debemos establecer la existencia de estas causas\ldots ya sea por la
  percepción inmediata de nuestra memoria o sentidos, o por la inferencia
  partiendo de otras causas; causas que debemos explicar de la misma manera por
  una impresión presente, o por una inferencia partiendo de sus causas, y así
  sucesivamente hasta que lleguemos a un objeto que vemos o recordamos. Es
  imposible para nosotros proseguir en nuestras inferencias al infinito, y lo
  único que puede detenerlas es una impresión de la memoria o los sentidos más
  allá de la cual no existe espacio para la duda o indagación.}

Ya en la sección II de esta misma parte del \emph{Treatise}, Hume ha planteado
cómo es la causalidad la conexión que nos asegura la existencia o acción de un
objeto que es seguido o precedido por la existencia o acción de
otro.\footnote{Cf. Treatise Sección II Parte III: ’Tis only causation, which
  produces such a connexion, as to give us assurance from the existence or
  action of one object, that ’twas follow’d or preceded by any other existence
  or action; nor can the other two relations be ever made use of in reasoning,
  except so far as they either affect or are affected by it. }
Ahora en la sección IV esta relación de causa y efecto será tomada como un
principio de asociación de ideas según el cual es posible inferir desde la
impresión de alguna cosa, una idea sobre otra cosa.

Desde esta noción de causalidad se explica la posibilidad de acceder a hechos
más allá de nuestra experiencia; estos son inferencias de efectos desde sus
causas. De este modo: \blockquote[{\cite[87]{anscombe1981hjc}}: For Hume, the
relation of cause and effect is the one bridge by which to reach belief in
matters beyond our present impressions or memories.]{Para Hume, la relación de
  causa y efecto es el único puente por el que se puede alcanzar creer en
  cuestiones más allá de nuestras impresiones presentes o memorias.}

El paso adicional que Hume propone en esta sección es que al realizar estas
inferencias es necesario establecer la existencia de las causas por medio de la
percepción inmediata de los sentidos o por medio de una ulterior inferencia. Sin
embargo, el establecimiento de la existencia de estas causas por medio de
inferencias no puede continuar infinitamente, sino que tiene que llegar a una
impresión de la memoria o los sentidos que sirva de justificación o fundamento
definitivo.

Para ilustrar este paso, Hume hace una invitación interesante:
  \blockquote[{\cite[?]{humetreatise}}: chuse any point of history, and consider
  for what reason we either believe or reject it.]{elegir cualquier punto en la
    historia, y considerar por qué razón lo creemos o rechazamos.} Acerca de una
  creencia histórica se nos invita a considerar sobre qué se sostiene su
  justificación. ¿Cuál es su fundamento?:
  \blockquote[{\cite[?]{humetratise}}: Thus we believe that Cæsar was kill’d in
  the senate-house on the ides of March; and that because this fact is establish’d
  on the unanimous testimony of historians, who agree to assign this precise time
  and place to that event. Here are certain characters and letters present either
  to our memory or senses; which characters we likewise remember to have been us’d
  as the signs of certain ideas; and these ideas were either in the minds of such
  as were immediately present at that action, and receiv’d the ideas directly from
  its existence; or they were deriv’d from the testimony of others, and that again
  from another testimony, by a visible gradation, ’till we arrive at those who
  were eye-witnesses and spectators of the event. ’Tis obvious all this chain of
  argument or connexion of causes and effects, is at first founded on those
  characters or letters, which are seen or remember’d, and that without the
  authority either of the memory or senses our whole reasoning wou’d be chimerical
  and without foundation.]{Así, creemos que César fue asesinado en el Senado en
    los idus de Marzo; y esto porque el hecho está establecido basándose en el
    testimonio unánime de los historiadores, que concuerdan en asignar a este
    evento este tiempo y lugar precisos. Aquí ciertos caracteres y letras se
    hallan presentes a nuestra memoria o sentidos; caracteres que recordamos
    igualmente que han sido usados como signos de ciertas ideas; y estas ideas
    estuvieron ya en las mentes de los que se hallaron inmediatamente presentes a
    esta acción y que obtuvieron las ideas directamente de su existencia; o fueron
    derivadas del testimonio de otros, y éstas a su vez de otro testimonio, por
    una graduación visible, hasta llegar a los que fueron testigos oculares y
    espectadores del suceso. Es manifiesto que toda esta cadena de argumentos o
    conexión de causas y efectos se halla fundada en un principio en los
    caracteres o letras que son vistos o recordados y que sin la autoridad de la
    memoria o los sentidos nuestro razonamiento entero sería quimérico o carecería
    de fundamento.}

Anscombe comienza por reaccionar afirmando:
\blockquote[{\cite[86]{anscombe1981hjc}}: This is not to infer effects from
causes, but rather causes from effects.]{Esto no es inferir efectos partiendo de
  sus causas, sino más bien causas desde los efectos.} Es decir, el ejemplo
histórico de Hume consiste en una inferencia de la causa original, el asesinato
de Julio César, desde su efecto remoto que es nuestra percepción en el presente.
Creemos en el asesinato de César porque lo inferimos como la causa última en una
cadena causal que llega hasta nuestra percepción de ciertas oraciones que
leemos. El hecho de que estemos leyendo esta información es la percepción que
justifica la creencia de que hay una cadena de causas y efectos que tiene como
efecto esta experiencia. Esta inferencia pasa a través de una cadena de efectos
de causas, que son efectos de causas\ldots ¿Dónde empieza la cadena? La
respuesta parece ser nuestra percepción presente. ¿Cómo hemos de entender,
entonces, el argumento de que la cadena no puede continuar infinitamente? La
propuesta de Hume es que la cadena ha de terminar en una impresión que no deje
lugar a dudas o busqueda mas allá, sin embargo, la cadena termina en el
asesinato de Julio César, no en nuestra percepción. La imagen que Hume pretende
ofrecer es la de una cadena fijada en sus dos extremos por algo distinto a los
eslabones que la componen, sin embargo, no lo logra, más bien parece describir
un voladizo, una estructura apoyada en un punto, pero sin apoyo en el otro
extremo.

La afirmación \blockquote['Tis impossible for us to carry on our inference in
infinitum]{Es imposible para nosotros proseguir en nuestras inferencias al
  infinito} viene a significar, según la interpretación de Anscombe, que
\blockquote[the justification of the grounds of our inferences cannot go on in
infinitum]{la justificación de los fundamentos de nuestras inferencias no pueden
  continuar al infinito}. El argumento aquí mas bien es que tiene que haber un
punto de partida para la inferencia de la causa original. La relación de
inferencias propuesta por Hume en su ilustración acabaría siendo una inferencia
hipotética según su propia definición. Anscombe explica diciendo:

\blockquote[hume in causality: We must suppose ourselves to start with the
familiar idea, merely as idea, of Caesar having been killed. Now if we ask why
we believe it we shall, as Hume does, point to historical testimony (the
‘characters and letters’), which doesn’t at this point figure as what stops
inference going on ad infinitum. However, if we want to explain the connection
we shall form the idea of Caesar’s death being recorded by eyewitnesses; and
these records having been received by others, who transmitted an account ...
etc. Here we really are arguing from the idea of an original cause to the idea
of an effect; we are ‘inferring effects from causes’, though only in the sense
of passing from the idea of the cause to the idea of the effect.]{Tendríamos que
  suponer que comenzamos con la idea familiar, meramente como una idea, de que
  César fue asesinado. Ahora si preguntamos por qué lo creemos hemos de, como
  hace Hume, señalar al testimonio histórico (los `caracteres y letras'), lo
  cual en este punto no figura como lo que detiene que la inferencia siga al
  infinito. Sin embargo, si queremos explicar la conexión tenemos que formular
  la idea de la muerte del Cesar siendo recordada por testigos; y esos recuentos
  siendo recibidos por otros, quienes transmitieron un informe\ldots etc. Aquí
  estamos realmente razonando desde la idea de una causa original a la idea de
  un efecto; estamos `infiriendo efectos de causas', pero solo en el sentido de
  pasar de la idea de la causa a la idea del efecto.}

Desde este análisis, Anscombe resume lo argumentado por Hume en cuatro partes:

\blockquote[humeandjulius 88: First, a chain of reasons for a belief must
terminate in something that is believed without being founded on anything else.
Second, the ultimate belief must be of a quite different character from derived
beliefs: it must be perceptual belief, belief in something perceived, or
presently remembered. Third, the immediate justification for a belief p, if the
belief is not a perception, will be another belief q, which follows from, just
as much as it implies, p. Fourth, we believe by inference through the links in a
chain of record

There is an implicit corollary: when we believe in historical information
belonging to the remote past, we believe that there has been a chain of record]{
  Primero, una cadena de razones para una creencia debe terminar en algo que se
  cree sin estar fundado en alguna otra cosa. Segundo, la creencia última debe
  ser de una naturaleza distinta a las creencias derivadas: Tiene que ser
  creencia perceptual, creer en algo percibido, or recordado en el presente.
  Tercero, la justificación inmediata de una creencia $p$, si la creencia no es
  una percepción, será otra creencia $q$, la cual se sigue, en la misma medida
  que implica, a $p$. Cuarto, creemos por inferencia a través de los eslabones
  en una cadena de relato.

  Hay un corolario implicito: cuando creemos en información histórica
  perteneciente a un pasado remoto, creemos que ha habido una cadena de relato.}

Sin embargo, Anscombe considera que esta no es la manera adecuada de establecer
esta relación. Mas bien: \blockquote[hjc 88: \emph{If} the written records that
we now see are grounds of our belief, they are first and foremost grounds for
belief in Caesar's killing, belief that the assassination is a solid bit of
history. Then our belief in that original event is a ground for belief in much
of the intermediate transmision.]{\emph{Si} los relatos escritos que vemos ahora
  son fundamento para nuestro creer, estos son primero y ante todo fundamento
  para la creencia en el asesinato de Cesar, creencia en que el asesinato es un
  pedazo sólido de historia. Entonces nuestra creencia en ese evento original es
  fundamento para el creer en mucha de la transimisión intermedia.}
¿Por qué creemos que hubo testigos del asesinato? Ciertamente porque creemos que
hubo un asesinato. La creencia de que hubo testigos es inferida de la creencia
en el hecho.

Anscombe compara este modo de entender la cadena de transmisión de información
histórica a nuestra creencia en la continuidad espacio-temporal. Si reconocemos
en una ocasión a una persona conocida como alguien que vimos la semana pasada,
nuestra creencia en que es la misma persona no es una inferencia de otra
creencia acerca de la continuidad espacio-temporal de un patrón humano desde
ahora hasta entonces, sino que más bien nuestra creencia en la continudad
espacio-temporal esta inferida del reconocimiento de la identidad de la persona.
Sin embargo, una evidencia sobre una interrupción en la continuidad sí alteraría
nuestra creencia en la identidad.

Elizabeth entonces concluye que: \blockquote[hjc 89: Belief in recorded history is
on the whole a belief that there has been a chain of tradition of reports and
records going back to contemporary knowledge; it is not a belief in the
historical facts by an inference that passes through the links of such a chain.
At most, that can very seldom be the case.]{La creencia en los registros de la
  historia consiste en general la creencia de que ha habido una cadena de
  tradición de informes y registros que van hacia el conocimiento contemporaneo;
  no es una creencia en hechos históricos por una inferencia que pasa por los
  eslabones de una cadena como esta. Como mucho, esto seria muy raramente el
  caso.}

Ahora bien, como se ha dicho antes, el interés de Anscombe no esta simplemente
en mostrar en qué se equivoca Hume, sino que considera que la cuestión toca el
nervio de un problema con cierta profundidad:
\blockquote[causality in hume 2855: The interesting problem that arises, then,
is why the things we are told and the writings that we see are the starting
points for our belief in the far distant events and so in the intermediate chain
of record.]{El problema interesante que surge, entonces, es por qué las cosas
  que se nos dicen y los escritos que vemos son puntos de partida para nuestro
  creer en eventos distantes y así también en la cadena del relato intermedia.}



Para discutir esta cuestión Anscombe recurre a las reflexiones de Wittgenstein
en \emph{On Certainty}. La motivación para estos ecritos de Wittgenstein son las
propuestas de Moore en \emph{Proof of the External World} y \emph{Defence of
  Common Sense}. En estas obras sostiene que hay una serie de proposiciones que
conocemos con seguridad, como \enquote{Aquí hay una mano, y aquí otra}, o
\enquote{La tierra ha existido por largo tiempo antes de mi nacimiento} y
\enquote{Nunca he estado lejos de la superficie de la tierra}. Estas reflexiones
ocuparon a Wittgenstein durante los últimos años de su vida.\footnote{Cf.
  preface On certainty}

Un tema que aparece en esta discusión de Wittgenstein es que la justificación
semántica, relacionada con el uso correcto del lenguaje, y la justificación
epistémica, relacionada como tal con el afirmar la verdad, están más unidas
entre sí de lo que se piensa. Según esto:\blockquote[teichmann 213: Wittgenstein
invites us to view the rules governing the correct use of words as comparable to
the rules governing the acceptance or rejection of beliefs (which are themselves
of course paradigmatically expressed in words); a ‘world view’ is determined as
much by our language and its attendant conceptual scheme as by what we would
ordinarily term our knowledge of things. The two aspects of world view, the two
kinds of justification, come together in the phenomenon of certainty. ‘I am
sure’, ‘I cannot doubt’ are related to ‘It must be’, which expression can be
prefixed to any statement of conceptual truth. One direction in which these
thoughts seem to take us is towards regarding certain world views, or sets of
beliefs, or very general beliefs, as no more susceptible of rational
justification or criticism than are concepts. –This is just how we go on’ looks
to be the final answer to a series of –Why?’ questions; and a language–game or
practice can appear to be sealed off from external assessment. An appeal to the
objective measure of Reality is empty in this context; we can of course –cite
reality’ when giving reasons in justification of a belief or practice, but that
our reasons count as good reasons is determined by norms or rules of reasoning
whose status as rules depends on the existence of a surrounding
language–game.]{Wittgenstein nos invita a ver las reglas que gobiernan el uso
  correcto de las palabras como comparables con las reglas que gobiernan la
  aceptación o rechazo de las creencias (que desde luego son ellas mismas
  paradigmáticamente expresadas en palabras); una `cosmovisión' está determinada
  tanto por nuestro lenguaje y su esquema conceptual relacionado como por lo que
  ordinariamente expresamos como nuestro conocimiento de las cosas. Los dos
  aspectos de la cosmovisión, los dos tipos de justificación, quedan unidos en
  el fenómeno de la certeza. [\ldots] Una dirección hacia la que estos
  pensamientos parecen dirigirnos es a considerar ciertas cosmovisiones, o
  colecciones de creencias, o creencias generales, como no más susceptibles de
  justificación racional o crítica que la que tienen los conceptos}.

Anscombe aplica las lecciones de \emph{On Certainty} al conocimiento histórico
en la linéa propuesta por Hume: ``elegir cualquier punto en la historia, y
considerar por qué razón lo creemos o rechazamos''. Elegir o rechazar una
creencia como la propuesta implica la identificación de una justificación
suficiente, y aquí esta busqueda esta regida por reglas comparables al correcto
uso de las palabras. Los dos puntos principales destacados por Anscombe serán:
\blockquote[grounds of belief 183: Hume's philosophical opinion was that these
ultimate groundless grounds were sense impressions. But I say that they are such
beliefs as those of which one will say `Everyone knows that!' or `Everyone who
knows anything on such matters at all, knows that!']{La opinion filosófica de
  Hume era que estos fundamentos-sin-fundamento definitivos eran impresiones de
  los sentidos. Pero yo digo que son ese tipo de creencias de las cuales uno
  dice `¡Todo el mundo sabe eso!' o `¡Todo el que sabe algo de ese tema, sabe
  eso!'}. Junto a esto, es también parte de su argumento:
\blockquote[teichmann 224: the mere statement that we can conceive of evidence
turning up which showed there had never been such a person as Julius Caesar is
no good until details are given of what sort of evidence that might be. If we
try to do this, however, we are likely to fail.]{la declaración de que puede ser
  concebido que aparezca evidencia que mostrara que nunca ha habido una persona
  como Julio César no es suficiente hasta que se den detalles acerca del tipo de
  evidencia que ésta pudiera ser. Si intentamos hacer esto, sin embargo, lo más
  probable es que fracasemos.}

Para entender su primera propuesta será útil recurrir a su explicación de este
punto como está planteado en \emph{On Certainty}: \blockquote[QLI, 130: Finding
grounds, testing, proving, reasoning, confirming, verifying are all processes
that go on within, say, one or another living linguistic practice which we have.
There are assumptions, beliefs, that are ‘immovable foundations’ of these
proceedings. By this, Wittgenstein means only that they are a foundation which
is not moved by any of these proceedings.]{Encontrar fundamentos, examinar,
  probar, razonar, confirmar, verificar son todos procesos que corresponden,
  diríamos, dentro de una u otra práctica linguística viva de las que tenemos.
  Hay supuestos, creencias, que son `fundamentos inmovibles' de estos modos de
  proceder. Con esto, Wittgenstein se refiere solamente a que son un fundamento
  que no es modificado por esos procesos.} En estos procesos o actividades hay
proposiciones que sirven como bisagras, donde se apoya el movimiento del
discurrir. Como tal, son creencias que si se ponen en duda impiden el progreso
del razonamiento. Estas creencias son esas que forman parte del conocimiento
común. En ese sentido, afirmar \enquote{aquí está mi mano} no es sostener algo
sobre el estado de las cosas en el mundo, sino establecer unas reglas para la
discusión. Por otra parte, poner en duda que tengo mi mano aquí delante
supondría tratar con escepticismo un conocimiento común de tal manera que se
podría decir \enquote{si esto es dudoso, ¿qué puede ser cierto?}, entonces
¿desde qué fundamento podríamos sostener una discusión o razonamiento sobre el
mundo en el que \enquote{aquí está mi mano} no es cierto?

Esto mismo ocurre con la creencia en el conocimiento común de la existencia de
Julio César, si nos planteamos la hipótesis de que nunca existió, nos
situaríamos entre dos alternativas, ya sea \blockquote[HJC 91: \textelp{} say:
``How could one explain all these references and implications, then?\ldots but,
but, \emph{but} if I doubt the existence of Caesar, if I say I may reasonably
call it in question, then with equal reason I must doubt the status of the
things I've just pointed to'']{\textelp{} decir ``¿Cómo se explican todas estas
  referencias e implicaciones entonces?\ldots pero, pero \emph{pero} si dudo de
  la existencia de César, si digo que podría razonablemente ponerlo en tela de
  juicio, entonces, con la misma razonabilidad tengo que dudar de la validez de
  las cosas que acabo de señalar''}. O por otra parte: \blockquote[HJC91:
\textelp{} I should realize straight away that the `doubt' put me in a vacuum in
which I could not produce reasons why such and such `historical facts' are more
or less doubtful.]{\textelp{} podría caer en cuenta inmediatamente de que la
  `duda' me ha encerrado en un vacío en el cual no podría producir razones por
  las cuales estos u otros `datos históricos' son más o menos dudosos.}

Hume escoge este punto histórico porque es un conocimiento presente en su
cultura con un grado particular de certeza. Podría haber sometido a prueba algun
detalle del suceso y cuestionar, por ejemplo, si podría dudarse la fecha o el
lugar del asesinato, sin embargo, el que ese hombre, César, existió, y su vida
terminó en un asesinato: esto solo podría cuestionarlo empleando la duda
Cartseiana.

Elizabeth alude a la analogía hecha por Otto Neurath en \emph{Anti-Spengler},
donde compara el conocimiento científico con un barco en el cual los que
investgan son como marinos que reconstruyen el barco en altamar, verificando y
reemplazando sus piezas mientras que se navega. Entonces propone que si la
ilustración implica que se puede ir examinando cada pieza y reemplazarla de tal
modo que se termina con un barco distinto, la analogía no sirve: \blockquote[HJC
92: For there are things that are on a level. A general epistemological reason
for doubting one will be a reason for doubting all, and then none of them would
have anything to test it by.]{Pues hay cosas que están sobre superficie. Una
  razón espistemológica general para dudar de una será razón para dudar de
  todas, y entonces ninguna tendría cosa alguna que sirviera para evaluarla.}


What would one REALLY have grounds for saying or thinking, in such a case?’ In
many of her articles, Anscombe refers to some view as a prejudice, or apparent
prejudice. When is a belief a prejudice, and when is it bedrock? When is it a
questionable ‘bit of Weltanschauung’, and when a ‘hinge proposition’? The answer
to these questions must in large part have to do with how much, and what sort
of, detail can be plausibly put into counter-examples to, or cases against, the
belief in question.

My knowledge of the things among which and the places in which I live is not so
much 'theory laden' as ‘common-knowledge laden'. I wish to say: it is a
falsification here to speak of testimony: to say, for example, that it is by
testimony that I know I was born. There is something else, not testimony, though
acquired by education from human beings, which is, so to speak, thicker than
testimony.

The work done, people could be taught what England was (no doubt still disputing
some regions). Now those who learned thereafter can hardly be said to have
knowledge by testimony. They were taught to call something 'England’—something
indeed which could in large part only be defined for them by hearsay; and they
so taught those who came after them. I am an heir of this tradition. Now, I know
I live in England. But by testimony? Some would say so. But there is something
queer about it. What do I know? That the world is divided up into countries
which have names, and that the one I live in is called England and is here on
the map of the globe. This involves understanding the use of the globe to
represent the earth. It is rather as if I had been taught to join in doing
something, than to believe something—but because everyone is taught to do such
things, an object of belief is generated. The belief is so certainly correct
(for it follows the practice) that it is knowledge; for here knowledge is no
other than certainly correct belief in pursuit of a practice. But the connection
with testimony is remote and indirect.
