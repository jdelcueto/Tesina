\section{Análisis Diacrónico}

\subsection{Hume and Julius Caesar}

Los artículos \emph{Hume and Julius Caesar} y \emph{``Whatever has a beginning
  of existence must have a cause'': Hume’s Argument Exposed} de Anscombe, fueron
publicados en la revista académica \emph{Analysis} en octubre de 1973 y abril de
1974 respectivamente. Ambos están relacionados por el tema de la causalidad en
Hume. En el trasfondo de los dos artículos está otro documento no publicado
hasta 2011 con el título \emph{Hume on causality: introductory}.

Una de las actitudes características de Anscombe es su tendencia a quedar
atraída por preguntas que representan cuestiones profundas, incluso en
discusiones cuyos argumentos, método o conclusiones no le parecen tan
interesantes.

Un autor que suele tener este efecto en ella es Hume. En \emph{Modern Moral
  Philosophy} dice:

\blockquote[{\cite[172]{anscombe1981mmph}}: The features of Hume’s philosophy
which I have mentioned, like many other features of it, would incline me to
think that Hume was a mere --—brilliant--— sophist; and his procedures are
certainly sophistical. But I am forced, not to reverse, but to add to this
judgement by a peculiarity of Hume’s philosophizing: namely that, although he
reaches his conclusions --—with which he is in love--— by sophistical methods,
his considerations constantly open up very deep and important problems. It is
often the case that in the act of exhibiting the sophistry one finds oneself
noticing matters which deserve a lot of exploring: the obvious stands in need of
investigation as a result of the points that Hume pretends to have made.]{Las
  características de la filosofía de Hume que he mencionado, como muchas otras
  de sus características, me hacen a inclinarme a pensar que Hume era un simple
  ---brillante--- sofista; y sus procedimientos son ciertamente sofísticos. Sin
  embargo me veo forzada, no a retractarme, sino a añadir a este juicio por la
  peculiaridad del filosofar de Hume: a saber, que aunque llega a sus
  conclusiones ---con las que está enamorado--- por métodos sofísticos, sus
  consideraciones constantemente abren problemas bien profundos e importantes.
  Frecuentemente es el caso que en el acto de exhibir la sofística uno se
  encuentra a sí mismo notando temas que merecen mucha exploración: lo obvio
  queda necesitado de investigación como resultado de los puntos que Hume
  pretende haber hecho.}

En el artículo \emph{Hume and Julius Caesar} la discusión que capta el interés
de Anscombe se encuentra en la sección IV de la tercera parte del \emph{Treatise
of Human Nature} sobre el tema de nuestro creer en cuestiones que están más allá
de nuestra experiencia y memoria. Anscombe cita el texto de Hume como sigue:

\blockquote[{\cite[86]{anscombe1981hjc}}When we infer effects from causes, we
must establish the existence of these causes\ldots either by an immediate
perception of our memory or senses, or by an inference from other causes; which
causes we must ascertain in the same manner either by a present impression, or
by an inference from their causes and so on, until we arrive at some object
which we see or remember. 'Tis impossible for us to carry on our inferences
\emph{in infinitum}, and the only thing that can stop them, is an impression of
the memory or senses, beyond which there is no room for doubt or enquiry.
(Selby-Bigge's edition, pp. 82--3)]{Cuando inferimos efectos partiendo de causas
  debemos establecer la existencia de estas causas\ldots ya sea por la
  percepción inmediata de nuestra memoria o sentidos, o por la inferencia
  partiendo de otras causas; causas que debemos explicar de la misma manera por
  una impresión presente, o por una inferencia partiendo de sus causas, y así
  sucesivamente hasta que lleguemos a un objeto que vemos o recordamos. Es
  imposible para nosotros proseguir en nuestras inferencias al infinito, y lo
  único que puede detenerlas es una impresión de la memoria o los sentidos más
  allá de la cual no existe espacio para la duda o indagación.}

Podemos identificar dos argumentaciones principales en este fragmento. En primer
lugar, describe el acceso a hechos más allá de nuestra experiencia como
inferencias de efectos desde sus causas.
\blockquote[{\cite[87]{anscombe1981hjc}}: For Hume, the relation of cause and
effect is the one bridge by which to reach belief in matters beyond our present
impressions or memories.]{Para Hume, la relación de causa y efecto es el único
  puente por el que se puede alcanzar creer en cuestiones más allá de nuestras
  impresiones presentes o memorias.}

En segundo lugar, determina que estas inferencias no pueden continuar
infinitamente.

Now there really is no difficulty about going on ad infinitum, or at any rate
about saying ‘and so on ad infinitum’, if the ‘inferring’ is simply deriving the
idea of the effect from that of the cause. But the inferring is more than
that—it is believing. It is in connection with this that Hume is saying ‘this
chain can’t go on for ever’.
