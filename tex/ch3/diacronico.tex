\section{Análisis Diacrónico}

\subsection{Hume and Julius Caesar}

Los artículos \emph{Hume and Julius Caesar} y \emph{``Whatever has a beginning
  of existence must have a cause'': Hume’s Argument Exposed} de Anscombe, fueron
publicados en la revista académica \emph{Analysis} en octubre de 1973 y abril de
1974 respectivamente. Ambos están relacionados por el tema de la causalidad en
Hume. En el trasfondo de los dos artículos está otro documento no publicado
hasta 2011 con el título \emph{Hume on causality: introductory}.

Una de las actitudes características de Anscombe es su tendencia a quedar
atraída por preguntas que representan cuestiones profundas, incluso en
discusiones cuyos argumentos, método o conclusiones no le parecen tan
interesantes.

Un autor que suele tener este efecto en ella es Hume. En \emph{Modern Moral
  Philosophy} dice:

\blockquote[{\cite[172]{anscombe1981mmph}}: The features of Hume’s philosophy
which I have mentioned, like many other features of it, would incline me to
think that Hume was a mere ---brilliant--— sophist; and his procedures are
certainly sophistical. But I am forced, not to reverse, but to add to this
judgement by a peculiarity of Hume’s philosophizing: namely that, although he
reaches his conclusions --—with which he is in love--— by sophistical methods,
his considerations constantly open up very deep and important problems. It is
often the case that in the act of exhibiting the sophistry one finds oneself
noticing matters which deserve a lot of exploring: the obvious stands in need of
investigation as a result of the points that Hume pretends to have made.]{Las
  características de la filosofía de Hume que he mencionado, como muchas otras
  de sus características, me hacen inclinarme a pensar que Hume era un simple
  ---brillante--- sofista; y sus procedimientos son ciertamente sofísticos. Sin
  embargo me veo forzada, no a retractarme, sino a añadir a este juicio por la
  peculiaridad del filosofar de Hume: a saber, que aunque llega a sus
  conclusiones ---con las que está enamorado--- por métodos sofísticos, sus
  consideraciones constantemente abren problemas bien profundos e importantes.
  Frecuentemente es el caso que en el acto de exhibir la sofística uno se
  encuentra a sí mismo notando temas que merecen mucha exploración: lo obvio
  queda necesitado de investigación como resultado de los puntos que Hume
  pretende haber hecho.}

En el artículo \emph{Hume and Julius Caesar} la discusión que capta el interés
de Anscombe se encuentra en la sección IV de la tercera parte del \emph{Treatise
  of Human Nature} sobre el tema de la justificación de nuestro creer en
cuestiones que están más allá de nuestra experiencia y memoria. Anscombe cita el
texto de Hume como sigue:

\blockquote[{\cite[86]{anscombe1981hjc}}When we infer effects from causes, we
must establish the existence of these causes\ldots either by an immediate
perception of our memory or senses, or by an inference from other causes; which
causes we must ascertain in the same manner either by a present impression, or
by an inference from their causes and so on, until we arrive at some object
which we see or remember. 'Tis impossible for us to carry on our inferences
\emph{in infinitum}, and the only thing that can stop them, is an impression of
the memory or senses, beyond which there is no room for doubt or enquiry.
(Selby-Bigge's edition, pp. 82--3)]{Cuando inferimos efectos partiendo de causas
  debemos establecer la existencia de estas causas\ldots ya sea por la
  percepción inmediata de nuestra memoria o sentidos, o por la inferencia
  partiendo de otras causas; causas que debemos explicar de la misma manera por
  una impresión presente, o por una inferencia partiendo de sus causas, y así
  sucesivamente hasta que lleguemos a un objeto que vemos o recordamos. Es
  imposible para nosotros proseguir en nuestras inferencias al infinito, y lo
  único que puede detenerlas es una impresión de la memoria o los sentidos más
  allá de la cual no existe espacio para la duda o indagación.}

Ya en la sección II del \emph{Treatise} Hume ha planteado cómo es la causalidad
la conexión que nos asegura la existencia o acción de un objeto que es seguido o
precedido por la existencia o acción de otro.\footnote{Cf. Treatise Sección II
  Parte III: ’Tis only causation, which produces such a connexion, as to give us
  assurance from the existence or action of one object, that ’twas follow’d or
  preceded by any other existence or action; nor can the other two relations be
  ever made use of in reasoning, except so far as they either affect or are
  affected by it. }

Ahora en la sección IV esta relación de causa y efecto será tomada como un
principio de asociación de ideas según el cual es posible inferir desde la
impresión de alguna cosa, una idea sobre otra cosa.

Desde esta noción de causalidad se explica la posibilidad de acceder a hechos
más allá de nuestra experiencia; estos son inferencias de efectos desde sus
causas. De este modo: \blockquote[{\cite[87]{anscombe1981hjc}}: For Hume, the
relation of cause and effect is the one bridge by which to reach belief in
matters beyond our present impressions or memories.]{Para Hume, la relación de
  causa y efecto es el único puente por el que se puede alcanzar creer en
  cuestiones más allá de nuestras impresiones presentes o memorias.}

El planteamiento que Hume establece ahora es que al realizar estas inferencias
es necesario establecer la existencia de las causas

Para ilustrar su propuesta Hume hace una invitación interesante:
  \blockquote[{\cite[?]{humetreatise}}: chuse any point of history, and consider
  for what reason we either believe or reject it.]{elegir cualquier punto en la
    historia, y considerar por qué razón lo creemos o rechazamos.} Acerca de una
  creencia histórica se nos invita a considerar sobre qué se sostiene su
  justificación. ¿Cuál es su fundamento? La opinión de Hume es que en definitiva
  se apoyan sobre impresiones de nuestros sentidos. Así lo describe diciendo:
  \blockquote[{\cite[?]{humetratise}}: Thus we believe that Cæsar was kill’d in
  the senate-house on the ides of March; and that because this fact is establish’d
  on the unanimous testimony of historians, who agree to assign this precise time
  and place to that event. Here are certain characters and letters present either
  to our memory or senses; which characters we likewise remember to have been us’d
  as the signs of certain ideas; and these ideas were either in the minds of such
  as were immediately present at that action, and receiv’d the ideas directly from
  its existence; or they were deriv’d from the testimony of others, and that again
  from another testimony, by a visible gradation, ’till we arrive at those who
  were eye-witnesses and spectators of the event. ’Tis obvious all this chain of
  argument or connexion of causes and effects, is at first founded on those
  characters or letters, which are seen or remember’d, and that without the
  authority either of the memory or senses our whole reasoning wou’d be chimerical
  and without foundation.]{Así, creemos que César fue asesinado en el Senado en
    los idus de Marzo; y esto porque el hecho está establecido basándose en el
    testimonio unánime de los historiadores, que concuerdan en asignar a este
    evento este tiempo y lugar precisos. Aquí ciertos caracteres y letras se
    hallan presentes a nuestra memoria o sentidos; caracteres que recordamos
    igualmente que han sido usados como signos de ciertas ideas; y estas ideas
    estuvieron ya en las mentes de los que se hallaron inmediatamente presentes a
    esta acción y que obtuvieron las ideas directamente de su existencia; o fueron
    derivadas del testimonio de otros, y éstas a su vez de otro testimonio, por
    una graduación visible, hasta llegar a los que fueron testigos oculares y
    espectadores del suceso. Es manifiesto que toda esta cadena de argumentos o
    conexión de causas y efectos se halla fundada en un principio en los
    caracteres o letras que son vistos o recordados y que sin la autoridad de la
    memoria o los sentidos nuestro razonamiento entero sería quimérico o carecería
    de fundamento.}

La reacción de Anscombe a esta propuesta de Hume es inmediata:
\blockquote[{\cite[86]{anscombe1981hjc}}: This is not to infer effects from
causes, but rather causes from effects.]{Esto no es inferir efectos partiendo de
  sus causas, sino más bien causas desde los efectos.} Es decir, el ejemplo
histórico de Hume consiste en una inferencia de la causa original, el asesinato
de Julio César desde su efecto remoto que es nuestra percepción en el presente
de ciertas letras que leemos.

Creemos en el asesinato del César porque lo inferimos como la causa última en
una cadena causal terminando en nuestra percepción de ciertas oraciones que
leemos. El hecho de que estemos leyendo esta información es la percepción que
justifica la creencia de que hay una cadena de causas y efectos que culminan en
esta experiencia.


Esta inferencia desde nuestra impresión presente
hasta la causa orignal pasa a través de una cadena de efectos de causas, que son
efectos de causas\ldots ¿Dónde empieza la cadena? ¿Puede decirse que es nuestra
percepción presente?

La imagen que Hume pretende ofrecer es la de una cadena fijada en sus dos
extremos por algo distinto a los eslabones que la componen

sin embargo hume no lo logra, más bien parece describir un voladizo, una
estructura apoyada en un punto, pero


The picture is that of a chain which must be nailed by and to something
different from the links of which it is composed. As the picture swims before
the imagination, the chain even so hangs forlornly down—one remembers that hint
of a nailing at the other end, where there were eyewitnesses of Caesar’s
assassination. But Hume cannot give us that picture. For the picture that he can
give us, the ordinary idea of a dangling chain is unfortunate: that of a
cantilever would be more satisfying. And no doubt the supported structure in a
cantilever construction could consist of


we believe in the killing of Caesar in the Senate House because we infer it as
ultimate cause in a chain of causality terminating in our perception of ‘certain
characters and letters’. Waiving the question (to be considered later) whether
this is a reasonable account of belief in historical testimony, we may grant
there is a chain of causality terminating in that perception, and that it is
because of our perception of sentences telling us of that event that we believe
it.


The end of the chain is the death of Caesar or the perception of it by
eyewitnesses, not our perception; that was surely the beginning of the
inference!

So after all there was reason to conceive the chain running the other
way.

But then how do we justify the starting point?

Our charitable
reconstruction has misfired. We must suppose ourselves to start with the
familiar idea, merely as idea, of Caesar having been killed.

Now if we ask why
we believe it we shall, as Hume does, point to historical testimony (the
‘characters and letters’), which doesn’t at this point figure as what stops
inference going on ad infinitum. However,


but we must reach a starting point in the justification of these inferences (and
that starting point must be perception)

’Tis obvious all this chain of argument or connexion of causes and effects, is
at first founded on those characters or letters, which are seen or remember’d,
and that without the authority either of the memory or senses our whole
reasoning wou’d be chimerical and without foundation. Every link of the chain
wou’d in that case hang upon another; but there wou’d not be any thing fix’d to
one end of it, capable of sustaining the whole; and consequently there wou’d be
no belief nor evidence. And this actually is the case with all hypothetical
arguments, or reasonings upon a supposition; there being in them, neither any
present impression, nor belief of a real existence.





We must suppose ourselves to start with the familiar idea, merely as idea, of
Caesar having been killed. Now if we ask why we believe it we shall, as Hume
does, point to historical testimony (the ‘characters and letters’), which
doesn’t at this point figure as what stops inference going on ad infinitum.
However, if we want to explain the connection we shall form the idea of Caesar’s
death being recorded by eyewitnesses; and these records having been received by
others, who transmitted an account ... etc. Here we really are arguing from the
idea of an original cause to the idea of an effect; we are ‘inferring effects
from causes’, though only in the sense of passing from the idea of the cause to
the idea of the effect.

His argument then falls into two parts.

  En segundo lugar, determina que estas inferencias no pueden continuar
  infinitamente. Si se tratara de mera relación especulativa de conceptos no
  representaría dificultad, pero se trata de creer, y la cadena no podría ofrecer
  una creencia si no tiene término. \blockquote[{\cite[2762]{anscombe2011hoc}}:
  Now there really is no difficulty about going on ad infinitum, or at any rate
  about saying ‘and so on ad infinitum’, if the ‘inferring’ is simply deriving the
  idea of the effect from that of the cause. But the inferring is more than that
  ---it is believing. It is in connection with this that Hume is saying ‘this
  chain can’t go on for ever’.]{Ahora realmente no hay dificultad en ir
    infinitamente, o en cualquier caso decir `así sucesivamente infinitamente', si
    el `inferir' es simplemente derivar la idea del efecto partiendo de su causa.
    Pero el inferir es más que eso ---es creer. Es en conexión con esto que Hume
    dice `esta cadena no puede seguir para siempre'}

First, a chain ‘Since p, q, etc’ in which p gives a believed-in (not perceived)
cause and q an inferred effect, cannot go on for ever but must terminate in a
proposition that is believed without inferring any consequences from it; and
from this proposition we then work back in reverse order to p.

This is a particular form of a familiar argument that not everything can be
argued from something else, that is: that it cannot be the case that everything
is argued from something else. I believe p because I believe q because I believe
r because I believe s ---this cannot go on for ever; it must end in something
which I believe, not because I believe something else. This argument appears to
be correct.

Hume’s second point is that not merely must the chain that he is concerned with
come to an end somewhere, but its terminus must be of a different kind from the
other members. ... without the authority either of the memory or the senses our
whole reasonings wou’d be chimerical and without foundation. Every link of the
chain wou’d in that case hang upon another; but there wou’d not be anything
fix’d to one end of it, capable of sustaining the whole; and consequently there
wou’d be no belief or evidence.[27]


The second part of his argument, which says that the terminus must be of a
different character from the links of the chain, is more doubtful than the first
part which only says there must be a terminus. Hume does not think that I have
to have a present perception (of memory or sense) in connection with my belief
that Caesar was killed in the Senate House: we can ‘reason upon our past
conclusions and principles, without having recourse to those impressions from
which they first arose.’ The convictions, however, must have been produced by
impressions, and ‘all reasonings concerning causes and effects are originally
deriv’d from some impression’.
