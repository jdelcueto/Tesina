\section{Análisis Diacrónico}

\subsection{Prophecy and Miracles}

El \emph{Philosophical Enquiry Group} se reunió anualmente entre 1954 y 1974 en
el Centro de Conferencias de los Dominicos en \emph{Spode House, Staffordshire}.
Los encuentros tenían como objetivo la discusión de cuestiones relacionadas con
las creencias y prácticas cristianas. Elizabeth Anscombe y Peter Geach
estuvieron entre los primeros ponentes invitados y colaboraron durante los
veinte años que se realizaron las
conferencias.\footnote{\cite[Cf.~][x]{anscombe2008faith}: \textelp{} no
  information was found about a number of papers. Features of their physical
  format suggested that the group of three (`Prophecy', `The Inmortality of the
  soul', and `On being in good faith', Nos. 3,9 and 12) were all given in the
  late 1950s and early 1960s to the Philosophical Enquiry Group wich met each
  year between 1954 and 1974 at the Dominican Conference Centre at Spode House
  in Staffordshire. \textelp{} Among the first invitees were Elizabeth Anscombe
  and Peter Geach \textelp{} The meetings focused on philosophical issues
  related to Christian belief and practice.} Una de estas colaboraciones se
encuentra en \emph{Prophecy and Miracles}, publicado en \emph{Faith in a Hard
  Ground} en 2008. Es con mucha probabilidad el texto de una ponencia ofrecida
por Anscombe en la reunión del grupo en 1957.\footnote{\cite[Cf.~][nota a pie de
  página 20]{anscombe2008faith:prophandmi}: From the undated typescript of a
  paper, probably delivered in 1957}

Elizabeth introduce su discusión ofreciendo tres documentos que servirán como
los ejes principales de su análisis:

\begin{enumerate}
\item La constitución dogmática \emph{Dei Filius}, específicamente el capítulo
  tercero: \blockquote[{\cite[\S\,3009]{vati1870df}}: Ut nihilominus fidei
  nostrae obsequium rationi consentaneum \textins{\emph{cf. Rm 12,1}} esset,
  voluit Deus cum internis Spiritus Sancti auxiliis externa iungi revelationis
  suae argumenta, facta scilicet divina, atque imprimis miracula et prophetias,
  quae cum Dei omnipotentiam et infinitam scientiam luculenter commonstrent,
  divinae revelationis signa sunt certissima et omnium intelligentiae
  accommodata \textins{\emph{can. 3 et 4}}. Quare tum Moyses et Prophetae, tum
  ipse maxime Christus Dominus multa et manifestissima miracula et prophetias
  ediderunt; et de Apostolis legimus: ``Illi autem profecti praedicaverunt
  ubique, domino cooperante, et sermonem confirmante, sequentibus signis''
  \textins{\emph{Mc 16,20}}. Et rursum scriptum est: ``Habemus firmiorem
  propheticum sermonem, cui bene facitis attendentes quasi lucernae lucenti in
  caliginoso loco'' \textins{\emph{2 Pt 1,19}}.]{Sin embargo, para que el
    obsequio de nuestra fe fuera conforme a la razón \textins{\emph{cf. Rm
        12,1}}, quiso Dios que a los auxilios internos del Espíritu Santo se
    juntaran argumentos externos de su revelación, a saber, hechos divinos y,
    ante todo, los milagros y profecías, que, mostrando de consuno luminosamente
    la omnipotencia y ciencia infinita de Dios, son signos ciertísimos y
    acomodados a la inteligencia de todos, de la revelación divina
    \textins{\emph{can. 3 et 4}}. Por eso, tanto Moisés y los profetas, como
    sobre todo el mismo Cristo Señor, hicieron y pronunciaron muchos y
    clarísimos milagros y profecías; y de los apóstoles leemos: <<Y ellos
    marcharon y predicaron por todas partes, cooperando el Señor y confirmando
    su palabra con los signos que se seguían>> \textins{\emph{Mc 16,20}}. Y
    nuevamente está: <<Tenemos palabra profética más firme, a la que hacéis bien
    en atender como a una antorcha que brilla en un lugar tenebroso>>
    \textins{\emph{2 Pe 1,19}}.}
\item La advertencia del Deuteronomio: \blockquote{Todo lo que yo os mando, lo
    debéis observar y cumplir; no añadirás ni suprimirás nada. Si surge en medio
    de ti un profeta o un visionario soñador y te propone: \enquote{Vamos en pos
      de otros dioses ---que no conoces--- y sirvámoslos}, aunque te anuncie una
    señal o un prodigio y se cumpla la señal o el prodigio, no has de escuchar
    las palabras de ese profeta o visionario soñador. (Dt 13, 1--4a)}
\item \emph{Sobre la Demostración de Espíritu y fuerza} de Lessing. Del cual
  considera varios puntos, pero se enfoca en su argumento central:
  \blockquote[La traducción al inglés de este fragmento es de Anscombe,
  {\cite[Cf.~][22]{anscombe2008faith:prophandmi}}: Who denies it ---I do not---
  that the reports of those miracles and prophecies are just as trustworthy as
  any historical truth can be? ---But now: if they are only so trustworthy, why
  are they so used as suddenly to make them infinitely more trustworthy? How? By
  building quite different things, and more things, on them, than one is
  entitled to build on historically evidenced truths. If no historical truth can
  be demonstrated, then neither can anything be demonstrated by historical
  truths. That is: accidental historical truths can never become the proof of
  necessary truths of reason.]{¿Quién lo niega ---no lo hago yo--- que los
    informes de esos milagros y profecías son tan dignos de confianza como puede
    ser cualquier verdad histórica? ---Pero ahora: si solo son tan merecedores
    de confianza, ¿por qué de repente son empleados como si fueran infinitamente
    confiables? ¿Cómo? Al construir cosas bastante distintas, y más cosas, sobre
    ellos, de las que se está en autoridad de construir sobre verdades de
    evidencia histórica. Si ninguna verdad histórica puede ser demostrada,
    entonces tampoco ninguna otra cosa puede ser demostrada por medio de
    verdades históricas. Esto es: verdades contingentes en tanto que históricas
    nunca pueden llegar a ser prueba de verdades de razón en tanto que
    necesarias.}
\end{enumerate}

Tras esta introducción, Anscombe comienza su análisis desenmarañando algunos
puntos de los argumentos del ensayo de Lessing. En una de sus premisas emplea
como ejemplo de verdad histórica nuestra creencia en que hubo en el pasado una
persona llamada Alejandro, que conquistó casi toda Asia en corto tiempo.
Entonces plantea el reto: \enquote{¿Quién, en consecuencia de esta creencia,
  estaría dispuesto a abjurar permanentemente de todo conocimiento que pueda
  entrar en conflicto con ella?}. Sugiere entonces considerar la idea de que,
después de todo, sería posible que la creencia en estas grandes conquistas
podrían estar fundadas simplemente en los poemas de Choerilus que acompañó a
Alejandro.\footnote{\cite[Cf.~][448]{lessing1982escritos:demo}}

Esta última propuesta resulta llamativa para Anscombe. Parece una alusión al
hecho de que conocemos de Cristo por una fuente o tradición \enquote{única}. Sin
embargo Anscombe piensa que más bien viene a apoyar la afirmación de que las
verdades históricas no pueden ser fundamentos de verdades necesarias. Una verdad
metafísica o una verdad matemática no puede seguirse de un hecho histórico, este
tendría que contar con el mismo grado de certeza que estas verdades de razón;
pero una verdad histórica es muy incierta, como lo serían las conquistas de
Alejandro, si solo supiéramos de ellas por los poemas de Choerilus. Ahora bien,
a juicio de Anscombe, esta premisa no merece gran atención. El supuesto de que
cualquier cosa creíble sobre Dios tiene que ser una verdad necesaria de razón le
parece una derivación de las nociones propuestas por Leibniz sobre la necesidad
en relación con Dios. En adición a esto, es una premisa apoyada sobre el
supuesto de que las verdades de la religión son de tal naturaleza que la razón
humana podría haber llegado a pensarlas por sí misma.

Anscombe sí encuentra valor en la premisa acerca de no afirmar certezas más allá
de las que las verdades históricas nos dan la autoridad de justificar. La
constitución del Vaticano~I habla de los milagros y profecías cumplidas como
sólidos argumentos externos. ¿Puede una verdad histórica contar con certeza
suficiente para representar un solido argumento externo? No es el papel de estas
manifestaciones ser una demostración que reemplace el rol del Espíritu en la
fundamentación de la fe. Entonces parece que verdades históricas que no puedan
ser estimadas más que como probabilidades podrían jugar ese papel. ¿Se podría
conceder que la fe no necesita de argumentos externos ciertos para ser abrazada?
¿Podrían emplearse errores históricos y argumentaciones equivocadas como una
escalera que se usa para llegar a la fe y luego se descarta? Para Anscombe sería
un error pensar que una \enquote{escalera} como esta podría acercarnos
adecuadamente a la fe. Aunque se descarte la idea de Lessing de que toda
creencia sobre Dios tiene que ser una verdad necesaria, hay algo de valor en la
idea de que una fe cierta no se puede afirmar simplemente sobre argumentos
externos con fundamentos inciertos.

Otro punto destacado por Anscombe es que la posición de Lessing ante el
cristianismo es incompatible con las creencias cristianas. Una de sus analogías
ilustra bien esta actitud:
\blockquote[{\cite[449]{lessing1982escritos:demo}}]{Supongamos que se diera una
  verdad matemática, grande y útil, a la que su descubridor hubiera llegado
  siguiendo un palmario sofisma \textelp{} ¿negaría yo por ello esa verdad y me
  negaría por eso a hacer uso de esa verdad? Pero ¿sería yo un ingrato
  calumniador del inventor, por no querer apoyarme en su agudeza, probada sí de
  otras maneras, para demostrar y mantener que el sofisma mediante el que dio
  con esa verdad no \emph{puede} ser un sofisma?} Su interés en Cristo es en la
enseñanza que este maestro pueda ofrecer. Adicionalmente, su opinión es que lo
que puede decirse sobre Dios, no solo no pueden ser proposiciones que derivan su
justificación desde afirmaciones históricas, sino que además no podrían ser
afirmaciones incompatibles con lo que podría ser razonable en estimar como
históricamente posible. Según esto, hace distinción entre la \emph{religión
  cristiana} y la \emph{religión de Cristo}. Esta última sería la que ofrece
enseñanzas claras y útiles, sin embargo ha quedado mezclada en su transmisión
con lo confuso y oscuro de la \emph{religión cristiana}.

Una aclaración adicional que Anscombe propone es que, a su juicio, Lessing
exagera la certidumbre que (desde un punto de vista externo) podría tener
Orígenes de los milagros y profecías cumplidas. Tanto en su tiempo como en el
nuestro los milagros serían hechos completamente extraordinarios y serían
estimados por los escépticos con tanta incredulidad entonces como ahora,
mientras que los católicos los aceptan.

Hechas estas consideraciones preliminares, Anscombe estudia el argumento central
establecido por Lessing. Su impresión es que la objeción de Lessing consiste
fundamentalmente en: \enquote{Pero estas cosas \emph{pueden} no ser verdad,
  ¿cómo puedo emplearlas para apoyar el cristianismo?}. El argumento es útil,
puesto que no se orienta a atacar la veracidad de los milagros o cumplimientos
de profecías que han quedado documentados, sino que pone en duda que estos
testimonios o relatos puedan ser fundamento suficiente para sostener la creencia
en el cristianismo como justificada. En esto está claramente en conflicto con la
enseñanza del Vaticano I.

Por su parte, para Anscombe, la afirmación de \emph{Dei Filius} es de
extraordinario interés ya que le parece que la experiencia más común es que
creamos en las profecías cumplidas y los milagros porque creemos en la religión
católica y estos forman parte de su enseñanza. Si tomamos esto en cuenta junto
con la enseñanza del Deuteronomio y una reflexión razonable acerca de lo que la
fe requiere, tendríamos que decir que para que se puedan tomar los milagros y
las profecías cumplidas como \enquote{sólidos argumentos externos}, estos
tendrían que quedar determinados como tales antes de que quede afirmada la
creencia en el cristianismo. Pero, ¿acaso no hay ya cierto elemento teológico en
designar algo como una profecía cumplida o milagro? ¿En que situación está un
juez o historiador indiferente de la religión que recibe noticias de un milagro
o de profecías cumplidas? ¿Pueden ser éstos sólidos argumentos externos para
creer en la religión católica?

El análisis de Anscombe se desarrollará en torno a la posibilidad de sostener
creencias ciertas teniendo como fundamento los informes de milagros; o la
certeza de los relatos históricos; o las profecías cumplidas que puedan ser
consideradas claras por su antigüedad, prioridad y realización.

En cuanto a los informes de milagros, Anscombe sostiene con Lessing que estos no
apelarían a un juez que sea externo a las creencias religiosas. Podemos estimar
la resurrección de Cristo como el signo principal empleado por la apologética. A
la noticia de este milagro Lessing le concede tanta certeza como la que pueda
tener un dato histórico, Anscombe, sin embargo, no está de acuerdo con esto. Le
parece que no es irrazonable decir:
\blockquote[{\cite[26]{anscombe2008faith:prophandmi}}: `Heaven knows what
happened to produce this belief; I do not. And I know much too little about what
may go on in human minds in the origins of embracing a new religious belief, to
draw any conclusions (as I am so often pressed to do) from the subsequent
careers of the Apostles (supposing them to be truly related in the main) or from
the sudden appearance and growth of a new religion, which after all is all I am
really perfectly certain of. I do know one thing: new religions sometimes spread
like wildfire. How this works, and how it gets established in them is obscure. I
concede that this is an impressive religion too; but then it had a very
impressive religion behind it: that of the Old Testament. Remember that beliefs
in miraculous events in connexion with the founders and heroes of religion are
quite common. The most I can grant is that the record is quite as if these
things had happened: the manner is not legendary, though the matter is!']{Dios
  sabe qué ocurrió para que se produjera esta creencia; yo no lo sé. Además
  conozco muy poco de lo que ocurre en las mentes humanas en los orígenes de
  abrazar una creencia religiosa nueva, como para sacar alguna conclusión
  \textelp{} de las subsiguientes misiones de los Apóstoles \textelp{} o de la
  repentina aparición y crecimiento de una nueva religión, de lo que después de
  todo es todo de lo que estoy perfectamente segura. Sí conozco una cosa: las
  religiones nuevas a veces se propagan como el fuego. Cómo funciona esto, y
  cómo queda establecido en ellas es oscuro. Concedo que esta es una religión
  impresionante también; pero ha tenido una religión impresionante detrás: la
  del Antiguo Testamento. Recuerda que las creencias de eventos milagrosos en
  conexión con los fundadores o héroes de una religión son bastante comunes. Lo
  mayor que puedo conceder es que la noticia es bastante como si estas cosas
  hubieran ocurrido: ¡el modo no es legendario, aunque la materia sí!}

Aquí la cuestión importante para Anscombe es cómo ha llegado a ocurrir que estos
informes aparentemente fácticos hayan llegado a quedar escritos y transmitidos
de este modo y qué tipo de hipótesis podría explicar este hecho. Si
efectivamente estos hechos han ocurrido, ¿de qué naturaleza esperaríamos que
fueran los documentos y noticias que nos los transmiten? Sin embargo, no sería
razonable pedir a un historiador indiferente que resuelva este problema, sobre
cómo han llegado a existir estos documentos y tradiciones, no sería irrazonable
para él dejar sin respuestas estas
preguntas.\footnote{\cite[Cf.~][37]{anscombe2008faith:prophandmi}: it is not
  reasonable to ask an indiferent historian to solve this problem, of how such
  records came to be written; he can reasonably just leave it unsolved.}

En donde Elizabeth estima que Lessing no tiene razón es en decir que ninguna
certeza histórica puede ser suficientemente fuerte como para tener un peso
absoluto. Lessing hace alusión al error que puede suponer saltar desde verdades
históricas a conclusiones que son verdades de una clase distinta, pero da
importancia también a esta otra cuestión sobre la fuerza que puede tener una
afirmación histórica para justificar nuestras creencias. Si es la fuerza de la
certeza lo que se está realmente poniendo en duda, le parece a Anscombe que no
es cierto que la certeza histórica sea siempre demasiado débil como para
fundamentar una certeza absoluta.

Lessing concede a un dato histórico como la existencia de Alejandro Magno el
grado de certeza de probabilidad. Anscombe juzga que la probabilidad, en
oposición a la total certeza, entra en juego más tarde para un dato como este.
Así afirma: \blockquote[{\cite[26]{anscombe2008faith:prophandmi}}: I should not
mind staking anything whatever on the existence of Alexander, or foreswearing
for ever any proferred appearance of knowledge that conflicted with it.]{No me
  importaría arriesgar cualquier cosa en la existencia de Alejandro, o renunciar
  para siempre a cualquier ofrecimiento de aparente conocimiento que entre en
  conflicto con esto.} Donde empezaríamos a hablar en términos de probabilidad
sería si nos preguntamos a quién nos referimos por \enquote{Alejandro} ---si en
algún momento fue reemplazado por un impostor, por ejemplo--- pero acerca de la
existencia de Alejandro la certeza es de mayor grado. En definitiva, no todos
los datos históricos tienen el mismo grado de certeza, y es un error no
distinguir el valor fundamental que llegan a tener ciertas afirmaciones
históricas; en palabras de Anscombe:
\blockquote[{\cite[27]{anscombe2008faith:prophandmi}}: I object to his lumping
together everything historical as of inferior certainty to my own
experience]{Estoy opuesta a su modo de amontonar todo lo histórico como de
  inferior certeza a mi propia experiencia}.

Para Anscombe hay proposiciones históricas que forman parte del conocimiento
común de tal manera que no se pueden poner en duda sin más, puesto que si se
duda de una proposición tan presente en el conocimiento general se hace
imposible afirmar el conocimiento que pueda ofrecer del todo cualquier otra
evidencia histórica. Es así que podríamos dudar de una experiencia personal, es
probable que lo que creemos conocer por nuestra experiencia no haya sido tal,
\blockquote[{\cite[27]{anscombe2008faith:prophandmi}}: whereas things making it
remotely probable that there was no Alexander are inconceivable]{mientras que
  cosas que hagan remotamente probable que no hubo un Alejandro son
  inconcebibles}. Esto se debe a que:
\blockquote[{\cite[27]{anscombe2008faith:prophandmi}}: there could be no reason
to think one knew what any historical evidence suggested at all, if a great
range of things in history were not quite solid. Experience, unless it is made
right by definition, is not more but less certain; and what I judge from
experience may, some of it, more easily be wrong.]{no podría haber razón alguna
  para pensar que sabemos qué podría sugerir del todo cualquier evidencia
  histórica, si un amplio rango de cosas en la historia no fuera del todo
  sólido. La experiencia, a no ser que sea hecha cierta por definición, no es
  mayor, sino de menor certeza; y lo que yo juzgo desde la experiencia puede, en
  parte, ser con mayor facilidad incorrecto.}

Ahora bien, ¿qué solidez tienen los datos históricos relacionados con Cristo?.
Que Jesús existió, y predicó como lo hacían los profetas del Antiguo Testamento,
y que fue al menos ostensiblemente crucificado bajo la autoridad romana y que
los creyentes lo tomaron como el Mesías y el Hijo de Dios y creyeron que
resucitó de los muertos; estos datos históricos cuentan con la solidez antes
descrita. Que Jesús declaró ser el Hijo de Dios, y que resucitó de los muertos
no son sólidos de esta manera. Si algún escrito, de Tácito digamos, afirmara que
los cristianos creían que Jesús se habría escondido y no moriría nunca y
visitaba en secreto a los creyentes; esto no sería evidencia de las genuinas
creencias de los discípulos y de que nos equivocamos en nuestras impresiones
actuales de estas creencias, sino que sería evidencia de que Tácito escribió
descripciones mal informadas de las creencias de los cristianos. El conocimiento
histórico general de las creencias de los cristianos de entonces sería la medida
para juzgar el escrito de Tácito y no al revés.

Hay ciertas afirmaciones históricas que son sólidas y que pueden emplearse como
justificación suficiente para certezas absolutas. Algunos datos relacionados con
Jesús pueden ser valorados así y por tanto no pueden ponerse en duda sin más.
Otras afirmaciones históricas sobre Jesús que no tienen esta solidez, sin
embargo tampoco pueden ser razonablemente afirmadas como falsas. El hecho de la
muerte, la ausencia de su cuerpo en el sepulcro, su reaparición tras la muerte,
y también su declaración de ser el Hijo de Dios:
\blockquote[{\cite[28]{anscombe2008faith:prophandmi}}: these belong to the very
large realm of historical assertions which it would indeed be absurd to claim
certainty for, but the time for disproving which is past \textelp{} with them
there is no danger of running up against a disproof of them, and the greater
part of them must be true: but of any particular one, we cannot say it is
perfectly certain. We may note that the death of Christ would be refuted, in
normal circumstances, just by his reappearance alive.]{estas pertenecen al
  amplio campo de afirmaciones históricas de las cuales sería ciertamente
  absurdo afirmar certeza, pero el tiempo para refutarlas ya ha pasado
  \textelp{} con estas no hay peligro de toparse con algo que las contradiga, y
  la mayor parte de ellas debe ser verdadera: pero de alguna en particular, no
  podemos decir que es perfectamente cierta. Podemos destacar que la muerte de
  Cristo sería refutada, en circunstancias ordinarias, justo por su reaparición
  en vida}. Anscombe piensa que Lessing no está consciente de la existencia de
esta clase de aserciones.

Tras estos análisis sobre las noticias de milagros y la fuerza de la certeza
histórica, Anscombe dirige su discusión hacia las profecías. En el centro de su
reflexión está el requisito propuesto por Lessing:
\blockquote[{\cite[29]{anscombe2008faith:prophandmi}}: in order to say `This was
predicted, and it happened' we have to judge that the thing that happened, not
merely was describable in the words occurring in the prediction, but was what
was predicted: otherwise `fulfilment' equals `applicability of these words'; and
can't this just be an accident?]{para poder decir `Esto fue predicho, y ocurrió'
  tendríamos que juzgar que lo ocurrido, no solo puede ser descrito por las
  palabras que aparecen en la predicción, sino que es lo que fue predicho de
  hecho: de otro modo `realización' es igual a `aplicabilidad de estas
  palabras'; y ¿puede no ser esto simplemente un accidente?}
Anscombe sostiene que hay dificultades especiales acerca de la noción de la
aplicabilidad de las palabras proféticas como \emph{accidental}.

Elizabeth ofrece una ilustración para esto. Un personaje en una obra teatral se
presenta como un personaje del pasado y describe hechos históricos de épocas
posteriores a la suya y que nosotros conocemos, el efecto sería ficticio, lo que
el autor quiere decir estaría claro. Sin embargo, si sale a relucir que estas
afirmaciones fueron realmente hechas por una persona en el pasado, entonces al
instante se convierten en palabras vagas y problemáticas.
\blockquote[{\cite[31]{anscombe2008faith:prophandmi}}: This is a logical point:
of the many, many utterances we might make now about the present or the past,
which have a good sharp sense, by far the greater number would look hopelessly
obscure if said earlier, of the future: even ones with proper names]{Esto es un
  punto lógico: de las muchas, muchas afirmaciones que podríamos hacer ahora
  acerca del presente o del pasado, las cuales tienen un sentido claro, por
  mucho la mayoría se vería irremediablemente oscura si hubiera sido dicha
  antes, sobre el futuro: incluso aquellas que contienen nombres propios}
Anscombe insiste en distinguir que las afirmaciones sobre el pasado o el
presente no significan de la misma manera que afirmaciones sobre el futuro. En
este sentido, si alguien afirmara un hecho verdadero del pasado y resulta que
ignoraba que había ocurrido, entonces es solo un accidente que sus palabras
aplicaran; \blockquote[{\cite[29]{anscombe2008faith:prophandmi}}: but it is
impossible to know the future of the world and of human affairs; so this test
for accident cannot be made]{pero es imposible conocer el futuro del mundo y de
  los asuntos humanos; así que esta prueba de accidente no puede ser hecha}. La
pregunta acerca de lo que un profeta quiso decir o qué tuvo en la mente cuando
afirmó lo que predijo es sin sentido:
\blockquote[{\cite[29]{anscombe2008faith:prophandmi}}: This point needs
stressing: someone who believes in a possibility of `precognition' comparable to
memory is thereby rendered incapable of understanding the nature of prophecy at
all]{Este punto merece insistencia: alguien que cree en la posibilidad de la
  `precognición' como comparable a la memoria queda así hecho incapaz de
  entender del todo la naturaleza de la profecía.}

La imposibilidad de especificar con certeza qué quiso decir el profeta, o qué
tenía en la mente al profetizar, impone una restricción severa al campo de lo
que pueda considerarse incluso como posible profecía. Quedaría limitado a
predicciones con nombres propios y predicados con un sentido bastante
definitivo. La consecuencia de esto es importante:
\blockquote[{\cite[31]{anscombe2008faith:prophandmi}}: This considerations
result in an interesting point: the critical principle that prophetical writings
must have been clearly intelligible in their own times is \emph{itself} a denial
of the possibility of all but prophecy of a very restricted type]{Estas
  consideraciones resultan en un punto interesante: el principio crítico de que
  los escritos proféticos tienen que haber sido claramente inteligibles en su
  propio tiempo es \emph{en sí mismo} una negación de la posibilidad de todo
  menos un restringido tipo de profecía.}
Lo cierto es, sin embargo, que para casi todas las profecías, tenerlas por
cumplidas, es interpretarlas, y la clave para interpretarlas es una noción
teológica.

Aquí podríamos preguntarnos \enquote{¿por qué me tendría que impresionar la
  profecía?}, \enquote{¿por qué debería de interesarme?}. La respuesta a esto
tiene que ver con el sentido o significado teológico de la profecía.
\blockquote[{\cite[32]{anscombe2008faith:prophandmi}}: a prophecy fulfiled, or a
miracle done, is supposed to \emph{attest} something]{una profecía cumplida, o
  un milagro realizado, se supone que \emph{testifica} algo}. Una predicción
cumplida que no testifica nada más allá de que lo predicho se ha realizado, no
tiene sentido profético. Esta consideración nos trae a una última afirmación
relacionada con la profecía.

Hay un sentido adicional a la noción de \enquote{accidental} distinto del
empleado por Lessing. Decir que el cumplimiento de una predicción \enquote{fue
  accidental} puede ser decir \enquote {esto no fue una profecía}. Si alguien
afirma algo sobre el futuro ---para ilustrar algo en una discusión, por
ejemplo--- y se cumple la predicción, entonces hay algo de sentido en afirmar
que \enquote{el cumplimiento fue accidental}. Pero si esto mismo se afirmara
como una profecía, entonces decir \enquote{fue accidental que se cumpliera}
puede significar que el hecho cumplido no fue lo que quiso decir la persona,
como afirmó Lessing, o que
\blockquote[{\cite[34]{anscombe2008faith:prophandmi}}: we do not allow this to
be prophecy, where `prophecy' has a \emph{theological} meaning]{no reconocemos
  que esto sea profecía, donde `profecía' tiene un sentido \emph{teológico}}.

Las conclusiones a las que Anscombe llega después de su análisis pueden
resumirse en dos cuestiones. En primer lugar se enfoca en el contraste entre dos
posiciones desde las que una persona podría acercarse al argumento de las
profecías y milagros. Una situación en la que puede estar una persona respecto
de los milagros y profecías es como un observador imparcial e indiferente. Este
solo tendría delante de él, como datos seguros, algunas profecías dispersas
relacionadas con personas y ciudades; también contaría con noticias de milagros
y del cumplimiento de profecías que, sin embargo, sería absurdo pretender que
debería de estimar como ciertamente verdaderas.

Es otra la situación en la que, a juicio de Anscombe, ha de hallarse alguien que
pueda ser interpelado por el argumento de los milagros y profecías:
\blockquote[{\cite[35]{anscombe2008faith:prophandmi}}: Only if a man is
impressed by the Old Testament, to the extent of being inclined to take it as
his teacher, has the argument from prophecies and miracles any serious
weight.]{Solo si un hombre queda impresionado por el Antiguo Testamento, hasta
  tal punto que esté inclinado a tomarlo como su maestro, tiene el argumento
  desde las profecías y los milagros algún peso serio.} Una persona que está en
esta situación se encuentra en una posición solida y razonable, sin embargo, es
tan específica y poco común hoy que puede explicar por qué el argumento no se
encuentra tan presente en la apologética actual.

La crítica de Lessing es contra un alegado peso que debería de tener un
argumento basado en los milagros y las profecías cumplidas y que para él no
tiene la fuerza para justificar la creencia en el Cristianismo. El Vaticano~I
alega, por su parte, que los milagros y profecías son sólidos argumentos
externos. Anscombe propone que estos argumentos externos presuponen una posición
específica de parte de quien pueda ser interpelado por ellos:
\blockquote[{\cite[37]{anscombe2008faith:prophandmi}}: That is to say: when St.
Augustine said that the fulfilment of the prophecies in Christ was the greatest
proof of his divinity, what he said was true; but the proof requires a very
special postiton on the part of someone who is to consider it. That is why the
kind of apologetic that Lessing argued against, which did not assume that
position, was so vulnerable and stupid.]{Es decir: cuando S. Agustín dijo que la
  realización de las profecías en Cristo es la mayor prueba de su divinidad, lo
  que dijo es verdadero; pero la prueba requiere una posición de parte de
  alguien que podría considerarla. Esta es la razón por la que el tipo de
  apologética en contra de la cual Lessing argumentó, en la que no se asume esta
  posición, queda tan vulnerable y estúpida.}

El argumento de los milagros y profecías cumplidas sí juega un papel razonable
como atestación que justifica la creencia en Cristo para una persona que ha
valorado suficientemente las enseñanzas del Antiguo Testamento como para tenerlo
como una fuente de instrucción y ha formado su mente de acuerdo a él. Una
persona que reconoce la solidez que pueden tener los milagros y profecías
cumplidas como signo del cumplimiento de las promesas del Antiguo Testamento en
Cristo podría entonces preguntarse sobre cómo se han transmitido estos relatos.
Anscombe llega entonces a la siguiente conclusión:
\blockquote[{\cite[37]{anscombe2008faith:prophandmi}}: The role of miracles,
which I have contended cannot possibly be accepted as certainly true ocurrences
by the indiferent historian, seems to me to be this: if one is seriously
entertaining the truth of the whole revelation in the way I have hinted at, the
miracles are consonant. That God attested Christ by miracles is possible, if
Christ is Christ ---i.e. is the Messiah promised in the Old Testament. Then the
problem, how on earth these seemingly factual records came to be written, of
such incredible things, is resolved by the hypothesis that they happened.
\textelp{} But I repeat, it is not reasonable to ask an indiferent historian to
solve this problem, of how such records came to be written; he \emph{can}
reasonably just leave it unsolved.]{El rol de los milagros, los cuales he
  argüido que no es posible aceptar como hechos ciertamente verdaderos por un
  historiador indiferente, me parece que es este: si alguien está seriamente
  considerando la verdad de toda la revelación en el modo que he sugerido, los
  milagros están en consonancia. Que Dios atestó a Cristo por medio de milagros
  es posible, si Cristo es Cristo ---es decir, es el Mesías prometido en el
  Antiguo Testamento. Entonces el problema, cómo es posible que estos informes
  aparentemente fácticos hayan llegado a quedar escritos, de estas cosas
  increíbles, se resuelve por la hipótesis de que ocurrieron. \textelp{} Pero
  repito, no es razonable pedir a un historiador indiferente que resuelva este
  problema, sobre cómo estos informes han llegado a quedar escritos; el
  \emph{puede} razonablemente dejarlo sin resolver.}

La segunda cuestión que Anscombe propone como conclusión tiene que ver con la
noción misma de la atestación divina. El hecho de que una persona que haga
prodigios o pronuncie profecías que se cumplen no demuestra necesariamente que
es un testigo de Dios o su enseñanza una atestación divina. Anscombe considera
que hay un criterio adicional para justificar esa creencia:
\blockquote[{\cite[38]{anscombe2008faith:prophandmi}}: So far as I can see there
has to be a thesis of natural theology, as I might call it, that if someone
works `a sign and a wonder' or utters a prophecy which gets fulfilled, in God's
name, then he is divinely attested. Now what does this rest on? It might rest on
faith.]{Hasta donde puedo ver tiene que haber una tesis de teología natural,
  como podría llamarla, que si alguien realiza `un signo y un prodigio' o
  pronuncia una profecía que queda cumplida, en el nombre de Dios, entonces está
  divinamente atestado. Ahora ¿en qué se basa esto? Puede estar respaldado por
  la fe.} Por ejemplo la fe en la promesa del Deuteronomio, de que vendrá otro
profeta como Moisés, ofrece como criterio que antes de preguntarse si se ha
cumplido lo profetizado, las enseñanzas de los profetas deberían ser tales que
se pueda pensar que pertenecen a la verdad revelada por Moisés. Es entonces que
si el profeta predice algo y se cumple, y si después de esto no trata de
conducir al pueblo a la idolatría, se puede tomar su profecía como atestación
divina. En este sentido se puede decir que el criterio para considerar a un
profeta como testigo divino es una cuestión de fe. Sin embargo:
\blockquote[{\cite[38]{anscombe2008faith:prophandmi}}: if \textins{what}
constitutes divine attestation is only learned by faith, what becomes of the
`solid external arguments' of the Vatican decree?]{si \textins{lo que}
  constituye una atestación divina solo se conoce por la fe, ¿en qué quedan los
  `sólidos argumentos externos' de la constitución del Vaticano?}. Si se tiene
esta enseñanza en cuenta tendría que ser posible un criterio que no tenga como
presupuesto la fe. Anscombe propone el siguiente análisis:
\blockquote[{\cite[38]{anscombe2008faith:prophandmi}}: I think the argument must
be rather that if a prophet who is apparently teaching the truth, dares foretell
something contingent, then this is presumption of him unless he has it from God
and must say it. Now if he teaches a lie straight away afterwards, or if the
thing does not happen, then he is proved presumptuous. But if he is not proved
presumptuous, then we ought not to dare not to believe and obey him: so long as
what he says does not conflict with the known truth.]{Pienso que el argumento ha
  de ser más bien que si un profeta que está aparentemente enseñando la verdad,
  se atreve a predecir algo contingente, entonces esto es presunción suya
  excepto si lo ha recibido de Dios y debe decirlo. Ahora si enseña una mentira
  inmediatamente después, o si lo predicho no ocurre, entonces queda probado
  como presuntuoso. Pero si no es probado presuntuoso, entonces no deberíamos
  atrevernos a no creerle y obedecerle: siempre que lo que dice no esté en
  conflicto con la verdad conocida.}

Anscombe termina haciendo una distinción; quizás podemos actuar según la
profecía \enquote{porque no deberíamos atrevernos a actuar de otro modo}, pero
¿sería esto justificación suficiente para afirmar una creencia?. Este criterio
puede servir para remover dudas a la hora de hacer un juicio razonable sobre una
alegada atestación divina, sin embargo, no ofrece una razón positiva para creer.
Esta razón positiva, según alude Elizabeth, se encuentra en la consonancia con
la doctrina conocida: \blockquote[{\cite[39]{anscombe2008faith:prophandmi}}:
Surely one wants positive reason to believe, and not merely absence of positive
reason to disbelieve? This, it seems to me, is correct, and goes with the thesis
that in some sense there cannot be a prophet with a new doctrine. ]{¿Sin duda
  quisiéramos razón positiva para creer, y no solo ausencia de razones positivas
  para dudar? Esto, según mi parecer, es correcto, y va con la tesis de que en
  cierto sentido no puede haber un profeta con una nueva doctrina.}

\subsection{Parmenides, Mystery and Contradiction}

En 1981 Anscombe publicó una colección de sus escritos en tres volúmenes
llamados \emph{The Collected Philosophical Papers of G.\,E.\,M.\,Anscombe}. El
primero de estos, titulado \emph{From Parmenides to Wittgenstein} recoge un tema
presente en el \emph{Tractatus} de Wittgenstein y que Anscombe trató con gran
interés: la relación entre lo concebible y lo posible. Lo concebible está
vinculado a lo que puede ser dicho, y la cuestión acerca de lo que puede ser
dicho claramente nos introduce en esos temas propios de Wittgenstein como son la
falta de significado, el sinsentido, lo misterioso y lo inefable. En este
volumen Anscombe agrupa distintos escritos suyos que examinan las reflexiones de
diversos autores en torno a esta cuestión. Establece así un diálogo con
filósofos como Parmenides, Platón, Hume y Wittgenstein en una única
discusión.\footnote{Cf.~ \cite[193]{teichmann2008ans}: Philosophers have
  grappled since ancient times with the problem of how thinkability and
  possibility are related, and it is characteristic of Anscombe to have drawn
  such diverse figures as Parmenides, Plato, Hume, and Wittgenstein into a
  single discussion.}

\emph{Parmenides, Mystery and Contradiction} es el primer artículo que se
encuentra en este volumen. El texto es de la ponencia ofrecida por Anscombe en
la reunión del \emph{Aristotelian Society} en \emph{21, Bedford Square} en
Londres el 24 de febrero de 1969. Una buena clave para acercarse al análisis de
este artículo se puede encontrar en la introducción al presente volumen de la
colección: \blockquote[{\cite[xi]{anscombe1981parmenides}}: It was left to the
moderns to deduce what could be from what could hold of thought, as we see Hume
to have done. This trend is still strong. But the ancients had the better
approach, arguing only that a thought was impossible because the thing was
impossible, or, as the Tractatus puts it, ``Was man nicht denken kann, das kann
man nicht denken'': an \emph{impossible} thought is an impossible
\emph{thought}.]{Se les dejó a los modernos el deducir lo que puede ser posible
  desde lo que puede ser sostenido en el pensamiento, como vemos hacer a Hume.
  Esta tendencia sigue siendo fuerte. Pero los antiguos tuvieron el mejor
  acercamiento, argumentando solo que un pensamiento sería imposible porque la
  cosa misma es imposible, o, como lo dice el Tractatus, ``Was man nicht denken
  kann, das kann man nicht denken'': un pensamiento \emph{imposible} es un
  \emph{pensamiento} imposible.}

\subsection{Hume and Julius Caesar}

Los artículos \emph{Hume and Julius Caesar} y \emph{``Whatever has a beginning
  of existence must have a cause'': Hume’s Argument Exposed} de Anscombe, fueron
publicados en la revista académica \emph{Analysis} en octubre de 1973 y abril de
1974 respectivamente. Ambos están relacionados por el tema de la causalidad en
Hume. En el trasfondo de los dos artículos está otro documento no publicado
hasta 2011 con el título \emph{Hume on causality: introductory}.

Anscombe again and again found in Hume a starting point for her discussions; and
we must not be misled bye her frequent dissent from his views into thinking of
her as `anti-Humean'. Indeed, in her treatment of the topic of causation
Anscombe can even be seen as continuing Hume's work---as out-Huming Hume.
teichmann 177

  Una de las actitudes características de Anscombe es su tendencia a quedar
  atraída por preguntas que representan cuestiones profundas, incluso en
  discusiones cuyos argumentos, método o conclusiones no le parecen tan
  interesantes.

  Un autor que suele tener este efecto en ella es Hume. En \emph{Modern Moral
    Philosophy} dice:

  \blockquote[{\cite[172]{anscombe1981mmph}}: The features of Hume’s philosophy
  which I have mentioned, like many other features of it, would incline me to
  think that Hume was a mere ---brilliant--— sophist; and his procedures are
  certainly sophistical. But I am forced, not to reverse, but to add to this
  judgement by a peculiarity of Hume’s philosophizing: namely that, although he
  reaches his conclusions --—with which he is in love--— by sophistical methods,
  his considerations constantly open up very deep and important problems. It is
  often the case that in the act of exhibiting the sophistry one finds oneself
  noticing matters which deserve a lot of exploring: the obvious stands in need of
  investigation as a result of the points that Hume pretends to have made.]{Las
    características de la filosofía de Hume que he mencionado, como muchas otras
    de sus características, me hacen inclinarme a pensar que Hume era un simple
    ---brillante--- sofista; y sus procedimientos son ciertamente sofísticos. Sin
    embargo me veo forzada, no a retractarme, sino a añadir a este juicio por la
    peculiaridad del filosofar de Hume: a saber, que aunque llega a sus
    conclusiones ---con las que está enamorado--- por métodos sofísticos, sus
    consideraciones constantemente abren problemas bien profundos e importantes.
    Frecuentemente es el caso que en el acto de exhibir la sofística uno se
    encuentra a sí mismo notando temas que merecen mucha exploración: lo obvio
    queda necesitado de investigación como resultado de los puntos que Hume
    pretende haber hecho.}

En el artículo \emph{Hume and Julius Caesar} la discusión que capta el interés
de Anscombe se encuentra en la sección IV de la tercera parte del \emph{Treatise
  of Human Nature} sobre el tema de la justificación de nuestro creer en
cuestiones que están más allá de nuestra experiencia y memoria. Anscombe cita el
texto de Hume como sigue:

\blockquote[{\cite[86]{anscombe1981hjc}}When we infer effects from causes, we
must establish the existence of these causes\ldots either by an immediate
perception of our memory or senses, or by an inference from other causes; which
causes we must ascertain in the same manner either by a present impression, or
by an inference from their causes and so on, until we arrive at some object
which we see or remember. 'Tis impossible for us to carry on our inferences
\emph{in infinitum}, and the only thing that can stop them, is an impression of
the memory or senses, beyond which there is no room for doubt or enquiry.
(Selby-Bigge's edition, pp. 82--3)]{Cuando inferimos efectos partiendo de causas
  debemos establecer la existencia de estas causas\ldots ya sea por la
  percepción inmediata de nuestra memoria o sentidos, o por la inferencia
  partiendo de otras causas; causas que debemos explicar de la misma manera por
  una impresión presente, o por una inferencia partiendo de sus causas, y así
  sucesivamente hasta que lleguemos a un objeto que vemos o recordamos. Es
  imposible para nosotros proseguir en nuestras inferencias al infinito, y lo
  único que puede detenerlas es una impresión de la memoria o los sentidos más
  allá de la cual no existe espacio para la duda o indagación.}

Ya en la sección II de esta misma parte del \emph{Treatise}, Hume ha planteado
cómo es la causalidad la conexión que nos asegura la existencia o acción de un
objeto que es seguido o precedido por la existencia o acción de
otro.\footnote{Cf. Treatise Sección II Parte III: ’Tis only causation, which
  produces such a connexion, as to give us assurance from the existence or
  action of one object, that ’twas follow’d or preceded by any other existence
  or action; nor can the other two relations be ever made use of in reasoning,
  except so far as they either affect or are affected by it. }
Ahora en la sección IV esta relación de causa y efecto será tomada como un
principio de asociación de ideas según el cual es posible inferir desde la
impresión de alguna cosa, una idea sobre otra cosa.

Desde esta noción de causalidad se explica la posibilidad de acceder a hechos
más allá de nuestra experiencia; estos son inferencias de efectos desde sus
causas. De este modo: \blockquote[{\cite[87]{anscombe1981hjc}}: For Hume, the
relation of cause and effect is the one bridge by which to reach belief in
matters beyond our present impressions or memories.]{Para Hume, la relación de
  causa y efecto es el único puente por el que se puede alcanzar creer en
  cuestiones más allá de nuestras impresiones presentes o memorias.}

El paso adicional que Hume propone en esta sección es que al realizar estas
inferencias es necesario establecer la existencia de las causas por medio de la
percepción inmediata de los sentidos o por medio de una ulterior inferencia. Sin
embargo, el establecimiento de la existencia de estas causas por medio de
inferencias no puede continuar infinitamente, sino que tiene que llegar a una
impresión de la memoria o los sentidos que sirva de justificación o fundamento
definitivo.

Para ilustrar este paso, Hume hace una invitación interesante:
  \blockquote[{\cite[?]{humetreatise}}: chuse any point of history, and consider
  for what reason we either believe or reject it.]{elegir cualquier punto en la
    historia, y considerar por qué razón lo creemos o rechazamos.} Acerca de una
  creencia histórica se nos invita a considerar sobre qué se sostiene su
  justificación. ¿Cuál es su fundamento?:
  \blockquote[{\cite[?]{humetratise}}: Thus we believe that Cæsar was kill’d in
  the senate-house on the ides of March; and that because this fact is establish’d
  on the unanimous testimony of historians, who agree to assign this precise time
  and place to that event. Here are certain characters and letters present either
  to our memory or senses; which characters we likewise remember to have been us’d
  as the signs of certain ideas; and these ideas were either in the minds of such
  as were immediately present at that action, and receiv’d the ideas directly from
  its existence; or they were deriv’d from the testimony of others, and that again
  from another testimony, by a visible gradation, ’till we arrive at those who
  were eye-witnesses and spectators of the event. ’Tis obvious all this chain of
  argument or connexion of causes and effects, is at first founded on those
  characters or letters, which are seen or remember’d, and that without the
  authority either of the memory or senses our whole reasoning wou’d be chimerical
  and without foundation.]{Así, creemos que César fue asesinado en el Senado en
    los idus de Marzo; y esto porque el hecho está establecido basándose en el
    testimonio unánime de los historiadores, que concuerdan en asignar a este
    evento este tiempo y lugar precisos. Aquí ciertos caracteres y letras se
    hallan presentes a nuestra memoria o sentidos; caracteres que recordamos
    igualmente que han sido usados como signos de ciertas ideas; y estas ideas
    estuvieron ya en las mentes de los que se hallaron inmediatamente presentes a
    esta acción y que obtuvieron las ideas directamente de su existencia; o fueron
    derivadas del testimonio de otros, y éstas a su vez de otro testimonio, por
    una graduación visible, hasta llegar a los que fueron testigos oculares y
    espectadores del suceso. Es manifiesto que toda esta cadena de argumentos o
    conexión de causas y efectos se halla fundada en un principio en los
    caracteres o letras que son vistos o recordados y que sin la autoridad de la
    memoria o los sentidos nuestro razonamiento entero sería quimérico o carecería
    de fundamento.}

Anscombe comienza por reaccionar afirmando:
\blockquote[{\cite[86]{anscombe1981hjc}}: This is not to infer effects from
causes, but rather causes from effects.]{Esto no es inferir efectos partiendo de
  sus causas, sino más bien causas desde los efectos.} Es decir, el ejemplo
histórico de Hume consiste en una inferencia de la causa original, el asesinato
de Julio César, desde su efecto remoto que es nuestra percepción en el presente.
Creemos en el asesinato de César porque lo inferimos como la causa última en una
cadena causal que llega hasta nuestra percepción de ciertas oraciones que
leemos. El hecho de que estemos leyendo esta información es la percepción que
justifica la creencia de que hay una cadena de causas y efectos que tiene como
efecto esta experiencia. Esta inferencia pasa a través de una cadena de efectos
de causas, que son efectos de causas\ldots ¿Dónde empieza la cadena? La
respuesta parece ser nuestra percepción presente. ¿Cómo hemos de entender,
entonces, el argumento de que la cadena no puede continuar infinitamente? La
propuesta de Hume es que la cadena ha de terminar en una impresión que no deje
lugar a dudas o busqueda mas allá, sin embargo, la cadena termina en el
asesinato de Julio César, no en nuestra percepción. La imagen que Hume pretende
ofrecer es la de una cadena fijada en sus dos extremos por algo distinto a los
eslabones que la componen, sin embargo, no lo logra, más bien parece describir
un voladizo, una estructura apoyada en un punto, pero sin apoyo en el otro
extremo.

La afirmación \blockquote['Tis impossible for us to carry on our inference in
infinitum]{Es imposible para nosotros proseguir en nuestras inferencias al
  infinito} viene a significar, según la interpretación de Anscombe, que
\blockquote[the justification of the grounds of our inferences cannot go on in
infinitum]{la justificación de los fundamentos de nuestras inferencias no pueden
  continuar al infinito}. El argumento aquí mas bien es que tiene que haber un
punto de partida para la inferencia de la causa original. La relación de
inferencias propuesta por Hume en su ilustración acabaría siendo una inferencia
hipotética según su propia definición. Anscombe explica diciendo:

\blockquote[hume in causality: We must suppose ourselves to start with the
familiar idea, merely as idea, of Caesar having been killed. Now if we ask why
we believe it we shall, as Hume does, point to historical testimony (the
‘characters and letters’), which doesn’t at this point figure as what stops
inference going on ad infinitum. However, if we want to explain the connection
we shall form the idea of Caesar’s death being recorded by eyewitnesses; and
these records having been received by others, who transmitted an account ...
etc. Here we really are arguing from the idea of an original cause to the idea
of an effect; we are ‘inferring effects from causes’, though only in the sense
of passing from the idea of the cause to the idea of the effect.]{Tendríamos que
  suponer que comenzamos con la idea familiar, meramente como una idea, de que
  César fue asesinado. Ahora si preguntamos por qué lo creemos hemos de, como
  hace Hume, señalar al testimonio histórico (los `caracteres y letras'), lo
  cual en este punto no figura como lo que detiene que la inferencia siga al
  infinito. Sin embargo, si queremos explicar la conexión tenemos que formular
  la idea de la muerte del Cesar siendo recordada por testigos; y esos recuentos
  siendo recibidos por otros, quienes transmitieron un informe\ldots etc. Aquí
  estamos realmente razonando desde la idea de una causa original a la idea de
  un efecto; estamos `infiriendo efectos de causas', pero solo en el sentido de
  pasar de la idea de la causa a la idea del efecto.}

Desde este análisis, Anscombe resume lo argumentado por Hume en cuatro partes:

\blockquote[humeandjulius 88: First, a chain of reasons for a belief must
terminate in something that is believed without being founded on anything else.
Second, the ultimate belief must be of a quite different character from derived
beliefs: it must be perceptual belief, belief in something perceived, or
presently remembered. Third, the immediate justification for a belief p, if the
belief is not a perception, will be another belief q, which follows from, just
as much as it implies, p. Fourth, we believe by inference through the links in a
chain of record

There is an implicit corollary: when we believe in historical information
belonging to the remote past, we believe that there has been a chain of record]{
  Primero, una cadena de razones para una creencia debe terminar en algo que se
  cree sin estar fundado en alguna otra cosa. Segundo, la creencia última debe
  ser de una naturaleza distinta a las creencias derivadas: Tiene que ser
  creencia perceptual, creer en algo percibido, or recordado en el presente.
  Tercero, la justificación inmediata de una creencia $p$, si la creencia no es
  una percepción, será otra creencia $q$, la cual se sigue, en la misma medida
  que implica, a $p$. Cuarto, creemos por inferencia a través de los eslabones
  en una cadena de relato.

  Hay un corolario implicito: cuando creemos en información histórica
  perteneciente a un pasado remoto, creemos que ha habido una cadena de relato.}

Sin embargo, Anscombe considera que esta no es la manera adecuada de establecer
esta relación. Mas bien: \blockquote[hjc 88: \emph{If} the written records that
we now see are grounds of our belief, they are first and foremost grounds for
belief in Caesar's killing, belief that the assassination is a solid bit of
history. Then our belief in that original event is a ground for belief in much
of the intermediate transmision.]{\emph{Si} los relatos escritos que vemos ahora
  son fundamento para nuestro creer, estos son primero y ante todo fundamento
  para la creencia en el asesinato de Cesar, creencia en que el asesinato es un
  pedazo sólido de historia. Entonces nuestra creencia en ese evento original es
  fundamento para el creer en mucha de la transimisión intermedia.}
¿Por qué creemos que hubo testigos del asesinato? Ciertamente porque creemos que
hubo un asesinato. La creencia de que hubo testigos es inferida de la creencia
en el hecho.

Anscombe compara este modo de entender la cadena de transmisión de información
histórica a nuestra creencia en la continuidad espacio-temporal. Si reconocemos
en una ocasión a una persona conocida como alguien que vimos la semana pasada,
nuestra creencia en que es la misma persona no es una inferencia de otra
creencia acerca de la continuidad espacio-temporal de un patrón humano desde
ahora hasta entonces, sino que más bien nuestra creencia en la continudad
espacio-temporal esta inferida del reconocimiento de la identidad de la persona.
Sin embargo, una evidencia sobre una interrupción en la continuidad sí alteraría
nuestra creencia en la identidad.

Elizabeth entonces concluye que: \blockquote[hjc 89: Belief in recorded history is
on the whole a belief that there has been a chain of tradition of reports and
records going back to contemporary knowledge; it is not a belief in the
historical facts by an inference that passes through the links of such a chain.
At most, that can very seldom be the case.]{La creencia en los registros de la
  historia consiste en general la creencia de que ha habido una cadena de
  tradición de informes y registros que van hacia el conocimiento contemporaneo;
  no es una creencia en hechos históricos por una inferencia que pasa por los
  eslabones de una cadena como esta. Como mucho, esto seria muy raramente el
  caso.}

Ahora bien, como se ha dicho antes, el interés de Anscombe no esta simplemente
en mostrar en qué se equivoca Hume, sino que considera que la cuestión toca el
nervio de un problema con cierta profundidad:
\blockquote[causality in hume 2855: The interesting problem that arises, then,
is why the things we are told and the writings that we see are the starting
points for our belief in the far distant events and so in the intermediate chain
of record.]{El problema interesante que surge, entonces, es por qué las cosas
  que se nos dicen y los escritos que vemos son puntos de partida para nuestro
  creer en eventos distantes y así también en la cadena del relato intermedia.}

El argumento de Hume, entonces, se compone de dos partes.
  En primer lugar, una cadena de inferencia en la cual "ya que p, q, etc..."
  en la que p da una causa creida (no percibida) y q un efecto inferido, no puede continuar
para siempre, sino que tiene que terminar n

  Determina que estas inferencias no pueden continuar infinitamente. Si se tratara
  de mera relación especulativa de conceptos no representaría dificultad, pero se
  trata de creer, y la cadena no podría ofrecer una creencia si no tiene término.
  \blockquote[{\cite[2762]{anscombe2011hoc}}: Now there really is no difficulty
  about going on ad infinitum, or at any rate about saying ‘and so on ad
  infinitum’, if the ‘inferring’ is simply deriving the idea of the effect from
  that of the cause. But the inferring is more than that ---it is believing. It is
  in connection with this that Hume is saying ‘this chain can’t go on for
  ever’.]{Ahora realmente no hay dificultad en ir infinitamente, o en cualquier
    caso decir `así sucesivamente infinitamente', si el `inferir' es simplemente
    derivar la idea del efecto partiendo de su causa. Pero el inferir es más que
    eso ---es creer. Es en conexión con esto que Hume dice `esta cadena no puede
    seguir para siempre'}

  First, a chain ‘Since p, q, etc’ in which p gives a believed-in (not perceived)
  cause and q an inferred effect, cannot go on for ever but must terminate in a
  proposition that is believed without inferring any consequences from it; and
  from this proposition we then work back in reverse order to p.

  This is a particular form of a familiar argument that not everything can be
  argued from something else, that is: that it cannot be the case that everything
  is argued from something else. I believe p because I believe q because I believe
  r because I believe s ---this cannot go on for ever; it must end in something
  which I believe, not because I believe something else. This argument appears to
  be correct.

Hume’s second point is that not merely must the chain that he is concerned with
come to an end somewhere, but its terminus must be of a different kind from the
other members. ... without the authority either of the memory or the senses our
whole reasonings wou’d be chimerical and without foundation. Every link of the
chain wou’d in that case hang upon another; but there wou’d not be anything
fix’d to one end of it, capable of sustaining the whole; and consequently there
wou’d be no belief or evidence.[27]


The second part of his argument, which says that the terminus must be of a
different character from the links of the chain, is more doubtful than the first
part which only says there must be a terminus. Hume does not think that I have
to have a present perception (of memory or sense) in connection with my belief
that Caesar was killed in the Senate House: we can ‘reason upon our past
conclusions and principles, without having recourse to those impressions from
which they first arose.’ The convictions, however, must have been produced by
impressions, and ‘all reasonings concerning causes and effects are originally
deriv’d from some impression’.



Para discutir esta cuestión Anscombe recurre a las reflexiones de Wittgenstein
en \emph{On Certainty}. La motivación para estos ecritos de Wittgenstein son las
propuestas de Moore en \emph{Proof of the External World} y \emph{Defence of
  Common Sense}. En estas obras sostiene que hay una serie de proposiciones que
conocemos con seguridad, como \enquote{Aquí hay una mano, y aquí otra}, o
\enquote{La tierra ha existido por largo tiempo antes de mi nacimiento} y
\enquote{Nunca he estado lejos de la superficie de la tierra}. Estas reflexiones
ocuparon a Wittgenstein durante los últimos años de su vida.\footnote{Cf.
  preface On certainty}

Un tema que aparece en esta discusión de Wittgenstein es que la justificación
semántica, relacionada con el uso correcto del lenguaje, y la justificación
epistémica, relacionada como tal con el afirmar la verdad, están más unidas
entre sí de lo que se piensa. Según esto:\blockquote[teichmann 213: Wittgenstein
invites us to view the rules governing the correct use of words as comparable to
the rules governing the acceptance or rejection of beliefs (which are themselves
of course paradigmatically expressed in words); a ‘world view’ is determined as
much by our language and its attendant conceptual scheme as by what we would
ordinarily term our knowledge of things. The two aspects of world view, the two
kinds of justification, come together in the phenomenon of certainty. ‘I am
sure’, ‘I cannot doubt’ are related to ‘It must be’, which expression can be
prefixed to any statement of conceptual truth. One direction in which these
thoughts seem to take us is towards regarding certain world views, or sets of
beliefs, or very general beliefs, as no more susceptible of rational
justification or criticism than are concepts. –This is just how we go on’ looks
to be the final answer to a series of –Why?’ questions; and a language–game or
practice can appear to be sealed off from external assessment. An appeal to the
objective measure of Reality is empty in this context; we can of course –cite
reality’ when giving reasons in justification of a belief or practice, but that
our reasons count as good reasons is determined by norms or rules of reasoning
whose status as rules depends on the existence of a surrounding
language–game.]{Wittgenstein nos invita a ver las reglas que gobiernan el uso
  correcto de las palabras como comparables con las reglas que gobiernan la
  aceptación o rechazo de las creencias (que desde luego son ellas mismas
  paradigmáticamente expresadas en palabras); una `cosmovisión' está determinada
  tanto por nuestro lenguaje y su esquema conceptual relacionado como por lo que
  ordinariamente expresamos como nuestro conocimiento de las cosas. Los dos
  aspectos de la cosmovisión, los dos tipos de justificación, quedan unidos en
  el fenómeno de la certeza. [\ldots] Una dirección hacia la que estos
  pensamientos parecen dirigirnos es a considerar ciertas cosmovisiones, o
  colecciones de creencias, o creencias generales, como no más susceptibles de
  justificación racional o crítica que la que tienen los conceptos}.

Anscombe aplica las lecciones de \emph{On Certainty} al conocimiento histórico
en la linéa propuesta por Hume: ``elegir cualquier punto en la historia, y
considerar por qué razón lo creemos o rechazamos''. Elegir o rechazar una
creencia como la propuesta implica la identificación de una justificación
suficiente, y aquí esta busqueda esta regida por reglas comparables al correcto
uso de las palabras. Los dos puntos principales destacados por Anscombe serán:
\blockquote[grounds of belief 183: Hume's philosophical opinion was that these
ultimate groundless grounds were sense impressions. But I say that they are such
beliefs as those of which one will say `Everyone knows that!' or `Everyone who
knows anything on such matters at all, knows that!']{La opinion filosófica de
  Hume era que estos fundamentos-sin-fundamento definitivos eran impresiones de
  los sentidos. Pero yo digo que son ese tipo de creencias de las cuales uno
  dice `¡Todo el mundo sabe eso!' o `¡Todo el que sabe algo de ese tema, sabe
  eso!'}. Junto a esto, es también parte de su argumento:
\blockquote[teichmann 224: the mere statement that we can conceive of evidence
turning up which showed there had never been such a person as Julius Caesar is
no good until details are given of what sort of evidence that might be. If we
try to do this, however, we are likely to fail.]{la declaración de que puede ser
  concebido que aparezca evidencia que mostrara que nunca ha habido una persona
  como Julio César no es suficiente hasta que se den detalles acerca del tipo de
  evidencia que ésta pudiera ser. Si intentamos hacer esto, sin embargo, lo más
  probable es que fracasemos.}

Para entender su primera propuesta será útil recurrir a su explicación de este
punto como está planteado en \emph{On Certainty}: \blockquote[QLI, 130: Finding
grounds, testing, proving, reasoning, confirming, verifying are all processes
that go on within, say, one or another living linguistic practice which we have.
There are assumptions, beliefs, that are ‘immovable foundations’ of these
proceedings. By this, Wittgenstein means only that they are a foundation which
is not moved by any of these proceedings.]{Encontrar fundamentos, examinar,
  probar, razonar, confirmar, verificar son todos procesos que corresponden,
  diríamos, dentro de una u otra práctica linguística viva de las que tenemos.
  Hay supuestos, creencias, que son `fundamentos inmovibles' de estos modos de
  proceder. Con esto, Wittgenstein se refiere solamente a que son un fundamento
  que no es modificado por esos procesos.} En estos procesos o actividades hay
proposiciones que sirven como bisagras, donde se apoya el movimiento del
discurrir. Como tal, son creencias que si se ponen en duda impiden el progreso
del razonamiento. Estas creencias son esas que forman parte del conocimiento
común. En ese sentido, afirmar \enquote{aquí está mi mano} no es sostener algo
sobre el estado de las cosas en el mundo, sino establecer unas reglas para la
discusión. Por otra parte, poner en duda que tengo mi mano aquí delante
supondría tratar con escepticismo un conocimiento común de tal manera que se
podría decir \enquote{si esto es dudoso, ¿qué puede ser cierto?}, entonces
¿desde qué fundamento podríamos sostener una discusión o razonamiento sobre el
mundo en el que \enquote{aquí está mi mano} no es cierto?

Esto mismo ocurre con la creencia en el conocimiento común de la existencia de
Julio César, si nos planteamos la hipótesis de que nunca existió, nos
situaríamos entre dos alternativas, ya sea \blockquote[HJC 91: \textelp{} say:
``How could one explain all these references and implications, then?\ldots but,
but, \emph{but} if I doubt the existence of Caesar, if I say I may reasonably
call it in question, then with equal reason I must doubt the status of the
things I've just pointed to'']{\textelp{} decir ``¿Cómo se explican todas estas
  referencias e implicaciones entonces?\ldots pero, pero \emph{pero} si dudo de
  la existencia de César, si digo que podría razonablemente ponerlo en tela de
  juicio, entonces, con la misma razonabilidad tengo que dudar de la validez de
  las cosas que acabo de señalar''}. O por otra parte: \blockquote[HJC91:
\textelp{} I should realize straight away that the `doubt' put me in a vacuum in
which I could not produce reasons why such and such `historical facts' are more
or less doubtful.]{\textelp{} podría caer en cuenta inmediatamente de que la
  `duda' me ha encerrado en un vacío en el cual no podría producir razones por
  las cuales estos u otros `datos históricos' son más o menos dudosos.}

Hume escoge este punto histórico porque es un conocimiento presente en su
cultura con un grado particular de certeza. Podría haber sometido a prueba algun
detalle del suceso y cuestionar, por ejemplo, si podría dudarse la fecha o el
lugar del asesinato, sin embargo, el que ese hombre, César, existió, y su vida
terminó en un asesinato: esto solo podría cuestionarlo empleando la duda
Cartseiana.

Elizabeth alude a la analogía hecha por Otto Neurath en \emph{Anti-Spengler},
donde compara el conocimiento científico con un barco en el cual los que
investgan son como marinos que reconstruyen el barco en altamar, verificando y
reemplazando sus piezas mientras que se navega. Entonces propone que si la
ilustración implica que se puede ir examinando cada pieza y reemplazarla de tal
modo que se termina con un barco distinto, la analogía no sirve: \blockquote[HJC
92: For there are things that are on a level. A general epistemological reason
for doubting one will be a reason for doubting all, and then none of them would
have anything to test it by.]{Pues hay cosas que están sobre superficie. Una
  razón espistemológica general para dudar de una será razón para dudar de
  todas, y entonces ninguna tendría cosa alguna que sirviera para evaluarla.}


What would one REALLY have grounds for saying or thinking, in such a case?’ In
many of her articles, Anscombe refers to some view as a prejudice, or apparent
prejudice. When is a belief a prejudice, and when is it bedrock? When is it a
questionable ‘bit of Weltanschauung’, and when a ‘hinge proposition’? The answer
to these questions must in large part have to do with how much, and what sort
of, detail can be plausibly put into counter-examples to, or cases against, the
belief in question.

My knowledge of the things among which and the places in which I live is not so
much 'theory laden' as ‘common-knowledge laden'. I wish to say: it is a
falsification here to speak of testimony: to say, for example, that it is by
testimony that I know I was born. There is something else, not testimony, though
acquired by education from human beings, which is, so to speak, thicker than
testimony.

The work done, people could be taught what England was (no doubt still disputing
some regions). Now those who learned thereafter can hardly be said to have
knowledge by testimony. They were taught to call something 'England’—something
indeed which could in large part only be defined for them by hearsay; and they
so taught those who came after them. I am an heir of this tradition. Now, I know
I live in England. But by testimony? Some would say so. But there is something
queer about it. What do I know? That the world is divided up into countries
which have names, and that the one I live in is called England and is here on
the map of the globe. This involves understanding the use of the globe to
represent the earth. It is rather as if I had been taught to join in doing
something, than to believe something—but because everyone is taught to do such
things, an object of belief is generated. The belief is so certainly correct
(for it follows the practice) that it is knowledge; for here knowledge is no
other than certainly correct belief in pursuit of a practice. But the connection
with testimony is remote and indirect.

\subsection{Que se puede entender de la fe sin tenerla}
En Oscott College, el seminario de la Archidiócesis de Birmingham, se comenzaron
a celebrar las conferencias llamadas Wiseman Lectures en 1971. Para estas
lecciones ofrecidas anualmente en memoria de Nicholas Wiseman se invitaba un
ponente que tratara algún tema relacionado con la filosofía de la religión o
alguna materia en torno al ecumenismo.\footcite[Cf.~][7]{wisemanlects}

El 27 de octubre de 1975, para la quinta edición de las conferencias, Anscombe
presentó una lección titulada simplemente \emph{Faith}. Allí planteaba la
siguiente cuestión: \citalitlar{Quiero decir qué puede ser entendido sobre la fe
  por alguien que no la tenga; alguien, incluso, que no necesariamente crea que
  Dios existe, pero que sea capaz de pensar cuidadosa y honestamente sobre ella.
  Bertrand Russell llamó a la fe `certeza sin prueba'. Esto parece correcto.
  Ambrose Bierce tiene una definición en su \emph{Devil's Dictionary}: `La
  actitud de la mente de uno que cree sin evidencia a uno que habla sin
  conocimiento cosas sin parangón'. ¿Qué deberíamos pensar de
  esto?\footnote{\cite[115]{anscombe2008faith:faith}: <<I want to say what might
    be understood about faith by someone who did not have it; someone, even, who
    does not necessarily believe that God exists, but who is able to think
    carefully and truthfully about it. Bertrand Russell called faith `certainty
    without proof'. That seems correct. Ambrose Bierce has a definition in his
    Devil's Dictionary: `The attitude of mind of one who believes without
    evidence one who tells without knowledge things without parallel.' What
    should we think of this?>>}}

\subsection{Descripción de `Fe'}
El objetivo de Elizabeth sitúa su investigación en un contexto específico.
Pretende describir el fenómeno de la fe como uno que tiene un carácter de
razonabilidad tal que se puede \citalitinterlin{pensar cuidadosa y
  honstamente}\footnote{\cite[115]{anscombe2008faith:faith}:<<think carefully
  and truthfully>>}. Su estrategia, carácterística del tipo de análisis empleado
por Wittgenstein, se muestra aquí de nuevo como una descripción de usos
familiares de la palabra siendo analizada que son articulados de tal manera que
los patrones de estos usos sean
estudiables\autocite[Cf.~][12]{bakerhacker2009understanding}. Se enfoca en un
modo antiguo de usar la palabra `fe' en el que se le empleaba para decir `creer
a alguien que $p$'. `Fe humana' era creer a una persona humana, `fe divina' era
creer a Dios\autocite[Cf.~][2]{anscombe2008faith:tobelieve}. Así por ejemplo:
<<Abrám creyó a Dios (\textgreek{ἐπίστευσεν τῷ Θεῷ}) y ésto se le contó como
justicia>>\footnote{Gn~15,6}. De tal modo que es llamado 'padre de la
fe'.\footnote{Cf.~Rm~4~y~Ga 3,7}. Este enfoque hace que la pregunta `¿qué es
creer a alguien?' quede situada en el centro del
análisis\autocite[Cf.~][14]{anscombe2008faith:faith}, y aquí también Anscombe
dedica su atención a las presuposiciones implicadas en esta creencia.

Pueden ser destacados tres movimientos principales en el análisis realizado por
Elizabeth en esta investigación. Primero establece una relación entre las
presuposiciones implicadas en el creer y lo que se ha llamado los preámbulos de
la fe. En segundo lugar describe lo relacionado a las presuposiciones implicadas
en creer a una persona humana cuando ésta comunica algo. En tercer lugar examina
el fenómeno particular de creer cuando la comunicación viene de Dios.

\subsubsection{Las presuposiciones del creer como descripción de la
  razonabilidad de la fe}
A lo largo de la investigación, Anscombe recurrirá a una descripción de las
presuposiciones implicadas en el creer como una descripción razonable de la fe.
Su apoyo para seguir esta ruta de análisis es el recuerdo de cierta discusión,
de cierta apologética\autocite[Cf.~][13]{anscombe2008faith:faith}. Trae a la
memoria que: \citalitlar{Hubo en una época pasada un profuso entusiasmo por la
  racionalidad, quizás inspirado por la enseñanza del Vaticano~I contra el
  fideísmo, ciertamente sostenidos por la promoción de estudios neo-tomístas
  [\ldots] la noticia era que la fe Cristiana Católica era \emph{racional}, y el
  problema, para aquellos capaces de sentirlo como tal, era cómo era
  \emph{gratuita} \footnote{\cite[11]{anscombe2008faith:faith}: <<There was in a
    preceding time a professed enthusiasm for rationality, perhaps inspired by
    the teaching of Vatican I against fideism, certainly carried along by the
    promotion of neo-thomist studies [\ldots] the word was that the Catholic
    Christian faith was \emph{rational}, and a problem, to those able to feel it
    as a problem, was how it was \emph{gratuitous}>>}.}

Distintas variantes de esta enseñanza ---destaca Anscombe--- ofrecían distintas
argumentaciones, algunas más sobrias que otras, que servían como procesos de
razonamientos que ofrecían una cierta demostración de la verdad de las
enseñanzas de la Iglesia\autocite[Cf.~][12]{anscombe2008faith:faith}.

\subsubsection{Las presuposiciones implicadas en creer a alguien}

\subsubsection{`Creer' cuando la comunicación viene de Dios}

\subsection{¿Qué es creer a alguien?}

\subsubsection{Naturaleza de la Investigación}
Es útil recordar aquí en términos generales el modo en el que Anscombe actua en
una investigación filosófica. Wittgenstein inicialmente describió el análisis
del lenguaje bajo la concepción de que la lógica conforma el orden que está
debajo y que sostiene todo lenguaje posible. El trabajo del filósofo es analizar
el lenguaje para sacar al descubierto el orden lógico que está debajo del
lenguaje ordinario y que es la forma de la realidad. Wittgenstein abandonó esta
concepción; en Investigaciones Filosóficas exclama: \citalitlar{Cuanto más de
  cerca examinamos el lenguaje actual, más crece el conflicto entre éste y
  nuestro requisito. (Pues la pureza cristalina de la lógica no era, por
  supuesto, algo que yo hubiera \emph{descubierto}: era un requisito.) El
  conflicto se hace intolerable; el requisito llega ahora a estar en peligro de
  tornarse vacuo. --- Nos hemos situado en hielo resbaladizo donde no hay
  fricción, y así, en cierto sentido, las condiciones son ideales; pero también,
  justo por eso, no somos capaces de caminar. Queremos caminar: así que
  necesitamos \emph{fricción}. ¡De vuelta al terreno
  escarpado!\footnote{\cite[\S107]{wittgenstein1953phiinv}: <<The more closely
    we examine actual language, the greater becomes the conflict between it and
    our requirement. (For the crystalline purity of logic was, of course, not
    something I had \emph{discovered}: it was a requirement.) The conflict
    becomes intolerable; the requirement is in danger of becoming vacuous. ---
    We have got on to slippery ice where there is no friction, and so, in a
    certain sense, the conditions are ideal; but also, just because of that, we
    are unable to walk. We want to walk: so we need \emph{friction}. Back to the
    rough ground!>>}.}

Los nombres, las proposiciones, el lenguaje, no tienen una forma esencial para
ser puesta al descubierto por el análisis, sino que son familias de estructuras
que están a plena vista y que pueden ser clarificadas por medio de la
descripción\autocite[Cf.~][12]{bakerhacker2009understanding}. Wittgenstein le
\citalitinterlin{da la vuelta a la
  busqueda}\autocite[\S108]{wittgenstein1953phiinv}, y trata a la lógica no como
lo que está debajo del lenguaje para ser descubierto, sino como
\citalitinterlin{una cuadrícula que imponemos sobre los argumentos para probar y
  demostrar su validez}\footnote{\cite[12]{bakerhacker2009understanding}: <<a
  grid we impose upon arguments to test and demonstrate their validity>>}.

Descartada la concepción sublime de la tarea filosófica, Wittgenstein describe
los problemas filosóficos como formas de malentendidos o falta de entendimiento
que pueden ser disueltos por medio de descripciones de los usos de las palabras.
La tarea de la filosofía es la \citalitinterlin{clarificación gramatical que
  disuelve la perplejidad conceptual y ofrece una visión amplia o representación
  estudiable de un segmento de la gramática de nuestro
  lenguaje}\footnote{\cite[12]{bakerhacker2009understanding}: <<grammatical
  clarification that dissolves conceptual puzzlement and gives an overview of or
  surveyable representation of a segment of the grammar of our language>>}. Esta
metodología, por tanto, no pretende ofrecer teorías explicativas fruto de la
deducción o la hipótesis; tampoco pretende ofrecer tesis dogmáticas o
esencialistas. Más bien busca describir usos familiares de las palabras y
ordenarlas de tal manera que los patrones de su uso sean
estudiables\autocite[Cf.~][12]{bakerhacker2009understanding}. La metodología de
Elizabeth está basada en esto.

\subsubsection{Investigación Gramática de `creer a $x$ que $p$'.}
Anscombe pone el interés de su investigación en la forma de la expresión `creer
a $x$ que $p$'\autocite[Cf.~][2]{anscombe2008faith:tobelieve}. Su análisis se va
desenvolviendo a lo largo de la descripción de los usos de la expresión.

\citalitinterlin{Si me dijeras `Napoleón perdió la batalla de Waterloo' y te
  digo `te creo' sería una
  broma}\footnote{\cite[4]{anscombe2008faith:tobelieve}: <<If you tell me
  `Napoleon lost the battle of Waterloo' and I say `I believe you' that is a
  joke.>>}. A primer golpe `creer a $x$ que $p$' parece que significa
simplemente creer lo que alguien me dice, o creer que lo que me dice es
verdadero. Sin embargo esto no es suficiente. Puede ser que ya crea lo que
alguien me venga a decir. Puede ser que la comunicación suscite que forme mi
propio juicio acerca de la verdad comunicada, pero aquí no podría decir que
estoy creyendo al que comunica o que estoy contando con él para mi creer que
$p$.

¿Entonces creer a alguien es creer algo apoyado en el hecho de que lo ha dicho?
\citalitinterlin{Puede que se le pregunte a un testigo `¿Por qué pensó que aquel
  hombre se estaba muriendo?' y que éste responda `Porque el doctor me lo dijo'
  [\ldots] `no me hice ninguna opinión propia --- yo sólo creí al
  doctor'}\footnote{\cite[4]{anscombe2008faith:tobelieve}: <<A witness might be
  asked `Why did you think the man was dying?' and reply `Because the doctor
  told me'. If asked further what his own judgement was, he may reply `I had no
  opinion of my own --- I just believed the doctor'.>>}. Éste puede ser un
ejemplo de contar con $x$ para la verdad de $p$. Esto, sin embargo, tampoco
parece ser suficiente. Puedo imaginar el caso en el que esté convencido de que
alguien a la vez cree lo opuesto a la verdad de $p$ y quiera mentirme. Según
este cálculo podría decir que creo en lo que ha dicho por el hecho de que me lo
ha dicho, pero no estaría diciendo que le creo a él.

¿Qué se puede decir del <<les creo a todos>> de Eutidemo en la cuestión
preliminar? Anscombe juzga que la exclamación no expresa simplemente una opinión
apresurada o excesiva credulidad, sino más bien suena a
locura\autocite[5]{anscombe2008faith:tobelieve}. Eutidemo no puede estar
diciendo la verdad cuando dice que les cree a todos. La expresión de $C$ da
pertinencia a lo que dice $B$, y la manera natural de entender lo que dice $B$
es como arrojando duda sobre lo que $A$ ha dicho. ¿Se puede pensar que $A$
todavía cree lo que ha dicho inicialmente? ¿Eutidemo puede creer a $A$ sin saber
cuál es su reacción a lo que $B$ y $C$ han dicho? Entonces Anscombe concluye,
\citalitinterlin{Para creer a $N$ uno debe creer que $N$ mismo cree lo que está
  diciendo}\footnote{\cite[5]{anscombe2008faith:tobelieve}: <<To believe $N$ one
  must believe that $N$ himself believes what he is saying>>.} Creer a $N$ sin
saber si $N$ cree lo que dice le suena a Elizabeth como una locura.

En este punto queda expuesta a la luz una segunda creencia involucrada en el
creer a $x$ que $p$. Anscombe fija su atención en esto. Creer a $x$ que $p$
conlleva otras creencias, éstas son presuposiciones implicadas en llegar a
plantearse si creer o no. En primer lugar, si se cree a alguien, tiene que ser
el caso que se cree que una comunicación es de
alguien\autocite[Cf.~][6]{anscombe2008faith:tobelieve}. Esta presuposición no
parece tan problemática si se piensa en las ocasiones en las que creemos a
alguien que es percibido. Sin embargo tiene más profundidad si se considera que
con frecuencia recibimos la comunicación sin que esté presente el que habla,
como cuando leemos un libro\autocite[Cf.~][5]{anscombe2008faith:tobelieve}.

Se puede imaginar aquí una situación problemática. Supongamos que alguien recibe
una carta en la que el autor no es el comunicador ostensible o aparente, es
decir, quien firma la carta no es quien la ha escrito. ¿Se puede decir que el
que recibe la carta cree o descree al autor o al comunicador ostensible? Creer
al autor, afirma Anscombe, conlleva un tipo de juicio y especulación que no son
mediaciones ordinarias en el creer a
alguien\autocite[Cf.~][7]{anscombe2008faith:tobelieve}. Para decir que creo al
autor tendría que discernir que la comunicación que viene bajo otro nombre es
realmente de esta otra persona que además me quiere decir esto.

Respecto de la posibilidad de decir que se cree al comunicador ostensible
Anscombe distingue entre un comunicador ostensible que exista o no. Ante una
comunicación que viene de parte de un comunicador aparente que no existe,
alguien puede responder diciendo que cree o descree al comunicador aparente,
pero la decisión de decir esto ---dice Anscombe--- \citalitinterlin{es una
  decisión de dar a estos verbos un uso `intencional', como el verbo `ir
  tras'}\footnote{\cite[7]{anscombe2008faith:tobelieve}: <<is a decision to give
  those verbs an `intentional' use like the verb `to look for'>> Ver:
  \cite{anscombe1981metaphysics:intsens}. Anscombe propone que un verbo es usado
  intencionalmente cuando tiene como objeto directo un `objeto intencional'
  (`objeto' no en el sentido material, sino de finalidad).}. Esto lo ilustra
añadiendo: \citalitlar{Y así uno podría hablar de alguien como creyendo al dios
  (Apolo, digamos), cuando consultó el oráculo del dios -- sin que por esto uno
  estuviera implicando que uno mismo cree en la existencia del dios. Todo lo que
  queremos es que necesitamos saber lo que es llamado que el dios le diga
  algo\footnote{\cite[7]{anscombe2008faith:tobelieve}: <<And so we might speak
    of someone as believing the god (Apollo, say), when he consulted the oracle
    of the god -- without thereby implying that one believed in the existence of
    the gos oneself. All we want is that we should know what is called the god's
    telling him something>>}.} `Creer' usado aquí intencionalmente viene a decir
que se busca o se desea creer a $x$ (Apolo en este caso) cuando se recibe
aquello que alguien entiende como una comunicación suya.

En el caso de que el comunicador ostensivo sí exista, la noción de creerle
manifiesta una cierta oscilación. Una tercera persona podría decir que `aquel,
pensando que $N$ dijo esto, le creyó', o el comunicador aparente puede decir
`veo que pensaste que fui yo quien dijo esto y me creiste', sin embargo, si el
que ha recibido la comunicación dijera `naturalmente te creí', el comunicador
aparente podría contestar `ya que no lo he dicho yo, no me estabas creyendo a
mi'\autocite[Cf.~][8]{anscombe2008faith:tobelieve}.

Estas consideraciones llevan a Anscombe a distinguir entre el que habla en una
comunicación y el productor inmediato de la
comunicación\autocite[Cf.~][8]{anscombe2008faith:tobelieve}. Éste puede ser
cualquiera que pase hacia adelante alguna comunicación, un maestro o mensajero,
o un interprete o traductor; éste es \citalitinterlin{el productor inmediato de
  aquello que se entiende, o incluye una reclamación interna de ser entendido
  como una comunicación de $NN$}\footnote{\cite[8]{anscombe2008faith:tobelieve}:
  <<we can speak of the immediate producer of what is taken, or makes an
  internal claim to be taken, as a communication from $NN$>>}. Si digo que creo
a un intérprete estoy afirmando que creo lo que ha dicho su principal, y mi
contar con el intérprete consiste en la creencia de que ha reproducido lo que
aquel ha dicho. En este sentido el intérprete no le falta rectitud si dice algo
que no es verdadero pero no ha representado falsamente lo que ha dicho su
principal. Por el contrario, al maestro sí le faltaría rectitud si lo que dice
no es verdadero. Cuando se cree al maestro, aún en el caso que no sea de ninguna
manera autoridad original de lo que comunica, se le cree a él sobre lo que
transmite. Para Anscombe no es necesario que cuando se cree a alguien se le
trate como una autoridad
original\autocite[Cf.~][5]{anscombe2008faith:tobelieve}. En esto el ejemplo del
maestro como distinto del intérprete es ilustrativo. Un maestro puede conocer lo
que enseña porque lo ha recibido de alguna tradición de información y al
transmitir lo que enseña se le está creyendo a él.

Asoma aquí otro aspecto relacionado con esta presuposición. Al creer que una
comunicación es de alguien se cree a una persona que puede tener distintos
grados de autoridad sobre lo que dice. El maestro del que se ha hablado antes
podría afirmar <<Leonardo da Vinci dibujó diseños para una máquina voladora>> y
en esto no es para nada una autoridad
original\autocite[Cf.~][6]{anscombe2008faith:tobelieve}. Conoce esto porque lo
ha escuchado, incluso si ha visto los diseños. Aún cuando los hubiera
descubierto él mismo, tendría que haber contado con alguna información recibida
de que esos diseños que ve son de Leonardo. En este caso sí seria una autoridad
original en notar que estos diseños que ha escuchado que son de Leonardo son de
máquinas voladoras. Anscombe explica la distinción diciendo:
\citalitlar{[Alguien] es \emph{una} autoridad original en aquello que él mismo
  ha hecho y visto y oido: digo \emph{una} autoridad original porque sólo quiero
  decir que él mismo sí contribuye algo, es algún tipo de testigo por ejemplo,
  en lugar de alguien que sólo transmite información recibida. Pero su informe
  de aquello de lo que es testigo es con frecuencia [\ldots] fuertemente
  influenciado o más bien casi del todo formado por la información que \emph{él}
  ha recibido\footnote{\cite[5]{anscombe2008faith:tobelieve}: <<He is \emph{an}
    original authority on what he himself has done and seen and heard: I say
    \emph{an} original authority because I only mean that he does himself
    contribute something, e.g. is in some sort a witness, as oposed to one who
    only transmits information received. But his account of what he is a witness
    to is very often [\ldots] heavily affected or ratherl all but completely
    formed by what information \emph{he} had received.>>}.} Además de ser
\emph{una} autoridad original sobre algún hecho, una persona puede ser una
autoridad \emph{totalmente} original. Si la distinción entre alguien que no es
una autoridad original y alguien que sí lo es ha sido descrita como la
contribución de algo propio que junto con la información recibida permite
construir un informe, lo particular de una autoridad totalmente original es que
no se apoya en ninguna información recibida para construir su informe de los
hechos. Anscombe no entiende el lenguaje como información recibida. Pone como
ejemplo de informe de una autoridad totalmente original a alguien que dice `esta
mañana comí una manzana' y dice: \citalitlar{si él está en la situación usual
  entre nosotros, sabe lo que una manzana es --- es decir, puede reconocer
  una. Así que aún cuando se le ha `enseñado el concepto' al aprender a usar el
  lenguaje en la vida ordinaria, no cuento esto como un caso de depender en
  información recibida.\footnote{\cite[6]{anscombe2008faith:tobelieve}: <<if he
    is in the situation usual among us, he knows what an apple is --- i.e. can
    recognise one. So though he was `taught the concept' in learning to use
    language in everyday life, I do not count that as a case of reliance on
    information received.>>}}

Hasta aquí se ha visto que el creer a $x$ que $p$ implica otras creencias que
son presuposiciones a la pregunta sobre si se cree o se descree a alguien y se
ha descrito lo que tiene que ver con la creencia de que una comunicación viene
de alguien. Anscombe examina otras presuposiciones más. También tiene que ser el
caso que creamos que por la comunicación, la persona que habla quiere decir
\emph{esto}. En situaciones ordinarias no es difícil distinguir si alguien está
diciendo o escribiendo algún lenguaje. Sin embargo, aún cuando el que habla use
palabras que puedo `hacer mías' y creer simplemente las palabras que dice, aquí
queda espacio para decir que hay una creencia adicional de que se ha dicho `tal
cosa' en la comunicación. Elaboramos en aquello que hemos creido y usamos otras
palabras distintas, nuestras creencias no están atadas a palabras específicas.
También podríamos pensar que alguien diga que cree \emph{esto} porque cree a $x$
y que se le cuestione su creencia preguntando `¿qué tomaste como $x$ dicicéndote
eso?'\autocite[Cf.~][8]{anscombe2008faith:tobelieve}.

Otra presuposición más sería que se cree que la comunicación está
\emph{dirigida} a alguien, aunque sea `a quien lea esto' o `a quien pueda
interesar'. Esta creencia se podría problematizar pensando en algún caso que
alguien reciba una comunicación con otro destinatario, ¿estaría creyendo al que
se comunica?. Asncombe opina que en un sentido extendido o reducido y considera
que el tema parece de poca
importancia\autocite[Cf.~][7]{anscombe2008faith:tobelieve}.

Una persona a quien se dirige una comunicación puede \emph{fallar en creerla} si
no nota la comunicación, o si notándola no la interpreta como lenguaje, o si
notándola como lenguaje no la toma como dirigida hacia ella; o puede que crea
todo esto, pero lo interprete incorrectamente, o puede que lo interprete bien
pero no crea que viene realmente de $N$. En este tipo de casos la persona no ha
descreido, sino que no ha llegado a estar en la situación de plantearse esa
pregunta. Para poder llegar a preguntar si alguien cree a $x$ que $p$ habría que
excluir o asumir como excluidos todos los casos en los que estas otras
presuposiciones no se han cumplido. Es así que Anscombe concluye:
\citalitlar{Supongamos que todas la presuposiciones están dadas. $A$ está
  entonces en la situación ---una muy común--- donde surge la pregunta sobre si
  creer o dudar (suspender el juicio ante) $NN$. Sin confusión por todas las
  preguntas que surgen por las presuposiciones, podemos ver que creer a alguien
  (en el caso particular) es confiar en él para la verdad -- en el caso
  particular. \footnote{\cite[9]{anscombe2008faith:tobelieve}: <<Let us suppose
    that all the presuppositions are in. $A$ is then in the situation ---a very
    normal one--- where the question arises of believing or doubting (suspending
    judgement in face of) $NN$. Unconfused by all the questions that arise
    because of the presuppositions, we can see that believing someone (in the
    particular case) is trusting him for the truth -- in the particular
    case.>>}.}
Que $A$ crea a $N$ que $p$ implica que $A$ cree que en una comunicación, que puede
venir de un productor inmediato, $N$ es el que habla y lo que dice es $p$ y esta
comunicación está dirigida hacia $A$; entonces $A$, creyendo que $N$ cree que
$p$, confia en $N$ sobre la verdad de $p$.

\subsubsection{Acceso al mundo más allá de la experiencia}
¿Cómo accedemos a una idea del mundo más allá de nuestra experiencia personal?
Una de las cuestiones que Anscombe plantea como preámbulo a su análisis sobre el
`creer a alguien' tiene que ver con esta pregunta.

Hume diría que el puente que permite nuestro contacto con la realidad más allá
de nuestra experiencia es la relación
causa-y-efecto\autocite[Cf.~][3]{anscombe2008faith:tobelieve}. Inferimos las
causas desde sus efectos porque estamos acostumbrados a ver que causa y efecto
van juntas. Estas causas inferidas las verificamos en la percepción inmediata de
nuestra memoria o nuestros sentidos, o por medio de la inferencia de otras
causas verificadas del mismo
modo\autocite[Cf.~][88]{anscombe1981parmenides:humeandjulius}. Hume entonces
propone que la relación entre el testimonio y la verdad es de la misma clase,
inferimos la verdad del testimonio porque estamos acostumbrados a que vayan
juntas\autocite[Cf.~][3]{anscombe2008faith:tobelieve}.

Anscombe tacha de absurda esta visión del rol del testimonio en el conocimiento
humano\autocite[Cf.~][3]{anscombe2008faith:tobelieve} y le parece que
\citalitinterlin{el misterio es cómo Hume la pudo haber llegado a
  sostener}\footnote{\cite[Cf.~][3]{anscombe2008faith:tobelieve}: <<the mystery
  is how Hume could ever have entertained it.>>}. Entonces explica:
\citalitlar{Hemos de reconocer al testimonio como el que nos da nuestro mundo
  más grande en no menor grado, o incluso en un grado mayor, que la relación de
  causa y efecto; y creerlo es bastante distinto en estructura que el creer en
  causas y efectos. Tampoco es lo que el testimonio nos da una parte
  completamente desprendible, como el borde de grasa en un pedazo de filete. Es
  más bien como las manchas y rayas de grasa que están distribuidas a través
  de la buena carne; aunque hay nudos de pura grasa
  también\footnote{\cite[3]{anscombe2008faith:tobelieve}:<<We must acknowledge
    testimony as giving us our larger world in no smaller degree, or even in a
    greater degree, than the relation of cause and effect; and believing it is
    quite dissimilar in structure from belief in causes and effects. Nor is what
    testimony gives us entirely a detachable part, like the thick fringe of fat
    on a chunk of steak. It is more like the flecks and streaks of fat that are
    distributed through good meat; though there are lumps of pure fat as
    well>>}.} Elizabeth considera que la mayor parte de nuestro conocimiento de
la realidad está apoyado en la creencia que tenemos en las cosas que se nos han
enseñado o dicho\autocite[Cf.~][3]{anscombe2008faith:tobelieve}. Para ella, la
investigación acerca de `creer a alguien' no sólo es del interés de la teología
o de la filosofía de la religión, sino de enorme importancia para la teoría del
conocimiento\autocite[Cf.~][3]{anscombe2008faith:tobelieve}.

La ruta de análisis que Anscombe abre con esta propuesta consiste en una
descripción más adecuada de la `estructura del creer en el testimonio' como una
estructura distinta de la relación causa y efecto. Aquí la descripción vista
sobre el `creer a algiuen' ha ofrecido ya pistas valiosas. Sin embargo, Anscombe
aborda el tema en otras discusiones y es necesario tenerlas en cuenta para hacer
una descripción más completa.

\subsubsection{`Creer a alguien' como `Fe'}
Una segunda cuestión que aparece como preámbulo en la investigación de Anscombe
es planteada así: \citalitinterlin{Si las palabras siempre guardaran sus
  antiguos valores, podría haber llamado mi tema `Fe'. Este corto término ha
  sido usado en el pasado justo con este significado, el de creer a
  alguien}\footnote{\cite{anscombe2008faith:tobelieve}: <<If words always kept
  their old values, I might have called my subject `Faith'. That short term has
  in the past been used in just this meaning, of believing someone.>>}. Este uso
de la expresión sera útil para Anscombe en su análisis del uso de la palabra
`fe'. Su descripción estará enfocada en `fe' como `creer a Dios que $p$'. Esta
segunda ruta será explorada más adelante.

\subsubsection{Creer a quien habla rectamente}
Al final de la investigación, Anscombe propone una cuestión que se queda
abierta. Tiene que ver con uno de los ejemplos relacionados a creer que la
comunicación viene de alguien. Allí proponia imaginar el caso en el que
estuvieramos convencidos de que alguien viene a decirnos lo que cree que es
falso, pero a la misma vez sabemos que lo que cree es lo contrario a la verdad.
Al decir lo que cree que es falso estaría afirmando la verdad. En ese caso,
afirmaba Anscombe, podría decir que creo en lo que dice y además creo porque lo
dice, pero no le creo a él. Se podría preguntar ¿cuál es la diferencia entre
llegar a la creencia de $p$ porque alguien que está en lo correcto y es veraz me
lo ha dicho, y llegar a la misma creencia porque me lo ha dicho alguien que está
equivocado y miente? Ambos casos parecen implicar un cálculo, en uno se calcula
que está en lo correcto y es veraz y en el otro se calcula que está equivocado y
miente. ¿Por qué estamos dispuestos a decir que creemos al que habla sólo en el
caso en que esté en lo correcto y sea veraz? ¿Acaso no llevan ambos casos a la
misma creencia que $p$?

Aquí Anscombe percibe que hay más que decir sobre la prioridad que damos a la
rectitud y la veracidad en el creer lo que se nos dice sobre la realidad. De las
tres rutas descritas, recorreremos ésta primero.
