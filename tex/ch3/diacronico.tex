\section{Análisis Diacrónico}

\subsection{Hume and Julius Caesar}

Los artículos \emph{Hume and Julius Caesar} y \emph{``Whatever has a beginning
  of existence must have a cause'': Hume’s Argument Exposed} de Anscombe, fueron
publicados en la revista académica \emph{Analysis} en octubre de 1973 y abril de
1974 respectivamente. Ambos están relacionados por el tema de la causalidad en
Hume. En el trasfondo de los dos artículos está otro documento no publicado
hasta 2011 con el título \emph{Hume on causality: introductory}.

Una de las actitudes características de Anscombe es su tendencia a quedar
atraída por preguntas que representan cuestiones profundas, incluso en
discusiones cuyos argumentos, método o conclusiones no le parecen tan
interesantes.

Un autor que suele tener este efecto en ella es Hume. En \emph{Modern Moral
  Philosophy} dice:

\blockquote[{\cite[172]{anscombe1981mmph}}: The features of Hume’s philosophy
which I have mentioned, like many other features of it, would incline me to
think that Hume was a mere ---brilliant--— sophist; and his procedures are
certainly sophistical. But I am forced, not to reverse, but to add to this
judgement by a peculiarity of Hume’s philosophizing: namely that, although he
reaches his conclusions --—with which he is in love--— by sophistical methods,
his considerations constantly open up very deep and important problems. It is
often the case that in the act of exhibiting the sophistry one finds oneself
noticing matters which deserve a lot of exploring: the obvious stands in need of
investigation as a result of the points that Hume pretends to have made.]{Las
  características de la filosofía de Hume que he mencionado, como muchas otras
  de sus características, me hacen inclinarme a pensar que Hume era un simple
  ---brillante--- sofista; y sus procedimientos son ciertamente sofísticos. Sin
  embargo me veo forzada, no a retractarme, sino a añadir a este juicio por la
  peculiaridad del filosofar de Hume: a saber, que aunque llega a sus
  conclusiones ---con las que está enamorado--- por métodos sofísticos, sus
  consideraciones constantemente abren problemas bien profundos e importantes.
  Frecuentemente es el caso que en el acto de exhibir la sofística uno se
  encuentra a sí mismo notando temas que merecen mucha exploración: lo obvio
  queda necesitado de investigación como resultado de los puntos que Hume
  pretende haber hecho.}

En el artículo \emph{Hume and Julius Caesar} la discusión que capta el interés
de Anscombe se encuentra en la sección IV de la tercera parte del \emph{Treatise
  of Human Nature} sobre el tema de la justificación de nuestro creer en
cuestiones que están más allá de nuestra experiencia y memoria. Anscombe cita el
texto de Hume como sigue:

\blockquote[{\cite[86]{anscombe1981hjc}}When we infer effects from causes, we
must establish the existence of these causes\ldots either by an immediate
perception of our memory or senses, or by an inference from other causes; which
causes we must ascertain in the same manner either by a present impression, or
by an inference from their causes and so on, until we arrive at some object
which we see or remember. 'Tis impossible for us to carry on our inferences
\emph{in infinitum}, and the only thing that can stop them, is an impression of
the memory or senses, beyond which there is no room for doubt or enquiry.
(Selby-Bigge's edition, pp. 82--3)]{Cuando inferimos efectos partiendo de causas
  debemos establecer la existencia de estas causas\ldots ya sea por la
  percepción inmediata de nuestra memoria o sentidos, o por la inferencia
  partiendo de otras causas; causas que debemos explicar de la misma manera por
  una impresión presente, o por una inferencia partiendo de sus causas, y así
  sucesivamente hasta que lleguemos a un objeto que vemos o recordamos. Es
  imposible para nosotros proseguir en nuestras inferencias al infinito, y lo
  único que puede detenerlas es una impresión de la memoria o los sentidos más
  allá de la cual no existe espacio para la duda o indagación.}

Ya en la sección II de esta misma parte del \emph{Treatise}, Hume ha planteado
cómo es la causalidad la conexión que nos asegura la existencia o acción de un
objeto que es seguido o precedido por la existencia o acción de
otro.\footnote{Cf. Treatise Sección II Parte III: ’Tis only causation, which
  produces such a connexion, as to give us assurance from the existence or
  action of one object, that ’twas follow’d or preceded by any other existence
  or action; nor can the other two relations be ever made use of in reasoning,
  except so far as they either affect or are affected by it. }
Ahora en la sección IV esta relación de causa y efecto será tomada como un
principio de asociación de ideas según el cual es posible inferir desde la
impresión de alguna cosa, una idea sobre otra cosa.

Desde esta noción de causalidad se explica la posibilidad de acceder a hechos
más allá de nuestra experiencia; estos son inferencias de efectos desde sus
causas. De este modo: \blockquote[{\cite[87]{anscombe1981hjc}}: For Hume, the
relation of cause and effect is the one bridge by which to reach belief in
matters beyond our present impressions or memories.]{Para Hume, la relación de
  causa y efecto es el único puente por el que se puede alcanzar creer en
  cuestiones más allá de nuestras impresiones presentes o memorias.}

El paso adicional que Hume propone en esta sección es que al realizar estas
inferencias es necesario establecer la existencia de las causas por medio de la
percepción inmediata de los sentidos o por medio de una ulterior inferencia. Sin
embargo, el establecimiento de la existencia de estas causas por medio de
inferencias no puede continuar infinitamente, sino que tiene que llegar a una
impresión de la memoria o los sentidos que sirva de justificación o fundamento
definitivo.

Para ilustrar este paso, Hume hace una invitación interesante:
  \blockquote[{\cite[?]{humetreatise}}: chuse any point of history, and consider
  for what reason we either believe or reject it.]{elegir cualquier punto en la
    historia, y considerar por qué razón lo creemos o rechazamos.} Acerca de una
  creencia histórica se nos invita a considerar sobre qué se sostiene su
  justificación. ¿Cuál es su fundamento?:
  \blockquote[{\cite[?]{humetratise}}: Thus we believe that Cæsar was kill’d in
  the senate-house on the ides of March; and that because this fact is establish’d
  on the unanimous testimony of historians, who agree to assign this precise time
  and place to that event. Here are certain characters and letters present either
  to our memory or senses; which characters we likewise remember to have been us’d
  as the signs of certain ideas; and these ideas were either in the minds of such
  as were immediately present at that action, and receiv’d the ideas directly from
  its existence; or they were deriv’d from the testimony of others, and that again
  from another testimony, by a visible gradation, ’till we arrive at those who
  were eye-witnesses and spectators of the event. ’Tis obvious all this chain of
  argument or connexion of causes and effects, is at first founded on those
  characters or letters, which are seen or remember’d, and that without the
  authority either of the memory or senses our whole reasoning wou’d be chimerical
  and without foundation.]{Así, creemos que César fue asesinado en el Senado en
    los idus de Marzo; y esto porque el hecho está establecido basándose en el
    testimonio unánime de los historiadores, que concuerdan en asignar a este
    evento este tiempo y lugar precisos. Aquí ciertos caracteres y letras se
    hallan presentes a nuestra memoria o sentidos; caracteres que recordamos
    igualmente que han sido usados como signos de ciertas ideas; y estas ideas
    estuvieron ya en las mentes de los que se hallaron inmediatamente presentes a
    esta acción y que obtuvieron las ideas directamente de su existencia; o fueron
    derivadas del testimonio de otros, y éstas a su vez de otro testimonio, por
    una graduación visible, hasta llegar a los que fueron testigos oculares y
    espectadores del suceso. Es manifiesto que toda esta cadena de argumentos o
    conexión de causas y efectos se halla fundada en un principio en los
    caracteres o letras que son vistos o recordados y que sin la autoridad de la
    memoria o los sentidos nuestro razonamiento entero sería quimérico o carecería
    de fundamento.}

Anscombe comienza por reaccionar afirmando:
\blockquote[{\cite[86]{anscombe1981hjc}}: This is not to infer effects from
causes, but rather causes from effects.]{Esto no es inferir efectos partiendo de
  sus causas, sino más bien causas desde los efectos.} Es decir, el ejemplo
histórico de Hume consiste en una inferencia de la causa original, el asesinato
de Julio César, desde su efecto remoto que es nuestra percepción en el presente.
Creemos en el asesinato de César porque lo inferimos como la causa última en una
cadena causal que llega hasta nuestra percepción de ciertas oraciones que
leemos. El hecho de que estemos leyendo esta información es la percepción que
justifica la creencia de que hay una cadena de causas y efectos que tiene como
efecto esta experiencia. Esta inferencia pasa a través de una cadena de efectos
de causas, que son efectos de causas\ldots ¿Dónde empieza la cadena? La
respuesta parece ser nuestra percepción presente. ¿Cómo hemos de entender,
entonces, el argumento de que la cadena no puede continuar infinitamente? La
propuesta de Hume es que la cadena ha de terminar en una impresión que no deje
lugar a dudas o busqueda mas allá, sin embargo, la cadena termina en el
asesinato de Julio César, no en nuestra percepción. La imagen que Hume pretende
ofrecer es la de una cadena fijada en sus dos extremos por algo distinto a los
eslabones que la componen, sin embargo, no lo logra, más bien parece describir
un voladizo, una estructura apoyada en un punto, pero sin apoyo en el otro
extremo.

La afirmación \blockquote['Tis impossible for us to carry on our inference in
infinitum]{Es imposible para nosotros proseguir en nuestras inferencias al
  infinito} viene a significar, según la interpretación de Anscombe, que
\blockquote[the justification of the grounds of our inferences cannot go on in
infinitum]{la justificación de los fundamentos de nuestras inferencias no pueden
  continuar al infinito}. El argumento aquí mas bien es que tiene que haber un
punto de partida para la inferencia de la causa original. La relación de
inferencias propuesta por Hume en su ilustración acabaría siendo una inferencia
hipotética según su propia definición. Anscombe explica diciendo:

\blockquote[hume in causality: We must suppose ourselves to start with the
familiar idea, merely as idea, of Caesar having been killed. Now if we ask why
we believe it we shall, as Hume does, point to historical testimony (the
‘characters and letters’), which doesn’t at this point figure as what stops
inference going on ad infinitum. However, if we want to explain the connection
we shall form the idea of Caesar’s death being recorded by eyewitnesses; and
these records having been received by others, who transmitted an account ...
etc. Here we really are arguing from the idea of an original cause to the idea
of an effect; we are ‘inferring effects from causes’, though only in the sense
of passing from the idea of the cause to the idea of the effect.]{Tendríamos que
  suponer que comenzamos con la idea familiar, meramente como una idea, de que
  César fue asesinado. Ahora si preguntamos por qué lo creemos hemos de, como
  hace Hume, señalar al testimonio histórico (los `caracteres y letras'), lo
  cual en este punto no figura como lo que detiene que la inferencia siga al
  infinito. Sin embargo, si queremos explicar la conexión tenemos que formular
  la idea de la muerte del Cesar siendo recordada por testigos; y esos recuentos
  siendo recibidos por otros, quienes transmitieron un informe\ldots etc. Aquí
  estamos realmente razonando desde la idea de una causa original a la idea de
  un efecto; estamos `infiriendo efectos de causas', pero solo en el sentido de
  pasar de la idea de la causa a la idea del efecto.}

Desde este análisis, Anscombe resume lo argumentado por Hume en cuatro partes:

\blockquote[humeandjulius 88: First, a chain of reasons for a belief must
terminate in something that is believed without being founded on anything else.
Second, the ultimate belief must be of a quite different character from derived
beliefs: it must be perceptual belief, belief in something perceived, or
presently remembered. Third, the immediate justification for a belief p, if the
belief is not a perception, will be another belief q, which follows from, just
as much as it implies, p. Fourth, we believe by inference through the links in a
chain of record

There is an implicit corollary: when we believe in historical information
belonging to the remote past, we believe that there has been a chain of record]{
  Primero, una cadena de razones para una creencia debe terminar en algo que se
  cree sin estar fundado en alguna otra cosa. Segundo, la creencia última debe
  ser de una naturaleza distinta a las creencias derivadas: Tiene que ser
  creencia perceptual, creer en algo percibido, or recordado en el presente.
  Tercero, la justificación inmediata de una creencia $p$, si la creencia no es
  una percepción, será otra creencia $q$, la cual se sigue, en la misma medida
  que implica, a $p$. Cuarto, creemos por inferencia a través de los eslabones
  en una cadena de relato.

  Hay un corolario implicito: cuando creemos en información histórica
  perteneciente a un pasado remoto, creemos que ha habido una cadena de relato.}

Sin embargo, Anscombe considera que esta no es la manera adecuada de establecer
esta relación. Mas bien: \blockquote[hjc 88: \emph{If} the written records that
we now see are grounds of our belief, they are first and foremost grounds for
belief in Caesar's killing, belief that the assassination is a solid bit of
history. Then our belief in that original event is a ground for belief in much
of the intermediate transmision.]{\emph{Si} los relatos escritos que vemos ahora
  son fundamento para nuestro creer, estos son primero y ante todo fundamento
  para la creencia en el asesinato de Cesar, creencia en que el asesinato es un
  pedazo sólido de historia. Entonces nuestra creencia en ese evento original es
  fundamento para el creer en mucha de la transimisión intermedia.}
¿Por qué creemos que hubo testigos del asesinato? Ciertamente porque creemos que
hubo un asesinato. La creencia de que hubo testigos es inferida de la creencia
en el hecho.

Anscombe compara este modo de entender la cadena de transmisión de información
histórica a nuestra creencia en la continuidad espacio-temporal. Si reconocemos
en una ocasión a una persona conocida como alguien que vimos la semana pasada,
nuestra creencia en que es la misma persona no es una inferencia de otra
creencia acerca de la continuidad espacio-temporal de un patrón humano desde
ahora hasta entonces, sino que más bien nuestra creencia en la continudad
espacio-temporal esta inferida del reconocimiento de la identidad de la persona.
Sin embargo, una evidencia sobre una interrupción en la continuidad sí alteraría
nuestra creencia en la identidad.

Elizabeth entonces concluye que: \blockquote[hjc 89: Belief in recorded history is
on the whole a belief that there has been a chain of tradition of reports and
records going back to contemporary knowledge; it is not a belief in the
historical facts by an inference that passes through the links of such a chain.
At most, that can very seldom be the case.]{La creencia en los registros de la
  historia consiste en general la creencia de que ha habido una cadena de
  tradición de informes y registros que van hacia el conocimiento contemporaneo;
  no es una creencia en hechos históricos por una inferencia que pasa por los
  eslabones de una cadena como esta. Como mucho, esto seria muy raramente el
  caso.}

Ahora bien, como se ha dicho antes, el interés de Anscombe no esta simplemente
en mostrar en que se equivoca Hume, sino que considera que la cuestión toca el
nervio de un problema con cierta profundidad:
\blockquote[causality in hume 2855: The interesting problem that arises, then,
is why the things we are told and the writings that we see are the starting
points for our belief in the far distant events and so in the intermediate chain
of record.]{El problema interesante que surge, entonces, es por qué las cosas
  que se nos dicen y los escritos que vemos son puntos de partida para nuestro
  creer en eventos distantes y así también en la cadena del relato intermedia.}



Para discutir esta cuestión Anscombe recurre a las reflexiones de Wittgenstein
en \emph{On Certainty}



8. The ditference between the concept of 'knowing' and the concept of 'being
certain' isn't of any great importance at all, except where "I know" is meant to
mean: I can't be wrong. In a law-court, for example, "I am certain" could
replace "I know" in every piece of testimony. We might even imagine its being
forbidden to say "I know" there. [A passage in WiZheZm Meister, where "You know"
or "You knew" is used in the sense "You were certain", the facts being different
from what he knew.]

12. -For "I know" seems to describe a state of affairs which guarantees what is
known, guarantees it as a fact. One always forgets the expression "I thought I
knew".

13. For it is not as though the proposition "It is so" could be inferred from
someone else's utterance: "I know it is so". Nor from the utterance together
with its not being a lie.-But can't I infer "It is so" from my own utterance "I
know etc."? Yes; and also "There is a hand there" follows from the proposition
"He knows that there's a hand there". But from his utterance "I know . . ." it
does not follow that he does know it.

14- That he does know takes some shewing. I j. It needs to be shewtz that no
mistake was possible. Giving the assurance "I know" doesn't suffice. For it is
after all only an assurance that I can't be making a mistake, and it needs to be
objectiueb established that I am not making a mistake about that.

2 1. Moore's view really comes down to this: the concept 'know' C is analogous
to the concepts 'believe', surmise', 'doubt', 'be convinced' in that the
statement "I know . .. ." can't be a mistake. And if that is so, then there can
be an inference from such an utterance to the truth of an assertion. And here
the form "I thought I knew" is being overlooked.-But if this latter is
inadmissible, then a mistake in the assertion must be logically impossible too.
And anyone who is acquainted with the language-game must realize thisan
assurance from a reliable man that he hows cannot contribute anything. 22. It
would surely be remarkable if we had to believe the reliable person who says "I
can't be wrong"; or who says "I am not wrong".

94. But I did not get my picture of the world by satisfying
myself of its correctness: nor do I have it because I am satisfied
of its correctness. No: it is the inherited background against
which I distinguish between true and false.



204. Giving grounds, however, justifying the evidence, comes
to an end;-but the end is not certain propositions' striking us
immediately as true, i.e. it is not a kind of seeing on our part;
it is our acting, which lies at the bottom of the language-game.
205. If the true is what is grounded, then the ground is not
true, nor yet false.
206. If someone asked us 'but is that true?" we might say "yes"
to him; and if he demanded grounds we might say "I can't give
you any grounds, but if you learn more you too will think the
same".
If this didn't come about, that would mean that he couldn't
for example learn history.
207. "Strange coincidence, that every man whose skull has
been opened had a brain!"


3 I I. Or imagine that the boy questioned the truth of history
(and everything that connects up with it)--and even whether the
earth had existed at all a hundred years before.
3 I 2. Here it strikes me as if this doubt were hollow. But in that
case-isn't belief in history hollow too ? No; there is so much that
this connects up with.


167. It is clear that our empirical propositions do not all have
the same status, since one can lay down such a proposition and
turn it from an empirical proposition into a norm of description.
Think of chemical investigations. Lavoisier makes experiments
with substances in his laboratory and now he concludes that this
and that takes place when there is burning. He does not say that
it might happen otherwise another time. He has got hold of a
definite world-picture-not of course one that he invented:
he learned it as a child. I say world-picture and not hypothesis,
because it is the matter-of-course foundation for his research and
as such also goes unmentioned.


24j To whom does anyone say that he knows something? To
himself, or to someone else. If he says it to himself, how is it
distinguished from the assertion that he is sure that things are like
that? There is no subjective sureness that I know something. The


certainty is subjective, but not the knowledge. So if I say "I
know that I have two hands", and that is not supposed to express
just my subjective certainty, I must be able to satisfy myself that
I am right. But I can't do that, for my having two hands is not
less certain before I have looked at them than afterwards. But I
could say: 'That I have two hands is an irreversible belief." That
would express the fact that I am not ready to let anything count
as a disproof of this proposition.
246. "Here I have arrived at a foundation of all my beliefs."
"This position I will holdl" But isn't that, precisely, ody because
I am completely convinced of it ?-What is 'being. completely convinced' like ?
247. What would it be like to doubt now whether I have two
hands ? Why can't I imagine it at all? What would I believe if I
didn't believe that? So far I have no system at all within which -
this doubt might exist.
248. I have arrived at the rock bottom of my convictions.
-~nd one might almost say that these fhdation-walls are
carried by the whole house.
249. One gives oneself a false picture of doubt.
2 j O. My having two hands is, in normal circumstances, as certain
as anything that I could produce in evidence for it.
That is why I am not in a position to take the sight of my hand
as evidence for it.
2 j I. Doesn't this mean: I shall proceed according to this belief
unconditionally, and not let anything confuse me ?
2~2. But it isn't just that I believe in this way tha.t I have two
hands, but that every reasonable person does.
25 3. At the foundation of well-founded belief lies belief that is
not founded.
2 j4. Any 'reasonable' person behaves like this
