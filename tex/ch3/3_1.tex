%SECCIÓN 1: ANSCOMBE Y WITTGENSTEIN
\section{Anscombe y Wittgenstein}

\subsection{El método de Wittgenstein}
El 16 de octubre de 1944 Ludwig Wittgenstein reanudó su itinerario de clases en
Cambridge con dos encuentros de dos horas cada semana. Aquél trimestre
'Michaelmas' contaba seis estudiantes en su curso, entre ellos Elizabeth
Anscombe. Las lecciones trataban sobre psicología filosófica, pero su interés no
era impartir una materia, sino un método. Era como un maestro de piano
---decía---, intentaba enseñar un estilo de pensar, una técnica. Su clase no era
como escuchar un concierto para sus estudiantes, más bien lo comparaban a
escuchar a un gran pianista practicando: eran testigos de aquellas acciones que
van dirigidas a construir una gran representación
musical.\footcite[p.~357]{pubnpriv}
 
En cierta ocasión Wittgenstein recibió a Anscombe con una pregunta: <<¿Por qué la
gente dice que era natural pensar que el sol giraba alrededor de la tierra en
lugar de que la tierra rotaba en su eje?>> Elizabeth contestó: <<Supongo que
porque se veía como si el sol girara alrededor de la tierra.>> <<Bueno\ldots>>,
añadió Wittgenstein, <<¿cómo se hubiera visto si se hubiera \emph{visto} como si
la tierra rotara en su propio eje?>> Anscombe reaccionó extendiendo las manos
delante de ella con las palmas hacia arriba y, levantándolas desde sus rodillas
con un movimiento circular, se inclinó hacia atrás asumiendo una expresión de
mareo. <<¡Exactamente!>> exclamó Wittgenstein.\footcite[cf.~][p.~151]{IWT}

Anscombe se percató del problema; la pregunta de Wittgenstein había puesto en
evidencia que hasta aquél momento no había ofrecido ningún significado relevante
para su expresión \emph{``se veía como si''} en su respuesta \emph{``se veía
    como si el sol girara alrededor de la tierra''}. 
 
¿Qué tipo de problema es este? 
¿Qué falta cuando una expresión carece de significado?

\subsection{El arte de hacer filosofía}

    \todo{``Within all great art there is a WILD animal: tamed.''} Wittgenstein
    pensaba que \citalitinterlin{dentro de todo buen arte hay un animal salvaje
      domado}\footcite[p.~43e]{cnv}, sin embargo él mismo juzgaba sobre su modo
    de hacer arte que \citalitinterlin{En mis actividades artísticas tengo
      meramente buenos modales.} \footcite[p.~29e]{cnv}

    \todo{``In my artistic activities I have merely good manners''} 

    \todo{``In the same sense: my house for Gretl is the product of a decidedly
      sensitive ear, good manners, the expression of great understanding. But
      primordial life, wild life striving to erupt in the open -- is lacking. And
      you might say, health is lacking (Kierkergaard)''} 

    ``Even in music, the art for which Wittgenstein had the Greatest
      feeling, he showed above all great understanding, rather than manifesting
      wild life... When he played music with others... his interest was in getting
      it right, in using his acutelysensitive ear to impose upon his fellow
      musicians an extraordinary exactitude of expression. One could even say he
      was not intereste in creating music, but in recreating it. When he played,
      he was not expressing himself, his own primordial life, but the thoughts,
      the life of others. He was probably right to regard himself not as creative
      but as reproductive ...It was only in philosophy that his creativity could
      really be awakened. Only then, as Russell had long ago noticed, does one see
      in him 'wild life striving to erupt in the open''


Vida salvaje luchando por hacer erupción en campo abierto.

    Ordinariamente tomamos parte en la actividad humana que es el
    lenguaje. Jugamos el juego del lenguaje. ---¿Jugarlo es entenderlo?--- 
    A la vista de Wittgenstein saltaban extraños problemas sobre las reglas de este
    juego; entonces no podía evitar escudriñarlas al
    detalle.\footcite[cf.~][loc.7099]{monk} 
    Esta tendencia ilustra una cualidad del carácter filosófico de Wittgenstein: su
    preocupación por investigaciones acerca de las cuestiones más fundamentales. 

    Estudiando ingeniería en Manchester se interesó por los fundamentos de las
    matemáticas. Desde las perspectivas de Frege y Russell identificó los problemas
    de explicar dichos fundamentos en términos de premisas lógicas. 
    Confrontando los problemas fundamentales de la lógica 
    describió las principales razones de los problemas de la filosofía como
    confusiones en el uso del lenguaje. 

    A lo largo de esta busqueda de respuestas fundamentales el pensamiento de
    Wittgenstein pasó por diversas transformaciones. Sin embargo una de las
    constantes importantes en el desarrollo de sus reflexiones fue su impresión de
    la naturaleza de los problemas filosóficos.
    Las cuestiones de la filosofía no son problemáticas por ser erróneas, sino por
    no tener significado.\footcite[cf.~][4.003]{tractatus} Una proposición sin
    significado que no es puesta al descubierto como tal
    atrapa al filósofo dentro de una confusión del lenguaje que no le permite
    acceder a la realidad. Salir de la confusión no consiste en refutar una doctrina
    y plantear una teoría alternativa, sino en examinar las operaciones hechas con
    las palabras para llegar a manejar una visión clara del empleo de nuestras
    expresiones. La filosofía no es un cuerpo doctrinal, sino una
    actividad\footcite[cf.~][4.112]{tractatus}y una
    terapia\footcite[cf.~][\S133]{PI}.   

    La actitud terapéutica adoptada por Wittgenstein en su atención de las
    confusiones filosóficas fue su respuesta más definitiva a la naturaleza de estos
    problemas. Para ello halló los más eficaces remedios en sus investigaciones sobre el
    significado y el sentido del lenguaje.

    Durante su vida sostuvo dos grandes descripciones del significado.
    Originalmente describió el lenguaje como una imagen que representa el posible
    estado de las cosas en el mundo.
    En una segunda etapa se distanció de esta analogía para describir al lenguaje
    como una herramienta cuyo significado consiste en la suma de las múltiples
    semejanzas familiares que aparecen en los distintos usos para los cuales el
    lenguaje es empleado en la actividad humana.
    Dentro de la primera descripción una expresión sin significado es una cuyos
    elementos no componen una representación del posible estado de las cosas.
    Dentro de la segunda descripción una expresión sin significado resulta del
    empleo de una expresión propia de un ``juego del lenguaje'' fuera de su
    contexto. 

    Estas dos etapas del pensamiento de Wittgenstein 
    son representadas por dos importantes tratados. 
    El \emph{'Tractatus Logico\=/Philosophicus'}, publicado en 1921, recoge sus
    esfuerzos por elaborar un gran tratado filosófico comenzados en 1911 y
    culminados durante la Primera Guerra Mundial. El segundo, \emph{'Philosophical 
        Investigations'}, traducido por Anscombe y publicado posthumamente en 1953,
    fue elaborado a partir de múltiples manuscritos desarrollados por Wittgenstein
    desde su regreso a Cambridge en 1929 hasta su muerte en 1951. 
    Ambas obras generaron un 'corte' en la historia de la filosofía. La manera de
    hacer filosofía cambió después de cada tratado.\footcite[cf.~][p.~181]{twocuts}

    %\begin{revision}
    %En ocasiones como esta la
    %discusión con Wittgenstein llevaba a Anscombe a afirmaciones para las cuales no
    %podía ofrecer mejor significado que los sugeridos por concepciones ingenuas. Una
    %concepción así no es otra cosa que ausencia de pensamiento, pero su falta de
    %significado no es evidente, sino que requiere de la fuerza de un `Copérnico'
    %para ponerla en cuestión efectivamente.\footcite[cf. 151]{IWT} 
    %\end{revision}

    %\begin{revision}
    %En lo concerniente a la filosofía, Wittgenstein siempre tendía a escudriñar las
    %reglas del juego, más que jugarlo. 
    %Anscombe encontró en la filosofía analítica ---en el método de Wittgenstein---
    %un método liberador, que le permitió involucrarse en el 'juego' de la filosofía
    %con enérgica fortaleza. 
    %\end{revision}

    Anscombe conoció a Wittgenstein en los años culminantes de su pensamiento
    filosófico. Comenzó a asistir a sus lecciones en el trimestre 'michaelmas' de
    1942. Eran unos diez estudiantes en clase, y la materia discutida era sobre los
    fundamentos de las matemáticas. En abril de 1943 Wittgenstein interrumpió sus
    clases para unirse a los esfuerzos por atender los daños de la Segunda Guerra
    Mundial trabajando en 'Guy's Hospital' en Newscastle. Regresó a Cambridge en
    octubre de 1944 y el 16 del mismo mes reanudó sus lecciones con seis
    estudiantes, Anscombe entre ellos. Los temas trabajados en estas lecciones son
    correspondientes con los números \S189--\S241 de 'Philosophical Investigations'.
    En el curso 1945--1946 Elizabeth asistió junto a otros dieciocho estudiantes a
    lecciones sobre filosofía de la psicología. El curso de 1946--1947 fue el último
    término de lecciones ofrecidas por Wittgenstein en Cambridge antes de su retiro
    en octubre de 1947. Durante ese curso le dedicó una tarde a la semana a Anscombe
    y W. A. Hijab en lecciones sobre filosofía de la religión.

    Al comienzo de sus lecciones en 1944 Wittgenstein escribía a su amigo Rush Rhees:
    \citalitinterlin{
        \ldots mis clases no han ido tan mal. Thouless esta asistiendo, y una mujer, 
        'Mrs so and so'
        que se llama a sí misma 
        'Miss Anscombe',
        que ciertamente es inteligente, aunque no del calibre de Kreisel.
        \footcite[p.~371]{cambridgeletters}
    }
    Un año mas tarde escribía a Norman Malcolm:
    \citalitinterlin{
        \ldots mi clase ahora es bastante grande, 19 personas. \ldots Smythies esta
        viniendo, y una mujer que es muy buena, es decir, más que solamente
        inteligente\ldots 
        \footcite[p.~388]{cambridgeletters}
    }
    Aquellos años no sólo creció en Wittgenstein la apreciación de la capacidad de
    Anscombe, sino que se afianzó entre ellos una estrecha amistad. 

    La influencia de Wittgenstein fue decisiva para el desarrollo filosófico de
    Elizabeth. Las lecciones con Wittgenstein eran directas y con franqueza. Esta
    metodología carente de cualquier parafernalia era inquietante para algunos,
    inspiradora para otros, pero fue tremendamente liberadora para
    ella.\footcite[loc 9853 Chapter 4, Section 24, \S5]{monk} Esta libertad
    quedaba demostrada en que Anscombe no se contentaba con repetir lo que decía
    Wittgenstein, sino que pensaba por sí misma; en esto precisamente era más fiel
    al espíritu de la filosofía que había aprendido de él. Sobre esta relación,
    Phillipa Foot, amiga de ambos, cuenta que durante mucho tiempo sostuvo
    objeciones a las afirmaciones de Wittgenstein, eventualmente, un comentario de
    Norman Malcom la hizo pensar que podía haber valor en lo que Wittgenstein decía.
    Cuestionó entonces a Anscombe: 
    ``¿Por qué no me dijiste?'', ella le contestó: ``Porque es importante que uno
    tenga sus resistencias''. Anscombe evidentemente pensaba ---continúa Foot: 
    \citalitlar{
        que un largo periodo de vigorosa objeción era la mejor manera de entender a
        Wittgenstein. Aun cuando era su amiga cercana y albacea literaria, y una de
        los primeros en reconocer su grandeza, nada podía ser más lejano de su
        carácter y modo de pensamiento que el discipulado.\footcite[p.~4]{teichmann}
    }


    \todo{introducir algunos contrastes y relaciones entre Anscombe y Wittgenstein
        para explicar la incursión en la vida/pensamiento de W.}

    %TERCERA CUESTIÓN: DE LA ILUSTRACIÓN AL TRACTATUS
    \ifdraft{\subsubsection{Desde la Ilustración hacia el desarrollo del Tractatus}}{}

    \todo{Con este párrafo nos remitimos desde la metodología a la elaboración del
        Tractatus, para llegar a los puntos fundamentales de la obra}

    %Para Ludwig Wittgenstein el método general adecuado de discutir los problemas
    %filosóficos era mostrar que la persona no ha provisto significado (o referencia)
    %para ciertos signos en sus expresiones.\footcite[cf. p. 151]{IWT} Creía
    %que el camino que lleva a formular estos problemas está frecuentemente trazado
    %por la mala comprensión de la lógica de nuestro lenguaje. Por tanto, el modo de
    %aclarar esta confusión consistía en identificar en el lenguaje el límite de lo
    %que expresa pensamiento; lo que queda al otro lado de esta frontera es
    %simplemente sinsentido. En otras palabras: \citalitinterlin{Lo que
    %    \todo{traducción difícil. \emph{``What can be said at all''}} 
    %    siquiera puede ser dicho puede ser dicho claramente; y de lo que uno no
    %    puede hablar, de eso, uno debe guardar silencio}. 
    %\footcite[prefacio]{tractatus}
    %Con esta expresión  Wittgenstein resumía el significado del libro que recoge su
    %esfuerzo para resolver este problema de la filosofía: el \emph{'Tractatus
    %    Logico\=/Philosophicus'}. 

    %Elaboración del Tractatus
    %En el 14 empezó la guerra, en el 15 W. escribió a R. con sus intenciones de
    %hacer un tratado. En el 18 lo acabó. En el 19 envió el manuscrito a R. En el 22
    %lo publicó.
    \subsection{El gran tratado de Wittgenstein}
    \ifdraft{\subsubsection{De Manchester a Cambridge}}{}

    \todo{El propósito de recorrer el desarrollo que lleva al Tractatus es ofrecer
        un trasfondo a los puntos que resaltamos más adelante.}
    Los primeros esfuerzos de Wittgenstein por escribir una obra sobre filosofía
    habían comenzado en 1911. En otoño de ese año en lugar de continuar sus estudios
    de ingeniería en Manchester, determinó irse a Cambridge donde Bertrand Russell
    ofrecía sus lecciones. Su hermana le describe en esa época:   
    \citalitlar{Fue repentinamente agarrado por la filosofía ---es decir, por la
        reflexión en problemas filosóficos--- tan violentamente y tan en contra de su
        voluntad que sufrió severamente por la doble y conflictiva llamada interior
        y se veía a sí mismo como roto en dos. Una de muchas transformaciones por las
        que pasaría en su vida había venido sobre él y le estremeció hasta lo más
        profundo. Estaba concentrado en escribir un trabajo filosófico y finalmente
        determinó mostrar el plan de su obra al Profesor Frege en Jena, quien
        había discutido preguntas similares. [\ldots] Frege alentó a Ludwig en su
        búsqueda filosófica y le aconsejó que fuera a Cambridge como alumno del
        Profesor Russell, cosa que Ludwig ciertamente hizo.\footcite[p. 73]{mcguinness}}

    Asistió a un término de lecciones con Russell y al finalizar no estaba seguro de
    abandonar la ingeniería por la filosofía, se cuestionaba si verdaderamente tenía
    talento para ella. Consultó a su nuevo profesor al respecto y éste le pidió que
    escribiera algo para ayudarle a hacer un juicio. 

    En enero de 1912 Wittgenstein regresó a Cambridge con un manuscrito que
    demostraba auténtica agudeza filosófica. Convencido de su gran capacidad,
    Russell alentó a Ludwig a continuar dedicándose a la filosofía. Este
    apoyo fue crucial para Wittgenstein, hecho puesto de manifiesto por el gran
    empeño con el que trabajó en sus estudios aquel curso. Al finalizar el termino
    Russell alegaba que Ludwig había aprendido todo lo que él podía
    enseñarle.\footcite[cap. 3 loc 865]{monk} 

    \ifdraft{\subsubsection{A Noruega a Resolver los problemas de la lógica}}{}
    Después de una temporada en Cambridge llena de eventos y desarrollos
    Wittgenstein anunció en septiembre de 1913 sus planes de retirarse para
    dedicarse exclusivamente a trabajar en resolver los problemas fundamentales de
    la lógica. Su idea era irse a Noruega, a algún lugar apartado, ya que pensaba
    que en Cambridge las interrupciones obstaculizarían su trabajo.\footcite[cap. 4
    loc 1844]{monk} 

    \ifdraft{\subsubsection{La Gran Guerra}}{}
    El trabajo en Noruega fue escabroso. En el verano de 1914 interrumpió su tarea
    para tomar un receso en Viena.\footcite[cap. 5 loc 2154]{monk} Había planificado
    regresar a Noruega después del verano, sin embargo la tensión entre las
    potencias europeas, agravada desde el atentado de Sarajevo a finales de junio de
    aquel año, detonó en el estallido de la Gran Guerra. El 7 de agosto de 1914
    Wittgenstein se enlistaba como voluntario al servicio militar. Sería en las
    trincheras donde culminaría su gran tratado filosófico.

    El 22 de octubre de 1915 Wittgenstein escribió a Russell desde el taller de
    artillería en Sokal, al norte de Lemberg, con lo que sería una primera versión
    de su libro.\footcite[cf. p.84]{cambridgeletters} Cuatro años más tarde, el 13
    de marzo, escribía a Russell desde Cassino donde se hallaba como prisionero de
    guerra en un campamento italiano\footcite[cf. p.268]{mcguinness}: 
    \citalitlar{He escrito un libro llamado ``Logisch-Philosophische Abhandlung''
        que contiene todo mi trabajo de los últimos seis años. Creo que finalmente he
        resuelto todos nuestros problemas. Esto puede sonar arrogante, pero no puedo
        evitar creerlo. Terminé el libro en agosto de 1918 y dos meses más tarde fui
    hecho 'Prigioniere'.\footcite[p.89]{cambridgeletters}}

    \ifdraft{\subsubsection{Aire de Misticismo}}{}
    En junio de aquel año logró enviar el manuscrito del libro a Russell por medio
    de John Maynard Keynes quien intervino con las autoridades italianas para
    permitir el envío seguro del texto\footcite[p.90 y 91]{cambridgeletters}. El 26
    de agosto de 1919 fue oficialmente liberado de sus funciones
    militares\footcite[p.277]{mcguinness} y en diciembre finalmente pudo encontrarse
    con Russell en la Haya. De aquel encuentro Russell escribe:
    \citalitlar{Había sentido un sabor a misticismo en su libro, pero me quedé
        asombrado cuando vi que se ha convertido en un completo místico. Lee a gente
        como Kierkergaard y Angelus Silesius, y ha contemplado seriamente el
        convertirse en un monje. Todo comenzó con ``Las variedades de la experiencia
        religiosa'' de William James y creció durante el invierno que pasó solo en
        Noruega antes de la guerra cuando casi se había vuelto loco. Luego, durante
        la guerra, algo curioso ocurrió. Estuvo de servicio en el pueblo de Tarnov
        en Galicia, y se encontró con una librería que parecía contener solamente
        postales. Sin embargo, entró y encontró que tenían un sólo libro: Los
        Evangelios abreviados de Tolstoy. Compró el libro simplemente porque no
        había otro. Lo leyó y releyó y desde entonces lo llevaba siempre consigo,
        estando bajo fuego y en todo momento. Aunque en su conjunto le gusta menos
        Tolstoy que Dostoeweski. Ha penetrado profundamente en místicos modos de
        pensar y sentir, aunque pienso que lo que le gusta del misticismo es su
        poder para hacerle dejar de pensar. No creo que realmente se haga monje, es
        una idea, no una intención. Su intención es ser profesor. Repartió todo su
        dinero entre sus hermanos y hermanas, pues encuentra que las posesiones
        terrenales son una carga. \footcite[p. 112]{cambridgeletters}}

    \ifdraft{\subsubsection{En busca de una experiencia religiosa}}{}
    Cuando Wittgenstein se enlistó en el ejercito para la guerra en 1914 tenía
    motivaciones más complejas que la defensa de su patria.\footcite[loc2276]{monk}
    Sentía que, de algún modo, la experiencia de encarar la muerte le haría mejor
    persona. Había leído sobre el valor espiritual de confrontarse con la muerte en
    ``Las variedades de la experiencia religiosa'':
    \citalitlar{No importa cuales sean las fragilidades de un hombre, si estuviera
        dispuesto a encarar la muerte, y más aún si la padece heroicamente, en el
        servicio que éste haya escogido, este hecho le consagra para
        siempre.\footcite[loc 2295]{monk}}

    Wittgenstein esperaba esta experiencia religiosa de la guerra.
    \citalitinterlin{Quizás}, escribía en su diario, \citalitinterlin{La cercanía de
        la muerte traerá luz a la vida. Dios me ilumine.}\footcite[loc2295]{monk}
    La guerra había coincidido con esta época en la que el deseo de convertirse en
    una persona diferente era más fuerte aún que su deseo de resolver los problemas
    fundamentales de la lógica.\footcite[loc2305]{monk}

    \ifdraft{\subsubsection{La Principal Contienda}}{}
    Esta transformación sorprendió a Russell en aquel encuentro en la Haya, pero
    además fue motivo de confusión en la tarea de entender el Tractatus. Cuando
    Russell recibió el manuscrito en agosto escribió a Wittgenstein cuestionando
    algunos puntos difíciles del texto. En su carta observaba: 
    \citalitlar{Estoy convencido de que estás en lo correcto en tu principal
        contienda, que las proposiciones lógicas son tautologías, las cuales no son
        verdad en el mismo modo que las proposiciones
        sustanciales.\footcite[p.96]{cambridgeletters}}

    Esta interpretación del texto se ajusta bien a la importancia que había tenido
    esta cuestión en las discusiones entre Russell y Wittgenstein. Así lo expresaba
    Russell en ``Introducción a la Filosofía Matemática'' publicado en mayo de aquel
    año: 
    \citalitlar{
        \todo{The importance of “tautology” for a definition of
        mathematics was pointed out to me by my former pupil Ludwig Wittgenstein,
        who was working on the problem. I do not know whether he has solved it, or
        even whether he is alive or dead.} 
        La importancia de la ``tautología'' para una definición de las
        matemáticas me fue señalada por mi ex-alumno Ludwig Wittgenstein, quien
        estaba trabajando en el problema. No sé si lo ha resuelto, o siquera si está
        vivo o muerto.\footcite[p.205]{introtomathphi}} 

    Sin embargo para el Tractatus la cuestión sobre las proposiciones lógicas como
    tautologías no es ya el tema principal, sino que enfatiza otra cuestión, así
    corrige Wittgenstein en su respuesta a la carta de Russell:
    \citalitlar{Ahora me temo que realmente no has captado mi principal contienda,
        para lo cual todo el asunto de las proposiciones lógicas es sólo corolario.
        El punto principal es la teoría sobre lo que puede ser expresado por
        proposiciones ---es decir, por el lenguaje--- (y, lo que viene a ser lo mismo,
        aquello que puede ser pensado) y lo que no puede ser expresado por medio de
        proposiciones, sino solamente mostrado; lo cual, creo, es el problema
        cardinal de la filosofía\ldots \footcite[p. 98]{cambridgeletters}}

    Esta respuesta de Wittgenstein no solo pone de manifiesto su cambio de enfoque,
    sino que ofrece una clave para introducirse en su obra. 

    %CUARTA CUESTIÓN: LA ``DOCTRINA'' DEL TRACTATUS
    %1. La filosofía como actividad
    %2. El pensamiento como representación
    %3. Los polos de verdad y falsedad de las proposiciones
    %4. La diferencia ente decir y mostrar
    \subsection{Las elucidaciones del Tractatus}
    \todo{Este párrafo resume los cuatro puntos del Tractatus que se desglosarán en
        los próximos párrafos} 
    Desde las proposiciones principales del Tractatus queda claro que el tema
    central del libro es la conexión entre el lenguaje, o el pensamiento, y la
    realidad.  
    \todo{1.Filosofía como actividad}
    En este nexo es donde la actividad filosófica ha de buscar esclarecer el
    pensamiento.
    \todo{2.El pensamiento como representación}
    La tesis básica sobre esta relación consiste en que las proposiciones, o su
    equivalente en la mente, son imágenes de los hechos.
    \todo{3.Las proposiciones como proyecciones con polos de verdad-falsedad}
    La proposición es la misma imagen tanto si es cierta como si es falsa, es decir,
    es la misma imagen sin importar que lo que se corresponde a ésta es el caso que
    es cierto o no. El mundo es la totalidad de los hechos, a saber, de lo
    equivalente en la realidad a las proposiciones verdaderas.
    \todo{4.La distinción entre el decir y el mostrar}
    Sólo las situaciones que pueden ser plasmadas en imágenes pueden ser afirmadas
    en proposiciones. Adicionalmente hay mucho que es inexpresable, lo cual no
    debemos intentar enunciar, sino más bien contemplar sin palabras.\footcite[cf.
    p.19]{IWT}

    \subsubsection{La filosofía como actividad}

    La filosofía es la actividad que tiene como objeto la clarificación lógica
    de los pensamientos.\footcite[4.112 p. 52]{tractatus} El problema de muchas de
    las proposiciones y preguntas que se han escrito acerca de asuntos filosóficos
    no es que sean falsas, sino carentes de significado. Wittgenstein continúa: 
    \citalitlar{4.003~En consecuencia no podemos dar respuesta a preguntas de este
        tipo, sino exponer su falta de sentido. Muchas cuestiones y proposiciones de
        los filósofos resultan del hecho de que no entendemos la lógica de nuestro
        lenguaje. (Son del mismo genero que la pregunta sobre si lo Bueno es más o
        menos idéntico a lo Bello). Y así no hay que sorprenderse ante el hecho de
        que los problemas más profundos realmente no son problemas.\footcite[4.003
        p. 45]{tractatus}} 

    Es así que el precipitado de la reflexión filosófica que el Tractatus recoge no
    pretende componer un cuerpo doctrinal articulado por proposiciones filosóficas,
    sino más bien ofrecer `elucidaciones' que sirven como etapas escalonadas y
    transitorias que al ser superadas conducen a ver el mundo correctamente. Este
    esfuerzo hace de pensamientos opacos e indistintos unos claros y con límites
    bien definidos.\footcite[cf. 4.112 y 6.54]{tractatus} 
    La posibilidad de llegar a una visión clara del mundo es fruto de la posibilidad
    de lograr aclarar la lógica del lenguaje. El lenguaje, a su vez, está compuesto
    de la totalidad de las proposiciones, y éstas, cuando tienen sentido,
    representan el pensamiento.\footcite[cf. 4 y 4.001]{tractatus} 
    Sin embargo, el mismo lenguaje que puede expresar el pensamiento lo disfraza:

    \citalitlar{4.002~El lenguaje disfraza el pensamiento; de tal manera que de la
        forma externa de sus ropajes uno no puede inferir la forma del pensamiento
        que estos revisten, porque la forma externa de la vestimenta esta elaborada
        con un propósito bastante distinto al de favorecer que la forma del cuerpo
        sea conocida.}

    El intento de llegar desde el lenguaje al pensamiento por medio de las
    proposiciones con significado es el esfuerzo por conocer una imagen de la
    realidad. El pensamiento es la imagen lógica de los hechos, en él se contiene la
    posibilidad del estado de las cosas que son pensadas y la totalidad de los
    pensamientos verdaderos es una imagen del mundo.\footcite[cf.][3 y
    3.001]{tractatus}

    \subsubsection{El pensamiento como representación}

    El pensamiento es representación de la realidad por la identidad existente entre
    la posibilidad de la estructura de una proposición y la posibilidad de la
    estructura un hecho:

    \citalitlar{Los objetos ---que son simples--- se combinan en situaciones
        elementales. El modo en el que se sujetan juntos en una situación tal es su
        estructura. Forma es la posibilidad de esa estructura. No todas las
        estructuras posibles son actuales: una que es actual es un `hecho
        elemental'. Nosotros formamos imágenes de los hechos, de hechos posibles
        ciertamente, pero algunos de ellos son actuales también. Una imagen consiste
        en sus elementos combinados en un modo específico. Al estar así presentan a
        los objetos denominados por ellos como combinados específicamente en ese
        mismo modo. La combinación de los elementos de la imagen ---la combinación
        siendo presentada--- se llama su estructura y su posibilidad se llama la
        forma de representación de la imagen.   
        Esta `forma de representación' es la posibilidad de que las cosas están
        combinadas como lo están los elementos de la imagen.
        \footnote{\cite[cf.][p.~171]{simplicity}; \cite[n.~2.15]{tractatus}}}

    La representación y los hechos tienen en común la forma lógica:
    \citalitlar{2.18~Lo que toda representación, de una forma cualquiera, debe tener
        en común con la realidad, de manera que pueda representarla ---cierta o
        falsamente--- de algún modo, es su forma lógica, esto es, la forma de la
        realidad.\footcite[p.34]{tractatus}}  

    \subsubsection{Las proposiciones como proyecciones con polos de verdad-falsedad}
    \todo{Añadir analogía sobre la verdad ---si es que no se va a usar en el próximo
    apartado---}
    La imagen de la realidad se convierte en proposición en el momento en que
    nosotros correlacionamos sus elementos con las cosas
    actuales.\footcite[cf.~][p.~73]{IWT}
    La condición de posibilidad de entablar dicha correlación es la relación interna
    entre los elementos de la imagen en una estructura con
    sentido.\footcite[cf.~][p.~68]{IWT}
    De este modo:
    \citalitlar{5.4733~Frege dice: Toda proposición legítimamente construida tiene
        que tener un sentido; y yo digo: Toda proposición posible está legítimamente
        construida, y si ésta no tiene sentido es sólo porque no hemos dado
        significado a alguna de sus partes constitutivas. (Incluso cuando pensemos
        que lo hemos hecho.)\footcite[p.~78]{tractatus}}

    La proposición expresa el pensamiento perceptiblemente por medio de signos.
    Usamos los signos de las proposiciones como proyecciones del estado de las cosas
    y las proposiciones son el signo proposicional en su relación proyectiva con el
    mundo. A la proposición le corresponde todo lo que le corresponde a la
    proyección, pero no lo que es proyectado, de tal modo, que la proposición no
    contiene aún su sentido, sino la posibilidad de expresarlo; la forma de su
    sentido, pero no su contenido.\footcite[cf.~][3.1,3.11-3.13]{tractatus} 

    La proposición no `contiene su sentido' porque la correlación la hacemos nosotros,
    al `pensar su sentido'. Hacemos esto cuando usamos los elementos de la
    proposición para representar los objetos cuya posible configuración estamos 
    reproduciendo en la disposición de los elementos de la proposición. Esto es lo
    que significa que la proposición sea llamada una imagen de la
    realidad.\footcite[cf.~][p.69]{IWT}  

    Toda proposición-imagen tiene dos acepciones. Puede ser una descripción de
    la existencia de una configuración de objetos o puede ser una descripción de la
    no-existencia de una configuración de objetos.\footcite[cf.~][p.~72]{IWT} 
    %Es una peculiaridad de la proyección el que de ésta y del método de proyección
    %se puede decir qué es lo que se está proyectando, sin que sea necesario que tal
    %cosa exista físicamente.\footcite[cf.~][p.~72]{IWT} 
    %La idea de la proyección es peculiarmente apta para explicar el carácter de una
    %proposición como teniendo sentido independientemente de los hechos, como
    %inteligible aún antes de que se sepa que es cierta; como algo que concierne lo
    %que se puede cuestionar sobre si es verdad, y saber lo que se pregunta antes de
    %conocer la respuesta.\footcite[cf.~][p.~73]{IWT}
    Esta doble acepción es el resultado de que la proposición-imagen puede ser una
    proyección hecha en sentido positivo o negativo.\footcite[cf.~][p.~74]{IWT} Esto
    queda ilustrado en una analogía:

    \citalitlar{4.463~La proposición, la imagen, el modelo, son en el sentido
        negativo como un cuerpo solido, que restringe el libre movimiento de otro:
        en el sentido positivo, son como un espacio limitado por una sustancia
        sólida, en la cual un cuerpo puede ser colocado.\footcite[p.~63]{tractatus}}

    De este modo toda proposición-imagen tiene dos polos; de verdad y de falsedad.
    Las tautologías y las contradicciones, por su parte, no son imagenes de la
    realidad ya que no representan ningún posible estado de las cosas. Así continúa
    la ilustración anterior:

    \citalitlar{4.463~Una tautología deja abierto para la realidad el total infinito
        del espacio lógico; una contradicción llena el total del espacio lógico no
        dejando ningún punto de él para la realidad. Así pues ninguna de las dos
        puede determinar la realidad de ningún modo.\footcite[p.~78]{tractatus}}

    La verdad de las proposiciones es posible, de las tautologías es cierta y de las
    contradicciones imposible. La tautología y la contradicción son los casos límite
    de la combinación de signos ---específicamente--- su
    disolución.\footcite[cf.~][4.464 y 4.466]{tractatus} Las tautologías son
    proposiciones sin sentido (carecen de polos de verdad y falsedad), su negación son
    las contradicciones. Los intentos de decir lo que sólo puede ser mostrado
    resultan en esto, en formaciones de palabras que carecen de sentido, es decir,
    son formaciones que parecen oraciones, cuyos componentes resultan no tener
    significado en esa forma de oración.\footcite[cf.~][p.~163~\S2]{IWT}.

    \subsubsection{La distinción entre el decir y el mostrar}
    La conexión entre las tautologías y aquello que no se puede decir, sino mostrar,
    es que éstas ---siendo proposiciones lógicas sin sentido--- muestran la 'lógica del
    mundo'.\footcite[cf.~][p.~163~\S3]{IWT}. Esta 'lógica del mundo' o 'de los
    hechos' es la que más prominentemente aparece en el Tractatus entre las cosas
    que no pueden ser dichas, sino mostradas. Esta lógica no solo se muestra en las
    tautologías, sino en todas las proposiciones. Queda exhibida en las proposiciones
    diciendo aquello que pueden decir. 

    La forma lógica no puede expresarse desde el lenguaje, pues es la forma del
    lenguaje mismo, se hace manifiesta en éste, no es representativa de los objetos
    y tampoco puede ser representada por signos, tiene que ser mostrada:
    \citalitlar{4.0312~La posibilidad de las proposiciones se basa en el principio de
        la representación de los objetos por medio de signos. Mi pensamiento
        fundamental es que las ``constantes lógicas'' no son representativas. Que la
        lógica de los hechos no puede ser representada.\footcite[p.~48]{tractatus}}

    La lógica es, por tanto, trascendental, no en el sentido de que las
    proposiciones sobre lógica afirmen verdades trascendentales, sino en que todas
    las proposiciones muestran algo que permea todo lo decible, pero es en sí mismo
    indecible.\footcite[cf.~][p.~166 \S2]{IWT}

    Otra cuestión notoria entre aquello que no puede ser dicho, sino mostrado es la
    cuestión acerca de la verdad del solipsismo. Los limites del mundo son los
    límites de la lógica, lo que no podemos pensar, no podemos pensarlo, y por tanto
    tampoco decirlo. Los límites de mi lenguaje significan los límites de mi
    mundo.\footcite[cf~.][5.6~y~5.61]{tractatus} De este modo:
    \citalitlar{5.62~[\ldots]Lo que el solipsismo \emph{significa}, es ciertamente
        correcto, sólo que no puede ser \emph{dicho}, pero se muestra a sí
        mismo. Que el mundo es \emph{mi} mundo, se muestra a sí mismo en el hecho
        de que los limites del lenguaje (de \emph{aquel} lenguaje que yo
        entiendo) significan los límites de mi
        mundo.\footcite[cf~.][p.~89]{tractatus}} 

    Así como la lógica del mundo y la verdad del solipsismo quedan mostradas,
    también, las verdades éticas y religiosas, aunque no expresables, se manifiestan
    a sí mismas en la vida. 

    Existe, por tanto lo inexpresable que se muestra a sí mismo, esto es lo
    místico.\footcite[cf.~][6.522]{tractatus}

    De la voluntad como sujeto de la ética no podemos
    hablar\footcite[cf.~][6.423]{tractatus}. El mundo es independiente de nuestra
    voluntad ya que no hay conexión lógica entre ésta y los hechos.
    La voluntad y la acción como fenómenos, por tanto, interesan sólo a la
    psicología.\footcite[cf.~][p.171 \S3]{IWT}

    El significado del mundo tiene que estar fuera del
    mundo\footcite[cf.~][6.41]{tractatus} y Dios no se revela \emph{en} el
    mundo\footcite[cf.~][6.432]{tractatus}. 
    Esto se sigue de la teoría de la representación; una proposición y su negación
    son ambas posibles, cuál es verdad es accidental.\footcite[cf.~][p.170 \S4]{IWT}
    Si hay un valor que valga la pena para el mundo tiene que estar fuera de lo que
    es el caso que es; lo que hace que el mundo tenga un valor no-accidental tiene
    que estar fuera de lo accidental, tiene que estar fuera del
    mundo.\footcite[cf.~][6.41]{tractatus} 

    Finalmente, aplicar el límite de lo que puede ser expresado a la actividad
    filosófica significa que:
    \citalitlar{6.53~El método correcto para la filosofía sería este. No decir nada
        excepto lo que pueda ser dicho, esto es, proposiciones de la ciencia
        natural, es decir, algo que no tiene nada que ver con la filosofía: y luego
        siempre, cuando alguien quiera decir algo metafísico, demostrarle que no ha
        logrado dar significado a ciertos signos en sus proposiciones. Este método
        sería insatisfactorio para la otra persona ---no tendría la impresión de que
        le estuviéramos enseñando filosofía--- pero este método sería el único
        estrictamente correcto.\footcite[p. 107--108]{tractatus}}
    \todo{Añadir como conclusión del resumen la finalidad ética del tratado.}

    \subsection{Formación filosófica de Elizabeth}
    \subsubsection{De Wittgenstein a Anscombe}
    En el 1929 Wittgenstein presentó el Tractatus Logico\=/Philosophicus como su
    tesis doctoral en Cambridge. Ese mismo año fue designado como profesor en
    ``Trinity College'', allí estaría hasta 1936.

    \subsubsection{Causalidad reflexiones iniciales de Anscombe}
    Por aquella época la joven Gertrude Elizabeth Margaret Anscombe, andaba buscando
    un buen argumento que demostrara que todo lo que existe tiene que tener una
    causa. ¿Por qué cuando algo ocurre estamos seguros de que tiene una causa? Nadie
    sabía darle una respuesta. Sin darse cuenta, se había despertado en Anscombe
    una pasión por la filosofía que le acompañaría el resto de su vida.

    El origen de su peculiar curiosidad por la causalidad se hallaba en una obra
    llamada `Teología Natural' escrita por un jesuita del siglo XIX. Había llegado a
    este libro motivada por su conversión a la Iglesia Católica ---fruto, a su vez,
    de lecturas hechas entre los doce y los quince---.\footcite[cf.~][p.~vii \S1]{M&PotM}
    El tratado presentaba un argumento sobre la existencia de la `Causa Primera' y
    como preliminar a éste ofrecía una demostración de un `principio de causalidad'
    según el cual todo cuanto existe tiene que tener una causa. Anscombe notó,
    escasamente escondido en una premisa, un presupuesto de la conclusión del propio
    argumento. Aquel ``petitio principii'' le pareció un simple descuido y resolvió,
    por tanto, escribir una versión mejorada de la demostración.
    Durante los siguientes dos o tres años produjo unas cinco versiones que le
    parecían satisfactorias, sin embargo eventualmente descubría que contenían la
    misma falacia, cada vez disimulada más astutamente.\footcite[cf.~][p.~vii
    \S2]{M&PotM} 

    \subsubsection{Oxford: La Percepción y el fenomenalismo de Price}
    Otra inquietud ocuparía sus reflexiones. Esta vez, como fruto de su lectura de
    `The Nature of Belief' de Martin D'Arcy, se interesó por el tema de la
    percepción. 
    \begin{revision}
    Estaba segura de que veía objetos, como paquetes de cigarrillos o tazas o\ldots
    cualquier cosa más o menos sustancial servía. Pero estaba más bien concentrada
    en artefactos, como los demás objetos de la vida urbana, y los primeros ejemplos
    mas naturales que le llamaron la atención fueron `madera' y el cielo. Lo segundo
    le golpeó en el centro porque andaba diciendo dogmáticamente que uno debe
    conocer la categoría del objeto del cual uno hablaba ---si era un color o un tipo
    de material, por ejemplo; eso pertenecía a la lógica del termino que uno estaba
    usando. No podía ser una cuestión de descubrimiento empírico el que algo
    perteneciera a una categoría distinta. El cielo la detuvo.

    Durante años ocupaba su tiempo, en cafeterías, por ejemplo, mirando fijamente
    objetos, diciendose a sí misma: 'Veo un paquete. ¿Pero qué veo realmente? ¿Cómo
    puedo decir que veo algo más que una extensión amarilla?

    Fue en las clases de Wittgenstein que el pensamiento central ``Tengo esto, y
    defino `amarillo' como esto'' fue efectivamente atacado. 

    En una ocasión en estas clases Wittgenstein estaba discutiendo la interpretación
    del letrero\footcite[p.~86~\S198]{PI}, y estallo en mi que el modo en que vas según éste es la
    interpretación final. 

    En otra ocasión salí con ``Pero todavía quiero decir: <<Azul esta ahí>>''.
    Wittgenstein respondió: <<Déjame pensar qué medicina necesitas\ldots>> <<Supón
    que tenemos la palabra `painy' ``(dolorante/doloreño)'', como una palabra para la
    propiedad de ciertas superficies>>. La medicina fue efectiva.
    Si dolorante fuera una palabra posible para una cualidad secundaria, ¿no podría
    el mismo motivo conducirme a decir: Dolorante esta aquí que lo que me condujo a
    decir azul está aquí? Mi expresión no significaba que ``azul'' es el nombre de
    esta sensación que estoy teniendo, ni cambié a ese pensamiento. 

    Durante años se le escapaba el tiempo mirando fijamente distintos
    objetos y cuestionandose: <<Veo este objeto, pero ¿qué estoy viendo
    realmente?>>.\footcite[cf.~][p.~viii \S1]{M&PotM}
    \end{revision}


    Después de graduarse de `Sydenham High School' en 1937, se matriculó en `St.
    Hugh's College'. Allí cursó `Literae Humaniores', el programa clásico de Oxford,
    compuesto por literatura clásica, historia y filosofía. Muy pronto se interesó
    por las lecciones de H. H. Price sobre percepción y fenomenalismo. De todos los
    que escuchó en Oxford fue quién le inspiró mayor respeto, no porque estuviera de
    acuerdo con lo que decía, sino porque hablaba de lo que había que hablar. El
    único libro suyo que le pareció realmente bueno fue ``Hume's Theory of the
    External World'' y lo leyó sin interrupción de principio a
    fin. Fue Price quien despertó en ella un intenso interés por el capítulo de Hume
    sobre ``Del escepticismo con respecto a los sentidos''.\footcite[cf.~][p.~viii
    \S1]{M&PotM} El desempeño de Anscombe en las pruebas finales en `St. Hugh's'
    manifestó su clara preferencia por la filosofía. Fue premiada con honores de
    primera clase aún cuando su desempeño en las pruebas de historia fue bastante
    menos que espectacular\footcite[p.~3~\S1]{teichmann}.

    \subsubsection{En Cambrdige con Wittgenstein}
    ANSCOMBE LLEGÓ A CAMBRIDGE EN M42 W. OFRECIA CLASES LOS SÁBADOS Y TRABAJABA EN
    GUY'S. DESDE E43 HASTA E44 NO DIO CLASES. EN M44 EMPEZÓ OTRA VEZ. EN L45 EMPEZÓ
    A TRABAJAR MATERIAL CORRESPONDIENTE A 189-421 PI!!!!!!

    1. Wittgenstein está en época de transición.
    \begin{verbatim}
    Philosophical Investigations:
    --Undertake an investigation, leading, not to the construction of new and
    surprising theories or explanations, but the examination of our life with
    language. This is a grammatical investigation PI~\S90 
    --The ideas of explanation and discovery are misleading and inappropiate when
    applied to questions like: what is meaning?
    --We feel as if we had to repair a spider web with our fingers PI~\s106
    --PI~\S129
    --By putting details together in the right way or by using a new analogy or
    comparison to prompt us to see our practice of using language in a new light, we
    find that we achieve the understanding that we thought would only come with the
    construction of an explanatory account. RFGB, p.30
    --Philosopher's questions must be treated like an illness is treated. PI~\S133
    and \S255.
    --The aim of grammatical investigations is perspicious representation PI~\S122
    --Meaning is use.
    --The question of a philosopher is: how do I go about this?
    \end{verbatim}


    \begin{revision}
    What marks the transition from early to later Wittgenstein can be summed up as
    the total rejection of dogmatism, i.e., as the working out of all the
    consequences of this rejection. The move from the realm of logic to that of
    ordinary language as the center of the philosopher's attention; from an emphasis
    on definition and analysis to ‘family resemblance’ and ‘language-games’; and
    from systematic philosophical writing to an aphoristic style—all have to do with
    this transition towards anti-dogmatism in its extreme. It is in the
    Philosophical Investigations that the working out of the transitions comes to
    culmination. Other writings of the same period, though, manifest the same
    anti-dogmatic stance, as it is applied, e.g., to the philosophy of mathematics
    or to philosophical psychology.
    \end{revision}


    \begin{revision}
    Philosophical Investigations was published posthumously in 1953. It was edited
    by G. E. M. Anscombe and Rush Rhees and translated by Anscombe. It comprised two
    parts. Part I, consisting of 693 numbered paragraphs, was ready for printing in
    1946, but rescinded from the publisher by Wittgenstein. Part II was added on by
    the editors, trustees of his Nachlass. 
    \end{revision}

    \begin{revision}
    “For a large class of cases of the employment of the word ‘meaning’—though not
    for all—this way can be explained in this way: the meaning of a word is its use
    in the language” (PI 43). This basic statement is what underlies the change of
    perspective most typical of the later phase of Wittgenstein's thought: a change
    from a conception of meaning as representation to a view which looks to use as
    the crux of the investigation. 
    \end{revision}

    2. La metodología terapéutica y franca de Wittgenstein fue liberadora
    \begin{revision}


    En 1941 Anscombe se graduó de St. Hugh's College en Oxford y el siguiente año se
    trasladó a Cambridge para sus estudios de posgrado en Newnham College. Cuando
    Wittgenstein regresó a Cambridge en 1944 Anscombe asistió a sus lecciones con
    entusiasmo. Incluso cuando se le concedió una beca de investigación en
    Somerville College en 1946 y regresó a Oxford, todavía durante aquel año y el
    siguiente, viajaba una vez a la semana a Cambridge para encontrarse con
    Wittgenstein.  

    El método terapeútico de Wittgenstein tuvo éxito en liberarla de confusiones
    filosóficas donde otras metodologíás mas teoréticas habían fallado. En sus
    estudios en St. Hugh's escuchaba a Price.....
    \end{revision}




    %El Tractatus Logico-Philosophicus fue publicado en el 1922 y ciertamente causó
    %un impacto en el modo de hacer filosofía. Anscombe emplea la idea de ``corte''
    %de Boguslaw Wolniewicz para describir el cambio causado por Wittgenstein. Este
    %corte efectuado en la historia de la filosofía por el Tractatus fue atestiguado
    %por un filósofo austriaco que describió a Anscombe el efecto cataclísmico
    %suscitado narrando cómo profesores largamente consolidados se deshacían de sus
    %viejos libros; la tarea consistía ahora en hacer filosofía en el modo indicado
    %por el Tractatus y el primer paso era, ciertamente, entenderlo.
    %\footcite[p.181]{twocuts} 


    %Este modo de criticar una proposición desvelando que no expresa un pensamiento
    %verdadero ilustra los principios propuestos en el \emph{Tractatus} y recuerda
    %una de sus tesis más conocidas: 

    %En el prefacio de las Investigaciones Filosóficas, con fecha de enero de 1945
    %Wittgenstein dice que los pensamientos que publica en el libro son el
    %precipitado de invetigaciones filosóficas que le han ocupado durante los pasados
    %16 años. En enero 1929 Wittgenstein estaba regresando a Cambridge.

    %En 1953 fue publicado el texto de las investigaciones filosóficas

    %En 1982 Anscombe afirma que el con el segundo corte causado por las
    %investigaciones filosóficas el proceso analogo al ocurrido con el tractatus
    %apenas ha comenzado.

    %El 29 de abril de 1951 murió en Cambridge. 

    \subsection{Wittgenstein y la fe}
    \todo{En casa de Anscombe, hablando de la fe}
    \todo{From IWT: la verdad de la teoría de la imagen sería el fin de la teología
        natural} 
    \todo{Inquietud respecto del esfuerzo de explicar racionalmente la fe} 
    \todo{Necesidad de contexto}

    \begin{revision}
    Es una gran bendición para mi poder trabajar hoy. ¡Pero cuán fácilmente olvido
    todas mis bendiciones!
    Estoy leyendo: ``Y ningún hombre puede decir Jesús es el Señor, sino el Espíritu
    Santo.''(1Co 3) Y es cierto: Yo no puedo llamarlo \emph{Señor}; porque eso no me
    dice absolutamente nada. Sí podría llamarlo 'el ejemplo por excelencia', 'Dios'
    incluso o quizás: puedo entenderlo cuando es llamado de ese modo; pero Yo no
    puedo pronunciar la palabra ``Señor'' significativamente. \emph{Porque yo no
    creo} que el vendrá a juzgarme; porque \emph{eso} no me dice nada. Y sólo me
    diría algo si yo viviera de un modo considerablemente distinto.

    ¿Qué me hace inclinarme incluso a mi a creer en la resurrección de Cristo?
    Entretengo la idea por así decirlo. ---Si él no ha resucitado de los muertos,
    entonces se descompuso en la tumba como cualquier otro ser humano. \emph{Esta
    muerto y descompuesto.} En ese caso es un maestro, como cualquier otro y
    entonces ya no puede \emph{ayudar} más; y estamos una vez más huérfanos y solos.
    Y tengo que arreglármelas con la sabiduría y la especulación. Es como si
    estuvieramos en un infierno, en el que solo podemos soñar y estamos dejados
    fuera del cielo, atrapados bajo el techo, diriamos. Pero si REALMENTE voy a ser
    redimido, ---necesito \emph{certeza}--- no sabiduría, sueños, especulación--- y
    esta certeza es la fe. Y fe es fe en lo que mi \emph{corazón}, mi \emph{alma},
    necesita, no mi intelecto especulativo. Pues mi alma, con sus pasiones, con su
    carne y sangre, diría, tiene que ser redimida, no mi mente abstracta. Quizás uno
    podría decir: Sólo el \emph{amor} puede creer la Resurrección. O: es el
    \emph{amor} lo que cree la Resurrección. Uno puede decir: el amor redentor cree
    incluso en la Resurrección; se sostiene firme incluso hasta la Resurrección. Lo
    que lucha con la duda es, por decirlo de algún modo, la redención. Sostenerse
    firmemente en esto tiene que ser mantenerse firme en esta creencia. Así esto
    significa: primero se redimido y sujétate firmemente de tu redención (sostente en tu
    redención) --- entonces veras que a lo que te estás sujetando es a esta
    creencia. Así que esto sólo puede ocurrir si ya no te sujetas de esta tierra,
    sino que te suspendes desde el cielo. Entonces \emph{todo} es distinto y 'no
    será sorpresa' el que puedas hacer entonces lo que ahora no puedes. (Es verdad
    que alguien que está suspendido se ve como alguien que está de pie, pero la
    interacción de fuerzas dentro de él es sin embargo una completamente distinta, y
    de ahí que sea capaz de hacer cosas bastante distintas de las que puede hacer
    alguien que está de pie). (Culture and Value p.38-39 MS 120 108 c: 12.12.1937)
    \end{revision}
