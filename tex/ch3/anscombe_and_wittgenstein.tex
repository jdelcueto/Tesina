% Hi-lock: (("\\\\todo{.*}" (0 (quote hi-green) prepend)))  

%SECCIÓN 1: ANSCOMBE Y WITTGENSTEIN

\section{Anscombe y Wittgenstein}

%PRIMERA CUESTIÓN: ILUSTRACIÓN DEL MÉTODO DE WITTGENSTEIN EN SUS LECCIONES CON
%ANSCOMBE
\subsection{El método de Wittgenstein}
\ifdraft{\subsubsection{Ilustración del Método de Wittgenstein}}{}

\todo{Esta anécdota ofrece un tono testimonial y genera preguntas que serán
    tratadas en adelante}
En cierta ocasión Wittgenstein recibió a Anscombe con una pregunta: <<¿Por qué la
gente dice que era natural pensar que el sol giraba alrededor de la tierra en
lugar de que la tierra rotaba en su eje?>> Elizabeth contestó: <<Supongo que
porque se veía como si el sol girara alrededor de la tierra.>> <<Bueno\ldots>>,
añadió Wittgenstein, <<¿cómo se hubiera visto si se hubiera \emph{visto} como si
la tierra rotara en su propio eje?>> A esta pregunta Anscombe reaccionó
extendiendo las manos delante de ella con las palmas hacia arriba y,
levantándolas desde sus rodillas con un movimiento circular, se inclinó hacia
atrás asumiendo una expresión de mareo. <<¡Exactamente!>> exclamó Wittgenstein.
\footcite[cf.][151]{IWT}   

%SEGUNDA CUESTIÓN: EXPLICACIÓN DE LA ILUSTRACIÓN
\ifdraft{\subsubsection{Explicación breve de la Ilustración}}{}

Anscombe se percató del problema; la pregunta de Wittgenstein había puesto en
evidencia que hasta aquél momento no había ofrecido ningún significado relevante
para su expresión \emph{``se veía como si''} en su respuesta \emph{``se veía
    como si el sol girara alrededor de la tierra''}.  En ocasiones como esta la
discusión con Wittgenstein llevaba a Anscombe a afirmaciones para las cuales no
podía ofrecer mejor significado que los sugeridos por concepciones ingenuas. Una
concepción así no es otra cosa que ausencia de pensamiento, pero su falta de
significado no es evidente, sino que requiere de la fuerza de un `Copérnico'
para ponerla en cuestión efectivamente.\footcite[cf. 151]{IWT} 

%TERCERA CUESTIÓN: DE LA ILUSTRACIÓN AL TRACTATUS
\ifdraft{\subsubsection{Desde la Ilustración hacia el desarrollo del Tractatus}}{}

\todo{Con este párrafo nos remitimos desde la metodología a la elaboración del
    Tractatus, para llegar a los puntos fundamentales de la obra}
Para Ludwig Wittgenstein mostrar que la persona no ha provisto significado (o
referencia) para ciertos signos en sus proposiciones era el método general
adecuado de discutir los problemas filosóficos.\footcite[cf. p. 151]{IWT} Creía
que el camino que lleva a formular estos problemas está frecuentemente trazado
por la mala comprensión de la lógica de nuestro lenguaje y el modo de aclarar
esta confusión consistía en identificar en el lenguaje el límite de lo que
expresa pensamiento; lo que queda al otro lado de esta frontera es simplemente
sinsentido. En otras palabras: \citalitinterlin{Lo que
    \todo{traducción difícil. \emph{``What can be said at all''}} 
    siquiera puede ser dicho puede ser dicho claramente; y de lo que uno no
    puede hablar, de eso, uno debe guardar silencio}. 
\footcite[prefacio]{tractatus}
Con esta expresión  Wittgenstein resumía el significado del libro que recoge su
esfuerzo para resolver este problema de la filosofía: el \emph{'Tractatus
    Logico\=/Philosophicus'}. 

%Elaboración del Tractatus
%En el 14 empezó la guerra, en el 15 W. escribió a R. con sus intenciones de
%hacer un tratado. En el 18 lo acabó. En el 19 envió el manuscrito a R. En el 22
%lo publicó.
\subsection{El gran tratado de Wittgenstein}
\ifdraft{\subsubsection{De Manchester a Cambridge}}{}

\todo{El propósito de recorrer el desarrollo que lleva al Tractatus es ofrecer
    un trasfondo a los puntos que resaltamos más adelante.}
Los primeros esfuerzos de Wittgenstein para escribir una obra sobre filosofía
habían comenzado en 1911. En otoño de ese año en lugar de continuar sus estudios
de ingeniería en Manchester, determinó irse a Cambridge donde Bertrand Russell
ofrecía sus lecciones. Su hermana le describe en esa época:   
\citalitlar{Fue repentinamente agarrado por la filosofía --es decir, por la
    reflexión en problemas filosóficos-- tan violentamente y tan en contra de su
    voluntad que sufrió severamente por la doble y conflictiva llamada interior
    y se veía a sí mismo como roto en dos. Una de muchas transformaciones por las
    que pasaría en su vida había venido sobre él y le estremeció hasta lo más
    profundo. Estaba concentrado en escribir un trabajo filosófico y finalmente
    se decidió en mostrar el plan de su obra al Profesor Frege en Jena, quien
    había discutido preguntas similares. [\ldots] Frege alentó a Ludwig en su
    búsqueda filosófica y le aconsejó que fuera a Cambridge como alumno del
    Profesor Russell, cosa que Ludwig ciertamente hizo.\footcite[p. 73]{mcguinness}}

Al finalizar el primer término de lecciones con Russell, Wittgenstein aún no
estaba seguro de abandonar la ingeniería por la filosofía, se cuestionaba si
verdaderamente tenía talento para ella, y consultó a Russell al respecto. El
profesor le pidió que escribiera algo para ayudarle a hacer un juicio. En enero
del 1912 Wittgenstein regresó a Cambridge con un manuscrito que convenció a
Russell de su gran capacidad y decidió alentar a Ludwig a continuar
dedicándose a la filosofía. El apoyo de Russell fue crucial para Wittgenstein y
durante el siguiente término puso tanto empeño en su estudio que para el final
de éste Russell decía que había aprendido todo lo que él podía
enseñarle.\footcite[cap. 3 loc 865]{monk} 

\ifdraft{\subsubsection{A Noruega a Resolver los problemas de la lógica}}{}
Después de una temporada en Cambridge llena de eventos y desarrollos
Wittgenstein anunció en septiembre de 1913 sus planes de retirarse para
dedicarse exclusivamente a trabajar en resolver los problemas fundamentales de
la lógica. Su idea era irse a Noruega, a algún lugar apartado, ya que pensaba
que en Cambridge las interrupciones obstaculizarían su trabajo.\footcite[cap. 4
loc 1844]{monk} 

\ifdraft{\subsubsection{La Gran Guerra}}{}
El trabajo en Noruega fue accidentado y para el verano de 1914, cuando regresó a
Vienna para un receso, no había terminado su obra.\footcite[cap. 5 loc
2154]{monk} Había planificado regresar a Noruega después del verano, pero en julio de
aquel año se desató la Primera Guerra Mundial. El 7 de agosto de 1914
Wittgenstein se enlistó como voluntario en la guerra. Allí continuaría su 
trabajo filosófico. 

El 22 de octubre de 1915 Wittgenstein escribió a Russell
desde el taller de artillería en Sokal, al norte de Lemberg, con lo que sería
una primera versión de su libro.\footcite[cf. p.84]{cambridgeletters} Cuatro años más
tarde, el 13 de marzo, escribía a Russell desde Cassino donde se hallaba como
prisionero de guerra en un campamento italiano\footcite[cf. p.268]{mcguinness}:
\citalitlar{He escrito un libro llamado ``Logisch-Philosophische Abhandlung''
    que contiene todo mi trabajo de los últimos seis años. Creo que finalmente he
    resuelto todos nuestros problemas. Esto puede sonar arrogante, pero no puedo
    evitar creerlo. Terminé el libro en agosto de 1918 y dos meses más tarde fui
hecho 'Prigioniere'.\footcite[p.89]{cambridgeletters}}

\ifdraft{\subsubsection{Aire de Misticismo}}{}
En junio de aquel año logró enviar el manuscrito del libro a Russell por medio
de John Maynard Keynes quien intervino con las autoridades italianas para
permitir el envío seguro del texto\footcite[p.90 y 91]{cambridgeletters}. El 26
de agosto de 1919 fue oficialmente liberado de sus funciones
militares\footcite[p.277]{mcguinness} y en diciembre finalmente pudo encontrarse
con Russell en la Haya. De aquel encuentro Russell escribe:
\citalitlar{Había sentido un sabor a misticismo en su libro, pero me quedé
    asombrado cuando vi que se ha convertido en un completo místico. Lee a gente
    como Kierkergaard y Angelus Silesius, y ha contemplado seriamente el
    convertirse en un monje. Todo comenzó con ``Las variedades de la experiencia
    religiosa'' de William James y creció durante el invierno que pasó solo en
    Noruega antes de la guerra cuando casi se había vuelto loco. Luego, durante
    la guerra, algo curioso ocurrió. Estuvo de servicio en el pueblo de Tarnov
    en Galicia, y se encontró con una librería que parecía contener solamente
    postales. Sin embargo, entró y encontró que tenían un sólo libro: Los
    Evangelios abreviados de Tolstoy. Compró el libro simplemente porque no
    había otro. Lo leyó y releyó y desde entonces lo llevaba siempre consigo,
    estando bajo fuego y en todo momento. Aunque en su conjunto le gusta menos
    Tolstoy que Dostoeweski. Ha penetrado profundamente en místicos modos de
    pensar y sentir, aunque pienso que lo que le gusta del misticismo es su
    poder para hacerle dejar de pensar. No creo que realmente se haga monje, es
    una idea, no una intención. Su intención es ser profesor. Repartió todo su
    dinero entre sus hermanos y hermanas, pues encuentra que las posesiones
    terrenales son una carga. \footcite[p. 112]{cambridgeletters}}

\ifdraft{\subsubsection{En busca de una experiencia religiosa}}{}
Cuando Wittgenstein se enlistó en el ejercito para la guerra en 1914 tenía
motivaciones más complejas que la defensa de su patria.\footcite[loc2276]{monk}
Sentía que, de algún modo, la experiencia de encarar la muerte le haría mejor
persona. Había leído sobre el valor espiritual de confrontarse con la muerte en
``Las variedades de la experiencia religiosa'':
\citalitlar{No importa cuales sean las fragilidades de un hombre, si estuviera
    dispuesto a encarar la muerte, y más aún si la padece heroicamente, en el
    servicio que éste haya escogido, este hecho le consagra para
    siempre.\footcite[loc 2295]{monk}}

Wittgenstein esperaba esta experiencia religiosa de la guerra.
\citalitinterlin{Quizás}, escribía en su diario, \citalitinterlin{La cercanía de
    la muerte traerá luz a la vida. Dios me ilumine.}\footcite[loc2295]{monk}
La guerra había coincidido con esta época en la que el deseo de convertirse en
una persona diferente era más fuerte aún que su deseo de resolver los problemas
fundamentales de la lógica.\footcite[loc2305]{monk}

\ifdraft{\subsubsection{La Principal Contienda}}{}
Esta transformación sorprendió a Russell en aquel encuentro en la Haya, pero
además fue motivo de confusión en la tarea de entender el Tractatus. Cuando
Russell recibió el manuscrito en agosto escribió a Wittgenstein cuestionando
algunos puntos difíciles del texto. En su carta observaba: 
\citalitlar{Estoy convencido de que estás en lo correcto en tu principal
    contienda, que las proposiciones lógicas son tautologías, las cuales no son
    verdad en el mismo modo que las proposiciones
    sustanciales.\footcite[p.96]{cambridgeletters}}

Esta interpretación del texto se ajusta bien a la importancia que había tenido
esta cuestión en las discusiones entre Russell y Wittgenstein. Así lo expresaba
Russell en ``Introducción a la Filosofía Matemática'' publicado en mayo de aquel
año: 
\citalitlar{
    \todo{The importance of “tautology” for a definition of
    mathematics was pointed out to me by my former pupil Ludwig Wittgenstein,
    who was working on the problem. I do not know whether he has solved it, or
    even whether he is alive or dead.} 
    La importancia de la ``tautología'' para una definición de las
    matemáticas me fue señalada por mi ex-alumno Ludwig Wittgenstein, quien
    estaba trabajando en el problema. No sé si lo ha resuelto, o siquera si está
    vivo o muerto.\footcite[p.205]{introtomathphi}} 

Sin embargo para el Tractatus la cuestión sobre las proposiciones lógicas como
tautologías no es ya el tema principal, sino que enfatiza otra cuestión, así
corrige Wittgenstein en su respuesta a la carta de Russell:
\citalitlar{Ahora me temo que realmente no has captado mi principal contienda,
    para lo cual todo el asunto de las proposiciones lógicas es sólo corolario.
    El punto principal es la teoría sobre lo que puede ser expresado por
    proposiciones --es decir, por el lenguaje-- (y, lo que viene a ser lo mismo,
    aquello que puede ser pensado) y lo que no puede ser expresado por medio de
    proposiciones, sino solamente mostrado; lo cual, creo, es el problema
    cardinal de la filosofía\ldots \footcite[p. 98]{cambridgeletters}}

Esta respuesta de Wittgenstein no solo pone de manifiesto su cambio de enfoque,
sino que ofrece una clave para introducirse en su obra. 

%CUARTA CUESTIÓN: LA ``DOCTRINA'' DEL TRACTATUS
%1. La filosofía como actividad
%2. El pensamiento como representación
%3. Los polos de verdad y falsedad de las proposiciones
%4. La diferencia ente decir y mostrar
\subsection{Las elucidaciones del Tractatus}
\todo{Este párrafo resume los cuatro puntos del Tractatus que se desglosarán en
    los próximos párrafos} 
Desde las proposiciones principales del Tractatus
queda claro que el tema central del libro es la conexión entre el lenguaje, o el
pensamiento, y la realidad. 
\todo{1.Filosofía como actividad}Es en este nexo donde la actividad filosófica
ha de buscar esclarecer el pensamiento. 
\todo{2.El pensamiento como representación}La tesis básica sobre esta relación 
consiste en que las proposiciones, o su equivalente en la mente, son imágenes de
los hechos. 
\todo{3.Los polos de verdad y falsedad}La proposición es la misma imagen tanto
si es cierta como si es falsa, es decir, es la misma imagen sin importar que lo
que se corresponde a ésta es el caso que es cierto o no. El mundo es la
totalidad de los hechos, es decir, de lo equivalente en la realidad a las
proposiciones verdaderas. 
\todo{4.La distinción entre el decir y el mostrar}Sólo las situaciones que
pueden ser plasmadas en imágenes pueden ser afirmadas en proposiciones.
Adicionalmente hay mucho que es inexpresable, lo cual no debemos intentar
enunciar, sino más bien contemplar sin palabras.\footcite[cf. p.19]{IWT}  

\ifdraft{\subsubsection{La filosofía como actividad}}{}

Considerada la relación entre la realidad, el pensamiento y el lenguaje de esta
manera, la filosofía queda definida como la actividad cuyo objeto es la
clarificación lógica de los pensamientos. \footcite[4.112 p. 52]{tractatus} El
problema de muchas de las proposiciones y preguntas que se han escrito acerca de
asuntos filosóficos no es que sean falsas, sino carentes de significado.
Wittgenstein continúa:   
\citalitlar{4.003 En consecuencia no podemos dar respuesta a preguntas de este
    tipo, sino exponer su falta de sentido. Muchas cuestiones y proposiciones de
    los filósofos resultan del hecho de que no entendemos la lógica de nuestro
    lenguaje. (Son del mismo genero que la pregunta sobre si lo Bueno es más o
    menos idéntico a lo Bello). Y así no hay que sorprenderse ante el hecho de
    que los problemas más profundos realmente no son problemas. \footcite[4.003
    p. 45]{tractatus}} 

Según esto el Tractatus no pretende ser cuerpo doctrinal compuesto de
proposiciones filosóficas, sino que busca ofrecer `elucidaciones' que sirven
como etapas escalonadas y transitorias que al ser superadas conducen a ver el
mundo correctamente. Este esfuerzo hace de pensamientos opacos e indistintos
unos claros y con límites bien definidos. \footcite[cf. 4.112 y 6.54]{tractatus}
La posibilidad de llegar a una visión clara del mundo es fruto de la posibilidad
de lograr aclarar la lógica del lenguaje. El lenguaje, a su vez, está
compuesto de la totalidad de las proposiciones, y éstas, cuando tienen sentido,
representan el pensamiento. \footcite[cf. 4 y 4.001]{tractatus} Sin embargo, el
mismo lenguaje que puede expresar el pensamiento puede velarlo: 

\citalitlar{4.002 El lenguaje disfraza el pensamiento; de tal manera que de la
    forma externa de sus ropajes uno no puede inferir la forma del pensamiento
    que estos revisten, porque la forma externa de la vestimenta esta elaborada
    con un propósito bastante distinto al de favorecer que la forma del cuerpo
    sea conocida.}

El intento de llegar desde el lenguaje al pensamiento por medio de las
proposiciones con significado es el esfuerzo de conocer una imagen de la
realidad. El pensamiento es la imagen lógica de los hechos, en él se contiene la
posibilidad del estado de las cosas que son pensadas y la totalidad de los
pensamientos verdaderos es una imagen del mundo. \footcite[cf. 3 y
3.001]{tractatus}

\ifdraft{\subsubsection{El pensamiento como representación}}{}
\todo{3.4.2 El pensamiento como representación}El pensamiento es representación
del mundo por la identidad existente entre la posibilidad de la estructura de
una proposición y la posibilidad de la estructura un hecho: 
\citalitlar{Los objetos --que son simples-- se combinan en situaciones
    elementales. El modo en el que se sujetan juntos en una situación tal es su
    estructura. Forma es la posibilidad de esa estructura. No todas las
    estructuras posibles son actuales: una que es actual es un `hecho
    elemental'. Nosotros formamos imágenes de los hechos, de hechos posibles
    ciertamente, pero algunos de ellos son actuales también. Una imagen consiste
    en sus elementos combinados en un modo específico. Al estar así presentan a
    los objetos denominados por ellos como combinados específicamente en ese
    mismo modo. La combinación de los elementos de la imagen --la combinación
    siendo presentada-- se llama su estructura y su posibilidad se llama la
    forma de representación de la imagen.   
    Esta `forma de representación' es la posibilidad de que las cosas están
    combinadas como lo están los elementos de la imagen. \footcite[p.
    171]{simplicity}}  

De esta identidad entre ambas posibilidades se deriva que las estructuras de las
realidades son estructuras lógicas:
\citalitlar{2.18 Lo que toda representación, de una forma cualquiera, debe tener
    en común con la realidad, de manera que pueda representarla --cierta o
    falsamente-- de algún modo, es su forma lógica, esto es, la forma de la
    realidad.\footcite[p.34]{tractatus}}  

\ifdraft{\subsubsection{Los polos de verdad y flasedad en las proposiciones}}{}
\todo{3.4.3 Los polos de verdad y falsedad de las proposiciones}El pensamiento
es la imagen lógica de la realidad, un pensamiento es una proposición con
sentido y la totalidad de las proposiciones son el lenguaje. Una proposición es
una imagen de la realidad, es un modelo de la realidad como pensamos que es. La
imagen o representación de la realidad en el pensamiento es distinta de la
proposición ya que no afirma nada, mientras que en la proposición se dice de
algo que es el caso. Las proposiciones pueden ser ciertas o falsas siendo
imágenes de la realidad. La proposición muestra como las cosas están si es cierta
y ésta dice que las cosas están de esa manera.
4.063

\ifdraft{\subsubsection{La distinción entre el decir y el mostrar}}{}
\todo{3.4.4 La diferencia entre decir y mostrar}La relación entre el pensamiento
y la realidad expresada en las proposiciones se refiere a lo que puede ser
dicho. Sin embargo hay mucho que no puede ser dicho por medio de proposiciones,
pero queda mostrado en el decir las proposiciones para expresar lo que puede ser
dicho. La forma lógica no puede expresarse desde el lenguaje, pues es la forma
del lenguaje mismo, se hace manifiesta en el lenguaje, tiene que ser mostrada.
Similarmente, las verdades éticas y religiosas, aunque no expresables, se
manifiestan a sí mismas en la vida. En esta distinción se encuentra la conexión
entre los pensamientos de Wittgenstein sobre lógica y sus reflexiones sobre el
significado de la vida.

Las lecciones con Wittgenstein eran directas y con franqueza. Esta metodología
carente de cualquier parafernalia era inquietante para algunos, inspiradora para
otros, pero tremendamente liberadora para Anscombe. \footcite[loc 9853 Chapter
4, Section 24, para. 5]{monk}

En 1941 Anscombe se graduó de St. Hugh's College en Oxford y el siguiente año se
trasladó a Cambridge para sus estudios de posgrado en Newnham College. Cuando
Wittgenstein regresó a Cambridge en 1944 Anscombe asistió a sus lecciones con
entusiasmo. Incluso cuando se le concedió una beca de investigación en
Somerville College en 1946 y regresó a Oxford, todavía durante aquel año y el
siguiente, viajaba una vez a la semana a Cambridge para encontrarse con
Wittgenstein.  

El método terapeútico de Wittgenstein tuvo éxito en liberarla de confusiones
filosóficas donde otras metodologíás mas teoréticas habían fallado. En sus
estudios en St. Hugh's escuchaba a Price.....

\todo{Right stuff...}

\todo{I always hated phenomenalism...}

\todo{Anscombe se inició en la filosofía como una ardua actividad.}

El Tractatus Logico-Philosophicus fue publicado en el 1922 y ciertamente causó
un impacto en el modo de hacer filosofía. Anscombe emplea la idea de ``corte''
de Boguslaw Wolniewicz para describir el cambio causado por Wittgenstein. Este
corte efectuado en la historia de la filosofía por el Tractatus fue atestiguado
por un filósofo austriaco que describió a Anscombe el efecto cataclísmico
suscitado narrando cómo profesores largamente consolidados se deshacían de sus
viejos libros; la tarea consistía ahora en hacer filosofía en el modo indicado
por el Tractatus y el primer paso era, ciertamente, entenderlo.
\footcite[p.181]{twocuts} 


Este modo de criticar una proposición desvelando que no expresa un pensamiento
verdadero ilustra los principios propuestos en el \emph{Tractatus} y recuerda
una de sus tesis más conocidas: 

\citalitlar{6.53 El método correcto para la filosofía sería este. No decir nada
    excepto lo que pueda ser dicho, esto es, proposiciones de la ciencia
    natural, es decir, algo que no tiene nada que ver con la filosofía: y luego
    siempre, cuando alguien quiera decir algo metafísico, demostrarle que no ha
    logrado dar significado a ciertos signos en sus proposiciones. Este método
    sería insatisfactorio para la otra persona --no tendría la impresión de que
    le estuviéramos enseñando filosofía-- pero este método sería el único
    estrictamente correcto. \footcite[p. 107--108]{tractatus}}


En el prefacio de las Investigaciones Filosóficas, con fecha de enero de 1945
Wittgenstein dice que los pensamientos que publica en el libro son el
precipitado de invetigaciones filosóficas que le han ocupado durante los pasados
16 años. En enero 1929 Wittgenstein estaba regresando a Cambridge.

En 1953 fue publicado el texto de las investigaciones filosóficas

En 1982 Anscombe afirma que el con el segundo corte causado por las
investigaciones filosóficas el proceso analogo al ocurrido con el tractatus
apenas ha comenzado.

El 29 de abril de 1951 murió en Cambridge. 
