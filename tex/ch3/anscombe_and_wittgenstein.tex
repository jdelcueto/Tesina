% Hi-lock: (("\\\\todo{.*}" (0 (quote hi-green) prepend)))  

%SECCIÓN 1: ANSCOMBE Y WITTGENSTEIN
\section{3.1 Anscombe y Wittgenstein}

%PRIMERA CUESTIÓN: ILUSTRACIÓN DEL MÉTODO DE WITTGENSTEIN EN SUS LECCIONES CON
%ANSCOMBE
\todo{3.1.1 Ilustración: El Método de W.}En una ocasión Wittgenstein
recibió a Anscombe con la pregunta: <<¿Por qué la gente dice que era natural
pensar que el sol giraba alrededor de la tierra en lugar de que la tierra rotaba
en su eje?>> Elizabeth contestó: <<Supongo que porque se veía como si el sol
girara alrededor de la tierra.>> <<Bueno\ldots>>, añadió Wittgenstein, <<¿cómo
se hubiera visto si se hubiera \emph{visto} como si la tierra rotara en su
propio eje?>> A esta pregunta Anscombe reaccionó extendiendo las manos delante
de ella con las palmas hacia arriba y, levantándolas desde sus rodillas con un
movimiento circular, se inclinó hacia atrás asumiendo una expresión de mareo.
<<¡Exactamente!>> exclamó Wittgenstein. \footcite[cf.][151]{IWT} 

%SEGUNDA CUESTIÓN: EXPLICACIÓN DE LA ILUSTRACIÓN
\todo{3.2 Explicación de la Ilustración}Anscombe se percató del problema; la
pregunta de Wittgenstein había puesto en evidencia que hasta aquél momento no
había ofrecido ningún significado relevante para su expresión \emph{``se veía como
si''} en su  respuesta \emph{``se veía como si el sol girara alrededor de la tierra''}. 
En ocasiones como esta la discusión con Wittgenstein llevaba a Anscombe a
afirmaciones para las cuales no podía ofrecer mejor significado que los
sugeridos por concepciones ingenuas. Una concepción así no es otra cosa que
ausencia de pensamiento, pero su falta de significado no es evidente, sino que
requiere de la fuerza de un `Copérnico' para ponerlas en cuestión
efectivamente.\footcite[cf. 151]{IWT} 

%TERCERA CUESTIÓN: DE LA ILUSTRACIÓN AL TRACTATUS

%En el 14 empezó la guerra, en el 15 W. escribió a R. con sus intenciones de
%hacer un tratado. En el 18 lo acabó. En el 19 envió el manuscrito a R. En el 22
%lo publicó.

\todo{3.3 De la ilustración al tractatus}Para Wittgenstein mostrar que la
persona no ha provisto significado (o referencia) para ciertos signos en sus
proposiciones era el método general adecuado de discutir los problemas
filosóficos.\footcite[cf. p. 151]{IWT} Creía que el camino que lleva a formular
estos problemas está frecuentemente trazado por la mala comprensión de la lógica
de nuestro lenguaje y el modo de aclarar esta confusión consistía en identificar
en el lenguaje el límite de lo que expresa pensamiento; lo que queda al otro
lado de éste es simplemente sinsentido. En otras palabras:
\citalitinterlin{Lo que 
    \todo{traducción difícil. \emph{``What can be said at all''}} 
    siquiera puede ser dicho puede ser dicho claramente; y de lo que uno no
    puede hablar, de eso, uno debe guardar silencio}. 
\footcite[prefacio]{tractatus}
Con estas palabras que se han hecho célebres Wittgenstein resumió el significado
del libro que recoge su esfuerzo para resolver este problema de la filosofía: el
\emph{'Tractatus Logico-Philosophicus'}.


%CUARTA CUESTIÓN: LA ``DOCTRINA'' DEL TRACTATUS
%1. La filosofía como actividad
%2. El pensamiento como representación
%3. Los polos de verdad y falsedad de las proposiciones
%4. La diferencia ente decir y mostrar
\todo{3.4 La ``Doctrina'' del Tractatus}
Los primeros esfuerzos de Wittgenstein para escribir una obra sobre filosofía
habían comenzado tan temprano como en 1911. En otoño de ese año en lugar de
continuar sus estudios de ingeniería en Manchester, determinó irse a
Cambridge donde Russell ofrecía sus lecciones. Su hermana le describe en esa
época: 
\citalitlar{Fue repentinamente agarrado por la filosofía --es decir, por la
    reflexión en problemas filosóficos-- tan violentamente y tan en contra de su
    voluntad que sufrió severamente por la doble y conflictiva llamada interior
    y se veía a sí mismo como roto en dos. Una de muchas transformaciones por las
    que pasaría en su vida había venido sobre él y le estremeció hasta lo más
    profundo. Estaba concentrado en escribir un trabajo filosófico y finalmente
    se decidió en mostrar el plan de su trabajo al Profesor Frege en Jena, quien
    había discutido preguntas similares. [\ldots] Frege alentó a Ludwig en su
    búsqueda filosófica y le aconsejó que fuera a Cambridge como alumno del
    Profesor Russell, cosa que Ludwig ciertamente hizo.\footcite[p. 73]{mcguinness}}

Al finalizar el primer término de lecciones con Russell, Wittgenstein aún no
estaba seguro de abandonar la ingeniería por la filosofía, se cuestionaba si
verdaderamente tenía talento para ella, y consultó a Russell al respecto. El
cprofesor le pidió que escribiera algo para ayudarle a hacer un juicio. En enero
del 1912 Wittgenstein regresó a Cambridge con un manuscrito que convenció a
Russell de su gran capacidad y le alentó a continuar dedicándose a la filosofía.
El apoyo de Russell fue crucial para Wittgenstein y durante el siguiente término
puso tanto empeño en su estudio que para el final de éste Russell decía que
había aprendido todo lo que él podía enseñarle.\footcite[cap. 3 loc 865]{monk}

Después de una temporada en Cambridge llena de eventos y desarrollos
Wittgenstein anunció en septiembre de 1913 sus planes de retirarse para
dedicarse exclusivamente a trabajar en resolver los problemas fundamentales de
la lógica. Su idea era irse a Noruega, a algún lugar apartado, ya que pensaba
que en Cambridge las interrupciones obstaculizarían su trabajo.\footcite[cap. 4
loc 1844]{monk} 

El trabajo en Noruega fue accidentado y para el verano de 1914 cuando regresó a
Vienna para un receso no había terminado su obra.\footcite[cap. 5 loc
2154]{monk} Su plan era regresar a Noruega después del verano, pero en julio de
aquel año se desató la Primera Guerra Mundial. El 7 de agosto de 1914
Wittgenstein se enlistó como voluntario en la guerra. Allí continuaría su 
trabajo filosófico. El 22 de octubre de 1915 Wittgenstein escribió a Russell
desde el taller de artillería en Sokal, al norte de Lemberg, con lo que sería
una primera versión de su libro.\footcite[cf. p.84]{cambridgeletters} Cuatro años más
tarde, el 13 de marzo escribía a Russell desde Cassino donde se hallaba como
prisionero de guerra en un campamento italiano\footcite[cf. p.268]{mcguinness}:
\citalitlar{He escrito un libro llamado ``Logisch-Philosophische Abhandlung''
    que contiene todo mi trabajo de los últimos seis años. Creo que finalmente he
    resuelto todos nuestros problemas. Esto puede sonar arrogante, pero no puedo
    evitar creerlo. Terminé el libro en agosto de 1918 y dos meses más tarde fui
hecho 'Prigioniere'.\footcite[p.89]{cambridgeletters}}

En junio de aquel año logró enviar el manuscrito del libro a Russell por medio
de J. M. Keynes quien intervino con las autoriadades italianas para permitir el
envío seguro del texto\footcite[p.90 y 91]{cambridgeletters}. El 26 de agosto
de 1919 fue oficialmente liberado de sus funciones
militares\footcite[p.277]{mcguinness} y en diciembre finalmente pudo encontrarse
con Russell en la Haya. De aquel encuentro Russell escribe:
\citalitlar{Había sentido un sabor a misticismo en su libro, pero me quede
    asombrado cuando vi que se ha convertido en un completo místico. Lee a gente
    como Kierkergaard y Angelus Silesius, y ha contemplado seriamente el
    convertirse en un monje. Todo comenzó con ``Las variedades de la experiencia
    religiosa'' de William James y creció durante el invierno que pasó solo en
    Noruega antes de la guerra cuando estaba casi loco. Luego durante la guerra
    algo curioso ocurrió. Estuvo de servicio en el pueblo de Tarnov en Galicia,
    y se encontró con una librería que parecía contener solamente postales. Sin
    embargo, entró y encontró que tenían un sólo libro: Los Evangelios
    abreviados de Tolstoy.\footcite[p. 112]{cambridgeletters}}




\todo{3.4.1 La filosofía como actividad}En su tratado Wittgenstein insiste en
la filosofía como una actividad cuyo objeto es la clarificación lógica de los
pensamientos. \footcite[4.112 p. 52]{tractatus} Esta labor atiende a la
naturaleza del problema de muchas de las proposiciones y preguntas que se han
escrito acerca de asuntos filosóficos: éstas no son falsas, sino carentes de
significado. Wittgenstein continúa: 

\citalitlar{4.003 En consecuencia no podemos dar respuesta a preguntas de este
    tipo, sino exponer su falta de sentido. Muchas cuestiones y proposiciones de
    los filósofos resultan del hecho de que no entendemos la lógica de nuestro
    lenguaje. (Son del mismo genero que la pregunta sobre si lo Bueno es más o
    menos idéntico a lo Bello). Y así no hay que sorprenderse ante el hecho de
    que los problemas más profundos realmente no son problemas. \footcite[4.003
    p. 45]{tractatus}}

\todo{Anscombe se inició en la filosofía como una ardua actividad.}

De acuerdo a esta afirmación principal Wittgenstein no produce un cuerpo
doctrinal compuesto de proposiciones filosóficas, sino que ofrece
`elucidaciones' que sirven como etapas escalonadas y transitorias que al ser
superadas conducen a ver el mundo correctamente, este esfuerzo hace de
pensamientos opacos e indistintos unos claros y con límites bien definidos.
\footcite[cf. 4.112 y 6.54]{tractatus} La posibilidad de llegar a una visión
clara del mundo es fruto de la posibilidad de lograr de aclarar la lógica del
lenguaje. El lenguaje, a su vez, está compuesto de la totalidad de las
proposiciones, y éstas, cuando tienen sentido, representan el pensamiento.
\footcite[cf. 4 y 4.001]{tractatus} Sin embargo, el mismo lenguaje que puede
expresar el pensamiento puede velarlo:

\citalitlar{4.002 El lenguaje disfraza el pensamiento; de tal manera que de la
    forma externa de sus ropajes uno no puede inferir la forma del pensamiento
    que estos revisten, porque la forma externa de la vestimenta esta elaborada
    con un propósito bastante distinto al de favorecer que la forma del cuerpo
    sea conocida.}

El intento de llegar desde el lenguaje al pensamiento por medio de las
proposiciones con significado es el esfuerzo de conocer una imagen de la
realidad. El pensamiento es la imagen lógica de los hechos, en él se contiene la
posibilidad del estado de las cosas que son pensadas y la totalidad de los
pensamientos verdaderos es una imagen del mundo. \footcite[cf. 3 y
3.001]{tractatus}


\todo{3.4.2 El pensamiento como representación}La conexión entre pensamientos y
hechos viene a responder a la pregunta ``¿qué relación hay entre pensamiento y
realidad?''. La respuesta consiste en la tesis sobre La identidad entre la
posibilidad de la estructura de una proposición y la posibilidad de la
estructura un hecho. Anscombe resume esta afirmación central del Tractatus de éste modo:

\citalitlar{Los objetos --que son simples-- se combinan en situaciones
    elementales. El modo en el que se sujetan juntos en una situación tal es su
    estructura. Forma es la posibilidad de esa estructura. No todas las
    estructuras posibles son actuales: una que es actual es un `hecho
    elemental'. Nosotros formamos imágenes de los hechos, de hechos posibles
    ciertamente, pero algunos de ellos son actuales también. Una imagen consiste
    en sus elementos combinados en un modo específico. Al estar así presentan a
    los objetos denominados por ellos como combinados específicamente en ese
    mismo modo. La combinación de los elementos de la imagen --la combinación
    siendo presentada-- se llama su estructura y su posibilidad se llama la
    forma de representación de la imagen.   
    Esta `forma de representación' es la posibilidad de que las cosas están
    combinadas como lo están los elementos de la imagen. \footcite[p.
    171]{simplicity}
}  

\todo{4. La diferencia entre decir y mostrar}
Es una tarea importante en el Tractatus el delimitar el pensamiento y su
expresión, y junto a esto hay gran cuidado en presentar muchas cosas que aunque
no puede ser dichas aún pueden ser mostradas. Esta distinción entre lo que puede
decirse y lo que sólo puede mostrarse es un asunto central en la reflexión del
libro. 

Dos años antes de su publicación Russell recibió el manuscrito del Tractatus y
escribió a Wittgenstein con algunos comentarios y preguntas. Como respuesta
recibió una carta de Wittgenstein, entonces en un campamento Italiano para
prisioneros de guerra después de la Primera Guerra Mundial, en donde decía:

\citalitlar{Ahora me temo que realmente no has captado mi principal contienda,
    para lo cual todo el asunto de las proposiciones lógicas es sólo corolario.
    El punto principal es la teoría sobre lo que puede ser expresado por
    proposiciones --es decir, por el lenguaje-- (y, lo que viene a ser lo mismo,
    aquello que puede ser pensado) y lo que no puede ser expresado por medio de
    proposiciones, sino solamente mostrado; lo cual, creo, es el problema
    cardinal de la filosofía\ldots \footcite[p. 161]{IWT}}

La conexión entre los pensamientos de W. sobre lógica y sus reflexiones sobre el
significado de la vida habrían de encontrarse en su distinción entre el decir y
el mostrar. La forma lógica no puede expresarse desde el lenguaje, pues es la
forma del lenguaje mismo, se hace manifiesta en el lenguaje, tiene que ser
mostrada. Similarmente, las verdades éticas y religiosas, aunque no expresables
se manifiestan a sí mismas en la vida. 

Las lecciones con Wittgenstein eran directas y con franqueza. Esta metodología
carente de cualquier parafernalia era inquietante para algunos, inspiradora para
otros, pero tremendamente liberadora para Anscombe. \footcite[loc 9853 Chapter
4, Section 24, para. 5]{monk}

En 1941 Anscombe se graduó de St. Hugh's College en Oxford y el siguiente año se
trasladó a Cambridge para sus estudios de posgrado en Newnham College. Cuando
Wittgenstein regresó a Cambridge en 1944 Anscombe asistió a sus lecciones con
entusiasmo. Incluso cuando se le concedió una beca de investigación en
Somerville College en 1946 y regresó a Oxford, todavía durante aquel año y el
siguiente, viajaba una vez a la semana a Cambridge para encontrarse con
Wittgenstein.  

El método terapeútico de Wittgenstein tuvo éxito en liberarla de confusiones
filosóficas donde otras metodologíás mas teoréticas habían fallado. En sus
estudios en St. Hugh's escuchaba a Price.....

\todo{Right stuff...}

\todo{I always hated phenomenalism...}

El Tractatus Logico-Philosophicus fue publicado en el 1922 y ciertamente causó
un impacto en el modo de hacer filosofía. Anscombe emplea la idea de ``corte''
de Boguslaw Wolniewicz para describir el cambio causado por Wittgenstein. Este
corte efectuado en la historia de la filosofía por el Tractatus fue atestiguado
por un filósofo austriaco que describió a Anscombe el efecto cataclísmico
suscitado narrando cómo profesores largamente consolidados se deshacían de sus
viejos libros; la tarea consistía ahora en hacer filosofía en el modo indicado
por el Tractatus y el primer paso era, ciertamente, entenderlo.
\footcite[p.181]{twocuts} 


Este modo de criticar una proposición desvelando que no expresa un pensamiento
verdadero ilustra los principios propuestos en el \emph{Tractatus} y recuerda
una de sus tesis más conocidas: 

\citalitlar{6.53 El método correcto para la filosofía sería este. No decir nada
    excepto lo que pueda ser dicho, esto es, proposiciones de la ciencia
    natural, es decir, algo que no tiene nada que ver con la filosofía: y luego
    siempre, cuando alguien quiera decir algo metafísico, demostrarle que no ha
    logrado dar significado a ciertos signos en sus proposiciones. Este método
    sería insatisfactorio para la otra persona --no tendría la impresión de que
    le estuviéramos enseñando filosofía-- pero este método sería el único
    estrictamente correcto. \footcite[p. 107--108]{tractatus}}


En el prefacio de las Investigaciones Filosóficas, con fecha de enero de 1945
Wittgenstein dice que los pensamientos que publica en el libro son el
precipitado de invetigaciones filosóficas que le han ocupado durante los pasados
16 años. En enero 1929 Wittgenstein estaba regresando a Cambridge.

En 1953 fue publicado el texto de las investigaciones filosóficas

En 1982 Anscombe afirma que el con el segundo corte causado por las
investigaciones filosóficas el proceso analogo al ocurrido con el tractatus
apenas ha comenzado.

El 29 de abril de 1951 murió en Cambridge. 
