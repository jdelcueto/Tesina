\subsection{¿Qué es tener la verdad?}
Elizabeth Anscombe visitó muchas veces la Universidad de Navarra junto con Peter
Geach. Allí impartió algunos seminarios y participó de las Reuniones
Filosóficas.\footcite[Cf.~][15]{torralbaynubiola2005fayeh} En una de sus
visitas, en octubre de 1983, ofreció dos lecciones tituladas: ``Verdad'' y ``La
unidad de la verdad''. Las dos investigaciones estan apoyadas en algunas
reflexiones de San Anselmo cuyos argumentos sirven a Anscombe para explorar
modos de hablar de aquello de lo que decimos que tiene verdad. Anscombe dio
inicio a su ponencia planteando la cuestión como sigue: \citalitlar{Hay verdad
  en muchas cosas. Mirando a mi título [`Truth'] me quedo algo sobrecogida por
  él, pues lo que salta de la página hacia mi es uno de los nombres de Dios. `He
  amado la verdad' me dijo una vez un profesor moribundo, después de hablarme de
  la dificultad que sentía sobre la idea de amar a Dios. Sin embargo: `He amado
  la verdad'. Y luego, temiendo que yo no malentendiera su afirmación: `No me
  refiero, cuando digo eso, que \emph{tenga} la verdad'.} \citalitlar{Tener la
  verdad, estar en la verdad---¿qué es esto?
  \footnote{\cite[71]{anscombe2011plato:truth}: <<There is truth in many things.
    Looking at my title [`Truth'] I am somewhat awed by it, for what leaps out
    of the page at me is one of the names of God. `I have loved the truth' a
    dying teacher once said to me, after speaking of the difficulty he felt
    about the idea of loving God. But:`I have loved the truth'. And then,
    fearing lest I misconstrue his statement: `I do not mean, when I say that,
    that I \emph{have} the truth'. To have the truth, to stand in the truth --
    what are these?>>}.}

%Y qué quiso decir Nuestro Señor al llamarse a \emph{sí mismo} la verdad? `No hay
%tal cosa como la verdad, sólo hay verdades', decía mi suegro a la primera esposa
%de Bertrand Russell. Russell fue su maestro; la influencia se ve con facilidad.}

%\citalitlar{¿Pero cuáles son las cosas que tienen verdad en ellas? ¿Tiene la
%  creación? ¿tienen las acciones? A qué se refería Aristóteles cuando dijo que
%  el bien de la razón práctica era `verdad de acuerdo con el recto deseo'? ¿Las
%  cosas hechas por los hombres tienen verdad en ellas? ¿Qué, de nuevo, quiso
%  decir Aristóteles cuando afirmó que el arte o la habilidad es una disposición
%  productiva con un logos verdadero? Mas allá todavía: Qué fuerza tiene contar
%  la verdad entre los `trascendentales', esas cosas que `atraviesan' todas las
%  categorías y todas las formas especiales de las cosas; y que no pertenecen
%  cada uno a una categoría, como el color: amarillo; o el area: un acre; o el
%  animal: un caballo.}

\subsection{La primacia de la verdad sobre la falsedad}
Este cuestionamiento lleva a Anscombe a indagar en una materia en la que
Wittgenstein y San Anselmo ---dice--- son `hermanos intelectuales': ¿cuál es la
primacía de la verdad sobre la
falsedad?\autocite[Cf.~][73]{anscombe2011plato:truth}.

San Anselmo queda prendado de esta pregunta como consecuencia de su indagación
en el capítulo segundo del \emph{De Veritate}: ¿qué es la verdad de la
enunciación?\footnote{\cite[Cf.~][493]{anselm1952obras:deveritate} Para las
  citas del texto de San Anselmo se ha empleado la traducción de
  \cite{anselm1952obras} donde `\emph{enuntiatio}' se traduce como
  `enunciación', Anscombe lo traducirá como `\emph{proposition}'. `Enunciación'
  y `proposición' se usarán aquí indistintamente.}. Anselmo elige indagar en las
enunciaciones o proposiciones como aquellas clases de las cuales más
naturalmente se puede pensar que contienen los posibles portadores del predicado
`verdadero'. Así lo expresa cuando dice \citalitinterlin{Busquemos, pues, en
  primer lugar qué es la verdad de la enunciación, puesto que es ésta la que
  calificamos con más frecuencia de verdadera o
  falsa}\autocite[493]{anselm1952obras:deveritate}.

Wittgenstein recorre una ruta análoga en los apartados que conforman el \S4.06
del Tractatus. Argumenta que \citalitinterlin{Una proposición puede ser
  verdadera o falsa sólo en virtud de ser una imagen de la
  realidad}\footnote{\cite[\S4.06]{wittgenstein1922tractatus}:<<Propositions can
  be true or false only by being pictures of the reality.>>}. Y advierte que
\citalitlar{No debe ser pasado por alto que una proposición tiene un sentido que
  es independiente de los hechos: de otra manera uno podría fácilmente suponer
  que verdadero y falso son relaciones igualmente justificadas entre los signos
  y aquello que
  significan\footnote{\cite[\S4.061]{wittgenstein1922tractatus}:<<If one does
    not observe that propositions have a sense independent of the facts, one can
    easily believe that true and false are two relations between signs and
    things signified with equal rights.>>}.}

Elizabeth realiza su investigación adentrándose en la misma cuestión trabajada
por ambos autores. El primer movimiento que hace en su análisis es indagar en la
distinción entre significado y verdad. Según se ha visto, la distinción es
familiar en las elucidaciones del Tractatus: \citalitinterlin{La proposición
  tiene un sentido que es independiente de los hechos}
\footnote{\cite[\S~4.061]{wittgenstein1922tractatus}: <<propositions have a
  sense independent of the fact>>} San Anselmo también lo considera. Una
proposición no pierde su significado cuando no es verdadera. Si el significado
(\emph{significatio}) de una proposición fuera su verdad, ésta
\citalitinterlin{semper esset vera}\autocite[492]{anselm1952obras:deveritate},
siempre sería verdadera. Sin embargo el significado de una proposición
\citalitinterlin{manent \ldots et cum est quod enunciat, et cum non
  est}\autocite[492]{anselm1952obras:deveritate}, permanece lo mismo cuando lo
que se afirma es el caso que es y cuando no lo es.

Significado y verdad en una proposición son distintos. Entonces, ¿qué es la
verdad de una proposición?. Se podría querer responder que es la
\citalitinterlin{res enunciata}, es decir, la realidad correspondiente, lo que
la proposición verdadera dice. Esta respuesta nos llevaría a confusión. ``La
verdad de una proposición es este hecho que es su significado''. Si esto es así,
entonces cuando deja de ser verdadera también pierde su significado, pues el
hecho que era su signifcado ya no es. Además, si la desaparición del hecho es la
desaparición del significado y la verdad, ¿no será entonces que el hecho es la
misma cosa que el significado y la
verdad?\autocite[Cf.~][72]{anscombe2011plato:truth}. Sin embargo no es así, el
hecho es lo que la hace verdadera: lo que la proposición verdadera dice, la
\emph{res enunciata} es la causa de la verdad de una proposición y no su verdad:
\citalitinterlin{non eius veritas, sed causa veritatis eius dicenda
  est}\autocite[492]{anselm1952obras:deveritate}.

La distinción abre otra línea de consideraciones. El hecho o la \emph{res
  enunciata} por la proposición verdadera es la causa de la verdad del
enunciado. La proposición tiene significado independientemente de si es
verdadera o falsa. En este sentido, una proposición con significado puede
guardar relación de verdad o de falsedad con los hechos. Una proposición falsa
no carece de toda relación con el hecho, sino que contiene una descripción del
hecho que hace a la proposición contraria
verdadera\autocite[Cf.~][73]{anscombe2011plato:truth}. Podríamos pensar,
entonces, que la proposición verdadera y la proposición falsa pueden
intercambiar roles.

Wittgenstein sugiere esto cuando afirma que el hecho de \citalitinterlin{que los
  signos ``$p$'' y ``${\sim}p$'' (``no $p$'') pueden intercambiar roles es
  importante, pues muestra que ``$\sim$'' (``no'') no corresponde con nada en la
  realidad}\footnote{\cite[\S4.0621]{wittgenstein1922tractatus}: <<That,
  however, the signs ``$p$'' and ``${\sim}p$'' can say the same thing is important,
  for it shows that the sign ``$\sim$'' corresponds to nothing in reality.>>}. Más
aún ``$p$'' y ``${\sim}p$'' son opuestos en significado pero a ambos enunciados
corresponde una sola realidad; esto es el hecho, la \emph{res enunciata} por el
enunciado verdadero. Esto permitiría sostener que verdadero y falso son tipos de
relaciones entre el signo y la cosa significada que están igualmente
justificadas. ``$p$'' y ``${\sim}p$'' significan la misma realidad, cualquiera
de las dos posibilidades que resulte ser la realidad correspondería con
ambas\autocite[Cf.~][73]{anscombe2011plato:truth}. La única distinción que
Wittgenstein se reserva entre ambas proposiciones es que una significa
falsamente lo que la otra significa verdaderamente. Sin embargo esta distinción
puede quedar disuelta con facilidad si se considera que `significa
verdaderamente' o `significa falsamente' no son descripciones de los sentidos de
las proposiciones verdaderas o falsas. Se puede entender el sentido de ``estoy
sentado'' o ``no estoy sentado'' sin conocer cuál enunciado se corresponde con
la realidad o cuál de ambas expresiones está significando verdaderamente y cuál
falsamente. En cuanto a la relación entre signo y significado ambas
proposiciones no tienen diferencia\autocite[Cf.~][74]{anscombe2011plato:truth}.

En San Anselmo esta noción de relaciones igualmente justificadas aparece bajo la
forma de una pregunta planteada por el discípulo en el diálogo con su maestro.
Dice: \citalitlar{enséñame a responder a aquel que me dijese que aun cuando el
  discurso exprese la existencia de lo que no existe, significa lo que debe,
  porque ha podido significar igualmente la existencia de lo que es y de lo que
  no es. En efecto, si no significara también la existencia de lo que no existe,
  no lo significaría. Por lo cual, aun cuando dice ser lo que no es, significa
  lo que debe. Pero si, al significar lo que debe, es recta y verdadera, como
  has demostrado, el discurso es verdadero aun cuando enuncia la existencia de
  lo que no existe\autocite[495]{anselm1952obras:deveritate}.} Las dos
relaciones son expresadas como una paridad: \citalitinterlin{pariter accepit
  significare esse, et quod est, et quod non
  est}\autocite[494]{anselm1952obras:deveritate}. Esta paridad es esencial ya
que si la proposición no significara lo que significa igualmente cuando lo que
significa es y también cuando tal cosa no es, no sería capaz de significar del
todo.

A propósito de esta paridad, Wittgenstein plantea: \citalitinterlin{¿Acaso no
  podríamos hacernos entender usando proposiciones falsas tal como hemos hecho
  hasta ahora por medio de las verdaderas, siempre y cuando sepamos que están
  significadas falsamente?}\footnote{\cite[\S4.062]{wittgenstein1922tractatus}:
  <<Can we not make ourselves understood by means of false propositions as
  hitherto with true ones, so long as we know that they are meant to be
  false?>>}. Anscombe compara este posible modo de actuar a una táctica de Santa
Juana de Arco. La Santa empleaba un código en las comunicaciones con sus
generales subordinados que consistía en que las cartas que ella marcaba con una
cruz contenían proposiciones que debían ser interpretadas en el sentido
contrario\autocite[Cf.~][73]{anscombe2011plato:truth}. El código es posible.

Hasta aquí Anscombe ha insitido en los argumentos de San Anselmo y de
Wittgenstein que apoyan la idea de que las proposiciones falsas y verdaderas
tienen igualdad de relación con la realidad significada. Wittgenstein ha
advertido del supuesto de entender ambas relaciones como igualmente
justificadas, sin embargo lo que ha propuesto hasta ahora parece apoyar esta
idea. La paridad propuesta ha resultado esencial para el significado, el sentido
o \emph{significatio} del tipo de proposiciones que pueden ser verdaderas o
falsas. La pregunta ahora es ¿qué, entonces, \emph{es} desigual entre ellas?
¿Cuál es la primacia de la verdad?

La respuesta de Wittgenstein a esta pregunta llegará a ser: no se puede
describir a alguien como comunicándose con proposiciones falsas entendidas como
significadas falsamente ya que se tornan en proposiciones verdaderas al ser
afirmadas\autocite[Cf.~][75]{anscombe2011plato:truth}. Esta es su respuesta a la
pregunta ¿podemos darnos a entender con proposiciones falsas?:
\citalitinterlin{¡No! Pues una proposición es verdadera si las cosas son así
  como estamos usándola para decir que son, y entonces si usamos ``$p$'' para
  decir que ${\sim}p$, y las cosas son como queremos decir que son, entonces
  ``$p$'' es vedadero en nuestro nuevo modo de tomarlo y no
  falso}\footnote{\cite[\S4.062]{wittgenstein1922tractatus}: <<No! For a
  proposition is true, if what we assert by means of it is the case; and if by
  ``$p$'' we mean ${\sim}p$, and what we mean is the case, then ``$p$'' in the
  new conception is true and not false.>>}. En la táctica antes descrita, Santa
Juana de Arco no mentía con su código y, si no estaba en error acerca de los
hechos, sus oraciones eran verdaderas y no
falsas\autocite[Cf.~][75]{anscombe2011plato:truth}.

Para Anscombe, esta descripción de la primacía de la verdad no parece explicar
cómo rechazar que verdadero y falso tengan relaciones igualmente justificadas
¿Acaso este tipo de imposibilidad general contiene toda la sustancia de las
`relaciones no igualmente justificadas'? Se puede aceptar que verdadero y falso
no son relaciones igualmente justificadas porque lo falso no podría hacerse
cargo del rol de lo verdadero en las afirmaciones y en el pensamiento. Sin
embargo, podemos mentir\ldots\, o equivocarnos. La imposibilidad general de
intercambiar los roles de verdadero y falso propuesta por Wittgenstein no
considera ni el error ni la mentira. Esta imposibilidad general puede ofrecer
una cierta primacia de la verdad dentro de la teoría del significado, pero ¿se
podría apoyar en esto el decir que la proposición verdadera tiene una relación
mas \emph{justificada} con la realidad que la
falsa?\autocite[Cf.~][75]{anscombe2011plato:truth}.

En San Anselmo, por su parte, se puede encontrar una propuesta sobre la primacía
de la verdad dentro de su definición de lo que la verdad es. Su punto de partida
ha sido la pregunta: \citalitinterlin{¿Cuál es el fin de la
  afirmación?}\autocite[495]{anselm1952obras:deveritate} El diálogo se
desarrolla de este modo: \citalitlar{\emph{Maestro.}---¿Cuál te parece ser aquí
  la verdad?\\
  \emph{Discípulo.}---No sé más que, cuando significa existir lo que existe
  realmente, está en ella la verdad y es verdadera.\\
  \emph{M.}---¿Cuál es el fin de la afirmación?\\
  \emph{D.}---Expresar lo que es.\\
  \emph{M.}---¿Debe, pues, hacerlo?\\
  \emph{D.}---Ciertamente.\\
  \emph{M.}---Por consiguiente, cuando expresa la existencia de lo que existe,
  expresa lo que debe.\\
  \emph{D.}---Es evidente.\\
  \emph{M.}---Y cuando expresa lo que debe, expresa con exactitud.\\
  \emph{D.}---Así es.\\
  \emph{M.}---Pero cuando expresa con rectitud, ¿su significación es exacta?\\
  \emph{D.}---Sin duda ninguna.\\
  \emph{M.}---Cuando expresa la existencia de lo que es, ¿la significación es recta?\\
  \emph{D.}---Es una conclusión que se impone.\\
  \emph{M.}---Igualmente, cuando significa la existencia de lo que existe, su
  significado es verdadero.\\
  \emph{D.}---Ciertamente es a la vez verdadera y recta cuando expresa la
  existencia de lo que es.\\
  \emph{M.}---¿Entonces es una misma y única cosa para ella el ser recta y
  verdadera, es decir, manifestar la existencia de lo que es?\\
  \emph{D.}---Es una sola y misma cosa.\\
  \emph{M.}---Por consiguiente, para ella, la verdad no es otra cosa que la
  rectitud.\\
  \emph{D.}---Sí; veo con claridad que la verdad no es más que esta rectitud.\\
  \emph{M.}---Lo mismo hay que decir cuando la enunciación expresa la no
  existencia de lo que existe\autocite[495]{anselm1952obras:deveritate}.}

El discípulo ha visto que la verdad del enunciado no es la \emph{res enunciata}
por una proposición verdadera, tampoco está en la significación, o en cualquier
cosa perteneciente a la definición, sino que \citalitinterlin{Nihil aliud scio
  nisi quia cum significat esse qous est, tunc est in ea veritas et est
  vera}\autocite[492]{anselm1952obras:deveritate}. Cuando una afirmación hace
aquello para lo que es, la significación (\emph{significatio}) está hecha
rectamente. Esta rectitud es lo que la verdad es. Es aquí que el discípulo
presenta la objeción antes expuesta: `Cuando una expresión significa que es algo
que no es, ¿se puede decir que está significando lo que debe?'. La respuesta del
maestro será: \citalitinterlin{veritatem tamen et rectitudinem habet, quia facit
  quod debet}\autocite[494]{anselm1952obras:deveritate}. Una expresión falsa
hace lo que debe en significar aquello que le ha sido dado significar, hace
aquello para lo que la expresión es. Sin embargo, teniendo este modo de ser
verdadera, no solemos llamarla verdadera pues habitualmente decimos que la
expresión es verdadera y correcta sólo cuando significa que es aquello que es y
no cuando significa que es aquello que no es, pues tiene mayor deber de hacer
aquello para lo que se le ha dado significar que para lo que no se le ha dado.
Es sorprendente que el maestro no rechace la descripción del discípulo, más aún
que la reitere. La objeción presentada no supone un impedimento para sostener
esta descripción de la verdad. El maestro retiene su explicación apoyada en que
la verdad de un enunciado es que hace lo que
debe\autocite[Cf.~][76]{anscombe2011plato:truth}.

¿En qué consiste entonces la primacía de la verdad? La proposición verdadera
hace lo que debe de dos maneras: significa justo aquello que se le ha dado
significar ---independientemente de si es el caso que es o no--- y significa
aquello para lo que se le ha dado esa significación, esto es, afirmar como que
es el caso lo que \emph{es} el caso. Calificamos de justa y verdadera la
proposición en virtud de ese hacer doblemente lo que debe, es decir, por su
rectitud y verdad.\autocite[Cf.~][497]{anselm1952obras:deveritate}.

Una observación adicional de Anselmo puede ser relacionada con la pregunta de
Wittgenstein: `¿Podríamos darnos a entender por medio de proposiciones falsas?'.
\citalitinterlin{[la enunciación] no ha sido hecha para expresar que una cosa
  existe cuando no existe o que no existe cuando sí existe, porque fue imposible
  hacer que expresase solamente la existencia cuando ésta existe, o la no
  existencia cuando no existe}\autocite[497]{anselm1952obras:deveritate}. A la
proposición no se le podía dar significar que algo es solamente cuando eso que
significa da el caso que es o su no ser sólo cuando es el caso que no es,
solamente por eso puede significar lo contrario de lo que existe, aunque no ha
sido hecha para eso\autocite[Cf~.][76]{anscombe2011plato:truth}. La observación
se acerca a la respuesta de Wittgenstein. En este sentido, lo falso sólo es
posible porque lo verdadero (en este tipo de proposiciones) no puede ser la
única posibilidad.

La descripción de la verdad que Anselmo comienza aquí le llevará por medio de
consideraciones sobre la verdad en el pensamiento, la voluntad, la acción y el
ser de las cosas a su conocida definición de la verdad como \emph{veritas est
  rectitudo sola mente perceptibilis}\autocite[522]{anselm1952obras:deveritate}.

\subsection{Solución de Anscombe}
Anscombe no llega a proponer una respuesta suya a la cuestión planteada en
\emph{Truth}. Culmina constatando como San Anselmo y Wittgenstein indican una
cierta primacia de la verdad en la materia del significado apoyados en distintas
razones. Sin embargo en \emph{Truth, Sense and
  Assertion}\autocite{anscombe2015logic:tsa} quedan recogidas sus notas para una
lección ofrecida en \emph{Johns Hopkins University} en abril de
1987\autocite[Cf.~][264 n.~1]{anscombe2015logic:tsa} en donde continúa su
análisis y ofrece una solución propia.

La pregunta fundamental que planteará Anscombe en este análisis será:
\citalitinterlin{¿Es la enunciación lo mismo que la
  significación?}\footnote{\cite[271]{anscombe2015logic:tsa}:<<Is enuntiation
  the same as signification?>>}.
