% SECCIÓN 1: La verdad
\section{Verdad y Significado}

\subsection{Qué es tener la verdad}
Elizabeth Anscombe visitó muchas veces la Universidad de Navarra junto con Peter
Geach y allí impartió algunos seminarios y participó de las Reuniones
Filosóficas.\footcite[cf.~][p.~15]{fa&esphom}

En una de sus visitas, en octubre de 1983, ofreció dos lecciones tituladas:
``Verdad'' y ``La unidad de la verdad''. En la primera lección planteó la
cuestión de este modo:

\citalitlar{Hay verdad en muchas cosas. Mirando a mi título `Truth' me quedo
  algo sobrecogida por él, pues lo que salta de la página hacia mi es uno de los
  nombres de Dios. `He amado la verdad' me dijo una vez un profesor moribundo,
  después de hablarme de la dificultad que sentía sobre la idea de amar a Dios.
  Sin embargo: `He amado la verdad'. Y luego, temiendo que yo no malentendiera
  su afirmación: `No me refiero, cuando digo eso, que \emph{tenga} la verdad'}
\citalitlar{Tener la verdad, estar en la verdad--¿qué es esto? Y qué quiso decir
  Nuestro Señor al llamarse a \emph{sí mismo} la verdad? `No hay tal cosa como
  la verdad, sólo hay verdades', dijo mi suegro a la primera esposa de Bertrand
  Russell. Russell fue su maestro; la influencia se ve con facilidad.}
\citalitlar{¿Pero cuáles son las cosas que tienen verdad en ellas? ¿Tiene la
  creación? ¿tienen las acciones? A qué se refería Aristóteles cuando dijo que
  el bien de la razón práctica era `verdad de acuerdo con el recto deseo'? ¿Las
  cosas hechas por los hombres tienen verdad en ellas? ¿Qué, de nuevo, quiso
  decir Aristóteles cuando afirmó que el arte o la habilidad es una disposición
  productiva con un logos verdadero? Mas allá todavía: Qué fuerza tiene contar
  la verdad entre los `trascendentales', esas cosas que `atraviesan' todas las
  categorías y todas las formas especiales de las cosas; y que no pertenecen
  cada uno a una categoría, como el color: amarillo; o el area: un acre; o el
  animal: un caballo.\footcite[p.~71]{FPW}}

La cuestión que Anscombe trae en esta lección tiene que ver con la primacia de
la verdad sobre la falsedad.\footcite[195]{teichmann}

Anscombe no se traga toda la teoría de la imagen de las proposiciones. Pero ella
ve lo que es probablemente la cosa mas iluminadora de la comparación de
Wittgenstein de imagenes y proposiciones; es decir, este `Janus-faced aspect' de
las proposiciones, un aspecto que puede ser expresado de diversos modos--como el
que `No' no se corresponde con nada en la realidad, o que P y no-P (los
símbolos) pueden ser sistematicamente inercambiados, cada uno asumiendo la
función del otro..
