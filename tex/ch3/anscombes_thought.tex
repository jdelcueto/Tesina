%SECCIÓN 1: ANSCOMBE Y WITTGENSTEIN
\section{Anscombe y Wittgenstein}

\subsection{El método de Wittgenstein}
El 16 de octubre de 1944 Ludwig Wittgenstein reanudó su itinerario de clases en
Cambridge con dos encuentros de dos horas cada semana. Aquél trimestre
'Michaelmas' contaba seis estudiantes en su curso, entre ellos Elizabeth
Anscombe. 

\engcitation{I am like a piano teacher. I am trying to teach a style of
  thinking, a technique, not a subject matter. Anscombe recalls (1995, p. 407):
  ``He himself in his classes sometimes said he was as it were giving examples
  of `five-finger exercises’ in thinking. These were certainly not limited in
  number like the set a piano teacher might employ, and were not like automatic
  formulae of investigation.'' (Cf. Gass 1971, p. 248: ``what you heard was
  something like a great pianist at practice: not a piece of music, but the very
  acts which went into making that performance.'')} En aquellas lecciones sobre
filosófia de la psicología se comportaba como un maestro de piano ---decía---
intentaba enseñar un estilo de pensar, una técnica. Lo que se escuchaba en sus
lecciones no eran piezas musicales, sino más bien las prácticas de un buen
pianista donde afina los movimientos que van dirigidos a construir su
concierto.\footcite[p.~357]{pubnpriv}

En cierta ocasión Wittgenstein recibió a Anscombe con una pregunta: <<¿Por qué
la gente dice que era natural pensar que el sol giraba alrededor de la tierra en
lugar de que la tierra rotaba en su eje?>> Elizabeth contestó: <<Supongo que
porque se veía como si el sol girara alrededor de la tierra.>> <<Bueno\ldots>>,
añadió Wittgenstein, <<¿cómo se hubiera visto si se hubiera \emph{visto} como si
la tierra rotara en su propio eje?>> Anscombe reaccionó extendiendo las manos
delante de ella con las palmas hacia arriba y, levantándolas desde sus rodillas
con un movimiento circular, se inclinó hacia atrás asumiendo una expresión de
mareo. <<¡Exactamente!>> exclamó Wittgenstein.\footcite[cf.~][p.~151]{IWT}

Anscombe se percató del problema; la pregunta de Wittgenstein había puesto en
evidencia que hasta aquél momento no había ofrecido ningún significado relevante
para su expresión \emph{``se veía como si''} en su respuesta \emph{``se veía
  como si el sol girara alrededor de la tierra''}.

¿Qué tipo de problema es este? ¿Qué falta cuando una expresión carece de
significado? \pnote{¿Es esta una pregunta sobre la representación que significan
  las palabras? ¿Es una pregunta sobre el uso que se les da a las palabras?}

\subsection{El arte de hacer filosofía}

\ifdraft{\subsubsection{Vida salvaje luchando por emerger abiertamente}}{}
\engcitation{``Within all great art there is a WILD animal: tamed.''}
Wittgenstein pensaba que
\citalitinterlin{dentro de todo buen arte hay un animal salvaje
  domado}\footcite[p.~43e]{cnv}.
Su talante artístico, sin embargo, no manifestaba esta primitiva vitalidad; o
como él mismo decía:
\engcitation{``In my artistic activities I have merely good manners''}
\citalitinterlin{en mis actividades artísticas tengo meramente buenos
  modales.}\footcite[p.~29e]{cnv}

Ejemplo de estos ``buenos modales'' fue el diseño que realizó para la casa de su
hermana Margaret en Viena, terminada en 1928.
\engcitation{``my house for Gretl is the product of a sensitive ear, good
  manners, the expression of great understanding... wild life striving to erupt
  in the open is lacking... health is lacking (Kierkergaard)''}
Trabajó como arquitecto de la casa con exhaustiva minuciosidad y el producto
manifestaba gran entendimiento, ``buen oido'', pero le escaseaba ``salud'',
pensaba él.\footcite[p.~43e]{cnv}
\engcitation{``Even in music... feeling, he showed above all great
  understanding, rather than manifesting wild life... When he played music with
  others... his interest was in getting it right... When he played, he was not
  expressing himself... but the thoughts... of others. He was probably right to
  regard himself not as creative but as reproductive ...It was only in
  philosophy that his creativity could really be awakened. Only then, as Russell
  had long ago noticed, does one see in him 'wild life striving to erupt in the
  open''}

También en la música, arte por la que tenía la mayor afición, era llamativa su
recia exactitud. Cuando tocaba con otros ponía su mayor interés en lograr una
expresión exacta y correcta, recreando música y pensamientos ajenos, más que
expresándose a sí mismo. Perseguía reproducir más que
crear.\footcite[loc.˜]{monk}

Esta fuerza creativa ausente en su rigurosa actitud hacia la actividad artística
estallaba, sin embargo, en su actividad filosófica. Aquella cualidad que él
encontraba característica del buen arte, esa ``vida salvaje luchando por emerger
abiertamente'',\footcite[cf.˜][loc.˜]{monk} quedaba expresada en su quehacer
filosífico.

\ifdraft{\subsubsection{Filosofía emergente}}{}
La filosofía nació así en Ludwig. Como una fuerza violenta. Se hallaba
estudiando ingeniería en Manchester y se interesó por los fundamentos de las
matemáticas. Este interés no tardó en convertirse en el deseo de elaborar un
trabajo filosófico. Su hermana Hermine le describe así en sus memorias de la
familia Wittgenstein
\footnote{Hermine Wittgenstein escribió la historia y memorias de su familia
  ``Familienerinnerungen'' durante la segunda Guerra Mundial.}:

\citalitlar{Fue repentinamente agarrado por la filosofía ---es decir, por la
  reflexión en problemas filosóficos--- tan violentamente y tan en contra de su
  voluntad que sufrió severamente por la doble y conflictiva llamada interior y
  se veía a sí mismo como roto en dos. Una de muchas transformaciones por las
  que pasaría en su vida había venido sobre él y le estremeció hasta lo más
  profundo. Estaba concentrado en escribir un trabajo filosófico y finalmente
  determinó mostrar el plan de su obra al Profesor Frege en Jena, quien había
  discutido preguntas similares. [\ldots] Frege alentó a Ludwig en su búsqueda
  filosófica y le aconsejó que fuera a Cambridge como alumno del Profesor
  Russell, cosa que Ludwig ciertamente hizo.\footcite[p. 73]{mcguinness}}

La investigación filosófica comenzada en aquel momento se convirtió en la tarea
del resto de su vida. Sus incipientes ideas filosóficas pasarían por diversas
transformaciones, pero expresaban ya desde el principio una preocupación por los
problemas fundamentales. Por las reglas del juego, se podría decir.

\ifdraft{\subsubsection{La Naturaleza de los problemas Filosóficos}}{}
Entre esas cuestiones fundamentales se halla una de las constantes importantes
en su pensamiento. Ésta es su definición de la naturaleza de los problemas
filosóficos. Para Wittgenstein las cuestiones de la filosofía no son
problemáticas por ser erróneas, sino por no tener
significado.\footcite[cf.~][4.003]{tractatus}

Una proposición sin significado que no es puesta al descubierto como tal atrapa
al filósofo dentro de una confusión del lenguaje que no le permite acceder a la
realidad. Salir de la confusión no consiste en refutar una doctrina y plantear
una teoría alternativa, sino en examinar las operaciones hechas con las palabras
para llegar a manejar una visión clara del empleo de nuestras expresiones. La
filosofía no es un cuerpo doctrinal, sino una
actividad\footcite[cf.~][4.112]{tractatus}y una
terapia\footcite[cf.~][\S133]{PI}.

La actitud terapéutica adoptada por Wittgenstein en su atención de las
confusiones filosóficas fue su respuesta más definitiva a la naturaleza de estos
problemas. Para ello halló los más eficaces remedios en sus investigaciones
sobre el significado y el sentido del lenguaje.

Ordinariamente tomamos parte en esta actividad humana que es el lenguaje.
Jugamos el juego del lenguaje. ---¿Jugarlo es entenderlo?--- A la vista de
Wittgenstein saltaban extraños problemas sobre las reglas de este juego;
entonces no podía evitar escudriñarlas al
detalle.\footcite[cf.~][loc.7099]{monk} En este análisis del lenguaje está la
raíz de sus ideas sobre el sentido, el significado y la verdad.

Durante su vida sostuvo dos grandes descripciones del significado. Originalmente
describió el lenguaje como una imagen que representa el posible estado de las
cosas en el mundo. En una segunda etapa se distanció de esta analogía para
describir al lenguaje como una herramienta cuyo significado consiste en la suma
de las múltiples semejanzas familiares que aparecen en los distintos usos para
los cuales el lenguaje es empleado en la actividad humana. Dentro de la primera
descripción una expresión sin significado es una cuyos elementos no componen una
representación del posible estado de las cosas. Dentro de la segunda descripción
una expresión sin significado resulta del empleo de una expresión propia de un
``juego del lenguaje'' fuera de su contexto.

\ifdraft{\subsubsection{Dos Cortes en la Filosofía}}{}
Estas dos etapas del pensamiento de Wittgenstein son representadas por dos
importantes tratados. El \emph{'Tractatus Logico\=/Philosophicus'}, publicado en
1921, recoge sus esfuerzos por elaborar un gran tratado filosófico comenzados en
1911 y culminados durante la Primera Guerra Mundial. El segundo,
\emph{'Philosophische Untersuchungen'}, o \emph{'Investigaciones Filosóficas'},
traducido por Anscombe y publicado posthumamente en 1953, fue elaborado a partir
de múltiples manuscritos desarrollados por Wittgenstein desde su regreso a
Cambridge en 1929 hasta su muerte en 1951.

\citalitinterlin{Wittgenstein es extraordinario entre los filósofos por haber
  generado dos épocas, o cortes\footnote{Anscombe toma el termino 'corte' de
    Boguslaw Wolniewicz, filósofo polaco y amigo.}, en la historia de la
  filosofía.}\footcite[p.~181]{twocuts} 
Con estas palabras Anscombe comenzaría su discurso inaugural para el 6to
Simposio Internacional de Wittgenstein unos treinta años después de la
publicación de las \emph{'Investigaciones Filosóficas'}. Y explica:
\citalitinterlin{un filósofo hace un corte si genera un cambio en el modo en que
  la filosofía es hecha: la filosofía tras el corte no puede ser la misma de
  antes.}\footcite[p.~181]{twocuts}

Estos cambios de época generados por la influencia de Wittgenstein vinieron
caracterizados por el esfuerzo de comprender cada libro tras su publicación,
tarea complicada en ambos casos por la dificultad intrínseca de los tratados,
ofuscada a su vez por los prejuicios filosóficos proyectados a cada obra por sus
lectores. La presunción, por ejemplo, de que \emph{'Investigaciones
  Filosóficas'} presenta una teoría del lenguaje ---quizás sobre cómo los
sonidos se tornan en discursos significativos--- nos dejaría situados lejos de
las preguntas que genuinamente ocupan a
Wittgenstein.\footcite[cf.~][p.~183]{twocuts} Ahora bien, la comprensión
adecuada de su pensamiento y método trae consigo, a juicio de Anscombe, cierto
efecto curativo.

\ifdraft{\subsubsection{Ver el mundo claramente}}{}
Quedar 'curados' es quedar liberados de la trampa de ciertas inclinaciones que
impiden llegar a concepciones verdaderas. El trabajo de Wittgenstein busca tener
este efecto en la filosofía. ¿Lo logra?

Elizabeth analiza uno de estos esfuerzos. Es una aflicción extendida entre los
filósofos la excesiva dependencia en explicaciones o conexiones necesarias. ¿Han
podido quedar curados los que han estudiado a Wittgenstein? Y añade:
\citalitlar{La filosofía profesional es en gran medida una gran fábrica para la
  manufactura de necesidades---sólo las necesidades nos dan paz mental. No es de
  extrañarse que Wittgenstein despierte cierto odio entre nosotros. Amenaza
  privarnos de nuestro empleo en la fábrica.\footcite[p~.184]{twocuts}}

La dependencia en estas explicaciones que \emph{`deben de ser'} para justificar
nuestras proposiciones nos impide tener una concepción clara del panorama de la
realidad. Anscombe lo ilustra de este modo:
\citalitlar{La descripción detallada de la distribución de manchas de color en
  un canvas no nos revela la imagen que está en él, sin embargo, si dices:
  ``Pero la imagen es \emph{también}. \emph{¿En qué consiste?} \emph{debe de}
  haber ahí algo más además de pintura en un canvas''--estarías embarcandote en
  una busqueda ilusoria. El vasto número de cosas que conocemos y hacemos y que
  indagamos son como la imagen en el canvas. Las realidades acerca de nuestro
  conocer, nuestro hacer y nuestro indagar son enormemente interesantes; pero
  necesidades de tipo absolutamente \emph{a priori} no pueden ser encontradas
  para justificar nuestras aserciones.\footcite[p.~185]{twocuts}}

En contraste con este uso engañoso de la necesidad hay un uso inocuo de ese
\emph{`deber de'} que ocurre en regiones más especializadas. Un ejemplo
notable es el modo en el que hacemos cuentas en una serie, o el modo en el que
calculamos el valor de una variable $\mathcal{Y}$ dado un cierto valor para
$\mathcal{X}$ en una fórmula. Podríamos decir que la serie está determinada ya
de antemano por la fórmula, al calcularla sólo ponemos en tinta, por así
decirlo, la parte de la serie que estamos computando. Aquí no estamos
exactamente manufacturando una necesidad, sino más bien
\citalitinterlin{tratando de formular el ideal de una necesidad que está siendo
  imitada por los cálculos cuando son de resultados que son `determinados', en
  ese sentido inofensivo de necesidad \footcite[p.~185]{twocuts}}.

Pues bien, para Wittgenstein la pregunta sobre la manera adecuada de continuar
una serie es la misma pregunta sobre cómo usar la palabra `rojo'. Así como la
serie tiene una cierta determinación por su formula, la palabra tiene una cierta
determinación por su uso. En este sentido, conocer el significado de una palabra
consiste en comprender ese \emph{`deber de'} que determina su futura aplicación.

Este camino en la busqueda del significado de las proposiciones puede ser
ocasión de otra inclinación:
\citalitinterlin{Aquí no estamos tan tentados de inventar o manufacturar
  necesidades, sino de descansar conformes con las que creemos haber
  comprendido.\footcite[p.~185]{twocuts}}

Esta podría ser nuestra actitud respecto de nuestro uso de las proposiciones
hasta que alguien nos interrumpe con una pregunta sobre la necesidad de estar en
lo correcto cuando usamos una palabra de cierto modo. Esta pregunta sería
esceptica sólo para aquel que asumiera que sus presunciones son
irrefragablemente correctas y la base del significado y la
verdad.\footcite[cfr.~][p.~186]{twocuts}

El impacto de Wittgenstein en la filosofía es para Anscombe una ruta que permite
llegar a concepciones verdaderas. Nos permite ver la pintura con claridad.
Siguiendo la anterior ilustración:

\citalitlar{Es un impedimento para llegar a mirar la imagen, si estás aferrado a
  la convicción de que debes una de dos; extraer la imagen desde la descripción
  del color de cada mancha de pintura en una fina cuadrícula extendida sobre
  esta, o que debes tener una teoría de lo que la imagen es aparte de lo que esa
  descripción describe. Si renuncias a ambas inclinaciones podrás llegar a mirar
  a la pintura y haciéndolo podrías encontrarte lleno de asombro. O, como
  Wittgenstein una vez lo dijera, puedes encontrarte a tí mismo `caminando en
  una montaña de maravillas'}

Según Anscombe el método general adecuado de discutir los problemas filosóficos
propuesto por Wittgenstein consiste en mostrar que la persona no ha provisto
significado (o referencia) para ciertos signos en sus expresiones.\footcite[cf.
p. 151]{IWT} Creía que el camino que lleva a formular estos problemas está
frecuentemente trazado por la mala comprensión de la lógica de nuestro lenguaje.

Cada obra de Wittgenstein representa su esfuerzo de superar estas confusiones y
propone un método para remediarlas. Su primera propuesta plantea que el modo de
aclarar las confusiones de los problemas filosóficos consiste en identificar en
el lenguaje el límite de lo que expresa pensamiento; lo que queda al otro lado
de esta frontera sería simplemente sinsentido. En otras palabras:
\citalitinterlin{ Lo que \ifdraft{ \todo{ traducción difícil: \emph{``What can
        be said at all''} } }{} siquiera puede ser dicho puede ser dicho
  claramente; y de lo que uno no puede hablar, de eso, uno debe guardar
  silencio}.\footcite[prefacio]{tractatus}

Con esta expresión Wittgenstein resumió el significado del \emph{'Tractatus
Logico\=/Philosophicus'}.

\subsection{El gran tratado de Wittgenstein}

\ifdraft{\subsubsection{De Manchester a Cambridge}}{}

\pnote{El propósito de recorrer el desarrollo que lleva al Tractatus es ofrecer
  un trasfondo a los puntos que resaltamos más adelante.}

Los primeros esfuerzos de Wittgenstein por escribir una obra sobre filosofía
habían comenzado en 1911. En otoño de ese año en lugar de continuar sus estudios
de ingeniería en Manchester, determinó irse a Cambridge donde Bertrand Russell
ofrecía sus lecciones.

Asistió a un término de lecciones con Russell y al finalizar no estaba seguro de
abandonar la ingeniería por la filosofía, se cuestionaba si verdaderamente tenía
talento para ella. Consultó a su nuevo profesor al respecto y éste le pidió que
escribiera algo para ayudarle a hacer un juicio.

En enero de 1912 Wittgenstein regresó a Cambridge con un manuscrito que
demostraba auténtica agudeza filosófica. Convencido de su gran capacidad,
Russell alentó a Ludwig a continuar dedicándose a la filosofía. Este apoyo fue
crucial para Wittgenstein, hecho puesto de manifiesto por el gran empeño con el
que trabajó en sus estudios aquel curso. Al finalizar el termino Russell alegaba
que Ludwig había aprendido todo lo que él podía enseñarle.\footcite[cap. 3 loc
865]{monk}

\ifdraft{\subsubsection{A Noruega a Resolver los problemas de la lógica}}{}
Después de una temporada en Cambridge llena de eventos y desarrollos
Wittgenstein anunció en septiembre de 1913 sus planes de retirarse para
dedicarse exclusivamente a trabajar en resolver los problemas fundamentales de
la lógica. Su idea era irse a Noruega, a algún lugar apartado, ya que pensaba
que en Cambridge las interrupciones obstaculizarían su trabajo.\footcite[cap. 4
loc 1844]{monk}

\ifdraft{\subsubsection{La Gran Guerra}}{} El trabajo en Noruega fue escabroso.
En el verano de 1914 interrumpió su tarea para tomar un receso en
Viena.\footcite[cap. 5 loc 2154]{monk} Había planificado regresar a Noruega
después del verano, sin embargo la tensión entre las potencias europeas,
agravada desde el atentado de Sarajevo a finales de junio de aquel año, detonó
en el estallido de la Gran Guerra. El 7 de agosto de 1914 Wittgenstein se
enlistaba como voluntario al servicio militar. Sería en las trincheras donde
culminaría su gran tratado filosófico.

El 22 de octubre de 1915 Wittgenstein escribió a Russell desde el taller de
artillería en Sokal, al norte de Lemberg, con lo que sería una primera versión
de su libro.\footcite[cf. p.84]{cambridgeletters} Cuatro años más tarde, el 13
de marzo, escribía a Russell desde Cassino donde se hallaba como prisionero de
guerra en un campamento italiano\footcite[cf. p.268]{mcguinness}: 
\citalitlar{He escrito un libro llamado ``Logisch-Philosophische Abhandlung''
  que contiene todo mi trabajo de los últimos seis años. Creo que finalmente
  he resuelto todos nuestros problemas. Esto puede sonar arrogante, pero no
  puedo evitar creerlo. Terminé el libro en agosto de 1918 y dos meses más
  tarde fui hecho 'Prigioniere'.\footcite[p.89]{cambridgeletters}}

\ifdraft{\subsubsection{Aire de Misticismo}}{}
En junio de aquel año logró enviar el manuscrito del libro a Russell por medio
de John Maynard Keynes quien intervino con las autoridades italianas para
permitir el envío seguro del texto\footcite[p.90 y 91]{cambridgeletters}. El 26
de agosto de 1919 fue oficialmente liberado de sus funciones
militares\footcite[p.277]{mcguinness} y en diciembre finalmente pudo encontrarse
con Russell en la Haya. De aquel encuentro Russell escribe:
\citalitlar{Había sentido un sabor a misticismo en su libro, pero me quedé
    asombrado cuando vi que se ha convertido en un completo místico. Lee a gente
    como Kierkergaard y Angelus Silesius, y ha contemplado seriamente el
    convertirse en un monje. Todo comenzó con ``Las variedades de la experiencia
    religiosa'' de William James y creció durante el invierno que pasó solo en
    Noruega antes de la guerra cuando casi se había vuelto loco. Luego, durante
    la guerra, algo curioso ocurrió. Estuvo de servicio en el pueblo de Tarnov
    en Galicia, y se encontró con una librería que parecía contener solamente
    postales. Sin embargo, entró y encontró que tenían un sólo libro: Los
    Evangelios abreviados de Tolstoy. Compró el libro simplemente porque no
    había otro. Lo leyó y releyó y desde entonces lo llevaba siempre consigo,
    estando bajo fuego y en todo momento. Aunque en su conjunto le gusta menos
    Tolstoy que Dostoeweski. Ha penetrado profundamente en místicos modos de
    pensar y sentir, aunque pienso que lo que le gusta del misticismo es su
    poder para hacerle dejar de pensar. No creo que realmente se haga monje, es
    una idea, no una intención. Su intención es ser profesor. Repartió todo su
    dinero entre sus hermanos y hermanas, pues encuentra que las posesiones
    terrenales son una carga. \footcite[p. 112]{cambridgeletters}}

\ifdraft{\subsubsection{En busca de una experiencia religiosa}}{}
Cuando Wittgenstein se enlistó en el ejercito para la guerra en 1914 tenía
motivaciones más complejas que la defensa de su patria.\footcite[loc2276]{monk}
Sentía que, de algún modo, la experiencia de encarar la muerte le haría mejor
persona. Había leído sobre el valor espiritual de confrontarse con la muerte en
``Las variedades de la experiencia religiosa'':
\citalitlar{No importa cuales sean las fragilidades de un hombre, si estuviera
    dispuesto a encarar la muerte, y más aún si la padece heroicamente, en el
    servicio que éste haya escogido, este hecho le consagra para
    siempre.\footcite[loc 2295]{monk}}

Wittgenstein esperaba esta experiencia religiosa de la guerra.
\citalitinterlin{Quizás}, escribía en su diario, \citalitinterlin{La cercanía de
    la muerte traerá luz a la vida. Dios me ilumine.}\footcite[loc2295]{monk}
La guerra había coincidido con esta época en la que el deseo de convertirse en
una persona diferente era más fuerte aún que su deseo de resolver los problemas
fundamentales de la lógica.\footcite[loc2305]{monk}

\ifdraft{\subsubsection{La Principal Contienda}}{}
Esta transformación sorprendió a Russell en aquel encuentro en la Haya, pero
además fue motivo de confusión en la tarea de entender el Tractatus. Cuando
Russell recibió el manuscrito en agosto escribió a Wittgenstein cuestionando
algunos puntos difíciles del texto. En su carta observaba: 
\citalitlar{Estoy convencido de que estás en lo correcto en tu principal
    contienda, que las proposiciones lógicas son tautologías, las cuales no son
    verdad en el mismo modo que las proposiciones
    sustanciales.\footcite[p.96]{cambridgeletters}}

Esta interpretación del texto se ajusta bien a la importancia que había tenido
esta cuestión en las discusiones entre Russell y Wittgenstein. Así lo expresaba
Russell en ``Introducción a la Filosofía Matemática'' publicado en mayo de aquel
año: 
\citalitlar{
    \todo{The importance of “tautology” for a definition of
    mathematics was pointed out to me by my former pupil Ludwig Wittgenstein,
    who was working on the problem. I do not know whether he has solved it, or
    even whether he is alive or dead.} 
    La importancia de la ``tautología'' para una definición de las
    matemáticas me fue señalada por mi ex-alumno Ludwig Wittgenstein, quien
    estaba trabajando en el problema. No sé si lo ha resuelto, o siquera si está
    vivo o muerto.\footcite[p.205]{introtomathphi}} 

Sin embargo para el Tractatus la cuestión sobre las proposiciones lógicas como
tautologías no es ya el tema principal, sino que enfatiza otra cuestión, así
corrige Wittgenstein en su respuesta a la carta de Russell:
\citalitlar{Ahora me temo que realmente no has captado mi principal contienda,
    para lo cual todo el asunto de las proposiciones lógicas es sólo corolario.
    El punto principal es la teoría sobre lo que puede ser expresado por
    proposiciones ---es decir, por el lenguaje--- (y, lo que viene a ser lo mismo,
    aquello que puede ser pensado) y lo que no puede ser expresado por medio de
    proposiciones, sino solamente mostrado; lo cual, creo, es el problema
    cardinal de la filosofía\ldots \footcite[p. 98]{cambridgeletters}}

Esta respuesta de Wittgenstein no solo pone de manifiesto su cambio de enfoque,
sino que ofrece una clave para introducirse en su obra. 

%CUARTA CUESTIÓN: LA ``DOCTRINA'' DEL TRACTATUS
%1. La filosofía como actividad
%2. El pensamiento como representación
%3. Los polos de verdad y falsedad de las proposiciones
%4. La diferencia ente decir y mostrar

\subsection{Las elucidaciones del Tractatus}
\todo{Este párrafo resume los cuatro puntos del Tractatus que se desglosarán en
    los próximos párrafos} 
Desde las proposiciones principales del Tractatus queda claro que el tema
central del libro es la conexión entre el lenguaje, o el pensamiento, y la
realidad.  
\todo{1.Filosofía como actividad}
En este nexo es donde la actividad filosófica ha de buscar esclarecer el
pensamiento.
\todo{2.El pensamiento como representación}
La tesis básica sobre esta relación consiste en que las proposiciones, o su
equivalente en la mente, son imágenes de los hechos.
\todo{3.Las proposiciones como proyecciones con polos de verdad-falsedad}
La proposición es la misma imagen tanto si es cierta como si es falsa, es decir,
es la misma imagen sin importar que lo que se corresponde a ésta es el caso que
es cierto o no. El mundo es la totalidad de los hechos, a saber, de lo
equivalente en la realidad a las proposiciones verdaderas.
\todo{4.La distinción entre el decir y el mostrar}
Sólo las situaciones que pueden ser plasmadas en imágenes pueden ser afirmadas
en proposiciones. Adicionalmente hay mucho que es inexpresable, lo cual no
debemos intentar enunciar, sino más bien contemplar sin palabras.\footcite[cf.
p.19]{IWT}

\ifdraft{\subsubsection{La filosofía como actividad}}{}

La filosofía es la actividad que tiene como objeto la clarificación lógica
de los pensamientos.\footcite[4.112 p. 52]{tractatus} El problema de muchas de
las proposiciones y preguntas que se han escrito acerca de asuntos filosóficos
no es que sean falsas, sino carentes de significado. Wittgenstein continúa: 
\citalitlar{4.003~En consecuencia no podemos dar respuesta a preguntas de este
    tipo, sino exponer su falta de sentido. Muchas cuestiones y proposiciones de
    los filósofos resultan del hecho de que no entendemos la lógica de nuestro
    lenguaje. (Son del mismo genero que la pregunta sobre si lo Bueno es más o
    menos idéntico a lo Bello). Y así no hay que sorprenderse ante el hecho de
    que los problemas más profundos realmente no son problemas.\footcite[4.003
    p. 45]{tractatus}} 

Es así que el precipitado de la reflexión filosófica que el Tractatus recoge no
pretende componer un cuerpo doctrinal articulado por proposiciones filosóficas,
sino más bien ofrecer `elucidaciones' que sirven como etapas escalonadas y
transitorias que al ser superadas conducen a ver el mundo correctamente. Este
esfuerzo hace de pensamientos opacos e indistintos unos claros y con límites
bien definidos.\footcite[cf. 4.112 y 6.54]{tractatus} 
La posibilidad de llegar a una visión clara del mundo es fruto de la posibilidad
de lograr aclarar la lógica del lenguaje. El lenguaje, a su vez, está compuesto
de la totalidad de las proposiciones, y éstas, cuando tienen sentido,
representan el pensamiento.\footcite[cf. 4 y 4.001]{tractatus} 
Sin embargo, el mismo lenguaje que puede expresar el pensamiento lo disfraza:

\citalitlar{4.002~El lenguaje disfraza el pensamiento; de tal manera que de la
    forma externa de sus ropajes uno no puede inferir la forma del pensamiento
    que estos revisten, porque la forma externa de la vestimenta esta elaborada
    con un propósito bastante distinto al de favorecer que la forma del cuerpo
    sea conocida.}

El intento de llegar desde el lenguaje al pensamiento por medio de las
proposiciones con significado es el esfuerzo por conocer una imagen de la
realidad. El pensamiento es la imagen lógica de los hechos, en él se contiene la
posibilidad del estado de las cosas que son pensadas y la totalidad de los
pensamientos verdaderos es una imagen del mundo.\footcite[cf.][3 y
3.001]{tractatus}

\ifdraft{\subsubsection{El pensamiento como representación}}{}

El pensamiento es representación de la realidad por la identidad existente entre
la posibilidad de la estructura de una proposición y la posibilidad de la
estructura un hecho:

\citalitlar{Los objetos ---que son simples--- se combinan en situaciones
    elementales. El modo en el que se sujetan juntos en una situación tal es su
    estructura. Forma es la posibilidad de esa estructura. No todas las
    estructuras posibles son actuales: una que es actual es un `hecho
    elemental'. Nosotros formamos imágenes de los hechos, de hechos posibles
    ciertamente, pero algunos de ellos son actuales también. Una imagen consiste
    en sus elementos combinados en un modo específico. Al estar así presentan a
    los objetos denominados por ellos como combinados específicamente en ese
    mismo modo. La combinación de los elementos de la imagen ---la combinación
    siendo presentada--- se llama su estructura y su posibilidad se llama la
    forma de representación de la imagen.   
    Esta `forma de representación' es la posibilidad de que las cosas están
    combinadas como lo están los elementos de la imagen.
    \footnote{\cite[cf.][p.~171]{simplicity}; \cite[n.~2.15]{tractatus}}}

La representación y los hechos tienen en común la forma lógica:
\citalitlar{2.18~Lo que toda representación, de una forma cualquiera, debe tener
    en común con la realidad, de manera que pueda representarla ---cierta o
    falsamente--- de algún modo, es su forma lógica, esto es, la forma de la
    realidad.\footcite[p.34]{tractatus}}

\ifdraft{\subsubsection{Las proposiciones como proyecciones con polos de verdad-falsedad}}{}
    \todo{Añadir analogía sobre la verdad ---si es que no se va a usar en el próximo
    apartado---}
    La imagen de la realidad se convierte en proposición en el momento en que
    nosotros correlacionamos sus elementos con las cosas
    actuales.\footcite[cf.~][p.~73]{IWT}
    La condición de posibilidad de entablar dicha correlación es la relación interna
    entre los elementos de la imagen en una estructura con
    sentido.\footcite[cf.~][p.~68]{IWT}
    De este modo:
    \citalitlar{5.4733~Frege dice: Toda proposición legítimamente construida tiene
        que tener un sentido; y yo digo: Toda proposición posible está legítimamente
        construida, y si ésta no tiene sentido es sólo porque no hemos dado
        significado a alguna de sus partes constitutivas. (Incluso cuando pensemos
        que lo hemos hecho.)\footcite[p.~78]{tractatus}}

    La proposición expresa el pensamiento perceptiblemente por medio de signos.
    Usamos los signos de las proposiciones como proyecciones del estado de las cosas
    y las proposiciones son el signo proposicional en su relación proyectiva con el
    mundo. A la proposición le corresponde todo lo que le corresponde a la
    proyección, pero no lo que es proyectado, de tal modo, que la proposición no
    contiene aún su sentido, sino la posibilidad de expresarlo; la forma de su
    sentido, pero no su contenido.\footcite[cf.~][3.1,3.11-3.13]{tractatus} 

    La proposición no `contiene su sentido' porque la correlación la hacemos nosotros,
    al `pensar su sentido'. Hacemos esto cuando usamos los elementos de la
    proposición para representar los objetos cuya posible configuración estamos 
    reproduciendo en la disposición de los elementos de la proposición. Esto es lo
    que significa que la proposición sea llamada una imagen de la
    realidad.\footcite[cf.~][p.69]{IWT}  

    Toda proposición-imagen tiene dos acepciones. Puede ser una descripción de
    la existencia de una configuración de objetos o puede ser una descripción de la
    no-existencia de una configuración de objetos.\footcite[cf.~][p.~72]{IWT} 
    %Es una peculiaridad de la proyección el que de ésta y del método de proyección
    %se puede decir qué es lo que se está proyectando, sin que sea necesario que tal
    %cosa exista físicamente.\footcite[cf.~][p.~72]{IWT} 
    %La idea de la proyección es peculiarmente apta para explicar el carácter de una
    %proposición como teniendo sentido independientemente de los hechos, como
    %inteligible aún antes de que se sepa que es cierta; como algo que concierne lo
    %que se puede cuestionar sobre si es verdad, y saber lo que se pregunta antes de
    %conocer la respuesta.\footcite[cf.~][p.~73]{IWT}
    Esta doble acepción es el resultado de que la proposición-imagen puede ser una
    proyección hecha en sentido positivo o negativo.\footcite[cf.~][p.~74]{IWT} Esto
    queda ilustrado en una analogía:

    \citalitlar{4.463~La proposición, la imagen, el modelo, son en el sentido
        negativo como un cuerpo solido, que restringe el libre movimiento de otro:
        en el sentido positivo, son como un espacio limitado por una sustancia
        sólida, en la cual un cuerpo puede ser colocado.\footcite[p.~63]{tractatus}}

    De este modo toda proposición-imagen tiene dos polos; de verdad y de falsedad.
    Las tautologías y las contradicciones, por su parte, no son imagenes de la
    realidad ya que no representan ningún posible estado de las cosas. Así continúa
    la ilustración anterior:

    \citalitlar{4.463~Una tautología deja abierto para la realidad el total infinito
        del espacio lógico; una contradicción llena el total del espacio lógico no
        dejando ningún punto de él para la realidad. Así pues ninguna de las dos
        puede determinar la realidad de ningún modo.\footcite[p.~78]{tractatus}}

    La verdad de las proposiciones es posible, de las tautologías es cierta y de las
    contradicciones imposible. La tautología y la contradicción son los casos límite
    de la combinación de signos ---específicamente--- su
    disolución.\footcite[cf.~][4.464 y 4.466]{tractatus} Las tautologías son
    proposiciones sin sentido (carecen de polos de verdad y falsedad), su negación son
    las contradicciones. Los intentos de decir lo que sólo puede ser mostrado
    resultan en esto, en formaciones de palabras que carecen de sentido, es decir,
    son formaciones que parecen oraciones, cuyos componentes resultan no tener
    significado en esa forma de oración.\footcite[cf.~][p.~163~\S2]{IWT}.

\ifdraft{\subsubsection{La distinción entre el decir y el mostrar}}{}
La conexión entre las tautologías y aquello que no se puede decir, sino
mostrar, es que éstas ---siendo proposiciones lógicas sin sentido--- muestran
la 'lógica del mundo'.\footcite[cf.~][p.~163~\S3]{IWT}. Esta 'lógica del
mundo' o 'de los hechos' es la que más prominentemente aparece en el Tractatus
entre las cosas que no pueden ser dichas, sino mostradas. Esta lógica no solo
se muestra en las tautologías, sino en todas las proposiciones. Queda exhibida
en las proposiciones diciendo aquello que pueden decir.

La forma lógica no puede expresarse desde el lenguaje, pues es la forma del
lenguaje mismo, se hace manifiesta en éste, no es representativa de los objetos
y tampoco puede ser representada por signos, tiene que ser mostrada:
\citalitlar{4.0312~La posibilidad de las proposiciones se basa en el principio de
    la representación de los objetos por medio de signos. Mi pensamiento
    fundamental es que las ``constantes lógicas'' no son representativas. Que la
    lógica de los hechos no puede ser representada.\footcite[p.~48]{tractatus}}

La lógica es, por tanto, trascendental, no en el sentido de que las
proposiciones sobre lógica afirmen verdades trascendentales, sino en que todas
las proposiciones muestran algo que permea todo lo decible, pero es en sí mismo
indecible.\footcite[cf.~][p.~166 \S2]{IWT}

Otra cuestión notoria entre aquello que no puede ser dicho, sino mostrado es la
cuestión acerca de la verdad del solipsismo. Los limites del mundo son los
límites de la lógica, lo que no podemos pensar, no podemos pensarlo, y por tanto
tampoco decirlo. Los límites de mi lenguaje significan los límites de mi
mundo.\footcite[cf~.][5.6~y~5.61]{tractatus} De este modo:
\citalitlar{5.62~[\ldots]Lo que el solipsismo \emph{significa}, es ciertamente
    correcto, sólo que no puede ser \emph{dicho}, pero se muestra a sí
    mismo. Que el mundo es \emph{mi} mundo, se muestra a sí mismo en el hecho
    de que los limites del lenguaje (de \emph{aquel} lenguaje que yo
    entiendo) significan los límites de mi
    mundo.\footcite[cf~.][p.~89]{tractatus}} 

Así como la lógica del mundo y la verdad del solipsismo quedan mostradas,
también, las verdades éticas y religiosas, aunque no expresables, se manifiestan
a sí mismas en la vida. 

Existe, por tanto lo inexpresable que se muestra a sí mismo, esto es lo
místico.\footcite[cf.~][6.522]{tractatus}

De la voluntad como sujeto de la ética no podemos
hablar\footcite[cf.~][6.423]{tractatus}. El mundo es independiente de nuestra
voluntad ya que no hay conexión lógica entre ésta y los hechos.
La voluntad y la acción como fenómenos, por tanto, interesan sólo a la
psicología.\footcite[cf.~][p.171 \S3]{IWT}

El significado del mundo tiene que estar fuera del
mundo\footcite[cf.~][6.41]{tractatus} y Dios no se revela \emph{en} el
mundo\footcite[cf.~][6.432]{tractatus}. 
Esto se sigue de la teoría de la representación; una proposición y su negación
son ambas posibles, cuál es verdad es accidental.\footcite[cf.~][p.170 \S4]{IWT}
Si hay un valor que valga la pena para el mundo tiene que estar fuera de lo que
es el caso que es; lo que hace que el mundo tenga un valor no-accidental tiene
que estar fuera de lo accidental, tiene que estar fuera del
mundo.\footcite[cf.~][6.41]{tractatus} 

Finalmente, aplicar el límite de lo que puede ser expresado a la actividad
filosófica significa que:
\citalitlar{6.53~El método correcto para la filosofía sería este. No decir nada
    excepto lo que pueda ser dicho, esto es, proposiciones de la ciencia
    natural, es decir, algo que no tiene nada que ver con la filosofía: y luego
    siempre, cuando alguien quiera decir algo metafísico, demostrarle que no ha
    logrado dar significado a ciertos signos en sus proposiciones. Este método
    sería insatisfactorio para la otra persona ---no tendría la impresión de que
    le estuviéramos enseñando filosofía--- pero este método sería el único
    estrictamente correcto.\footcite[p. 107--108]{tractatus}}
\todo{Añadir como conclusión del resumen la finalidad ética del tratado.}

\subsection{Formación filosófica de Elizabeth}

\ifdraft{\subsubsection{De Wittgenstein a Anscombe}}{} 
En el 1929 Wittgenstein presentó el Tractatus Logico\=/Philosophicus como su
tesis doctoral en Cambridge. Ese mismo año fue designado como profesor en
``Trinity College'', allí estaría hasta 1936.

\ifdraft{\subsubsection{Causalidad reflexiones iniciales de Anscombe}}{}

Por aquella época de mediados de los 30 la joven Gertrude Elizabeth Margaret
Anscombe, andaba buscando un buen argumento que demostrara que todo lo que
existe tiene que tener una causa. ¿Por qué cuando algo ocurre estamos seguros de
que tiene una causa? Nadie sabía darle una respuesta.\footcite[cf.~][p.~vii
\S1]{M&PotM} Así, sin darse cuenta, se iniciaba Anscombe en la ardua tarea de la
filosofía. Rigurosa y enérgica desde el principio.

El origen de su peculiar curiosidad por la causalidad se hallaba en una obra
llamada `Teología Natural' escrita por un jesuita del siglo XIX. Había llegado a
este libro motivada por su conversión a la Iglesia
Católica.\footcite[cf.~][p.~vii \S1]{M&PotM} El tratado le resultó problemático
en dos cuestiones.

La primera fue la doctrina de la \emph{`scientia media'}, según la cual Dios
tiene conocimiento, por ejemplo, de lo que alguien podría haber hecho si no
hubiera muerto cuando murió. A Elizabeth le parecía que lo que hubiera ocurrido
si lo que pasó no hubiera pasado simplemente no existe; no hay qué conocer. Y no
podía creer esto. Anscombe tuvo la oportunidad de discutir esta preocupación con
Richard Kehoe durante su preparación religiosa en su primer año en Oxford. La
dificultad para creer aquella doctrina le parecía un límite para aceptar la fe
católica. Richard le aclaró que no hacía falta que creyera en eso. Con el tiempo
entendió que se trataba de una discusión de escuela, en la que los jesuítas y
dominicos entablaron una ardua disputa y que la postura que ella había adoptado
era la defendida por los dominicos.\footcite[cf.~][p.~vii]{M&PotM}

La segunda cuestión problematica la encontró en un argumento sobre la existencia
de la `Causa Primera'. El tratado ofrecía como preliminar al argumento una
demostración de un `principio de causalidad' según el cual todo cuanto existe
tiene que tener una causa. Anscombe notó, escasamente escondido en una premisa,
un presupuesto de la conclusión del propio argumento. Aquel ``petitio
principii'' le pareció un simple descuido y resolvió, por tanto, escribir una
versión mejorada de la demostración. Durante los siguientes dos o tres años
produjo unas cinco versiones que le parecían satisfactorias, sin embargo
eventualmente descubría que contenían la misma falacia, cada vez disimulada más
astutamente. Todo este esfuerzo lo realizó sin ninguna enseñanza formal en
filosofía, incluso su último intento de argumento lo hizo antes de estudiar
`Greats'.\footcite[cf.~][p.~vii \S2]{M&PotM}

\ifdraft{\subsubsection{Oxford: La Percepción y el fenomenalismo de Price}}{}

Sus lecturas en torno a su conversión fueron motivo de más reflexiones. Esta
vez, como fruto de su lectura de `The Nature of Belief' de Martin D'Arcy, se
interesó por el tema de la percepción. Durante años ocupaba su tiempo, en
cafeterías, por ejemplo, mirando fijamente objetos, diciendose a sí misma: 'Veo
un paquete. ¿Pero qué veo realmente? ¿Cómo puedo decir que veo algo más que una
extensión amarilla?\footcite[cf.~][p.~viii \S1]{M&PotM}

Al principio su impresión era que lo que veía eran objetos:
\citalitinterlin{Estaba segura de que veía objetos, como paquetes de cigarrillos
  o tazas o\ldots~cualquier cosa más o menos sustancial
  servía.}\footcite[p.~viii \S1]{M&PotM} Además creía que debemos de conocer la
categoría de un objeto cuando hablamos de él, eso corresponde a la lógica del
término usado para hablar del objeto y no de algún descubrimiento empírico.
Estas ideas, sin embargo, las había desarrollado fijándose en artefactos
urbanos. Los ejemplos de percepción de la naturaleza que más la impactaron
fueron `madera' y el cielo. Este último le hizo retractarse de su creencia sobre
el conocimiento lógico de la categoría de los objetos.\footcite[cf.~][p.~viii
\S1]{M&PotM}

Sus indagaciones sobre la percepción, así como le ocurrió con la causalidad,
fueron previas al periodo de `Greats' donde estudiaría formalmente la filosofía.
Ya desde `Mods' asistía a las lecciones de H. H. Price sobre percepción y
fenomenalismo. De todos los que escuchó en Oxford fue quién le inspiró mayor
respeto, no porque estuviera de acuerdo con lo que decía, sino porque hablaba de
lo que había que hablar. El único libro suyo que le pareció realmente bueno fue
\emph{Hume's Theory of the External World} y lo leyó sin interrupción de
principio a fin. Fue Price quien despertó en ella un intenso interés por el
capítulo de Hume sobre ``Del escepticismo con respecto a los sentidos''. Aunque
le parecía que Price tendía a suavizar a Hume, el hecho de que escribiera sobre
él le parecia que era escribir sobre las cosas mismas que merecía la pena
discutir. Asncombe, sin embargo, odiaba el fenomenalismo y se sentía atrapada
por él, pero no sabía salir de él, o rebatirlo. La postura escéptica tampoco la
convencía como para adoptarla y no la dejaba satisfecha. Esta insatisfacción no
haría más que crecer en sus años en Oxford.\footcite[cf.~][p.~viii \S1]{M&PotM}

\ifdraft{\subsubsection{En Cambrdige con Wittgenstein}}{}

Fue en las clases de Wittgenstein que el pensamiento central ``Tengo esto, y
 defino `amarillo' como esto'' fue efectivamente atacado.

 En una ocasión en estas clases Wittgenstein estaba discutiendo la interpretación
 del letrero\footcite[p.~86~\S198]{PI}, y estalló en mi que el modo en que vas
 según éste es la interpretación final.

 En otra ocasión salí con ``Pero todavía quiero decir: <<Azul esta ahí>>''.
 Wittgenstein respondió: <<Déjame pensar qué medicina necesitas\ldots>> <<Supón
 que tenemos la palabra `painy' ``(dolorante/doloreño)'', como una palabra para
 la propiedad de ciertas superficies>>. La medicina fue efectiva. Si dolorante
 fuera una palabra posible para una cualidad secundaria, ¿no podría el mismo
 motivo conducirme a decir: Dolorante esta aquí que lo que me condujo a decir
 azul está aquí? Mi expresión no significaba que ``azul'' es el nombre de esta
 sensación que estoy teniendo, ni cambié a ese pensamiento.

Anscombe conoció a Wittgenstein en los años culminantes de su pensamiento
filosófico. 
Al comienzo de sus lecciones en 1944 Wittgenstein escribía a su amigo Rush Rhees:
\citalitinterlin{
    \ldots mis clases no han ido tan mal. Thouless esta asistiendo, y una mujer, 
    'Mrs so and so'
    que se llama a sí misma 
    'Miss Anscombe',
    que ciertamente es inteligente, aunque no del calibre de Kreisel.
    \footcite[p.~371]{cambridgeletters}
}
Un año mas tarde escribía a Norman Malcolm:
\citalitinterlin{
    \ldots mi clase ahora es bastante grande, 19 personas. \ldots Smythies esta
    viniendo, y una mujer que es muy buena, es decir, más que solamente
    inteligente\ldots 
    \footcite[p.~388]{cambridgeletters}
}
Aquellos años no sólo creció en Wittgenstein la apreciación de la capacidad de
Anscombe, sino que se afianzó entre ellos una estrecha amistad. 

La influencia de Wittgenstein fue decisiva para el desarrollo filosófico de
Elizabeth. Las lecciones con Wittgenstein eran directas y con franqueza. Esta
metodología carente de cualquier parafernalia era inquietante para algunos,
inspiradora para otros, pero fue tremendamente liberadora para
ella.\footcite[loc 9853 Chapter 4, Section 24, \S5]{monk} Esta libertad
quedaba demostrada en que Anscombe no se contentaba con repetir lo que decía
Wittgenstein, sino que pensaba por sí misma; en esto precisamente era más fiel
al espíritu de la filosofía que había aprendido de él. Sobre esta relación,
Phillipa Foot, amiga de ambos, cuenta que durante mucho tiempo sostuvo
objeciones a las afirmaciones de Wittgenstein, eventualmente, un comentario de
Norman Malcom la hizo pensar que podía haber valor en lo que Wittgenstein decía.
Cuestionó entonces a Anscombe: 
``¿Por qué no me dijiste?'', ella le contestó: ``Porque es importante que uno
tenga sus resistencias''. Anscombe evidentemente pensaba ---continúa Foot: 
\citalitlar{
    que un largo periodo de vigorosa objeción era la mejor manera de entender a
    Wittgenstein. Aun cuando era su amiga cercana y albacea literaria, y una de
    los primeros en reconocer su grandeza, nada podía ser más lejano de su
    carácter y modo de pensamiento que el discipulado.\footcite[p.~4]{teichmann}
}


\pnote{introducir algunos contrastes y relaciones entre Anscombe y Wittgenstein
    para explicar la incursión en la vida/pensamiento de W.}

\subsection{Wittgenstein y la fe}
\todo{En casa de Anscombe, hablando de la fe}
\todo{From IWT: la verdad de la teoría de la imagen sería el fin de la teología
    natural} 
\todo{Inquietud respecto del esfuerzo de explicar racionalmente la fe} 
\todo{Necesidad de contexto}

\begin{revision}
Es una gran bendición para mi poder trabajar hoy. ¡Pero cuán fácilmente olvido
todas mis bendiciones!
Estoy leyendo: ``Y ningún hombre puede decir Jesús es el Señor, sino el Espíritu
Santo.''(1Co 3) Y es cierto: Yo no puedo llamarlo \emph{Señor}; porque eso no me
dice absolutamente nada. Sí podría llamarlo 'el ejemplo por excelencia', 'Dios'
incluso o quizás: puedo entenderlo cuando es llamado de ese modo; pero Yo no
puedo pronunciar la palabra ``Señor'' significativamente. \emph{Porque yo no
creo} que el vendrá a juzgarme; porque \emph{eso} no me dice nada. Y sólo me
diría algo si yo viviera de un modo considerablemente distinto.

¿Qué me hace inclinarme incluso a mi a creer en la resurrección de Cristo?
Entretengo la idea por así decirlo. ---Si él no ha resucitado de los muertos,
entonces se descompuso en la tumba como cualquier otro ser humano. \emph{Esta
muerto y descompuesto.} En ese caso es un maestro, como cualquier otro y
entonces ya no puede \emph{ayudar} más; y estamos una vez más huérfanos y solos.
Y tengo que arreglármelas con la sabiduría y la especulación. Es como si
estuvieramos en un infierno, en el que solo podemos soñar y estamos dejados
fuera del cielo, atrapados bajo el techo, diriamos. Pero si REALMENTE voy a ser
redimido, ---necesito \emph{certeza}--- no sabiduría, sueños, especulación--- y
esta certeza es la fe. Y fe es fe en lo que mi \emph{corazón}, mi \emph{alma},
necesita, no mi intelecto especulativo. Pues mi alma, con sus pasiones, con su
carne y sangre, diría, tiene que ser redimida, no mi mente abstracta. Quizás uno
podría decir: Sólo el \emph{amor} puede creer la Resurrección. O: es el
\emph{amor} lo que cree la Resurrección. Uno puede decir: el amor redentor cree
incluso en la Resurrección; se sostiene firme incluso hasta la Resurrección. Lo
que lucha con la duda es, por decirlo de algún modo, la redención. Sostenerse
firmemente en esto tiene que ser mantenerse firme en esta creencia. Así esto
significa: primero se redimido y sujétate firmemente de tu redención (sostente en tu
redención) --- entonces veras que a lo que te estás sujetando es a esta
creencia. Así que esto sólo puede ocurrir si ya no te sujetas de esta tierra,
sino que te suspendes desde el cielo. Entonces \emph{todo} es distinto y 'no
será sorpresa' el que puedas hacer entonces lo que ahora no puedes. (Es verdad
que alguien que está suspendido se ve como alguien que está de pie, pero la
interacción de fuerzas dentro de él es sin embargo una completamente distinta, y
de ahí que sea capaz de hacer cosas bastante distintas de las que puede hacer
alguien que está de pie). (Culture and Value p.38-39 MS 120 108 c: 12.12.1937)
\end{revision}
