%SECCIÓN 1: 
\section{Actividad Filosófica de Elizabeth Anscombe}

\subsection{Los primeros arduos esfuerzos}

\ifdraft{\subsubsection{De Wittgenstein a Anscombe}}{} 
En el 1929 Wittgenstein presentó el Tractatus Logico\=/Philosophicus como su
tesis doctoral en Cambridge. Ese mismo año fue designado como profesor en
``Trinity College'', allí estaría hasta 1936.

\ifdraft{\subsubsection{Causalidad reflexiones iniciales de Anscombe}}{}

Por aquella época de mediados de los 30 la joven Gertrude Elizabeth Margaret
Anscombe, andaba buscando un buen argumento que demostrara que todo lo que
existe tiene que tener una causa. ¿Por qué cuando algo ocurre estamos seguros de
que tiene una causa? Nadie sabía darle una respuesta.\autocite[cf.~][p.~vii
]{anscombe1981metaphysicsintro} Así, sin darse cuenta, se iniciaba en la ardua
tarea de la filosofía. Rigurosa y enérgica desde el principio.

El origen de su peculiar curiosidad por la causalidad se hallaba en una obra
llamada `Teología Natural' escrita por un jesuita del siglo XIX. Había llegado a
este libro motivada por su conversión a la Iglesia
Católica.\autocite[cf.~][p.~vii]{anscombe1981metaphysicsintro} El tratado le
resultó problemático en dos cuestiones.

La primera fue la doctrina de la \emph{`scientia media'}, según la cual Dios
tiene conocimiento, por ejemplo, de lo que alguien podría haber hecho si no
hubiera muerto cuando murió. A Elizabeth le parecía que lo que hubiera ocurrido
si lo que pasó no hubiera pasado simplemente no existe; no hay qué conocer. Y no
podía creer esto. Anscombe tuvo la oportunidad de discutir esta preocupación con
Richard Kehoe durante su preparación religiosa en su primer año en Oxford. La
dificultad para creer aquella doctrina le parecía un límite para aceptar la fe
católica. Richard le aclaró que no hacía falta que creyera en eso. Con el tiempo
entendió que se trataba de una discusión de escuela, en la que los jesuítas y
dominicos entablaron una ardua disputa y que la postura que ella había adoptado
era la qu había sido defendida por los
dominicos.\autocite[cf.~][p.~vii]{anscombe1981metaphysicsintro}

La segunda cuestión problematica la encontró en un argumento sobre la existencia
de la `Causa Primera'. El tratado ofrecía como preliminar al argumento una
demostración de un `principio de causalidad' según el cual todo cuanto existe
tiene que tener una causa. Anscombe notó, escasamente escondido en una premisa,
un presupuesto de la conclusión del propio argumento. Aquel \emph{petitio
  principii} le pareció un simple descuido y resolvió, por tanto, escribir una
versión mejorada de la demostración. Durante los siguientes dos o tres años
produjo unas cinco versiones que le parecían satisfactorias, sin embargo
eventualmente descubría que contenían la misma falacia, cada vez disimulada más
astutamente. Todo este esfuerzo lo realizó sin ninguna enseñanza formal en
filosofía, incluso su último intento de argumento lo hizo antes de estudiar
`Greats'.\autocite[cf.~][p.~vii]{anscombe1981metaphysicsintro}

\ifdraft{\subsubsection{Oxford: La Percepción y el fenomenalismo de Price}}{}

Sus lecturas en torno a su conversión fueron motivo de más reflexiones. Esta
vez, como fruto de \emph{The Nature of Belief} de Martin D'Arcy, se interesó por
el tema de la percepción. Durante años ocupaba su tiempo, en cafeterías, por
ejemplo, mirando fijamente objetos, diciendose a sí misma: <<Veo un paquete.
¿Pero qué veo realmente? ¿Cómo puedo decir que veo algo más que una extensión
amarilla?>>\autocite[cf.~][p.~viii]{anscombe1981metaphysicsintro}

Al principio su impresión era que lo que veía eran objetos:
\citalitinterlin{Estaba segura de que veía objetos, como paquetes de cigarrillos
  o tazas o\ldots~cualquier cosa más o menos sustancial
  servía.}\autocite[p.~viii]{anscombe1981metaphysicsintro} Además creía que
debemos de conocer la categoría de un objeto cuando hablamos de él, eso
corresponde a la lógica del término usado para hablar del objeto y no de algún
descubrimiento empírico. Estas ideas, sin embargo, las había desarrollado
fijándose en artefactos urbanos. Los ejemplos de percepción de la naturaleza que
más la impactaron fueron `madera' y el cielo. Este último le hizo retractarse de
su creencia sobre el conocimiento lógico de la categoría de los
objetos.\autocite[cf.~][p.~viii]{anscombe1981metaphysicsintro}

Sus indagaciones sobre la percepción, así como le ocurrió con la causalidad,
fueron previas al periodo de `Greats' donde estudiaría formalmente la filosofía.
Ya desde `Mods' asistía a las lecciones de H.H.~Price sobre percepción y
fenomenalismo. De todos los que escuchó en Oxford fue quién le inspiró mayor
respeto, no porque estuviera de acuerdo con lo que decía, sino porque hablaba de
lo que había que hablar. El único libro suyo que le pareció realmente bueno fue
\emph{Hume's Theory of the External World} y lo leyó sin interrupción de
principio a fin. Fue Price quien despertó en ella un intenso interés por el
capítulo de Hume sobre ``Del escepticismo con respecto a los sentidos''. Aunque
le parecía que Price tendía a suavizar a Hume, el hecho de que escribiera sobre
él le parecia que era escribir sobre las cosas mismas que merecía la pena
discutir. Asncombe, sin embargo, odiaba el fenomenalismo y se sentía atrapada
por él, pero no sabía salir de él, o rebatirlo. La postura escéptica tampoco la
convencía como para adoptarla y no la dejaba satisfecha. Esta insatisfacción no
haría más que crecer en sus años en Oxford.
\autocites[cf.~][p.~viii]{anscombe1981metaphysicsintro}
[~y~][p.~26]{torralba2005accion}

\ifdraft{\subsubsection{En Cambrdige con Wittgenstein}}{}

  En las lecciones con Wittgenstein en Cambridge fue que el pensamiento central
  <<Tengo \emph{esto}, y defino `amarillo' como \emph{esto}>> fue efectivamente
  atacado. Anscombe misma lo narra usando dos ejemplos:

  \citalitlar{En cierto punto Wittgenstein estaba discutiendo en sus clases la
    interpretación del letrero (sign-post), y estalló en mi que el modo en que
    vas según éste es la interpretación
    final.\autocite[p.~viii]{andcombe1981metaphysicsintro}}

En \emph{Investigaciones Filosóficas} \S198 

toda interpretación queda sostenida en el aire junto con lo que interpreta, y no
puede darle a ésto ningún apoyo. Las interpretaciones por sí solas no determinan
el significado.

  Aquí Elizabeth se refiere a \autocite[p.~86~\S198]{PI}

  \citalitlar{En otra ocasión salí con: <<Pero todavía quiero decir: ``Azul esta
    ahí''>>. [\ldots] [Wittgenstein] dijo: <<Déjame pensar qué medicina
    necesitas\ldots>> <<Supón que tenemos la palabra \emph{`painy'}, como una
    palabra para la propiedad de ciertas superficies>>. La `medicina' fue efectiva
    [\ldots] Si \emph{`painy'} fuera una palabra posible para una cualidad
    secundaria, ¿no podría el mismo motivo conducirme a decir: \emph{`painy'} esta
    aquí que lo que me condujo a decir azul está aquí? Mi expresión no significaba
    que `azul' es el nombre de esta sensación que estoy teniendo, ni cambié a ese
    pensamiento.\autocite[p.~viii]{andcombe1981metaphysicsintro}}

  The issue's significance can be seen by considering how the argument is
  embedded in the structure of Philosophical Investigations. Immediately
  prior to the introduction of the argument (§§241f), Wittgenstein suggests
  that the existence of the rules governing the use of language and making
  communication possible depends on agreement in human behaviour—such as the
  uniformity in normal human reaction which makes it possible to train most
  children to look at something by pointing at it. (Unlike cats, which react
  in a seemingly random variety of ways to pointing.) One function of the
  private language argument is to show that not only actual languages but
  the very possibility of language and concept formation depends on the
  possibility of such agreement.

  Another, related, function is to oppose the idea that metaphysical
  absolutes are within our reach, that we can find at least part of the
  world as it really is in the sense that any other way of conceiving that
  part must be wrong (cf. Philosophical Investigations p. 230). Philosophers
  are especially tempted to suppose that numbers and sensations are examples
  of such absolutes, self-identifying objects which themselves force upon us
  the rules for the use of their names. Wittgenstein discusses numbers in
  earlier sections on rules (185–242). Some of his points have analogues in
  his discussion of sensations, for there is a common underlying confusion
  about how the act of meaning determines the future application of a
  formula or name. In the case of numbers, one temptation is to confuse the
  mathematical sense of ‘determine’ in which, say, the formula y = 2x
  determines the numerical value of y for a given value of x (in contrast
  with y > 2x, which does not) with a causal sense in which a certain
  training in mathematics determines that normal people will always write
  the same value for y given both the first formula and a value for x—in
  contrast with creatures for which such training might produce a variety of
  outcomes (cf. §189). This confusion produces the illusion that the result
  of an actual properly conducted calculation is the inevitable outcome of
  the mathematical determining, as though the formula's meaning itself were
  shaping the course of events.

  In the case of sensations, the parallel temptation is to suppose that they
  are self-intimating. Itching, for example, seems like this: one just feels
  what it is directly; if one then gives the sensation a name, the rules for
  that name's subsequent use are already determined by the sensation itself.
  Wittgenstein tries to show that this impression is illusory, that even
  itching derives its identity only from a sharable practice of expression,
  reaction and use of language. If itching were a metaphysical absolute,
  forcing its identity upon me in the way described, then the possibility of
  such a shared practice would be irrelevant to the concept of itching: the
  nature of itching would be revealed to me in a single mental act of naming
  it (the kind of mental act which Russell called ‘acquaintance’); all
  subsequent facts concerning the use of the name would be irrelevant to how
  that name was meant; and the name could be private. The private language
  argument is intended to show that such subsequent facts could not be
  irrelevant, that no names could be private, and that the notion of having
  the true identity of a sensation revealed in a single act of acquaintance
  is a confusion.




    \begin{revision}
       ``For a large class of cases of the employment of the word ‘meaning’—though not
       for all—this way can be explained in this way: the meaning of a word is its use
       in the language'' (PI 43). This basic statement is what underlies the change of
       perspective most typical of the later phase of Wittgenstein's thought: a change
       from a conception of meaning as representation to a view which looks to use as
       the crux of the investigation. 
       \end{revision}

      \begin{revision}
      Philosophical Investigations:
      --Undertake an investigation, leading, not to the construction of new and
      surprising theories or explanations, but the examination of our life with
      language. This is a grammatical investigation PI~\S90 
      --The ideas of explanation and discovery are misleading and inappropiate when
      applied to questions like: what is meaning?
      --We feel as if we had to repair a spider web with our fingers PI~\S106
      --PI~\S129
      --By putting details together in the right way or by using a new analogy or
      comparison to prompt us to see our practice of using language in a new light, we
      find that we achieve the understanding that we thought would only come with the
      construction of an explanatory account. RFGB, p.30
      --Philosopher's questions must be treated like an illness is treated. PI~\S133
      and \S255.
      --The aim of grammatical investigations is perspicious representation PI~\S122
      --Meaning is use.
      --The question of a philosopher is: how do I go about this?
      \end{revision}


      \begin{revision}
      What marks the transition from early to later Wittgenstein can be summed up as
      the total rejection of dogmatism, i.e., as the working out of all the
      consequences of this rejection. The move from the realm of logic to that of
      ordinary language as the center of the philosopher's attention; from an emphasis
      on definition and analysis to ‘family resemblance’ and ‘language-games’; and
      from systematic philosophical writing to an aphoristic style—all have to do with
      this transition towards anti-dogmatism in its extreme. It is in the
      Philosophical Investigations that the working out of the transitions comes to
      culmination. Other writings of the same period, though, manifest the same
      anti-dogmatic stance, as it is applied, e.g., to the philosophy of mathematics
      or to philosophical psychology.
      \end{revision}




      2. La metodología terapéutica y franca de Wittgenstein fue liberadora
      \begin{revision}



      El método terapeútico de Wittgenstein tuvo éxito en liberarla de confusiones
      filosóficas donde otras metodologíás mas teoréticas habían fallado. En sus
      estudios en St. Hugh's escuchaba a Price/ldots
      \end{revision}


      \begin{revision}
      Este modo de criticar una proposición desvelando que no expresa un pensamiento
      verdadero ilustra los principios propuestos en el \emph{Tractatus} y recuerda
      una de sus tesis más conocidas: 
      En el prefacio de las Investigaciones Filosóficas, con fecha de enero de 1945
      Wittgenstein dice que los pensamientos que publica en el libro son el
      precipitado de invetigaciones filosóficas que le han ocupado durante los pasados
      16 años. En enero 1929 Wittgenstein estaba regresando a Cambridge.
      \end{revision}


      \begin{revision}
      En ocasiones como esta la
      discusión con Wittgenstein llevaba a Anscombe a afirmaciones para las cuales no
      podía ofrecer mejor significado que los sugeridos por concepciones ingenuas. Una
      concepción así no es otra cosa que ausencia de pensamiento, pero su falta de
      significado no es evidente, sino que requiere de la fuerza de un `Copérnico'
      para ponerla en cuestión efectivamente.\autocite[cf. 151]{IWT} 
      \end{revision}

Anscombe conoció a Wittgenstein en los años culminantes de su pensamiento
     filosófico. 
     Al comienzo de sus lecciones en 1944 Wittgenstein escribía a su amigo Rush Rhees:
     \citalitinterlin{
         \ldots mis clases no han ido tan mal. Thouless esta asistiendo, y una mujer, 
         'Mrs so and so'
         que se llama a sí misma 
         'Miss Anscombe',
         que ciertamente es inteligente, aunque no del calibre de Kreisel.
         \autocite[p.~371]{cambridgeletters}
     }
     Un año mas tarde escribía a Norman Malcolm:
     \citalitinterlin{
         \ldots mi clase ahora es bastante grande, 19 personas. \ldots Smythies esta
         viniendo, y una mujer que es muy buena, es decir, más que solamente
         inteligente\ldots 
         \autocite[p.~388]{cambridgeletters}
     }
     Aquellos años no sólo creció en Wittgenstein la apreciación de la capacidad de
     Anscombe, sino que se afianzó entre ellos una estrecha amistad. 

     La influencia de Wittgenstein fue decisiva para el desarrollo filosófico de
     Elizabeth. Las lecciones con Wittgenstein eran directas y con franqueza. Esta
     metodología carente de cualquier parafernalia era inquietante para algunos,
     inspiradora para otros, pero fue tremendamente liberadora para
     ella.\autocite[loc 9853 Chapter 4, Section 24, \S5]{monk} Esta libertad
     quedaba demostrada en que Anscombe no se contentaba con repetir lo que decía
     Wittgenstein, sino que pensaba por sí misma; en esto precisamente era más fiel
     al espíritu de la filosofía que había aprendido de él. Sobre esta relación,
     Phillipa Foot, amiga de ambos, cuenta que durante mucho tiempo sostuvo
     objeciones a las afirmaciones de Wittgenstein, eventualmente, un comentario de
     Norman Malcom la hizo pensar que podía haber valor en lo que Wittgenstein decía.
     Cuestionó entonces a Anscombe: 
     ``¿Por qué no me dijiste?'', ella le contestó: ``Porque es importante que uno
     tenga sus resistencias''. Anscombe evidentemente pensaba ---continúa Foot: 
     \citalitlar{
         que un largo periodo de vigorosa objeción era la mejor manera de entender a
         Wittgenstein. Aun cuando era su amiga cercana y albacea literaria, y una de
         los primeros en reconocer su grandeza, nada podía ser más lejano de su
         carácter y modo de pensamiento que el discipulado.\autocite[p.~4]{teichmann}
     }

     Peter geach que dice que les ayudó que estudiaron otros filósofos antes de
     Wittgenstein.

\pnote{introducir algunos contrastes y relaciones entre
       Anscombe y Wittgenstein para explicar la incursión en la vida/pensamiento
       de W.}
