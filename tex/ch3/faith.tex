\subsection{Que se puede entender de la fe sin tenerla}
En Oscott College, el seminario de la Archidiócesis de Birmingham, se comenzaron
a celebrar las conferencias llamadas Wiseman Lectures en 1971. Para estas
lecciones ofrecidas anualmente en memoria de Nicholas Wiseman se invitaba un
ponente que tratara algún tema relacionado con la filosofía de la religión o
alguna materia en torno al ecumenismo.\footcite[cf.~][p.~7]{wisemanlects}

El 27 de octubre de 1975, para la quinta edición de las conferencias, Anscombe
presentó una lección titulada simplemente ``Faith''. Allí planteaba la
siguiente cuestión: \citalitlar{Quiero decir qué puede ser entendido sobre la fe
  por alguien que no la tenga; alguien, incluso, que no necesariamente crea que
  Dios existe, pero que sea capaz de pensar cuidadosa y honestamente sobre ella.
  Bertrand Russell llamó a la fe ``certeza sin prueba''. Esto parece correcto.
  Ambrose Bierce tiene una definición en su \emph{Devil's Dictionary}: ``La
  actitud de la mente de uno que cree sin evidencia a uno que habla sin
  conocimiento cosas sin parangón''. ¿Qué deberíamos pensar de
  esto?\footcite[p.~115]{faith}}

\subsection{La razonabilidad de la fe}
Hubo una época en la que se vivió gran entusiasmo por la racionalidad de la
fe.

El carácter racional de la fe estaba sujeto a los llamados preambulos y el
paso de estos a la fe. Anscombe entiende que éstos son construcciones ideales.
Al menos parte de ellos, sería más apropiado llamarles
presuposiciones.

\subsection{``Solíamos creer que la fe católica era racional''}
Anscombe comienza su investigación

cómo se ha dicho de la fe que es racional?
.



Habían pasado casi diez años de la clausura del Concilio Vaticano~II; Anscombe
comenzó su ponencia recordando cómo a finales de los años sesenta muchas
homilias comenzaban: ``Solíamos creer que\ldots''. ``Soliamos creer --escuchó
una vez-- que no había peor pecado que faltar a misa el domingo''. Escuchar la
frase le traía un desaliento alarmado, ya que la implicita oposición que se
pretendía establecer con la expresión, por lo general, era desaecertada.

Ahora, hay un ``soliamos creer'' que se podía haber usado con algo de acierto.
Hubo una época en el que se profesó gran entusiasmo por la racionalidad. Quizás
inspirado por las enseñanzas del Concilio Vaticano I contra el fideismo, pero
ciertamente promovido por los estudios neo-tomistas. Se decía entre los
entendidos que la fe Católica era racional, el problema parecía ser más bien
cómo era un regalo de la gracia. ¿Por qué sería necesaria la gracia para seguir
un proceso de razonamiento?

Es como si se tuviera la seguirdad de que hay una línea de demostración.

La fe es ciertamente distinta que el conocimiento--pero eso podía ser explicado
por el carácter extrinseco de las pruebas de las doctrinas de de fide

el conocimiento que podía ser contrastado con la fe sería el conocimiento por
pruebas intrínsecas a la matería en cuestión, no por pruebas haber dicho de
alguien que estas cosas son verdad

para asuntos que eran estrictamente de la fe las pruebas intrínsecas no eran
posibles, y eso era por lo que la fe contrastaba con el conocimiento
