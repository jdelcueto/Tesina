\subsection{Parmenides, Mystery and Contradiction}

En 1981 Anscombe publicó una colección de sus escritos en tres volúmenes llamados \emph{The Collected Philosophical Papers of G.\,E.\,M.\,Anscombe}. El primero de estos, titulado \emph{From Parmenides to Wittgenstein}, recoge un tema presente en el \emph{Tractatus} de Wittgenstein y que Anscombe trató con gran interés: la relación entre lo concebible y lo posible. En el contexto del pensamiento de Wittgenstein la cuestión de lo concebible se encuentra dentro de la discusión sobre lo que puede ser dicho claramente. Ahí se encuentran también característicos temas Wittgensteinianos como la falta de significado, el sinsentido, lo misterioso y lo inefable; nociones que estarán presentes en el análisis de Anscombe.

El volumen reúne a autores como Parménides, Platón, Hume y Wittgenstein en la discusión sobre esta cuestión\footnote{Cf.~ \cite[193]{teichmann2008ans}: Philosophers have grappled since ancient times with the problem of how thinkability and possibility are related, and it is characteristic of Anscombe to have drawn such diverse figures as Parmenides, Plato, Hume, and Wittgenstein into a single discussion.} y, como es característico de Anscombe, en cada artículo se le encuentra identificando rutas interesantes tomadas por los distintos autores y profundizando todavía más por caminos de reflexión que ella juzga poco explorados o no valorados del todo.

El primer artículo: \emph{Parmenides, Mystery and Contradiction}, es el texto de una ponencia ofrecida por Anscombe en la reunión del \emph{Aristotelian Society} en \emph{21, Bedford Square} en Londres el 24 de febrero de 1969. En esta discusión Elizabeth estudia la manera en que Parménides construye su argumento acerca de lo posible y lo concebible y qué oportunidades ofrece para un análisis de esta relación. Anscombe percibe en su época la tendencia propia del modernismo de deducir lo posible desde lo concebible, sin embargo le parece más atractivo el acercamiento de Parménides y los antiguos: \blockquote[{\cite[xi]{anscombe1981parmenides}}: It was left to the moderns to deduce what could be from what could hold of thought, as we see Hume to have done. This trend is still strong. But the ancients had the better approach, arguing only that a thought was impossible because the thing was impossible, or, as the Tractatus puts it, ``Was man nicht denken kann, das kann man nicht denken'': an \emph{impossible} thought is an impossible \emph{thought}.]{Se les dejó a los modernos el deducir lo que puede ser posible desde lo que puede ser sostenido en el pensamiento, como vemos hacer a Hume. Esta tendencia sigue siendo fuerte. Pero los antiguos tuvieron el mejor acercamiento, argumentando solo que un pensamiento sería imposible porque la cosa misma es imposible, o, como lo dice el Tractatus, ``Was man nicht denken kann, das kann man nicht denken'': un pensamiento \emph{imposible} es un \emph{pensamiento} imposible.}

Los razonamientos Parmenidianos ofrecen muchas oportunidades de reflexión a Elizabeth. Ella recoge de entre ellos una argumentación sobre lo que puede existir y lo que puede ser pensado que le servirá como el foco de su análisis:
\blockquote[{\cite[3]{anscombe1981parmenides:pmc}}: Parmenides' arguments runs:\\
It is the same thing that can be thought and can be\\
What is not can't be\\
$\therefore$ What is not can't be thought\\
Ver también en {\cite[22--25]{parmenides2007poema}}: Algunos fragmentos relacionados con el argumento presentado por Anscombe pueden ser: \ldots\textgreek{τὸ γὰρ αὐτὸ νοεῖν ἐστίν τε καὶ εἶναι.} (III); \textgreek{Χρὴ τὸ λέγειν τε νοεῖν τ' ἐὸν ἔμμεναι· ἔστι γὰρ εἶναι, μηδὲν δ' οὐκ ἔστιν} (VI); \textelp{} \textgreek{οὐ γὰρ φατὸν οὐδὲ νοητόν ἔστιν ὅπως οὐκ ἔστι.} (VIII)]{El argumento de Parmenides va así:\\
  Es la misma cosa lo que puede ser pensado y lo que puede ser\\
  Lo que no es no puede ser\\
  $\therefore$ Lo que no es no puede ser pensado} En otro lugar Anscombe plantea el mismo argumento de esta otra manera: \blockquote[{\cite[vii]{anscombe1981parmenides}}: Parmenides himself argues: What can be thought can be, What is nothing cannot be, Therefore whatever can be actually is. Therefore whatever can be thought actually is.]{Lo que puede ser pensado puede ser,\\
  Lo que es nada no puede ser,\\
  Por tanto lo que sea que pueda ser es de hecho.\\
  Por tanto lo que sea que pueda ser pensado es de hecho.} Una lectura poco atenta podría dejar la impresión de que el argumento consiste en: \blockquote[{\cite[vii]{anscombe1981parmenides}}: Only what can be thought can be, What is not cannot be thought, Therefore what is not cannot be.]{Solo lo que puede ser pensado puede ser,\\
  Lo que no es no puede ser pensado,\\
  Por tanto lo que no es no puede ser.} Sin embargo, Parmenides no argumentó así.\footnote{\cite[Cf.~][6]{anscombe1981parmenides:pmc}: \textelp{} one might, if reading inattentively, think that Parmenides did argue like that.} La segunda premisa del argumento, las proposiciones \enquote{Lo que no es no puede ser} o \enquote{Lo que es nada no puede ser}, están basadas en que \enquote{Lo que no es, es nada}\footnote{\cite[Cf.~][vii]{anscombe1981parmenides}: these arguments \textelp{} use as a premise: What is not is nothing}. El argumento, por tanto, \blockquote[{\cite[vii]{anscombe1981parmenides}}: \textins{doesn't} derive the nothingness of what-is-not from its unthinkability, but rather unthinkability from its nothingness or from its impossibility.]{no deriva la inexistencia de lo-que-no-es de su ser inconcebible, sino más bien su ser inconcebible desde su inexistencia o su imposibilidad.} Y así Anscombe insiste: \blockquote[{\cite[viii]{anscombe1981parmenides}}: If I am right, the ancients never argued from constraints on what could be a thought to restrictions on what could be, but only the other way around.]{Si estoy en lo correcto, los antiguos nunca argumentaron desde las limitaciones de lo que podría constituir un pensamiento a las restricciones sobre lo que puede ser, sino en la manera inversa.}

Esta segunda premisa todavía ofrece más vía de reflexión. La modalidad según la cual sea interpretada le otorga distintas acepciones. Entendida \emph{in sensu composito}, es decir, como una proposición general, la verdad de la premisa \enquote{Lo que no es no puede ser} puede ser entendida como la imposibilidad de la afirmación \enquote{Lo que no es, es}.\footnote{\cite[Cf.~][vii]{anscombe1981parmenides}: \textelp{} the impossibility of the proposition ``What is not is'' ---i.e. the truth of ``What is not cannot be'', taken in \emph{sensu composito}} Si, por otra parte, se entiende \emph{in sensu diviso}, o como una proposición particular, puede ser interpretada como \blockquote[{\cite[3]{anscombe1981parmenides:pmc}}: Concerning that which is not, it holds that \emph{that} cannot be]{Concerniendo aquello que no es, se sostiene que \emph{eso} no puede ser}.

Ahora bien, Anscombe establece que el argumento completo no es valido si esta segunda premisa es entendida \emph{in sensu composito}. Sin embargo, si se interpreta \emph{in sensu diviso}, la premisa misma no es creíble. Esto lo explica diciendo: \blockquote[{\cite[vii]{anscombe1981parmenides}}: The impossibility of what is not isn't just the impossibility of the proposition ``What is not, is'' ---i.e. the truth of ``What is not cannot be'', taken \emph{in sensu composito}. \emph{That} could be swept aside as irrelevant. What is not can't be indeed, but it may come to be, and in this sense what is not is possible. When it \emph{has} come to be, of course it no longer is what is not, so in calling it possible we aren't claiming that ``What is not is'' is possible. So it can't be shown to be impossible that it should come to be just by pointing to the impossibility that it is. ---But this can't be the whole story. That what is not is nothing implies that there isn't anything to come to be. So ``What is not can be'' taken in \emph{sensu diviso}, namely as: ``Concerning what is not, \emph{that} can be'' is about nothing at all. If it were about something, then it would be about something that is not, and so there'd be an example of ``What is not is'' that was true.]{La imposibilidad de lo que no es, no es solo la imposibilidad de la proposición ``lo que no es, es'' ---es decir, la verdad de ``Lo que no es no puede ser'', tomado \emph{in sensu composito}. \emph{Eso} puede ser descartado como irrelevante. Lo que no es, ciertamente no puede estar siendo, pero puede llegar a ser, y en este sentido lo que no es es posible. Cuando \emph{haya} llegado a ser, ciertamente ya no es lo que no es, así que en llamarlo posible no estamos declarando que ``Lo que no es, es'' es posible. Entonces no puede mostrarse como imposible que pueda llegar a ser solo por señalar la imposibilidad de que este siendo. ---Pero esta no puede ser toda la historia. Que lo que no es, es nada implica que no hay nada ahí para llegar a ser. Así ``Lo que no es puede ser'' tomado en \emph{sensu diviso}, digase como: ``Con respecto a lo que no es, eso puede ser'' es acerca de nada en absoluto. Si fuera acerca de algo, entonces sería sobre algo que no es, y así habría un ejemplo de ``Lo que no es, es'' que sería verdadero.} Si la premisa se toma en sentido general su significado es irrelevante para el argumento. Si se toma en sentido particular es relevante para el argumento, pero es una proposición que no es creíble; lo mismo ocurre con la conclusión: \blockquote[{\cite[3]{anscombe1981parmenides:pmc}}: Concerning that which is not, it holds that \emph{that} cannot be thought.]{Con respecto a aquello que no es, se sostiene que \emph{eso} no puede ser pensado} la cual también es increíble.

En el análisis de esta premisa Anscombe hace una distinción que caracteriza su punto de vista sobre lo concebible y lo existente. Destaca que Parmenides actúa en su argumentación bajo \blockquote[{\cite[x]{anscombe1981parmenides}}: \textelp{} the assumption that a significant term is a name of an object which is either expressed or characterized by the term.]{\textelp{} el presupuesto de que un término significativo es un nombre de un objeto que queda expresado o caracterizado por el término.}  Platón da por hecho este mismo supuesto y, a juicio de Anscombe: \blockquote[{\cite[xi]{anscombe1981parmenides}}: <<The assumption common to Plato and Parmenides is an ancestor of much philosophical theorizing and perplexity. In Aristotle \textelp{} the theory of substance and the inherence in substances of individualized forms of properties and relations of various kinds \textelp{} In Descartes \textelp{} the assertion that the descriptive terms which we use to construct even false pictures of the world must themselves stand for realities \textelp{} In Hume \textelp{} the assumption that `an object' corresponds to a term, even such a term as ``a cause'' as it occurs in ``A beginning of existence must have a cause.'' \textelp{} Brentano thinks that the mere predicative connection of terms is an `acknowledgement' \textelp{} Wittgenstein himself in the \emph{Tractatus} has language pinned to reality by its (postulated) simple names, which mean simple objects.>>]{El presupuesto común a Platón y Parmenides es un ancestro de mucha especulación y perplejidad. En Aristóteles \textelp{} la teoría de la sustancia y la inherencia en sustancias de formas individualizadas de propiedades y relaciones de varias clases \textelp{} En Descartes \textelp{} la aseveración de que los términos descriptivos que usamos para construir incluso falsas imágenes del mundo tienen que ser ellos mismos representaciones de realidades \textelp{} En Hume \textelp{} el presupuesto de que `un objeto' corresponde con un termino, incluso con un término como ``una causa'' como aparece en ``El comienzo de una existencia tiene que tener una causa.'' \textelp{} Brentano piensa que la mera conexión predicativa de términos es un `reconocimiento' \textelp{} Wittgenstein mismo en el \emph{Tractatus} tiene al lenguaje atado a la realidad por medio de sus (postulados) nombres simples, que significan objetos simples.}



\blockquote[{\cite[5]{anscombe1981parmenides:pmc}}: it is false that one mentions either properties or objects when one uses the quantifiers binding property variables and object variables; though it has to be granted that some authors, sucha as Quine, are accostumed to speak of the reference of variables. But if this is given up, as it ought to be, Parmenides is deprived of his claim that we are commited to self-contradiction in existence just because we are willing to use a self-contradictory predicate --- e.g. in the sentence saying that nothing has a self-contradictory predicate true of it --- so that out property-variable is admitted to range over self-contradictory properties.]{es falso que }

una variable no tiene que estar atada a una referencia como dice quine sino que puede ser empleada para evaluar la validez de una proposición teniendola como variable

Parmenides tiene como objetivo evitar la auto contradicción en lo que existe, Anscombe parece insistir en que no es creíble que lo que no es no puede ser pensado entonces lo que dice que cree es que lo que no es puede ser en el pensamiento y para parmenides esto es una auto contradicción (self-contradiction in what exists is just what I set out to avoid, and you pretended that I could do that without accepting the conclusion ``What is not cannot be thought'' But your insistence that what is not can be has landed you in self-contradiction after all...)

Anscombe está hablando del lenguaje aquí como algo que no está atrapado por la realidad (no es representativo), sino como dice en qli es como una herramienta que tiene el logical shape de la esencia que expresa.

Para él ``ser'' es el término que expresa el ser, sin embargo, otros términos que no son nombres de nada son también nombres del ser, \blockquote[{\cite[x]{anscombe1981parmenides}}: What they express is what is true of being, so they characterize it as well as naming it]{Lo que expresan es lo que es verdadero del ser, así que lo caracterizan además de denominarlo.}

dificultad para entender a qué se refiere con being

Si combinamos esto con su idea de que ser es un objeto entonces obtenemos sus resultados más alocados

la segunda premisa entendida en sensu diviso ya sea como: lo que no existe no puede existir como lo que no es el caso no puede ser el caso no es creíble


también hay una dificultad sobre los dos caminos del conocimiento

lo notable es la combinación de es con no puede no ser y no es con no puede ser: el argumento para esto es lo que no es es nada y no es posible que lo que es nada sea; por tanto lo que sea que puede ser debe ser, y lo que puede ser pensado debe ser; puesto que es lo mismo que lo que puede ser.


