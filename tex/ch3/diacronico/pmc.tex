\subsection{Parmenides, Mystery and Contradiction}

En 1981 Anscombe publicó una colección de sus escritos en tres volúmenes
llamados \emph{The Collected Philosophical Papers of G.\,E.\,M.\,Anscombe}. El
primero de estos, titulado \emph{From Parmenides to Wittgenstein}, recoge un tema
presente en el \emph{Tractatus} de Wittgenstein y que Anscombe trató con gran
interés: la relación entre lo concebible y lo posible. En el contexto del
pensamiento de Wittgenstein la cuestión de lo concebible se encuentra dentro de
la discusión sobre lo que puede ser dicho claramente. Ahí se encuentran también
característicos temas Wittgensteinianos como la falta de significado, el
sinsentido, lo misterioso y lo inefable; nociones que estarán presentes en el
análisis de Anscombe.

El volumen reúne a autores como Parménides, Platón, Hume y Wittgenstein en la
discusión sobre esta cuestión\footnote{Cf.~ \cite[193]{teichmann2008ans}:
  Philosophers have grappled since ancient times with the problem of how
  thinkability and possibility are related, and it is characteristic of Anscombe
  to have drawn such diverse figures as Parmenides, Plato, Hume, and
  Wittgenstein into a single discussion.} y, como es característico de
Anscombe, en cada artículo se le encuentra identificando rutas interesantes
tomadas por los distintos autores y profundizando todavía más por caminos de
reflexión que ella juzga poco explorados o no valorados del todo.

El primer artículo: \emph{Parmenides, Mystery and Contradiction}, es el texto de
una ponencia ofrecida por Anscombe en la reunión del \emph{Aristotelian Society}
en \emph{21, Bedford Square} en Londres el 24 de febrero de 1969. En esta
discusión Elizabeth estudia la manera en que Parménides construye su argumento
acerca de lo posible y lo concebible y qué oportunidades ofrece para un análisis
de esta relación. Anscombe percibe en su época la tendencia propia del
modernismo de deducir lo posible desde lo concebible, sin embargo le parece más
atractivo el acercamiento de Parménides y los antiguos:
\blockquote[{\cite[xi]{anscombe1981parmenides}}: It was left to the moderns to
deduce what could be from what could hold of thought, as we see Hume to have
done. This trend is still strong. But the ancients had the better approach,
arguing only that a thought was impossible because the thing was impossible, or,
as the Tractatus puts it, ``Was man nicht denken kann, das kann man nicht
denken'': an \emph{impossible} thought is an impossible \emph{thought}.]{Se les
  dejó a los modernos el deducir lo que puede ser posible desde lo que puede ser
  sostenido en el pensamiento, como vemos hacer a Hume. Esta tendencia sigue
  siendo fuerte. Pero los antiguos tuvieron el mejor acercamiento, argumentando
  solo que un pensamiento sería imposible porque la cosa misma es imposible, o,
  como lo dice el Tractatus, ``Was man nicht denken kann, das kann man nicht
  denken'': un pensamiento \emph{imposible} es un \emph{pensamiento} imposible.}


