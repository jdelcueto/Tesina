\subsection{Parmenides, Mystery and Contradiction (1969)}

En 1981 Anscombe publicó una colección de sus escritos en tres volúmenes llamados \emph{The Collected Philosophical Papers of G.\,E.\,M.\,Anscombe}. El primero de estos, titulado \emph{From Parmenides to Wittgenstein}, recoge un tema que juega un papel importante en el \emph{Tractatus} de Wittgenstein y que Anscombe trató con gran interés: la relación entre lo concebible y lo posible. En el contexto del pensamiento de Wittgenstein la cuestión de lo concebible se encuentra dentro de la discusión sobre lo que puede ser dicho claramente. Ahí se encuentran también característicos temas Wittgensteinianos como la falta de significado, el sinsentido, lo misterioso y lo inefable; nociones que estarán presentes en el análisis de Anscombe.

El volumen reúne a autores como Parménides, Platón, Hume y Wittgenstein en la discusión sobre esta cuestión\footnote{\cite[Cf.][193]{teichmann2008ans}: \enquote{Philosophers have grappled since ancient times with the problem of how thinkability and possibility are related, and it is characteristic of Anscombe to have drawn such diverse figures as Parmenides, Plato, Hume, and Wittgenstein into a single discussion}.} y, como es característico de Anscombe, en cada artículo se le encuentra identificando rutas interesantes tomadas por los distintos autores y profundizando todavía más por caminos de reflexión que ella juzga poco explorados o no valorados del todo.

El primer artículo: \emph{Parmenides, Mystery and Contradiction}, es el texto de una ponencia ofrecida por Anscombe en la reunión del \emph{Aristotelian Society} en Londres el 24 de febrero de 1969. En esta discusión Elizabeth estudia la manera en que Parménides construye su argumento acerca de lo posible y lo concebible y qué oportunidades ofrece para un análisis de esta relación.

Una importante clave de interpretación de este artículo se encuentra en el lugar que ocupa como parte de esta colección. El título del volumen no es casual, el primer artículo es dedicado a Parménides, y el último, \emph{The Question of Linguistic Idealism}, es un examen de nociones importantes en la filosofía de Wittgenstein en donde reaparecen temas que Anscombe plantea ya en esta investigación dedicada a las ideas de Parménides. En este sentido, su análisis de los argumentos de Parménides pone en marcha una discusión que atraviesa todos los artículos del volumen. ¿En qué consiste esta discusión que Anscombe juzga presente ya en Parménides y viva todavía en Wittgenstein? En la introducción de la colección la describe diciendo: \blockquote[{\Cite[xi]{anscombe1981parmenides}}: \enquote{At the present day we are often perplexed with enquiries about what makes true, or what something's being thus or so \emph{consists in}; and the answer to this is thought to be an explanation of meaning. If there is no external answer, we are apparently committed to a kind of idealism}.]{En la época actual con frecuencia nos quedamos perplejos con preguntas sobre qué hace a algo verdadero, o \emph{en qué consiste} el que algo sea de un modo u otro; y la respuesta a esto se piensa que es una explicación del significado. Si no hay una respuesta externa, aparentemente estamos comprometidos con un tipo de idealismo}.\label{subsec:intextq}

Esta preocupación de la época, aludida por Anscombe, tiene una presencia importante en \emph{Investigaciones Filosóficas}. Las \S\S428-465, en donde Wittgenstein se detiene a reflexionar sobre la intencionalidad, contienen implícitamente una crítica a ese modo de concebir el pensamiento, el lenguaje, la realidad y sus relaciones que sirvió para orientar las ideas del \emph{Tractatus}; específicamente son atacados: \blockquote[{\Cite[3]{hacker2000mind}}: \enquote{the underlying assumptions that characterize the whole tradition of philosophical reflection of which it was the culmination}.]{los presupuestos subyacentes que han caracterizado toda la tradición de reflexión filosófica de la cual \textelp{el \emph{Tractatus}} fue la culminación}. Entre estos presupuestos se cuestiona enfáticamente \blockquote[{\Cite[3]{hacker2000mind}}: \enquote{the venerable idea that the meaning of signs, their capacity to represent what they represent, is parasitic upon thought, upon mental processes of thinking and meaning}.]{la venerable idea de que el significar de los signos, su capacidad para representar lo que representan, depende del pensamiento, de procesos mentales de pensar y significar}. Esta idea, juzga Wittgenstein, es un producto de la concepción de los pensamientos como representación. Sobre los pensamientos así concebidos ha girado cierta discusión en la que se ha debatido acerca de qué es lo que constituye los pensamientos. Así: \blockquote[{\Cite[3]{hacker2000mind}}: \enquote{the empiricists characteristically held them to be mental images or ideas; others, like the author of the \emph{Tractatus}, were more reticent, content to leave the matter to future psychological discovery, insisting only that thought-constituents must stand to reality in the same sort of relation as words}.]{los empiristas característicamente sostenían que estos eran imágenes mentales o ideas; otros, como el autor del \emph{Tractatus}, fueron más reticentes, contentándose con dejar el asunto al futuro descubrimiento psicológico, insistiendo solamente en que los constituyentes de pensamiento tienen que tener, respecto de la realidad, el mismo tipo de relación que las palabras}.

Dentro de este debate, la intencionalidad de los pensamientos, ---y aquí `pensamientos' pueden ser creencias, expectativas, esperanzas, temores, dudas, deseos, etc.--- era explicada también de modos distintos por los empiristas y por el autor del \emph{Tractatus}. Los primeros sosteniendo que la relación entre un pensamiento y la realidad correspondiente con este es externa, y el segundo que la relación es interna. La posibilidad de esta relación interna aparece explicada en el \emph{Tractatus}: \blockquote[{\Cite[3]{hacker2000mind}}: \enquote{in terms of a pre-established metaphysical harmony between thought and reality. This harmony was conceived to consist in an essential isomorphism between representation and what is represented, wether truly or falsely}.]{en términos de una armonía metafísica preestablecida entre el pensamiento y la realidad. Esta armonía fue concebida como consistiendo en un isomorfismo esencial entre la representación y lo que es representado, ya sea verdadera como falsamente}. La concepción empirista \blockquote[{\Cite[3]{hacker2000mind}}: \enquote{attempted to explain the intentionality of thought in causal terms \textelp{} construing the relation between thought and reality (between belief and what makes it true, or between desire and what fulfills it) as external}.]{intentó explicar la intencionalidad del pensamiento en términos causales \textelp{} interpretando la relación entre pensamiento y realidad (entre el creer y lo que lo hace verdadero, o entre el deseo y lo que lo realiza) como externa}. En \emph{Investigaciones Filosóficas} se critican estas dos posturas aunque se mantiene la idea de que la relación entre pensamiento y realidad es interna.

Estas discusiones son importantes porque están en el trasfondo de la perspectiva de Elizabeth, cuya postura es análoga a la que se encuentra en \emph{Investigaciones Filosóficas}. Todavía se descubre otro elemento de esta reflexión en el análisis que Anscombe hace de los argumentos de Parménides. En las \S\S89-133 Wittgenstein examina la naturaleza de la filosofía y critica la impresión de que el pensamiento sea algo misterioso o extraño. En las \S\S93-94 se fija en que la proposición puede parecer algo extraordinario que aparenta esconder un intermediario puro (la forma lógica) que está entre los signos y los hechos. \S95 sugiere que también el pensar parece algo de naturaleza singular puesto que es posibile que en el pensamiento se contemple algo que no es. El \emph{Tractatus} intentó dar una explicación de esto con una elaborada doctrina, sin embargo,
%: \blockquote[{\Cite[4]{hacker2000mind}}: \enquote{for what we mean when we say that such-and-such is the case does not stop short of the fact that makes what we say true. We mean that very fact, and not something that stands in some relation (e.g. of correspondence) to it. We, as it were, reach right up to it. On the other hand, we can think what is \emph{not} the case. But if it is not the case, then it seems that there is nothing to reach right up to. Yet what we think when we think what is the case and what we think when we think what is not the case are not intrinsically different. How is this possible? The \emph{Tractatus} resolved the difficulty by arguing that what we think is the sense of a sentence, which is a \emph{possible} state of affairs, actual if what we think is the case and unactualized if what we think is not the case. For this a complex metaphysics and ontology and an elaborate doctrine of the depth grammar of all possible languages were introduced.}]{pues lo que significamos cuando decimos que alguna cosa es de hecho no se queda detenido ante el hecho que hace que lo que decimos sea verdadero. Significamos el mismo hecho y no algo que está situado en relación alguna (de correspondencia por ejemplo) con este. Nosotros, podría decirse, lo tenemos al alcance. Por otra parte, podemos pensar lo que \emph{no} es de hecho. Pero si no es de hecho, entonces parece que no hay nada para alcanzar. Sin embargo lo que pensamos cuando pensamos lo que es de hecho y lo que pensamos cuando pensamos lo que no es de hecho no es intrínsecamente distinto. ¿Cómo es esto posible? El \emph{Tractatus} resolvió la dificultad argumentando que lo que pensamos es el sentido de una oración, que es un \emph{posible} estado de las cosas, actual si lo que pensamos es de hecho y no actualizado si lo que pensamos no es de hecho. Para esto se introdujo una compleja metafísica y ontología y una elaborada doctrina sobre la gramática profunda de todos los lenguajes.}
en \emph{Investigaciones Filosóficas} la noción misma del lenguaje o del pensamiento como algo singular, o la idea de que entender el lenguaje es algo extraordinario cuya comprensión tiene que pasar a través del medio que es el pensamiento, es una superstición producida por ilusiones de la gramática.

Para Anscombe lo que parece misterioso del lenguaje no es una armonía formal a priori entre el pensamiento y la realidad, sino precisamente la intencionalidad del pensamiento. Sin embargo, la intención de referir una expresión a algo en el mundo no establece una conexión esencial entre palabra y realidad, sino que experesa una regla gramatical\footnote{\cite[4]{hacker2000mind}: \enquote{A thought seems queer and mysterious when we reflect on it in philosophy. What is mysterious is precisely its intentionality. \S429 introduces the \emph{Tractatus} idea of the `harmony between thought and reality', which constituted an explanation of the `mysteries' of thinking and of the nature of representation by means of language. This misconception is laid to rest (with excessive brevity) by an intra-grammatical move that implicitly repudiates the earlier conception of a connection between language and reality. An ostensive definition does not forge a connection between word and world of a kind which the \emph{Tractatus} had thought essential, but is a rule of grammar. So language is, in this sense, autonomous and self-contained}.}.
%: \blockquote[{\Cite[4]{hacker2000mind}}: \enquote{A thought seems queer and mysterious when we reflect on it in philosophy. What is mysterious is precisely its intentionality. \S429 introduces the \emph{Tractatus} idea of the `harmony between thought and reality', which constituted an explanation of the `mysteries' of thinking and of the nature of representation by means of language. This misconception is laid to rest (with excessive brevity) by an intra-grammatical move that implicitly repudiates the earlier conception of a connection between language and reality. An ostensive definition does not forge a connection between word and world of a kind which the \emph{Tractatus} had thought essential, but is a rule of grammar. So language is, in this sense, autonomous and self-contained.}]{Un pensamiento parece extraño y misterioso cuando reflexionamos sobre él en la filosofía. Lo que es misterioso es precisamente su intencionalidad. \S429 introduce la idea del \emph{Tractatus} de la `armonía entre pensamiento y realidad', que constituye una explicación de los `misterios' del pensar y de la naturaleza de la representación por medio del lenguaje. A esta idea equivocada se le pone fin (con excesiva brevedad) por medio de un movimiento intra-gramático que implícitamente repudia la anterior concepción de una conexión entre el lenguaje y la realidad. Una definición ostensiva no forja una conexión entre palabra y mundo del tipo del que el \emph{Tractatus} había pensado como esencial, sino que es una regla de la gramática. Así que el lenguaje es, en este sentido, autónomo e independiente.}
Cuando se abandona esta relación a priori entre el pensamiento y la realidad también la lógica queda en situación distinta. Mientras que en el \emph{Tractatus} el rigor de la lógica se entendía como la imagen-reflejo de este orden a priori del mundo, \S108 de \emph{Investigaciones Filosóficas} corrige esta visión proponiendo que más bien es un modo de representación\footnote{\cite[242]{bakerhacker2009understanding}: \enquote{We can re-present sentences of natural language in the forms of sentences of the predicate calculus. We can recast our arguments in these forms and display their validity (or invalidity). We can perspicuously disambiguate certain kinds of equivocations in ordinary language by means of quantifier shifts in the calculus}.}.
%:\blockquote[{\Cite[242]{bakerhacker2009understanding}}: \enquote{We can re-present sentences of natural language in the forms of sentences of the predicate calculus. We can recast our arguments in these forms and display their validity (or invalidity). We can perspicuously disambiguate certain kinds of equivocations in ordinary language by means of quantifier shifts in the calculus.}]{Podemos re-presentar oraciones del lenguaje natural en las formas de oraciones del cálculo predicado. Podemos reestructurar nuestros argumentos en estas formas y mostrar su validez (o invalidez). Podemos inteligiblemente eliminar la ambigüedad de ciertos tipos de equivocaciones en el lenguaje ordinario por medio de desplazamientos de los cuantificadores en el cálculo.}

Por último podemos preguntarnos si se ha abandonado esta descripción del modo en que las palabras significan, ¿qué es lo que les otorga significado según la visión de \emph{Investigaciones Filosóficas}? Sobre esto se puede ver \S430-432 y \S454: \blockquote[{\Cite[4]{hacker2000mind}}: \enquote{One must resist the temptation of thinking that what gives life to a sign is a psychic act, e.g. thinking, understanding or meaning. The life of a sign lies in its rule-governed use in a practice, in the application that a living being, who has mastered the techniques of its use, makes of it}.]{Debemos resistir la tentación de pensar que lo que da vida a un signo es un acto psíquico, como pensar, entender o significar, por ejemplo. La vida de un signo se encuentra en el uso gobernado por reglas que se hace de este en la práctica, en la aplicación que un ser vivo, que domina las técnicas de su uso, hace de él}.

Teniendo en cuenta todo este trasfondo podemos distinguir los movimientos que Anscombe realiza en su análisis. El argumento de Parménides que será examinado lo presenta como sigue: \blockquote[{\Cite[3]{anscombe1981parmenides:pmc}}: \enquote{Parmenides' arguments runs: It is the same thing that can be thought and can be; What is not can't be; $\therefore$ What is not can't be thought} Ver también en {\cite[22-25]{parmenides2007poema}}: Algunos fragmentos relacionados con el argumento presentado por Anscombe pueden ser: \enquote{\ldots\textgreek{τὸ γὰρ αὐτὸ νοεῖν ἐστίν τε καὶ εἶναι.} (III); \textgreek{Χρὴ τὸ λέγειν τε νοεῖν τ' ἐὸν ἔμμεναι· ἔστι γὰρ εἶναι, μηδὲν δ' οὐκ ἔστιν} (VI); \textelp{} \textgreek{οὐ γὰρ φατὸν οὐδὲ νοητόν ἔστιν ὅπως οὐκ ἔστι.} (VIII)}.]{El argumento de Parménides va así:\\
Es la misma cosa lo que puede ser pensado y lo que puede ser\\
Lo que no es no puede ser\\
$\therefore$ Lo que no es no puede ser pensado}

Ahora bien, Elizabeth aclara que Parménides tiene algunos presupuestos que es preciso tener en cuenta para interpretar sus premisas. En primer lugar, un presupuesto que tiene en común con Platón, es \blockquote[{\Cite[x]{anscombe1981parmenides}}: \enquote{that a significant term is a name of an object which is either expressed or characterized by the term}.]{que un término significativo es el nombre de un objeto que está expresado o caracterizado por el término}. Este presupuesto, propone Anscombe, \blockquote[{\Cite[xi]{anscombe1981parmenides}}: \enquote{is an ancestor of much philosophical theorizing and perplexity}; En el texto continúa dando ejemplos de esta tradición que coinciden con las discusiones que están recogidas en este volumen de la colección: \enquote{In Aristotle \textelp{} the theory of substance and the inherence in substances of individualized forms of properties and relations of various kinds \textelp{} In Descartes \textelp{} the assertion that the descriptive terms which we use to construct even false pictures of the world must themselves stand for realities \textelp{} In Hume \textelp{} the assumption that `an object' corresponds to a term, even such a term as ``a cause'' as it occurs in ``A beginning of existence must have a cause.'' \textelp{} Brentano thinks that the mere predicative connection of terms is an `acknowledgement' \textelp{} Wittgenstein himself in the \emph{Tractatus} has language pinned to reality by its (postulated) simple names, which mean simple objects}.]{es un ancestro de mucha teorización y perplejidad filosófica}.
%y continúa: \blockquote[{\Cite[xi]{anscombe1981parmenides}}: \enquote{In Aristotle \textelp{} the theory of substance and the inherence in substances of individualized forms of properties and relations of various kinds \textelp{} In Descartes \textelp{} the assertion that the descriptive terms which we use to construct even false pictures of the world must themselves stand for realities \textelp{} In Hume \textelp{} the assumption that `an object' corresponds to a term, even such a term as ``a cause'' as it occurs in ``A beginning of existence must have a cause.'' \textelp{} Brentano thinks that the mere predicative connection of terms is an `acknowledgement' \textelp{} Wittgenstein himself in the \emph{Tractatus} has language pinned to reality by its (postulated) simple names, which mean simple objects.}]{En Aristóteles \textelp{} la teoría de la sustancia y la inherencia en sustancias de formas individualizadas de propiedades y relaciones de varias clases \textelp{} En Descartes \textelp{} la aseveración de que los términos descriptivos que usamos para construir incluso falsas imágenes del mundo tienen que ser ellos mismos representaciones de realidades \textelp{} En Hume \textelp{} el presupuesto de que `un objeto' corresponde con un término, incluso con un término como ``una causa'' así como aparece en ``El comienzo de una existencia tiene que tener una causa.'' \textelp{} Brentano piensa que la mera conexión predicativa de términos es un `reconocimiento' \textelp{} Wittgenstein mismo en el \emph{Tractatus} tiene al lenguaje atado a la realidad por medio de sus (postulados) nombres simples, que significan objetos simples.} Estos temas son los que Anscombe estudia en los ensayos que componen este volumen de la colección.
Esta tradición de \enquote*{teorización y perplejidad} que Anscombe traza culminando en el \emph{Tractatus} hace referencia al modelo de representación que se encuentra criticado en \emph{Investigaciones Filosóficas}. Anscombe nota en el argumento de Parménides un germen de la tradición subyacente a la conexión a priori entre el lenguaje y la realidad que aparece en el \emph{Tractatus}.

%Otros dos presupuestos se encuentran en las premisas del argumento parmenidiano; uno tiene que ver con lo que Parménides entiende por `ser' y el otro con su descripción sobre las dos `rutas' posibles para el pensamiento sobre algo. Para Parménides los términos son nombres de objetos, y según esto, para él, `ser' es el nombre de un objeto. Sin embargo el uso que hace del término en sus premisas no es tan simple: \blockquote[{\Cite[x]{anscombe1981parmenides}}: \enquote{``being'' might be an abstract noun, equivalent to the infinitive ``to be''. But Parmenides does not treat \emph{to be} as an object, but rather \emph{being}, i.e. something being or some being thing \textelp{} we might get closer to the sense by saying ``what is''}.]{``el ser'' puede ser un nombre abstracto, equivalente al infinitivo ``ser''. Pero Parménides no trata ``ser'' como un objeto, sino más bien ``el ser'', es decir algo que esta siendo, o alguna cosa que es \textelp{} nos podemos aproximar a este sentido diciendo ``lo que es''}. También plantea dificultades lo que Parménides propone como las dos rutas posibles del pensamiento. Estas son \enquote*{es, y no puede no ser} (\textgreek{ἔστιν τε καὶ ὡς οὐκ ἔστι μὴ εἶναι}) y \enquote*{no es y necesariamente no puede ser} (\textgreek{οὐκ ἔστιν τε καὶ ὡς χρεών ἐστι μὴ εἶναι}). Anscombe lo pone en estas palabras: \blockquote[{\Cite[x]{anscombe1981parmenides}}: \enquote{``These are the only ways for enquiry for thought: one is `is and cannot not be',\ldots the other `is not, and needs must not be'.'' That is: Whatever enquiry one is making, one's thoughts can only go two ways, saying `is, and must be', or `is not, and can't be'}.]{``Estos son los únicos caminos para indagar con el pensamiento: uno es `es y no puede no ser',\ldots el otro `no es, y necesariamente no puede ser'.'' Esto es: Cualquier indagación que estemos haciendo, nuestros pensamientos solo pueden ir en una de dos direcciones, decir `es, y debe ser', o `no es, y no puede ser'}. Anscombe destaca que es notable la combinación de `es' con `debe ser'  y `no es' con `no puede ser'.
%, la justificación de Parménides para esta relación puede verse presente en el argumento antes citado si este mismo se entiende como: \blockquote[{\Cite[vii]{anscombe1981parmenides}}: \enquote{Parmenides himself argues: What can be thought can be, What is nothing cannot be, Therefore whatever can be actually is. Therefore whatever can be thought actually is.}]{Lo que puede ser pensado puede ser,\\
%Lo que es nada no puede ser,\\
%Por tanto lo que sea que pueda ser es de hecho.\\
%Por tanto lo que sea que pueda ser pensado es de hecho}.

%Una lectura poco atenta, advierte Anscombe, podría dejar la impresión de que el argumento consiste en: \blockquote[{\Cite[vii]{anscombe1981parmenides}}: \enquote{Only what can be thought can be, What is not cannot be thought, Therefore what is not cannot be}.]{Solo lo que puede ser pensado puede ser,\\
%Lo que no es no puede ser pensado,\\
%Por tanto lo que no es no puede ser}. Sin embargo, Parménides no argumentó así\footnote{\cite[Cf.][6]{anscombe1981parmenides:pmc}: \enquote{\textelp{} one might, if reading inattentively, think that Parmenides did argue like that}.}. La segunda premisa del argumento, las proposiciones \enquote*{Lo que no es no puede ser} o \enquote*{Lo que es nada no puede ser}, están basadas en que \enquote*{Lo que no es, es nada}\footnote{\cite[Cf.][vii]{anscombe1981parmenides}: \enquote{these arguments \textelp{} use as a premise: What is not is nothing}.}. El argumento, por tanto, \blockquote[{\Cite[vii]{anscombe1981parmenides}}: \enquote{\textins{doesn't} derive the nothingness of what-is-not from its unthinkability, but rather unthinkability from its nothingness or from its impossibility}.]{no deriva la inexistencia de lo-que-no-es de su ser inconcebible, sino más bien su ser inconcebible desde su inexistencia o su imposibilidad}. Y así Anscombe insiste: \blockquote[{\Cite[viii]{anscombe1981parmenides}}: \enquote{If I am right, the ancients never argued from constraints on what could be a thought to restrictions on what could be, but only the other way around}.]{Si estoy en lo correcto, los antiguos nunca argumentaron desde las limitaciones de lo que podría constituir un pensamiento a las restricciones sobre lo que puede ser, sino en la manera inversa}. Este punto es del interés de Anscombe. Es decir, la reflexión de Parménides no solo resulta interesante a Anscombe por la tradición filosófica que representa, sino además porque percibe en su época la tendencia propia del modernismo de deducir lo posible desde lo concebible, sin embargo le parece más atractivo el acercamiento de Parménides y los antiguos\footnote{\cite[xi]{anscombe1981parmenides}: \enquote{It was left to the moderns to deduce what could be from what could hold of thought, as we see Hume to have done. This trend is still strong. But the ancients had the better approach, arguing only that a thought was impossible because the thing was impossible, or, as the Tractatus puts it, ``Was man nicht denken kann, das kann man nicht denken'': an \emph{impossible} thought is an impossible \emph{thought}}.}.
%: \blockquote[{\Cite[xi]{anscombe1981parmenides}}: \enquote{It was left to the moderns to deduce what could be from what could hold of thought, as we see Hume to have done. This trend is still strong. But the ancients had the better approach, arguing only that a thought was impossible because the thing was impossible, or, as the Tractatus puts it, ``Was man nicht denken kann, das kann man nicht denken'': an \emph{impossible} thought is an impossible \emph{thought}.}]{Se les dejó a los modernos el deducir lo que puede ser posible desde lo que puede ser sostenido en el pensamiento, como vemos hacer a Hume. Esta tendencia sigue siendo fuerte. Pero los antiguos tuvieron el mejor acercamiento, argumentando solo que un pensamiento sería imposible porque la cosa misma es imposible, o, como lo dice el \emph{Tractatus}, ``Was man nicht denken kann, das kann man nicht denken'': un pensamiento \emph{imposible} es un \emph{pensamiento} imposible}.
%Con esto Elizabeth vuelve a hacer referencia al debate sobre la relación entre la realidad y el pensamiento en donde los planteamientos empiristas de su época estan en continuidad con los planteamientos de la modernidad y en donde también se identifica la presencia de la tradición recogida en el \emph{Tractatus}. De este modo el ensayo dedicado a Parménides sirve a Anscombe para representar distintas perspectivas y argumentaciones que ella identifica presentes en el debate de su época. Estudiando estas ideas desde las propuestas de \emph{Investigaciones Filosóficas} sienta las bases de la discusión que la ocupará a lo largo de los distintos escritos que se encuentran en este volumen de la colección.

%Aclaradas las intenciones de Anscombe, podemos adentrarnos en su investigación, ¿qué tiene ella que decir sobre el argumento de Parménides? En primer lugar examina la segunda premisa: \enquote*{Lo que no es no puede ser}. La modalidad según la cual se interprete la premisa le otorga distintas acepciones. Entendida \emph{in sensu composito}, es decir, como una proposición general, la verdad de la premisa \enquote*{Lo que no es no puede ser} puede ser entendida como la imposibilidad de la afirmación \enquote*{Lo que no es, es}\footnote{\cite[Cf.][vii]{anscombe1981parmenides}: \enquote{\textelp{} the impossibility of the proposition ``What is not is'' ---i.e. the truth of ``What is not cannot be'', taken in \emph{sensu composito}}.}. Si, por otra parte, se entiende \emph{in sensu diviso}, o como una proposición particular, puede ser interpretada como \blockquote[{\Cite[3]{anscombe1981parmenides:pmc}}: \enquote{Concerning that which is not, it holds that \emph{that} cannot be}.]{Concerniendo aquello que no es, se sostiene que \emph{eso} no puede ser}. Es importante notar aquí los dos modos de usar el término `ser' antes descritos, `lo que no es' lo emplea como nombre de un objeto, y `no puede ser' como una propiedad de este objeto o un predicado de este. Igualmente puede notarse la ruta \enquote*{no es y no puede ser} examinada también anteriormente.

%Tras distinguir la diferencia de sentido de la premisa según la modalidad que se le interprete, Anscombe 
La crítica general de Anscombe al razonamiento de Parmenides consiste en que el argumento completo no es válido si la segunda premisa es entendida \emph{in sensu composito}. Sin embargo, si se interpreta \emph{in sensu diviso}, la premisa misma no es creíble\footnote{\Cite[vii]{anscombe1981parmenides}: \enquote{The impossibility of what is not isn't just the impossibility of the proposition ``What is not, is'' ---i.e. the truth of ``What is not cannot be'', taken \emph{in sensu composito}. \emph{That} could be swept aside as irrelevant. What is not can't be indeed, but it may come to be, and in this sense what is not is possible. When it \emph{has} come to be, of course it no longer is what is not, so in calling it possible we aren't claiming that ``What is not is'' is possible. So it can't be shown to be impossible that it should come to be just by pointing to the impossibility that it is. ---But this can't be the whole story. That what is not is nothing implies that there isn't anything to come to be. So ``What is not can be'' taken in \emph{sensu diviso}, namely as: ``Concerning what is not, \emph{that} can be'' is about nothing at all. If it were about something, then it would be about something that is not, and so there'd be an example of ``What is not is'' that was true}.}.
%Esto lo explica diciendo: \blockquote[{\Cite[vii]{anscombe1981parmenides}}: \enquote{The impossibility of what is not isn't just the impossibility of the proposition ``What is not, is'' ---i.e. the truth of ``What is not cannot be'', taken \emph{in sensu composito}. \emph{That} could be swept aside as irrelevant. What is not can't be indeed, but it may come to be, and in this sense what is not is possible. When it \emph{has} come to be, of course it no longer is what is not, so in calling it possible we aren't claiming that ``What is not is'' is possible. So it can't be shown to be impossible that it should come to be just by pointing to the impossibility that it is. ---But this can't be the whole story. That what is not is nothing implies that there isn't anything to come to be. So ``What is not can be'' taken in \emph{sensu diviso}, namely as: ``Concerning what is not, \emph{that} can be'' is about nothing at all. If it were about something, then it would be about something that is not, and so there'd be an example of ``What is not is'' that was true.}]{La imposibilidad de lo que no es, no es solo la imposibilidad de la proposición ``lo que no es, es'' ---es decir, la verdad de ``Lo que no es no puede ser'', tomado \emph{in sensu composito}. \emph{Eso} puede ser descartado como irrelevante. Lo que no es, ciertamente no puede estar siendo, pero puede llegar a ser, y en este sentido lo que no es es posible. Cuando \emph{haya} llegado a ser, ciertamente ya no es lo que no es, así que en llamarlo posible no estamos declarando que ``Lo que no es, es'' es posible. Entonces no puede mostrarse como imposible que pueda llegar a ser solo por señalar la imposibilidad de que este siendo. ---Pero esta no puede ser toda la historia. Que lo que no es, es nada implica que no hay nada ahí para llegar a ser. Así ``Lo que no es puede ser'' tomado en \emph{sensu diviso}, dígase como: ``Con respecto a lo que no es, eso puede ser'' es acerca de nada en absoluto. Si fuera acerca de algo, entonces sería sobre algo que no es, y así habría un ejemplo de ``Lo que no es, es'' que sería verdadero.}
Si la premisa se toma en sentido general su significado es irrelevante para el argumento. Si se toma en sentido particular es relevante para el argumento, pero es una proposición que no es creíble; lo mismo ocurre con la conclusión: \blockquote[{\Cite[3]{anscombe1981parmenides:pmc}}: \enquote{Concerning that which is not, it holds that \emph{that} cannot be thought}.]{\enquote{Con respecto a aquello que no es, se sostiene que \emph{eso} no puede ser pensado}}. la cual también es increíble.

Al decir que la premisa no es creíble porque es \enquote*{es acerca de nada en absoluto} Anscombe no esta situada desde la comprensión del lenguaje como representación, es decir, no está afirmando que la premisa no representa un objeto posible, sino que está criticando que la premisa misma no dice nada, no puede ser aplicada.
%El problema se encuentra en la proposición misma; \blockquote[{\Cite[5]{anscombe1981parmenides:pmc}}: \enquote{whether we interpret the premise as saying: `What doesn't exist can't exist' or as: `What isn't the case can't be the case' the proposition is not credible}]{ya sea que interpretemos la premisa como diciendo: `Lo que no existe no puede existir' o como: `Lo que no es de hecho no puede ser de hecho' la proposición no es creíble}. Para ilustrar esto de otra manera Anscombe representa la premisa según su estructura lógica de este modo:
%    \begin{center}$(F)(x)\enspace {\sim}F\enspace x\quad \longrightarrow\quad Nec\enspace {\sim}Fx$\end{center}
%Donde la variable $x$ representa un objeto y la variable $F$ representa una propiedad predicada del objeto $x$. La implicacíon de que concerniendo un objeto concreto con una propiedad concreta predicada de él, necesariamente, de la negación del predicado del objeto, se sigue la negación de la conjunción del objeto y su predicado, no es una afirmación creíble. Para que Parménides pueda juzgar creíble su afirmación tiene que basarse en el supuesto de que necesariamente una propiedad predicada de un objeto tiene que ser existente, tiene que tener un referente en la realidad. Según esto la variable de la propiedad $F$ tiene que ser representativa de una propiedad existente. Contra esto, Anscombe enfatiza la independencia del uso lenguaje respecto de la realidad: \blockquote[{\Cite[5]{anscombe1981parmenides:pmc}}: \enquote{it is false that one mentions either properties or objects when one uses the quantifiers binding property variables and object variables; though it has to be granted that some authors, such as Quine, are accostumed to speak of the reference of variables. But if this is given up, as it ought to be, Parmenides is deprived of his claim that we are commited to self-contradiction in existence just because we are willing to use a self-contradictory predicate --- e.g. in the sentence saying that nothing has a self-contradictory predicate true of it --- so that our property-variable is admitted to range over self-contradictory properties.}]{es falso que mencionamos propiedades u objetos cuando usamos cuantificadores para enlazar variables de propiedades con variables de objetos; aunque habría que reconocer que algunos autores, como Quine, están acostumbrados a hablar de la referencia de las variables. Pero si esto es abandonado, como debería de serlo, Parménides queda privado de su declaración de que estamos comprometidos con la auto-contradicción en la existencia solo porque estamos dispuestos a usar un predicado auto-contradictorio --- por ejemplo en la afirmación de que no hay algo que tenga un predicado auto-contradictorio verdadero de ello --- así que se le puede permitir a nuestra variable-propiedad abarcar también propiedades auto-contradictorias.}

A Anscombe le parece acertada la dirección de la argumentación de Parménides en sostener lo concebible desde lo posible, sin embargo rechaza que para afirmar esto haya que establecer un vínculo metafísico entre lo posible y lo concebible. Igualmente rechaza que sea necesario creer que \enquote*{Lo que no es no puede ser pensado} para evitar sostener la creencia de que lo existente puede ser auto-contradictorio. Para Anscombe lo que no es puede ser pensado y esto no implica creer que lo existente puede ser auto-contradictorio.

Aquí tenemos en el centro del análisis de Anscombe la visión de \emph{Investigaciones Filosóficas} sobre la relación entre la realidad, el lenguaje y el pensamiento. En la \S429 se afirma: \blockquote[{\Cite[\S429]{wittgenstein1953phiinv}}: \enquote{The agreement, the harmony, between thought and reality consists in this: that if I say falsely that something is \emph{red}, then all the same, it is \emph{red} that it isn't. And in this: that if I want to explain the word ``red'' to someone, in the sentence ``That is not red'', I do so by pointing to something that \emph{is} red}.]{La concordancia, la armonía, entre pensamiento y realidad consiste en esto: que si digo falsamente que algo es \emph{rojo}, entonces aún así, es \emph{rojo} eso que eso no es. Y en esto otro: que si quiero explicar la palabra ``rojo'' a alguien, en la oración ``Eso no es rojo'', lo haría por medio de señalar a algo que \emph{es} rojo}. Lo que esta sección propone es que contrario a la comprensión del \emph{Tractatus} de que la realidad y el pensamiento están unidos porque comparten la forma lógica, el pensamiento y la realidad, más bien, quedan unidos en el uso que se hace del lenguaje.
%: \blockquote[{\Cite[17-18]{hacker2000mind}}: \enquote{It was a mistake to conceive of the agreement or harmony between language and reality as an agreement of form. It is misguided to think of the \emph{grammatical} proposition `If I say falsely that something is \emph{red}, then, for all that, it isn't \emph{red}' as displaying a harmony \emph{between} thought and reality, a harmony which demands an elaborate logico-metaphysical explanation of the essential projective co-ordination of language and world. The apparent harmony is not orchestrated between a thought and a situation (which may or may not obtain) or between names and their isomorphic meanings which constitute the substance of the world, but rather \emph{between one proposition and another}. For it is a rule of our language that `It is false that $p$' = `not-$p$'. It is a grammatical proposition, not a metaphysical truth about the relation between language and reality, that if it is false that this is red, then this is not red. Indeed, it is impossible that there be a language in which what we describe by `not-$p$' would be expressed without using `$p$'. `Like everything metaphysical, the harmony between thought and reality is to be found in the grammar of the language'. \textelp{} It is correct that one can read off from the proposition that $p$ the fact that makes it true, but that does not betoken a pre-established harmony between language and reality. It is merely \emph{a move in grammar} licensed by the substitution-rule: `the proposition that $p$' = `the proposition which the fact that $p$ makes true'}]{Fue un error concebir la concordancia o la armonía entre lenguaje y realidad como una concordancia de forma. Es desacertado pensar que la proposición \emph{gramática} `Si digo falsamente que algo es \emph{rojo}, entonces, con todo y eso, eso no es \emph{rojo}' está mostrando armonía \emph{entre} pensamiento y realidad, una armonía que reclama una elaborada explicación lógico-metafísica de la esencial co-ordinación proyectiva de lenguaje y mundo. La aparente armonía no esta orquestada entre un pensamiento y una situación (que puede ser de hecho o no) o entre nombres y sus significados isomórficos que constituyen la sustancia del mundo, sino más bien \emph{entre una proposición y otra}. Pues es una regla de nuestro lenguaje que `Es falso que $p$' = `no-$p$'. Es una proposición gramática, no una verdad metafísica sobre la relación entre el lenguaje y la realidad, que si es falso que esto es rojo, entonces esto no es rojo. Ciertamente, es imposible que haya un lenguaje en el cual lo que describimos por medio de `no-$p$' se expresara sin usar `$p$'. `Como todo lo metafísico, la armonía entre pensamiento y realidad se encuentra en el lenguaje'. \textelp{} Es correcto que podemos leer desde la proposición que $p$ el hecho que la hace verdadera, pero eso no anuncia una armonía pre-establecida entre lenguaje y realidad. Es meramente \emph{un movimiento en la gramática} permitido por la regla de substitución: `la proposición que $p$' = `la proposición a la cual el hecho que $p$ hace verdadera'}
Desde esta perspectiva Anscombe propone que se debe abandonar la inclinación a vincular los signos del lenguaje a algún referente en la realidad a la hora de analizar una proposición como hace ella con la segunda premisa del argumento parmenidiano.

Después de estudiar y abandonar esta segunda premisa , Anscombe se fija en la primera premisa y dice: \blockquote[{\Cite[5]{anscombe1981parmenides:pmc}}: \enquote{That other arm of his first premise, which he does not in fact use, remains tantalizing. What he used was `Only that can be thought, which can be'; the other arm of his premise is `Only that can be, which can be thought'}.]{Esa otra rama de su primera premisa, que él de hecho no usa, sigue siendo prometedora. Lo que él usó fue `Solo eso puede ser pensado, lo que puede ser'; la otra rama de su premisa es `Solo eso puede ser, lo que puede ser pensado'}. Entonces propone: \blockquote[{\Cite[5]{anscombe1981parmenides:pmc}}: \enquote{We might call this arm of the premise the `No Mystery' arm. If some way of charactherizing what can be thought could be found, then if this proposition is true, there's a quick way of excluding mysteries}.]{Podemos calificar a esta rama de la premisa como la rama del `No misterio'. Si alguna manera de caracterizar lo que puede ser pensado puede encontrarse, entonces si esta proposición es verdadera, hay aquí una manera rápida de excluir los misterios}. Sobre la rama que sí usa Parménides, Elizabeth dirá que si se interpreta como: \blockquote[{\Cite[6]{anscombe1981parmenides:pmc}}: \enquote{Only what can exist or be the case can, without misunderstanding, logical error, or confusion, be thought to exist or be the case}.]{Solo lo que puede existir o ser de hecho puede, sin malentendidos, error lógico, o confusión, ser pensado como existiendo o siendo de hecho}, puede ser una proposición quizás aceptable. Sin embargo Anscombe se enfocará en la rama de la premisa que Parménides no usa, y se concentrará entonces en describir en qué puede consistir caracterizar lo que puede ser pensado.

Anscombe se cuestiona \blockquote[{\Cite[7]{anscombe1981parmenides:pmc}}: \enquote{What are we to make of this premise anyway?}]{¿Pero cómo hemos de tomar esta premisa?}, despues de todo: \blockquote[{\Cite[7]{anscombe1981parmenides:pmc}}: \enquote{It appears to draw attention to the possibilities for thought --- and who knows what they are? If I say I can think something, what of it? If I say I can't, does that mean I can't manage to do what I do in the other case? Again, what of it?}]{Parece que dirige la atención hacia las posibilidades del pensamiento --- y ¿quién sabe cuáles son? Si digo que puedo pensar algo, ¿de qué vale? Si digo que no puedo, ¿entonces quiere decir que no puedo lograr hacer eso de lo que soy capaz en el otro caso? De nuevo, ¿y qué con eso?}. Si intentamos negar la proposición: \blockquote[{\Cite[7]{anscombe1981parmenides:pmc}}: \enquote{There may be what can't be thought. (Not: what one can't invest with the feeling of having thought it, but what eludes explanation, what remains enigmatic)}.]{Puede haber lo que no puede ser pensado. (No: lo que no podemos otorgarle el sentimiento de haberlo pensado, sino lo que escapa a la explicación, lo que permanece como enigmático)}, parece ser una noción inofensiva; entendida como \blockquote[{\Cite[7]{anscombe1981parmenides:pmc}}: \enquote{Something that can't be thought may be}.]{Algo que no puede ser pensado puede ser} parece que se trata de un pensamiento que aún no es de nada en particular. Sin embargo, ¿no sería preferible poder refutar: \blockquote[{\Cite[7]{anscombe1981parmenides:pmc}}: \enquote{There may be what can't be thought}.]{Puede haber lo que no puede ser pensado} o \blockquote[{\Cite[7]{anscombe1981parmenides:pmc}}: \enquote{Something may be which can't be grasped in thought}.]{Puede haber algo que no puede ser captado en el pensamiento}? Si esto pudiera refutarse \blockquote[{\Cite[7]{anscombe1981parmenides:pmc}}: \enquote{no one could have any right to produce a \emph{particular} sentence and say: this is true, but what it says is irreducibly enigmatic}.]{nadie podría tener el derecho a producir una afirmación \emph{particular} y decir: esto es verdadero, pero lo que dice es irreduciblemente enigmático}.

Desde luego, dice Elizabeth, si una afirmación es simplemente `abracadabra', es decir, puro sinsentido, no hay que prestarle atención, pero ¿qué sucede con las expresiones que no son sinsentido, pero que aún presentan dificultades a la hora de determinar para ellas un sentido inobjetable? En esos casos ¿podríamos descartar la posibilidad de que este sentido enigmático sea una verdad? Anscombe sugiere que si pudiera quedar demostrado el principio de Parménides, de la rama de la premisa que no usó, \blockquote[{\Cite[6]{anscombe1981parmenides:pmc}}: \enquote{Only what can be thought of can be}.]{Solo aquello de lo que puede pensarse puede ser}, entonces: \blockquote[{\Cite[8]{anscombe1981parmenides:pmc}}: \enquote{Since the sentence cannot be taken as expressing a clear thought ---i.e. a thought which is clearly free from contradiction or other conceptual disorder---therefore it doesn't say anything, and therefore not anyting true. And that would be very agreeable. We could perhaps become quite satisfied that a sentence was in that sense irreducibly enigmatic --- and so we could convince ourselves we had the right to dismiss it}.]{Puesto que la oración no puede ser tenida como expresión de un pensamiento claro ---es decir, un pensamiento que está claramente libre de contradicción o algún otro desorden conceptual--- entonces no dice nada, y por tanto nada verdadero. Y esto sería muy aceptable. Podríamos quizas llegar a estar muy satisfechos de que una oración fuera en este sentido irreduciblemente enigmática --- y entonces podríamos convencernos de que hemos tenido el derecho de descartarla}.

Con esto, Anscombe identifica lo que parece ofrecer un modo de caracterizar lo que puede ser pensado: \blockquote[{\Cite[8]{anscombe1981parmenides:pmc}}: \enquote{This suggests as the sense of ``can be grasped in thougth''; ``can be presented in a sentence which can be seen to have an unexceptionable non-contradictory sense''. A form of: whatever can be said at all can be said clearly}.]{Esto sugiere como el sentido de ``puede ser captado en el pensamiento''; ``puede ser presentado en una oración que pueda ser vista como teniendo un irreprochable sentido no-contradictorio''. Una forma de: todo lo que puede ser expresado en absoluto puede ser expresado claramente}.

Sin embargo, aunque para Anscombe sería aceptable pensar en ``ser presentado en una afirmación que pueda verse que tiene un inobjetable sentido no-contradictorio'' como la manera de afirmar lo que podría ser captado en el pensamiento, le parece que esto no sirve para establecer que haya alguna cosa que no pueda ser pensada: \blockquote[{\Cite[8]{anscombe1981parmenides:pmc}}: \enquote{Someone who thought this \emph{might} think ``There may be the inexpressible.'' And so in that sense think ``There may be what can't be thought''. ---But he wouldn't be exercised by any definite claimant to be that which can't be grasped in thought. \emph{Mystery} would be illusion\,---\,either the thought expressing something mysterious could be clarified, and then no mystery, or the impossibility of clearing it up would show it was really a non-thought. The trouble is, there doesn't seem to be any ground for holding this position. It is a sort of prejudice}.]{Alguien que piense esto \emph{puede} pensar ``Puede haber lo inexpresable.'' Y entonces en ese sentido ``Puede haber lo que no puede ser pensado''. ---Pero no estaría siendo movido por alguna cosa determinada que le estuviera reclamando ser aquello que no puede ser captado en el pensamiento. El \emph{misterio} sería una ilusión\,---\,una de dos, el pensamiento expresando algo misterioso podría ser clarificado, y entonces no hay misterio, o la imposibilidad de aclararlo mostraría que era verdaderamente un no-pensamiento. El problema es, que no parece haber ningún fundamento para sostener esta posición. Es una especie de prejuicio}.

Con estas últimas expresiones Anscombe ha dejado expuestos los elementos que componen su discusión sobre la relación entre lo concebible y lo posible y junto a esto el modo en el que puede ser caracterizado lo que puede ser pensado y lo que pueda ser sinsentido y la peculiaridad del misterio. Anscombe compara su proposición acerca de lo que puede caracterizar lo que puede ser pensado con la afirmación que se encuentra en el prefacio del \emph{Tractatus}, \enquote*{lo que puede ser expresado en absoluto puede ser expresado claramente}; sin embargo, juzga como un prejuicio la creencia, expresada también en el \emph{Tractatus}, de que esto implique que \enquote*{hay lo inexpresable}, o \enquote*{hay lo que no puede ser pensado}. Aquí Anscombe está acuñando una herramienta útil del modo en el que el \emph{Tractatus} efectivamente propone examinar las proposiciones para mostrar si expresan pensamiento: \blockquote[{\Cite[151]{anscombe1959iwt}}: \enquote{The criticism of sentences as expressing no real thought, according to the principles of the \emph{Tractatus}, could never be of any very simple general form; each criticism would be \emph{ad hoc}, and fall within the subject-matter with which the sentence professed to deal}.]{La crítica de las proposiciones como no expresando ningún pensamiento real, de acuerdo con los principios del \emph{Tractatus}, nunca podría consistir de alguna forma general muy simple; cada crítica tendría que ser \emph{ad hoc}, y estar relacionada con el sujeto de la materia con la cual la proposición está profesamente lidiando}. Lo que Elizabeth rechaza es que haya un principio general que \emph{a priori} sirva para descartar alguna clase de proposiciones porque no expresan pensamiento. Cada proposición tiene que ser examinada.

Junto a esto, el análisis que Anscombe propone acerca de la relación entre la realidad y el pensamiento está dirigido hacia el uso del lenguaje. En el uso de los signos del lenguaje dentro de la vida es donde se encuentran pensamiento y realidad, esto como contrario a la idea de que la relación entre pensamiento y realidad se encuentra en una armonía metafísica \emph{a priori}. De ahí que su propuesta sobre lo que puede caracterizar un pensamiento dirija la atención a la posibilidad de presentar el pensamiento en el lenguaje.

Teniendo en cuenta estas ideas, para Anscombe, creer en un misterio no presupone una actitud acrítica que abrace la contradicción, sino que consiste mas bien en la disposición de examinar el uso que se hace de las expresiones en el lenguaje y la actividad humana, teniendo en cuenta que los misterios son expresiones que no pueden quedar definitivamente demostradas, pero que tampoco pueden quedar descartadas como no expresando un pensamiento posible.

El tema que se ha empezado a discutir en este artículo y que continúa en \emph{The Question of Linguistic Idealism} consiste en un esfuerzo por analizar \enquote{la aptitud del lenguaje humano para hablar de forma significativa y verdadera incluso de lo que supera toda experiencia humana} (FR 67). 

Este análisis esta relacionado con la cuestión de la analogía. 

, componen una discusión en la que Anscombe estudia elementos importantes de la relación entre lenguaje, pensamiento y realidad dentro de la obra de Wittgenstein. La noción del lenguaje que esta discusión detalla el terreno en el que el testimonio es la práctica lingüística
y \emph{On Transubstantiation} recogen las ideas de Anscombe relacionadas más directamente con la tercera cuestión planteada en el primer capítulo. 



 
FR 83 DH 3016
La vida misma del creyente, la vida de la Iglesia son signo visible de la presencia de Dios vivo
El ser humano es imagen de Dios

posturas previas a la definicion de la verdad 

se propone
\blockquote[{\Cite[Cf.][354-356]{dominguez2009at}. Destaca tres estadios de la actitud ante la verdad en la historia: \enquote{En la época medieval, \emph{grosso modo}, la verdad era testimoniada \textelp{} el amante de la verdad, el hombre corriente, era consciente de ser ``testigo'' de una verdad que le excedía, y que le había sido dada \textelp{} la pretensión humana de ser autónomo chocó de frente con esta actitud \textelp{dando} paso a la concepción ilustrada donde la verdad era el confereciante mismo \textelp{y posteriormente} la época contemporánea post-idealista niega la existencia de la verdad ontológica.}}]{mostrar qué actitudes no hacen justicia a la noción de libertad en el hombre que se sigue de su ser \emph{imago Dei}, y cuál sí}.

La postura que se tiene previo a la definición de la verdad la actitud hacia la condición de posibilidad de algún tipo de existencia ---o inexistencia de la verdad--- es una ``consideración mística''. 

\blockquote[357]{Cabría decir que ésta es una consideración \emph{mística}. Con místico me refiero, en principio al uso wittgensteniano: ``Nicht wie die Welt ist, ist das Mystische, sondern daß sie ist''. En efecto, a la actitud mística pertenece la persuasión de que la verdad es un tema vital que trasciende, la certeza de que ante ella el ejercicio racional culmina en contemplación. Es, de nuevo, un momento en el que la racionalidad filosófica de pone en busca de otro nivel que la supera: la Revelación.

Esta dimensión mistica no está lejos de la metafísica, ni de la ética. \textelp{} En consecuencia, la actitud previa que se ha de mantener ante la verdad es, no cabe duda, martirial; sí, testimonial en el sentido cabal de la palabra}.

\blockquote[{\Cite[410-411]{dominguez2009at}}]{Este Absoluto concreto, por el que entramos en la vida de la Trinidad, no es una ``abstracción '' inexistenete, sino que está presente en la expresión más viva de la experiencia de la fe.}

La tercera cuestión planteada en el primer capítulo 
Una cuestión compleja relacionada con el testimonio tiene que ver con el hecho de que En qué consiste la pregunta sobre qué hace significativo al lenguaje.
Relación entre lo concebible y lo posible
cual es la respuesta dentro de la filosofía de Anscombe
a en qué consiste que algo sea verdadero o que algo sea de tal cosa
es una pregunta sobre el significado

esta discusión y qli nos dan una noción de los fundamentos que 
la relación de este tema con el testimonio está en la noción misma del lenguaje
