\subsection{On Transubstantiation (1974)}

Originalmente publicado en un panfleto por el \emph{Catholic Truth Society} en Londres en 1974, \emph{On Transubstantiation} es uno de los escritos recogidos en \emph{Ethics, Religion and Politics}, el tercer volumen de los \emph{Collected Philosophical Papers} de Anscombe. El volumen contiene escritos dirigidos a un público general, o publicados en revistas o reuniones filosóficas; también incluye otros escritos compuestos pensando en lectores católicos, como es el caso de este documento. El dato permite anticipar que Anscombe escribe aquí como católica, dando por hecho presupuestos propios del trasfondo de fe que comparte con los católicos a los que se dirige en su discusión.

Considerando el objetivo de su reflexión, resulta llamativo el modo en que Anscombe comienza y termina el artículo. En el centro de su atención está el misterio de la presencia de Jesús en la Eucaristía y elige comenzar su discusión diciendo: \blockquote[
%{\Cite[85]{torralbaynubiola2005fayeh:ot}}.
{\Cite[108]{anscombe1981erp:ot}}: \enquote{It is easiest to tell what transubstantiation is by saying this: little children should be taught about it as early as possible. Not of course using the word ``transubstantiation'', because it is not a little child's word.} El texto en español de este artículo se ha tomado de {\cite{torralbaynubiola2005fayeh:ot}}.
]{El modo más sencillo de expresar lo que es la transubstanciación es decir que ha de enseñarse a los niños pequeños tan pronto como sea posible, sin usar, por supuesto, la palabra ``transubstanciación'' porque no pertenece al vocabulario infantil}. Esta propuesta invita ya varias consideraciones. Anscombe toma eso que la expresión `transubstanciación' quiere denominar y sugiere que ese misterio puede enseñarsele a un niño, sin usar la palabra `transubstanciación', que el niño no entendería. Esto, además, mientras más pronto se enseñe mejor.

La propuesta hasta aquí sugiere que un misterio puede ser enseñado empleando otro recurso que no sea un concepto. También que una persona familiarizada con el misterio puede compartirlo con alguien que lo ignora, y ambos estarían creyendo el mismo misterio independientemente de la capacidad de comprensión de cada uno. En este mismo tono se encuentra la conclusión del artículo: \blockquote[
%{\Cite[94]{torralbaynubiola2005fayeh:ot}}.
{\Cite[112]{anscombe1981erp:ot}}: \enquote{It is the mystery of faith which is the same for the simple and for the learned. For they believe the same, and what is grasped by the simple is not better understood by the learned: their service is to clear away the rubbish which the human reason so often throws in the way to create obstacles}.
]{Éste es el misterio de la fe, que es el mismo para los ignorantes y para los sabios, pues creen lo mismo; y lo que los ignorantes entienden no es comprendido mejor por los sabios, cuya tarea es quitar de en medio la basura que tan a menudo la razón humana arroja en el camino para crear obstáculos}. Esta tarea de los entendidos describe también la relación de un concepto como `transubstanciación' con el misterio al que se refiere. No pretende ser la herramienta que se emplea para demostrar de una vez por todas que el misterio es perfectamente posible, sino que se usa para desenredar las objeciones que puedan ser presentadas en contra del misterio.

Estas consideraciones son valiosas porque caracterizan el modo en que Anscombe actúa cuando habla del misterio. Distingue entre el misterio y los conceptos que se emplean para hablar de él e insiste en el papel que juegan estas expresiones. Distingue también en qué consiste la actitud de aquellos que creen en el misterio, sostener la creencia no significa abrazar la contradicción. Su insistencia es que precisamente porque no se persigue afirmar que lo absurdo puede ser verdadero cuando se está creyendo un misterio, se cree que hay respuestas a cualquier argumento que pretenda demostrar el misterio como un absurdo: \blockquote[
%{\Cite[88-89]{torralbaynubiola2005fayeh:ot}}.
{\Cite[109]{anscombe1981erp:ot}}: \enquote{in the philosophy of scholastic Aristotelianism in which those distinctions \textins{between substance of a thing and its accidents} were drawn, transubstantiation is as difficult, as `impossible', as it seems to any ordinary reflection. And it is right that it should be so. When we call something a mystery, we mean that we cannot iron out the difficulties about understanding it and demonstrate once for all that it is perfectly possible. Nevertheless we do not believe that contradictions and absurdities can be true, or that anything logically demonstrable from things known can be false. And so we believe that there are answers to supposed proofs of absurdity, whether or not we are clever enought to find them}.
]{en la filosofía de la escolástica aristotélica en la que se trazaron aquellas distinciones \textins{entre la substancia de algo y sus accidentes}, la transubstanciación resulta tan difícil, tan ``imposible'', como lo parece en la reflexión ordinaria. Y es bueno que sea así. Cuando llamamos a algo un misterio, queremos decir que no podemos solventar las dificultades en su comprensión ni demostrar de una vez por todas que es perfectamente posible. Sin embargo, no creemos que las contradicciones y los absurdos puedan ser verdaderos o que algo demostrable lógicamente a partir de lo ya conocido pueda resultar falso. Y, por tanto, creemos que hay respuestas para las supuestas pruebas de su carácter absurdo, seamos o no lo suficientemene listos para encontrarlas}.

Podemos regresar a la propuesta hecha por Anscombe en el comienzo. ¿Cómo se enseña a un niño sobre la transubstanciación sin emplear este concepto? Elizabeth responde: \blockquote[
%{\Cite[85]{torralbaynubiola2005fayeh:ot}}.
{\Cite[107]{anscombe1981erp:ot}}: \enquote{the thing can be taught, and it is best taught at mass at the consecration, the one part where a small child should be got to fix its attention on what is going on}.
]{puede enseñárseles, y la mejor manera de hacerlo es en la Misa durante la consagración, que es la única parte en la que ha de conseguirse que el niño pequeño atienda a lo que está ocurriendo}. En ese momento se le puede enseñar al niño diciéndole en voz baja \blockquote[
%{\Cite[85]{torralbaynubiola2005fayeh:ot}}.
{\Cite[107]{anscombe1981erp:ot}}: \enquote{Look! Look what the priest is doing \ldots He is saying Jesus' words that change the bread into Jesus' body. Now he's lifting it up. Look! Now bow your head and say `My Lord and my God'}.
]{¡Mira! Mira lo que el sacerdote está haciendo\ldots  Está diciendo las palabras que convierten el pan en el cuerpo de Jesús. Ahora lo está elevando. ¡Mira! Ahora incilina tu cabeza y di `Señor mío y Dios mío'}. Y, luego, cuando se eleva el cáliz: \blockquote[
%{\Cite[85]{torralbaynubiola2005fayeh:ot}}.
{\Cite[107]{anscombe1981erp:ot}}: \enquote{Look, now he's taken hold of the cup. He's saying the words that change the wine into Jesus' blood. Look up at the cup. Now bow your head and say `We believe, we adore your precious blood, O Christ of God'}.
]{Mira, ahora ha cogido el caliz. Está diciendo las palabras que convierten el vino en la sangre de Jesús. Mira el cáliz. Ahora inclina la cabeza y di `Creemos y adoramos tu preciosa Sangre, oh Cristo de Dios'}.

La invitación que se le está haciendo al niño no es simplemente a observar lo que está ocurriendo en el momento de la consagración, sino a unirse en adoración a quien ahora está presente sobre el altar. Esta adoración \blockquote[
%{\Cite[86]{torralbaynubiola2005fayeh:ot}}.
{\Cite[107]{anscombe1981erp:ot}}: \enquote{carries with it implicitly the belief in the divinity and the resurrection of the Lord. And if we do believe in his divinity and in his resurrection then we must worship what is now there on the altar}.
]{lleva consigo implícitamente la creencia en la divinidad y en la resurrección del Señor. Y si creemos en su divinidad y en su resurrección, entonces debemos adorar lo que está ahora allí sobre el altar}. De este modo \blockquote[
%{\Cite[85]{torralbaynubiola2005fayeh:ot}}.
{\Cite[107]{anscombe1981erp:ot}}: \enquote{If the person who takes a young child to mass always does this \textelp{} the child thereby learns a great deal}.
]{Si la persona que lleva a un niño a Misa actúa siempre así \textelp{} el niño aprenderá mucho}.

La propuesta de Anscombe consiste en introducir al niño a la práctica de la comunidad y relacionarse con el misterio, permitiendo que sus gestos de adoración le ayuden a conectar lo que está ocurriendo en el momento de la consagración con la fe en Jesucristo vivo. Para Elizabeth esta es la mejor manera de educar al niño sobre el misterio: \blockquote[
%{\Cite[86]{torralbaynubiola2005fayeh:ot}}.
{\Cite[107]{anscombe1981erp:ot}}: \enquote{Thus by this sort of instruction the little child learns a great deal of the faith. And it learns in the best possible way: as part of an action; a concerning something going on before it; as actually unifying and connecting beliefs, which is clearer and more vivifying than being taught only later, in a classroom perhaps, that we have all these beliefs}.
]{Así, mediante una enseñanza de este tipo, el niño pequeño aprende mucho de la fe. Y lo aprende del mejor modo posible: como parte de una acción; como relacionado con algo que sucede ante él; como algo que unifica y conecta efectivamente las creencias; esto es más claro y vivificante que aprender sólo más tarde, quizá en una clase, que todos nosotros tenemos esas creencias}.

La descripción de este escenario no solo responde a la importancia que tiene en sí mismo, sino que le parece a Anscombe que es el modo de sacar a la luz más claramente lo que `transubstanciación' significa. Lo que decimos cuando usamos esta palabra es exactamente lo que enseñamos a un niño cuando el sacerdote, en el lugar de Cristo y usando sus palabras, por el poder divino hace que el pan quede cambiado de modo que ya no está ahí, sino que es el cuerpo de Jesús. El término `transubstanciación' apunta a esa conversión de una realidad física en otra que ya existe. ¿Es posible este cambio? Si se sostiene que es imposible ha de mostrarse una contradicción determinada. Por otra parte, creer en esto implica creer que toda pretensión de refutarlo como contradictorio puede ser refutada. Para ser creído no necesita ser expuesto de tal modo que no hubiera en él ningún misterio.

Para Anscombe, sin embargo, lo más misterioso del sacrificio Eucarístico no es el cambio del pan en el cuerpo de Cristo, sino su significado, el hecho misterioso de que Cristo haya querido alimentarnos consigo mismo. Quizás estamos acostumbrados a la idea de la comunión, pero suele pasar desapercibido cuán misteriosa es la idea. En antiguas discusiones se encuentran los debates entre protestantes y católicos acerca de si lo que comemos es el cuerpo de Cristo realmente o solo un símbolo. Parece que solo es extravagante la creencia católica de que está presente realmente, mientras que los protestantes tendrían la posición más razonable de comer el cuerpo y beber la sangre de Cristo solo simbólicamente, la extrañeza de comer y beber el cuerpo y la sangre, incluso de manera simbólica no queda atendida. En tiempos más recientes algunos teólogos han querido explicar la transubstanciación como transignificación. Aquí, una vez más, lo extraño pasa desapercibido, que lo que queda transignificado en la eucaristía no es el pan y el vino, sino el cuerpo y la sangre de Cristo, que quedan transignificados en alimento, ese es el misterio.

Si examinamos lo que Jesús hace en la Última Cena: toma el pan, reza, lo parte y lo da a sus discípulos; vemos que hace la acción de gracias en la celebración de la Pascua. Y a su oración añade \enquote*{Esto es mi cuerpo}, y luego toma el cáliz y dice \enquote*{Es mi sangre que será derramada por vosotros}. De este modo muestra que su muerte será el sacrificio del que Él mismo es sacerdote. Sus acciones muestran que para nosotros Él mismo ha reemplazado el cordero pascual, asume el lugar del cordero que se ofrece en sacrificio de comunión al invitarnos a comer de él. Anscombe considera que este darnos de comer de su cuerpo es un símbolo: \blockquote[
%{\Cite[91]{torralbaynubiola2005fayeh:ot}}.
{\Cite[110]{anscombe1981erp:ot}}: \enquote{So his flesh and blood are given us for food, and this is surely a great mystery. It is clearly a symbol: we are not physically nourished by Christ's flesh and blood as the Jews were by the paschal lamb}.
]{De este modo su Carne y su Sangre se nos dan como alimento, lo que es, por supuesto, un gran misterio. Es claramente un símbolo pues nosotros no somos alimentados físicamente con el Cuerpo y la Sangre de Cristo como lo fueron los judíos con el cordero pascual}. Aquí lo que Anscombe quiere decir no es que es simbólico el que se esté comiendo el cuerpo de Cristo, sino que ya sea comer y beber simbólica o literalmente su cuerpo y sangre, esa comida y bebida son en sí mismas simbólicas; y lo que representa no es un símbolo natural, sino que es difícil de comprender qué significa el comer y beber el cuerpo y la sangre de Jesús.

Las palabras de Jesús: \enquote*{Yo soy el pan que ha bajado del cielo} pueden ser entendidas como una metáfora en la que el Señor esta afirmando: \enquote*{Yo mismo seré el alimento de la vida de que hablo}. Cristo no dice \enquote*{Yo tengo alimento para vosotros}, del mismo modo que no dice \enquote*{Mi camino es el camino} o \enquote*{Yo os muestro la verdad}, sino que afirma \enquote*{Yo soy el camino\ldots}, \enquote*{Yo soy la verdad\ldots}, \enquote*{Yo soy el pan\ldots}. \blockquote[
%{\Cite[93]{torralbaynubiola2005fayeh:ot}}.
{\Cite[110]{anscombe1981erp:ot}}: \enquote{The commanded action of eating his flesh creates the very same metaphor as the words ---wehter we take the description of the action literally or symbolically. For, even if the words ``I am the bread (i.e. the food) that came down from heaven'' are to be taken literally, still that which they say, and which on \emph{that} understanding is literally so, symbolizes something \emph{else}}.
]{La acción que nos ordenó de comer su Cuerpo constituye exactamente la misma metáfora que esas palabras, tanto si se toma la descripción de la acción simbólicamente como literalmente. Porque, aun cuando las palabras ``Yo soy el pan (esto es, la comida) que ha bajado del cielo'' se tomen literalmente, lo que dicen ---que bajo \emph{esta} comprensión es lo literal--- todavía simboliza alguna \emph{otra cosa}}.

Para Anscombe la más clara de sus metáforas es la de la vid. Podemos decir de modo no metafórico lo que aquí se afirma; la vida de la que Jesús habla es su propia vida y esta es la que comparte con sus discípulos como la vid a los sarmientos. Esto aclara algo del misterio. Cristo no solo quiere comunicar a sus discípulos sus enseñanzas, sino compartirles su propia vida divina. En ese sentido podríamos entender que no nos diga que él puede mostrarnos el camino, sino que Él es el camino. Sin embargo nuestra comprensión vuelve a encontrarse con un límite, porque \blockquote[
%{\Cite[93]{torralbaynubiola2005fayeh:ot}}.
{\Cite[110]{anscombe1981erp:ot}}: \enquote{no one can know what it means to live with the life of God himself}.
]{nadie puede saber qué significa para nosotros vivir con la vida de Dios mismo}. A esto es que se refiere Elizabeth cuando afirma que le parece que lo que comer el cuerpo y beber la sangre de Jesús simboliza es profundamente misterioso.

Estos modos de hablar de Jesús apuntan a la unidad de vida con Él y su mandato de comer de su cuerpo y beber de su sangre es un compartirnos su propia vida divina. Esto también nos constituye en una unidad a todos los que comemos de su cuerpo y bebemos su sangre. De esta unidad también hay modos de hablar. Agustín dice: \enquote*{Nos da su cuerpo para convertirnos en su cuerpo}. También llamamos a la Iglesia el \enquote*{cuerpo místico de Cristo}. Se habla de que todos nacemos \enquote*{miembros de Adán} y en el bautismo somos injertados en el cuerpo de un nuevo Adán. En estas maneras de hablar se emplea la metáfora de que somos como los miembros de un único cuerpo; sin embargo \blockquote[
%{\Cite[94]{torralbaynubiola2005fayeh:ot}}.
{\Cite[112]{anscombe1981erp:ot}}: \enquote{\emph{the unity of the life that is pointed to} in the figure of speech is \emph{no} metaphor}.
]{\emph{la unidad de la vida a la que se alude} en la figura lingüística \emph{no} es una metafora}.

Este es el misterio que creemos y que el sabio no comprende mejor que el ignorante. La vida divina en la que quedamos unidos; \blockquote[
%{\Cite[94]{torralbaynubiola2005fayeh:ot}}.
{\Cite[112]{anscombe1981erp:ot}}: \enquote{Of this life Christ called himself the food. It is the food of the divine life which is promised and started in us: the viaticum of our perpetual flight from Egypt which is the bondage of sin; the sacrificial offering by which we were reconciled; the sign of our unity with one another in him}.
]{Cristo se llamó a sí mismo el alimento de esa vida. Es el alimento de la vida divina que se nos prometió y comenzó en nosotros: el viático de nuestra perpetua huida del Egipto que representa la esclavitud del pecado; el ofrecimiento sacrificial mediante el que fuimos reconciliados; el signo de nuestra unidad de unos con otros en Él}.

\vspace{2.83334em}
La reflexión de Elizabeth en este artículo sirve como contrapunto a sus investigaciones en \emph{Parmenides, Mystery and Contradiction} y \emph{The Question for Linguistic Idealism}. Su modo de describir la capacidad del lenguaje humano para comunicar el misterio divino en este ensayo constituye un modo distinto y más sencillo de abordar esta cuestión, pero armoniza con las argumentaciones más densas que se encuentran en los otros dos artículos.

El aspecto que esta discusión permite destacar es la conexión entre el testimonio particular y el contexto comunitario. La vida de la comunidad en la que se introduce al niño le enseña lo que `transubstanciación' significa. Esto es un ejemplo de como Anscombe comprende la presencia de la Verdad en la práctica lingüística humana; no como una idea que se abstrae, sino como Logos encarnado. En esto consideramos que hay en su pensamiento una idea análoga a una afirmación cristológica como puede ser: \blockquote[{\Cite[410-411]{dominguez2009at}}]{Cristo es el Verbo de Dios hecho hombre \textelp{} Él es el Logos encarnado \textelp{} Este Absoluto concreto, por el que entramos en la vida de la Trinidad, no es una ``abstracción '' inexistente, sino que está presente en la expresión más viva de la experiencia de la fe}.
%Esta experiencia viva de la fe es descrita por Anscombe como adoración junto a la comunidad de Cristo vivo y presente en el altar; también como el alimentarnos de un mismo cuerpo que significa participar de la unidad de la vida divina que Cristo nos comparte. En esta experiencia se expresa la verdad de Dios de tal modo que, como destaca Elizabeth, puede acogerle tanto el ignorante como el sabio.
%Según la concepción de Anscombe es esta experiencia viva de la fe la que introduce en la relación con el misterio
En esta reflexión la expresión viva de la experiencia de la fe que Anscombe describe consiste en la adoración comunitaria de Cristo vivo y presente en el altar; también en el alimentarnos de un mismo cuerpo que significa participar de la unidad de la vida divina que Cristo nos comparte. La aportación de los entendidos, según explica Elizabeth, está en servicio de esta expresión viva. Esta idea la encontraremos nuevamente en \emph{Faith} y está relacionada con la noción de misterio presente aquí y en otros ensayos. El servicio del sabio consiste en quitar de en medio los obstáculos que \enquote{tan a menudo la razón humana arroja en el camino}.
