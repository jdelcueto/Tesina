\subsection{On Transubstantiation}

Originalmente publicado en un panfleto por el \emph{Catholic Truth Society} en Londres en 1974, \emph{On Transubstantiation} es uno de los escritos recogidos en \emph{Ethics, Religion and Politics}, el tercer volumen de los \emph{Collected Philosophical Papers} de Anscombe. El volumen contiene escritos dirigidos a un público general, o empleados para revistas o reuniones filosóficas; también incluye otros escritos compuestos pensando en lectores católicos, como es el caso de este documento. El dato permite anticipar que Anscombe escribe aquí como católica, dando por hecho presupuestos propios del trasfondo de fe que comparte con los católicos a los que se dirige en su discusión.

Considerando el objetivo de su reflexión, resulta llamativo el modo en que Anscombe comienza y termina el artículo. En el centro de su atención está el misterio de la presencia de Jesús en la Eucaristía ---entonces elige comenzar su discusión diciendo: \blockquote[{\cite[108]{anscombe1981erp:ot}}: It is easiest to tell what transubstantiation is by saying this: little children should be taught about it as early as possible. Not of course using the word ``transubstantiation'', because it is not a little child's word. El texto en español de este artículo se ha tomado de {\cite{torralbaynubiola2005fayeh:ot}}]{El modo más sencillo de expresar lo que es la tansubstanciación es decir que ha de enseñarse a los niños pequeños tan pronto como sea posible, sin usar, por supuesto, la palabra ``transubstanciación'' porque no pertenece al vocabulario infantil.} Esta propuesta invita ya varias consideraciones. Anscombe toma eso que la expresión \enquote{transubstanciación} quiere denominar y sugiere que ese misterio puede enseñarsele a un niño, sin usar la palabra \enquote{transubstanciación}, que el niño no entendería. Esto, además, mientras más pronto se enseñe mejor.

La propuesta hasta aquí sugiere que un misterio puede ser enseñado empleando otro recurso que no sea un concepto y también que una persona familiarizada con el misterio puede compartirlo con alguien que lo ignora, y ambos estrían creyendo el mismo misterio independientemente de la capacidad de comprensión de cada uno. En este mismo tono se encuentra la conclusión del artículo: \blockquote[{\cite[112]{anscombe1981erp:ot}}: It is the mystery of faith which is the same for the simple and for the learned. For they believe the same, and what is grasped by the simple is not better understood by the learned: their service is to clear away the rubbish which the human reason so often throws in the way to create obstacles.]{Éste es el misterio de la fe, que es el mismo para los ignorantes y para los sabios, pues creen lo mismo; y lo que los ignorantes entienden no es comprendido mejor por los sabios, cuya tarea es quitar de en medio la basura que tan a menudo la razón humana arroja en el camino para crear obstáculos.} Esta tarea de los entendidos describe también la relación de un concepto como \enquote{transubstanciación} con el misterio al que se refiere. No pretende ser la herramienta que se emplea para demostrar de una vez por todas que el misterio es perfectamente posible, sino que se usa para desenredar las objeciones que puedan ser presentadas en contra del misterio.

Estas consideraciones son valiosas porque caracterizan el modo en que Anscombe actua cuando habla del misterio. Distingue entre el misterio y los conceptos que se emplean para hablar de él e insiste en el papel que juegan estas expresiones. Distingue también en qué consiste la actitud de aquellos que creen en el misterio, sostener la creencia no significa abrazar la contradicción. Su insistencia es que precisamente porque no se persigue afirmar que lo absurdo puede ser verdadero cuando se está creyendo un misterio, se cree que hay respuestas a cualquier argumento que pretenda demostrar el misterio como un absurdo: \blockquote[{\cite[109]{anscombe1981erp:ot}}: in the philosophy of scholastic Aristotelianism in which those distinctions \textins{between substance of a thing and its accidents} were drawn, transubstantiation is as difficult, as `impossible', as it seems to any ordinary reflection. And it is right that it should be so. When we call something a mystery, we mean that we cannot iron out the difficulties about understanding it and demonstrate once for all that it is perfectly possible. Nevertheless we do not believe that contradictions and absurdities can be true, or that anything logically demonstrable from things known can be false. And so we believe that there are answers to supposed proofs of absurdity, whether or not we are clever enought to find them.]{en la filosofía de la escolástica aristotélica en la que se trazaron aquellas distinciones \textins{entre la substancia de algo y sus accidentes}, la transubstanciación resulta tan difícil, tan ``imposible'', como lo parece en la reflexión ordinaria. Y es bueno que sea así. Cuando llamamos a algo un misterio, queremos decir que no podemos solventar las dificultades en su comprensión ni demostrar de una vez por todas que es perfectamente posible. Sin embargo, no creemos que las contradicciones y los absurdos puedan ser verdaderos o que algo demostrable lógicamente a partir de lo ya conocido pueda resultar falso. Y, por tanto, creemos que hay respuestas para las supuestas pruebas de su carácter absurdo, seamos o no lo suficientemene listos para encontrarlas.}
