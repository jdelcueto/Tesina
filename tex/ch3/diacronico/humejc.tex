\subsection{Hume and Julius Caesar (1973)}

Los artículos \emph{Hume and Julius Caesar} y \emph{``Whatever has a beginning of existence must have a cause'': Hume’s Argument Exposed}, de Anscombe, fueron publicados en la revista académica \emph{Analysis} en octubre de 1973 y abril de 1974 respectivamente. Ambos están relacionados por el tema de la causalidad en Hume. En el trasfondo de los dos artículos está otro documento no publicado hasta 2011 con el título \emph{Hume on causality: introductory}. Anscombe añadió el artículo \emph{Hume and Julius Caesar} al primer volumen de sus \emph{Collected Philosophical Papers} donde, según se ha comentado antes, agrupa ensayos que consideran de diversos modos la relación entre lo concebible y lo posible.

Una de las actitudes características de Anscombe es su tendencia a quedar atraída por preguntas que representan cuestiones profundas, incluso en discusiones cuyos argumentos, método o conclusiones no le parecen tan interesantes. Con esa actitud se detiene en diversas ocasiones en las argumentaciones de Hume. El Prof. Roger Teichmann, en su libro dedicado a la filosofía de Elizabeth, describe esta tendencia en estos términos: \blockquote[{\cite[177]{teichmann2008ans}}: \enquote{Anscombe again and again found in Hume a starting point for her discussions; and we must not be misled by her frequent dissent from his views into thinking of her as `anti-Humean'. Indeed, in her treatment of the topic of causation Anscombe can even be seen as continuing Hume's work---as out-Huming Hume.}]{Anscombe una y otra vez encontró en Hume un punto de partida para sus discusiones; y no hemos de quedar engañados por su recurrente desacuerdo con sus perspectivas en pensar de ella como `anti Humeana'. Ciertamente, en su forma de tratar el tema de la causalidad Anscombe incluso puede verse como continuando el trabajo de Hume---como siendo más Humeana que Hume}. Elizabeth misma ofrece un juicio de la filosofía de Hume donde expresa su interés en los problemas estudiados por él, en \emph{Modern Moral Philosophy} dice: \blockquote[{\cite[28]{anscombe1981erp:mmph}}: \enquote{The features of Hume’s philosophy which I have mentioned, like many other features of it, would incline me to think that Hume was a mere ---brilliant--- sophist; and his procedures are certainly sophistical. But I am forced, not to reverse, but to add to this judgement by a peculiarity of Hume’s philosophizing: namely that, although he reaches his conclusions ---with which he is in love--- by sophistical methods, his considerations constantly open up very deep and important problems. It is often the case that in the act of exhibiting the sophistry one finds oneself noticing matters which deserve a lot of exploring: the obvious stands in need of investigation as a result of the points that Hume pretends to have made.}]{Las características de la filosofía de Hume que he mencionado, como muchas otras de sus características, me hacen inclinarme a pensar que Hume era un simple ---brillante--- sofista; y sus procedimientos son ciertamente sofísticos. Sin embargo me veo forzada, no a retractarme, sino a añadir a este juicio por la peculiaridad del filosofar de Hume: a saber, que aunque llega a sus conclusiones ---con las que está enamorado--- por métodos sofísticos, sus consideraciones constantemente abren problemas bien profundos e importantes. Frecuentemente es el caso que en el acto de exhibir la sofística uno se encuentra a sí mismo notando temas que merecen mucha exploración: lo obvio queda necesitado de investigación como resultado de los puntos que Hume pretende haber hecho}.

En el artículo \emph{Hume and Julius Caesar} la discusión que capta el interés de Anscombe se encuentra en la sección IV de la tercera parte del \emph{Treatise of Human Nature} sobre el tema de la justificación para nuestro creer en cuestiones que están más allá de nuestra experiencia y memoria. Anscombe cita el texto de Hume como sigue: \blockquote[{\cite[86]{anscombe1981parmenides:humeandjulius}}: \enquote{When we infer effects from causes, we must establish the existence of these causes\ldots either by an immediate perception of our memory or senses, or by an inference from other causes; which causes we must ascertain in the same manner either by a present impression, or by an inference from their causes and so on, until we arrive at some object which we see or remember. 'Tis impossible for us to carry on our inferences \emph{in infinitum}, and the only thing that can stop them, is an impression of the memory or senses, beyond which there is no room for doubt or enquiry. (Selby-Bigge's edition, pp. 82-3)}]{Cuando inferimos efectos partiendo de causas debemos establecer la existencia de estas causas\ldots ya sea por la percepción inmediata de nuestra memoria o sentidos, o por la inferencia partiendo de otras causas; causas que debemos explicar de la misma manera por una impresión presente, o por una inferencia partiendo de sus causas, y así sucesivamente hasta que lleguemos a un objeto que vemos o recordamos. Es imposible para nosotros proseguir en nuestras inferencias al infinito, y lo único que puede detenerlas es una impresión de la memoria o los sentidos más allá de la cual no existe espacio para la duda o indagación}.

Ya en la sección II de esta misma parte del \emph{Treatise}, Hume ha planteado cómo es la causalidad la conexión que nos asegura la existencia o acción de un objeto que es seguido o precedido por la existencia o acción de otro\footnote{\cite[Cf.~][53]{hume1740treatise}: \enquote{’Tis only causation, which produces such a connexion, as to give us assurance from the existence or action of one object, that ’twas follow’d or preceded by any other existence or action; nor can the other two relations be ever made use of in reasoning, except so far as they either affect or are affected by it.}}. Ahora en la sección IV esta relación de causa y efecto será tomada como un principio de asociación de ideas según el cual es posible inferir desde la impresión de alguna cosa, una idea sobre otra cosa.

Desde esta noción de causalidad se explica la posibilidad de acceder a hechos más allá de nuestra experiencia; estos son inferencias de efectos desde sus causas. De este modo: \blockquote[{\cite[87]{anscombe1981parmenides:humeandjulius}}: \enquote{For Hume, the relation of cause and effect is the one bridge by which to reach belief in matters beyond our present impressions or memories.}]{Para Hume, la relación de causa y efecto es el único puente por el que se puede alcanzar creer en cuestiones más allá de nuestras impresiones presentes o memorias}.

El paso adicional que Hume propone en esta sección es que al realizar estas inferencias es necesario establecer la existencia de las causas por medio de la percepción inmediata de los sentidos o por medio de una ulterior inferencia. Sin embargo, el establecimiento de la existencia de estas causas por medio de inferencias no puede continuar infinitamente, sino que tiene que llegar a una impresión de la memoria o los sentidos que sirva de justificación o fundamento definitivo.

Para ilustrar este paso, Hume hace una invitación interesante: \blockquote[{\cite[58]{hume1740treatise}}: \enquote{choose any point of history, and consider for what reason we either believe or reject it.}]{elegir cualquier punto en la historia, y considerar por qué razón lo creemos o rechazamos}. Acerca de una creencia histórica se nos invita a considerar sobre qué se sostiene su justificación. ¿Cuál es su fundamento?: \blockquote[{\cite[58-59]{hume1740treatise}}: \enquote{Thus we believe that Cæsar was kill’d in the senate-house on the ides of March; and that because this fact is establish’d on the unanimous testimony of historians, who agree to assign this precise time and place to that event. Here are certain characters and letters present either to our memory or senses; which characters we likewise remember to have been us’d as the signs of certain ideas; and these ideas were either in the minds of such as were immediately present at that action, and receiv’d the ideas directly from its existence; or they were deriv’d from the testimony of others, and that again from another testimony, by a visible gradation, ’till we arrive at those who were eye-witnesses and spectators of the event. ’Tis obvious all this chain of argument or connexion of causes and effects, is at first founded on those characters or letters, which are seen or remember’d, and that without the authority either of the memory or senses our whole reasoning wou’d be chimerical and without foundation.}]{Así, creemos que César fue asesinado en el Senado en los idus de Marzo; y esto porque el hecho está establecido basándose en el testimonio unánime de los historiadores, que concuerdan en asignar a este evento este tiempo y lugar precisos. Aquí ciertos caracteres y letras se hallan presentes a nuestra memoria o sentidos; caracteres que recordamos igualmente que han sido usados como signos de ciertas ideas; y estas ideas estuvieron ya en las mentes de los que se hallaron inmediatamente presentes a esta acción y que obtuvieron las ideas directamente de su existencia; o fueron derivadas del testimonio de otros, y éstas a su vez de otro testimonio, por una graduación visible, hasta llegar a los que fueron testigos oculares y espectadores del suceso. Es manifiesto que toda esta cadena de argumentos o conexión de causas y efectos se halla fundada en un principio en los caracteres o letras que son vistos o recordados y que sin la autoridad de la memoria o los sentidos nuestro razonamiento entero sería quimérico o carecería de fundamento}.

Anscombe comienza por reaccionar afirmando: \blockquote[{\cite[86]{anscombe1981parmenides:humeandjulius}}: \enquote{This is not to infer effects from causes, but rather causes from effects.}]{Esto no es inferir efectos partiendo de sus causas, sino más bien causas desde los efectos}. Es decir, el ejemplo histórico de Hume consiste en una inferencia de la causa original, el asesinato de Julio César, desde su efecto remoto que es nuestra percepción en el presente. Creemos en el asesinato de César porque lo inferimos como la causa última en una cadena causal que llega hasta nuestra percepción de ciertas oraciones que leemos. El hecho de que estemos leyendo esta información es la percepción que justifica la creencia de que hay una cadena de causas y efectos que tiene como efecto esta experiencia. Esta inferencia pasa a través de una cadena de efectos de causas, que son efectos de causas\ldots ¿Dónde empieza la cadena? La respuesta parece ser nuestra percepción presente. ¿Cómo hemos de entender, entonces, el argumento de que la cadena no puede continuar infinitamente? La propuesta de Hume es que la cadena ha de terminar en una impresión que no deje lugar a dudas o búsqueda mas allá, sin embargo, la cadena termina en el asesinato de Julio César, no en nuestra percepción. La imagen que Hume pretende ofrecer es la de una cadena fijada en sus dos extremos por algo distinto a los eslabones que la componen, sin embargo, no lo logra, más bien parece describir un voladizo, una estructura apoyada en un punto, pero sin apoyo en el otro extremo.

La afirmación \enquote*{Es imposible para nosotros proseguir en nuestras inferencias al infinito} viene a significar, según la interpretación de Anscombe, que \blockquote[{\cite[Cf.~][87]{anscombe1981parmenides:humeandjulius}}: \enquote{\emph{the justification of the grounds of our inferences cannot go on in infinitum}}]{\emph{la justificación de los fundamentos de nuestras inferencias no pueden continuar al infinito}}. El argumento aquí mas bien es que tiene que haber un punto de partida para la inferencia de la causa original. La relación de inferencias propuesta por Hume en su ilustración acabaría siendo una inferencia hipotética según su propia definición. Anscombe explica diciendo: \blockquote[{\cite[117]{anscombe2011plato:humecaus}}: \enquote{We must suppose ourselves to start with the familiar idea, merely as idea, of Caesar having been killed. Now if we ask why we believe it we shall, as Hume does, point to historical testimony (the ‘characters and letters’), which doesn’t at this point figure as what stops inference going on ad infinitum. However, if we want to explain the connection we shall form the idea of Caesar’s death being recorded by eyewitnesses; and these records having been received by others, who transmitted an account \ldots etc. Here we really are arguing from the idea of an original cause to the idea of an effect; we are ‘inferring effects from causes’, though only in the sense of passing from the idea of the cause to the idea of the effect.}]{Tendríamos que suponer que comenzamos con la idea familiar, meramente como una idea, de que César fue asesinado. Ahora si preguntamos por qué lo creemos hemos de, como hace Hume, señalar al testimonio histórico (los `caracteres y letras'), lo cual en este punto no figura como lo que detiene que la inferencia siga al infinito. Sin embargo, si queremos explicar la conexión tenemos que formular la idea de la muerte del César siendo recordada por testigos; y esos recuentos siendo recibidos por otros, quienes transmitieron un informe\ldots etc. Aquí estamos realmente razonando desde la idea de una causa original a la idea de un efecto; estamos `infiriendo efectos de causas', pero solo en el sentido de pasar de la idea de la causa a la idea del efecto}.

Teniendo estas cosas en cuenta, Anscombe distingue cuatro puntos en el argumento de Hume, él tendría que creer esto para poder establecer que la cadena de información es una cadena de inferencias via causa y efecto: \blockquote[{\cite[88]{anscombe1981parmenides:humeandjulius}}: \enquote{First, a chain of reasons for a belief must terminate in something that is believed without being founded on anything else. Second, the ultimate belief must be of a quite different character from derived beliefs: it must be perceptual belief, belief in something perceived, or presently remembered. Third, the immediate justification for a belief $p$, if the belief is not a perception, will be another belief $q$, which follows from, just as much as it implies, $p$. Fourth, we believe by inference through the links in a chain of record.   There is an implicit corollary: when we believe in historical information belonging to the remote past, we believe that there has been a chain of record.}]{Primero, una cadena de razones para una creencia debe terminar en algo que se cree sin estar fundado en alguna otra cosa. Segundo, la creencia última debe ser de una naturaleza distinta a las creencias derivadas: Tiene que ser creencia perceptual, creer en algo percibido, o recordado en el presente. Tercero, la justificación inmediata de una creencia $p$, si la creencia no es una percepción, será otra creencia $q$, la cual se sigue, en la misma medida que implica, a $p$. Cuarto, creemos por inferencia a través de los eslabones en una cadena de relato. Hay un corolario implícito: cuando creemos en información histórica perteneciente a un pasado remoto, creemos que ha habido una cadena de relato}.

Sin embargo, Anscombe considera que este no es el orden en que quedan fundadas nuestras creencias, sino que más bien: \blockquote[{\cite[88]{anscombe1981parmenides:humeandjulius}}: \enquote{\emph{If} the written records that we now see are grounds of our belief, they are first and foremost grounds for belief in Caesar's killing, belief that the assassination is a solid bit of history. Then our belief in that original event is a ground for belief in much of the intermediate transmision.}]{\emph{Si} los relatos escritos que vemos ahora son fundamento para nuestro creer, estos son primero y ante todo fundamento para la creencia en el asesinato de Cesar, creencia en que el asesinato es un pedazo sólido de historia. Entonces nuestra creencia en ese evento original es fundamento para el creer en mucha de la transmisión intermedia}. ¿Por qué creemos que hubo testigos del asesinato? Ciertamente porque creemos que hubo un asesinato. La creencia de que hubo testigos es inferida de la creencia en el hecho.

Anscombe compara este modo de entender la cadena de transmisión de información histórica a nuestra creencia en la continuidad espacio-temporal. Si reconocemos en una ocasión a una persona conocida como alguien que vimos la semana pasada, nuestra creencia en que es la misma persona no es una inferencia de otra creencia acerca de la continuidad espacio-temporal de un patrón humano desde ahora hasta entonces, sino que más bien nuestra creencia en la continuidad espacio-temporal está inferida del reconocimiento de la identidad de la persona. Sin embargo, una evidencia sobre una interrupción en la continuidad sí alteraría nuestra creencia en la identidad.

Elizabeth entonces concluye que: \blockquote[{\cite[89]{anscombe1981parmenides:humeandjulius}}: \enquote{Belief in recorded history is on the whole a belief that there has been a chain of tradition of reports and records going back to contemporary knowledge; it is not a belief in the historical facts by an inference that passes through the links of such a chain. At most, that can very seldom be the case.}]{La creencia en los registros de la historia consiste en general en la creencia de que ha habido una cadena de tradición de informes y registros que van hacia el conocimiento contemporáneo; no es una creencia en hechos históricos por una inferencia que pasa por los eslabones de una cadena como esta. Como mucho, esto sería muy raramente el caso}.

Ahora bien, como se ha dicho antes, el interés de Anscombe no está simplemente en mostrar en qué se equivoca Hume, sino que considera que la cuestión toca el nervio de un problema con cierta profundidad: \blockquote[{\cite[122]{anscombe2011plato:humecaus}}: \enquote{The interesting problem that arises, then, is why the things we are told and the writings that we see \emph{are} the starting points for our belief in the far distant events and so in the intermediate chain of record.}]{El problema interesante que surge, entonces, es por qué las cosas que se nos dicen y los escritos que vemos \emph{son} puntos de partida para nuestro creer en eventos distantes y así también en la cadena del relato intermedia}.

Ahora Anscombe propone una noción que evolucionó en la mente de Wittgenstein y a la que debe mucho en su propia argumentación. Según como aparece en \emph{Investigaciones Filosóficas} es, en opinión de Anscombe, una de \enquote{las raras piezas de estupidez en los escritos de Wittgenstein} se encuentra en el \S56: \blockquote[{\cite[89]{anscombe1981parmenides:humeandjulius}}: \enquote{That it is thinkable that we may find Caesar's body hangs directly together with the sense of a propoisiton about Caesar. But so too does the possibility of finding something written, from which it emerges that no such man ever lived, and his existence was made up for particular ends.}]{Que es concebible que podamos encontrar todavía el cuerpo de César está sujeto directamente junto al sentido cualquier proposición acerca de César. Pero también lo está la posibilidad de encontrar algo escrito, desde lo cual surja que tal hombre no vivió nunca, y su existencia fue inventada para fines particulares}. Elizabeth se cuestiona \enquote{¿Qué documento o inscripción podría ser evidencia de que Julio César nunca existió?}. Wittgenstein cambia su manera de pensar sobre esto; en una época más tardía de su pensamiento, él mismo cuestionaría la posibilidad de una evidencia que probara que Julio César no existió preguntando: \enquote*{¿qué quedaría juzgado por qué aquí?}.

El modo más tardío del pensamiento de Wittgenstein al que Anscombe hace aquí referencia es el que se encuentra en \emph{Sobre la Certeza}. La motivación para este escrito de Wittgenstein son las propuestas de Moore en \emph{Proof of the External World} y \emph{Defence of Common Sense}. En estas obras Moore sostiene que hay una serie de proposiciones que conocemos con seguridad, como \enquote*{Aquí hay una mano, y aquí otra}, o \enquote*{La tierra ha existido por largo tiempo antes de mi nacimiento} y \enquote*{Nunca he estado lejos de la superficie de la tierra}. Estas reflexiones ocuparon a Wittgenstein durante los últimos años de su vida\footnote{\cite[Cf.~][vi]{wittgenstein1969oncertes}: \enquote{Hacia la mitad de 1949, visitó los Estados Unidos por invitación de Norman Malcolm, residiendo en la casa de éste en Ithaca. Malcolm reavivó su interés por la ``defence of common sense'' de Moore. Es decir, por la pretensión de \emph{saber} con seguridad que una serie de proposiciones son verdaderas, por ejemplo: ``Aquí hay una mano y aquí hay otra''}}. Un tema que aparece en esta discusión de Wittgenstein es que la justificación semántica, relacionada con el uso correcto del lenguaje, y la justificación epistémica, relacionada como tal con el afirmar la verdad, están más unidas entre sí de lo que se piensa. Según esto:\blockquote[{\cite[213]{teichmann2008ans}}: \enquote{Wittgenstein invites us to view the rules governing the correct use of words as comparable to the rules governing the acceptance or rejection of beliefs (which are themselves of course paradigmatically expressed in words); a ‘world view’ is determined as much by our language and its attendant conceptual scheme as by what we would ordinarily term our knowledge of things. The two aspects of world view, the two kinds of justification, come together in the phenomenon of certainty. \textelp{} One direction in which these thoughts seem to take us is towards regarding certain world views, or sets of beliefs, or very general beliefs, as no more susceptible of rational justification or criticism than are concepts.}]{Wittgenstein nos invita a ver las reglas que gobiernan el uso correcto de las palabras como comparables con las reglas que gobiernan la aceptación o rechazo de las creencias (que desde luego son ellas mismas paradigmáticamente expresadas en palabras); una `cosmovisión' está determinada tanto por nuestro lenguaje y su esquema conceptual relacionado como por lo que ordinariamente expresamos como nuestro conocimiento de las cosas. Los dos aspectos de la cosmovisión, los dos tipos de justificación, quedan unidos en el fenómeno de la certeza. \textelp{} Una dirección hacia la que estos pensamientos parecen dirigirnos es a considerar ciertas cosmovisiones, o colecciones de creencias, o creencias generales, como no más susceptibles de justificación racional o crítica que la que tienen los conceptos}. Dicho en términos simples una afirmación como \enquote*{aquí hay una mano} presentada en medio de una discusión, no viene a ser una declaración acerca de cómo es el mundo o cómo es la realidad de hecho, sino que la proposición sirve más bien para establecer una regla para la discusión. Si no puede haber un acuerdo de que esta proposición es cierta, la discusión no es posible. El acuerdo permite hablar de la realidad en términos significativos. Una actitud escéptica ante una proposición como esta resta valor a los fundamentos y solo genera parálisis.

El desarrollo de la discusión de Anscombe sigue esta línea de pensamiento. Tiene en su objetivo cómo lo que se nos dice o lo que leemos lo tenemos como fundamento para justificar creencias que juzgamos como conocimiento cierto.

\blockquote[{\cite[90]{anscombe1981parmenides:humeandjulius}}: \enquote{We know about Caesar from the testimony of ancient historians, we even have his own writings! And how do you know \emph{that} those are ancient historians, and these, works of Caesar? You were told it. And how did your teachers know? They were told it. We know it from being taught; not just from explicit teaching, but by its being implicit in a lot else that we are taught explicitly. But it is very difficult to characterize the peculiar solidity involved, or its limits.}]{Conocemos de César por el testimonio de los historiadores antiguos, ¡hasta tenemos sus propios escritos! Y ¿cómo sabes \emph{eso}, que esos son historiadores antiguos, y estos, escritos de César? Te lo dijeron. Y ¿cómo lo supieron tus maestros? Se lo dijeron a ellos. Conocemos de esto por que se nos ha enseñado; no solamente por medio de la lección explícita, sino por su presencia implícita en muchas otras cosas que se nos enseñan explícitamente. Sin embargo es muy difícil caracterizar la solidez peculiar involucrada con esto, o sus límites}. Aquí está el punto principal de la preocupación de Elizabeth. Esa característica solidez que presenta la certeza que puede justificarse para una enseñanza que forma parte del conocimiento común de nuestra cultura, y cuál pueden ser sus límites. Anscombe destaca que no es casualidad que Hume elija esta ilustración para su argumento. Ha escogido este punto histórico porque es un conocimiento presente en su cultura con un grado particular de certeza. Podría haber sometido a prueba algún detalle del suceso y cuestionar, por ejemplo, si podría dudarse la fecha o el lugar del asesinato. Y sin embargo al poner en duda un conocimiento como este, y afirmar que lo que puede servir como justificación para creerlo como cierto sólo puede ser esa propuesta cadena de inferencias, ha invitado a cuestionarse qué es lo que verdaderamente sirve de fundamento a un conocimiento como este, y adicionalmente, qué consecuencias tiene ponerlo en duda. Para Anscombe poner en duda que ese hombre, César, existió, y su vida terminó en un asesinato, sólo es posible si \blockquote[{\cite[Cf.~][90]{anscombe1981parmenides:humeandjulius}}: \enquote{by indulging in Cartesian doubt}]{nos permitimos el lujo de la duda cartesiana}.

Efectivamente, dudar de una creencia tan presente en nuestra cosmovisión, en nuestro conocimiento común, como la existencia de Julio César nos deja atrapados en una situación en la que no tenemos fundamento para afirmar otra proposición histórica. Es decir, si nos planteáramos la hipótesis de que Julio César nunca existió, nos situaríamos entre dos alternativas, ya sea el enredo de la confusión: \blockquote[{\cite[91]{anscombe1981parmenides:humeandjulius}}: \enquote{\textelp{} say: ``How could one explain all these references and implications, then?\ldots but, but, \emph{but} if I doubt the existence of Caesar, if I say I may reasonably call it in question, then with equal reason I must doubt the status of the things I've just pointed to''}]{\textelp{} decir ``¿Cómo se explican todas estas referencias e implicaciones entonces?\ldots pero, pero \emph{pero} si dudo de la existencia de César, si digo que podría razonablemente ponerlo en tela de juicio, entonces, con la misma razonabilidad tengo que dudar de la validez de las cosas que acabo de señalar''}. O, por otra parte, la conciencia del callejón sin salida por dónde nos hemos metido: \blockquote[{\cite[91]{anscombe1981parmenides:humeandjulius}}: \enquote{\textelp{} I should realize straight away that the `doubt' put me in a vacuum in which I could not produce reasons why such and such `historical facts' are more or less doubtful.}]{\textelp{} podría caer en cuenta inmediatamente de que la `duda' me ha encerrado en un vacío en el cual no podría producir razones por las cuales estos u otros `datos históricos' serían más o menos dudosos}.

Lo que se pierde de vista cuando se pone en duda un conocimiento como este es qué puede ser tenido como evidencia para justificar la certeza de cualquier conocimiento de la misma naturaleza del que se ha negado. En ese sentido, hay creencias que forman parte del conocimiento común que no pueden ser negadas sin más, sino que forman parte de los fundamentos de la cosmovisión dentro de la cuál se está discutiendo. Nuestro conocimiento está dentro de una cosmovisión y esta cosmovisión tiene coordenadas fijas que se van desarrollando, pero no pueden ser intercambiadas por nociones completamente distintas.

Esto lo ilustra Elizabeth en su conclusión recurriendo a la analogía hecha por Otto Neurath en \emph{Anti-Spengler}, donde compara el conocimiento científico con un barco en el cual los que investigan son como marinos que reconstruyen el barco en alta mar, verificando y reemplazando sus piezas mientras que se navega. Entonces propone que si la ilustración implica que se puede ir examinando cada pieza y reemplazarla de tal modo que se termina con un barco distinto, la analogía no sirve: \blockquote[{\cite[92]{anscombe1981parmenides:humeandjulius}}: \enquote{For there are things that are on a level. A general epistemological reason for doubting one will be a reason for doubting all, and then none of them would have anything to test it by.}]{Pues hay cosas que están fijas. Una razón epistemológica general para dudar de una será razón para dudar de todas, y entonces ninguna tendría criterio alguno que sirviera para evaluarla}.
