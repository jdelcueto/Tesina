\subsection{The Question of Linguistic Idealism (1976)}

\emph{The Question of Linguistic Idealism} fue publicado en 1976 en \emph{Acta Philosophica Fennica} junto a otros ensayos sobre Wittgenstein en honor de G.\,H.\,von Wright. Georg Henrik von Wright fue sucesor de Wittgenstein en la cátedra de filosofía en Cambrdige entre 1948--1951, puesto que Anscombe ocuparía en 1970; también fue con Elizabeth uno de los responsables del legado literario de Wittgenstein.

Como fue anticipado en la discusión del artículo \emph{Parmenides, Mystery and Contradiction} este ensayo sirve como conclusión al primer volumen de los \emph{Collected Philosophical Papers} dedicados a distintas reflexiones en torno a la relación entre la realidad, el pensamiento y el lenguaje. En aquel artículo la tradición subyacente al \emph{Tractatus} fue examinada por Anscombe desde la perspectiva de \emph{Investigaciones Filosóficas}. Aquí Elizabeth examina esta segunda etapa del pensamiento de Wittgenstein y se pregunta si logra aquella difícil empresa planteada por Ludwig: \blockquote[{\cite[112]{wittgenstein1956remmath}}: \enquote{Not empiricsm and yet realism in philosophy, that is the hardest thing}]{Realismo en la filosofía sin caer en empirismo, eso es lo más complicado}. El artículo está dividido en dos partes, la primera dedicada al aspecto semántico del tema, derivado de la expresión \enquote*{la esencia es expresada en la gramática}; la segunda se enfoca más en los aspectos epistemológicos de la cuestión según aparecen en la discusión de \emph{On Certainty}.\footnote{\cite[Cf.~][215]{teichmann2008ans}: \enquote{The essay is in two parts, these correspond roughly to the semantic and epistemological aspects of the topic.}}

El modo en que Elizabeth enmarca la investigación recuerda sus palabras en la introducción de esta colección: \blockquote[{\cite[xi]{anscombe1981parmenides}}: \enquote{At the present day we are often perplexed with enquiries about what makes true, or what something's being thus or so \emph{consists in}; and the answer to this is thought to be an explanation of meaning. If there is no external answer, we are apparently committed to a kind of idealism.}]{En la época actual con frecuencia nos quedamos perplejos con preguntas sobre qué hace a algo verdadero, o \emph{en qué consiste} el que algo sea de un modo u otro; y la respuesta a esto se piensa que es una explicación del significado. Si no hay una respuesta externa, aparentemente estamos comprometidos con un tipo de idealismo.} En \emph{Investigaciones Filosóficas} la relación entre la realidad y el pensamiento se plantea como una relación interna. Anscombe se pregunta sobre la posibilidad de que se encuentre en esta etapa del pensamiento de Wittgenstein un planteamiento idealista. Toma como punto de partida el siguiente pasaje: \blockquote[{\cite[112]{anscombe1981parmenides:qli}}: \enquote{``If anyone believes that certain concepts are absolutely the right ones, and that having different concepts would mean not realizing something that we realize\,---\,then let him imagine certain very general facts of nature to be different from what we are used to, and the formation of  concepts different from usual ones will become intelligible to him'' (Philosophical Investigations \textins{PI}, II, XII).}]{``Si alguna persona cree que ciertos conceptos son absolutamente los correctos, y que tener otros conceptos significaría que no se apreciaría algo de lo que nosotros apreciamos\,---\,entonces que imagine ciertos hechos muy generales de la naturaleza como siendo distintos de lo que estamos acostumbrados, y la formación de conceptos distintos de los usuales se le harán inteligibles'' (Investigaciones Filosóficas \textins{IF}, II, XII).} Entonces plantea: \blockquote[{\cite[112]{anscombe1981parmenides:qli}}: \enquote{This is one of the passages from Wittgenstein arousing ---in my mind at least--- the question: have we in his last philosophical thought what migth be called linguistic idealism? Linguistic, because he describes concepts in terms of linguistic practices. And he also wrote: ``\emph{Essence} is expressed by grammar'' (PI, I, \S371).}]{Este es uno de los pasajes de Wittgenstein que despierta ---en mi mente al menos--- la pregunta: ¿tenemos en su pensamiento filosófico tardío lo que podríamos llamar idealismo linguístico? Linguistico, porque describe los conceptos en terminos de prácticas linguísticas. Y también escribió: ``La \emph{esencia} es expresada por la gramática'' (IF, I, \S371).} El término `gramática' es un concepto presente en la filosofía más tardía de Wittgenstein que consiste en \blockquote[{\cite[215]{teichmann2008ans}}: \enquote{linguistic or conceptual necessities that relate to particular concepts or concept-groups}]{las necesidades linguísticas o conceptuales que están realcionadas con conceptos particulares o grupos de conceptos}. Esas necesidades no se corresponden con algún lenguaje formal específico, sino que por ejemplo: \blockquote[{\cite[215]{teichmann2008ans}}: \enquote{`there is a crude grammar common to all \textins{words in different languages for \emph{horse}}, by which each is e.g. a count-noun which is the name of a kind of whole living thing'}]{`hay una gramática en bruto común a todas las \textins{palabras en diferentes lenguajes para \emph{caballo}}, por la cual este es p. ej. un sustantivo-enumerador que es el nombre de un tipo de totalidad de ser viviente'}.

Ahora bien, si estamos de acuerdo con que la esencia está expresada en la gramática entonces tendríamos que decir que las palabras que usamos para hablar de algo tienen que tener una gramática específica. Pero, ¿qué significa esto? ¿Sería lo mismo que decir \enquote*{la esencia es creada por la gramática}? Esta propiedad gramática que se adscribe a estas expresiones ¿es propia del objeto del que la expresión habla, o del lenguaje?

Habría que decir que la propiedad es del lenguaje, y por tanto no caracteriza al objeto sino al lenguaje, es decir, si esta expresión no tiene esta propiedad, esta gramática, deja de ser lenguaje acerca de este objeto. En este sentido la gramática \emph{corresponde} con la esencia del objeto y el objeto mismo es independiente del lenguaje. De acuerdo con esto, Anscombe distingue que, efectivamente, la esencia es expresada por la gramática y no creada por ella, y así si imagináramos otro lenguaje distinto con otra gramática y otros conceptos y también personas que usaran este otro lenguaje, estas personas, en efecto, no estarían usando un lenguaje cuya gramática expresara las mismas esencias que nosotros; sin embargo, este lenguaje diferente con otros conceptos no determinaría necesariamente que estas personas no serían capaces de apreciar en la realidad cosas que nosotros somos capaces de apreciar.\footnote{\cite[Cf.~][115]{anscombe1981parmenides:qli}: \enquote{Essence is expressed by grammar. But we can conceive of different concepts, i.e. of language without the same grammar. People using this would then not be using language whose grammar expressed the same essences. However, they might not thereby be missing anything that we realize.}}

Anscombe nota: \blockquote[{\cite[115]{anscombe1981parmenides:qli}}: \enquote{It is enormously difficult to steer in the narrow channel here: to avoid the falsehoods of idealism and the stupidities of empiricist realism.}]{Es enormemente difícil conducirse en el canal estrecho aquí: evitar las falsedades del idealismo y las necedades del realismo empírico.} y propone llanamente: \blockquote[{\cite[116]{anscombe1981parmenides:qli}}: \enquote{if we want to know wether Wittgenstein is a `linguistic idealist'. We shall ask the question: Does this existence, or this truth, depend upon human linguistic practice? That the \emph{meaning of expressions} is so dependent is evident; that human possesion of concepts is so dependent is not quite so evident.}]{si queremos saber si Wittgenstein es un `idealista linguistico'. Hemos de hacer la pregunta: ¿Acaso esta existencia, o esta verdad, depende de la práctica linguística humana? Que el \emph{significado de las expresiones} es de este modo dependiente es evidente; que la posesión humana de conceptos es de tal manera dependiente no es tan evidente.}

El uso competente del lenguaje es un criterio para la posesión de los conceptos simbolizados en él, en este sentido, tener ciertas prácticas linguísticas específicas implica tener ciertos conceptos específicos. Por `uso competente del lenguaje' o por `práctica linguística humana' no hemos de pensar simplemente en producir palabras ordenadas de tal manera que componen una oración pertinente, sino que hemos de entender todas aquellas actividades dentro de las cuales el uso del lenguaje está entretejido: medir, pesar, dar y recibir, situar en algún lugar correspondiente, realizar movimientos de maneras particulares, y también actuar según la consulta de tablas, calendarios o signos.\footnote{\cite[Cf.~][117]{anscombe1981parmenides:qli}: \enquote{The competent use of language is \emph{a} criterion for the possession of the concepts symbolized in it, and so we are at liberty to say: to have such-and-such linguistic practices is to have such-and-such concepts. ``Linguistic practice'' here does not mean merely the production of words properly arranged into sentences on occasions which we vaguely call ``suitable''. It is important that it includes activities \emph{other} than the production of language, into which a use of language is interwoven. For example, activities of measuring, of weighing, of giving and receiving and putting into special places, of moving about in a huge variety of ways, of consulting tables and calendars and signs and acting in a way which is connected with that consultation.}} Este complejo compuesto por actividad y lenguaje en un contexto específico es lo que Wittgenstein llama `juego de lenguaje'.\footnote{\cite[Cf.~][62]{bakerhacker2009understanding}: \enquote{`language-game' refers to the complex consisting of activity and language use.}}

Al examinar el pasaje de \emph{Investigaciones Filosóficas} citado al principio, Anscombe ha establecido que la práctica linguística está relacionada con la existencia de ciertos conceptos, pero que de esto no se sigue que las realidades que son expresadas por estos conceptos dependen en modo alguno del pensamiento o lenguaje humanos. Desde un punto de vista semántico, la postura idealista consistiría en \enquote*{la esencia es creada por la gramática}, y esta idea ha sido rechazada, diciendo más bien que la gramática expresa la esencia o se corresponde con ella. Sin embargo, ¿podría haber lo que podríamos llamar un idealismo parcial? Con esta pregunta, Elizabeth dirige su atención a la lógica como el orden según el cuál los conceptos son empleados. ¿Está determinado por la práctica linguística? ¿Se podría decir que la \enquote*{esencia es creada por la gramática} en el sentido de que las necesidades lógicas y conceptuales dependen de la práctica linguística humana?\footnote{\cite[Cf.~][220]{teichmann2008ans}: \enquote{there is a lesser mode of applying the notion of `dependence' through and through: one by which logical and conceptual necessities are made out to depend on practices that are esentially \emph{arbitrary}. If essence is created by grammar, and if grammar is not only autonomous but arbitrary, then the cloud that is metaphysics has apparently been condensed into a droplet of no significance, and the overview of our language which philosophy brings us is an overview of random human action.}} Anscombe cita a Kronecker que dice: \enquote{Dios hizo los números enteros, lo demás es construcción humana}, ¿a qué se refiere? Parece sugerir que hay una parte del orden lógico que es dado por la naturaleza, y otra que es invención humana. ¿Cómo se puede describir esto?

Anscombe ha establecido que la existencia de los objetos expresados por la gramatica no dependen de la práctica linguística, pero propone que hay un cierto tipo de necesidad que sí está establecido por esta práctica: \blockquote[{\cite[118]{anscombe1981parmenides:qli}}: \enquote{But there are, of course, a great many things whose existence does depend on human linguistic practice. The dependence is in many cases an unproblematic and trivial fact. But in others it is not trivial\,---\,it touches the nerve of great philosophical problems. The cases I have in mind are three: namely rules, rights and promises.}]{hay, desde luego, una gran cantidad de cosas cuya existencia sí depende de la práctica linguística humana. La dependencia es en muchos casos un dato no problemático y trivial. Pero en otros no es trivial\,---\,sino que toca el nervio de grandes problemas filosóficos. Los casos que tengo en mente son tres: a saber, reglas, derechos y promesas.} Estos tres casos tienen asociados con ellos el uso de nociones modales, es decir hay un `tener que' relacionado con ellos: de acuerdo a las \emph{reglas} de un juego o procedimiento hay ciertas acciones que tienen que ser hechas y otras que no deben hacerse, cuando alguien tiene el \emph{derecho} de hacer algo no se le puede detener, si se ha establecido un \emph{contrato} se debe de cumplir esto o no se debe hacer algo en contra de ello. Es posible pensar en distintas prácticas que son definidas por estas reglas y que no representan ninguna dificultad, sin embargo ¿qué ocurre en el caso de las reglas de la lógica? ¿Dependen de la práctica linguística?

Si alguien cambia las reglas de un juego, o de un baile, se diría que ha construido una variante, \enquote*{esto ya no es ajedrez, sino otro juego}. ¿Se puede decir lo mismo de la lógica? La actividad que la lógica tiene como objetivo es la inferencia válida. ¿Se pueden construir variantes de inferencias validas usando otras reglas? Para responder a esto hay que pensar en estas reglas como siendo puestas en práctica, entonces, ¿de acuerdo a qué reglas se hace esta deducción para su aplicación, esta transición desde reglas dadas a prácticas particulares? Anscombe explica que: \blockquote[{\cite[121]{anscombe1981parmenides:qli}}: \enquote{Always there is the logical \emph{must}: you can't have this \emph{and} that; you can't do that if you are going by this rule; you must grant this in face of that. And just as ``You can't move your king'' is the more basic expression for one learning chess, since it lies at the bottom of his learning the concept of the game and its rules, so these ``You must's'' and ``You cant's'' are the more basic expressions in logical thinking. But they are not what Hume calls ``naturally intelligible''\,---\,that is to say, they are not expressions of perception or experience. They are understood by those of normal intelligence as they are trained in the practices of reasoning.}]{Siempre está ahí el \emph{tener que} lógico: no puedes tener esto \emph{y} aquello; no puedes hacer eso si estás siguiendo esta regla; tienes que conceder esto teniendo en cuenta esto otro. Y así como ``No puedes mover tu rey'' es la expresión más básica para alguien que está aprendiendo ajedrez, puesto que está en el fondo de su aprendizaje del concepto del juego y sus reglas, así estos ``Tienes que'' y ``No puedes'' son las expresiones más basicas en el pensamiento lógico. Pero estas no son lo que Hume llama ``naturalmente inteligible''\,---\,es decir, estas no son expresiones de percepción o experiencia. Son entendidas por aquellos de inteligencia ordinaria al ser adiestrados en las prácticas de razonar.} Al considerar la realización de una inferencia válida como una práctica según una regla, la posibilidad de generar una variante queda limitada por estos 'Tienes que' que rigen la práctica de la inferencia. Sin embargo \blockquote[{\cite[219]{teichmann2008ans}}: \enquote{A justification for a `You must' will not come from outside the practice, but from within it. Anscombe takes it that for Wittgenstein, conceptual and logical necessity are both expressed by means of this `You must'}]{La justificación para un `Tienes que' no vendrá desde fuera de la práctica, sino desde dentro de ella. Anscombe entiende que para Wittgenstein, la necesidad lógica y conceptual quedan ambas expresadas por medio de este `Tienes que'}. Esto aún representa otro problema: \blockquote[{\cite[220]{teichmann2008ans}}: \enquote{if the rules of linguistic practice cannot be justified from without, and rest ultimately on the brute fact that human beings learn to respond to `You must' in a way that produces agreement in response, then surely \textins{that logical necessities depend on practices arbitrarias} is the true picture?}]{si las reglas de la práctica linguística no pueden ser justificadas externamente, y se fundan en definitiva en el hecho en bruto de que los seres humanos aprenden a responder al `Tienes que' con respuestas que establecen un acuerdo en el modo de responder, entonces ¿sin duda \textins{el que las necesidades lógicas dependen en prácticas arbitrarias} es el panorama verdadero?}

Entonces ---todavía--- enquote{¿Es esta verdad, esta existencia, el producto de la práctica linguistica humana?}. Anscombe ha dado ya una respuesta parcial a su pregunta; en el caso de las realidades que quedan expresadas en el uso del lenguaje, conceptos como un caballo, los colores o las figuras, estos no son producto de la práctica linguística; ni de hecho, ni en la filosofía de Wittgenstein. Y entonces ¿qué de las necesidades conceptuales y lógicas que pertenecen a la naturaleza de estas cosas? ¿Dependen estas necesidades metafísicas de la práctica linguística según la filosofía de Wittgenstein?

Con estas consideraciones Anscombe esta buscando ser precisa al describir la actitud de Ludwig hacia la lógica. Parece que para Wittgenstein las necesidades metafísicas dependen de las reglas gramáticas que ordenan la práctica linguística. En \emph{Investigaciones Filosóficas} \S372 hace referencia a una noción delinieada en el \emph{Tractatus}: que el correlato en el lenguaje de las necesidades de la naturaleza, es decir, de las posibilidades determinadas al objeto por su naturaleza, son las arbitrarias reglas de la gramática.\footnote{\cite[Cf.~][121]{anscombe1981parmenides:qli}: \enquote{``Is this truth, this existence, the product of human linguistic practice?'' This was my test question. I should perhaps have divided it up: Is it so actually? Is it so according to Wittgenstein's philosophy? Now we have partial answers. Horses and giraffes, colours and shapes\,---\,the existence of these is not such a product, either in fact or in Wittgenstein. But the metaphysical necessities belonging to the nature of such things\,---\,these \emph{seem} to be regarded by him as `grammatical rules'. ``Consider `The only correlate in language to a necessity of nature is an arbitrary rule. It is the only thing one can milk out of a necessity of nature into a proposition'''}} Las llama arbitrarias pues el hecho de que sean estas y no otras no responde a ninguna realidad específica. Anscombe interpreta que el modo en que Ludwig invita a considerar esta noción sugiere que es una propuesta en la que ve algo de correcto, pero de la que no está convencido. ¿Se podría sostener esto de manera general? ¿Son verdaderamente arbitrarias? En casos particulares Wittgenstein da la impresión de sotener que algo que aparece como una necesidad metafísica es una proposición gramatical. ¿Es arbitraria la gramática?\footnote{\cite[Cf.~][122]{anscombe1981parmenides:qli}: \enquote{He always seemed to say in particular cases that something that appears as a metaphysical necessity is a proposition of grammar. Is grammar `arbitrary'?}}

En cualquier caso, Wittgenstein sin duda afirma que la necesidad de hacer \emph{esto} específico si es que vamos a actuar según \emph{esta} regla específica depende de la práctica linguística. Y, de nuevo, esta práctica no se reduce a construir oraciones que expresen pensamientos en situaciones pertinentes, sino que: \blockquote[{\cite[131]{anscombe1981parmenides:qli}}: \enquote{It refers e.g. to \emph{action} on the rule; actually going \emph{this} way by the signpost. The signpost or any directive arrow may be interpreted by some new rule. When I see an arrow at an airport pointing vertically upwards, I mentally `reinterpret' this, and might put my interpretation in the form of another arrow, horizontal and pointing in the direction I am looking in when I see the first. But the arrows and their interpretations await action: what one actually does, which is counted as what was meant: \emph{that} is what fixes the meaning: And so it is about following the rules of correct reasoning. One draws the conclusion as one `must'. That is what ``thinking'' means (RFM, I, 131).}]{Se refiere por ejemplo a la \emph{acción} de acuerdo a la regla; a ir de hecho de \emph{esta} manera según el signo indicador. El signo o cualquier flecha señalando dirección puede ser interpretada según una ulterior regla. Cuando veo una flecha en el aeropuerto apuntando verticalmente hacia arriba, mentalmente `reinterpreto' esto, y puedo poner mi interpretación en la forma de otra flecha, horizontal y apuntando en la dirección que estoy mirando cuando veo la flecha original. Pero las flechas y sus interpretaciones esperan acción: lo que hacemos finalmente, que es lo que cuenta como lo que se quiso significar: \emph{esto} es lo que fija el significado: Y entonces consiste en seguir las reglas del razonamiento correcto. Sacamos la conclusión como `debemos'. Esto es lo que ``pensar'' significa (RFM, I, 131).} Anscombe ve en esta descripción la posibilidad de un idealismo linguístico si la dependencia de las reglas en la práctica se entiende de este modo: \blockquote[{\cite[131]{anscombe1981parmenides:qli}}: \enquote{Rules, with their interpretations, cannot finally dictate how you go, can't tell you what is the next step in applying them \textelp{} In the end you take the rule \emph{this} way, not in the sense of an interpretation, but by acting, by taking the step. Rules and the particular rule are defined by practice: a rule doesn't tell you how you `must' apply it; interpretations, like reasons, give out in the end.}]{Las reglas, con sus interpretaciones, no pueden dictar definitivamente cómo vamos según ellas, no pueden decirnos cuál es el próximo paso en aplicarlas \textelp{} Al final decidimos tomar la regla de \emph{esta} manera, no en el sentido de una interpretación, sino actuando, dando el paso. Las reglas en general y la regla en particular quedan definidas por la práctica: una regla no nos dice cómo `debemos' aplicarla; las interpretaciones, como las razones, se agotan al final.} Parece que la aplicación de las reglas nos dejan con tal incertidumbre que sería posible concebir que su interpretación es dudosa y puede ser cuestionada. En otro lugar Wittgenstein plantea: \blockquote[{\cite[I, \S135--136]{wittgenstein1956remmath}}: \enquote{Imagine the following queer possibility: we have always gone wrong up to now in multiplying $12\times12$. True, it is unintelligible how this can have happened, but it has happened. So everything worked out this way is wrong! ------ But what does it matter? It does not matter at all! \textelp{} But then, is it impossible for me to have gone wrong in my calculation? And what if a devil deceives me, so that I keep on overlooking something however often I go over the sum step by step? So that if I were to awake from the enchantment I should say: ``Why, was I blind?'' --- But what difference does it make for me to `assume' this? I might say: ``Yes to be sure, the calculation is wrong --- but that is how I calculate. And this is what I now call adding, and this `the sum of these two numbers'.''}]{Imagina la siguiente extraña posibilidad: hasta ahora siempre hemos actuado equivocadamente al multiplicar $12\times12$. Cierto, es incomprensible cómo puede haber ocurrido esto, pero ha ocurrido. Entonces ¡todo lo que se ha calculado de esta manera está equivocado! ------ Pero ¿que importancia tiene? ¡No importa para nada! \textelp{} Pero entonces, ¿es imposible que haya actuado equivocadamente en mi cálculo? ¿Y si un genio maligno me engaña, de modo que sigo pasando algo por alto cada vez que voy sobre la suma paso por paso? De tal manera que si despertara del hechizo tendría que decir: ``Pero, ¿estaba ciego?'' --- Pero ¿qué diferencia hace que `asuma' esto? Podría decir: ``Sí, desde luego, el cálculo está equivocado --- pero así es como yo hago ese cálculo. Y esto es lo que ahora llamo adición, y esto `la suma de estos dos números'.''} La consideración peculiar podría usarse para sostener que es concebible dudar de nuestras acciónes según las reglas, incluso en casos familiares como un cálculo aritmético. Y ante esto habria que cuestionarse, ¿esta duda concebible es un conflicto real? Imaginar la posibilidad de que podemos estar siendo engañados en nuestra aplicación de una regla ¿estaría en servicio de tratarlas con mayor rigor? Por otra parte la manera de responder a esta incertidumbre parece insistir en la dependencia del uso de estas reglas al arbitrio humano.

¿Es así como debe ser interpretada la perspectiva de Wittgenstein? En \emph{Investigaciones Filosóficas} \S520 Ludwig sondea la dependencia de la posibilidad lógica en la gramática y la consecuente arbitrariedad que entonces parece pertenecer a lo que puede ser considerado como lógicamente posible:
\blockquote[{\cite[\S520]{wittgenstein1953phiinv}}: \enquote{``Even if one conceives of a proposition as a picture of a possible state of affairs, and says that it shows the possibility of the state of affairs, still, the most that a proposition can do is what a painting or relief or film does; and so it can, at any rate, not present what is not the case. So does what is, and what is not, called (logically) possible depend wholly on our grammar --- that is, on what it permits?''}]{``Incluso si concebimos una proposición como una imagen de una posible situación de hecho, y decimos que muestra la posibilidades del estado de las cosas, aún así, lo más que una proposición puede hacer es lo mismo que haría una pintura o un relieve o un filme; y por tanto no podría, en cualquier caso, representar eso que no está siendo de hecho. Entonces ¿lo que es, y lo que no es, considerado (lógicamente) posible depende completamente en nuestra gramática?''} Wittgenstein establece que una proposición tiene la capacidad de representar una situación de hecho \emph{posible}. Se cuestiona entonces cómo quedan establecidos estos límites de la posibilidad lógica. ¿Dependen por completo de lo que nuestra gramática permite? Es decir, ¿el hecho de que una cierta combinación de palabras tenga sentido, sea capaz de representar un estado posible de las cosas, es algo que depende completamente de su concordancia con ciertas reglas gramaticales?\footnote{\cite[Cf.~][216]{hacker2000mind}: \enquote{The proposition, the sentence with its sense (\emph{der sinnvolle Satz}), can be said to depict a \emph{possible} state of affairs. (An order represents a \emph{possible} action, an action which \emph{is to be} carried out (\S519).) The moot question now is: how are the bounds of logical possibility determined? Does it depend wholly on what our grammar permits? Does the fact that a certain combination of words make sense, depicts a possible state of affairs, depend on nothing more than its agreement with a set of grammatical rules?}} Ante esto Wittgenstein exclama \blockquote[{\cite[\S520]{wittgenstein1953phiinv}}: \enquote{But surely that is arbitrary! --- Is it arbitrary?}]{¡Pero sin duda eso es arbitrario! --- ¿Es arbitrario?}. Es decir, las reglas de la gramática no son reglas definidas por algún objetivo que pueda atribuirse al lenguaje, tampoco puede decirse que sean correctas o incorrectas porque estén de acuerdo o no con algún aspecto de la realidad. Parece que estas reglas están al arbitrio de la práctica humana. ¿Entonces puede una decisión arbitraria dar sentido a una expresión contradictoria?\footnote{\cite[Cf.~][216]{hacker2000mind}: \enquote{for surely, grammatical rules are arbitrary. They are not technical, means-ends rules, and cannot be said to be correct ore incorrect because they agree or fail to agree with reality. Does this mean that we can transform nonsense into sense by fiat, shift the bounds of sense at will? Could an arbitrary \emph{decision} transform the words `This is red and green all over simultaneously' into a legitimate sentence? Could we make it a rule that the words `red and green all over simultaneously' are licit?}}

A esto, Wittgenstein replíca: \blockquote[{\cite[\S520]{wittgenstein1953phiinv}}: \enquote{It is not every sentence-like formation that we know how to do something with, not every technique that has a use in our life; and when we are tempted in philosophy to count something quite useless as a proposition, that is often because we have not reflected sufficiently on its application.}]{No toda formación que asemeje una oración es algo con lo que sepamos qué hacer, no toda técnica es una que tenga un uso en nuestra vida; y cuando estamos tentados en la filosofía de estimar algo del todo inútil como una proposición, es con frecuencia porque no hemos reflexionado suficientemente en su aplicación.} Con esto señala que las reglas gramaticales no son arbitrarias en el sentido antes aludido. Si se estableciera como gramaticalmente lícita una expresión contradictoria, todavía habría que establecer en qué consiste su gramática, es decir, como ha de usarse la expresión. ¿Cómo quedaría verificada? ¿Qué se sigue de ella? ¿Cómo puede integrarse en el resto de nuestra gramática?\footnote{\cite[Cf.~][216]{hacker2000mind}: \enquote{\textins{grammatical rules} are not a matter of whim or of \emph{ad hoc} decision. Saying, stipulating, that `A is red and green all over simultaneously' is grammatically licit, i.e. makes sense, effects nothing unless one goes on to specify \emph{what sense} it makes. As it stands, it is a sentence-like formation which we do not know how to use.}}
El hecho de que sea necesario especificar esta aplicación, libera a la gramática de la arbitrariedad: \blockquote[{\cite[220]{teichmann2008ans}}: \enquote{That a technique, a rule, has or is capable of having a real application in our life is what prevents the essence created by grammar from being arbitrary. In virtue of what does a rule have such a real application? In our being the sort of creatures who find it natural to \emph{give} it certain applications in our lives, and who agree in so finding it. But this doesn't mean: a description of the sort of creatures we are (say, in terms of biology, or evolution, or empirical psychology) will provide a justification for the rule.}]{Que una técnica, una regla, tenga o sea capaz de tener una aplicación real en nuestras vidas es lo que impide que la esencia creada por la gramática sea arbitraria. ¿En virtud de qué puede tener una regla esta aplicación real? En nuestro ser el tipo de creaturas que encuentra natural el \emph{darle} a esta ciertas aplicaciones en nuestras vidas, y que estamos de acuerdo en encontrarle esta aplicación. Pero esto no significa: una descripción del tipo de creaturas que somos (diríamos, en terminos biológicos, o evolutivos o de psicología empírica) nos proveería una justificación para la regla.} Sobre esto puede ser pertinente el comentario de Ludwig: \blockquote[{\cite[114]{kerr1997theo}}: \enquote{Did we invent human speech? No more than we invented walking on two legs (RPP II, 435)}]{¿Es invento nuestro el hablar humano? No más que lo que pueda ser nuestra invención el caminar en dos piernas (RPP II, 435)}.

Anscombe entiende que no se le rebaja a la lógica su rigor si se concibe de este modo, aún cuando se le considera claramente como una creación linguística.\footnote{\cite[Cf.~][124]{anscombe1981parmenides:qli}: \enquote{The non-arbitrariness of logic, therefore, is not a way of `bargaining its rigour out of it' (PI, I, \S108). And yet it, with its rigour, is quite clearly regarded as linguístic creation.}} No es posible recurrir a una especie de duda metódica para construir el rigor lógico, pues el conflicto entre la regla y su aplicación así concebido es aparente: \blockquote[{\cite[Cf.~][124]{anscombe1981parmenides:qli}}: \enquote{those \textelp{are} cases where the `doubt', which in fact, of course, I hardly ever have as I apply a rule, has no real content, and disagreement is just imagined by the philosopher}]{estos \textelp{son} casos donde la `duda', que de hecho, desde luego, difícilmente sostengo al aplicar una regla, no tiene contenido real, y el desacuerdo es simplemente imaginado por el filósofo} es así que \blockquote[{\cite[Cf.~][124]{anscombe1981parmenides:qli}}: \enquote{The argument from mere conceivability leads only to empty, ornamental doubt, as in face of the idea of the deceiving demon}]{El argumento partiendo de la mera posibilidad de concebirla lleva solo a una duda vacía y ornamental, como en el caso de la idea del genio maligno}. Por otra parte sería también ficticio pensar que las posibilidades lógicas quedan definidas por reglas arbitrariamente, las expresiones de estas reglas tienen que contar con una aplicación posible dentro de nuestra actividad. Wittgenstein ofrece una ilustración que Elizabeth considera interesante para ejemplificar esto. \S521 invita a comparar `lógicamente posible' con `químicamente posible': \blockquote[{\cite[\S520]{wittgenstein1953phiinv}}: \enquote{One might perhaps call a combination chemically posible if a formula with the right valencies existed (e.g. H - O - O - O - H\,). Of course, such a combination need not exist; but even the formula HO$_2$ cannot have less than no combination corresponding to it in reality.}]{Podríamos quizás decir que cierta combinación es químicamente posible si existiera una fórmula con valencias correctas (p. ej. H - O - O - O - H\,). Desde luego, no es necesario que exista una combinación como esta; pero incluso la fórmula HO$_2$ no puede tener menos que ninguna combinación que se le corresponda en la realidad.} En este apendice a lo establecido en la sección anterior, Ludwig compara la fórmula H$_2$O$_3$ con HO$_2$; según el sistema dentro del que estas expresiones están construidas la primera puede considerarse `químicamente posible' y la segunda `químicamente imposible', el punto de Wittgenstein es que la primera no es más posible que la segunda o la segunda más imposible que la primera, en ambos casos nada se corresponde en la realidad con estas expresiones. Esto para afirmar que decir que algo es lógicamente posible o imposible no atribuye a una expresión ningún vinculo con alguna posibilidad en la realidad independiente del lenguaje.\footnote{\cite[Cf.~][219]{hacker2000mind}: \enquote{H$_2$O$_3$ might be called `chemically possible' in the sense that the right valencies exist for such a molecule. But nothing corresponds to this possibility in reality. HO$_2$, is, in this sense, chemically impossible. Nothing corresponds to it in reality either --- but it cannot have \emph{less than nothing} to correspond to it, i.e. less than what corresponds to H$_2$O$_3$. So what? So it is a mistake to suppose that grammar is justified by reference to objective logical possibilities, \emph{as if logical possibilities were shadowy actualities}.}} Lo que resulta interesante para Anscombe es que: \blockquote[{\cite[Cf.~][124]{anscombe1981parmenides:qli}}: \enquote{The notation enables us to construct the formula HO$_2$, but the system then rules it out. Impossibility even has a certain role: one examines a formula to see that the valencies are right. The exclusion belongs to the system, a human construction. It is objective; that is, it is not up to me to decide what is allowable here.}]{La notación nos permite construir la formula HO$_2$, pero el sistema la prohíbe. La imposibilidad incluso tiene un rol: examinamos la fórmula para ver que las valencias están correctas. La exclusión pertence al sistema, un constructo humano. Es objetivo; esto es, no depende de uno el decidir qué está permitido aquí.}

Con esto Anscombe llega a un último examen sobre la posibilidad de algún tipo de idealismo. Así comienza la segunda parte de su artículo, donde aborda el tema desde su aspecto epistemológico. Para componer su pregunta recurre a una expresión del mismo Wittgenstein: \blockquote[{\cite[Cf.~][124]{anscombe1981parmenides:qli}}: \enquote{``So you are saying that human agreement decides what is true and what is false? --- It is what humans \emph{say} that is true and false, and they agree in the \emph{language} they use. That is not agreement in opinions\ldots''(PI,I,\S241). What are the implications of `agreement in language'?}]{``Entonces ¿estás diciendo que el acuerdo humano decide lo que es verdadero y lo que es falso? --- Lo que los humanos \emph{dicen} es lo que es verdadero y falso, y en lo que se ponen de acuerdo es en el \emph{lenguaje} que usan. Eso no es estar de acuerdo sobre opiniones\ldots''(PI,I,\S241). ¿Cuáles son las implicaciones de `acuerdo en el lenguaje'?} Con esta pregunta Anscombe se adentra en una cuestión en la que el pensamiento de Wittgenstein maduró durante los últimos años de su vida. Esta tiene que ver con la posibilidad de justificar creencias propias de una \emph{imagen del mundo} y un contexto con sus prácticas en el uso del lenguaje, dentro de otro contexto distinto con una \emph{imagen del mundo} diferente. En el trabajo de Wittgenstein hasta \emph{On Certainty}, dice Elizabeth: \blockquote[{\cite[Cf.~][124]{anscombe1981parmenides:qli}}: \enquote{we might think we could discern a straightforward thesis: there can be no such things as `rational grounds' for our criticizing practices and beliefs that are so different from our own. These alien practices and language games are simply there. They are not ours, we cannot move in them.}]{podemos pensar que es posible discernir una tesis clara: no puede haber cosa alguna como `fundamentos racionales' para nuestras prácticas en crítica de creencias que son tan distintas de las nuestras. Estas prácticas y juegos de lenguaje foraneos simplemente están ahí. No son nuestros, no podemos movernos en ellos.} Sin embargo en \emph{On Certainty} Ludwig estudia la justificación posible que puede tener G.\,E.\,Moore para afirmar, como lo hace en \emph{Proof of the External World} y \emph{Defence of Common Sense}, que él \emph{conoce}, entre otras cosas, que nunca ha estado lejos de la superficie de la tierra, o que el mundo ha existido desde mucho antes de que él naciera. La investigación se realiza proponiendo cómo justificar estas creencias en contextos y sistemas de conocimiento basados en \emph{imágenes del mundo} distintas a las de Moore. Así, por ejemplo: \blockquote[{\cite[\S264--266]{wittgenstein1969oncert}}: \enquote{I could imagine Moore being captured by a wild tribe, and their expressing the suspicion that he has come from somewhere between the earth and the moon. Moore tells them that he knows etc. but he can't give them the grounds for his certainty, because they have fantastic ideas of human ability to fly and know nothing about physics. \textelp{} But what does it say, beyond ``I have never been to such and such a place, and have compelling grounds for believing that''? And here one would have to say what are compelling grounds.}]{Podría imaginar a Moore siendo capturado por alguna tribu salvaje, y ellos expresando la sospecha de que su procedencia sea algún lugar entre la tierra y la luna. Moore entonces les explica que él conoce etc. pero no es capaz de ofrecerles fundamentos para su certeza, pues ellos sostienen ideas fantásticas sobre la capacidad de los humanos para volar y no conocen nada de física. \textelp{} Pero ¿qué se diría, más allá de ``Yo no he estado en tal o cual lugar, y tengo fundamentos convincentes para creer eso''? Y aquí tendríamos que decir qué son fundamentos convincentes.} El interés de Wittgenstein es describir cómo Moore está empleando el término `conocer' y la diferencia de emplearlo en un escenario como este a usarlo en el contexto del sistema de conocimiento del que Moore forma parte. Lo que interesa a Anscombe, por su parte, es si depende de la práctica del lenguaje de un contexto específico el poder justificar una creencia cierta y verdadera. O dicho de otra manera, si el conocer depende completamente del juego de lenguaje de algún contexto específico.

A lo largo de \emph{On Certainty} Wittgenstein plantea otra serie de escenarios. Entre ellos, imagina a una tribu de adultos que conceden que no hay un modo ordinario de llegar a la luna, pero creen que las personas a veces viajan allá, quizás en esto consiste para ellos el soñar. ¿Qué podríamos replicar para justificar que conocemos que eso no es verdadero? ¿Sería igual el caso de un niño que cree la historia que le contó un adulto sobre su viaje a la luna? ¿Qué respuesta podríamos darle? (\S106--108) Imagina también el caso de un hombre que ha crecido bajo la enseñanza de que la tierra empezó a existir hace cincuenta años. ¿Qué sería enseñarle la verdad, o darle a conocer lo que nosotros conocemos? (\S262) O también un rey que ha sido educado en la creencia de que el mundo comenzó con él. ¿Qué conllevaría darle a conocer el mundo como nosotros lo conocemos? (\S92)

Acerca de todos estos escenarios Anscombe dice: \blockquote[{\cite[130]{anscombe1981parmenides:qli}}: \enquote{we should not regard the struggling investigations of \emph{On Certainty} as all saying the same thing. Doubts whether this is a tree or whether his name was L.\,W. or whether the world has existed a long time or whether the kettle will heat on the fire or whether he had never been to the moon are themselves not subjected to the same treatment. Not all these things, for example, are part of a `world-picture'.}]{no deberíamos de considerar las esforzadas investigaciones de \emph{On Certainty} como todas diciendo la misma cosa. Las dudas sobre si esto es un árbol o si su nombre era L.\,W. o si el mundo ha existido por un largo tiempo o si la tetera se calentará en el fuego o si él nunca ha estado en la luna no son todas ellas sometidas al mismo tratamiento. No todas estas cosas, por ejemplo, son parte de una `imagen del mundo'.} Toda esta variedad de escenarios en los que Wittgenstein se pregunta en qué consiste `dudar' o qué sería `conocer' viene a ser una ocasión para poner en práctica su consigna: \enquote{te mostraré diferencias}. En algunos de estos casos atender la duda tiene que ver con qué justificación puede ser ofrecida, en otros con qué réplica puede ser adecuada, en algunos las dificultades se hallan en el \emph{sistema de conocimiento} de los interlocutores, y en otros lo que entra en juego es la diferencia de \emph{imágenes del mundo}.

El caso específico en el que el conflicto es causado por una diferencia de imágenes del mundo o de sistema de conocimiento, para Anscombe, ejemplifica un conflicto de principios real donde el desacuerdo no consiste en los datos o en el uso de las palabras, sino en el trasfondo que sirve como justificación para la certeza en las evidencias.\footnote{\cite[Cf.~][222]{teichmann2008ans}: \enquote{there are disagreements, actual and hypothetical, where what is lacking is just this background agreement as to what count as reasons \emph{pro} and \emph{con}}} Este tipo de conflicto sera el foco su respuesta a la cuestión sobre si el conocimiento depende completamente del acuerdo en el lenguaje o del juego de lenguaje de un contexto. La relación entre imagen del mundo como fundamento de la certeza, el juego de lenguaje y la justificación del conocimiento son los tres elementos que interactuan en esta respuesta.

Lo que Wittgenstein llama `imagen del mundo' (`\emph{Weltbild}' como distinto de `\emph{Weltanschauung}') sirve como el \blockquote[{\cite[\S94]{wittgenstein1969oncert}}: \enquote{inherited background against which I distinguish between true and false}]{trasfondo heredado desde el cual distinguimos entre verdadero y falso}. Este trasfondo está compuesto por proposiciones que están en un estado de fluctuación y otras, sólidas, que sirven como cauce para las primeras. Wittgenstein lo describe así: \blockquote[{\cite[\S95--99]{wittgenstein1969oncert}}: \enquote{The propositions describing this world-picture might be part of a kind of mythology. And their role is like that of rules of a game; and the game can be learned purely practically, without learning any explicit rules. It might be imagined that some propositions, of the form of empirical propositions, were hardened and functioned as channels for such empirical propositions as were not hardened but fluid; and that this relation altered with time, in that fluid propositions hardened, and hard ones became fluid. The mythology may change back into a state of flux, the river-bed of thoughts may shift. But I distinguish between the movement of the waters on the river-bed and the shift of the bed itself; though there is not a sharp division of the one from the other. But if someone were to say ``So logic too is an empirical science'' he would be wrong. Yet this is right: the same proposition may get treated at one time as something to test by experience, at another as a rule of testing. And the bank of that river consists partly of hard rock, subject to no alteration or only to an imperceptible one, partly of sand, which now in one place now in another gets washed away, or deposited.}]{Las proposiciones que describen esta imagen del mundo podrían pertenencer a una especie de mitología. Y su función es así como las reglas de un juego; y el juego puede ser aprendido de un modo puramente práctico, sin necesidad de aprender ninguna regla explicita. Puede imaginarse que algunas proposiciones, que tienen la forma de proposiciones empíricas, se solidifican y funcionan como canales para aquellas otras proposiciones empíricas que no se han solidificado, sino que fluctuan; y que esta relación se altera con el tiempo, en el que las proposiciones fluctuantes quedan solidificadas, y las sólidas se tornan fluídas. La mitología puede cambiar de nuevo a un estado de fluctuación, el lecho del río de los pensamientos puede desplazarse. Pero distingo entre el agitamiento de las aguas cercanas al lecho del río y el desplazamiento del suelo mismo; aunque no hay una división nítida entre uno y otro. Pero si alguien dijera ``Entonces la lógica también es una ciencia empírica'' estaría equivocado. Y sin embargo esto sí es cierto: la misma proposición puede ser tratada en una ocasión como algo que se evalúa por la experiencia, y en otra como una regla para evaluar. Y la orilla de ese río consiste en parte de roca dura, no sometida a la alteración o solo a un cambio imperceptible, y en parte de arena, que de un lado a otro es arrastrada por la corriente o que queda depositada.} A lo largo de \emph{On Certainty}, uno de los temas principales es el carácter infundado de esta imagen del mundo, Anscombe explica: \blockquote[{\cite[130]{anscombe1981parmenides:qli}}: \enquote{Finding grounds, testing, proving, reasoning, confirming, verifying are all processes that go on \emph{within}, say, one or another living linguistic practice which we have.}]{Encontrar fundamentos, poner a prueba, demostrar, razonar, confirmar, verificar son todos procesos que ocurren \emph{dentro}, diríamos, de una u otra prática linguística viva de las que tenemos.} Que conocemos el significado de ciertas palabras, que otras tengo que consultarlas en el diccionario, que esto es lo que consultar un diccionario implica, este son el tipo de cosas que son, continúa Anscombe: \blockquote[{\cite[130]{anscombe1981parmenides:qli}}: \enquote{There are assumptions, beliefs, that are the `immovable foundation' of these proceedings. \textelp{} they are a foundation which is not moved by any of these proceedings. I cannot doubt or question anything unless there are some things I do not doubt or question}]{supuestos, creencias, que son la `fundación inmóvil' de nuestros modos de proceder. \textelp{} estos son una fundación que no queda trastocada por ninguno de estos procedimientos. No puedo dudar o cuestionar nada a no ser que haya algunas cosas de las que no dudo o cuestiono}. La anterior descripción de proposiciones en fluctuación orientadas por otras sólidas que sirven como el cauce un río aquí es útil. Los procesos dentro de nuestra actividad fluyen teniendo como trasfondo una serie de proposiciones que no son alteradas por esta actividad.

Lo que Anscombe quiere aclarar es si en la perspectiva de Wittgenstein hay algún elemento externo relacionado con la composición de la imagen del mundo a través del tiempo o si este trasfondo de nuestras certezas es el derivado de arbitrarios acuerdos en el lenguaje. Cuando hay un desacuerdo entre imagenes del mundo, Wittgenstein parece rechazar la idea de que uno de ellos esté en lo correcto y el otro equivocado. La imagen del mundo forma parte del fundamento del sistema de conocimiento y parece que es cuestionable el derecho de juzgar como error desde mi sistema, algo que es tenido como conocimiento en otro.\footnote{\cite[Cf.~][131--132]{anscombe1981parmenides:qli}: \enquote{it may seem that if ever world-pictures are incompatible, Wittgenstein rejects the idea of one of them's being right, the other wrong. A world-picture partly lies behind a knowledge system. One knowledge system may be far richer than another, just as it may be connected with far greater capacities of travel, for example. But when, speaking with \emph{this} knowledge system behind one, one calls something an error which \emph{counts as knowledge} in another system, the question arises: has one the right to do that? Or has one to be `moving within the system' to call anything error?}} Wittgenstein describe esta situación como una \blockquote[{\cite[\S611--612]{wittgenstein1969oncert}}: \enquote{Where two principles really do meet which cannot be reconciled with one another, then each man declares the other a fool and a heretic. I said I would `combat' the other man, ---but wouldn't I give him \emph{reasons}? Certainly; but how far do they go? At the end of reasons comes \emph{persuasion}. (Think what happens when missionaries convert natives.)}]{Donde dos principios realmente se encuentran y que no pueden ser reconciliados el uno con el otro, entonces cada persona declara a la otra ignorante y hereje. He dicho que en estos casos `combatiría' con esa otra persona, ---pero ¿acaso no le daría \emph{razones}? Ciertamente; pero ¿cuán lejos llegaría con ellas? Al final de las razones está la \emph{persuasión}. (Piensa en lo que ocurre cuando misioneros convierten nativos.)} Aquí Elizabeth centra su atención en el conflicto de principios. Es decir, no está preocupada de hablar sobre relativismo cultural, sino de la situación en que dos principios se encuentran y ocurre un desacuerdo real.\footnote{\cite[Cf.~][131]{anscombe1981parmenides:qli}: \enquote{So what is in question here is not: cultural relativism. For the assumption is of ``two principles which really meet and can't be reconciled'' and ``each man declares the other a fool and a heretic'' (Cert.,\S611). That is to say, we have a ``disagreement in the language they use'' --- but it really is a disagreement.}} La respuesta de Wittgenstein: \enquote*{Al final de las razones está la persuasión}, hace que Anscombe se cuestione: \blockquote[{\cite{anscombe1981parmenides:qli}}: \enquote{Is it futile to say here: But won't the persuasion be right or wrong, an intellectual disaster or intellectual enlightenment?}]{Sería inútil decir aquí: Pero ¿acaso la persuasión no será correcta o incorrecta, un desastre o una iluminación intelectual?} De qué nos sirve decir que al final de las razones está la persuasión. ¿Estamos aquí ante otro caso como el de las reglas que no pueden decirnos como ir? Sobre esa especie de duda metódica quedó establecido que el conflicto entre regla y aplicación no era real, sino imaginado por el filósofo. En este caso el conflicto no es vacío. Wittgenstein pone otro ejemplo: \blockquote[{\cite[\S641]{wittgenstein1969oncert}}: \enquote{``He told me about it today --- I can't be making a mistake about that.'' --- But what if it does turn out to be wrong?! --- Musn't one make a distinction between the ways in which something `turns out wrong'? --- How \emph{can} it \emph{be shewn} that my statement was wrong? Here evidence is facing evidence, and it must be \emph{decided} which is to give way.}]{``Él me habló sobre esto hoy --- no puedo estar equivocado al respecto.'' --- Pero ¿¡qué si resulta que sí estoy equivocado!? --- No deberíamos hacer una distinción entre los modos en los que algo `resulta ser equivocado'? --- ¿Cómo \emph{puede} \emph{mostrarse} que mi afirmación fue equivocada? Aquí tenemos evidencia contra evidencia, y tiene que ser \emph{decidido} cúal ha de ceder.}

Tenemos entonces estos dos escenarios en donde un conflicto real ocurre; en uno la resolución culmina en persuación y en el otro en decisión. Elzabeth comenta \blockquote[{\cite{anscombe1981parmenides:qli}}: \enquote{And now isn't it as if Wittgenstein were saying: there isn't a right or wrong --- but only the conflict, or persuasion, or decision?}]{Entonces ¿acaso no es como si Wittgenstein estuviera diciendo: no hay correcto o incorrecto --- solo el conflicto, la persuasión o la decisión?} ¿Estas son las tres resoluciones posibles cuando dos principios irecconciliables se encuentran?

Wittgenstein continúa: \blockquote[{\cite[\S643--645]{wittgenstein1969oncert}}: \enquote{Admittedly one can imagine a case ---and cases do exist--- where after the ``awakening'' one never has any more doubt which was imagination and which was reality. But such a case, or its possibility, doesn't discredit the proposition ``I can't be wrong''. For otherwise, wouldn't all assertion be discredited in this way? I can't be making a mistake, ---but some day, rightly or wrongly, I may think I realize that I was not competent to judge.}]{Ciertamente podríamos imaginar un caso ---y sí existen casos así--- donde tras el ``despertar'' ya no se vuelve a tener ninguna duda sobre lo que fue imaginación y lo que es realidad. Pero un caso como ese, o su posibilidad, no desacredita la proposición ``no puedo estar equivocado''. Pues si no fuera así, ¿no quedaría de este modo desacreditada toda aseveración? No puedo estar equivocado, ---pero algún día, correcta o incorrectamente, puedo llegar a pensar que he caído en la cuenta de que no fui competente al juzgar.} Podemos interpretar este pasaje desde la descripción antes vista de la imagen del mundo. El juego de lenguaje de la aseveración no queda desacreditado porque haya proposiciones de las que hayamos tenido el derecho de decir `no puedo estar equivocándome' y de las cuales más tarde lleguemos a creer que no fuimos competentes en ese juicio.\footnote{\cite[Cf.~][223]{teichmann2008ans}: \enquote{The `immovable foundations' could be moved, e.g. by extreme experiences, or by the development of wholly new techniques (such as mathematical techniques). It is this fact which lies behind the possibility that, within some practice, I may be quite justified in saying `I can't be making a mistake about this', while at the same time admitting that given certain changes, I could come to deny the truth of what I am now saying (QLI 132; cf. \emph{On Certainty, paras. 643--5}). Within the practice, I can't see how $P$ could be doubted without this disrupting the whole practice (e.g. maths, or ancient history); so I can say, `I can't be mistaken that $P$'. But I can simultaneously admit the general possibility of experiences or developments that would radically alter the practice, and so alter what counted in it as bedrock, while leaving enough of it intact for `$P$' still to count as making the same claim as before --- a claim, however, that no longer enjoyed its status as `immovable', but on the contrary counted as false.}} Para Anscombe lo complejo aquí es que sea posible que esa creencia de no haber sido competentes al juzgar pueda ser justificada como correcta. En otras palabras, le parece que decir que podríamos llegar a pensar correctamente que no fuimos competentes al juzgar algo de lo que estuvimos en el derecho de afirmar que no podríamos estar equivocándonos suena como afirmar \enquote*{no puedo estar equivocándome, pero puedo estar equivocado}. Si no fuera por la distinción que Wittgenstein hace entre lo que es una equivocación y lo que puede ser otra cosa, tendríamos aquí una contradicción.\footnote{\cite[Cf.~][124]{anscombe1981parmenides:qli}: \enquote{It runs through Wittgenstein's thought that you haven't got a \emph{mistake} just because you have as a complete utterance a string of words contrary to one in which some truth is expressed. I can be accused of \emph{making a mistake} when I know what it is for a given proposition (say) to be true, and things aren't like that but I suppose that they are. (There is a corresponding condition for being right.) This means that I have to be actually operating the language. My proceedings with it have to belong in the system of thought that is in question. Otherwise such an utterance may be nothing at all; it may be `superstition' (PI,I,\S110) or `a queer reaction' or a manifestation of some different `picture of the world', or of a special form of belief which flies in the face of what would be understood to falsify it but for its peculiarity; it may be some strange secondary application of words; it may be a mere manifestation of ignorance like a child's. It may be madness. But in none of these cases is Wittgenstein willing to speak of a `mistake'.}} Si, dados ciertos cambios, llegaramos a estar en desacuerdo con nosotros mismos sobre algo que teníamos como fundamento cierto, entonces esa creencia original no sería propiamente una equivocación,\footnote{\cite[Cf.~][223]{teichmann2008ans}: \enquote{If (given certain changes) I were to disagree with my former self, and that disagreement had to do with a change in what I took to be bedrock or immovable, then it seems that by Wittgenstein's lights I shouldn't call my former assertion a \emph{mistake}.}} sino `otra cosa': \blockquote[{\cite[Cf.~][223]{teichmann2008ans}}: \enquote{The `something else' would be: not being competent to judge. On the face of it, this doesn't look very like the diagnosis of anything as radical as a difference of world-picture, as Wittgenstein seems to regard the latter when imagining cases of disagreement between two parties; for in the end, \emph{such} disagreement (he says) is only resoluble, if at all, by persuasion or conversion. \textelp{} The matter need not detain us long, however, for as Anscombe says, the crux of the matter comes with Wittgenstein's next remark}]{La `otra cosa' sería: no ser competentes para juzgar. De primera impresión, esto no parece muy semejante al diagnóstico de algo tan radical como un conflicto entre imágenes del mundo, que es como Wittgenstein valora esto segundo cuando imagina casos de desacuerdo entre dos partidos; dado que al final, \emph{ese tipo} de desacuerdo (dice él) solo puede resolverse, si del todo, por persuasión o conversión. \textelp{} Este asunto no debería detenernos mucho, si embargo, pues como dice Anscombe, el punto decisivo de la cuestión se encuentra en el comentario que Wittgenstein hace a continuación}

En su observación siguiente Ludwig habla de \enquote*{no puedo estar equivocándome} y de llegar a pensar que no fui competente para juzgar en esa ocasión como dos fenómenos que pueden seguirse uno del otro, y entonces dice: \blockquote[{\cite[\S646]{wittgenstein1969oncert}}: \enquote{Admittedly, if that always or often happened it would completely alter the character of the language-game.}]{Ciertamente, si esto ocurriera siempre o con frecuencia alteraría completamente el carácter del juego de lenguaje.} Y con esta observación, juzga Anscombe, Wittgesntein llega a una conclusión que no es idealista.  \blockquote[{\cite[133]{anscombe1981parmenides:qli}}: \enquote{\emph{That one knows something is not guaranteed by the language-game}. \textelp{} But the language-game of assertion, which for speaking humans is so important a part of the whole business of knowing and being certain, depends for its character on a `general fact of nature'; namely that that sequence of phenomena is rare.}]{\emph{Que conozcamos algo no está garantizado por el juego de lenguaje}. Sino que el juego de lenguaje de la aseveración, que juega un papel tan importante para los seres humanos que se comunican en todo el proceso de conocer y tener certeza, depende para su carácter en un `hecho general de la naturaleza'; es decir en que esa secuencia de fenómenos es rara.} O como diría Wittgesntein: \blockquote[{\cite[\S505; 509]{wittgenstein1969oncert}}: \enquote{It is always by favour of Nature that one knows something. \textelp{} I really want to say that a language-game is only possible if one trusts something\ldots}]{Es siempre por gracia de la Naturaleza que conocemos alguna cosa. \textelp{} Lo que en realidad quiero decir es que un juego de lenguaje solamente es posible si se confía en algo\ldots}.

Con este argumento final Anscombe entiende que Wittgenstein evitó el idealismo linguístico, alcanzando en su lugar `realismo sin empirismo': \blockquote[{\cite[224]{teichmann2008ans}}: \enquote{For Wittgenstein, `that one knows something is not guaranteed by the language-game' (QLI 133) --- for there is such a thing as radical change of view, however rare, and the natural expresssion of this is `I was wrong'. The possibility of radical change of view is compatible with the fact that, in the absence of such change, `I know' and `I am certain' are justifiable forms of expression within the language-game.}]{Para Wittgenstein, `que conozcamos algo no está garantizado por el juego de lenguaje' (QLI 133) --- pues hay tal cosa como un cambio radical de perspectiva, por más raro que sea, y la expresión natural de esto es `estaba equivocado'. La posibilidad de un cambio radical de perspectiva es compatible con el hecho de que, en la ausencia de tal cambio, `yo conozco' y `tengo certeza' son formas justificables de expresión dentro del juego de lenguaje.}

Estas afirmaciones sirven para culminar la investigación y esta discusión sobre la relación entre la realidad, el lenguaje y el pensamiento. Sin embargo, para terminar nuestra discusión sobre el artículo es preciso volver sobre algunos puntos relacionados con el lenguaje religioso. Algunos comentarios de Anscombe sirven para distinguir algunos elementos de su comprensión del las proposiciones religiosas y para descubrir algunos puntos donde la cuestión del lenguaje religioso interactua con toda la discusión que hemos recorrido.

Elizabeth observa que: \blockquote[{\cite[123]{anscombe1981parmenides:qli}}: \enquote{Wittgenstein's attitude to the whole of religion in a way assimilated it to the mysteries: thus he detested natural theology. But again, what part of this was philosophical (and therefore something which, if right, others ought to see) and what personal, it is difficult to say.}]{La actitud de Wittgenstein hacia el todo de la religión la asimilaba en cierto modo a los misterios: por consiguiente detestaba la teología natural. Pero de nuevo, qué parte de esto era filosófico (y por tanto algo que, si correcto, otros han de ver) y qué parte era personal, es difícil decir.} En \emph{On Certainty} hace referencia a distintas ideas y creencias religiosas, específicamente, por ejemplo: \blockquote[{\cite[\S239]{wittgenstein1969oncert}}: \enquote{I believe that every human being has two human parents; but Catholics believe that Jesus only had a human mother. \textelp{} Catholics believe as well that in certain circumstances a wafer completely changes its nature, and at the same time that all evidence proves the contrary. And so if Moore said ``I know that this is wine and not blood'', Catholics would contradict him.}]{Creo que todo ser humano tiene dos padres humanos; pero los católicos creen que Jesús solo tuvo una madre humana. \textelp{} Los católicos creen también que en ciertas circunstancias un trozo de pan completamente cambia su naturaleza, y al mismo tiempo que toda evidencia demuestra lo contrario. Y así si Moore dijera ``Yo conozco que eso es vino y no sangre'', los católicos lo contradirían.} Anscombe cuenta también otro ejemplo:\blockquote[{\cite[122]{anscombe1981parmenides:qli}}: \enquote{At the Moral Science Club he once quoted a passage from St Augustine about God which with the characteristic rhetoric of St Augustine sounded contradictory, Wittgenstein even took ``he moves without moving'' as a contradcition in intent, and was impatient being told that that at least was not so, the first ``moves'' being transitive and the second intransitive (\emph{movet, non movetur}).}]{En una ocasión citó en el \emph{Moral Science Club} un pasaje de San Agustín acerca de Dios el cual con la retórica característica de San Agustín sonaba contradictorio, Wittgenstein incluso tomó ``mueve sin moverse'' como una contradicción de propósito, y se mostró impaciente al decírsele que eso al menos no era así, el primer ``mueve'' siendo transitivo y el segundo intransitivo (\emph{movet, non movetur}).} En ambos casos Ludwig ve proposiciones que trata como misterios, sin hacer distinción entre argumentaciones de teología natural o creencias en misterios como la Eucaristía. Su actitud ante los misterios no era contraria, sino que, por ejemplo, en el caso del argumento de Agustín: \blockquote[{\cite[122]{anscombe1981parmenides:qli}}: \enquote{He wished to take the contradiction as seriously intended and at the same time to treat it with respect.}]{Él deseaba tomar la contradicción como seriamente intencional y al mismo tiempo quería tratarla con respeto.}

Anscombe atribuye todo esto a el desagrado de Wittgenstein de describir la religión como racional: \blockquote[{\cite[122]{anscombe1981parmenides:qli}}: \enquote{This was connected with his dislike of rationality or would-be rationality in religion. He would describe this with a characteristic simile: there is something all jagged and irregular, and some people have a desire to encase it in a smooth ball: looking within you see the jagged edges and spikes, but a smooth surface has been constructed. He preferred it left jagged. I don't know how to distribute this between philosophical observation on the one hand and personal reaction on the other.}]{Esto estaba conectado con su desagrado de la racionalidad o potencial racionalidad de la religión. Describía esto con un símil característico: hay algo todo escarpado e irregular, y algunas personas tienen el deseo de encerrarlo en una esfera lisa: mirando dentro de ella se pueden ver las espinas e irregularidades, pero una superficie lisa ha sido construida sobre estas. Él prefería que se dejara escarpado. No se como distribuir esto entre observación filosófica por una parte y reacción personal por otra.} Adicionalmente dentro del sistema de pensamiento de Ludwig no es posible justificar el tipo de proposiciones que la teología natural emplea: \blockquote[{\cite[123]{anscombe1981parmenides:qli}}: \enquote{In natural theology there is attempted reasoning from the objects of the world to something outside the world. Wittgenstein certainly worked and thought in a tradition for which this was impossible.}]{En la teología natural hay un intento de razonamiento desde los objetos del mundo a algo fuera del mundo. Wittgenstein ciertamente trabajó y pensó en una tradición para la cual esto era imposible.}

Anscombe claramente no comparte estas opiniones de Wittgenstein. Tras afirmar que para Ludwig `pensar' significa \enquote*{actuar según las reglas de razonamiento correcto} se pregunta: \blockquote[{\cite[122]{anscombe1981parmenides:qli}}: \enquote{If so, then what will Wittgenstein say about `illogical' thinking? As I would, that it isn't thinking?}]{Si esto es así, entonces ¿qué diría Wittgenstein sobre el pensamiento `ilógico'? ¿Como diría yo, que no es pensar?} Y continúa: \blockquote[{\cite[122]{anscombe1981parmenides:qli}}: \enquote{In the Catholic faith, certain beliefs (such as the Trinity, the Incarnation, the Eucharist) are called ``mysteries''; this means at the very least that it is neither possible to demonstrate them nor possible to show once for all that they are not contradictory and absurd. On the other hand contradiction and absurdity is not embraced; ``This can be disproved, but I still believe it'' is not an attitude of faith at all. So ostenisble proofs of absurdity are assumed to be rebuttable, each one in turn.}]{En la fe católica, ciertas creencias (como la Trinidad, la Encarnación, la Eucaristía) son llamadas ``misterios''; esto significa en el mejor de los casos que ni es posible demostrarlas ni tampoco es posible mostrar de una vez por todas que no son contradictorias y absurdas. Por otra parte la contradicción y lo absurdo no son abrazados; ``Esto puede ser refutado, pero aún así lo creo'' no es para nada una actitud de fe. Entonces las ostensibles demostraciones de absurdidad son asumidas como rebatibles, cada una en su turno.} Esta distinción entre no pensar y el misterio es característica de Elizabeth que en diversas ocasiones propone que la capacidad de argumentar está en servicio de disipar los ataques que pretendan demostrar como definitivamente absurdas las proposiciones que expresan misterios. La actitud que acompaña esta perspectiva, estar dispuestos a atender cada crítica, era una que compartía con Ludwig: \blockquote[{\cite[122]{anscombe1981parmenides:qli}}: \enquote{Now this process Wittgenstein himself once described: ``You can ward off \emph{each} attack as it comes'' (Personal Conversation).}]{Ahora, este proceso Wittgenstein mismo lo describió en una ocasión: ``Puedes mantener a raya \emph{cada} ataque según venga'' (Conversación personal).}

Elizabeth mantiene en sus distintos artículos esa distinción entre el sinsentido o los meros pensamientos ilógicos y lo que puede ser valorado como un misterio. Reconoce su naturaleza extraordinaria: \blockquote[{\cite[122--123]{anscombe1981parmenides:qli}}: \enquote{religious mysteries are not a theory, the product of reasoning; their source is quite other}]{los misterios religiosos no son una teoría, el producto del razonamiento; su fuente es totalmente otra}, y sin embargo afirmarlos no va en contra de considerar, como ella lo hace, que hay razonabilidad en la fe: \blockquote[{\cite[122]{anscombe1981parmenides:qli}}: \enquote{the attitude of one who does that, or wishes that that should be done, is not that of willingness to profess contradiction. On the contrary.}]{la actitud de uno que hace esto, o que desea que eso se haga, no es la de una disposición a profesar la contradicción. Al contrario.}
