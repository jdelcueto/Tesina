\subsection{The Question of Linguistic Idealism}

\emph{The Question of Linguistic Idealism} fue publicado en 1976 en \emph{Acta Philosophica Fennica} junto a otros ensayos sobre Wittgenstein en honor de G.\,H.\,von Wright. Georg Henrik von Wright fue sucesor de Wittgenstein en la cátedra de filosofía en Cambrdige entre 1948--1951, puesto que Anscombe ocuparía en 1970; también fue con Elizabeth uno de los responsables del legado literario de Wittgenstein.

Como fue anticipado en la discusión del artículo \emph{Parmenides, Mystery and Contradiction} este ensayo sirve como conclusión al primer volumen de los \emph{Collected Philosophical Papers} dedicados a distintas reflexiones en torno a la relación entre la realidad, el pensamiento y el lenguaje. En aquel artículo la tradición subyacente al \emph{Tractatus} fue examinada por Anscombe desde la perspectiva de \emph{Investigaciones Filosóficas}. Aquí Elizabeth examina esta segunda etapa del pensamiento de Wittgenstein y se pregunta si logra aquella difícil empresa planteada por Ludwig: \blockquote[{\cite[112]{wittgensteinrfm}}: Not empiricsm and yet realism in philosophy, that is the hardest thing]{Realismo en la filosofía sin caer en empirismo, eso es lo más complicado}.

El modo en que Elizabeth enmarca la investigación recuerda sus palabras en la introducción de esta colección: \blockquote[{\cite[xi]{anscombe1981parmenides}}: At the present day we are often perplexed with enquiries about what makes true, or what something's being thus or so \emph{consists in}; and the answer to this is thought to be an explanation of meaning. If there is no external answer, we are apparently committed to a kind of idealism.]{En la época actual con frecuencia nos quedamos perplejos con preguntas sobre qué hace a algo verdadero, o \emph{en qué consiste} el que algo sea de un modo u otro; y la respuesta a esto se piensa que es una explicación del significado. Si no hay una respuesta externa, aparentemente estamos comprometidos con un tipo de idealismo.} En \emph{Investigaciones Filosóficas} la relación entre la realidad y el pensamiento se plantea como una relación interna. Anscombe se pregunta sobre la posibilidad de que se encuentre en esta etapa del pensamiento de Wittgenstein un planteamiento idealista. Toma como punto de partida el siguiente pasaje: \blockquote[{\cite[112]{anscombe1981parmenides:qli}}: ``If anyone believes that certain concepts are absolutely the right ones, and that having different concepts would mean not realizing something that we realize\,---\,then let him imagine certain very general facts of nature to be different from what we are used to, and the formation of  concepts different from usual ones will become intelligible to him'' (Philosophical Investigations \textins{PI}, II, XII).]{``Si alguna persona cree que ciertos conceptos son absolutamente los correctos, y que tener otros conceptos significaría que no se apreciaría algo de lo que nosotros apreciamos\,---\,entonces que imagine ciertos hechos muy generales de la naturaleza como siendo distintos de lo que estamos acostumbrados, y la formación de conceptos distintos de los usuales se le harán inteligibles'' (Investigaciones Filosóficas \textins{IF}, II, XII).} Entonces plantea: \blockquote[{\cite[112]{anscombe1981parmenides:qli}}: This is one of the passages from Wittgenstein arousing ---in my mind at least--- the question: have we in his last philosophical thought what migth be called linguistic idealism? Linguistic, because he describes concepts in terms of linguistic practices. And he also wrote: ``\emph{Essence} is expressed by grammar'' (PI, I, \S371).]{Este es uno de los pasajes de Wittgenstein que despierta ---en mi mente al menos--- la pregunta: ¿tenemos en su pensamiento filosófico tardío lo que podríamos llamar idealismo linguístico? Linguistico, porque describe los conceptos en terminos de prácticas linguísticas. Y también escribió: ``La \emph{esencia} es expresada por la gramática'' (IF, I, \S371).}

Si estamos de acuerdo con que la esencia está expresada en la gramática entonces tendríamos que decir que las palabras que usamos para hablar de algo tienen que tener una gramática específica. Pero ¿esto qué significa? Esta propiedad gramática que se adscribe a estas expresiones ¿es propia del objeto del que la expresión habla, o del lenguaje? La manera de decirlo tendría que ser que la propiedad es del lenguaje, y por tanto no caracteriza al objeto sino al lenguaje, es decir, si esta expresión no tiene esta propiedad, esta gramática, deja de ser lenguaje acerca de este objeto. En este sentido la gramática \emph{corresponde} con la esencia del objeto y el objeto mismo es independiente del lenguaje. Según esto, Anscombe destaca que, efectivamente, la esencia es expresada por la gramática, sin embargo si imagináramos otro lenguaje distinto con otra gramática y otros conceptos y también personas que usaran este otro lenguaje, estas personas, en efecto, no estarían usando un lenguaje cuya gramática expresara las mismas esencias que nosotros; sin embargo, este lenguaje diferente con otros conceptos no determinaría necesariamente que estas personas no serían capaces de apreciar en la realidad cosas que nosotros somos capaces de apreciar.\footnote{\cite[Cf.~][115]{anscombe1981parmenides:qli}: Essence is expressed by grammar. But we can conceive of different concepts, i.e. of language without the same grammar. People using this would then not be using language whose grammar expressed the same essences. However, they might not thereby be missing anything that we realize.}

Anscombe nota: \blockquote[{\cite[115]{anscombe1981parmenides:qli}}: It is enormously difficult to steer in the narrow channel here: to avoid the falsehoods of idealism and the stupidities of empiricist realism.]{Es enormemente difícil conducirse en el canal estrecho aquí: evitar las falsedades del idealismo y las necedades del realismo empírico.} y propone llanamente: \blockquote[{\cite[116]{anscombe1981parmenides:qli}}: if we want to know wether Wittgenstein is a `linguistic idealist'. We shall ask the question: Does this existence, or this truth, depend upon human linguistic practice? That the \emph{meaning of expressions} is so dependent is evident; that human possesion of concepts is so dependent is not quite so evident.]{si queremos saber si Wittgenstein es un `idealista linguistico'. Hemos de hacer la pregunta: ¿Acaso esta existencia, o esta verdad, depende de la práctica linguística humana? Que el \emph{significado de las expresiones} es de este modo dependiente es evidente; que la posesión humana de conceptos es de tal manera dependiente no es tan evidente.}

Anscombe enfatiza que por \enquote{práctica linguística humana} no se refiere simplemente a producir palabras ordenadas de tal manera que componen una oración pertinente, sino que por práctica linguística entiende todas aquellas actividades dentro de las cuales el uso del lenguaje está entretejido: medir, pesar, dar y recibir, situar en algún lugar correspondiente, realizar movimientos de maneras particulares, y también actuar según la consulta de tablas, calendarios o signos. \footnote{\cite[Cf.~][117]{anscombe1981parmenides:qli}: The competent use of language is \emph{a} criterion for the possession of the concepts symbolized in it, and so we are at liberty to say: to have such-and-such linguistic practices is to have such-and-such concepts. ``Linguistic practice'' here does not mean merely the production of words properly arranged into sentences on occasions which we vaguely call ``suitable''. It is important that it includes activities \emph{other} than the production of language, into which a use of language is interwoven. For example, activities of measuring, of weighing, of giving and receiving and putting into special places, of moving about in a huge variety of ways, of consulting tables and calendars and signs and acting in a way which is connected with that consultation.}

Al examinar el pasaje de \emph{Investigaciones Filosóficas}, citado al principio, ya Anscombe ha establecido que la práctica linguística está relacionada con la existencia de ciertos conceptos, pero que de esto no se sigue que las realidades que son expresadas por estos conceptos dependen en modo alguno del pensamiento o lenguaje humanos. Y hasta ahí no se puede hablar de idealismo. Ahora bien, ¿podría haber lo que podríamos llamar un idealismo parcial? Con esta pregunta, Elizabeth dirige su atención a la lógica como el orden según el cuál los conceptos son empleados. ¿Está determinado por la práctica linguística? Anscombe cita a Kronecker que dice: ``Dios hizo los números enteros, lo demás es construcción humana'', ¿a qué se refiere? Parece sugerir que hay una parte del orden lógico que es dado por la naturaleza, y otra que es invención humana. ¿Cómo se puede describir esto?

Anscombe se fija en que hay un cierto orden de necesidad que está establecido por la práctica linguística: \blockquote[{\cite[118]{anscombe1981parmenides:qli}}: But there are, of course, a great many things whose existence does depend on human linguistic practice. The dependence is in many cases an unproblematic and trivial fact. But in others it is not trivial\,---\,it touches the nerve of great philosophical problems. The cases I have in mind are three: namely rules, rights and promises.]{hay, desde luego, una gran cantidad de cosas cuya existencia sí depende de la práctica linguística humana. La dependencia es en muchos casos un dato no problemático y trivial. Pero en otros no es trivial\,---\,sino que toca el nervio de grandes problemas filosóficos. Los casos que tengo en mente son tres: a saber, reglas, derechos y promesas.} Estos tres casos tienen asociados un cierto uso de nociones modales, es decir hay un \enquote{tener que} relacionado con ellos: de acuerdo a las \emph{reglas} de un juego o procedimiento hay ciertas acciones que tienen que ser hechas y otras que no deben hacerse, cuando alguien tiene el \emph{derecho} de hacer algo no se le puede detener, si se ha establecido un \emph{contrato} se debe de cumplir esto o no se debe hacer algo en contra de esto. Es posible pensar en distintas prácticas que son definidas por estas reglas y que no representan ninguna dificultad, sin embargo ¿qué ocurre en el caso de las reglas de la lógica? ¿Dependen de la práctica linguística?

Si alguien cambia las reglas de un juego, o de un baile, se diría que ha construido una variante, \enquote{esto ya no es ajedrez, sino otro juego}. ¿Se puede decir lo mismo de la lógica? ¿Se pueden construir variantes usando otras reglas? Para responder a esto hay que pensar en estas reglas como siendo puestas en práctica, entonces, ¿de acuerdo a qué reglas se hace esta deducción, esta transición desde reglas dadas a prácticas particulares? Anscombe destaca que: \blockquote[{\cite[121]{anscombe1981parmenides:qli}}: Always there is the logical \emph{must}: you can't have this \emph{and} that; you can't do that if you are going by this rule; you must grant this in face of that. And just as ``You can't move your king'' is the more basic expression for one learning chess, since it lies at the bottom of his learning the concept of the game and its rules, so these ``You must's'' and ``You cant's'' are the more basic expressions in logical thinking. But they are not what Hume calls ``naturally intelligible''\,---\,that is to say, they are not expressions of perception or experience. They are understood by those of normal intelligence as they are trained in the practices of reasoning.]{Siempre está ahí el \emph{tener que} lógico: no puedes tener esto \emph{y} aquello; no puedes hacer eso si estás siguiendo esta regla; tienes que conceder esto teniendo en cuenta esto otro. Y así como ``No puedes mover tu rey'' es la expresión más básica para alguien que está aprendiendo ajedrez, puesto que está en el fondo de su aprendizaje del concepto del juego y sus reglas, así estos ``Tienes que'' y ``No puedes'' son las expresiones más basicas en el pensamiento lógico. Pero estas no son lo que Hume llama ``naturalmente inteligible''\,---\,es decir, estas no son expresiones de percepción o experiencia. Son entendidas por aquellos de inteligencia ordinaria al ser adiestrados en las prácticas de razonar.}

Entonces, ``¿Es esta verdad, esta existencia, el producto de la práctica linguistica humana?''. Anscombe ha dado ya una respuesta parcial a su pregunta; en el caso de las realidades que quedan expresadas en el uso del lenguaje, conceptos como un caballo, los colores o las figuras, estos no son producto de la práctica linguística; ni de hecho, ni en la filosofía de Wittgenstein. Y entonces ¿qué de las necesidades metafísicas que pertenecen a la naturaleza de estas cosas? ¿Dependen de la practica linguistica en la filosofía de Wittgenstein? Parece que para Wittgenstein estas dependen de las reglas gramáticas que ordenan la práctica linguística. En \emph{Investigaciones Filosóficas} \S372 sugiere que el correlato en el lenguaje de las necesidades de la naturaleza, es decir, de las posibilidades determinadas al objeto por su naturaleza, son las arbitrarias reglas de la gramática. Se refiere a estas como arbitrarias puesto que no responden a ninguna realidad específica.\footnote{\cite[Cf.~][121]{anscombe1981parmenides:qli}: ``Is this truth, this existence, the product of human linguistic practice?'' This was my test question. I should perhaps have divided it up: Is it so actually? Is it so according to Wittgenstein's philosophy? Now we have partial answers. Horses and giraffes, colours and shapes\,---\,the existence of these is not such a product, either in fact or in Wittgenstein. But the metaphysical necessities belonging to the nature of such things\,---\,these \emph{seem} to be regarded by him as `grammatical rules'. ``Consider `The only correlate in language to a necessity of nature is an arbitrary rule. It is the only thing one can milk out of a necessity of nature into a proposition'''} Junto a esto, en casos particulares Wittgenstein da la impresión de sotener que algo que aparece como una necesidad metafísica es una proposición gramatical.\footnote{\cite[Cf.~][122]{anscombe1981parmenides:qli}: He always seemed to say in particular cases that something that appears as a metaphysical necessity is a proposition of grammar. Is grammar `arbitrary'?} Uno de sus ejemplos: \enquote{Toda vara tiene longitud.}

Anscombe vuelve a enfatizar que es necesario entender por práctica linguística algo más que el ordenar palabras en oraciones y mencionarlas en contextos apropiados.

Estas afirmac
Anscombe examines W's attitude to logic.

\blockquote[{\cite[122]{anscombe1981parmenides:qli}}: The dependence of logical possibility on grammar, and the arbitrarines that then seems to belong to what is counted as logically possible, are canvassed in the following passage: ``If a proposition is conceived as a picture of a possible state of affairs and said to show its possiblity, still it can at most do what a painting or relief or film does: and so at any rate it can't put there what is not the case. (I take this to mean: what is not the case, if what it represents \emph{is} the case.) So does it depend wholly on our grammar what will be called (logically) possible and what not\,---\,i.e. what that grammar permits?''\,---But that is surely arbitrary!\,---Is it arbitrary?\,---It is not every sentence-like formation that we know how to do something with, not every technique has its application in our life; and when we are tempted in philosophy to count some quite useless thing as a proposition, that is often because we have not considered its application sufficiently (PI, I, \S520).]{La dependencia de la posibilidad lógica en la gramática, y la arbitrariedad que desde ahí parece pertencer a lo que puede ser contado como lógicamente posible, quedan exploradas en el siguiente pasaje: ``Si una proposición es concebida como la imagen de un posible estado de las cosas y se dice que muestra su posibilidad, aún así podría lograr como mucho lo que una pintura o un relieve o un filme hace: y entonces en cualquier caso no podría establecer lo que no es de hecho. (Interpreto esto como: lo que no es de hecho, si lo que está representado \emph{es} de hecho.) ¿Entonces depende completamente en nuestra gramática qué puede llamarse (lógicamente) posible y qué no\,---\,a saber, lo que esa gramática permite?''}

\blockquote[{\cite[122]{anscombe1981parmenides:qli}}: However this may be if there is such a thing as idealism about rules and about the necessity of doing \emph{this} if you are to be in conformity with \emph{this} rule, then here Wittgenstein was a linguistic idealist. He insists that these things are the creation of human linguistic practice. To repeat, this does not mean just the practices of arranging words together and uttering them in appropriate contexts. It refers to e.g. \emph{action} on the rule; actually going \emph{this} way by the signpost.]{En cualquier caso si hay alguna cosa como idealismo acerca de reglas y acerca de la necesidad de hacer \emph{esto} si se va a estar en confromidad con \emph{esta} regla, entonces aquí Wittgenstein es un idealista linguístico. El insiste que estas cosas son la creación de la práctica humana linguística. Para repetir, esto no significa solo las prácticas de ordenar palabras y decirlas en contextos apropiados. Se refiere a por ejemplo \emph{acción} desde una regla; actualmente yendo \emph{de esta} manera según el letrero.}

La existencia de conceptos humanos puede ser generalmente igualada con la existencia de una gran variedad de practicas linguisticas, pero eso no implica para nada ninguna dpendencia en el pensamiento y lenguaje humano, en la prate de las cosas que caen bajo estos conceptos.

peero si hay cosas que dependen

si hay algo como idealismo acerca de reglas y sobre la necesidad de hacer esto si vas a estar en conformidad con esta regla, entonces aquí W. era un idealista linguístico

esto se refiere a acción desde la regla

thinking means drawing the conclusion as one must

what would he say about illogical thinking?

So In all this Anscombe did see a sort of linguistic idealism NO!!!! p131 y 118

\blockquote[{\cite[122]{anscombe1981parmenides:qli}}: the arrows and their interpretations await action: what one actually does, which is counted as what was meant: \emph{that} is what fixes the meaning: And so it is about following the rules of correct reasoning. One draws the conclusion as one `must'. That is what ``thinking'' means (RFM I, 131).]{las flechas y sus interpretaciones esperan acción: lo que hacemos de hecho, eso es lo que cuenta como lo que se quiso significar: \emph{eso} es lo que fija el significadoL y así es acerca de seguir las reglas del razonamiento correcto. Sacamos la conclusion así como `debemos'. Eso es lo que ``pensar'' significa (RFM I, 131).}

\blockquote[{\cite[122]{anscombe1981parmenides:qli}}: If so, then what will Wittgenstein say about `illogical' thinking? As I would, that it isn't thinking?]{Si esto es así, entonces ¿qué diría Wittgenstein sobre el pensamiento `ilógico'? ¿Como diría yo, que no es pensar?}

\blockquote[{\cite[122]{anscombe1981parmenides:qli}}: At the Moral Science Club he once quoted a passage from St Augustine about God which with the characteristic rhetoric of St Augustine sounded contradictory, Wittgenstein even took ``he moves without moving'' as a contradcition in intent, and was impatient being told that that at least was not so, the first ``moves'' being transitive and the second intransitive (\emph{movet, non movetur}).]{En una ocasión citó en el \emph{Moral Science Club} un pasaje de San Agustín acerca de Dios el cual con la retórica característica de San Agustín sonaba contradictorio, Wittgenstein incluso tomó ``mueve sin moverse'' como una contradicción de propósito, y se mostró impaciente al decírsele que eso al menos no era así, el primer ``mueve'' siendo transitivo y el segundo intransitivo (\emph{movet, non movetur}).}

\blockquote[{\cite[122]{anscombe1981parmenides:qli}}: He wished to take the contradiction as seriously intended and at the same time to treat it with respect.]{Él deseaba tomar la contradicción como seriamente intencional y al mismo tiempo quería tratarla con respeto.}

\blockquote[{\cite[122]{anscombe1981parmenides:qli}}: This was connected with his dislike of rationality or would-be rationality in religion. He would describe this with a characteristic simile: there is something all jagged and irregular, and some people have a desire to encase it in a smooth ball: looking within you see the jagged edges and spikes, but a smooth surface has been constructed. He preferred it left jagged. I don't know how to distribute this between philosophical observation on the one hand and personal reaction on the other.]{Esto estaba conectado con su desagrado de la racionalidad o potencial racionalidad de la religión. Describía esto con un símil característico: hay algo todo escarpado e irregular, y algunas personas tienen el deseo de encerrarlo en una esfera lisa: mirando dentro de ella se pueden ver las espinas e irregularidades, pero una superficie lisa ha sido construida sobre estas. Él prefería que se dejara escarpado. No se como distribuir esto entre observación filosófica por una parte y reacción personal por otra.}

\blockquote[{\cite[122]{anscombe1981parmenides:qli}}: In the Catholic faith, certain beliefs (such as the Trinity, the Incarnation, the Eucharist) are called ``mysteries''; this means at the very least that it is neither possible to demonstrate them nor possible to show once for all that they are not contradictory and absurd. On the other hand contradiction and absurdity is not embraced; ``This can be disproved, but I still believe it'' is not an attitude of faith at all. So ostenisble proofs of absurdity are assumed to be rebuttable, each one in turn.]{En la fe católica, ciertas creencias (como la Trinidad, la Encarnación, la Eucaristía) son llamadas ``misterios''; esto significa en el mejor de los casos que ni es posible demostrarlas ni tampoco es posible mostrar de una vez por todas que no son contradictorias y absurdas. Por otra parte la contradicción y lo absurdo no son abrazados; ``Esto puede ser refutado, pero aún así lo creo'' no es para nada una actitud de fe. Entonces las ostensibles demostraciones de absurdidad son asumidas como rebatibles, cada una en su turno.}

\blockquote[{\cite[122]{anscombe1981parmenides:qli}}: Now this process Wittgenstein himself once described: ``You can ward off \emph{each} attack as it comes'' (Personal Conversation).]{Ahora, este proceso Wittgenstein mismo lo describió en una ocasión: ``Puedes mantener a raya \emph{cada} ataque según venga'' (Conversación personal).}

\blockquote[{\cite[122--123]{anscombe1981parmenides:qli}}: But the attitude of one who does that, or wishes that that should be done, is not that of willingness to profess contradiction. On the contrary. On the other hand, religious mysteries are not a theory, the product of reasoning; their source is quite other. Wittgenstein's attitude to the whole of religion in a way assimilated it to the mysteries: thus he detested natural theology. But again, what part of this was philosophical (and therefore something which, if right, others ought to see) and what personal, it is difficult to say.]{Pero la actitud de uno que hace esto, o que desea que eso se haga, no es la de una disposición a profesar la contradicción. Al contrario. Por otra parte, los misterios religiosos no son una teoría, el producto del razonamiento; su fuente es totalmente otra. La actitud de Wittgenstein a el todo de la religión la asimilaba en cierto modo a los misterios: por consiguiente detestaba la teología natural. Pero de nuevo, qué parte de esto era filosófico (y por tanto algo que, si correcto, otros han de ver) y qué parte era personal, es difícil decir.}

\blockquote[{\cite[123]{anscombe1981parmenides:qli}}: In natural theology there is attempted reasoning from the objects of the world to something outside the world. Wittgenstein certainly worked and thought in a tradition for which this was impossible.]{En la teología natural hay un intento de razonamiento desde los objetos del mundo a algo fuera del mundo. Wittgenstein ciertamente trabajó y pensó en una tradición para la cual esto era imposible.}
