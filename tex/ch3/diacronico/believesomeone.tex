\subsection{What is it to believe someone? (1979)}

\emph{What is it to believe someone?} fue originalmente publicado en 1979 en \emph{Rationality and Religious Belief} junto a otros siete ensayos. Sobre esta colección, editada por C.\,F.\,Delaney, el comentario escrito por Robert Masson para la revista \emph{Horizon}, tenía esto que decir: \blockquote[{\Cite[440]{masson1981}}: \enquote{Delaney promises that the eight original essays he has collected \textelp{} contribute to the ongoing discussion in the philosophy of religion in basically two ways: they demonstrate that the question about the rationality of religious belief is ``as much about rationality as about religion,'' and they show why people raising this question ought to examine religion ``concretely as a human practice rather than abstractly as a system of propopsitions''}.]{Delaney promete que los ocho ensayos originales que ha agrupado \textelp{} contribuyen a la discusión en curso en la filosofía de la religion básicamente de dos maneras: demuestran que la cuestión acerca de la racionalidad del creer religioso es ``tanto sobre racionalidad como sobre religión,'' y muestran por qué las personas que proponen esta pregunta deben examinar la religión ``concretamente como una práctica humana más que abstractamente como un sistema de proposiciones''}. En su ensayo, Anscombe considera el papel que la `fe humana' juega en nuestro conocimiento y \blockquote[{\Cite[xvii]{anscombe2008faith}}: \enquote{This problem, of what it is to believe \emph{someone}, which we do all the time, is obviously one which is interesting independently of questions having to do with divine faith}.]{Este problema, acerca de qué es creer a \emph{alguien}, que hacemos todo el tiempo, es obviamente uno que es interesante independientemente de las preguntas que tienen que ver con la fe divina}.

  Anscombe comienza su análisis con una sugerencia algo extraña. Propone un escenario construido según un patrón argumental\footnote{El patrón de argumento al que aquí se refiere es estudiado con más detalle por Anscombe en \cite{anscombe2015logic:qpa}.} que tiene la peculiaridad de que la conjunción de sus premisas no es suficiente para justificar la creencia expresada en la conclusión y, por tanto, no puede ser valorada como conocimiento o juicio razonable si no se tiene en cuenta otro fundamento externo. Dicho de otra manera, el escenario es una ilustración de un caso en el que la creencia depositada en lo que alguien dice no tiene como fundamento la combinación de las premisas, sino un elemento o circunstancia externa. En la escena cada premisa aparece atribuida a una persona distinta y la conclusión a un cuarto personaje, el pequeño relato aparece como sigue: \blockquote[{\Cite[1]{anscombe2008faith:tobelieve}}: \enquote{There were three men, $A$, $B$ and $C$, talking in a certain village. $A$ said ``If that tree falls down, it'll block the road for a long time.'' ``That's not so if there's a tree-clearing machine working'', said $B$. $C$ remarked ``There \emph{will} be one, if the tree doesn't fall down.'' The famous sophist Euthydemus, a stranger in the place, was listening. He immediately said ``I believe you all. So I infer that the tree will fall and the road will be blocked.''}.]{Había tres hombres, $A$, $B$ y $C$, hablando en cierta aldea. $A$ dijo: ``Si ese árbol cae, interrumpirá el paso por el camino durante mucho tiempo.'' ``No será así si una máquina para remover árboles está funcionando'', dijo $B$. $C$ destacó: ``\emph{Habrá} una, si el árbol no cae.'' El famoso sofista Eutidemo, un extraño en el lugar, estaba escuchando. Inmediatamente dijo: ``Les creo a todos. Así que infiero que el árbol caerá e interrumpirá el paso por el camino.''}

  ¿En qué se equivoca Eutidemo? Si se evalúa la lógica del argumento no aparece ninguna contradicción, sin embargo hay algo extraño en la afirmación \enquote*{les creo a todos}. Si la lógica del argumento parece permitir que la inferencia de Eutidemo sea posible, ¿por qué suena tan extraña la posibilidad de que les crea a todos y juzgue esa conclusión?

  Pero antes, sin embargo, quizás surge la inquietud: ¿creer a alguien?, ¿acaso no hacemos eso todo el tiempo?, ¿merece la pena atender esta cuestión filosóficamente? Anscombe piensa que sí, y espera mostrar que es un tema de gran importancia para la vida y la filosofía y que además representa suficiente dificultad como para merecer investigación filosófica.

Elizabeth desarrollará su ensayo como una investigación sobre la expresión \enquote*{creer a $x$ que $p$}. Antes de la investigación propone dos nociones a modo de preámbulo. En primer lugar plantea: \blockquote[{\Cite[1]{anscombe2008faith:tobelieve}}: \enquote{If words always kept their old values, I might have called my subject `Faith'. That short term has in the past been used in just this meaning, of believing someone}.]{Si las palabras siempre guardaran sus antiguos valores, podría haber llamado mi tema `Fe'. Este corto término ha sido usado en el pasado justo con este significado, el de creer a alguien}. Con esto Anscombe no pretende simplemente rescatar esta antigua acepción del término, sino que al hacer referencia a este modo de hablar establece varias conexiones entre lo que la fe implica y lo que es creer a alguien o el uso de la expresión `creer' con un objeto personal. Trata la expresión como `fe humana'. Esto también tiene como consecuencia que tanto el análisis de la `fe divina' se ve enriquecido por la comprensión sobre lo que significa creer a alguien, como que el análisis de lo que significa creer a alguien se beneficia del uso que hacemos de la expresión `fe'. En este punto Elizabeth insiste. La discusión sobre la fe divina pierde mucho cuando se abandona esta acepción del término como creer a Dios. \enquote{En esta época}, dice, \blockquote[{\Cite{anscombe2008faith:tobelieve}}: \enquote{Nowadays it is used to mean much the same thing as `religion' or possibly `religious belief'. Thus belief in God would now generally be called `faith'\,---\,belief in God at all, not belief that God will help one for example}.]{se usa para decir básicamente lo mismo que `religión' o posiblemente `creencia religiosa'. Así creer en Dios se llamaría ahora generalmente `fe'\,---\,creer en Dios del todo, no creer que Dios nos ayuda por ejemplo}. La consecuencia es que se ha perdido cierta riqueza: \blockquote[{\Cite{anscombe2008faith:tobelieve}}: \enquote{This is a great pity. It has had a disgusting effect on thought about religion. The astounding idea that there should be such a thing as \emph{believing God} has been lost sight of}.]{Esto es una gran lástima. Ha tenido un efecto desagradable en el pensamiento sobre la religión. La asombrosa idea de que existe tal cosa como \emph{creer a Dios} se ha perdido de vista}.

Otra consecuencia de esta relación de nociones está en la distinción que permite hacer respecto de \enquote*{creer que $N$ existe}. Esta creencia con Dios como objeto no podría ser llamada `fe divina'. Si se entiende fe como \enquote*{creer a $x$ que $p$} esto puede ser visto claramente, sería extraño decir que creemos a $N$ que $N$ existe. Creer en la existencia de alguien y creerle sobre algo que me comunica son dos modos distintos de creer. La creencia en la existencia de alguien que se comunica tiene que ver con aceptar la comunicación como aquello que pretende ser: una comunicación de $N$. La creencia en lo comunicado sería entonces creer a $N$ que $p$.

La segunda idea que se presenta en el preámbulo tiene que ver con la pregunta \enquote*{¿Cómo accedemos a una idea del mundo más allá de nuestra experiencia personal?} Hume diría que el puente que permite nuestro contacto con la realidad más allá de nuestra experiencia es la relación causa-y-efecto.\footnote{\cite[Cf.][3]{anscombe2008faith:tobelieve}: \enquote{Hume thought that the idea of cause-and-effect was the bridge enabling us to reach any idea of a world beyond personal experience}.} Inferimos las causas desde sus efectos porque estamos acostumbrados a ver que causa y efecto van juntas. Estas causas inferidas las verificamos en la percepción inmediata de nuestra memoria o nuestros sentidos, o por medio de la inferencia de otras causas verificadas del mismo modo.\footnote{\cite[Cf.][87]{anscombe1981parmenides:humeandjulius}: \enquote{For Hume, the relation of cause and effect is the only bridge by which to reach belief in matters beyind our present impressions or memories}.} Hume entonces propone que la relación entre el testimonio y la verdad es de la misma clase, inferimos la verdad del testimonio porque estamos acostumbrados a que vayan juntas.\footnote{\cite[Cf.][3]{anscombe2008faith:tobelieve}:\enquote{ We believe in a cause, he tought, because we perceive the effect and cause and effect have been found to always go together. Similarly we believe in the truth of testimony because we perceive the testimony and we have (well! often have) found testimony and truth to go together!}}

Anscombe tacha de absurda esta visión del rol del testimonio en el conocimiento humano y le parece que \blockquote[{\Cite[Cf.][3]{anscombe2008faith:tobelieve}}: \enquote{the mystery is how Hume could ever have entertained it}.]{el misterio es cómo Hume la pudo haber llegado a sostener}. Entonces explica: \blockquote[{\Cite[3]{anscombe2008faith:tobelieve}}: \enquote{We must acknowledge testimony as giving us our larger world in no smaller degree, or even in a greater degree, than the relation of cause and effect; and believing it is quite dissimilar in structure from belief in causes and effects. Nor is what testimony gives us entirely a detachable part, like the thick fringe of fat on a chunk of steak. It is more like the flecks and streaks of fat that are distributed through good meat; though there are lumps of pure fat as well}.]{Hemos de reconocer que el testimonio nos da nuestro mundo más grande en grado no menor, o incluso en un grado mayor, que la relación de causa y efecto; y creerlo es bastante distinto en su estructura al creer en causas y efectos. Tampoco lo que el testimonio nos da es una parte que se puede desprender completamente, como el borde de grasa en un pedazo de filete. Es más bien como las manchas y rayas de grasa que están distribuidas a través de la buena carne; aunque hay nudos de pura grasa también} Elizabeth considera que la mayor parte de nuestro conocimiento de la realidad está apoyado en la creencia que tenemos en las cosas que se nos han enseñado o dicho. Estas creencias, maduradas a lo largo del tiempo, van componiendo una imagen del mundo y un sistema de conocimiento. Para ella, la investigación acerca de `creer a alguien' no solo es del interés de la teología o de la filosofía de la religión, sino de enorme importancia para la teoría del conocimiento.

Con esto ya estamos a la puerta de la investigación de Elizabeth. El objetivo propuesto es profundizar en una descripción más acertada sobre la `estructura del creer en el testimonio' como distinta de la inadecuada relación causa y efecto. Esta descripción será un análisis de \enquote*{creer a $x$ que $p$} entendido como `fe humana'.

\label{subsec:presups}
\blockquote[{\Cite[4]{anscombe2008faith:tobelieve}}: \enquote{If you tell me `Napoleon lost the battle of Waterloo' and I say `I believe you' that is a joke}.]{Si me dijeras `Napoleón perdió la batalla de Waterloo' y te digo `te creo' sería una broma}. De primera impresión \enquote*{creer a $x$ que $p$} parece que significa simplemente creer lo que alguien me dice, o creer que lo que me dice es verdadero. Sin embargo esto no es suficiente. Puede ser que ya crea lo que alguien me venga a decir. Puede ser que la comunicación suscite que forme mi propio juicio acerca de la verdad comunicada, pero aquí no podría decir que estoy creyendo al que comunica o que estoy contando con él para mi creer que $p$.

¿Entonces creer a alguien es creer algo apoyado en el hecho de que lo ha dicho? \blockquote[{\Cite[4]{anscombe2008faith:tobelieve}}: \enquote{A witness might be asked `Why did you think the man was dying?' and reply `Because the doctor told me' \textelp{} If asked further what his own judgement was, he may reply `I had no opinion of my own\,---\,I just believed the doctor'}.]{Puede que se le pregunte a un testigo `¿Por qué pensó que aquel hombre se estaba muriendo?' y que este responda `Porque el doctor me lo dijo' \textelp{} `no me hice ninguna opinión propia\,---\,yo solo creí al doctor'}. Este puede ser un ejemplo de contar con $x$ para la verdad de $p$. Esto, sin embargo, tampoco parece ser suficiente. Puedo imaginar el caso en el que esté convencido de que alguien a la vez cree lo opuesto a la verdad de $p$ y quiera mentirme. Según este cálculo podría decir que creo en lo que ha dicho por el hecho de que me lo ha dicho, pero no estaría diciendo que le creo a él.

¿Qué se puede decir del \enquote*{les creo a todos} de Eutidemo en la cuestión preliminar? Anscombe juzga que la exclamación no expresa simplemente una opinión apresurada o excesiva credulidad, sino más bien suena a locura\footnote{\cite[Cf.][5]{anscombe2008faith:tobelieve}: \enquote{\emph{insane} is just what Euthydemus' remark is and sounds\,---\,it is not, for example, like the expression of a somewhat rash opinion, or of excessive credulity}.}. Eutidemo no puede estar diciendo la verdad cuando dice que les cree a todos. La expresión de $C$ da pertinencia a lo que dice $B$, y la manera natural de entender lo que dice $B$ es arrojar duda sobre lo que $A$ ha dicho. ¿Se puede pensar que $A$ todavía cree lo que ha dicho inicialmente? ¿Eutidemo puede creer a $A$ sin saber cuál es su reacción a lo que $B$ y $C$ han dicho? Entonces Anscombe concluye, \blockquote[{\Cite[5]{anscombe2008faith:tobelieve}}: \enquote{To believe $N$ one must believe that $N$ himself believes what he is saying}.]{Para creer a $N$ uno debe creer que $N$ mismo cree lo que está diciendo}. Creer a $N$ sin saber si $N$ cree lo que dice le suena a Elizabeth como una locura.

En este punto Anscombe fija su atención en las otras creencias involucradas en el `creer a $x$ que $p$'. Para esto atrae nuestra atención sobre el hecho de que con frecuencia lo que tenemos ante nosotros es la comunicación y no al que habla, como cuando leemos un libro. Si se tiene esto en cuenta también, es posible ver mejor cómo `creer a $x$ que $p$' conlleva otras creencias. Estas son presuposiciones relacionadas con la comunicación y en circunstancias ordinarias no tienen por qué ser dudosas, pero están implicadas en el llegar a plantearse si creer o no una comunicación recibida.

En primer lugar, si se cree a alguien, tiene que ser el caso que se cree que una comunicación es de alguien\footnote{\cite[Cf.][6]{anscombe2008faith:tobelieve}: \enquote{futher beliefs that are involved in believing someone. First of all, it must be the case that you believe that something is a communication from him (or `from someone')}.}. Esta presuposición no parece tan problemática si se piensa en las ocasiones en las que creemos a alguien que es percibido. Aquí resulta útil la consideración de los casos en los que recibimos la comunicación sin que esté presente el que habla\footnote{\cite[Cf.][5]{anscombe2008faith:tobelieve}: \enquote{often all we have is the communication without the speaker}.}. Al respecto, podríamos imaginar una situación problemática. Supongamos que alguien recibe una carta en la que el autor no es el comunicador ostensible o aparente, es decir, quien firma la carta no es quien la ha escrito. ¿Se puede decir que el que recibe la carta cree o no cree al autor o al comunicador ostensible? Creer al autor, afirma Anscombe, conlleva un tipo de juicio y especulación que no son mediaciones ordinarias en el creer a alguien\footnote{\cite[Cf.][7]{anscombe2008faith:tobelieve}: \enquote{This case, where there is intervening judgement and speculation, should alert us to the fact that in the most ordinary cases of believing someone there is no such mediation}.}. Para decir que creo al autor tendría que discernir que la comunicación que viene bajo otro nombre es realmente de esta otra persona que además me quiere decir esto.

Acerca de la posibilidad de decir que se cree al comunicador ostensible Anscombe distingue entre un comunicador ostensible que exista o no. Ante una comunicación que viene de parte de un comunicador aparente que no existe, alguien puede responder diciendo que cree o no cree al comunicador aparente, pero la decisión de decir esto ---dice Anscombe--- \blockquote[{\Cite[7]{anscombe2008faith:tobelieve}}: \enquote{is a decision to give those verbs an `intentional' use like the verb `to look for'}. Anscombe propone que un verbo es usado intencionalmente cuando tiene como objeto directo un `objeto intencional' (`objeto' no en el sentido material, sino de finalidad) en: {\cite[9]{anscombe1981metaphysics:intsens}, lo describe como sigue: \enquote{We must ask: does any phrase that gives the direct object of an intentional verb in a sentence necessarily give an intentional object? No. Consider: ``These people worship Ombola; that is to say, they worship a mere hunk of wood.'' (cf. ``They worship sticks and stones.'') Or ``They worship the sun, that is, they worship what is nothing but a great mass of frightfully hot stuff.'' The worshippers themselves will not acknowledge the descriptions. Their idol is for them a divinized piece of wood, one that is somehow also a god; and similarly for the sun. An intentional object is given by a word or a phrase which gives a \emph{description under which}}}.]{es una decisión de dar a estos verbos un uso `intencional', como el verbo `ir tras'}. Esto lo ilustra añadiendo: \blockquote[{\Cite[7]{anscombe2008faith:tobelieve}}: \enquote{And so we might speak of someone as believing the god (Apollo, say), when he consulted the oracle of the god --- without thereby implying that one believed in the existence of the god oneself. All we want is that we should know what is called the god's telling him something}.]{Y así uno podría hablar de alguien en cuanto que cree al dios (Apolo, digamos), cuando consultó el oráculo del dios --- sin que por esto uno estuviera implicando que uno mismo cree en la existencia del dios. Todo lo que queremos es que tendríamos que saber aquello que se denomina que el dios le diga algo}. `Creer' usado aquí intencionalmente viene a decir que se busca o se desea creer a $x$ (Apolo en este caso) cuando se recibe aquello que alguien entiende como una comunicación suya.

En el caso de que el comunicador aparente sí exista, la noción de creerle manifiesta una cierta oscilación que depende de que la expresión `creer' se use en primera, segunda o tercera persona. Una tercera persona podría decir que \enquote*{aquel, pensando que $N$ dijo esto, le creyó}, o el comunicador aparente puede decir \enquote*{veo que pensaste que fui yo quien dijo esto y me creíste}, sin embargo, si el que ha recibido la comunicación dijera \enquote*{naturalmente te creí}, el comunicador aparente podría contestar \enquote*{ya que no lo he dicho yo, no me estabas creyendo a mi}.

Estas consideraciones llevan a Anscombe a distinguir entre el que habla en una comunicación y el productor inmediato de la comunicación. Este puede ser cualquiera que transmita alguna comunicación, un maestro o mensajero, o un interprete o traductor; este es \blockquote[{\Cite[8]{anscombe2008faith:tobelieve}}: \enquote{we can speak of the immediate producer of what is taken, or makes an internal claim to be taken, as a communication from $NN$}.]{el productor inmediato de aquello que se entiende, o incluye una reclamación interna de ser entendido como una comunicación de $NN$}. Si digo que creo a un intérprete estoy afirmando que creo lo que ha dicho su principal, y mi contar con el intérprete consiste en la creencia de que ha reproducido lo que aquel ha dicho. En este sentido al intérprete no le falta rectitud si dice algo que no es verdadero pero no ha representado falsamente lo que ha dicho su principal. Por el contrario, al maestro sí le faltaría rectitud si lo que dice no es verdadero. Cuando se cree al maestro, aún en el caso que no sea de ninguna manera autoridad original de lo que comunica, se le cree a él sobre lo que transmite. Para Anscombe no es necesario que cuando se cree a alguien se le trate como una autoridad original\footnote{\cite[Cf.][5]{anscombe2008faith:tobelieve}: \enquote{To believe a person is not necessarily to treat him as an original authority}.}. En esto el ejemplo del maestro como distinto del intérprete es ilustrativo. Un maestro puede conocer lo que enseña porque lo ha recibido de alguna tradición de información y al transmitir lo que enseña se le está creyendo a él.

Asoma aquí otro aspecto relacionado con esta presuposición. Al creer que una comunicación es de alguien se cree a una persona que puede tener distintos grados de autoridad sobre lo que dice. El maestro del que se ha hablado antes podría afirmar \enquote*{Leonardo da Vinci dibujó diseños para una máquina voladora} y en esto no es para nada una autoridad original\footnote{\cite[Cf.][6]{anscombe2008faith:tobelieve}: \enquote{he may not be an original authority at all, as if he says that Leonardo made drawings fo a flying machine. In this latter case he almost certainly knows it from having been told, \emph{even} if he's seen the drawings}.}. Conoce esto porque lo ha escuchado, incluso si ha visto los diseños. Aún cuando los hubiera descubierto él mismo, tendría que haber contado con alguna información recibida de que esos diseños que ve son de Leonardo. En este caso sí seria una autoridad original en notar que estos diseños que ha escuchado que son de Leonardo son de máquinas voladoras. Anscombe explica la distinción diciendo: \blockquote[{\Cite[5]{anscombe2008faith:tobelieve}}: \enquote{He is \emph{an} original authority on what he himself has done and seen and heard: I say \emph{an} original authority because I only mean that he does himself contribute something, e.g. is in some sort a witness, as oposed to one who only transmits information received. But his account of what he is a witness to is very often \textelp{} heavily affected or ratherl all but completely formed by what information \emph{he} had received}.]{\textins{Alguien} es \emph{una} autoridad original en aquello que él mismo ha hecho y visto y oído: digo \emph{una} autoridad original porque solo quiero decir que él mismo sí contribuye algo, es algún tipo de testigo por ejemplo, en lugar de alguien que solo transmite información recibida. Pero su informe de aquello de lo que es testigo es con frecuencia \textelp{} fuertemente influenciado o más bien casi del todo formado por la información que \emph{él} ha recibido} Además de ser \emph{una} autoridad original sobre algún hecho, una persona puede ser una autoridad \emph{totalmente} original. Si la distinción entre alguien que no es una autoridad original y alguien que sí lo es ha sido descrita como la contribución de algo propio que junto con la información recibida permite construir un informe, lo particular de una autoridad totalmente original es que no se apoya en ninguna información recibida para construir su informe de los hechos. Anscombe no entiende el lenguaje como información recibida. Pone como ejemplo de informe de una autoridad totalmente original a alguien que dice \enquote*{esta mañana comí una manzana} y dice: \blockquote[{\Cite[6]{anscombe2008faith:tobelieve}}: \enquote{if he is in the situation usual among us, he knows what an apple is\,---\,i.e. can recognise one. So though he was `taught the concept' in learning to use language in everyday life, I do not count that as a case of reliance on information received}.]{si él está en la situación usual entre nosotros, sabe lo que es una manzana\,---\,es decir, puede reconocer una. Así que aún cuando se le ha `enseñado el concepto' al aprender a usar el lenguaje en la vida ordinaria, no cuento esto como un caso de depender de información recibida}.

Hasta aquí se ha visto que el `creer a $x$ que $p$' implica otras creencias que son presuposiciones a la pregunta sobre si se cree o no se cree a alguien y se ha descrito lo que tiene que ver con la creencia de que una comunicación viene de alguien. Anscombe examina otras presuposiciones más. También tiene que ser el caso que creamos que por la comunicación, la persona que habla quiere decir \emph{esto}. En situaciones ordinarias no es difícil distinguir si alguien está diciendo o escribiendo algún lenguaje. Sin embargo, aún cuando el que habla use palabras que puedo `hacer mías' y creer simplemente las palabras que dice, aquí queda espacio para decir que hay una creencia adicional de que se ha dicho `tal cosa' en la comunicación. Elaboramos aquello que hemos creído y usamos otras palabras distintas, nuestras creencias no están atadas a palabras específicas. También podríamos pensar que alguien diga que cree \emph{esto} porque `cree a $x$' y que se le cuestione su creencia preguntando \enquote*{¿qué entendiste como $x$ diciéndote eso?}\footnote{\cite[Cf.][8]{anscombe2008faith:tobelieve}: \enquote{So when someone says that he believes such-and-such because he believes $NN$, we may say `We suspect a misunderstanding. What did you take as $NN$'s telling you that?'}}.

Otra presuposición sería que se cree que la comunicación está \emph{dirigida} a alguien, aunque sea \enquote*{a quien lea esto} o \enquote*{a quien pueda interesar}. Esta creencia se podría problematizar pensando en algún caso que alguien reciba una comunicación con otro destinatario, ¿estaría creyendo al que se comunica? Anscombe opina que en un sentido extendido o reducido y considera que el tema parece de poca importancia\footnote{\cite[Cf.][7]{anscombe2008faith:tobelieve}: \enquote{Suppose that someone gets hold of written communications, but they are not addressed to him at all, not even meant to reach him. Can he be said to believe the writer if he believes what they tell the addressee? Only in a reduced or extended sense, though the matter is perhaps not onte of any importance}.}.

Una persona a quien se dirige una comunicación puede \emph{fallar en el creerla} si no nota la comunicación, o si notándola no la interpreta como lenguaje, o si notándola como lenguaje no la toma como dirigida hacia ella; o puede que crea todo esto, pero lo interprete incorrectamente, o puede que lo interprete bien pero no crea que viene realmente de $N$. En este tipo de casos la persona no ha dejado de creer, sino que no ha llegado a estar en la situación de plantearse esa pregunta. Para poder llegar a preguntar si alguien `cree a $x$ que $p$' habría que excluir o asumir como excluidos todos los casos en los que estas otras presuposiciones no se han cumplido. En los casos en los que todos estos presupuestos no presentan problema o duda, llegamos a estar en la situación que Anscombe describe a modo de conclusión: \blockquote[{\Cite[9]{anscombe2008faith:tobelieve}}: \enquote{Let us suppose that all the presuppositions are in. $A$ is then in the situation ---a very normal one--- where the question arises of believing or doubting (suspending judgement in face of) $NN$. Unconfused by all the questions that arise because of the presuppositions, we can see that believing someone (in the particular case) is trusting him for the truth\,---\,in the particular case}.]{Supongamos que todas las presuposiciones están dadas. $A$ está entonces en la situación ---muy común--- donde surge la pregunta sobre si creer o dudar (suspender el juicio ante) $NN$. Sin confusión por todas las preguntas que surgen por las presuposiciones, podemos ver que creer a alguien (en el caso particular) es confiar en él para la verdad\,---\,en el caso particular}. Que `$A$ crea a $N$ que $p$' implica que $A$ cree que en una comunicación, que puede venir de un productor inmediato, $N$ es el que habla y lo que dice es $p$ y esta comunicación está dirigida hacia $A$; entonces $A$, creyendo que $N$ cree que $p$, confía en $N$ sobre la verdad de $p$.

Anscombe termina con una cuestión que deja abierta. Tiene que ver con uno de los ejemplos relacionados con creer que la comunicación viene de alguien. Allí proponía imaginar el caso en el que estuviéramos convencidos de que alguien viene a decirnos lo que cree que es falso, pero a la vez sabemos que lo que cree es lo contrario a la verdad. Al decir lo que cree que es falso estaría afirmando la verdad. En ese caso, afirmaba Anscombe, podría decir que creo en lo que dice y además creo porque lo dice, pero no le creo a él. Se podría preguntar ¿cuál es la diferencia entre llegar a la creencia de $p$ porque alguien que está en lo correcto y es veraz me lo ha dicho, y llegar a la misma creencia porque me lo ha dicho alguien que está equivocado y miente? Ambos casos parecen implicar un cálculo, en uno se calcula que está en lo correcto y es veraz y en el otro se calcula que está equivocado y miente. ¿Por qué estamos dispuestos a decir que creemos al que habla solo en el caso en que esté en lo correcto y sea veraz? ¿Acaso no llevan ambos casos a la misma creencia que $p$?

Aquí Anscombe percibe que hay que decir todavía más sobre la prioridad que damos a la rectitud y la veracidad en la dinámica de creer lo que se nos dice sobre la realidad.

Este artículo, junto a \emph{Faith}, sirve como el eslabón que nos permite relacionar la actitud de confianza que Anscombe tiene ante la presencia de la verdad en la actividad del lenguaje y la creencia que llamamos fe como juicio incondicional que implica la confianza en la rectitud de quien se comunica y de su comunicación. En esta reflexión Elizabeth establece esta conexión al hablar de `fe humana' como análoga a `fe divina'. También alude a esta misma relación con sus comentarios finales sobre la concepción de la verdad como teniendo una relación más justificada con la aserción y el creer (Cf. \S\ref{subsec:verdad}, p.~\pageref{subsec:verdad}).
