\subsection{Que se puede entender de la fe sin tenerla}
En Oscott College, el seminario de la Archidiócesis de Birmingham, se comenzaron
a celebrar las conferencias llamadas Wiseman Lectures en 1971. Para estas
lecciones ofrecidas anualmente en memoria de Nicholas Wiseman se invitaba un
ponente que tratara algún tema relacionado con la filosofía de la religión o
alguna materia en torno al ecumenismo.\footcite[Cf.~][7]{wisemanlects}

El 27 de octubre de 1975, para la quinta edición de las conferencias, Anscombe
presentó una lección titulada simplemente \emph{Faith}. Allí planteaba la
siguiente cuestión: \citalitlar{Quiero decir qué puede ser entendido sobre la fe
  por alguien que no la tenga; alguien, incluso, que no necesariamente crea que
  Dios existe, pero que sea capaz de pensar cuidadosa y honestamente sobre ella.
  Bertrand Russell llamó a la fe `certeza sin prueba'. Esto parece correcto.
  Ambrose Bierce tiene una definición en su \emph{Devil's Dictionary}: `La
  actitud de la mente de uno que cree sin evidencia a uno que habla sin
  conocimiento cosas sin parangón'. ¿Qué deberíamos pensar de
  esto?\footnote{\cite[115]{anscombe2008faith:faith}: <<I want to say what might
    be understood about faith by someone who did not have it; someone, even, who
    does not necessarily believe that God exists, but who is able to think
    carefully and truthfully about it. Bertrand Russell called faith `certainty
    without proof'. That seems correct. Ambrose Bierce has a definition in his
    Devil's Dictionary: `The attitude of mind of one who believes without
    evidence one who tells without knowledge things without parallel.' What
    should we think of this?>>}}

\subsection{Descripción de `Fe'}
El objetivo de Elizabeth sitúa su investigación en un contexto específico.
Pretende describir el fenómeno de la fe como uno que tiene un carácter de
razonabilidad tal que se puede \citalitinterlin{pensar cuidadosa y
  honstamente}\footnote{\cite[115]{anscombe2008faith:faith}:<<think carefully
  and truthfully>>}. Su estrategia, carácterística del tipo de análisis empleado
por Wittgenstein, se muestra aquí de nuevo como una descripción de usos
familiares de la palabra siendo analizada que son articulados de tal manera que
los patrones de estos usos sean
estudiables\autocite[Cf.~][12]{bakerhacker2009understanding}. Se enfoca en un
modo antiguo de usar la palabra `fe' en el que se le empleaba para decir `creer
a alguien que $p$'. `Fe humana' era creer a una persona humana, `fe divina' era
creer a Dios\autocite[Cf.~][2]{anscombe2008faith:tobelieve}. Así por ejemplo:
<<Abrám creyó a Dios (\textgreek{ἐπίστευσεν τῷ Θεῷ}) y ésto se le contó como
justicia>>\footnote{Gn~15,6}. De tal modo que es llamado 'padre de la
fe'.\footnote{Cf.~Rm~4~y~Ga 3,7}. Este enfoque hace que la pregunta `¿qué es
creer a alguien?' quede situada en el centro del
análisis\autocite[Cf.~][14]{anscombe2008faith:faith}, y aquí también Anscombe
dedica su atención a las presuposiciones implicadas en esta creencia.

Pueden ser destacados tres movimientos principales en el análisis realizado por
Elizabeth en esta investigación. Primero establece una relación entre las
presuposiciones implicadas en el creer y lo que se ha llamado los preámbulos de
la fe. En segundo lugar describe lo relacionado a las presuposiciones implicadas
en creer a una persona humana cuando ésta comunica algo. En tercer lugar examina
el fenómeno particular de creer cuando la comunicación viene de Dios.

\subsubsection{Las presuposiciones del creer como descripción de la
  razonabilidad de la fe}
A lo largo de la investigación, Anscombe recurrirá a una descripción de las
presuposiciones implicadas en el creer como una descripción razonable de la fe.
Su apoyo para seguir esta ruta de análisis es el recuerdo de cierta discusión,
de cierta apologética\autocite[Cf.~][13]{anscombe2008faith:faith}. Trae a la
memoria que: \citalitlar{Hubo en una época pasada un profuso entusiasmo por la
  racionalidad, quizás inspirado por la enseñanza del Vaticano~I contra el
  fideísmo, ciertamente sostenidos por la promoción de estudios neo-tomístas
  [\ldots] la noticia era que la fe Cristiana Católica era \emph{racional}, y el
  problema, para aquellos capaces de sentirlo como tal, era cómo era
  \emph{gratuita} \footnote{\cite[11]{anscombe2008faith:faith}: <<There was in a
    preceding time a professed enthusiasm for rationality, perhaps inspired by
    the teaching of Vatican I against fideism, certainly carried along by the
    promotion of neo-thomist studies [\ldots] the word was that the Catholic
    Christian faith was \emph{rational}, and a problem, to those able to feel it
    as a problem, was how it was \emph{gratuitous}>>}.}

Distintas variantes de esta enseñanza ---destaca Anscombe--- ofrecían distintas
argumentaciones, algunas más sobrias que otras, que servían como procesos de
razonamientos que ofrecían una cierta demostración de la verdad de las
enseñanzas de la Iglesia\autocite[Cf.~][12]{anscombe2008faith:faith}.

\subsubsection{Las presuposiciones implicadas en creer a alguien}

\subsubsection{`Creer' cuando la comunicación viene de Dios}
