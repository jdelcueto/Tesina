\subsection{Faith (1975)}

En \emph{Oscott College}, el seminario de la Archidiócesis de Birmingham, se comenzaron a celebrar las conferencias llamadas \emph{Wiseman Lectures} en 1971. Para estas lecciones ofrecidas anualmente en memoria de Nicholas Wiseman se invitaba un ponente que tratara algún tema relacionado con la filosofía de la religión o alguna materia en torno al ecumenismo\footnote{\cite[Cf.~][7]{wisemanlects}}.

El 27 de octubre de 1975, para la quinta edición de las conferencias, Anscombe presentó una lección titulada simplemente \emph{Faith}. Allí planteaba la siguiente cuestión: \blockquote[{\cite[115]{anscombe1981erp:faith}}: \enquote{I want to say what might be understood about faith by someone who did not have it; someone, even, who does not necessarily believe that God exists, but who is able to think carefully and truthfully about it. Bertrand Russell called faith `certainty without proof'. That seems correct. Ambrose Bierce has a definition in his Devil's Dictionary: `The attitude of mind of one who believes without evidence one who tells without knowledge things without parallel.' What should we think of this?}]{Quiero decir qué es lo que puede ser entendido sobre la fe por alguien que no la tenga; alguien, incluso, que no necesariamente crea que Dios existe, pero que sea capaz de pensar cuidadosa y honestamente sobre ella. Bertrand Russell llamó a la fe `certeza sin prueba'. Esto parece correcto. Ambrose Bierce tiene una definición en su \emph{Devil's Dictionary}: `La actitud de la mente de uno que cree sin evidencia a uno que habla sin conocimiento cosas sin parangón'. ¿Qué deberíamos pensar de esto?}

El objetivo de Elizabeth, hablar de la fe para quien no tiene esa experiencia, determina un enfoque específico a su investigación. La descripción del fenómeno de la fe tiene que ser realizada razonablemente, de modo que pueda ser considerada por alguien \enquote*{que sea capaz de pensar cuidadosa y honestamente} sobre ella. Su estrategia consiste aquí de nuevo en una descripción de usos familiares de la palabra analizada que son articulados de tal manera que los patrones de estos usos sean estudiables\footnote{\cite[Cf.~][12]{bakerhacker2009understanding}: \enquote{There is no room in philosophy for explanatory (hypothetico-deductive) theory, on the model of science, or for dogmatic (essentialist) thesis, on the model of metaphysics. Its task is grammatical clarifiaction that dissolves conceptual puzzlement and gives an overview or surveyable representation of a segment of the grammar of our language \textelp{} It describes the familiar uses of words and arranges them so that the patterns of their use become surveyable, and our entanglement in the web of grammar becomes perspicuous.}}. Se enfoca en un modo antiguo de usar la palabra `fe' en el que se le empleaba para decir \enquote*{creer a alguien que $p$}. `Fe humana' era creer a una persona humana, `fe divina' era creer a Dios\footnote{\cite[Cf.~][2]{anscombe2008faith:tobelieve}: \enquote{At one time there was the following way of speaking: faith was distinguished as human and divine. Human faith was believing a mere human being; divine faith was believing God.}}. Así por ejemplo: \enquote*{Abrám creyó a Dios (\textgreek{ἐπίστευσεν τῷ Θεῷ}) y esto se le contó como justicia} (Gn 15,6). De tal modo que es llamado \enquote*{padre de la fe} (Cf.~Rm 4 y Ga 3,7). La pregunta \enquote*{¿qué es creer a alguien?} queda situada en el centro de este análisis\footnote{\cite[Cf.~][116]{anscombe1981erp:faith}: \enquote{It is clear that the topic I introduced of \emph{believing somebody} is in the middle of our target.}}. Anscombe emplea esta noción para indagar sobre la estructura del creer que está relacionada con la dinámica de la fe. Creer a alguien implica ciertas presuposiciones, al hablar de la fe como \enquote*{creer a Dios que $p$} le atribuye la misma implicación. La cuestión acerca de lo que es creer a alguien resultará de suficiente interés a Anscombe como para dedicarle su propio artículo y en esta investigación, sin duda, juega un papel importante.

Para exponer el desarrollo del análisis que Elizabeth recorre en su discusión podemos atender a tres movimientos principales realizados en su argumentación. Primero se fija en el carácter racional de la fe y recuerda una cierta apologética en la que se le atribuyó este carácter a los llamados preámbulos y el paso de estos a la fe misma; y establece que la la designación correcta de estos `preámbulos de la fe', al menos para parte de ellos, es más bien `presuposiciones'. En segundo lugar describe cuáles son las presuposiciones implicadas en creer a una persona humana cuando esta comunica algo. En tercer lugar examina el fenómeno particular del creer cuando la comunicación viene de Dios.

Elizabeth nos introduce a su reflexión recordando una época en la que la racionalidad de la fe estuvo en el foco de cierta discusión teológica: \blockquote[{\cite[113]{anscombe1981erp:faith}}: \enquote{There was in a preceding time a professed enthusiasm for rationality, perhaps inspired by the teaching of Vatican I against fideism, certainly carried along by the promotion of neo-thomist studies \textelp{} the word was that the Catholic Christian faith was \emph{rational}, and a problem, to those able to feel it as a problem, was how it was \emph{gratuitous}\,---\,a special gift of grace. Why would it \emph{essentially} need the promptings of grace to follow a process of reasoning?}]{Hubo en una época pasada un profuso entusiasmo por la racionalidad, quizás inspirado por la enseñanza del Vaticano~I contra el fideísmo, ciertamente sostenidos por la promoción de estudios neo-tomistas [\ldots] la noticia era que la fe Cristiana Católica era \emph{racional}, y el problema, para aquellos capaces de sentirlo como tal, era cómo era \emph{gratuita}\,---\,un don especial de la gracia. ¿Por qué tendría que ser \emph{esencialmente} necesaria la ayuda de la gracia para seguir un proceso de razonamiento?} Este proceso de razonamiento consistía en una especie de cadena de demostraciones; se afirmaba a Dios, y luego la divinidad de Jesús, y después la institución de la Iglesia por él con el Papa a la cabeza con la encomienda de enseñar. Cada demostración permitiendo justificar la certeza de la verdad de las enseñanzas de la Iglesia\footnote{\cite[Cf.~][12]{anscombe1981erp:faith}: \enquote{It was as if we were assured there was a chain of proof. First God. Then, the divinity of Jesus Christ. Then, \emph{his} establishment of a church with a Pope at the head of it and with a teaching commission from him. This body was readily identifiable. Hence you could demonstrate the truth of what the Church taught}}. Esta breve descripción representa una postura quizás más extravagante, y otras variantes más sobrias enfatizaban más la figura de la Iglesia, o la divinidad de Jesús. Esta actitud más sobria o crítica ante aquellos que pretendían defender la razonabilidad de la fe como una casi demostrabilidad sirvió en beneficio de la veracidad y la honestidad. Ciertamente estas opiniones presentaban problemas. Era obvio que identificar la Iglesia católica que conocemos con la Iglesia que Cristo instituyó no era tarea fácil y necesitaba conocimiento y técnica. Entonces ¿qué carácter tiene la certeza atribuida a la fe? \blockquote[{\cite[114]{anscombe1981erp:faith}}: \enquote{The so-called preambles of faith could not possibly have the sort of certainty that \emph{it} had. And if less, then where was the vaunted rationality?}]{Los llamados preámbulos de la fe no podrían tener el tipo de certeza que \emph{esta} tiene. Y si es menos, entonces ¿dónde esta la racionalidad proclamada?}. Otro problema tenía que ver con la fe de los doctos y los sencillos, ¿aquellos que no conocen estos argumentos tienen un tipo de fe inferior a los doctos? Por otra parte, los que han estudiado ¿realmente conocen todas estas cosas? Ser racional en tener fe implicaba sostener la creencia de que el conocimiento estaba ahí para argumentar y demostrar la verdad de Dios, de Cristo y de la Iglesia, quizá repartido entre algunos expertos o al menos de manera teorética. Todo esto hacía problemáticas estas opiniones.

Anscombe describe brevemente estas discusiones y este modo de hacer apologética que fue empleado en el pasado y ya no se usa en las discusiones de su época. Esto, dice, \blockquote[{\cite[114]{anscombe1981erp:faith}}: \enquote{not necessarily because better thoughts about faith are now common; there is a vacuum where these ideas once were prominent}]{no necesariamente porque sean comunes mejores pensamientos sobre la fe; hay un vacío en donde estas ideas antes fueron prominentes}. Sin embargo opina que no hay que lamentar que estas opiniones hayan pasado, y añade: \blockquote[{\cite[114]{anscombe1981erp:faith}}: \enquote{They attached the character of `rationality' entirely to what were called the preambles and to the passage from the preambles to faith itself. But both these preambles and that passage were in fact an `ideal' construction \textelp{} `fanciful', indeed dreamed up according to prejudices: prejudices, that is, about what it is to be reasonable in holding a belief.}]{Estas atribuían el carácter de `racionalidad' por entero a lo que se llamaron los preámbulos y al paso de estos preámbulos a la fe misma. Pero tanto estos preámbulos como ese paso eran realmente una construcción `ideal' \textelp{} `imaginaria', ciertamente soñada de acuerdo a prejuicios: esto es, prejuicios sobre qué es lo que es ser razonable en sostener una creencia}.

De acuerdo al objetivo trazado al inicio de su discusión, Anscombe busca presentar una descripción del carácter racional de la fe libre de estos prejuicios. En el centro de su propuesta está la comprensión de `fe' como `creer a $x$ que $p$' y, partiendo de esto, el valor de los presupuestos involucrados en creer una comunicación. Comienza, entonces, proponiendo un ejemplo: \blockquote[{\cite[114]{anscombe1981erp:faith}}: \enquote{You receive a letter from someone you know, let's call him Jones. In it, he tells you that his wife has died. You believe him. That is, you now believe that his wife has died because you believe \emph{him}. Let us call this just what it used to be called, ``human faith''. That sense of ``faith'' still occurs on our language. ``Why'', someone may be asked, ``do you believe such-and-such?'', and he may reply ``I just took it on faith\,---\,so-and-so told me''.}]{Recibes una carta de alguien que conoces, llamémosle Jones. En ella te dice que su esposa ha muerto. Tu le crees. Esto es, ahora crees que su esposa ha muerto porque le crees a él. Llamemos a esto justo como solía ser llamado, ``fe humana''. Este sentido de ``fe'' todavía ocurre en nuestro lenguaje. ``Por qué'', se le puede preguntar a alguien, ``crees esto y aquello?'', y podría responder ``Lo tome en buena fe\,---\,fulano me dijo''}. Al especificar este uso de `fe', Elizabeth busca justificar que la designación más adecuada para los llamados `preámbulos' de la fe, al menos para parte de ellos, es `presuposiciones'. En el ejemplo propuesto hay tres creencias implicadas con haberle creído a Jones, estas \blockquote[{\cite[114]{anscombe1981erp:faith}}: \enquote{three convictions or assumptions are, logically, pressupositions that \emph{you} have if your belief that Jones' wife has died is a case of your believing Jones}]{tres convicciones o supuestos son, lógicamente, presuposiciones que \emph{tú} tienes si tu creencia de que la esposa de Jones ha muerto es un caso de que crees a Jones}.

Al creerlo presupones primero que tu amigo Jones existe, segundo, que la carta viene verdaderamente de él, y tercero, que esto que crees es verdaderamente lo que la carta dice. Estas son presuposiciones tuyas, el que puedas llegar a creer la comunicación de la carta no presupone estas tres cosas de hecho, sino que tú crees estas tres cosas.

Ahora bien, `fe' en la tradición en la que ese concepto se origina se refiere a `fe divina' y significa `creer a Dios'. Según esta acepción la fe es absolutamente cierta, puesto que es creer a Dios y, si las presuposiciones son ciertas, conlleva creer sobre los mejores fundamentos a uno habla con conocimiento perfecto. Lo problemático aquí sería en qué consiste creer a Dios.
%, pero antes de indagar más sobre esto, Anscombe estudia con más detalle las presuposiciones relacionadas con creer a una persona humana.

Después de discutir cómo puede atribuírsele a la fe algún carácter de racionalidad y haberse decidido por valorar las convicciones implicadas en la certeza que depositamos en lo que creemos porque creemos a alguien,
%Anscombe ahora nos adentra en el análisis de estas presuposiciones y la utilidad que puedan tener para comprender el fenómeno de la fe.
Anscombe se plantea algunas preguntas relacionadas con estas presuposiciones que discutiremos más adelante: (\S\ref{subsec:presups}, p.~\pageref{subsec:presups}). Aquí solo destacamos dos elementos adicionales sobre ellas discutidos por Anscombe.
%¿Qué es creer a alguien? Anscombe vuelve a su ejemplo. Creer a Jones, que su esposa ha muerto, ¿significa que el hecho de que Jones me diga esto es la \emph{causa} de mi creencia? o ¿significa que el hecho de que se comunique es mi \emph{evidencia} para creer en la muerte de su esposa? ¿Esto sería creer a Jones? No del todo. Puesto que podría ser que la comunicación llama mi atención sobre la cuestión, pero llego a la creencia por mi propio juicio. O puedo tomar lo que me están diciendo y pensar que la persona que me habla me está engañando y a la misma vez está equivocada en lo que me dice, entonces podría decir que creo lo que me dice porque me lo ha dicho, pero no estaría creyendo a la persona. Entonces ¿creer a alguien significa creer que la persona cree lo que me está diciendo? Ordinariamente asumimos esto, pero incluso puede imaginarse el caso en el que alguien me dice algo que cree, pero yo sé que en el origen de su creencia hay una falsedad y por tanto creo lo contrario de lo que esta persona cree y me dice, entonces tampoco estaría creyendole a ella. Sin embargo, en el caso de creer a un maestro, un profesor de historia por ejemplo, sería suficiente para creerle \emph{a él} que creas lo que dice porque lo ha dicho y piensas que no está mintiendo y piensas que lo que él cree es verdadero.
%Estas dificultades no aparecen si se puede establecer con certeza que la persona conoce lo que dice y no miente, sin embargo el tema de creer a alguien no es asunto sencillo. Hay, además, otras preguntas relacionadas con las presuposiciones involucradas en creer a alguien. Al creer lo que dice la comunicación presupones que Jones existe, que escribió la carta y que esta dice lo que has llegado a creer. Pero estos son tus presupuestos y no son condiciones de hecho. ¿Qué se puede decir del caso en el que de hecho no existe la persona que se cree que es quien se comunica? ¿Se puede decir que se está creyendo a Jones si es el caso que de hecho no existe? Si insistiéramos en decir que no se está creyendo en la persona que no existe, afirma Anscombe, \blockquote[{\cite[117]{anscombe1981erp:faith}}: \enquote{you will deprive yourself of the best way of describing his situation: ``he believed this non-existent person''}]{te estarías privando de la mejor manera de describir esta situación: ``le creyó a esta persona no existente''}. De un antiguo que creyó en el oráculo del dios Apolo, por ejemplo, se puede decir efectivamente que creyó en Apolo\,---\,que no existe. Lo mismo se podría decir del caso en el que de hecho existe la persona, pero esta comunicación que se cree que viene de ella no proviene de ella de hecho.
%Dos elementos adicionales son destacados por Anscombe acerca de las presuposiciones.

Primero comenta que \blockquote[{\cite[117]{anscombe1981erp:faith}}: \enquote{the presuppositions of faith are not themselves part of the content of what in a narrow sense is believed by faith}]{las presuposiciones de la fe no son ellas mismas parte del contenido de lo que en un sentido estricto es creido por la fe}. En segundo lugar explica que hay también una \blockquote[{\cite[118]{anscombe1981erp:faith}}: \enquote{difference between presuppositions of believing $N$ and believing such-and-such as coming from $N$. ``Pre-suppositions'' don't have to be temporarily prior beliefs}]{diferencia entre las presuposiciones de creer a $N$ y creer esto o aquello como viniendo de $N$. Las ``pre-suposiciones'' no tienen que ser creencias temporalmente previas}. Elizabeth ilustra esto imaginando el caso en el que la carta dijera que viene de alguien: \enquote*{Esta es una carta de tu viejo amigo Jones}, y al leerla se ponga en duda esta afirmación, o incluso no se ponga en duda sino que se lea acríticamente, sin pensar en ello, entonces se cree lo que dice la carta, pero no se está contando con la credibilidad de Jones como garantía de que la carta viene de él, se tiene en cuenta lo que la carta dice, incluido el que viene de él, pero no se le está creyendo a él y en este sentido las presuposiciones y el contenido de lo que es la fe propiamente son distintos. Otra ilustración puede ser el caso en el que no se tiene un conocimiento previo de la persona que se comunica: \enquote*{Esto es de parte de un amigo desconocido\,---\,llámame $N$}. Imaginemos un prisionero que recibe una comunicación de esta naturaleza y en ella se le ofrecen ayudas para sus necesidades, no sabe si son genuinas, pero responde a la comunicación y recibe las ayudas prometidas. Este prisionero recibe otras comunicaciones que parecen ser de la misma persona y estas contienen nueva información. Al creer esta información el prisionero cree a $N$, pero su creencia en que $N$ existe y que las cartas vienen de él no son creer algo apoyándose en que $N$ lo ha dicho. Es en este sentido que \blockquote[{\cite[118]{anscombe1981erp:faith}}: \enquote{the beliefs which \emph{are} cases of believing $N$ and the belief that $N$ exists are logically different}]{las creencias que \emph{son} casos de creer a $N$ y la creencia de que $N$ existe son lógicamente diferentes}.

En todos estos ejemplos Anscombe ha recurrido a comunicaciones entre personas humanas. ¿Qué se puede decir del caso en que la comunicación viene de Dios? \blockquote[{\cite[118]{anscombe1981erp:faith}}: \enquote{Suarez said that in every revelation God reveals that he reveals}]{Suarez dijo que en cada revelación Dios revela que Él revela} y esto es como decir \blockquote[{\cite[118]{anscombe1981erp:faith}}: \enquote{in every bit of information $N$ is also claiming (implicitly or explicitly, it doesn't matter which) that he is giving the prisioner information}]{en cada pedazo de información $N$ está también declarando (implícita o explícitamente, no importa como) que está dando información al prisionero}. Y aquí hay una dificultad central en el asunto de la fe: \blockquote[{\cite[118]{anscombe1981erp:faith}}: \enquote{In all other cases we have been considering, it can be made clear \emph{what} it is for someone to believe someone. But what can it mean ``to believe God''? Could a learned clever man inform me on the authority of his learning, that the evidence is that God has spoken? No. The only posssible use of a learned clever man is as a \emph{causa removens prohibens}. There are gross obstacles in the received opinion of my time and in its characteristic ways of thinking, and someone learned and clever may be able to dissolve these.}]{En todos los otros casos que hemos estado considerando, puede ser aclarado \emph{qué} es que alguien crea a alguien. Pero ¿qué puede significar ``creer a Dios''? ¿Podría un hombre docto e inteligente informarme sobre la autoridad de su conocimiento, que la evidencia es que Dios ha hablado? No. El único uso posible para un hombre docto e inteligente es como \emph{causa removens prohibens}. Hay grandes obstáculos en la opinion aceptada en mi época y en sus característicos modos de pensar, y alguien con inteligencia y conocimiento podría ser capaz de disolverlos}.

Con esto llegamos al tercer esfuerzo de Anscombe por arrojar luz sobre este tema. ¿Qué estamos creyendo cuando creemos que Dios ha hablado? Para hablar sobre esto Elizabeth recurre a una noción rabínica llamada \emph{Bath Qol} o la `hija de la voz': \blockquote[{\cite[118-119]{anscombe1981erp:faith}}: \enquote{You hear a sentence as you stand in a crowd\,---\,a few words out of what someone is saying perhaps: it leaps at you, it `speaks to your condition'. Thus there was a man standing in a crowd and he heard a woman saying ``Why are you wasting your time?'' He had been dithering about, putting off the question of becoming a Catholic. The voice struck him to the heart and he acted in obedience to it. Now, he did not have to suppose, nor did he suppose, that that remark was not made in the course of some exchange between the woman and her companion, which had nothing to do with him. But he believed that God had spoken to him in that voice. The same thing happened to St Augustine, hearing the child's cry, ``Tolle lege''.}]{Escuchas una oración mientras que estás en medio de una muchedumbre\,---\,algunas palabras de entre lo que alguien está diciendo: saltan hacia ti, `hablan a tu condición'. Así había un hombre que entre la muchedumbre escuchó una mujer que estaba diciendo ``¿Por qué estas desperdiciando tu tiempo?'' Había estado vacilando, ignorando la cuestión de hacerse católico. La voz le golpeó en el corazón y actuó en obediencia a ella. Ahora, él no tenía que suponer, ni de hecho supuso, que este comentario no fuera hecho en el curso de alguna conversación entre la mujer y su acompañante, la cuál no tenía nada que ver con él. Lo mismo ocurrió a San Agustín, al escuchar el grito del niño ``Tolle lege''}.

Ahora bien, todavía hace falta una aclaración adicional respecto de qué significa decir que se cree que Dios habla. En los ejemplos anteriores estaba claro qué significa para alguien que \enquote*{cree a $X$} el que \enquote*{$X$ está hablando}. Incluso en el caso de que no exista. Pero no es claro qué es que Dios sea el que hable. Aquí, entender deidad como el objeto de adoración no es útil puesto que habría que definir adoración como el honor ofrecido a una deidad. En este sentido por `Dios' Anscombe no se refiere al objeto de esta o aquella adoración; `Dios' no es un nombre propio, sino una `descripción definitiva' en el sentido técnico. Es decir es equivalente a `el uno y único dios verdadero'. Un ateo cree que Dios está entre los dioses que no son dioses, pero podría entender la identidad de `Dios' con `el uno y único dios'. En este sentido decir que Dios es el dios de Israel es decir lo que Israel ha adorado como dios es `el uno y único dios verdadero'. Esto podría ser afirmado o negado por alguien incluso que considerara que esa expresión es vacía o no se refiere a nada.

Con esto, Anscombe llega a una descripción conclusiva: \blockquote[{\cite[119-120]{anscombe1981erp:faith}}: \enquote{And so we can say this: the supposition that someone has faith is the supposition that he believes that something ---it may be a voice, it may be something he has been taught--- comes as a word from God. Faith is then the belief he accords to that word.}]{Y entonces podemos decir esto: la suposición de que alguien tiene fe es la suposición de que cree que algo ---puede ser una voz, puede ser algo que ha aprendido--- viene como una palabra de Dios. Fe es entonces la creencia que otorga a esa palabra}. Esto puede ser entendido por alguien que no tiene fe, sea que su actitud ante este fenómeno sea de reverencia, indiferencia u hostilidad. Esto además puede ser dicho en términos generales sobre el fenómeno de la fe. En el caso específico del que cree en Cristo: \blockquote[{\cite[120]{anscombe1981erp:faith}}: \enquote{the Christian adds that such a belief is sometimes the truth, and that the consequent belief is only then what \emph{he} means by faith}]{el cristiano añade que esta creencia es en ocasiones la verdad, y esta creencia consecuente es solo lo que \emph{él} entiende por fe}.
