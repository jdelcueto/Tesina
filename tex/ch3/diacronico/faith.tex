\subsection{Faith}

En Oscott College, el seminario de la Archidiócesis de Birmingham, se comenzaron a celebrar las conferencias llamadas Wiseman Lectures en 1971. Para estas lecciones ofrecidas anualmente en memoria de Nicholas Wiseman se invitaba un ponente que tratara algún tema relacionado con la filosofía de la religión o alguna materia en torno al ecumenismo.\footnote{\cite[Cf.~][7]{wisemanlects}}

El 27 de octubre de 1975, para la quinta edición de las conferencias, Anscombe presentó una lección titulada simplemente \emph{Faith}. Allí planteaba la siguiente cuestión: \blockquote[{\cite[115]{anscombe1981erp:faith}}: I want to say what might be understood about faith by someone who did not have it; someone, even, who does not necessarily believe that God exists, but who is able to think carefully and truthfully about it. Bertrand Russell called faith `certainty without proof'. That seems correct. Ambrose Bierce has a definition in his Devil's Dictionary: `The attitude of mind of one who believes without evidence one who tells without knowledge things without parallel.' What should we think of this?]{Quiero decir qué puede ser entendido sobre la fe por alguien que no la tenga; alguien, incluso, que no necesariamente crea que Dios existe, pero que sea capaz de pensar cuidadosa y honestamente sobre ella. Bertrand Russell llamó a la fe `certeza sin prueba'. Esto parece correcto. Ambrose Bierce tiene una definición en su \emph{Devil's Dictionary}: `La actitud de la mente de uno que cree sin evidencia a uno que habla sin conocimiento cosas sin parangón'. ¿Qué deberíamos pensar de esto?}

El objetivo de Elizabeth, hablar de la fe para quien no tiene esa experiencia, determina un enfoque específico a su investigación. La descripción del fenómeno de la fe tiene que ser realizada razonablemente, de modo que pueda ser considerada por alguien \enquote{que sea capaz de pensar cuidadosa y honstamente} sobre ella. Su estrategia, la carácterística \enquote{investigación gramatica}, consiste aquí de nuevo en una descripción de usos familiares de la palabra que está siendo analizada que son articulados de tal manera que los patrones de estos usos sean estudiables\autocite[Cf.~][12]{bakerhacker2009understanding}. Se enfoca en un modo antiguo de usar la palabra \enquote{fe} en el que se le empleaba para decir \enquote{creer a alguien que $p$}. \enquote{Fe humana} era creer a una persona humana, \enquote{fe divina} era creer a Dios\autocite[Cf.~][2]{anscombe2008faith:tobelieve}. Así por ejemplo: \enquote{Abrám creyó a Dios (\textgreek{ἐπίστευσεν τῷ Θεῷ}) y ésto se le contó como justicia} (Gn~15,6). De tal modo que es llamado \enquote{padre de la fe} (Cf.~Rm~4 y Ga 3,7). La pregunta \enquote{¿qué es creer a alguien?} queda situada en el centro de este análisis\footnote{\cite[Cf.~][116]{anscombe1981erp:faith}: It is clear that the topic I introduced of \emph{believing somebody} is in the middle of our target.}. Anscombe emplea essta noción para indagar sobre la estructura del creer que está relacionada con la dinámica de la fe. Creer a alguien implica ciertas presuposiciones, al hablar de la fe como \enquote{creer a Dios que $p$} le atribuye la misma implicación. La cuestión acerca de lo que es creer a alguien resultará de suficiente interés a Anscombe como para dedicarle su propio artículo y en esta investigación, sin duda, juega un papel importante.

Para exponer el desarrollo del análisis que Elizabeth recorre en su discusión podemos atender a tres movimientos principales realizados en su argumentación. Primero se fija en el carácter racional de la fe y recuerda una cierta apologética en la que se le atribuyó este carácter a los llamados preámbulos y el paso de éstos a la fe misma; y establece que la la designación correcta de estos \enquote{preámbulos de la fe}, al menos para parte de ellos, es más bien \enquote{presuposiciones}. En segundo lugar describe cuáles son las presuposiciones implicadas en creer a una persona humana cuando esta comunica algo. En tercer lugar examina el fenómeno particular del creer cuando la comunicación viene de Dios.

Elizabeth nos introduce a su reflexión recordando una época en la que la racionalidad de la fe estuvo en el foco de cierta discusión teológica: \blockquote[{\cite[113]{anscombe1981erp:faith}}: There was in a preceding time a professed enthusiasm for rationality, perhaps inspired by the teaching of Vatican I against fideism, certainly carried along by the promotion of neo-thomist studies \textelp{} the word was that the Catholic Christian faith was \emph{rational}, and a problem, to those able to feel it as a problem, was how it was \emph{gratuitous} --- a special gift of grace. Why would it \emph{essentially} need the promptings of grace to follow a process of reasoning?]{Hubo en una época pasada un profuso entusiasmo por la racionalidad, quizás inspirado por la enseñanza del Vaticano~I contra el fideísmo, ciertamente sostenidos por la promoción de estudios neo-tomístas [\ldots] la noticia era que la fe Cristiana Católica era \emph{racional}, y el problema, para aquellos capaces de sentirlo como tal, era cómo era \emph{gratuita} --- un don especial de la gracia. ¿Por qué tendría que ser \emph{esencialmente} necesaria la ayuda de la gracia para seguir un proceso de razonamiento?}. Este proceso de razonamiento consistía en una especie de cadena de demostraciones; se afirmaba a Dios, y luego la divinidad de Jesús, y después la institución de la Iglesia por él con el Papa a la cabeza con la encomienda de enseñar. Cada demostración permitiendo justificar la certeza de la verdad de las enseñanzas de la Iglesia.\footnote{\cite[Cf.~][12]{anscombe1981erp:faith}: It was as if we were assured there was a chain of proof. First God. Then, the divinity of Jesus Christ. Then, \emph{his} establishment of a church with a Pope at the head of it and with a teaching commission from him. This body was readily identifiable. Hence you could demonstrate the truth of what the Church taught} Esta breve descripción representa una postura quizás más extravagante, y otras variantes más sobrias enfatizaban más la figura de la Iglesia, o la divinidad de Jesús. Esta actitud más sobria o crítica ante aquellos que pretendían defender la razonabilidad de la fe como una casi demonstrabilidad sirvió en beneficio de la veracidad y la honestidad. Ciertamente estas opiniones presentaban problemas. Era obvio que identificar la Iglesia católica que conocemos con la Iglesia que Cristo instituyó no era tarea fácil y necesitaba conocimiento y técnica. Entonces ¿qué carácter tiene la certeza atribuida a la fe? \blockquote[{\cite[114]{anscombe1981erp:faith}}: The so-called preambles of faith could not possibly have the sort of certainty that \emph{it} had. And if less, then where was the vaunted rationality?]{Los llamados preámbulos de la fe no podrían tener el tipo de certeza que \emph{esta} tiene. Y si es menos, entonces ¿dónde esta la racionalidad proclamada?}. Otro problema tenía que ver con la fe de los doctos y los sencillos, ¿aquellos que no conocen estos argumentos tienen un tipo de fe inferior a los doctos? Por otra parte, los que han estudiado ¿realmente conocen todas estas cosas? Ser racional en tener fe implicaba sostener la creencia de que el conocimiento estaba ahí para argumentar y demostrar la verdad de Dios, de Cristo y de la Iglesia, quizá repartido entre algunos expertos o al menos de manera teorética. Todo esto hacía problemáticas estas opiniones.

Anscombe describe brevemente estas discusiones y este modo de hacer apologética que fue empleado en el pasado y ya no se usa en las discusiones de su época. Esto, dice, \blockquote[{\cite[114]{anscombe1981erp:faith}}: not necessarily because better thoughts about faith are now common; there is a vacuum where these ideas once were prominent]{no necesariamente porque sean comunes mejores pensamientos sobre la fe; hay un vacío en donde estas ideas antes fueron prominentes}. Sin embargo opina que no hay que lamentar que estas opiniones hayan pasado, y añade: \blockquote[{\cite[114]{anscombe1981erp:faith}}: They attached the character of `rationality' entirely to what were called the preambles and to the passage from the preambles to faith itself. But both these preambles and that passage were in fact an `ideal' construction \textelp{} `fanciful', indeed dreamed up according to prejudices: prejudices, that is, about what it is to be reasonable in holding a belief.]{Estas atribuían el carácter de `racionalidad' por entero a lo que se llamaron los preámbulos y al paso de estos preámbulos a la fe misma. Pero tanto estos preámbulos como ese paso eran realmente una construcción `ideal' \textelp{} `imaginaria', ciertamente soñada de acuerdo a prejuicios: esto es, prejuicios sobre qué es lo que es ser razonable en sostener una creencia.}

Esto trae a Anscombe a una de sus propuestas principales, que explica proponiendo un ejemplo: \blockquote[{\cite[114]{anscombe1981erp:faith}}: You receive a letter from someone you know, let's call him Jones. In it, he tells you that his wife has died. You believe him. That is, you now believe that his wife has died because you believe \emph{him}. Let us call this just what it used to be called, ``human faith''. That sense of ``faith'' still occurs on our language. ``Why'', someone may be asked, ``do you believe such-and-such?'', and he may reply ``I just took it on faith ---so-and-so told me''.]{Recibes una carta de alguien que conoces, llamémosle Jones. En ella te dice que su esposa ha muerto. Tu le crees. Esto es, ahora crees que su esposa ha muerto porque le crees a él. Llamemos a esto justo como solía ser llamado, ``fe humana''. Este sentido de ``fe'' todavía ocurre en nuestro lenguaje. ``Por qué'', se le puede preguntar a alguien, ``crees esto y aquello?'', y podría responder ``Lo tome en buena fe ---fulano me dijo''.} Al proponer este uso de ``fe'', Elizabeth justifica que la designación más adecuada para los llamados ``preámbulos'' de la fe, al menos para parte de ellos, es ``presuposiciones''.\footnote{``presuppositions''} En el ejemplo propuesto hay tres creencias implicadas con haberle creido a Jones, estas \blockquote[{\cite[114]{anscombe1981erp:faith}}: three convictions or assumptions are, logically, pressupositions that \emph{you} have if your belief that Jones' wife has died is a case of your believing Jones]{tres convicciones o supuestos son, lógicamente, presuposiciones que \emph{tú} tienes si tu creencia de que la esposa de Jones ha muerto es un caso de que crees a Jones}.

Al creerlo presupones primero que tu amigo Jones existe, segundo, que la carta viene verdaderamente de él, y tercero, que esto que crees es verdaderamente lo que la carta dice. Estas son presupociones tuyas, el que puedas llegar a creer la comunicación de la carta no presupone estas tres cosas de hecho, sino que tú crees estas tres cosas.

Ahora bien, ``fe'' en la tradición en la que ese concepto se origina se refiere a ``fe divina'' y significa ``creer a Dios''. Según esta acepción la fe es absolutamente cierta, puesto que es creer a Dios y, si las presuposiciones son ciertas, conlleva creer sobre los mejores fundamentos a uno habla con conocimiento perfecto.

\subsubsection{Las presuposiciones implicadas en creer a alguien}

\subsubsection{`Creer' cuando la comunicación viene de Dios}
