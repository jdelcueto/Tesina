\subsection{Prophecy and Miracles (1957)}
\label{subsec:prophnmi}

El \emph{Philosophical Enquiry Group} se reunió anualmente entre 1954 y 1974 en el Centro de Conferencias de los Dominicos en \emph{Spode House, Staffordshire}. Los encuentros tenían como objetivo la discusión de cuestiones relacionadas con las creencias y prácticas cristianas. Elizabeth Anscombe y Peter Geach estuvieron entre los primeros ponentes invitados y colaboraron durante los veinte años que se realizaron las conferencias\footnote{\cite[Cf.~][x]{anscombe2008faith}: \enquote{\textelp{} no information was found about a number of papers. Features of their physical format suggested that the group of three (`Prophecy', `The Inmortality of the soul', and `On being in good faith', Nos. 3,9 and 12) were all given in the late 1950s and early 1960s to the Philosophical Enquiry Group which met each year between 1954 and 1974 at the Dominican Conference Centre at Spode House in Staffordshire. \textelp{} Among the first invitees were Elizabeth Anscombe and Peter Geach \textelp{} The meetings focused on philosophical issues related to Christian belief and practice.}}. Una de estas colaboraciones se encuentra en \emph{Prophecy and Miracles}, publicado en \emph{Faith in a Hard Ground} en 2008. Es con mucha probabilidad el texto de una ponencia ofrecida por Anscombe en la reunión del grupo en 1957\footnote{\cite[Cf.~][nota a pie de página 20]{anscombe2008faith:prophandmi}: \enquote{From the undated typescript of a paper, probably delivered in 1957}}.

Elizabeth introduce su discusión ofreciendo tres documentos que servirán como los ejes principales de su análisis:
\begin{enumerate}
\item La constitución dogmática \emph{Dei Filius}, específicamente el capítulo tercero: \blockquote[{\cite[\S\,3009]{vati1870df}}: \enquote{Ut nihilominus fidei nostrae obsequium rationi consentaneum \textins{\emph{cf. Rm 12,1}} esset, voluit Deus cum internis Spiritus Sancti auxiliis externa iungi revelationis suae argumenta, facta scilicet divina, atque imprimis miracula et prophetias, quae cum Dei omnipotentiam et infinitam scientiam luculenter commonstrent, divinae revelationis signa sunt certissima et omnium intelligentiae accommodata \textins{\emph{can. 3 et 4}}. Quare tum Moyses et Prophetae, tum ipse maxime Christus Dominus multa et manifestissima miracula et prophetias ediderunt; et de Apostolis legimus: ``Illi autem profecti praedicaverunt ubique, domino cooperante, et sermonem confirmante, sequentibus signis'' \textins{\emph{Mc 16,20}}. Et rursum scriptum est: ``Habemus firmiorem propheticum sermonem, cui bene facitis attendentes quasi lucernae lucenti in caliginoso loco'' \textins{\emph{2 Pt 1,19}}.}]{Sin embargo, para que el obsequio de nuestra fe fuera conforme a la razón \textins{\emph{cf. Rm 12,1}}, quiso Dios que a los auxilios internos del Espíritu Santo se juntaran argumentos externos de su revelación, a saber, hechos divinos y, ante todo, los milagros y profecías, que, mostrando de consuno luminosamente la omnipotencia y ciencia infinita de Dios, son signos ciertísimos y acomodados a la inteligencia de todos, de la revelación divina \textins{\emph{can. 3 et 4}}. Por eso, tanto Moisés y los profetas, como sobre todo el mismo Cristo Señor, hicieron y pronunciaron muchos y clarísimos milagros y profecías; y de los apóstoles leemos: <<Y ellos marcharon y predicaron por todas partes, cooperando el Señor y confirmando su palabra con los signos que se seguían>> \textins{\emph{Mc 16,20}}. Y nuevamente está: <<Tenemos palabra profética más firme, a la que hacéis bien en atender como a una antorcha que brilla en un lugar tenebroso>> \textins{\emph{2 Pe 1,19}}}.
\item La advertencia del Deuteronomio: \blockquote{Todo lo que yo os mando, lo debéis observar y cumplir; no añadirás ni suprimirás nada. Si surge en medio de ti un profeta o un visionario soñador y te propone: \enquote{Vamos en pos de otros dioses ---que no conoces--- y sirvámoslos}, aunque te anuncie una señal o un prodigio y se cumpla la señal o el prodigio, no has de escuchar las palabras de ese profeta o visionario soñador (Dt 13, 1-4a).}
\item \emph{Sobre la Demostración de Espíritu y fuerza} de Lessing. Del cual considera varios puntos, pero se enfoca en su argumento central: \blockquote[La traducción al inglés de este fragmento es de Anscombe, {\cite[Cf.~][22]{anscombe2008faith:prophandmi}}: \enquote{Who denies it ---I do not--- that the reports of those miracles and prophecies are just as trustworthy as any historical truth can be? ---But now: if they are only so trustworthy, why are they so used as suddenly to make them infinitely more trustworthy? How? By building quite different things, and more things, on them, than one is entitled to build on historically evidenced truths. If no historical truth can be demonstrated, then neither can anything be demonstrated by historical truths. That is: accidental historical truths can never become the proof of necessary truths of reason.}]{¿Quién lo niega ---no lo hago yo--- que los informes de esos milagros y profecías son tan dignos de confianza como puede ser cualquier verdad histórica? ---Pero ahora: si solo son tan merecedores de confianza, ¿por qué de repente son empleados como si fueran infinitamente confiables? ¿Cómo? Al construir cosas bastante distintas, y más cosas, sobre ellos, de las que se está en autoridad de construir sobre verdades de evidencia histórica. Si ninguna verdad histórica puede ser demostrada, entonces tampoco ninguna otra cosa puede ser demostrada por medio de verdades históricas. Esto es: verdades contingentes en tanto que históricas nunca pueden llegar a ser prueba de verdades de razón en tanto que necesarias}.
\end{enumerate}

Tras esta introducción, Anscombe comienza su análisis desenmarañando algunos puntos de los argumentos del ensayo de Lessing. En una de sus premisas él emplea como ejemplo de verdad histórica nuestra creencia en que hubo en el pasado una persona llamada Alejandro, que conquistó casi toda Asia en corto tiempo. Entonces ofrece el reto: \enquote*{¿Quién, en consecuencia de esta creencia, estaría dispuesto a abjurar permanentemente de todo conocimiento que pueda entrar en conflicto con ella?}. Sugiere entonces considerar la idea de que, después de todo, sería posible que la creencia en estas grandes conquistas podrían estar fundadas simplemente en los poemas de Choerilus que acompañó a Alejandro\footnote{\cite[Cf.~][448]{lessing1982escritos:demo}}.

Esta última propuesta resulta llamativa para Anscombe. Parece una alusión al hecho de que conocemos de Cristo por una fuente o tradición `única'. Sin embargo Anscombe piensa que más bien viene a apoyar la afirmación de que las verdades históricas no pueden ser fundamentos de verdades necesarias. Una verdad metafísica o una verdad matemática no puede seguirse de un hecho histórico, este tendría que contar con el mismo grado de certeza que estas verdades de razón; pero una verdad histórica es muy incierta, como lo serían las conquistas de Alejandro, si solo supiéramos de ellas por los poemas de Choerilus. Ahora bien, a juicio de Anscombe, esta premisa no merece gran atención. El supuesto de que cualquier cosa creíble sobre Dios tiene que ser una verdad necesaria de razón le parece una derivación de las nociones propuestas por Leibniz sobre la necesidad en relación con Dios. En adición a esto, es una premisa apoyada sobre el supuesto de que las verdades de la religión son de tal naturaleza que la razón humana podría haber llegado a pensarlas por sí misma.

Anscombe sí encuentra valor en la premisa acerca de no afirmar certezas más allá de las que las verdades históricas nos dan la autoridad de justificar. La constitución del Vaticano~I habla de los milagros y profecías cumplidas como sólidos argumentos externos. ¿Puede una verdad histórica contar con certeza suficiente para representar un sólido argumento externo? No es el papel de estas manifestaciones ser una demostración que reemplace el rol del Espíritu en la fundamentación de la fe. Entonces parece que verdades históricas que no puedan ser estimadas más que como probabilidades podrían jugar ese papel. ¿Se podría conceder que la fe no necesita de argumentos externos ciertos para ser abrazada? ¿Podrían emplearse errores históricos y argumentaciones equivocadas como una escalera que se usa para llegar a la fe y luego se descarta? Para Anscombe sería un error pensar que una `escalera' como esta podría acercarnos adecuadamente a la fe. Aunque se descarte la idea de Lessing de que toda creencia sobre Dios tiene que ser una verdad necesaria, hay algo de valor en la idea de que una fe cierta no se puede afirmar simplemente sobre argumentos externos con fundamentos inciertos.

%Otro punto destacado por Anscombe es que la posición de Lessing ante el cristianismo es incompatible con las creencias cristianas. Una de sus analogías ilustra bien esta actitud: \blockquote[{\cite[449]{lessing1982escritos:demo}}]{Supongamos que se diera una verdad matemática, grande y útil, a la que su descubridor hubiera llegado siguiendo un palmario sofisma \textelp{} ¿negaría yo por ello esa verdad y me negaría por eso a hacer uso de esa verdad? Pero ¿sería yo un ingrato calumniador del inventor, por no querer apoyarme en su agudeza, probada sí de otras maneras, para demostrar y mantener que el sofisma mediante el que dio con esa verdad no \emph{puede} ser un sofisma?} Su interés en Cristo es en la enseñanza que este maestro pueda ofrecer. Adicionalmente, su opinión es que lo que puede decirse sobre Dios, no solo no pueden ser proposiciones que derivan su justificación desde afirmaciones históricas, sino que además no podrían ser afirmaciones incompatibles con lo que podría ser razonable en estimar como históricamente posible. Según esto, hace distinción entre la \emph{religión cristiana} y la \emph{religión de Cristo}. Esta última sería la que ofrece enseñanzas claras y útiles, sin embargo ha quedado mezclada en su transmisión con lo confuso y oscuro de lo que él llama la \emph{religión cristiana}.

%Una aclaración adicional que Anscombe propone es que, a su juicio, Lessing exagera la certidumbre que (desde un punto de vista externo) podría tener Orígenes de los milagros y profecías cumplidas. Tanto en su tiempo como en el nuestro los milagros serían hechos completamente extraordinarios y serían estimados por los escépticos con tanta incredulidad entonces como ahora, mientras que los católicos los aceptan.

Hechas estas consideraciones preliminares, Anscombe estudia el argumento central establecido por Lessing. Su impresión es que la objeción de Lessing consiste fundamentalmente en: \enquote*{Pero estas cosas \emph{pueden} no ser verdad, ¿cómo puedo emplearlas para apoyar el cristianismo?}. El argumento es útil, puesto que no se orienta a atacar la veracidad de los milagros o cumplimientos de profecías que han quedado documentados, sino que pone en duda que estos testimonios o relatos puedan ser fundamento suficiente para sostener la creencia en el cristianismo como justificada. En esto está claramente en conflicto con la enseñanza del Vaticano I.

Por su parte, la afirmación de \emph{Dei Filius} es de extraordinario interés para Anscombe ya que le parece que la experiencia más común es que creamos en las profecías cumplidas y los milagros porque creemos en la religión católica y estos forman parte de su enseñanza. Si tomamos esto en cuenta junto con la enseñanza del Deuteronomio y una reflexión razonable acerca de lo que la fe requiere, tendríamos que decir que para que se puedan tomar los milagros y las profecías cumplidas como \enquote*{sólidos argumentos externos}, estos tendrían que quedar determinados como tal antes de que quede afirmada la creencia en el cristianismo. Pero, ¿acaso no hay ya cierto elemento teológico en designar algo como una profecía cumplida o milagro? ¿En qué situación está un juez o historiador indiferente de la religión que recibe noticias de un milagro o de profecías cumplidas? ¿Pueden ser éstos sólidos argumentos externos para creer en la religión católica?

El análisis de Anscombe se desarrollará en torno a la posibilidad de sostener creencias ciertas teniendo como fundamento los informes de milagros; o la certeza de los relatos históricos; o las profecías cumplidas que puedan ser consideradas claras por su antigüedad, prioridad y realización.

En cuanto a los informes de milagros, Anscombe sostiene con Lessing que estos no apelarían a un juez que sea externo a las creencias religiosas. Podemos estimar la resurrección de Cristo como el signo principal empleado por la apologética. A la noticia de este milagro Lessing le concede tanta certeza como la que pueda tener un dato histórico, Anscombe, sin embargo, no está de acuerdo con esto. Le parece que no es irrazonable decir: \blockquote[{\cite[26]{anscombe2008faith:prophandmi}}: \enquote{`Heaven knows what happened to produce this belief; I do not. And I know much too little about what may go on in human minds in the origins of embracing a new religious belief, to draw any conclusions (as I am so often pressed to do) from the subsequent careers of the Apostles (supposing them to be truly related in the main) or from the sudden appearance and growth of a new religion, which after all is all I am really perfectly certain of. I do know one thing: new religions sometimes spread like wildfire. How this works, and how it gets established in them is obscure. I concede that this is an impressive religion too; but then it had a very impressive religion behind it: that of the Old Testament. Remember that beliefs in miraculous events in connexion with the founders and heroes of religion are quite common. The most I can grant is that the record is quite as if these things had happened: the manner is not legendary, though the matter is!'}]{Dios sabe qué ocurrió para que se produjera esta creencia; yo no lo sé. Además conozco muy poco de lo que ocurre en las mentes humanas en los orígenes de abrazar una creencia religiosa nueva, como para sacar alguna conclusión \textelp{} de las subsiguientes misiones de los Apóstoles \textelp{} o de la repentina aparición y crecimiento de una nueva religión, de lo que después de todo es todo de lo que estoy perfectamente segura. Sí conozco una cosa: las religiones nuevas a veces se propagan como el fuego. Cómo funciona esto, y cómo queda establecido en ellas es oscuro. Concedo que esta es una religión impresionante también; pero ha tenido una religión impresionante detrás: la del Antiguo Testamento. Recuerda que las creencias de eventos milagrosos en conexión con los fundadores o héroes de una religión son bastante comunes. Lo mayor que puedo conceder es que la noticia es bastante como si estas cosas hubieran ocurrido: ¡el modo no es legendario, aunque la materia sí!}

Aquí la cuestión importante para Anscombe es cómo ha llegado a ocurrir que estos informes aparentemente fácticos hayan llegado a quedar escritos y transmitidos de este modo y qué tipo de hipótesis podría explicar este hecho. Si efectivamente estos hechos han ocurrido, ¿de qué naturaleza esperaríamos que fueran los documentos y noticias que nos los transmiten? Sin embargo, no sería razonable pedir a un historiador indiferente que resuelva este problema, sobre cómo han llegado a existir estos documentos y tradiciones, no sería irrazonable para él dejar sin respuestas estas preguntas\footnote{\cite[Cf.~][37]{anscombe2008faith:prophandmi}: \enquote{it is not reasonable to ask an indiferent historian to solve this problem, of how such records came to be written; he can reasonably just leave it unsolved.}}.

En donde Elizabeth estima que Lessing no tiene razón es en decir que ninguna certeza histórica puede ser suficientemente fuerte como para tener un peso absoluto. Lessing hace alusión al error que puede suponer saltar desde verdades históricas a conclusiones que son verdades de una clase distinta, pero da importancia también a esta otra cuestión sobre la fuerza que puede tener una afirmación histórica para justificar nuestras creencias. Si es la fuerza de la certeza lo que se está realmente poniendo en duda, le parece a Anscombe que no es cierto que la certeza histórica sea siempre demasiado débil como para fundamentar una certeza absoluta.

Lessing concede a un dato histórico como la existencia de Alejandro Magno el grado de certeza de probabilidad. Anscombe juzga que la probabilidad, en oposición a la total certeza, entra en juego más tarde para un dato como este. Así afirma: \blockquote[{\cite[26]{anscombe2008faith:prophandmi}}: \enquote{I should not mind staking anything whatever on the existence of Alexander, or foreswearing for ever any proferred appearance of knowledge that conflicted with it.}]{No me importaría arriesgar cualquier cosa en la existencia de Alejandro, o renunciar para siempre a cualquier ofrecimiento de aparente conocimiento que entre en conflicto con esto}. Donde empezaríamos a hablar en términos de probabilidad sería si nos preguntamos a quién nos referimos por `Alejandro' ---si en algún momento fue reemplazado por un impostor, por ejemplo--- pero acerca de la existencia de Alejandro la certeza es de mayor grado. En definitiva, no todos los datos históricos tienen el mismo grado de certeza, y es un error no distinguir el valor fundamental que llegan a tener ciertas afirmaciones históricas; en palabras de Anscombe: \blockquote[{\cite[27]{anscombe2008faith:prophandmi}}: \enquote{I object to his lumping together everything historical as of inferior certainty to my own experience}]{Estoy opuesta a su modo de amontonar todo lo histórico como de inferior certeza a mi propia experiencia}.

Para Anscombe hay proposiciones históricas que forman parte del conocimiento común de tal manera que no se pueden poner en duda sin más, puesto que si se duda de una proposición tan presente en el conocimiento general se hace imposible afirmar el conocimiento que pueda ofrecer del todo cualquier otra evidencia histórica. Es así que podríamos dudar de una experiencia personal, es probable que lo que creemos conocer por nuestra experiencia no haya sido tal, \blockquote[{\cite[27]{anscombe2008faith:prophandmi}}: \enquote{whereas things making it remotely probable that there was no Alexander are inconceivable}]{mientras que cosas que hagan remotamente probable que no hubo un Alejandro son inconcebibles}. Esto se debe a que: \blockquote[{\cite[27]{anscombe2008faith:prophandmi}}: \enquote{there could be no reason to think one knew what any historical evidence suggested at all, if a great range of things in history were not quite solid. Experience, unless it is made right by definition, is not more but less certain; and what I judge from experience may, some of it, more easily be wrong.}]{no podría haber razón alguna para pensar que sabemos qué podría sugerir del todo cualquier evidencia histórica, si un amplio rango de cosas en la historia no fuera del todo sólido. La experiencia, a no ser que sea hecha cierta por definición, no es mayor, sino de menor certeza; y lo que yo juzgo desde la experiencia puede, en parte, ser con mayor facilidad incorrecto}.

Ahora bien, ¿qué solidez tienen los datos históricos relacionados con Cristo?. Que Jesús existió, y predicó como lo hacían los profetas del Antiguo Testamento, y que fue al menos ostensiblemente crucificado bajo la autoridad romana y que los creyentes lo tomaron como el Mesías y el Hijo de Dios y creyeron que resucitó de los muertos; estos datos históricos cuentan con la solidez antes descrita. Que Jesús declaró ser el Hijo de Dios, y que resucitó de los muertos no son sólidos de esta manera. Si algún escrito, de Tácito digamos, afirmara que los cristianos creían que Jesús se habría escondido y no moriría nunca y visitaba en secreto a los creyentes; esto no sería evidencia de las genuinas creencias de los discípulos y de que nos equivocamos en nuestras impresiones actuales de estas creencias, sino que sería evidencia de que Tácito escribió descripciones mal informadas de las creencias de los cristianos. El conocimiento histórico general de las creencias de los cristianos de entonces sería la medida para juzgar el escrito de Tácito y no al revés.

Hay ciertas afirmaciones históricas que son sólidas y que pueden emplearse como justificación suficiente para certezas absolutas. Algunos datos relacionados con Jesús pueden ser valorados así y por tanto no pueden ponerse en duda sin más. Otras afirmaciones históricas sobre Jesús que no tienen esta solidez, sin embargo tampoco pueden ser razonablemente afirmadas como falsas. El hecho de la muerte, la ausencia de su cuerpo en el sepulcro, su reaparición tras la muerte, y también su declaración de ser el Hijo de Dios: \blockquote[{\cite[28]{anscombe2008faith:prophandmi}}: \enquote{these belong to the very large realm of historical assertions which it would indeed be absurd to claim certainty for, but the time for disproving which is past \textelp{} with them there is no danger of running up against a disproof of them, and the greater part of them must be true: but of any particular one, we cannot say it is perfectly certain. We may note that the death of Christ would be refuted, in normal circumstances, just by his reappearance alive.}]{estas pertenecen al amplio campo de afirmaciones históricas de las cuales sería ciertamente absurdo afirmar certeza, pero el tiempo para refutarlas ya ha pasado \textelp{} con estas no hay peligro de toparse con algo que las contradiga, y la mayor parte de ellas debe ser verdadera: pero de alguna en particular, no podemos decir que es perfectamente cierta. Podemos destacar que la muerte de Cristo sería refutada, en circunstancias ordinarias, justo por su reaparición en vida}. Anscombe piensa que Lessing no está consciente de la existencia de esta clase de proposiciones.

Tras estos análisis sobre las noticias de milagros y la fuerza de la certeza histórica, Anscombe dirige su discusión hacia las profecías. En el centro de su reflexión está el requisito propuesto por Lessing: \blockquote[{\cite[29]{anscombe2008faith:prophandmi}}: \enquote{in order to say `This was predicted, and it happened' we have to judge that the thing that happened, not merely was describable in the words occurring in the prediction, but was what was predicted: otherwise `fulfilment' equals `applicability of these words'; and can't this just be an accident?}]{para poder decir `Esto fue predicho, y ocurrió' tendríamos que juzgar que lo ocurrido, no solo puede ser descrito por las palabras que aparecen en la predicción, sino que es lo que fue predicho de hecho: de otro modo `realización' es igual a `aplicabilidad de estas palabras'; y ¿puede no ser esto simplemente un accidente?} Anscombe sostiene que hay dificultades especiales acerca de la noción de la aplicabilidad de las palabras proféticas como \emph{accidental}.

Elizabeth ofrece una ilustración para esto. Un personaje en una obra teatral se presenta como un personaje del pasado y describe hechos históricos de épocas posteriores a la suya y que nosotros conocemos, el efecto sería ficticio, lo que el autor quiere decir estaría claro. Sin embargo, si sale a relucir que estas afirmaciones fueron realmente hechas por una persona en el pasado, entonces al instante se convierten en palabras vagas y problemáticas. \blockquote[{\cite[31]{anscombe2008faith:prophandmi}}: \enquote{This is a logical point: of the many, many utterances we might make now about the present or the past, which have a good sharp sense, by far the greater number would look hopelessly obscure if said earlier, of the future: even ones with proper names}]{Esto es un punto lógico: de las muchas, muchas afirmaciones que podríamos hacer ahora acerca del presente o del pasado, las cuales tienen un sentido claro, por mucho la mayoría se vería irremediablemente oscura si hubiera sido dicha antes, sobre el futuro: incluso aquellas que contienen nombres propios} Anscombe insiste en distinguir que las afirmaciones sobre el pasado o el presente no significan de la misma manera que afirmaciones sobre el futuro. En este sentido, si alguien afirmara un hecho verdadero del pasado y resulta que ignoraba que había ocurrido, entonces es solo un accidente que sus palabras aplicaran; \blockquote[{\cite[29]{anscombe2008faith:prophandmi}}: \enquote{but it is impossible to know the future of the world and of human affairs; so this test for accident cannot be made}]{pero es imposible conocer el futuro del mundo y de los asuntos humanos; así que esta prueba de accidente no puede ser hecha}. La pregunta acerca de lo que un profeta quiso decir o qué tuvo en la mente cuando afirmó lo que predijo es sin sentido: \blockquote[{\cite[29]{anscombe2008faith:prophandmi}}: \enquote{This point needs stressing: someone who believes in a possibility of `precognition' comparable to memory is thereby rendered incapable of understanding the nature of prophecy at all}]{Este punto merece insistencia: alguien que cree en la posibilidad de la `precognición' como comparable a la memoria queda así hecho incapaz de entender del todo la naturaleza de la profecía}.

La imposibilidad de especificar con certeza qué quiso decir el profeta, o qué tenía en la mente al profetizar, impone una restricción severa al campo de lo que pueda considerarse incluso como posible profecía. Quedaría limitado a predicciones con nombres propios y predicados con un sentido bastante definitivo. La consecuencia de esto es importante: \blockquote[{\cite[31]{anscombe2008faith:prophandmi}}: \enquote{This considerations result in an interesting point: the critical principle that prophetical writings must have been clearly intelligible in their own times is \emph{itself} a denial of the possibility of all but prophecy of a very restricted type}]{Estas consideraciones resultan en un punto interesante: el principio crítico de que los escritos proféticos tienen que haber sido claramente inteligibles en su propio tiempo es \emph{en sí mismo} una negación de la posibilidad de todo menos un restringido tipo de profecía}. Lo cierto es, sin embargo, que para casi todas las profecías, tenerlas por cumplidas, es interpretarlas, y la clave para interpretarlas es una noción teológica.

Aquí podríamos preguntarnos, \enquote*{¿por qué me tendría que impresionar la profecía?}, \enquote*{¿por qué debería de interesarme?}. La respuesta a esto tiene que ver con el sentido o significado teológico de la profecía. \blockquote[{\cite[32]{anscombe2008faith:prophandmi}}: \enquote{a prophecy fulfiled, or a miracle done, is supposed to \emph{attest} something}]{una profecía cumplida, o un milagro realizado, se supone que \emph{testifica} algo}. Una predicción cumplida que no testifica nada más allá de que lo predicho se ha realizado, no tiene sentido profético.
%Esta consideración nos trae a una última afirmación relacionada con la profecía.

%Hay un sentido adicional a la noción de `accidental' distinto del empleado por Lessing. Decir que el cumplimiento de una predicción \enquote*{fue accidental} puede ser decir \enquote*{esto no fue una profecía}. Si alguien afirma algo sobre el futuro ---para ilustrar algo en una discusión, por ejemplo--- y se cumple la predicción, entonces hay algo de sentido en afirmar que \enquote*{el cumplimiento fue accidental}. Pero si esto mismo se afirmara como una profecía, entonces decir \enquote*{fue accidental que se cumpliera} puede significar que el hecho cumplido no fue lo que quiso decir la persona, como afirmó Lessing, o que \blockquote[{\cite[34]{anscombe2008faith:prophandmi}}: \enquote{we do not allow this to be prophecy, where `prophecy' has a \emph{theological} meaning}]{no reconocemos que esto sea profecía, donde `profecía' tiene un sentido \emph{teológico}}.

Las conclusiones a las que Anscombe llega después de su análisis pueden resumirse en dos cuestiones. En primer lugar se enfoca en el contraste entre dos posiciones desde las que una persona podría acercarse al argumento de las profecías y milagros. Una situación en la que puede estar una persona respecto de los milagros y profecías es como un observador imparcial e indiferente. Este solo tendría delante de él, como datos seguros, algunas profecías dispersas relacionadas con personas y ciudades; también contaría con noticias de milagros y del cumplimiento de profecías que, sin embargo, sería absurdo pretender que debería de estimar como ciertamente verdaderas.

Es otra la situación en la que, a juicio de Anscombe, ha de hallarse alguien que pueda ser interpelado por el argumento de los milagros y profecías: \blockquote[{\cite[35]{anscombe2008faith:prophandmi}}: \enquote{Only if a man is impressed by the Old Testament, to the extent of being inclined to take it as his teacher, has the argument from prophecies and miracles any serious weight.}]{Solo si un hombre queda impresionado por el Antiguo Testamento, hasta tal punto que esté inclinado a tomarlo como su maestro, tiene el argumento desde las profecías y los milagros algún peso serio}. Una persona que está en esta situación se encuentra en una posición solida y razonable, sin embargo, es tan específica y poco común hoy que puede explicar por qué el argumento no se encuentra tan presente en la apologética actual.

La crítica de Lessing es contra un alegado peso que debería de tener un argumento basado en los milagros y las profecías cumplidas y que para él no tiene la fuerza para justificar la creencia en el Cristianismo. El Vaticano~I alega, por su parte, que los milagros y profecías son sólidos argumentos externos. Anscombe propone que estos argumentos externos presuponen una posición específica de parte de quien pueda ser interpelado por ellos: \blockquote[{\cite[37]{anscombe2008faith:prophandmi}}: \enquote{That is to say: when St. Augustine said that the fulfilment of the prophecies in Christ was the greatest proof of his divinity, what he said was true; but the proof requires a very special postiton on the part of someone who is to consider it. That is why the kind of apologetic that Lessing argued against, which did not assume that position, was so vulnerable and stupid.}]{Es decir: cuando S. Agustín dijo que la realización de las profecías en Cristo es la mayor prueba de su divinidad, lo que dijo es verdadero; pero la prueba requiere una posición de parte de alguien que podría considerarla. Esta es la razón por la que el tipo de apologética en contra de la cual Lessing argumentó, en la que no se asume esta posición, queda tan vulnerable y estúpida}.

El argumento de los milagros y profecías cumplidas sí juega un papel razonable como atestación que justifica la creencia en Cristo para una persona que ha valorado suficientemente las enseñanzas del Antiguo Testamento como para tenerlo como una fuente de instrucción y ha formado su mente de acuerdo a él. Una persona que reconoce la solidez que pueden tener los milagros y profecías cumplidas como signo del cumplimiento de las promesas del Antiguo Testamento en Cristo podría entonces preguntarse sobre cómo se han transmitido estos relatos. Anscombe llega entonces a la siguiente conclusión: \blockquote[{\cite[37]{anscombe2008faith:prophandmi}}: \enquote{The role of miracles, which I have contended cannot possibly be accepted as certainly true ocurrences by the indiferent historian, seems to me to be this: if one is seriously entertaining the truth of the whole revelation in the way I have hinted at, the miracles are consonant. That God attested Christ by miracles is possible, if Christ is Christ ---i.e. is the Messiah promised in the Old Testament. Then the problem, how on earth these seemingly factual records came to be written, of such incredible things, is resolved by the hypothesis that they happened. \textelp{} But I repeat, it is not reasonable to ask an indiferent historian to solve this problem, of how such records came to be written; he \emph{can} reasonably just leave it unsolved.}]{El rol de los milagros, los cuales he argüido que no es posible aceptar como hechos ciertamente verdaderos por un historiador indiferente, me parece que es este: si alguien está seriamente considerando la verdad de toda la revelación en el modo que he sugerido, los milagros están en consonancia. Que Dios atestó a Cristo por medio de milagros es posible, si Cristo es Cristo ---es decir, es el Mesías prometido en el Antiguo Testamento. Entonces el problema, cómo es posible que estos informes aparentemente fácticos hayan llegado a quedar escritos, de estas cosas increíbles, se resuelve por la hipótesis de que ocurrieron. \textelp{} Pero repito, no es razonable pedir a un historiador indiferente que resuelva este problema, sobre cómo estos informes han llegado a quedar escritos; el \emph{puede} razonablemente dejarlo sin resolver}.

La segunda cuestión que Anscombe propone como conclusión tiene que ver con la noción misma de la atestación divina. El hecho de que una persona haga prodigios o pronuncie profecías que se cumplen no demuestra necesariamente que es un testigo de Dios o su enseñanza una atestación divina. Anscombe considera que hay un criterio adicional para justificar esa creencia: \blockquote[{\cite[38]{anscombe2008faith:prophandmi}}: \enquote{So far as I can see there has to be a thesis of natural theology, as I might call it, that if someone works `a sign and a wonder' or utters a prophecy which gets fulfilled, in God's name, then he is divinely attested. Now what does this rest on? It might rest on faith.}]{Hasta donde puedo ver tiene que haber una tesis de teología natural, como podría llamarla, que si alguien realiza `un signo y un prodigio' o pronuncia una profecía que queda cumplida, en el nombre de Dios, entonces está divinamente atestado. Ahora ¿en qué se basa esto? Puede estar respaldado por la fe}. Por ejemplo la fe en la promesa del Deuteronomio, de que vendrá otro profeta como Moisés, ofrece como criterio que antes de preguntarse si se ha cumplido lo profetizado, las enseñanzas de los profetas deberían ser tales que se pueda pensar que pertenecen a la verdad revelada por Moisés. Es entonces que si el profeta predice algo y se cumple, y si después de esto no trata de conducir al pueblo a la idolatría, se puede tomar su profecía como atestación divina. En este sentido se puede decir que el criterio para considerar a un profeta como testigo divino es una cuestión de fe. Sin embargo: \blockquote[{\cite[38]{anscombe2008faith:prophandmi}}: \enquote{if \textins{what} constitutes divine attestation is only learned by faith, what becomes of the `solid external arguments' of the Vatican decree?}]{si \textins{lo que} constituye una atestación divina solo se conoce por la fe, ¿en qué quedan los `sólidos argumentos externos' de la constitución del Vaticano?}. Si se tiene esta enseñanza en cuenta tendría que ser posible un criterio que no tenga como presupuesto la fe. Anscombe propone el siguiente análisis: \blockquote[{\cite[38]{anscombe2008faith:prophandmi}}: \enquote{I think the argument must be rather that if a prophet who is apparently teaching the truth, dares foretell something contingent, then this is presumption of him unless he has it from God and must say it. Now if he teaches a lie straight away afterwards, or if the thing does not happen, then he is proved presumptuous. But if he is not proved presumptuous, then we ought not to dare not to believe and obey him: so long as what he says does not conflict with the known truth.}]{Pienso que el argumento ha de ser más bien que si un profeta que está aparentemente enseñando la verdad, se atreve a predecir algo contingente, entonces esto es presunción suya excepto si lo ha recibido de Dios y debe decirlo. Ahora si enseña una mentira inmediatamente después, o si lo predicho no ocurre, entonces queda probado como presuntuoso. Pero si no es probado presuntuoso, entonces no deberíamos atrevernos a no creerle y obedecerle: siempre que lo que dice no esté en conflicto con la verdad conocida}.\label{subsec:argprof}

Anscombe termina haciendo una distinción; quizás podemos actuar según la profecía \enquote*{porque no deberíamos atrevernos a actuar de otro modo}, pero ¿sería esto justificación suficiente para afirmar una creencia?. Este criterio puede servir para remover dudas a la hora de hacer un juicio razonable sobre una alegada atestación divina, sin embargo, no ofrece una razón positiva para creer. Esta razón positiva, según alude Elizabeth, se encuentra en la consonancia de la profecía con la doctrina conocida: \blockquote[{\cite[39]{anscombe2008faith:prophandmi}}: \enquote{Surely one wants positive reason to believe, and not merely absence of positive reason to disbelieve? This, it seems to me, is correct, and goes with the thesis that in some sense there cannot be a prophet with a new doctrine.}]{¿Sin duda quisiéramos razón positiva para creer, y no solo ausencia de razones positivas para dudar? Esto, según mi parecer, es correcto, y va con la tesis de que en cierto sentido no puede haber un profeta con una nueva doctrina}.
