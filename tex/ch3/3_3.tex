%SECCIÓN 1: Fe y Conocimiento
\section{Fe y Conocimiento}

En Oscott College, el seminario de la Archidiócesis de Birmingham, se comenzaron a celebrar las conferencias llamadas \eng{Wiseman Lectures} en 1971. 
Para estas lecciones ofrecidas anualmente en memoria de Nicolás Wiseman se invitaba un ponente que tratara algún tema relacionado con la filosofía de la religión o alguna materia en torno al ecumenísmo.\footcite[cf.~][p.~7]{wisemanlects}

El 27 de octubre de 1975, para la quinta edición de las conferencias, Anscombe presentó una lección titulada simplemente \eng{Faith}. 
Allí planteaba la siguiente cuestión:\citalitlar{Quiero decir qué puede ser entendido sobre la fe por alguien que no la tenga; alguien, incluso, que no necesariamente crea que Dios existe, pero que sea capaz de pensar cuidadosa y honestamente sobre ella. 
Bertrand Russell llamó a la fe ``certeza sin prueba''. 
Esto parece correcto. 
Ambrose Bierce tiene una definición en su \eng{`Devil's Dictionary'}: ``La actitud de la mente de uno que cree sin evidencia a uno que habla sin conocimiento cosas sin parangón''.}
\citalitlar{¿Qué deberíamos pensar de esto?\footcite[p.~115]{faith}}



\subsection{``Solíamos creer que la fe católica era racional''}
Habían pasado casi diez años de la clausura del Concilio \mbox{Vaticano II}; Anscombe comenzó su ponencia recordando cómo en los finales de los años sesenta muchas homilias comenzaban: ``Solíamos creer que\ldots''. ``Soliamos creer ---escuchó una vez--- que no había peor pecado que faltar a misa el domingo''. Escuchar la frase le traía un desaliento alarmado, ya que la implicita oposición que se pretendía establecer con la expresión, por lo general, era desecertada. 

Ahora, hay un ``soliamos creer'' que se podía haber usado con algo de acierto. Hubo una tiempo en el que se profesó gran entusiasmo por la racionalidad. Quizás inspirado por las enseñanzas del Concilio Vaticano I contra el fideismo, pero ciertamente promovido por los estudios neo-tomistas. Se decía entre los entendidos que la fe Católica era racional, el problema era más bein

\subsection{El significado de la palabra fe}

En el trasfondo del análisis de Anscombe sobre la fe se halla otro trabajo suyo titulado \eng{`What Is It to Believe Someone?'}. 
Un eslabón importante entre ambas investigaciones se haya en la valoración del uso del concepto `fe'. 
Ella propone: 
\citalitinterlin{En la tradición donde el concepto tiene su origen, `fe' es la forma breve de `fe divina' y significa `creer a Dios'.} De esa manera fue usada la expresión, al menos por los pensadores cristianos. 
Según este modo de hablar 'fe' se distinguía como humana y divina. Fe humana era creer a una persona humana, fe divina era creer a Dios.

En el uso moderno 'fe' tiende a significar 'creencia religiosa' o 'religión'. Se le llama generalmente 'fe', por ejemplo, a la creencia en la existencia de Dios. 

<<Abrám creyó a Dios (\textgreek{ἐπίστευσεν τῷ Θεῷ}) y ésto se le contó como justicia.>>\footnote{Gn~15,6} De tal modo que es llamado 'padre de la fe'.\footnote{cfr.~Rm~4~y~Ga 3,7} He aquí una expresión sorprendente: <<creer a Dios>>. Abrahám creyó a Dios que su descendiencia sería tan numerosa como las estrellas del cielo, de este modo se describe su fe.
