%SECCIÓN 1: Fe y Conocimiento
\section{Fe y Conocimiento}

En Oscott College, el seminario de la Archidiócesis de Birmingham, se comenzaron a celebrar las conferencias llamadas \eng{Wiseman Lectures} en 1971. Para estas lecciones ofrecidas anualmente en memoria de Nicolás Wiseman se invitaba un ponente que tratara algún tema relacionado con la filosofía de la religión o alguna materia en torno al ecumenísmo.\footcite[cf.~][p.~7]{wisemanlects}

El 27 de octubre de 1975, para la quinta edición de las conferencias, Anscombe presentó una lección titulada simplemente \eng{Faith}. Allí planteaba la siguiente cuestión:\citalitlar{Quiero decir qué puede ser entendido sobre la fe por alguien que no la tenga; alguien, incluso, que no necesariamente crea que Dios existe, pero que sea capaz de pensar cuidadosa y honestamente sobre ella. Bertrand Russell llamó a la fe ``certeza sin prueba''. Esto parece correcto. Ambrose Bierce tiene una definición en su \eng{`Devil's Dictionary'}: ``La actitud de la mente de uno que cree sin evidencia a uno que habla sin conocimiento cosas sin parangón''.}
\citalitlar{¿Qué deberíamos pensar de esto?\footcite[p.~115]{faith}}

\subsection{El significado de la palabra fe}
<<Abrám creyó a Dios (\textgreek{ἐπίστευσεν τῷ Θεῷ}) y ésto se le contó como justicia.>>\footnote{Gn~15,6} De tal modo que es llamado 'padre de la fe'.\footnote{cfr.~Rm~4~y~Ga 3,7} He aquí una expresión sorprendente: <<creer a Dios>>. Abrahám creyó a Dios que su descendiencia sería tan numerosa como las estrellas del cielo. ¿Existe acaso la posibilidad de ``creer a Dios''? Este modo de hablar de la fe
