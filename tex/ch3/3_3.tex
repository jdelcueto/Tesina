%SECCIÓN 1: Fe y Conocimiento
\section{Fe y Conocimiento}

Quiero decir qué puede ser entendido sobre la fe por alguien que no la tenga; alguien, incluso, que no necesariamente crea que Dios existe, pero que sea capaz de pensar cuidadosa y honestamente sobre ella. Bertrand Russell llamó a la fe `certeza sin prueba'. Esto parece correcto. Ambrose Bierce tiene una definición en su \emph{Diccionario del Diablo}: `La actitud de la mente de uno que cree sin evidencia a uno que habla sin conocimiento cosas sin parangón'.

¿Qué deberíamos pensar de esto?
