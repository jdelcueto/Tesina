\subsection{¿Hay justificación para valorar un hecho histórico como atestación divina?}

En el apartado anterior destacamos la importancia que tiene para Anscombe la creencia de la comunicación de Dios como una `voz pública' y presente en la actividad humana del lenguaje. Este dato ha sido importante en el análisis de su comprensión de la fe, pero también lo encontramos presente en otras discusiones dentro de su obra. En \emph{Rules, Rights and Promises} emplea como premisa una conclusión de Hume: \blockquote[{\Cite[99]{anscombe1981erp:rrp}}: \enquote{``that promises have no \emph{force} antecedent to human conventions''}.]{``las promesas no tienen \emph{fuerza} ninguna antecedente a las convenciones humanas''} y sobre esto añade: \blockquote[{\Cite[99]{anscombe1981erp:rrp}}: \enquote{If this is found offensive, that will be by a misunderstanding. God himself can make no promises to man except in a human language}.]{Si esto parece ofensivo, es por no entenderlo bien. Dios mismo no puede hacer promesas al ser humano si no es en lenguaje humano}. En \emph{Authority in Morals} destaca que hay verdades morales que conocemos solamente porque Dios lo ha revelado
\footnote{\cite[48-49]{anscombe1981erp:am}: \enquote{some dogmatic beliefs are revealed and could not be known otherwise \textelp{} some of the facts, of what is the case, will help to determine moral truth \textelp{} some such truths about what is the case are revealed; original sin for example. There are also revealed some conditional promises, to disregard which is to despise the goodness of God. Both of these things lead us to infer the rightness of ascetism \textelp{} given the facts about original sin and the promise of the possibility of a man's joining his sufferings to those of Christ, the goodness of severe ascetical practices \textelp{} is obvious; there is no such thing as a revelation that such-and-such is good or bad not for any reason, not because of any facts, not because of any hopes or prospects, but simply: such-and-such is good to do, this is to be believed, and could not be known or inferred from anything else}.}.
%: \blockquote[{\Cite[48-49]{anscombe1981erp:am}}: \enquote{some dogmatic beliefs are revealed and could not be known otherwise \textelp{} some of the facts, of what is the case, will help to determine moral truth \textelp{} some such truths about what is the case are revealed; original sin for example. There are also revealed some conditional promises, to disregard which is to despise the goodness of God. Both of these things lead us to infer the rightness of ascetism \textelp{} given the facts about original sin and the promise of the possibility of a man's joining his sufferings to those of Christ, the goodness of severe ascetical practices \textelp{} is obvious; there is no such thing as a revelation that such-and-such is good or bad not for any reason, not because of any facts, not because of any hopes or prospects, but simply: such-and-such is good to do, this is to be believed, and could not be known or inferred from anything else.}]{algunas creencias dogmáticas son reveladas y no podrían ser conocidas de otro modo \textelp{} algunos datos, sobre lo que es de hecho, ayudan a determinar lo que es una verdad moral \textelp{} algunas de esas verdades acerca de lo que es de hecho son reveladas; el pecado original, por ejemplo. Hay también reveladas algunas promesas sujetas a condiciones, que ignorarlas conlleva un desprecio a la bondad de Dios. Estas dos cosas nos conducen a inferir la idoneidad del ascetismo \textelp{} dados los datos sobre el pecado original y la promesa de que es posible que una persona se una en sus sufrimientos a los de Cristo, la bondad de severas prácticas ascéticas \textelp{} es obvia; no hay tal cosa como una revelación de que esto es bueno o malo, no por ninguna razón, no por algún dato, no por algún prospecto o expectativa, simplemente: esto es bueno hacerlo, esto hay que creerlo, y no podría haber sido conocido o inferido de otra cosa}.
Hemos visto ya el ejemplo que toma prestado de la tradición rabínica, la `hija de la voz' o \emph{Bath Qol}, que describe como la experiencia de escuchar algo que salta a nosotros, que `habla a nuestra condición'. También cómo Dios habla en las enseñanzas de la Iglesia. Otro ejemplo ha sido el caso de los milagros realizados y las profecías cumplidas, que ella especifica que `dan testimonio', es decir, que testifican algo más allá de la realización del hecho profetizado o la acción milagrosa. También ha hablado del Antiguo Testamento que puede ser tomado como maestro y las enseñanzas de Jesús, como cuando nos dice que estamos unidos a él como los sarmientos a la vid, y así conocemos que él quiere hacernos partícipes de su vida divina. También las palabras de Jesús que son usadas por el sacerdote en la consagración y cambian el pan y el vino en el cuerpo y la sangre y son las palabras de Jesús en la última cena que él encomendó a los apóstoles que hicieran en memoria suya. Todos estos ejemplos que encontramos en las distintas discusiones de Anscombe son ocasiones en las que ella diría ``Dios habla''.

\label{subsec:fecorrel}
La comprensión de `fe' que ha sido estudiada por Elizabeth viene acompañada de esa noción importante: \blockquote[{\Cite[185]{conesa1994cc}}.]{\emph{Creo a Dios} presupone así que Dios ha hablado: \enquote{La naturaleza especial de la creencia (\emph{belief}) que es la fe consiste en ser una creencia en algo como revelado por Dios; es creencia en una proposición por la palabra de Dios. La fe, así definida, es un correlato de la revelación}}. Ella habla de esta noción como una `idea asombrosa' y considera que tenerla en cuenta enriquece la discusión y el pensamiento sobre la religión. Podemos decir que esta noción enriquece también su propia filosofía. Al interesarse por tener en cuenta en medio de sus discusiones cómo actuamos cuando creemos a Dios sobre alguna proposición está tomando de la riqueza de su vida de fe para iluminar su análisis, esto es así dado que: \blockquote[{\Cite[185]{conesa1994cc}}]{Aunque, como señala Anscombe, filosóficamente podemos encontrar problemas en el análisis del significado de la proposición \enquote{Dios habla}, para el cristiano su significado es claro. \emph{Creer a Dios} para el cristiano es creer su palabra, tener por verdad inquebrantable y regla de vida lo que nos ha revelado}. Elizabeth insiste en distintas discusiones que nuestra creencia en los misterios de la fe no consiste en creer teorías que son el producto de nuestro razonamiento o que pretendan ser explicaciones de fenómenos, sino que la fuente de estas creencias es otra, es \enquote*{aquello que creemos que viene a nosotros como palabra de Dios}. Según esto, insiste también en que los entendidos y estudiosos y sus argumentos no constituyen un fundamento que justifique estas creencias, sino que su lugar es disipar las objeciones. Así decía en \emph{Faith}: \enquote*{¿qué puede significar ``creer a Dios''? ¿Podría un hombre docto e inteligente informarme sobre la autoridad de su conocimiento, que la evidencia es que Dios ha hablado? No. El único uso posible para un hombre docto e inteligente es como \emph{causa removens prohibens}}.

En esta clave podemos entender la discusión que tratamos en este apartado. Anscombe, como `mujer docta e inteligente', no ofrece sobre la autoridad de su conocimiento un informe sobre lo que constituye una evidencia de que Dios ha hablado, más bien se enfoca en remover ciertas objeciones que pretenden demostrar que el fenómeno `Dios habla' no es posible. La discusión relacionada con esto se encuentra en un artículo que hemos visto, \emph{Prophecy and Miracles} y también podemos tener en cuenta otro artículo no publicado con el título \emph{Hume on Miracles}. La aportación de Anscombe constituye, más que una serie de conclusiones, una línea de reflexión abierta que podemos resumir en dos puntos. El primero sobre objeciones ante la idea de los milagros y profecías cumplidas como sólidos argumentos externos de la revelación. El segundo sobre la objeción contra el testimonio de los milagros y profecías cumplidas como signo de probabilidad de los hechos que narra.

\subsubsection{\enquote*{Una `tesis de teología natural' sobre la atestación divina.}}

En la introducción del libro de Benedicto XVI \emph{Jesús de Nazaret} la promesa del Deuteronomio aparece como clave para entender la figura de Jesús. Dios promete por medio de Moisés: \enquote{El Señor, tu Dios, te suscitará un profeta como yo de entre tus hermanos. A él le escucharéis} (Dt 18,15) y sin embargo, el pueblo de la Alianza queda en la espera del cumplimiento de esta promesa: \enquote{Pero no surgió en Israel otro profeta como Moisés, con quien el Señor trataba cara a cara\ldots} (Dt 34,10). Lo prometido por Dios se realiza en Cristo: \blockquote[{\Cite[28]{ratzinger2007jdenaz}}.]{En Jesús se cumple la promesa del nuevo profeta. En Él se ha hecho plenamente realidad lo que en Moisés era sólo imperfecto: Él vive ante el rostro de Dios no sólo como amigo, sino como Hijo; vive en la más íntima unidad con el Padre. Sólo partiendo de esta afirmación se puede entender verdaderamente la figura de Jesús, tal como se nos muestra en el Nuevo Testamento}

Anscombe hace referencia a esta promesa del Deuteronomio de dos maneras en su discusión sobre las objeciones de Lessing en \emph{Prophecy and Miracles}. Por un lado es un criterio de la fe; creemos la promesa del Señor y creemos que se cumple en Jesús. Este juicio respaldado por la fe nos permite reconocer en los signos y profecias de Jesús una atestación divina de que él es el Mesías prometido en el Antiguo Testamento. Por otra parte, el texto del Deuteronomio sirve como criterio para lo que Elizabeth llama una `tesis de teología natural' como argumento para descartar la falsa profecía. El fragmento al que se refiere afirma: \enquote{Y si dices en tu corazón: ``¿Cómo reconoceré una palabra que no ha dicho el Señor?''. Cuando un profeta hable en nombre del Señor y no suceda ni se cumpla su palabra, es una palabra que no ha dicho el Señor: ese profeta habla por arrogancia, no le tengas miedo} (Dt 18, 21-22). Desde esta enseñanza ella propone el argumento que hemos visto (III, \S\ref{subsec:argprof}, p.~\pageref{subsec:argprof}).
%: \blockquote[]{si un profeta que está aparentemente enseñando la verdad, se atreve a predecir algo contingente, entonces esto es presunción suya excepto si lo ha recibido de Dios y debe decirlo. Ahora si enseña una mentira inmediatamente después, o si lo predicho no ocurre, entonces queda probado como presuntuoso. Pero si no es probado presuntuoso, entonces no deberíamos atrevernos a no creerle y obedecerle: siempre que lo que dice no esté en conflicto con la verdad conocida}.
Elizabeth propone ese argumento como un criterio en el espíritu de la expresión de \emph{Dei Filius}. La enseñanza de alguien que realiza prodigios, o de un profeta que no resulta presuntuoso puede ser tomada como argumento externo, \enquote*{signo ciertísimo y acomodado a la inteligencia de todos, de la revelación divina}. El criterio sin embargo, no deja de ser un argumento para descartar la falsedad, no para afirmar la veracidad. En esto es un argumento similar al que Anscombe usa para hablar de los misterios, los cuales creemos con el supuesto de que cualquier pretendida demostración definitiva de que son una completa contradicción puede ser rebatida. Podemos justificar nuestra creencia en una profecía cumplida bajo el supuesto de que el profeta no ha resultado presuntuoso, y en esto tenemos razones para no dudar.

Además de este argumento, la reflexión de Anscombe nos permite formular la pregunta: ¿qué postura puede ser más representativa de una disposción razonable ante los testimonios de milagros y profecías cumplidas?, ¿la del historiador indiferente o la de quien ha valorado el Antiguo Testamento como para tomarlo como maestro? En la reflexión de Lessing el historiador indiferente es representativo de la disposción más razonable; no encuentra fuerza en el testimonio de estos hechos extraordinarios y por tanto no ve en ellos razones para considerarlos como argumentos sólidos que justifiquen la creencia en los hechos que narran.

En el artículo de Anscombe el que ha tomado el Antiguo Testamento como maestro puede cuestionarse ¿cómo es posible que estos informes pretendidamente fácticos, sobre estos hechos extraordinarios, hayan quedado escritos? y considerar que esta pregunta se resuelve por la hipótesis de que los hechos ocurrieron. Los milagros realizados por Jesús y las profecías cumplidas en él son para esta persona testimonio de que Jesús es el Mesías. Anscombe añade que una persona que está en esta situación está en una posición sólida y razonable. Si tenemos en cuenta lo que Elizabeth ha dicho sobre la estructura propia del creer en hechos históricos que forman parte del conocimiento tradicional, que los informes son justificación para creer en el hecho, y que, a su vez, la creencia en el hecho es la justificación para creer en la transmisión intermedia (Cf. III, \S\ref{subsec:notchain}, p.~\pageref{subsec:notchain}); podríamos decir que una persona que recibe estos informes sobre milagros y profecías puede considerar la hipótesis de que la razón de que exista esta tradición intermedia es que los hechos ocurrieron. En esto estaría realizando un juicio sólido. Así, aún cuando el historiador apático puede razonablemente dejar sin resolver la pregunta sobre el hecho de que existan estos informes, su posición no es representativa de la única disposición razonable.

Cabe destacar un punto adicional en la descripción que Anscombe hace de los fundamentos de nuestra certeza en la creencia que podemos tener sobre los testimonios o informes de que Dios ha hablado, específicamente en la solidez de los testimonios de milagros o profecías cumplidas como argumentos externos de la Revelación. Como vimos en el apartado anterior, Elizabeth propone que hay certezas históricas que forman parte de la estructura de nuestro conocimiento tradicional. En \emph{Prophecy and Miracles}, habla también de certezas históricas que no pueden ser razonablemente afirmadas como falsas, puesto que el tiempo para refutarlas ha pasado. De estas, consideradas en general, no es común que se encuentre algo que las contradiga definitivamente y \enquote*{la mayor parte de ellas debe ser verdadera}, aunque considerada alguna de ellas en particular, no es posible afirmarlas como completamente ciertas. Este tipo de datos, a juicio de Anscombe, son justificación suficiente para afirmar certezas absolutas. Las afirmaciones históricas relacionadas con Jesús están compuestas por proposiciones de estas dos categorías y como tal no carecen de justificación, sino que son apoyadas por el grado de certeza que puede atribuírsele al conocimiento tradicional o al estatuto general de las afirmaciones históricas cuyo tiempo de refutar ha pasado.

\subsubsection{\enquote*{Una persona sabia ajusta sus creencias a las evidencias.}}

En \emph{Prophecy and Miracles} Anscombe resume la preocupación principal de Lessing como un asunto de probabilidad. Si lo que pretende ser un argumento sólido para justificar esta serie de creencias es poco menos que probable, ¿cómo puede ser razonable sostenerlas?. En \emph{Hume on Miracles} ella también toma el consejo de Hume al `sabio y entendido' como una cuestión de probabilidad. El criterio sugerido por Hume, como vimos (I, \S\ref{subsec:humarg}, p.~\pageref{subsec:humarg}),
pretendía ofrecer un examen definitivo para descartar `ideas supersticiosas'.
%fue: \blockquote[]{ningún testimonio es suficiente para establecer un milagro, excepto si el testimonio es de tal tipo, que su falsedad sea más milagrosa que el hecho que se esfuerza por establecer; e, incluso en este caso, hay una mutua destrucción de argumentos; y el superior sólo nos da certeza apropiada al grado de fuerza que permanece después de restar el inferior}.
Para Anscombe este capítulo del \emph{Enquiry} de Hume es `brillante propaganda'\footnote{\cite[46]{anscombe2008faith:hummi}: \enquote{Broad may say, like someone criticising a student's essay, that Hume doesn't in this essay maintain his otherwise `extremely high standards'; he mistook what Hume was at. The essay is brilliant propaganda}.}. En su análisis del argumento, ella repasa siete críticas que considera sólidas contra el razonamiento de Hume. Tres de ellas son relevantes para nuestra discusión. En primer lugar, el argumento de Hume busca demostrar que el carácter milagroso de un evento es razón suficiente para rechazar cualquier testimonio sobre este. A los críticos de Hume esto les parece una conclusión extraña para un argumento que comienza con la tesis de que \enquote*{un hombre sabio adecúa su creencia a la evidencia}\footnote{\cite[Cf.][44]{anscombe2008faith:hummi}: \enquote{Hume's aim is to procure (what has indeed been procured) that the miraculous character of an event shall be \emph{sufficient} reason to reject the story of it having ocurred without investigation of any evidence. This is a strange termination of an argument which starts with the thesis that a wise man proportions his belief to the evidence}.}. En segundo lugar, Hume se equivoca en su descripción del rol del testimonio en nuestro conocimiento. Para él es el hábito lo que nos permite darle algún crédito a lo que nos dice un testigo. Según esta descripción, en el caso del testimonio de un hecho extraordinario, la alta probabilidad de que el testimonio sea verdadero compite con la poca probabilidad de un hecho que es extraño a nuestra experiencia habitual. El ejemplo de Anscombe para ilustrar la crítica contra esto es el siguiente: \enquote*{Bueno, yo no solo raramente, sino nunca, he experimentado un terremoto; sin embargo no hay conflicto, o principio de experiencia que en este caso me ofrezca un `grado de garantía contra el hecho' que los testigos de terremotos intentan establecer}\footnote{\cite[Cf.][44]{anscombe2008faith:hummi}: \enquote{Hume misdescribes the role of testimony in human knowledge. `The reason', he says, `why we place any credit in witnesses and historians, is not derived from any \emph{connexion}, which we perceive \emph{a priori}, between testimony and reality, but because we are accustomed to find a conformity between them. But when the fact attested is such a one as has seldom fallen under our observation, here is a contest of two opposite experiences.' Well, I have not merely not often, but never, experienced an earthquake; yet there is no conflict, no principle of experience which in this case gives me a `degree of assurance against the fact' that witnesses to earthquakes endeavour to establish}.}. En tercer lugar, según la descripción de Hume sobre lo que es `creer', es imposible creer en milagros. La creencia de un hecho depende de la conjunción habitual de un objeto que tenemos ante nuestros sentidos o memoria en relación con otros objetos. No es razonable creer en la religión cristiana si no es apoyados en la evidencia de los milagros. Sin embargo, la creencia en la veracidad de estos es un milagro mismo que opera la fe en nosotros dándonos la determinación para creer lo que es contrario a la experiencia habitual\footnote{\cite[Cf.][45-46]{anscombe2008faith:hummi}: \enquote{All belief of matter of fact or real existence is derived merely from some object, present to the memory or senses, and a customary conjunction between that and some other object. \textelp{} \emph{Christian Religion} not only was at first attended with miracles, but even at this day cannot be believed by any reasonable person without one. Mere reason is insufficient to convince us of its veracity: And whoever is moved by \emph{Faith} to assent to it, is conscious of a continued miracle in his own person, which subverts all the principles of his understanding, and gives him a determination to believe what is most contrary to custom and experience}.}. Si esto es así no sería necesario un criterio para valorar si tenemos justificación para creer el testimonio de un milagro.

Tras repasar estas y otras críticas al argumento de Hume, Anscombe juzga que hay algo más que decir contra el criterio de que para atribuir algún grado posible de certeza al testimonio de un milagro su falsedad debe ser más milagrosa que el hecho que narra. La crítica de Anscombe no va dirigida hacia la probabilidad de los hechos, sino contra la idea de Hume de que ningún testimonio puede ofrecer justificación para juzgar que un hecho milagroso o profecía cumplida ha ocurrido.

Hume no distingue entre milagros y hechos extraordinarios, en este sentido, su argumento es aplicable en cualquier caso de testimonio de un hecho poco probable según lo habitual o según el trasfondo de un contexto. Anscombe piensa que \blockquote[{\Cite[47]{anscombe2008faith:hummi}}: \enquote{Hume's argument that the more improbable the event the less weight has testimony to it is sound enough}.]{el argumento de Hume de que mientras más improbable un evento menor es el peso que tiene el testimonio de este, es suficientemente sólido}. Ahora bien, la tesis de Hume no sería adecuada si se trata de un hecho imposible. Es decir, si se trata de una imposibilidad absoluta, no hay probabilidades necesitadas de justificación. Entonces el criterio de Hume se aplica ante creencias sobre probabilidades dentro de un límite. Así considerado, el argumento viene a decir que \blockquote[{\Cite[47]{anscombe2008faith:hummi}}: \enquote{testimony cannot add to probability at all where lying or deceived testimony is more probable than the event}.]{el testimonio no puede añadir probabilidad cuando la mentira o un testimonio engañoso es más probable que el hecho}. A Elizabeth le parece que esto hace falaz el criterio de Hume. Si se considera un testimonio acerca de un hecho extraordinario se está reconociendo al hecho, al menos retóricamente, un grado de probabilidad dentro de un límite. Entonces la pregunta sobre si el testimonio tiene peso para justificar la creencia en el hecho se hace desde el juicio de que: \blockquote[{\Cite[47]{anscombe2008faith:hummi}}: \enquote{the ratio of the probability that the event will be reported \emph{if} it has ocurred (near certainty for some events of an extraordinary nature, if publicly ocurring) to the probability that, if has \emph{not} ocurred, that particular lie should be invented, may be high. It is in this ratio that the consequent odds (odds after testimony) exceed the antecedent odds in favour of the event}.]{la ratio de la probabilidad de que el hecho sea reportado \emph{si} ha ocurrido (cerca de la certeza para ciertos eventos de naturaleza extraordinaria, si ocurrieron públicamente) contra la probabilidad de que, si \emph{no} ha ocurrido, se invente esta mentira particular, puede ser alta. Es en esta ratio donde las probabilidades consecuentes (las probabilidades tras el testimonio) exceden las probabilidades antecedentes en favor del hecho}. No es \enquote*{que la falsedad del testimonio sea más milagrosa que el hecho} lo que nos justifica para juzgar la probabilidad de un hecho desde la existencia del testimonio que lo narra, sino que la existencia del testimonio mismo representa una justificación para juzgar la probabilidad del hecho narrado. Esto está en sintonía con lo que Anscombe ha dicho sobre una persona que tiene una disposición positiva hacia la Sagrada Escritura y responde a la pregunta sobre cómo ha llegado a suceder que estos informes aparentemente fácticos hayan quedado escritos sobre estos hechos extraordinarios con la hipótesis de que verdaderamente ocurrieron. Esta persona tiene una justificación razonable.
