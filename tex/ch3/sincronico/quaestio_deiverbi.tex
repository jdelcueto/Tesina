\subsection{¿Hay justificación para valorar un hecho histórico como atestación divina?}

En el apartado anterior destacamos la importancia que tiene para Anscombe la creencia de la comunicación de Dios como una `voz pública' y presente en la actividad humana del lenguaje. Este dato ha sido importante en el análisis de su comprensión de la fe, pero también lo encontramos presente en otras discusiones dentro de su obra. En \emph{Rules, Rights and Promises} emplea como premisa una conclusión de Hume: \blockquote[{\cite[99]{anscombe1981erp:rrp}}: \enquote{``that promises have no \emph{force} antecedent to human conventions''}]{``las promesas no tienen \emph{fuerza} ninguna antecedente a las convenciones humanas''} y sobre esto añade: \blockquote[{\cite[99]{anscombe1981erp:rrp}}: \enquote{If this is found offensive, that will be by a misunderstanding. God himself can make no promises to man except in a human language.}]{Si esto parece ofensivo, es por no entenderlo bien. Dios mismo no puede hacer promesas al ser humano si no es en lenguaje humano.} En \emph{Authority in Morals} destaca que hay verdades morales que conocemos solamente porque Dios lo ha revelado: \blockquote[{\cite[48--49]{anscombe1981erp:am}}: \enquote{some dogmatic beliefs are revealed and could not be known otherwise \textelp{} some of the facts, of what is the case, will help to determine moral truth \textelp{} some such truths about what is the case are revealed; original sin for example. There are also revealed some conditional promises, to disregard which is to despise the goodness of God. Both of these things lead us to infer the rightness of ascetism \textelp{} given the facts about original sin and the promise of the possibility of a man's joining his sufferings to those of Christ, the goodness of severe ascetical practices \textelp{} is obvious; there is no such thing as a revelation that such-and-such is good or bad not for any reason, not because of any facts, not because of any hopes or prospects, but simply: such-and-such is good to do, this is to be believed, and could not be known or inferred from anything else.}]{algunas creencias dogmáticas son reveladas y no podrían ser conocidas de otro modo \textelp{} algunos datos, sobre lo que es de hecho, ayudan a determinar lo que es una verdad moral \textelp{} algunas de esas verdades acerca de lo que es de hecho son reveladas; el pecado original, por ejemplo. Hay también reveladas algunas promesas sujetas a condiciones, que ignorarlas conlleva un desprecio a la bondad de Dios. Estas dos cosas nos conducen a inferir la idoneidad del ascetismo \textelp{} dados los datos sobre el pecado original y la promesa de que es posible que una persona se una en sus sufrimientos a los de Cristo, la bondad de severas prácticas ascéticas \textelp{} es obvia; no hay tal cosa como una revelación de que esto es bueno o malo, no por ninguna razón, no por algún dato, no por algún prospecto o expectativa, simplemente: esto es bueno hacerlo, esto hay que creerlo, y no podría haber sido conocido o inferido de otra cosa.} Hemos visto ya el ejemplo que toma prestado de la tradición rabínica, la `hija de la voz' o Bath Qol, que describe como la experiencia de escuchar algo que salta a nosotros, que `habla a nuestra condición'. También cómo Dios habla en las enseñanzas de la Iglesia. Otro ejemplo ha sido el caso de los milagros realizados y las profecias cumplidas, que ella especifica que `dan testimonio', es decir, que testifican algo más allá de la realización del hecho profetizado o la acción milagrosa. También ha hablado del Antiguo Testamento como uno que puede ser tomado como maestro y las enseñanzas de Jesús, como cuando nos dice que estamos unidos a él como los sarmientos a la vid, y así conocemos que él quiere participarnos su vida divina. También las palabras de Jesús que son usadas por el sacerdote en la consagración y cambian el pan y el vino en el cuerpo y la sangre y son las palabras de Jesús en la última cena que él encomendó a los apostoles que hicieran en memoria suya. Todos estos ejemplos que encontramos en las distintas discusiones de Anscombe son ocasiones en las que ella diría ``Dios habla''.

La comprensión de `fe' que ha sido estudiada por Elizabeth viene acompañada de esa noción importante: \blockquote[{\cite[185]{conesa1994cc}}]{\emph{Creo a Dios} presupone así que Dios ha hablado: \enquote{La naturaleza especial de la creencia (\emph{belief}) que es la fe consiste en ser una creencia en algo como revelado por Dios; es creencia en una proposición por la palabra de Dios. La fe, así definida, es un correlato de la revelación}.} Ella habla de esta noción como una `idea asombrosa' y considera que tenerla en cuenta enriquece la discusión y el pensamiento sobre la religión. Podemos decir que esta noción enriquece también su propia filosofía. Al interesarse por tener en cuenta en medio de sus discusiones cómo actuamos cuando creemos a Dios sobre alguna proposición está tomando de la riqueza de su vida de fe para iluminar su análisis, esto es así dado que: \blockquote[{\cite[185]{conesa1994cc}}]{Aunque, como señala Anscombe, filosóficamente podemos encontrar problemas en el análisis del significado de la proposición \enquote{Dios habla}, para el cristiano su significado es claro. \emph{Creer a Dios} para el cristiano es creer su palabra, tener por verdad inquebrantable y regla de vida lo que nos ha revelado.} Elizabeth insiste en distintas discusiones que nuestra creencia en los misterios de la fe no consiste en creer teorías que son el producto de nuestro razonamiento o que pretendan ser explicaciones de fenómenos, sino que la fuente de estas creencias es otra, es \enquote*{aquello que creemos que viene a nosotros como palabra de Dios}. Según esto, insiste también en que los entendidos y estudiosos y sus argumentos no constituyen un fundamento que justifique estas creencias, sino que su lugar es disipar las objeciones. Así decía en \emph{Faith}: \enquote*{¿qué puede significar ``creer a Dios''? ¿Podría un hombre docto e inteligente informarme sobre la autoridad de su conocimiento, que la evidencia es que Dios ha hablado? No. El único uso posible para un hombre docto e inteligente es como \emph{causa removens prohibens}.}

En esta clave podemos entender la discusión de la cuestión que tratamos en este apartado. Anscombe, como `mujer docta e inteligente' no ofrece, sobre la autoridad de su conocimiento, un informe sobre lo que constituye una evidencia de que Dios ha hablado, más bien se enfoca en confrontar ciertas objeciones que pretenden demostrar que el fenómeno `Dios habla' no es posible. La discusión relacionada con esto se encuentra en un artículo que hemos visto, \emph{Prophecy and Miracles} y también podemos tener en cuenta otro artículo no publicado con el título \emph{Hume on Miracles}. La aportación de Anscombe constituye, más que una serie de conclusiones, una línea de reflexión abierta que podemos resumir en dos puntos. El primero sobre objeciones ante la idea de los milagros y profecías cumplidas como sólidos argumentos externos de la revelación. El segundo sobre la objeción contra el testimonio de los milagros y profecías cumplidas como signo de probabilidad de los hechos que narra.

\subsubsection{\enquote*{Una `tesis de teología natural' sobre la atestación divina.}}

En la introducción de \emph{Jesús de Nazaret} la promesa de Deuteronomio aparece como clave para entender la figura de Jesús. Dios promete por medio de Moisés: \enquote{El Señor, tu Dios, te suscitará un profeta como yo de entre tus hermanos. A él le escucharéis} (Dt 18,15) y sin embargo, el pueblo de la Alianza queda en la espera del cumplimiento de esta promesa: \enquote{Pero no surgió en Israel otro profeta como Moisés, con quien el Señor trataba cara a cara\ldots} (Dt 34,10). Lo prometido por Dios se realiza en Cristo: \blockquote[{\cite[28]{ratzinger2007jdenaz}}]{En Jesús se cumple la promesa del nuevo profeta. En Él se ha hecho plenamente realidad lo que en Moisés era sólo imperfecto: Él vive ante el rostro de Dios no sólo como amigo, sino como Hijo; vive en la más íntima unidad con el Padre. Sólo partiendo de esta afirmación se puede entender verdaderamente la figura de Jesús, tal como se nos muestra en el Nuevo Testamento}

Anscombe hace referencia a esta promesa del Deuteronomio de dos maneras en su discusión de las objeciones de Lessing. Por un lado es un criterio de la fe; creemos la promesa del Señor y creemos que se cumple en Jesús. Este juicio respaldado por la fe nos permite reconocer en los signos y profecias de Jesús una atestación divina de que él es el Mesías prometido en el Antiguo Testamento. Por otra parte el texto del Deuteronomio sirve como criterio para lo que Elizabeth llama una `tesis de teología natural' como argumento para descartar la falsa profecía. El fragmento al que se refiere es: \enquote{Y si dices en tu corazón: ``¿Cómo reconoceré una palabra que no ha dicho el Señor?''. Cuando un profeta hable en nombre del Señor y no suceda ni se cumpla su palabra, es una palabra que no ha dicho el Señor: ese profeta habla por arrogancia, no le tengas miedo.} (Dt 18, 21--22) Desde esta enseñanza es que ella propone el argumento que hemos visto: \blockquote{si un profeta que está aparentemente enseñando la verdad, se atreve a predecir algo contingente, entonces esto es presunción suya excepto si lo ha recibido de Dios y debe decirlo. Ahora si enseña una mentira inmediatamente después, o si lo predicho no ocurre, entonces queda probado como presuntuoso. Pero si no es probado presuntuoso, entonces no deberíamos atrevernos a no creerle y obedecerle: siempre que lo que dice no esté en conflicto con la verdad conocida.}

\subsubsection{\enquote*{Una persona sabia ajusta sus creencias a las evidencias.}}
