\subsection{¿Tiene carácter `veritativo' el lenguaje religioso?}

\subsubsection{\enquote{El modo más sencillo de expresar lo que es la transubstanciación\ldots}}

A lo largo de los artículos de Anscombe que hemos discutido hay una posición clara: sostener esas creencias que dentro de la fe católica llamamos `misterios' no consiste en una disposición a profesar la contradicción. Decir \enquote*{Esto puede ser demostrado como falso, pero aún así lo creo}, no es declarar para nada una actitud de fe. En esto, creer un misterio no es lo mismo que creer cosas ilógicas o sin sentido.

Junto a esta noción, se encuentra también en ella la interesante idea de \enquote*{expresar el misterio}, posibilidad que caracteriza diciendo: \enquote*{puede ser enseñado}; a un niño, por ejemplo. Y en esto también hay algo que empieza a diferenciar una afirmación relacionada con una creencia de fe como distinta de afirmaciones que no expresan pensamiento o que no tienen significado.

Dentro de la obra de Anscombe, además, todo el lenguaje relacionado con la religión no está compuesto solamente por afirmaciones que expresan creencias en misterios de fe, sino que en sus discusiones utiliza también proposiciones de teología natural y proposiciones acerca de las presuposiciones involucradas en creer a Dios.

Vamos a profundizar más en estos aspectos del pensamiento de Anscombe sobre las características que describen el modo en el que el lenguaje religioso es lenguaje significativo. Para esto será útil comparar su perspectiva con la de Wittgenstein.

En el \emph{Tractatus}, termina ocupando un lugar prominente lo \enquote*{inexpresable, lo que se muestra; que es lo místico} (\S6.522). En esta categoría de `lo que no puede ser dicho pero queda mostrado', se encuentran las proposiciones éticas y estéticas: \blockquote[{\Cite[\S6.421]{wittgenstein1922tractatus}}: \enquote{It is clear that ethics cannot be expressed. Ethics is transcendental. (Ethics and aesthetics are one.)}.]{Queda por tanto claro que la ética no puede expresarse mediante palabras. La ética es transcendental. (La ética y la estética son una y la misma.)}. Wittgenstein tuvo gran interés por esto que consideraba una tendencia de la mente humana: el deseo de poner en palabras lo que no puede ser dicho. Esta tendencia la reconocía en el corazón de la ética, cuyas proposiciones juzgaba como sinsentido, aunque su actitud hacia ellas era de respeto\footnote{\cite[Cf.][211]{teichmann2008ans}: \enquote{In his `Lecture on Ethics' of 1929, he cites certain experiences, saying of them that their natural expression takes the form of utterances which can only count as nonsensical, as attempts to `\emph{go beyond} the world and that is to say beyond significant language'. These experiences and utterances he takes to be at the heart of ethics, about which he writes:`it is a document of a tendency in the human mind which I personally cannot help respecting deeply and I would not for my life ridicule it'}.}.

A la ética y la estética añadiría la religión. Este tipo de proposiciones también intentan ir más allá del mundo y de lo que puede considerarse como lenguaje significativo, y por tanto estos intentos de poner en palabras lo que no puede ser dicho también constituyen afirmaciones sin sentido. Sin embargo, su actitud hacia las afirmaciones religiosas ---así como hacia la ética--- era tomarlas en serio, con respeto. En este sentido puede entenderse la anecdota recordada por Anscombe en \emph{The Question of Linguistic Idealism}. Wittgenstein prefería tratar con respeto las proposiciones religiosas en tanto que contradictorias, puesto que rechazaba la idea de considerar la religión como racional. Así es que el intento de presentar la religión como algo que pudiera ser visto racionalmente le parecía que era como encerrar un objeto irregular dentro de una lisa esfera de cristal; las irregularidades no dejan de ser visibles, así que consideraba más adecuado atender el objeto sin disimularle sus aristas.

Anscombe se apoya en esto cuando dice que la  actitud de Wittgenstein al conjunto de la religión, en cierto modo, la asimilaba al misterio. Rechazaba las proposiciones de teología natural y no cabían dentro de su sistema de pensamiento. Hemos visto cómo el \emph{Círculo de Viena} articuló un rechazo sistemático de las proposiciones teológicas apoyados en el \emph{Tractatus} de Wittgenstein. Anscombe reconoce que dentro del pensamiento de Ludwig no es posible la teología natural en particular, pero ante la interpretación del \emph{Círculo} se mostró crítica:\blockquote[{\Cite[78]{anscombe1959iwt}}: \enquote{Here it is worth remarking that the truth of the \emph{Tractatus} theory would be death to natural theology; not because of any jejune positivism or any `verificationism', but simply because of the picture theory of the `significant proposition'. For it is essential to this that the picturing proposition has two poles, and in each sense it represents what may perfectly well be true. Which of them is true is just what \emph{happens} to be the case. But in natural theology this is an impermissible notion; its propositions are not supposed to be the ones that happen to be true out of pairs of possibilities; nor are they supposed to be logical or mathematical propositions either}.]{Aquí vale la pena comentar que la verdad de la teoría del \emph{Tractatus} conllevaría la muerte de la teología natural; no por ningún inmaduro positivismo o ningún `verificacionismo', sino simplemente por la teoría de la imagen relacionada con lo que es una `proposición significativa'. Puesto que es esencial para esta que la proposición que ofrece una imagen tenga dos polos, y en cada sentido represente lo que pudiera ser perfectamente bien la verdad. Pero en la teología natural esto es una noción inadmisible; sus proposiciones no son tales que se supone que son las que de hecho son verdaderas de entre un par de posibilidades; ni se tiene por supuesto que sean proposiciones lógicas o matemáticas tampoco}. Aún cuando, en la segunda etapa de su pensamiento, Wittgenstein desarrolló nuevas ideas en su modo de comprender el lenguaje, no dejó de pensar que no es posible el intento de razonar desde los objetos del mundo hacia algo fuera de este, como se pretende en las afirmaciones de la teología natural. En \emph{The Question of Linguistic Idealism} Anscombe ofrece como evidencia de esta objeción una cita de \emph{Observaciones sobre los fundamentos de la matemática}: \blockquote[{\Cite[VII, 25]{wittgenstein1956remmathes}}.
%: \enquote{A \emph{logical} conclusion is being drawn, when no experience can contradict the conclusion without contradicting the premises. I.e., when the inference is only a movement within the means of representation.}]{Se hace una inferencia \emph{lógica} cuando ninguna experiencia puede contradecir la conclusión porque entonces contradiría las premisas. Es decir, cuando la inferencia es solo un movimiento en los medios de la representación}.
]{Se hace una inferencia \emph{lógica} cuando ninguna experiencia puede contradecir la conclusión porque entonces contradiría las premisas. Es decir, cuando la inferencia es sólo un movimiento en los medios de representación}.
Elizabeth relaciona este comentario con la premisa de \emph{Investigaciones Filosóficas}: \blockquote[{\Cite[\S126]{wittgenstein1953phiinv}}: \enquote{Philosophy just puts everythig before us, and neither explains nor deduces anything. --- Since everything lies open to view, there is nothing to explain. For whatever may be hidden is of no interest to us. The name ``philosophy'' might also be given to what is possible \emph{before} all new discoveries and inventions}.]{La filosofía meramente expone todo ante nosotros, y no explica ni deduce nada.\,---\,Ya que todo está abiertamente a la vista, no hay nada que explicar. Pues lo que sea que esté oculto no es de ningún interés para nosotros. Se podría llamar también ``filosofía'' a lo que es posible antes de todos los nuevos descubrimientos e invenciones}. Tales afirmaciones representan nociones propias de la etapa más tardía del pensamiento de Ludwig. Para él la actividad filosófica debe realizar sus inferencias dentro de los medios de representación que pueden ser usados por el lenguaje. Los elementos que componen estos medios de representación no se obtienen desde deducciones de realidades ocultas, sino que están a la vista en la actividad misma de usar el lenguaje. Según esto, el intento de razonar desde los objetos del mundo sobre algo más allá del mundo está en contra de lo que Wittgenstein llamaría filosofía.

Quizás por el tono anecdótico de algunas de las descripciones que Anscombe narra sobre la actitud de Wittgenstein hacia la religión, resulta ambiguo si sus creencias constituyen una posición filosófica o solamente una opinión personal. Ciertamente, a lo largo de su vida, la cuestión de la religión fue para él un asunto personalmente problemático y así no deja de aparecer como un tema cargado de cierta ambigüedad en sus reflexiones filosóficas. Hemos escuchado de Elizabeth sobre la actitud de Ludwig hacia el argumento Agustiniano en su ponencia en el \emph{Moral Science Club} y él mismo ha expresado la dificultad que representa una creencia católica como lo es la Eucaristía en la discusión en \emph{Sobre la Certeza}. En relación con este segundo ejemplo es interesante el comentario de Ray Monk, que en su biografía de Wittgenstein atribuye su inquietud sobre la Eucaristía a las conversaciones que sostuvo con Elizabeth en esta temporada que se hospedó con los `Geachcombes': \blockquote[{\Cite[572]{monk1991duty}}: \enquote{This remark \textins{(in \emph{On Certainty} \S239)} was possibly prompted by a conversation about Transsubstantiation \textins{sic} that Wittgenstein had with Anscombe about this time. He was, it seems, surprised to hear from Anscombe that it really was Catholic belief that ‘in certain circumstances a wafer completely changes its nature’. It is presumably an example of what he had in mind when he remarked to Malcolm about Anscombe and Smythies: ‘I could not possibly bring myself to believe all the things that they believe.’ Such beliefs could find no place in his own world picture. His respect for Catholicism, however, prevented him from regarding them as mistakes or ‘transient mental disturbances’ \textins{(\emph{On Certainty} \S73)}}.]{Esta afirmación \textins{(en \emph{Sobre la Certeza} \S239)} fue motivada posiblemente por alguna conversación sobre la Transubstanciación que Wittgenstein tuvo con Anscombe alrededor de esta época. Al parecer, quedó sorprendido de escuchar de Anscombe que es verdaderamente una creencia Católica que `en ciertas circunstancias un trozo de pan completamente cambia en su naturaleza'. Esto es quizás un ejemplo de lo que tenía en mente cuando comentó a Malcolm sobre Anscombe y sobre Smythies: `No sería capaz de convencerme a mí mismo para llegar a creer todas esas cosas que ellos creen.' Creencias de este tipo no podrían encontrar un lugar en su imagen del mundo. Su respeto por el Catolicismo, sin embargo, le impedía considerarlas como equivocaciones o `perturbaciones mentales pasajeras' \textins{(\emph{Sobre la Certeza} \S73)}}.

El tema evoca la discusión de Wittgenstein sobre el papel que juega la imagen del mundo como justificación de ciertas creencias. La interpretación de Monk es que dentro del pensamiento de Wittgenstein la justificación para sostener creencias religiosas se encuentra en lo que él llamó la `imagen del mundo' y que describió como el \enquote*{trasfondo heredado desde el cual distinguimos verdadero de falso} (\emph{Sobre la Certeza \S94}). Si tenemos en cuenta la insistencia de Wittgenstein en que las creencias deben ser criticadas dentro de su propio contexto o sistema, esta interpretación parece correcta. Según esto parece que cualquier creencia religiosa estaría justificada dentro de su contexto o dentro de la imagen del mundo que sirve como su justificación. Sin embargo para Wittgenstein sí hay una diferencia entre lo que él consideraría ideas religiosas e ideas supersticiosas, de modo que tiene que ser posible criticar una expresión que se presenta como religiosa pero no lo es.

Otra narración de Anscombe puede servir para ilustrar mejor esto. En una de sus lecciones, ofrecida en 1984 con el título \emph{Paganism, Superstition and Philosophy}, ella distingue dos modos de usar la expresión `superstición' al referirse a creencias relacionadas con las religiones. Una aplicación para la palabra sería como un: \blockquote[{\Cite[57]{anscombe2008faith:paganism}}: \enquote{term of abuse for a religion deemed false by the speaker, and calling this religion `superstition' would be an expression of condemnation as false, in a culture where the acceptable religions were not regarded as true, but simply as the normal human practices}.]{insulto contra una religión considerada falsa por el que habla, y llamar a esta religión `superstición' consistiría en una expresión de condena por tenerla como falsa, dentro de una cultura donde no es el caso que las religiones aceptables sean consideradas como verdaderas, sino más bien como lo normal dentro de las prácticas humanas}. El segundo modo de usar la expresión es para denominar \blockquote[{\Cite[57]{anscombe2008faith:paganism}}: \enquote{something else which very many people of different religions would agree in calling `superstition': things like the use of charms, \textelp{} thinking certain numbers are unlucky or the sight of a black cat lucky}.]{algo distinto que mucha gente de diferentes religiones estarían de acuerdo en llamar `superstición': cosas como el uso de amuletos, \textelp{} pensar que ciertos números traen mala suerte o que es buena fortuna ver un gato negro}. Elizabeth entiende por `superstición' esto segundo y añade que \blockquote[{\Cite[57]{anscombe2008faith:paganism}}: \enquote{About such things people will sometimes say: `I'm afraid I \emph{am} superstitious', and here it is tempting to make Wittgenstein's remark: `Don't be proud of \emph{seeming} a fool, you may be one'}.]{Sobre estas cosas la gente dice en ocasiones `Me temo que \emph{soy} supersticioso', y aquí es tentador replicarles con el comentario de Wittgenstein: `No te enorgullezcas de \emph{parecer} un tonto, es posible que lo seas'}. Ahora bien, hemos visto que en el contexto filosófico Wittgenstein distingue entre una superstición y una equivocación y considera la superstición como la consecuencia de quedar engañados por una ilusión gramatical (\emph{Investigaciones Filosóficas \S110}). Anscombe, sin embargo, se interesó por lo que Ludwig comprendía por `superstición' en el contexto de la religión: \blockquote[{\Cite[57-58]{anscombe2008faith:paganism}}: \enquote{I once asked Wittgenstein what he understood by ‘superstition’. He said that he imagined he meant the same as I did. I thought it was not in the ‘false-religion’ sense that he was thinking of it, but the other one; he wasn’t offering a definition, but would call the same things superstition as I would. That he did not intend it in the ‘false-religion’ sense (in which neither am I accustomed to use the word) looks likely from his hostility to the ‘science has shown us that this is a mistake’ attitude about such things as poison oracles and other magical practices. Speaking of such matters I once asked him whether, if he had a friend, an African whose plan or possibility after being in England for a bit, was to go back home and take a training and then practise as a witch doctor, whether he, Wittgenstein, would want to stop him from doing this. We walked in silence for a space and then he said: ‘I would, but I don’t know why’. We talked of it no more. I incline to think that a vestige of the true religion spoke in him then; for that religion, whether in its ancient Hebrew or its Christian phase, has always said ‘No’ to such things}.]{En una ocasión pregunté a Wittgenstein qué entendía por `superstición'. Me dijo que imaginaba que para él significaba lo mismo que para mí. Lo interpreté pensando que él no lo entendía en el sentido de `falsa-religión', sino en el otro modo; no estaba ofreciendo una definición, pero él llamaría superstición a las mismas cosas que yo. Que no tenía la intención de usarla con con el sentido de `falsa-religión' (en el que yo tampoco estoy acostumbrada a usar la palabra) parece probable desde su hostilidad a la actitud: `la ciencia ha demostrado que esto es una equivocación' en casos relacionados con cosas como oráculos basados en los efectos del veneno u otras prácticas mágicas. Hablando de este tipo de cosas, en una ocasíon le pregunté si, en el caso de que tuviera un amigo, alguien de África cuyo plan o posibilidad fuera estar en Inglaterra por un tiempo, y que tuviera la intención de, al regresar a casa, entrenarse y practicar como un chamán, si él, Wittgenstein, querría disuadirlo de hacer esto. Caminamos en silencio por un rato y entonces respondió: `Lo intentaría, pero no sé por qué'. No hablamos más del tema. Me siento inclinada a pensar que un vestigio de la religión verdadera habló en él en esa ocasíon; pues esta religión, ya fuera en la etapa de la antigüedad hebrea o en la época cristiana, siempre ha dicho `No' a este tipo de cosas}. Para Ludwig era absurdo pedir a la ciencia que demostrara que las creencias mágicas son equivocaciones, puesto que \blockquote[{\Cite[125]{anscombe1981parmenides:qli}}: \enquote{he thought it stupid to take magic for mistaken science}.]{pensaba que era una necedad entender la magia como una ciencia equivocada}. La crítica a una idea mágica tiene que ser justificada en su propio campo, así como la ciencia tiene el suyo: \blockquote[{\Cite[125]{anscombe1981parmenides:qli}}: \enquote{Science can correct only scientific error, can detect error only in its own domain; in thoughts belonging to its own system of proceedings. About the merits of other proceedings it has nothing to say except perhaps for making predictions}.]{La ciencia solo puede corregir el error científico, puede detectar el error solo en su propio campo; en los pensamientos correspondientes a su propio sistema de procedimientos. Acerca de los méritos de otro tipo de procedimientos no tiene nada que decir, excepto quizás para hacer predicciones}. Elizabeth describe más llanamente lo que Wittgenstein no terminó de explicarle sobre su objeción a la decisión del hipotético amigo; el terreno desde el cual rechazaba la práctica mágica como supersticiosa era el de la religión. Para Anscombe, los fundamentos que podía tener Ludwig para objetar a una práctica mágica eran religiosos. ``Un vestigio de la religión verdadera habló en él''. Para Anscombe hay tal cosa como una religión verdadera y esta ofrece criterios para distinguir una práctica o creencia que se presenta como religiosa y no lo es. Para Wittgenstein no hay tal cosa.

Ludwig rechaza la idea de que haya alguna religión dentro de la cual el decir que se ``cree en Dios'' sea algo que se puede justificar como verdadero en el sentido de que puede demostrarse de manera comprensible. Alguien que dice que ``cree en Dios'' lo hace apoyado en una imagen del mundo, en algo ``que la vida le ha enseñado'': \blockquote[{\Cite[97]{wittgenstein1998cnv}}: \enquote{A proof of God ought really to be something by means of which you can convince yourself of God's exsistence. But I think that \emph{believers} who offered such proofs wanted to analyse \& make a case for their `belief' with their intellect, although they themselves would never have arrived at belief with their intellect, although they themselves would never have arrived at belief by way of such proofs. ``Convincing someone of God's existence'' is something you might do by means of a certain upbringing, shaping his life in such \& such a way.
Life can educate you to ``believing in God''. And \emph{experiences} too are what do this but not visions, or other sense experiences, which show us the ``existence of this being'', but e.g. sufferings of various sorts. And they do not show us God as a sense experience does an object, nor do they give rise to \emph{conjectures} about him. Experiences, thoughts,\,---\,life can force this concept on us.
So perhaps it is similar to the concept `object'}.]{Una demostración de Dios realmente debería ser algo por medio de lo que pudiéramos convencernos de la existencia de Dios. Pero pienso que los \emph{creyentes} que han ofrecido este tipo de demostraciones han querido analizar y presentar un argumento para su `creer' usando el intelecto, aún cuando ellos mismos nunca habrían llegado a creer por medio de este tipo de demostraciones. ``Convencer a alguien de la existencia de Dios'' es algo que podríamos hacer por medio de cierta crianza, moldeando la vida de esa persona en cierto modo.

La vida puede educarnos para ``creer en Dios''. Y las \emph{experiencias} también son las que hacen esto aunque no visiones, u otras experiencias de los sentidos, que nos mostrarían la ``existencia de este ser'', sino p.\,ej. sufrimientos de diversa índole. Y estos no nos muestran a Dios como una experiencia sensorial muestra un objeto, tampoco propician el surgimiento de \emph{conjeturas} sobre él. Las experiencias, los pensamientos,\,---\,la vida puede forzar este concepto en nosotros.

Así que quizás sí es similar al concepto de `objeto'}.

Desde luego que este tipo de ``creer en Dios'' no podría llegar a ser ``confiar en la Eucaristía'', por ejemplo, o creer en algún misterio o palabra de la revelación. Consiste más bien en una actitud hacia Dios y hacia el mundo, una especie de revalorización que se hace de las cosas de la vida desde lo que la creencia religiosa propone como lo profundamente importante.
%Wittgenstein lo explica así: %\blockquote[{\Cite[96-97]{wittgenstein1998cnv}}: \enquote{If the believer in God looks around and asks ``Where does everything I see come from?'' ``Where does all that come from?'', what he hankers after is not a (causal) explanation; and the point of his question is that it is the expression of this hankering. He is expressing, then, a stance towards all explanations.\,---\,But how is this manifested in his life? It is the attitude of taking a certain matter seriously, but then \underline{at a certain point} not taking it seriously after all, \& declaring that something else is still more serious. Someone may for instance say that it is a very grave matter that such \& such a person has died before he could complete a certain piece of work; \& in another sense that is not what matters. At this point one uses the words ``in a deeper sense''. Really what I should like to say is that here too what is important is not the \emph{words} you use or what you think while saying them, so much as the difference that they make at different points in your life. How do I know that two people mean the same thing when each says he believes in God? And just the same thing goes for the Trinity. Theology that insists \emph{certain} words \& phrases \& prohibits others makes nothing clearer. (Karl Barth) It gesticulates with words, as it were, because it wants to say something \& does not know how to express it. \emph{Practice} gives the words their sense.}]{Si el creyente en Dios mira a su alrededor y pregunta ``¿De dónde proviene todo esto que veo?'' ``¿De dónde ha surgido todo esto?'', lo que está anhelando no es una explicación (causal); y el punto de su pregunta es que ella misma es la expresión de su anhelo. Lo que está expresando, entonces, es una actitud hacia toda explicación.\,---\,Pero, ¿cómo se manifiesta esto en su vida?
%
%Lo hace en la actitud de tomar cierto asunto sériamente, pero entonces, \underline{en cierto punto} no tomándolo sériamente después de todo, y declarando que algo distinto merece todavía más seriedad. Por ejemplo alguien puede decir que es un asunto muy grave que tal o cual persona ha muerto antes de poder completar cierta obra; considerado según otro sentido eso no es lo que importa. En este punto usamos las palabras ``en un sentido más profundo''.
%
%Lo que quiero decir realmente es que aquí también lo importante no son las \emph{palabras} que usamos o lo que estamos pensando mientras las decimos, sino más bien la diferencia que hacen en distintos puntos de nuestra vida. ¿Cómo conozco que dos personas distintas quieren decir lo mismo cuando cada una dice que cree en Dios? Y exactamente lo mismo ocurre con la Trinidad. Una teología que insiste en palabras y frases \emph{específicas} y prohíbe otras no logra aclarar nada. (Karl Barth)
%
%Gesticula con palabras, podría decirse, porque quiere decir algo y no sabe cómo expresarlo. La \emph{práctica} es la que da a las palabras su sentido}.
Desde esta perspectiva, la noción de lo que un cristiano llamaría `fe' consistiría en esta actitud respecto de la vida y del mundo y de Dios, justificada por el trasfondo que van dejando las enseñanzas que la vida comunica por medio de experiencias como enfrentar el sufrimiento o la muerte. Además esta `fe' no se comunica en palabras precisas o verdaderas, sino que queda manifestada en la práctica, puesto que la `fe' misma consiste en esa actitud que se tiene hacia la vida.

Dentro de una concepción como esta, las proposiciones relacionadas con verdades reveladas o misterios quedan reducidas a una cierta actitud hacia las cosas, pero no expresan pensamientos: \blockquote[{\Cite[211]{teichmann2008ans}}: \enquote{It might be thought that a religious person who regards certain articles of faith as `mysteries' is more or less bound to embrace nonsense or self-contradiction; for what \emph{is} a mystery such as that of the Trinity, or of the Incarnation, or of the Eucharistic Transubstantiation, if not something whose appearance of incoherence cannot be dispelled by reason? If somebody utters `I believe' in connection with such mysteries, won't we be entitled to say, along with Wittgenstein: `But is this a belief, a thought at all? Perhaps there is a state of enlightenment, or an urge to find expression for certain experiences of life\,---\,but for there to be a belief, you would need to be able, at least in principle, to state that belief clearly and without contradiction'?}]{Puede ser pensado que una persona religiosa que considera ciertos artículos de fe como `misterios' está en mayor o menor grado obligada a abrazar el sinsentido o la auto-contradicción; pues ¿qué \emph{es} un misterio como el de la Trinidad, o el de la Encarnación, o el de la Transubstanciación Eucarística, sino algo cuya apariencia de incoherencia no puede ser disipada por la razón? Si alguien dice `Yo creo' en conexión con tales misterios, ¿no estaríamos autorizados a cuestionar, junto con Wittgenstein: `¿Pero es esto una creencia, un pensamiento en absoluto? Quizás haya ahí un estado de iluminación, o un deseo de encontrar expresión para ciertas experiencias de la vida\,---\,pero para que haya una creencia, deberías ser capaz, al menos en principio, de enunciar esa creencia claramente y sin contradicción'?}

Si escuchamos ahora a Anscombe en \emph{What is it to Believe Someone?}, lamentando que en su época se discuta sobre la fe haciéndola equivaler a `creencia religiosa' y que se haya perdido de vista \enquote*{la asombrosa noción de una cosa tal como \emph{creer a Dios}}, no es difícil distinguir una voz bastante diferente a la de Wittgenstein. Mientras que en él encontramos la tajante afirmación: \blockquote[{\Cite[\S6.432]{wittgenstein1922tractatus}}: \enquote{How the world is, is completely indifferent for what is higher. God does not reveal himself in the world}.]{Cómo sean las cosas en el mundo es un asunto completamente indiferente para lo superior. Dios no se revela en el mundo.}; Anscombe propone llanamente interpretar la fe como saber por testimonio\footnote{\cite[Cf.][87-88]{conesa1994cc}.}, es decir, como la creencia que se pone en aquello que se cree que viene a nosotros como palabra de Dios.

Entonces, ¿cómo responde Elizabeth a la objeción anterior? Al decir \enquote*{yo creo en la Encarnación} ¿expresamos un pensamiento, una creencia? Roger Teichmann propone que las ideas que están en el trasfondo de la descripción que Anscombe hace del misterio son las que expresa en los argumentos finales del artículo \emph{Parmenides, Mystery and Contradiction}. Allí vimos cómo Anscombe estudiaba la equivalencia de `puede ser captado en el pensamiento' con `puede ser presentado en una afirmación que pueda ser vista como portadora de un inobjetable sentido no contradictorio'. Esta equivalencia, además, la comparaba con la expresión del prefacio del \emph{Tractatus}: `aquello que pueda decirse del todo en palabras puede ser dicho claramente' y añadía que \enquote*{alguien que pensara esto podría pensar que puede existir lo inexpresable}, y en este sentido que \enquote*{puede haber lo que no puede ser pensado}. La interpretación de Teichmann es que: \blockquote[{\Cite[212]{teichmann2008ans}}: \enquote{the equivalence is rejected by Anscombe; or rather it is taken as wanting justification, as is shown by the closing words \textelp{}: `The trouble is, there doesn't seem to be any ground for holding this position. It is a sort of prejudice'}.]{la equivalencia es rechazada por Anscombe; o más bien la considera como necesitada de justificación, como queda mostrado en las palabras finales \textelp{}: `El problema es que no parece haber ningún fundamento para sostener esta posición. Es una especie de prejuicio'}. Para Teichmann es llamativo el detalle de que Anscombe no propone simplemente que \enquote*{puede ser captado en el pensamiento} podría ser equivalente a \enquote*{puede ser presentado en una oración \emph{que tenga} un inobjetable sentido no contradictorio}, sino que especifíca: \enquote*{\emph{que pueda verse} que tiene un inobjetable sentido no contradictorio}. Él entiende que en este detalle se está relacionando este principio, que pretende ser un criterio para caracterizar lo que puede ser considerado un pensamiento, con la capacidad empírica humana. Una perspectiva como esta podría atribuirse al Wittgenstein del \emph{Tractatus}, pero en la etapa más tardía de su pensamiento la rechazaría. En cualquier caso, para Teichmann, Anscombe no termina de aceptar la equivalencia.

Por otra parte, sí se podría decir que: \blockquote[{\Cite[212]{teichmann2008ans}}: \enquote{Anscombe would certainly admit that `can be grasped in thought' is incompatible with `can only be presented in a sentence with a contradictory sense'}.]{Anscombe ciertamente admitiría que `puede ser captado en el pensamiento' es incompatible con `solo puede ser presentado en una oración con un sentido contradictorio'}. Esta incompatibilidad la encontramos expresada en \emph{The Question for Linguistic Idealism}. Allí, tras explicar que para Wittgenstein el pensar consiste en actuar según una regla, Elizabeth comentaba \enquote*{¿Qué diría Wittgenstein del pensamiento ilógico? ¿Como yo?, ¿que no es pensar?}. Para Anscombe una proposición ilógica, es decir, una que no tiene aplicación en la actividad humana, que no expresa un ir según una regla, que solo puede ser presentada en una oración con un sentido contradictorio, no es una proposición que exprese pensamiento. Sin embargo, el misterio no es lo mismo que esto. En la conclusión de \emph{Parmenides, Mystery and Conradiction} ella establecía esa distinción. Si entendiéramos que el misterio es aquello que existe, pero que no puede ser pensado, estaríamos cayendo en una ilusión, puesto que \enquote*{el pensamiento expresando lo misterioso podría quedar esclarecido y entonces no queda misterio}, o \enquote*{la imposibilidad de aclararlo del todo mostraría que realmente no era un pensamiento}. La conclusión de Teichmann ante estas afirmaciones es que: \blockquote[{\Cite[212]{teichmann2008ans}}: \enquote{What Anscombe is trying to make room for is the idea of grasping a thought which cannot be cleared up, i.e. cannot be shown to have a non-contradictory sense. And this means: cannot be shown \emph{by us} to have a non-cotradictory sense. She is raising the possibility of a person's grasping a thought, even though the sentence expressing it `cannot be seen to have an unexceptionable non-contradictory sense'\,---\,seen by us, that is}.]{Para lo que Anscombe está tratando de hacer espacio es para la idea de captar un pensamiento que no puede ser aclarado, es decir que no puede ser mostrado como teniendo un sentido no contradictorio. Y esto significa: no puede ser mostrado \emph{por nosotros} como teniendo un sentido no-contradictorio. Ella está planteando la posibilidad de que una persona pueda captar un pensamiento, aún cuando la oración que lo expresa `no pueda ser vista como teniendo un inobjetable sentido no contradictorio'\,---\,vista por nosotros, es decir}.

En esto, la perspectiva de Anscombe es distinta a la de Wittgenstein. Ella sostiene que \blockquote[{\Cite[213]{teichmann2008ans}}: \enquote{\emph{we} might be able to grasp a thought which \emph{we} cannot clear up\,---\,cannot, because of our human finitude}.]{\emph{nosotros} podemos captar un pensamiento que \emph{nosotros} no podemos aclarar\,---\,no podemos, por nuestra finitud humana}. Esta perspectiva es la que Elizabeth expresa cuando en \emph{The Question for Linguistic Idealism} afirma que cuando la fe católica llama `misterios' a ciertas creencias quiere decir por lo menos que no es posible demostrarlas ni mostrar definitivamente que no son contradictorias y absurdas; sin embargo esto no implica que se profese abrazar la contradicción y lo absurdo.

Teichmann identifica dos cuestiones problemáticas relacionadas con esta noción. La primera es planteada por Anscombe en \emph{Parmenides, Mystery and Contradiction}. Allí habla de una oración que sea `abracadabra', es decir, sinsentido, y a esta \enquote*{nadie le haría caso}. También, dice, podríamos pensar en alguien que produzca una oración y diga \enquote*{esto es verdad, pero lo que dice es irreduciblemente enigmático}. En un caso en que la oración no sea mero `abracadabra', pero aún así presente dificultades para declarar un sentido que sea inobjetable para la expresión, ¿deberíamos descartar la posibilidad de que este sentido enigmático pueda ser una verdad? El problema que aparece aquí es cómo distinguir entre el misterio y el sinsentido, entre algo como la Transubstanciación y el puro `abracadabra'. \blockquote[{\Cite[213]{teichmann2008ans}}: \enquote{How then are we to know when to `take no notice', and when to take seriously?}]{¿Cómo, entonces, podríamos saber cuando `no hacer caso', y cuando tomar en serio?}

Una posible respuesta la encuentra Teichmann en el modo en el que Elizabeth enmarca su descripción de la Transubstanciación: \blockquote[{\Cite[213]{teichmann2008ans}}: \enquote{One reason why the doctrine of Transubstantiation is not \emph{mere} abracadabra is that you can teach it, explain it\,---\,or at any rate do something that looks like teaching and explaining}.]{Una razón por la que la doctrina de la Transubstanciación no es \emph{mero} abracadabra es que podemos enseñarla, explicarla\,---\,o en cualquier caso hacer algo que se ve como enseñar y explicar.} Con esto se refiere a la reflexión hecha en \emph{On Transubstantiation} donde Elizabeth propone que se puede expresar de modo sencillo lo que la Transubstanciación es considerando que puede ser enseñada a un niño durante la consagración. Ahora bien, para Teichmann esto no sería suficiente argumento. Tendríamos que dar por supuesto que estas enseñanzas sí tienen un sentido, después de todo, enteras escuelas filosóficas se han fundado en la promulgación de enigmático sinsentido\footnote{\cite[Cf.][213]{teichmann2008ans}: \enquote{The child will understand and learn. Only, of course, on the assumption that these sentences do make sense; which is why, in the context of distinguishing mystery from e.g. philosophical nonsense, the data about teaching are inconclusive: for whole schools of philosophy have been based on the promulgation of enigmatic nonsense}.}.

La propuesta que más comúnmente encontramos en los artículos de Anscombe sobre la distinción entre el sinsentido y el misterio es lo que aparece expresado en \emph{The Question for Linguistic Idealism}: \enquote*{Se asume que cualquier ostensible demostración de absurdidad es rebatible, una a la vez. Este proceso Wittgenstein mismo lo describió diciendo: `Puedes mantener a raya \emph{cada} ataque según venga.'}. Para Elizabeth es una diferencia importante entre el sinsentido y el misterio que las demostraciones que pretendan probar definitivamente que la creencia del misterio es absurda pueden ser rebatidas. En palabras de Teichmann: \blockquote[{\Cite[213-214]{teichmann2008ans}}: \enquote{You can show that `I can change the past' is an absurdity. It may take some philosophical delving, but it can be done. For Anscombe, a (proper) Catholic will believe that this cannot be done for those articles of faith called `mysteries'. \textelp{} You cannot show once and for all that the sentence in question has a non-contradictory sense, but you can rebut each attempt to prove that it lacks one \textelp{} the difference is between `It is possible that: for every proof $P$, you rebut $P$' and `For every proof $P$, it is possible that you rebut $P$'}.]{Podemos mostrar que `Puedo cambiar el pasado' es un absurdo. Puede requerir algo de indagación filosófica, pero puede hacerse. Para Anscombe, un católico (de verdad) creerá que esto no puede hacerse con aquellos artículos de la fe llamados `misterios'. \textelp{} No podemos mostrar de una vez por todas que la afirmación en cuestión tiene un sentido no contradictorio, pero puedes rebatir cada intento de demostrar que carece de ello \textelp{} la diferencia es entre `Es posible que: por cada demostración $P$, rebatimos $P$' y `Por cada demostración $P$, es posible rebatir que $P$'}. Con esta consideración aparece la segunda cuestión problemática que Teichmann identifica en relación con la noción del misterio que Anscombe propone. ¿Cómo justificamos de antemano el presupuesto de que estas creencias pueden ser defendidas de cualquier intento de demostrarlas como contradictorias? Teichmann responde: \blockquote[{\Cite[213-214]{teichmann2008ans}}: \enquote{The grounds for thinking that a mystery can always be defended from attack will not lie within the mystery itself. They will lie elsewhere: among the grounds for a person's religious belief. A Catholic will have been taught that the bread of the Mass is the body of Christ. She will believe what she has been taught as she believes, and as we all believe, things taught\,---\,not because we have established the reliability of the teacher, but because of the set-up of teaching and learning}.]{Los fundamentos para pensar que un misterio puede ser defendido de ataques no se encontrará dentro del misterio mismo. Estarán en otro lugar: entre los fundamentos para el creer religioso de la persona. A un católico se le habrá enseñado que el pan de la misa es el cuerpo de Cristo. Ella creerá lo que se le ha enseñado así como cree, y como todos creemos, las cosas que son enseñadas\,---\,no porque hemos establecido la veracidad del maestro, sino por el orden establecido del enseñar y aprender}. Esta propuesta nos trae de nuevo a la idea de un sistema de conocimiento como justificación para nuestras creencias.

El análisis de Teichmann es de gran valor para componer la visión que Anscombe tiene sobre el lenguaje religioso como lenguaje significativo y ofrece claves interesantes para identificar cómo el carácter testimonial que Elizabeth le atribuye a lo que aprendemos por la fe tiene que ver con el valor proposicional del lenguaje religioso.

La discusión recorrida hasta aquí nos deja ante la idea de que hay creencias que en la fe católica llamamos misterios que son realidades que nosotros podemos captar en el pensamiento, aunque por nuestra finitud humana no podemos expresar mostrándolas definitivamente como teniendo un sentido no contradictorio; estas creencias son profesadas con el presupuesto de que es posible rebatir cualquier intento de demostrar que son definitivamente contradictorias y absurdas y este presupuesto esta justificado por los fundamentos del creer religioso que provee el sistema de conocimiento dentro del que se nos enseñan estas creencias. Considerados estos aspectos de nuestras creencias en proposiciones que llamamos misterios de fe, todavía es posible añadir algo más sobre lo que Anscombe tiene que decir acerca de enseñar o atestiguar el misterio.

\subsubsection{\enquote{Poned en práctica la palabra y no os contentéis con oírla\ldots}}

En su artículo \emph{Authority in Morals} Anscombe cuenta que se encontró la frase \enquote*{\emph{Poned en práctica la palabra y no os contentéis con oírla}} (St 1,22) como lema del capítulo de un libro de matemáticas avanzadas. La enseñanza del apóstol aplicada a las matemáticas le parece interesante. Quizás recuerda al modo en que para Wittgenstein la manera en la que vamos según un cálculo matemático, en una función por ejemplo, es similar al modo en el que usamos las palabras en el lenguaje. En este caso la comparación alude a que aquel que quiere aprender matemáticas tiene que poner en práctica la enseñanza como está llamado a hacerlo el discípulo que recibe el Evangelio. La intención de Anscombe con la frase tiene que ver con el modo en que aprender matemáticas se parece a aprender verdades morales. Una de las premisas de su argumento en el artículo queda ilustrada por el hecho de que hay un modo acertado y otro equivocado de interpretar esta similitud. El modo equivocado de tomarla sería pensar que nuestra moralidad \emph{debe} ser algo que hemos formulado nosotros mismos juzgando entre bien y mal, como lo hace uno que aprende matemáticas y tiene que aprender a resolver él mismo las demostraciones\footnote{\cite[Cf.][45]{anscombe1981erp:am}: \enquote{one does learn mathematics by learning that mathematical propsitions are truths, but by working out their proofs. Similarly it might be held that one's morality \emph{must} be something one has formulated for oneself, seeing the rightness and wrongness of each of the things one judges to be right or wrong}.}. Pensar así de la enseñanza moral es un error. Ahora bien, lo que sí es correcto de la comparación, dice Anscombe, es que: \blockquote[{\Cite[47]{anscombe1981erp:am}}: \enquote{You have to do the mathematics; and the teacher can get you to do it: that is what teaching mathematics is. Similarly teaching morals will be, not getting the pupil to think something, not giving him a statement to believe, but getting him to act; this can be done by someone who brings up children}.]{Tienes que hacer las matemáticas; y el maestro puede lograr que lo hagas: en esto consiste enseñar matemáticas. Similarmente, enseñar moral será, no lograr que el alumno piense algo, no darle una afirmación para que crea, sino ayudarlo a actuar; esto puede ser hecho por alguien que educa a un niño}.

Esta aplicación de la enseñanza de Santiago es iluminadora para indagar en qué aspecto la posibilidad de enseñar un misterio de fe sirve para afirmar que es un lenguaje significativo. Si volvemos a la escena del padre invitando al hijo a adorar en la consagración podemos entender cómo este actúa como testigo y como maestro. El verbo adorar, como explica Anscombe, es un `verbo intencional'\footnote{\cite[4-5]{anscombe1981metaphysics:intsens}: \enquote{Obvious examples of intentional verbs are ``to think of'', ``to worship'', ``to shoot at''. (The verb ``to intend'' comes by metaphor from the last\,---\,``\emph{intendere arcum in}'', leading to ``\emph{intendere animum in}''.) Where we have such a verb taking an object, features analogous to the three features of intentionalness in action relate to some descriptions occurring as object-phrases after the verb}.}, y, si recordamos la aclaración hecha acerca de `Dios' como `descripción definitiva' que se refiere a `el uno y único Dios verdadero', podemos decir que la invitación del padre al hijo expresa un testimonio de aquello que el padre tiene la disposición de tratar como presencia del Dios verdadero. En su adoración, el padre actúa de acuerdo a la fe, cree a Dios, e invita a su hijo a participar de su acción de fe. Haciendo esto le enseña al hijo invitándole a unirse en el hacer de la comunidad, y en esto el padre se apoya también en la tradición que ha recibido. En esta dinámica vemos como están involucrados los distintos aspectos de la fe que se han discutido hasta ahora.

Hay un último aspecto del pensamiento de Anscombe relacionado con el testimonio que queda por destacar. El testimonio que el padre da al hijo cuando le enseña a adorar a Jesús en la Eucaristía tiene sentido o es significativo porque está enseñando al niño a \emph{hacer lo bueno}. El acto de adoración implica que se cree a Dios que promete \enquote*{en lenguaje humano}, es un acto de fe de acuerdo a sus promesas.

\label{subsec:logicbien}
En la discusión sobre lo que hace que nuestro lenguaje sea significativo en \emph{The Question for Linguistic Idealism} advertíamos que lo que previene que la gramática sea arbitraria es que una regla o técnica encuentra aplicación real en nuestras vidas. Según esto, la pregunta ¿en virtud de qué tienen las reglas esta aplicación real?, tenía como respuesta: \enquote*{En virtud de que somos el tipo de criaturas que encuentra natural darles ciertas aplicaciones y estamos de acuerdo en descubrir estos usos.} Este modo de responder a esta pregunta está relacionado con el análisis que Wittgenstein hace de la `necesidad lógica'. Dentro del pensamiento de Elizabeth hay otro modo de atender la pregunta \enquote*{¿por qué tenemos estas reglas?} desde lo que ella llama `necesidad Artistotélica'. Ella nota que \blockquote[{\Cite[100]{anscombe1981erp:rrp}}: \enquote{Aristotle indeed made a little noted observation that one sense of ``necessary'' is: ``that without which some good will not be attained or some evil avoided''}.]{Aristóteles ciertamente hizo una observación poco reconocida de que hay un sentido de ``necesario'' que es: ``aquello sin lo cual un bien no podría ser alcanzado o un mal evitado''}. Para Elizabeth la actividad humana del lenguaje no está regida por la necesidad lógica solamente, sino que también está orientada por aquello que es necesario para alcanzar el bien. En esto ella reconoce una relación entre la actividad del lenguaje y la ética: \blockquote[{\Cite[221]{teichmann2008ans}}: \enquote{it is Anscombe, rather than Wittgenstein, who has demystified such notions as \emph{good for}, as well as reinstating that notion of non-logical necessity which has been dubbed `Aristotelian necessity'. If these notions are ethical, or have an ethical aspect, then ethics is not, as Wittgesntein thought, a matter of what is shown and not said, or not only that}.]{es Anscombe, más que Wittgenstein, quien ha desmitificado tales nociones como \emph{bueno para}, además de reintegrar la noción de necesidad no-lógica que ha sido llamada `necesidad Aristotélica'. Si estas nociones son éticas, o tienen un aspecto ético, entonces la ética no es, como pensó Wittgenstein, materia de lo que se muestra y no se dice, o no es solo eso}. Prometer y actuar de acuerdo a una promesa es para Anscombe un ejemplo claro de esta interacción entre el lenguaje y el bien: \blockquote[{\Cite[18]{anscombe1981erp:pj}}: \enquote{such a procedure as that language-game is an instrument whose use is part and parcel of an enormous amount of human activity and hence of human good; of the supplying both of human needs and of human wants so far as the satisfactions of these are compossible}.]{un proceso como este juego de lenguaje es un instrumento cuyo uso es parte integral de una enorme cantidad de actividad humana, y por tanto de bien humano; del proveer para ambos el querer y las necesidades humanas, siempre que la satisfacción de estos sea compatible}.

Hay que aclarar que con esto no se afirma que los conceptos de \emph{bueno} o \emph{necesidad} son los que proveen justificación para nuestras prácticas lingüísticas, sino lo que necesitamos y buscamos como bien. En esto, como en el caso de la `necesidad lógica' no se está afirmando una justificación arbitraria: \blockquote[{\Cite[221]{teichmann2008ans}}: \enquote{our wants are not a source of arbitrariness, for in the end they cannot be completely hived off from our needs, as Anscombe recognized.} Ver también {\cite[31]{anscombe1981erp:mmph}}: \enquote{there is some sort of necessary connection between what you think \emph{you} need, and what you want. The connection is a complicated one; it is possible \emph{not} to want something that you judge you need. But e.g., it is not possible never to want \emph{anything}that you judge you need}.]{nuestros deseos no son una fuente de arbitrariedad, puesto que al final no pueden ser separados completamente de nuestras necesidades, como reconoció Anscombe}.

¿El testimonio de la fe en la presencia de Jesús en la Eucaristía tiene significado?, ¿que Dios es nuestro Padre?, ¿que Jesús es el Cristo? Anscombe diría que no resolvemos la pregunta afirmando que el testimonio constituye una variante lógica, es decir, hablar de la lógica del testimonio como un orden alternativo que justifica nuestra creencia, dado que bajo ese nuevo orden todavía habría que decir en qué consiste actuar según la regla lógica\footnote{\cite[Cf.][102]{anscombe1981erp:rrp}: \enquote{When it comes to rules of logic, it is otherwise. Let us not speak of variant logics; that is a mere distraction. For even in a variant logic, there will always be the question whether a rule has been followed}.}. La pregunta sobre si afirmaciones como estas son significativas nos lleva en definitiva a cuestionar sobre cuál es su uso en la actividad del lenguaje y la vida humana (necesidad lógica) y si pueden tener un rol para alcanzar el bien (necesidad Aristotélica). Cuando decimos que el cristiano da un testimonio de un misterio de la fe como lo hicieron los apóstoles o cuando se acoge el testimonio de una enseñanza que se comunica como la verdad de Dios comunicada por Cristo, el creyente juzga estas afirmaciones como significativas porque tienen un lugar dentro de nuestra vida y actividad, y sirven para alcanzar el bien.
