\subsection{¿Tiene carácter veritativo el lenguaje religioso?}

\subsubsection{``El modo más sencillo de expresar el misterio''}

Puntos de Teichmann
- Tractatus: ethics, aesthetics are things that are shown or show themselves, though they cannot be said or thought but can nevertheless be seen or understood
- For W. Religion should be added to that. He takes religious propositions as nonsensical and contradictory, and treats them with respect as such, this because he would rather have these not attempted to be presented as rational.
- Following this, it might be thought that a religious person who regards certain articles of faith as `mysteries' is more or less bound embrace nonsense and contradiction
- In PMC Anscombe presents an alternative:
  + considers the view
  + compares this view with whatever can be said...
  + she goes on:


  \blockquote[{\cite[211]{teichmann2008ans}}: It might be thought that a religious person who regards certain articles of faith as `mysteries' is more or less bound to embrace nonsense or self-contradiction; for what \emph{is} a mystery such as that of the Trinity, or of the Incarnation, or of the Eucharistic Transubstantiation, if not something whose appearance of incoherence cannot be dispelled by reason? If somebody utters `I believe' in connection with such mysteries, won't we be entitled to say, along with Wittgenstein: `But is this a belief, a thought at all? Perhaps there is a state of enlightenment, or an urge to find expression for certain experiences of life---but for there to be a belief, you would need to be able, at least in principle, to state that belief clearly and without contradiction'?]{Puede ser pensado que una persona religiosa que considera ciertos artículos de fe como `misterios' está en mayor o menor grado obligada a abrazar el sinsentido o la auto-contradicción; pues ¿qué \emph{es} un misterio como el de la Trinidad, o el de la Encarnación, o el de la Transubstanciación Eucarística, si no algo cuya apariencia de incoherencia no puede ser disipada por la razón? Si alguien dice `Yo creo' en conexión con tales misterios, ¿no estaríamos autorizados a cuestionar, junto con Wittgenstein: `¿Pero es esto una creencia, un pensamiento en absoluto? Quizás haya ahí un estado de iluminación, o un deseo de encontrar expresión para ciertas experiencias de la vida\,---\,pero para que haya una creencia, deberías ser capaz, al menos en principio, de enunciar esa creencia claramente y sin contradicción'?}

  Teichmann 212:
  It is an interesting question whether the later Wittgenstein can still be seen commited to the equivalence mentioned by Anscombe, between `can be grasped in thought' and `can be presented in a sentence which can be sen to have an unexceptionable non-contradictory sense', given a reading of `can be seen to have' which connects it with empirical human possibility. Whatever the answer to that question, the equivalence is rejected by Anscombe; or rather it is taken as wanting justification, as is shown by the closing words of PMC, which follow immediately after the passage just quoted: `The trouble is, there doesn't seem to be any ground for holding this position. It is a sort of prejudice' (PMC, 8).

  Anscombe would certainly admit that `can be grasped in thought' is incompatible with `can only be presented in a sentence with a contradictory sense' (****without the can be seen to have)
  What would W. say about illogical... what I would? that it isn't thinking?

  What Anscombe is trying to make room for is the idea of grasping a thought which cannot be cleared up, i.e. cannot be shown to have a non-contradictory sense. And this means: cannot be shown \emph{by us} to have a non-contradictory sense. She is reaising the possibility of a person's grasping a thought, even thought the sentence expressing it `cannot be seen to have an unexceptionable non-contradictory sense'---seen by us, that is. It is this idea that lies behind her account of what a mystery is:
  In the catholic faith...

  The departure from Wittgenstein consists in saying that we might be able to grasp a thought which we cannot clear up\,---\,cannot, because of our human finitude. The problem for Anscombe is how to distinguish a mystery from sheer nonsense.

  How then are we to know when to `take no notice', and when to take seriously?

  One reason why the doctrine of Transubstantiation is not \emph{mere} abracadabra is that you can teach it, explain it\,---\,or at any rate do something that looks like teaching and explaining.

  The child will understand and learn. Only, of course, on the assumption that these sentences do make sense; which is why, in the context of distinguishing mystery from e.g. philosophical nonsense, the data about teaching are inconclusive: for whole schools of philosophy have been based on the promulgation of enigmatic nonsense.

  You can show that `I can change the past' is an absurdity.
  For Anscombe, a (proper) Catholic will believe that this cannot be done for those articles of faith called `mysteries'.

  teichmann 220-221
  una explicación de por qué tenemos estas reglas anscombe rather than w. has demystified

En on transubstantiation anscombe no propone que se anime al niño a visualizar en la mente, sino a entrar en una actividad



La respuesta se encuentra en On transubstantiation que es engañoso en su simplicidad

In grounds of belief she makes a distintion between tradition or common knowledge and testimony, arguing that things justified on being thaught are justified on something thicker than testimony. Tradition or common knowledge is described by her as being thaught to join in doing something, not to believe something. But because everyone is taught to do such things, an object of belief is generated. The belief is so certainly correct (for it follows the practice) that it is knowledge. \emph{Here knowledge is no other than certainly correct belief in pursuit of a practice. But the connection with testimony is remote and
indirect.}

Intentionality of sensation: Worship is an intentional verb

``Under a description''

On promising and its justice 16: What I have skteched here us what W. called a language-game and we may say it is a fact of nature that humans beings very readily take to it
17: what you do is not a move in the game unless the game is being played and you are one of the players

authority in morals: be ye doers, you have to do the math and the teacher can get you to do it, teaching morals, getting him to act, some truths about what is the case are revealed

IWT 170 Wittgenstein took the term over from Russell, who used it in a special way, with reference to an entirely ordinary feeling; one that is well expressed at 6,52: `We feel that even if all \emph{possible} scientific questions have been answered, still the problems of life have not been touched at all.' And his further comment on this: `Of course there then just is no question left, an just this is the answer.'

IWT 170 he speaks of people `to whom the meaning of life has become clear'. But he says of them that they have not been able to say it. Now such people have not failed for want of trying; they have usually said a great deal. He means that they have failed to state what they wished to state; that it was never possible to state it as it is possible to state indifferent truth. He probably had Tolstoy especially in mind, whose explanations of what he thought he understood are miserable failures; but whose understanding is manifested, and whose preaching comes through, in a story like \emph{Hadji Murad}.

\subsubsection{La verdad como un transcendental}

IWT 166 Thus when the \emph{Tractatus} tells us that `Logic is trascendental', it does not mean that the propositions of logic state transcendental truths; it means that they, like all other propositions, shew something that percades everything sayable an is itself unsayable. If it were sayable, then failure to accord with it would have to be expressible too, and thus would be a possibility.

\subsubsection{The possibility of natural religion}

Anscombe distingue entre presuposiciones que

W. no distingue entre teologia natural y misterios... tampoco racionalidad de la fe.... QLI 123



IWT 78: Here it is worth remarking that the truth of the \emph{Tractatus} theory would be death to natural theology; not because of any jejune positivism or any `verificationism', but simply because of the picture theory of the `significant proposition'. For it is essential to this that the picturing proposition has two poles, and in each sense it represents what may perfectly well be true. Which of them is true is just what \emph{happens} to be the case. But in natural theology this is an impermissible notion; its propositions are not supposed to be the ones that happen to be true out of pairs of possibilities; nor are they supposed to be logical or mathematical propositions either.

\subsubsection{``Ward each attack as it comes''}

IWT 161 `there is no picture that is true \emph{a priori}'. That is to say, if a proposition has a negation which is a perfectly good possibility, then it cannot be settled whether the proposition is true or false just by considering what it means.

\subsubsection{El extraordinario fenómeno de creer a Dios}

Conesa 260: al creer a Dios, el hombre se apoya en la veracidad divina  y por lo mismo se confía al Dios de la verdad
