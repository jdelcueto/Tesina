\subsection{¿Tiene carácter veritativo el lenguaje religioso?}

\subsubsection{\enquote{El modo más sencillo de expresar lo que es la transubstanciación\ldots}}

A lo largo de la discusión de los artículos de Anscombe es posible distinguir en ella una posición clara acerca de la actitud con la que un creyente sostiene esas creencias que dentro de la fe católica llamamos `misterios': \enquote*{Esto puede ser demostrado falso, pero aún así lo creo} no es para nada una actitud de fe, es decir, sostener este tipo de creencias no consiste en una disposición a profesar la contradicción. En esto, creer un misterio no es lo mismo que creer cosas ilógicas o sin sentido.

Junto a esta noción, se encuentra también en ella la interesante idea de \enquote*{expresar el misterio}, posibilidad que caracteriza diciendo: \enquote*{puede ser enseñado}; a un niño, por ejemplo. Y en esto también hay algo que empieza a diferenciar una afirmación relacionada con una creencia de fe de afirmaciones que no expresan pensamiento o que no tienen significado.

Dentro del pensamiento de Anscombe, además, el todo del lenguaje relacionado con la religión no está compuesto solamente por afirmaciones que expresan creencias en misterios de fe, sino que además utiliza proposiciones de teología natural y proposiciones sobre las presuposiciones involucradas en creer a Dios.

Para profundizar más en el pensamiento de Anscombe sobre estas características que describen el modo en el que el lenguaje religioso es lenguaje significativo, es útil comparar su postura con la de Wittgenstein.

En el \emph{Tractatus}, termina ocupando un lugar prominente lo \enquote*{inexpresable, lo que se muestra; que es lo místico} (\S6.522). En esta categoría de `lo que no puede ser dicho pero queda mostrado', se encuentran las proposiciones éticas y estéticas: \blockquote[{\cite[\S6.421]{wittgenstein1922tractatus}}: \enquote{It is clear that ethics cannot be expressed. Ethics is transcendental. (Ethics and aesthetics are one.)}]{Queda por tanto claro que la ética no puede expresarse mediante palabras. La ética es transcendental. (La ética y la estética son una y la misma.)}. Wittgenstein tuvo gran interés por esto que consideraba una tendencia de la mente humana: el deseo de poner en palabras lo que no puede ser dicho. Esta tendencia la reconocía en el corazón de la ética, cuyas proposiciones juzgaba como sinsentido, aunque su actitud hacia ellas era de respeto.\footnote{\cite[Cf.~][211]{teichmann2008ans}: \enquote{In his `Lecture on Ethics' of 1929, he cites certain experiences, saying of them that their natural expression takes the form of utterances which can only count as nonsensical, as attempts to `\emph{go beyond} the world and that is to say beyond significant language'. These experiences and utterances he takes to be at the heart of ethics, anout which he writes:`it is a document of a tendency in the human mind which I personally cannot help respecting deeply and I would not for my life ridicule it'.}}

A la ética y la estética ha de añadirse la religión. Este tipo de proposiciones también intentan ir más allá del mundo y de lo que puede considerarse como lenguaje significativo y por tanto estos intentos de poner en palabras lo que no puede ser dicho también constituyen afrimaciones sin sentido. Sin embargo, su actitud hacia las afirmaciones religiosas ---así como hacia la ética--- era tomarlas en serio con respeto. En este sentido puede entenderse la anecdota recordada por Anscombe en \emph{The Question of Linguistic Idealism}. Wittgenstein prefería tratar con respeto las proposiciones religiosas en tanto que contradictorias, puesto que rechazaba la idea de considerar la religión como racional. Así es que el intento de presentar la religión como algo que pudiera ser visto como racional le parecía que era como encerrar un objeto irregular dentro de una lisa esfera de cristal; las irregularidades no dejan de ser visibles, así que consideraba más adecuado atender el objeto sin disimularle sus aristas. En esto Anscombe se apoya para decir que la  actitud de Wittgenstein al todo de la religión, en cierto modo, la asimilaba al misterio. Rechazaba las proposiciones de teología natural y no cabían dentro de su sistema de pensamiento. Hemos visto cómo el \emph{Círculo de Viena} interpretó el rechazo de las proposiciones de teología natural dentro del sistema de Wittgenstein. Anscombe reconoce que dentro del pensamiento de Ludwig la teología natural no es posible, pero ante la interpretación del \emph{Círculo} se mostró crítica:\blockquote[{\cite[78]{anscombe1959iwt}}: \enquote{Here it is worth remarking that the truth of the \emph{Tractatus} theory would be death to natural theology; not because of any jejune positivism or any `verificationism', but simply because of the picture theory of the `significant proposition'. For it is essential to this that the picturing proposition has two poles, and in each sense it represents what may perfectly well be true. Which of them is true is just what \emph{happens} to be the case. But in natural theology this is an impermissible notion; its propositions are not supposed to be the ones that happen to be true out of pairs of possibilities; nor are they supposed to be logical or mathematical propositions either.}]{Aquí vale la pena comentar que la verdad de la teoría del \emph{Tractatus} conllevaría la muerte de la teología natural; no por ningún inmaduro positivismo o ningún `verificacionismo', sino simplemente por la teoría de la imagen relacionada con lo que es una `proposición significativa'. Puesto que es esencial para esta que la proposición que ofrece una imagen tenga dos polos, y en cada sentido represente lo que pudiera ser perfectamente bien la verdad. Pero en la teología natural esto es una noción inadmisible; sus proposiciones no son tales que se supone que son las que de hecho son verdaderas de entre un par de posibilidades; ni se tiene por supuesto que sean proposiciones lógicas o matemáticas tampoco.} Aún cuando Wittgenstein abandonó en gran parte el modo de comprender el lenguaje descrito en el \emph{Tractatus}, siguió juzgando que el intento de razonar desde los objetos del mundo a algo fuera de este, como se pretende en las afirmaciones de la teología natural, no era posible. En \emph{The Question of Linguistic Idealism} Anscombe ofrece como evidencia de esta objeción una cita de \emph{Observaciones sobre los fundamentos de la matemática}: \blockquote[{\cite[VII, 25]{wittgenstein1956remmath}}: \enquote{A \emph{logical} conclusion is being drawn, when no experience can contradict the conclusion without contradicting the premises. I.e., when the inference is only a movement within the means of representation.}]{Se hace una inferencia \emph{lógica} cuando ninguna experiencia puede contradecir la conclusion porque entonces contradiría las premisas. Es decir, cuando la inferencia es solo un movimiento en los medios de la representación.} Elizabeth relaciona este comentario con la premisa de \emph{Investigaciones Filosóficas}: \blockquote[{\cite[\S126]{wittgenstein1953phiinv}}: \enquote{Philosophy just puts everythig before us, and neither explains nor deduces anything. --- Since everything lies open to view, there is nothing to explain. For whatever may be hidden is of no interest to us. The name ``philosophy'' might also be given to what is possible \emph{before} all new discoveries and inventions.}]{La filosofía meramente expone todo ante nosotros, y no explica ni deduce nada.\,---\,Ya que todo está abiertamente a la vista, no hay nada que explicar. Pues lo que sea que esté oculto no es de ningún interés para nosotros. Se podría llamar también ``filosofía'' a lo que es posible antes de todos los nuevos descubrimientos e invenciones.} Con esto, Anscombe hace referencia al argumento de Wittgenstein de que la actividad filosófica realiza sus inferencias dentro de los medios de representación que pueden ser usados por el lenguaje y los elementos que componen estos medios de representación no se obtienen desde deducciones de realidades ocultas, sino que están a la vista en la actividad misma de usar el lenguaje. Según esto, el intento de razonar desde los objetos del mundo sobre algo más allá del mundo está en contra de lo que Wittgenstein considera filosofía.



\blockquote[{\cite[211]{teichmann2008ans}}: It might be thought that a religious person who regards certain articles of faith as `mysteries' is more or less bound to embrace nonsense or self-contradiction; for what \emph{is} a mystery such as that of the Trinity, or of the Incarnation, or of the Eucharistic Transubstantiation, if not something whose appearance of incoherence cannot be dispelled by reason? If somebody utters `I believe' in connection with such mysteries, won't we be entitled to say, along with Wittgenstein: `But is this a belief, a thought at all? Perhaps there is a state of enlightenment, or an urge to find expression for certain experiences of life---but for there to be a belief, you would need to be able, at least in principle, to state that belief clearly and without contradiction'?]{Puede ser pensado que una persona religiosa que considera ciertos artículos de fe como `misterios' está en mayor o menor grado obligada a abrazar el sinsentido o la auto-contradicción; pues ¿qué \emph{es} un misterio como el de la Trinidad, o el de la Encarnación, o el de la Transubstanciación Eucarística, si no algo cuya apariencia de incoherencia no puede ser disipada por la razón? Si alguien dice `Yo creo' en conexión con tales misterios, ¿no estaríamos autorizados a cuestionar, junto con Wittgenstein: `¿Pero es esto una creencia, un pensamiento en absoluto? Quizás haya ahí un estado de iluminación, o un deseo de encontrar expresión para ciertas experiencias de la vida\,---\,pero para que haya una creencia, deberías ser capaz, al menos en principio, de enunciar esa creencia claramente y sin contradicción'?}

verdad es uno de los nombres de Dios, para ella es un trascendental. Dios promete en lenguaje humano, se involucra en la actividad humana del lenguaje. Aquí es importante su comprensión de la fe como creer a alguien que se comunica, o entender alguna experiencia como palabra de Dios.

\subsubsection{\enquote{Poned en práctica la palabra y no os contentéis con oírla\ldots}}
