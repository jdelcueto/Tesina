\subsection{¿Tiene carácter veritativo el lenguaje religioso?}

\subsubsection{\enquote{El modo más sencillo de expresar lo que es la transubstanciación\ldots}}

A lo largo de los artículos de Anscombe que hemos discutido es posible distinguir en ella una posición clara: sostener esas creencias que dentro de la fe católica llamamos `misterios' no consiste en una disposición a profesar la contradicción. Decir \enquote*{Esto puede ser demostrado falso, pero aún así lo creo}, no es declarar para nada una actitud de fe. En esto, creer un misterio no es lo mismo que creer cosas ilógicas o sin sentido.

Junto a esta noción, se encuentra también en ella la interesante idea de \enquote*{expresar el misterio}, posibilidad que caracteriza diciendo: \enquote*{puede ser enseñado}; a un niño, por ejemplo. Y en esto también hay algo que empieza a diferenciar una afirmación relacionada con una creencia de fe como distinta de afirmaciones que no expresan pensamiento o que no tienen significado.

Dentro de la obra de Anscombe, además, el todo del lenguaje relacionado con la religión no está compuesto solamente por afirmaciones que expresan creencias en misterios de fe, sino que en sus discusiones utiliza también proposiciones de teología natural y proposiciones acerca de las presuposiciones involucradas en creer a Dios.

Vamos a profundizar más en estos aspectos del pensamiento de Anscombe sobre las características que describen el modo en el que el lenguaje religioso es lenguaje significativo. Para esto será útil comparar su perspectiva con la de Wittgenstein.

En el \emph{Tractatus}, termina ocupando un lugar prominente lo \enquote*{inexpresable, lo que se muestra; que es lo místico} (\S6.522). En esta categoría de `lo que no puede ser dicho pero queda mostrado', se encuentran las proposiciones éticas y estéticas: \blockquote[{\cite[\S6.421]{wittgenstein1922tractatus}}: \enquote{It is clear that ethics cannot be expressed. Ethics is transcendental. (Ethics and aesthetics are one.)}]{Queda por tanto claro que la ética no puede expresarse mediante palabras. La ética es transcendental. (La ética y la estética son una y la misma.)}. Wittgenstein tuvo gran interés por esto que consideraba una tendencia de la mente humana: el deseo de poner en palabras lo que no puede ser dicho. Esta tendencia la reconocía en el corazón de la ética, cuyas proposiciones juzgaba como sinsentido, aunque su actitud hacia ellas era de respeto.\footnote{\cite[Cf.~][211]{teichmann2008ans}: \enquote{In his `Lecture on Ethics' of 1929, he cites certain experiences, saying of them that their natural expression takes the form of utterances which can only count as nonsensical, as attempts to `\emph{go beyond} the world and that is to say beyond significant language'. These experiences and utterances he takes to be at the heart of ethics, about which he writes:`it is a document of a tendency in the human mind which I personally cannot help respecting deeply and I would not for my life ridicule it'.}}

A la ética y la estética añadiría la religión. Este tipo de proposiciones también intentan ir más allá del mundo y de lo que puede considerarse como lenguaje significativo, y por tanto estos intentos de poner en palabras lo que no puede ser dicho también constituyen afirmaciones sin sentido. Sin embargo, su actitud hacia las afirmaciones religiosas ---así como hacia la ética--- era tomarlas en serio, con respeto. En este sentido puede entenderse la anecdota recordada por Anscombe en \emph{The Question of Linguistic Idealism}. Wittgenstein prefería tratar con respeto las proposiciones religiosas en tanto que contradictorias, puesto que rechazaba la idea de considerar la religión como racional. Así es que el intento de presentar la religión como algo que pudiera ser visto racionalmente le parecía que era como encerrar un objeto irregular dentro de una lisa esfera de cristal; las irregularidades no dejan de ser visibles, así que consideraba más adecuado atender el objeto sin disimularle sus aristas.

Anscombe se apoya en esto cuando dice que la  actitud de Wittgenstein al todo de la religión, en cierto modo, la asimilaba al misterio. Rechazaba las proposiciones de teología natural y no cabían dentro de su sistema de pensamiento. Hemos visto cómo el \emph{Círculo de Viena} articuló un rechazo sistemático de las proposiciones teológicas apoyados en el \emph{Tractatus} de Wittgenstein. Anscombe reconoce que dentro del pensamiento de Ludwig no es posible la teología natural en particular, pero ante la interpretación del \emph{Círculo} se mostró crítica:\blockquote[{\cite[78]{anscombe1959iwt}}: \enquote{Here it is worth remarking that the truth of the \emph{Tractatus} theory would be death to natural theology; not because of any jejune positivism or any `verificationism', but simply because of the picture theory of the `significant proposition'. For it is essential to this that the picturing proposition has two poles, and in each sense it represents what may perfectly well be true. Which of them is true is just what \emph{happens} to be the case. But in natural theology this is an impermissible notion; its propositions are not supposed to be the ones that happen to be true out of pairs of possibilities; nor are they supposed to be logical or mathematical propositions either.}]{Aquí vale la pena comentar que la verdad de la teoría del \emph{Tractatus} conllevaría la muerte de la teología natural; no por ningún inmaduro positivismo o ningún `verificacionismo', sino simplemente por la teoría de la imagen relacionada con lo que es una `proposición significativa'. Puesto que es esencial para esta que la proposición que ofrece una imagen tenga dos polos, y en cada sentido represente lo que pudiera ser perfectamente bien la verdad. Pero en la teología natural esto es una noción inadmisible; sus proposiciones no son tales que se supone que son las que de hecho son verdaderas de entre un par de posibilidades; ni se tiene por supuesto que sean proposiciones lógicas o matemáticas tampoco.} Aún cuando, en la segunda etapa de su pensamiento, Wittgenstein desarrolló nuevas ideas en su modo de comprender el lenguaje, no dejó de pensar que no es posible el intento de razonar desde los objetos del mundo a algo fuera de este, como se pretende en las afirmaciones de la teología natural. En \emph{The Question of Linguistic Idealism} Anscombe ofrece como evidencia de esta objeción una cita de \emph{Observaciones sobre los fundamentos de la matemática}: \blockquote[{\cite[VII, 25]{wittgenstein1956remmath}}: \enquote{A \emph{logical} conclusion is being drawn, when no experience can contradict the conclusion without contradicting the premises. I.e., when the inference is only a movement within the means of representation.}]{Se hace una inferencia \emph{lógica} cuando ninguna experiencia puede contradecir la conclusión porque entonces contradiría las premisas. Es decir, cuando la inferencia es solo un movimiento en los medios de la representación.} Elizabeth relaciona este comentario con la premisa de \emph{Investigaciones Filosóficas}: \blockquote[{\cite[\S126]{wittgenstein1953phiinv}}: \enquote{Philosophy just puts everythig before us, and neither explains nor deduces anything. --- Since everything lies open to view, there is nothing to explain. For whatever may be hidden is of no interest to us. The name ``philosophy'' might also be given to what is possible \emph{before} all new discoveries and inventions.}]{La filosofía meramente expone todo ante nosotros, y no explica ni deduce nada.\,---\,Ya que todo está abiertamente a la vista, no hay nada que explicar. Pues lo que sea que esté oculto no es de ningún interés para nosotros. Se podría llamar también ``filosofía'' a lo que es posible antes de todos los nuevos descubrimientos e invenciones.} Tales afirmaciones representan nociones propias de la etapa más tardía del pensamiento de Ludwig. Para él la actividad filosófica debe realizar sus inferencias dentro de los medios de representación que pueden ser usados por el lenguaje. Los elementos que componen estos medios de representación no se obtienen desde deducciones de realidades ocultas, sino que están a la vista en la actividad misma de usar el lenguaje. Según esto, el intento de razonar desde los objetos del mundo sobre algo más allá del mundo está en contra de lo que Wittgenstein llamaría filosofía.

Quizás por el tono anecdótico de algunas de las descripciones que Anscombe narra sobre la actitud de Wittgenstein hacia la religión, resulta ambiguo si sus creencias constituyen una posición filosófica o solamente una opinión personal. Ciertamente, a lo largo de su vida, la cuestión de la religión fue para él un asunto personalmente problemático y así no deja de aparecer como un tema cargado de cierta ambigüedad en sus reflexiones filosóficas. Hemos escuchado de Elizabeth sobre la actitud de Ludwig hacia el argumento Agustiniano en su ponencia en el \emph{Moral Science Club} y él mismo ha expresado la dificultad que representa una creencia católica como lo es la Eucaristía en la discusión en \emph{Sobre la Certeza}. En relación con este segundo ejemplo es interesante el comentario de Ray Monk, que en su biografía de Wittgenstein atribuye su inquietud sobre la Eucaristía a las conversaciones que sostuvo con Elizabeth en esta temporada que se hospedó con los `Geachcombes': \blockquote[{\cite[572]{monk1991duty}}: \enquote{This remark \textins{(in \emph{On Certainty} \S239)} was possibly prompted by a conversation about Transsubstantiation \textins{sic} that Wittgenstein had with Anscombe about this time. He was, it seems, surprised to hear from Anscombe that it really was Catholic belief that ‘in certain circumstances a wafer completely changes its nature’. It is presumably an example of what he had in mind when he remarked to Malcolm about Anscombe and Smythies: ‘I could not possibly bring myself to believe all the things that they believe.’ Such beliefs could find no place in his own world picture. His respect for Catholicism, however, prevented him from regarding them as mistakes or ‘transient mental disturbances’ \textins{(\emph{On Certainty} \S73)}.}]{Esta afirmación \textins{(en \emph{Sobre la Certeza} \S239)} fue motivada posiblemente por alguna conversación sobre la Transubstanciación que Wittgenstein tuvo con Anscombe alrededor de esta época. Al parecer, quedó sorprendido de escuchar de Anscombe que es verdaderamente una creencia Católica que `en ciertas circunstancias un trozo de pan completamente cambia en su naturaleza'. Esto es quizás un ejemplo de lo que tenía en mente cuando comentó a Malcolm sobre Anscombe y sobre Smythies: `No sería capaz de convencerme a mí mismo para llegar a creer todas esas cosas que ellos creen.' Creencias de este tipo no podrían encontrar un lugar en su imagen del mundo. Su respeto por el Catolicismo, sin embargo, le impedía considerarlas como equivocaciones o `perturbaciones mentales pasajeras' \textins{(\emph{Sobre la Certeza} \S73)}.}

El tema evoca la discusión de Wittgenstein sobre el papel que juega la imagen del mundo como justificación de ciertas creencias. La interpretación de Monk es que dentro del pensamiento de Wittgenstein la justificación para sostener creencias religiosas se encuentra en lo que él llamó la `imagen del mundo' y que describió como el \enquote*{trasfondo heredado desde el cual distinguimos verdadero de falso} (\emph{Sobre la Certeza \S94}). Si tenemos en cuenta la insistencia de Wittgenstein en que las creencias deben de ser criticadas dentro de su propio contexto o sistema, esta interpretación parece correcta. Según esto parece que cualquier creencia religiosa estaría justificada dentro de su contexto o dentro de la imagen del mundo que sirve como su justificación. Sin embargo para Wittgenstein sí hay una diferencia entre lo que él consideraría ideas religiosas e ideas supersticiosas, de modo que tiene que ser posible criticar una expresión que se presenta como religiosa pero no lo es.

Otra narración de Anscombe puede servir para ilustrar mejor esto. En una de sus lecciones, ofrecida en 1984 con el título \emph{Paganism, Superstition and Philosophy}, ella distingue dos modos de usar la expresión `superstición' al referirse a creencias relacionadas con las religiones. Una aplicación para la palabra sería como un: \blockquote[{\cite[57]{anscombe2008faith:paganism}}: \enquote{term of abuse for a religion deemed false by the speaker, and calling this religion `superstition' would be an expression of condemnation as false, in a culture where the acceptable religions were not regarded as true, but simply as the normal human practices.}]{insulto contra una religión considerada falsa por el que habla, y llamar a esta religión `superstición' consistiría en una expresión de condena por tenerla como falsa, dentro de una cultura donde no es el caso que las religiones aceptables sean consideradas como verdaderas, sino más bien como lo normal dentro de las prácticas humanas}. El segundo modo de usar la expresión es para denominar \blockquote[{\cite[57]{anscombe2008faith:paganism}}: \enquote{something else which very many people of different religions would agree in calling `superstition': things like the use of charms, \textelp{} thinking certain numbers are unlucky or the sight of a black cat lucky.}]{algo distinto que mucha gente de diferentes religiones estarían de acuerdo en llamar `superstición': cosas como el uso de amuletos, \textelp{} pensar que ciertos números traen mala suerte o que es buena fortuna ver un gato negro.} Elizabeth entiende por `superstición' esto segundo y añade que \blockquote[{\cite[57]{anscombe2008faith:paganism}}: \enquote{About such things people will sometimes say: `I'm afraid I \emph{am} superstitious', and here it is tempting to make Wittgenstein's remark: `Don't be proud of \emph{seeming} a fool, you may be one'}]{Sobre estas cosas la gente dice en ocasiones `Me temo que \emph{soy} supersticioso', y aquí es tentador replicarles con el comentario de Wittgenstein: `No te enorgullezcas de \emph{parecer} un tonto, es posible que lo seas'}. Ahora bien, hemos visto que en el contexto filosófico Wittgenstein distingue entre una superstición y una equivocación y considera la superstición como la consecuencia de quedar engañados por una ilusión gramatical (\emph{Investigaciones Filosóficas \S110}). Anscombe, sin embargo, se interesó por lo que Ludwig comprendía por `superstición' en el contexto de la religión: \blockquote[{\cite[57--58]{anscombe2008faith:paganism}}: \enquote{I once asked Wittgenstein what he understood by ‘superstition’. He said that he imagined he meant the same as I did. I thought it was not in the ‘false-religion’ sense that he was thinking of it, but the other one; he wasn’t offering a definition, but would call the same things superstition as I would. That he did not intend it in the ‘false-religion’ sense (in which neither am I accustomed to use the word) looks likely from his hostility to the ‘science has shown us that this is a mistake’ attitude about such things as poison oracles and other magical practices. Speaking of such matters I once asked him whether, if he had a friend, an African whose plan or possibility after being in England for a bit, was to go back home and take a training and then practise as a witch doctor, whether he, Wittgenstein, would want to stop him from doing this. We walked in silence for a space and then he said: ‘I would, but I don’t know why’. We talked of it no more. I incline to think that a vestige of the true religion spoke in him then; for that religion, whether in its ancient Hebrew or its Christian phase, has always said ‘No’ to such things.}]{En una ocasión pregunté a Wittgenstein qué él entendía por `superstición'. Me dijo que imaginaba que para él significaba lo mismo que para mí. Lo interpreté pensando que él no lo entendía en el sentido de `falsa-religión', sino en el otro modo; no estaba ofreciendo una definción, pero él llamaría superstición a las mismas cosas que yo. Que no tenía la intención de usarla con con el sentido de `falsa-religión' (en el que yo tampoco estoy acostumbrada a usar la palabra) parece probable desde su hostilidad a la actitud: `la ciencia ha demostrado que esto es una equivocación' en casos relacionados con cosas como oráculos basados en los efectos del veneno u otras prácticas mágicas. Hablando de este tipo de cosas, en una ocasíon le pregunté si, en el caso de que tuviera un amigo, alguien de Africa cuyo plan o posibilidad fuera estar en Inglaterra por un tiempo, y que tuviera la intención de, al regresar a casa, entrenarse y practicar como un chamán, si él, Wittgenstein, querría disuadirlo de hacer esto. Caminamos en silencio por un rato y entonces respondió: `Lo intentaría, pero no sé por qué'. No hablamos más del tema. Me siento inclinada a pensar que un vestigio de la religión verdadera habló en él en esa ocasíon; pues esta religión, ya fuera en la etapa de la antiguedad hebrea o en la época cristiana, siempre ha dicho `No' a este tipo de cosas.} Para Ludwig era absurdo pedir a la ciencia que demostrara que las creencias mágicas son equivocaciones, puesto que \blockquote[{\cite[125]{anscombe1981parmenides:qli}}: \enquote{he thought it stupid to take magic for mistaken science.}]{pensaba que era una necedad entender la magia como una ciencia equivocada.} La crítica a una idea mágica tiene que ser justificada en su propio campo, así como la ciencia tiene el suyo: \blockquote[{\cite[125]{anscombe1981parmenides:qli}}: \enquote{Science can correct only scientific error, can detect error only in its own domain; in thoughts belonging to its own system of proceedings. About the merits of other proceedings it has nothing to say except perhaps for making predictions.}]{La ciencia solo puede corregir el error científico, puede detectar el error solo en su propio campo; en los pensamientos correspondientes a su propio sistema de procedimientos. Acerca de los méritos de otro tipo de procedimientos no tiene nada que decir, excepto quizás para hacer predicciones.} Elizabeth describe más llanamente lo que Wittgenstein no terminó de explicarle sobre su objeción a la decisión del hipotético amigo; el terreno desde el cual rechazaba la práctica mágica como supersticiosa era el de la religión. Para Anscombe, los fundamentos que podía tener Ludwig para objetar a una práctica mágica eran religiosos. ``Un vestigio de la religión verdadera habló en él''. Para Anscombe hay tal cosa como una religión verdadera y esta ofrece criterios para distinguir una práctica o creencia que se presenta como religiosa y no lo es. Para Wittgenstein no hay tal cosa.

Ludwig rechaza la idea de que haya alguna religión dentro de la cual el decir que se ``cree en Dios'' sea algo que se puede justificar como verdadero en el sentido de que puede demostrarse de manera comprensible. Alguien que dice que ``cree en Dios'' lo hace apoyado en una imagen del mundo, en algo ``que la vida le ha enseñado'': \blockquote[{\cite[97]{wittgenstein1998cnv}}: \enquote{A proof of God ought really to be something by means of which you can convince yourself of God's exsistence. But I think that \emph{believers} who offered such proofs wanted to analyse \& make a case for their `belief' with their intellect, although they themselves would never have arrived at belief with their intellect, although they themselves would never have arrived at belief by way of such proofs. ``Convincing someone of God's existence'' is something you might do by means of a certain upbringing, shaping his life in such \& such a way.
Life can educate you to ``believing in God''. And \emph{experiences} too are what do this but not visions, or other sense experiences, which show us the ``existence of this being'', but e.g. sufferings of various sorts. And they do not show us God as a sense experience does an object, nor do they give rise to \emph{conjectures} about him. Experiences, thoughts,\,---\,life can force this concept on us.
So perhaps it is similar to the concept `object'.}]{Una demostración de Dios realmente debería ser algo por medio de lo que pudiéramos convencernos de la exsitencia de Dios. Pero pienso que los \emph{creyentes} que han ofrecido este tipo de demostraciones han querido analizar y presentar un argumento para su `creer' usando el intelecto, aún cuando ellos mismos nunca habrían llegado a creer por medio de este tipo de demostraciones. ``Convencer a alguien de la existencia de Dios'' es algo que podríamos hacer por medio de cierta crianza, moldeando la vida de esa persona en cierto modo.

La vida puede educarnos para ``creer en Dios''. Y las \emph{experiencias} también son las que hacen esto aunque no visiones, u otras experiencias de los sentidos, que nos mostrarían la ``existencia de este ser'', sino p.\,ej. sufrimientos de diversa índole. Y estos no nos muestran a Dios como una experiencia sensorial muestra un objeto, tampoco propician el surgimiento de \emph{conjeturas} sobre él. Las experiencias, los pensamientos,\,---\,la vida puede forzar este concepto en nosotros.

Así que quizás sí es similar al concepto de `objeto'.}

Desde luego que este tipo de ``creer en Dios'' no podría llegar a ser ``confiar en la Eucaristía'', por ejemplo, o creer en algún misterio o palabra de la revelación. Consiste más bien en una actitud hacia Dios y hacia el mundo, una especie de revalorización que se hace de las cosas de la vida desde lo que la creencia religiosa propone como lo profundamente importante. Wittgenstein lo explica así: \blockquote[{\cite[96--97]{wittgenstein1998cnv}}: \enquote{If the believer in God looks around and asks ``Where does everything I see come from?'' ``Where does all that come from?'', what he hankers after is not a (causal) explanation; and the point of his question is that it is the expression of this hankering. He is expressing, then, a stance towards all explanations.\,---\,But how is this manifested in his life?

It is the attitude of taking a certain matter seriously, but then \underline{at a certain point} not taking it seriously after all, \& declaring that something else is still more serious. Someone may for instance say that it is a very grave matter that such \& such a person has died before he could complete a certain piece of work; \& in another sense that is not what matters. At this point one uses the words ``in a deeper sense''.

Really what I should like to say is that here too what is important is not the \emph{words} you use or what you think while saying them, so much as the difference that they make at different points in your life. How do I know that two people mean the same thing when each says he believes in God? And just the same thing goes for the Trinity. Theology that insists \emph{certain} words \& phrases \& prohibits others makes nothing clearer. (Karl Barth)

It gesticulates with words, as it were, because it wants to say something \& does not know how to express it. \emph{Practice} gives the words their sense.}]{Si el creyente en Dios mira a su alrededor y pregunta ``¿De dónde proviene todo esto que veo?'' ``¿De dónde ha surgido todo esto?'', lo que está anhelando no es una explicación (causal); y el punto de su pregunta es que ella misma es la expresión de su anhelo. Lo que está expresando, entonces, es una actitud hacia toda explicación.\,---\,Pero, ¿cómo se manifiesta esto en su vida?

Lo hace en la actitud de tomar cierto asunto sériamente, pero entonces, \underline{en cierto punto} no tomándolo sériamente después de todo, y declarando que algo distinto merece todavía más seriedad. Por ejemplo alguien puede decir que es un asunto muy grave que tal o cual persona ha muerto antes de poder completar cierta obra; considerado según otro sentido eso no es lo que importa. En este punto usamos las palabras ``en un sentido más profundo''.

Lo que quiero decir realmente es que aquí también lo importante no son las \emph{palabras} que usamos o lo que estamos pensando mientras las decimos, sino más bien la diferencia que hacen en distintos puntos de nuestra vida. ¿Cómo conozco que dos personas distintas quieren decir lo mismo cuando cada una dice que cree en Dios? Y exactamente lo mismo ocurre con la Trinidad. Una teología que insiste en palabras y frases \emph{específicas} y prohibe otras no logra aclarar nada. (Karl Barth)

Gesticula con palabras, podría decirse, porque quiere decir algo y no sabe cómo expresarlo. La \emph{práctica} es la que da a las palabras su sentido.} Desde esta perspectiva, la noción de lo que un cristiano llamaría `fe' consistiría en esta actitud respecto de la vida y del mundo y de Dios, justificada por el trasfondo que van dejando las enseñanzas que la vida comunica por medio de experiencias como enfrentar el sufrimiento o la muerte. Además esta `fe' no se comunica en palabras precisas o verdaderas, sino que queda manifestada en la práctica, puesto que la `fe' misma consiste en esa actitud que se tiene hacia la vida.

Dentro de una concepción como esta, las proposiciones relacionadas con verdades reveladas o misterios quedan reducidas a una cierta actitud hacia las cosas, pero no expresan pensamientos: \blockquote[{\cite[211]{teichmann2008ans}}: It might be thought that a religious person who regards certain articles of faith as `mysteries' is more or less bound to embrace nonsense or self-contradiction; for what \emph{is} a mystery such as that of the Trinity, or of the Incarnation, or of the Eucharistic Transubstantiation, if not something whose appearance of incoherence cannot be dispelled by reason? If somebody utters `I believe' in connection with such mysteries, won't we be entitled to say, along with Wittgenstein: `But is this a belief, a thought at all? Perhaps there is a state of enlightenment, or an urge to find expression for certain experiences of life\,---\,but for there to be a belief, you would need to be able, at least in principle, to state that belief clearly and without contradiction'?]{Puede ser pensado que una persona religiosa que considera ciertos artículos de fe como `misterios' está en mayor o menor grado obligada a abrazar el sinsentido o la auto-contradicción; pues ¿qué \emph{es} un misterio como el de la Trinidad, o el de la Encarnación, o el de la Transubstanciación Eucarística, si no algo cuya apariencia de incoherencia no puede ser disipada por la razón? Si alguien dice `Yo creo' en conexión con tales misterios, ¿no estaríamos autorizados a cuestionar, junto con Wittgenstein: `¿Pero es esto una creencia, un pensamiento en absoluto? Quizás haya ahí un estado de iluminación, o un deseo de encontrar expresión para ciertas experiencias de la vida\,---\,pero para que haya una creencia, deberías ser capaz, al menos en principio, de enunciar esa creencia claramente y sin contradicción'?}

Si escuchamos a Anscombe en \emph{What is it to Believe Someone?}, lamentando que en su época se discuta sobre la fe haciendola equivaler a `creencia religiosa' y que se haya perdido de vista \enquote*{la asombrosa noción de una cosa tal como \emph{creer a Dios}}, no es difícil distinguir una voz diferente a la de Wittgenstein. ¿Cómo responde Elizabeth a la objeción anterior? Al decir \enquote*{yo creo en la Encarnación} ¿expresamos un pensamiento, una creencia?

Roger Teichmann propone que las ideas que están en el trasfondo de la descripción que Anscombe hace del misterio son las que expresa en los argumentos finales del artículo \emph{Parmenides, Mystery and Contradiction}. Allí vimos cómo Anscombe estudiaba la equivalencia de `puede ser captado en el pensamiento' con `puede ser presentado en una afirmación que pueda ser vista como teniendo un inobjetable sentido no contradictorio'. Esta equivalencia, además, la comparaba con la expresión del prefacio del \emph{Tractatus}: `aquello que pueda decirse del todo en palabras puede ser dicho claramente' y añadía que \enquote*{alguien que pensara esto pordría pensar que puede haber lo inexpresable}, y en este sentido que \enquote*{puede haber lo que no puede ser pensado}. La interpretación de Teichmann es que \blockquote[{\cite[212]{teichmann2008ans}}: \enquote{the equivalence is rejected by Anscombe; or rather it is taken as wanting justification, as is shown by the closing words of \textelp{Parmenides, Mystery and Contradiction}: `The trouble is, there doesn't seem to be any ground for holding this position. It is a sort of prejudice'}]{la equivalencia es rechazada por Anscombe; o más bien es juzgada como necesitada de justificación, como queda mostrado en las palabras finales de \textelp{\emph{Parmenides, Mystery and Contradiction}}: `El problema es que no parece haber ningún fundamento para sostener esta posición. Es una especie de prejuicio'}. Y añade: \blockquote[{\cite[212]{teichmann2008ans}}: \enquote{Anscombe would certainly admit that `can be grasped in thought' is incompatible with `can only be presented in a sentence with a contradictory sense'}]{Anscombe ciertamente admitiría que `puede ser captado en el pensamiento' es incompatible con `solo puede ser presentado en una oración con un sentido contradictorio'}. Esta incompatibilidad la encontramos expresada en \emph{The Question for Linguistic Idealism}. Allí, tras explicar que para Wittgenstein el pensar consiste en actuar según una regla, Elizabeth comentaba \enquote*{¿Qué diría Wittgenstein del pensamiento ilógico? ¿Como yo, que no es pensar?}.
\blockquote[{\cite[212]{teichmann2008ans}}: \enquote{What Anscombe is trying to make room for is the idea of grasping a thought which cannot be cleared up, i.e. cannot be shown to have a non-contradictory sense.]{}

La idea de creer a Dios nos rescata de mucha ambiguedad.
  verdad es uno de los nombres de Dios, para ella es un trascendental. Dios promete en lenguaje humano, se involucra en la actividad humana del lenguaje. Aquí es importante su comprensión de la fe como creer a alguien que se comunica, o entender alguna experiencia como palabra de Dios.

\subsubsection{\enquote{Poned en práctica la palabra y no os contentéis con oírla\ldots}}
