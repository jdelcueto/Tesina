\subsection{¿Cuál es el valor espistemológico del testimonio?}

\subsubsection{La `estructura' de creer en el testimonio}

\subsubsection{Traditional knowledge}

The work of determining England and fixing the meaning of the name \emph{would} depend on testimony\,---\,the testimony of many different people for different parts of it. The work done, people could be taught what Engalnd was (no doubt still disputing some regions). Now those who learned thereafter can hardly be said to have knowledge by testimony. They were taught to \emph{call} something `England'\,---\,something indeed which could in large part only be defined for them by hearsay; and they so taught those who came after them. I am an heir of this tradition. Now, I know I live in England. But by testimony? Some would say so. But there is something queer about it. \emph{What} do I know? That the world is divided up into countries which have names, and that the one I live in is called England and is here on the map of the globe. This involves understanding the use of the globe to represent the earth. It is rather as if I had been taught to join in \emph{doing} something, than to believe something\,---\,but because everyone is taught to do such things, an object of belief is generated. The belief is so certainly correct (for it follows the practice) that it is knowledge; for here knowledge is no other that certainly correct belief in pursuit of a practice. But the connection with testimony is remote and indirect.
