\subsection{¿Cuál es el valor epistemológico del testimonio?}

\emph{Creer y Conocer}, de Francisco Conesa, es una investigación sobre el valor cognoscitivo de la fe en la filosofía analítica. En su estudio, Conesa sitúa a Anscombe entre los autores que \blockquote[{\cite[84]{conesa1994cc}}.]{entienden la fe primordialmente como un saber por testimonio}. El análisis que el autor ofrece como fundamento para este modo de entender la perspectiva de Anscombe se enfoca en dos puntos. El primero es que para Anscombe el significado de la palabra `fe' es `creer a Dios'. Conesa resume este punto refiriéndose a la discusión del artículo \emph{Faith}: \blockquote[{\cite[87-88]{conesa1994cc}}.]{<<En la tradición donde el concepto tiene su origen, \emph{fe} es una abreviación de \emph{fe divina} y significa \emph{creer a Dios}>>. Y ¿qué puede significar \emph{creer a Dios}? Todos los casos de <<creer a ``$x$''>> suponen que ``$x$'' habla. Que alguien tiene fe quiere decir que cree que algo es palabra de Dios: <<fe es la creencia que él presta a esa palabra>>}. El segundo tema que compone la explicación de Conesa es del artículo \emph{Hume and Julius Caesar}: \blockquote[{\Cite[88]{conesa1994cc}}.]{Creer en el testimonio es muy distinto de creer en causas y efectos. Este punto es desarrollado por la filósofa al estudiar el conocimiento histórico: <<Creer en un relato histórico es absolutamente creer que ha habido una cadena de tradición de relatos y documentos que llega hasta el conocimiento contemporáneo; no es creer en los hechos históricos mediante una inferencia que vaya siguiendo cada nudo de esa cadena>>}.

Desde el punto de vista de Conesa, este modo de entender `fe', como `saber por testimonio', sirve para caracterizar el valor cognoscitivo que tienen las creencias que se sostienen sobre el fundamento de la fe. Su propuesta es que: \blockquote[{\Cite[88]{conesa1994cc}}.]{Desde esta perspectiva comprendemos el valor epistemológico de la fe religiosa, que consiste en \emph{creer a Dios}. Ella forma parte de ese conocimiento que depende del testimonio de otros. En este caso, además, creemos a alguien que conoce. Entonces es claro que accedemos a su conocimiento}. Aquí el autor afirma que el valor epistemológico que tiene la fe es el del `saber por testimonio', y en pocas palabras describe el valor epistemológico de este saber como el conocimiento al que accedemos cuando creemos a alguien que conoce, en este caso a Dios. En este apartado veremos con más detalle cómo Anscombe describe el valor epistemológico de estas creencias que sostenemos por el testimonio que hemos recibido. Las dos cuestiones que Conesa tiene en cuenta al valorar el pensamiento de Elizabeth nos servirán como marco de referencia para esta discusión.

\subsubsection{La `estructura' de creer en el testimonio}

Mary Geach, hija de Anscombe, cuenta una anécdota que le escuchó a su madre; cuando Elizabeth estaba en sus estudios universitarios se topó con un pasaje de Russell en su comentario de Leibniz que sostenía que un argumento construido desde los datos del mundo no sería válido para afirmar la existencia de Dios, pues no es posible deducir una conclusión necesaria desde una premisa contingente. En ese momento Anscombe no sabía qué está equivocado en la noción de que las necesidades solamente pueden ser deducidas de premisas necesarias, sin embargo, sí sabía que el negar la posibilidad de conocer la existencia de Dios por medio de las cosas creadas a la luz de la razón era negar una doctrina de fe definida por la enseñanza de la Iglesia. Decidió, entonces, ir a una iglesia y hacer un acto de fe. Más tarde en su carrera filosófica llegó a ver cómo argumentar que pueden deducirse conclusiones necesarias de premisas contingentes, pero en aquel momento su acto de fe le evitó caer en un error.

Mary destaca un aspecto interesante de esta anécdota: \blockquote[{\Cite[xvi-xvii]{anscombe2008faith}}: \enquote{Faith, \textelp{} is believing God, but this story shows how public she believed the voice of God could be, speaking as it has done in the teaching of the Church}.]{La fe, \textelp{} es creer a Dios, y esta historia muestra cuán pública ella creía que la voz de Dios puede ser, hablando como lo hace en la enseñanza de la Iglesia}. Es difícil entender bien el modo en que Elizabeth habla de la fe si no se tiene en cuenta esta creencia suya. Anscombe habla de Dios como uno que está involucrado en la actividad humana del lenguaje, tiene una `voz pública'. En términos generales, incluso, se puede decir que Anscombe entiende por `fe', en sentido estricto, `creer a Dios', y `fe humana' es en cierto modo el uso análogo. La decisión tomada por Anscombe fue creer a Dios creyendo que Él habla en la enseñanza de la Iglesia. Mary Geach valora esta actitud en su reflexión de la anécdota y comenta que \blockquote[{\Cite[xvii]{anscombe2008faith}}: \enquote{philosophers nowadays accept on authority much that they do not themselves have the expertise to know firsthand, and they do not see it as a limitation on their freedom}.]{hoy en día los filósofos aceptan mucho que ellos mismos no tienen la capacidad para conocer de primera mano, y esto no lo ven como una limitación de su libertad}. Aceptamos creencias apoyados en la autoridad de peritos y, si están en lo correcto, esta aceptación no implica una limitación de nuestra libertad. Lo mismo se puede considerar respecto de la enseñanza de la Iglesia: \blockquote[{\Cite[xvi-xvii]{anscombe2008faith}}: \enquote{To proceed on the assumption that this teaching is true is seen by some as a limitation on one's freedom, but this is only the case if the Church does not have the teaching authority she claims to have}.]{Proceder con el presupuesto de que esta enseñanza es verdadera es visto por algunos como una limitación a nuestra libertad, pero esto solo es el caso si la Iglesia no tiene la autoridad para enseñar que declara tener}.

El nexo entre la discusión tratada en los artículos \emph{What is it to believe someone?} y \emph{Faith} es ese dato: `fe' como la creencia depositada en lo que se nos comunica ---apoyados, entre otras cosas, en la autoridad del que comunica--- y estas creencias como componentes `no desprendibles' de nuestro conocimiento de la realidad más allá de nuestra experiencia personal. Anscombe parte de la descripción de Hume: la justificación para que sea razonable creer el testimonio consiste en la inferencia que hacemos de que al testimonio le sigue la verdad como se siguen los efectos de las causas. Tras expresar su desacuerdo, ella propone en cambio que hemos de reconocer al testimonio como un medio que nos da acceso a una visión más amplia del mundo del mismo modo, o incluso en mayor grado que la relación causa y efecto. A esto añade que \enquote*{creerlo es muy distinto en estructura que la creencia en causas y efectos}. Este comentario sugiere la pregunta: ¿en qué consiste, desde su perspectiva, la `estructura' de la creencia en el testimonio?

Podemos situar la respuesta de Anscombe en torno a las dos argumentaciones antes referidas por Conesa. La primera es la descripción que ella hace de lo que significa creer a alguien. Su propuesta es que una persona está en la situación de atender la pregunta acerca de creer o dudar (suspender el juicio ante) alguien cuando están dadas toda una serie de presuposiciones; entonces, libre de confusiones por las preguntas que podrían surgir relacionadas con estos presupuestos, creer a alguien acerca de algo en particular es confiar en esa persona sobre la verdad de ese asunto en particular.

La segunda argumentación está en su discusión sobre el conocimiento histórico. En efecto, como piensa Hume, el hecho de que tenemos creencias justificadas sobre fundamentos que se consideran premisas de argumentos, presupone que hay creencias sin fundamento, o al menos, que no tienen como fundamento algo que pueda considerarse como premisa de un argumento. Es decir, debe haber un fundamento último para nuestras creencias que no sea otra inferencia, sino de otra naturaleza. Para Hume estos fundamentos últimos son las impresiones de nuestros sentidos. Anscombe no piensa así. Se pregunta: ¿por qué las cosas que se nos dicen y los escritos que vemos \emph{son} el punto de partida para nuestro creer en eventos distantes y también en la cadena de transmisión de esta información?, ¿por qué creemos los testimonios e informes que recibimos de estos hechos? Su respuesta es que los fundamentos últimos de estas creencias se encuentran en el conocimiento tradicional o común, aquellas creencias de las cuales diríamos \enquote*{¡Todo el mundo sabe eso!}.

La comprensión de lo que Elizabeth llama la `estructura' de creer en el testimonio nos servirá para responder a la pregunta sobre su valor epistemológico. Con este objetivo examinaremos ambas cuestiones más detenidamente.

\subsubsection{¿Qúe es creer a alguien?}

Se podría decir que la actitud que caracteriza la reflexión de Anscombe sobre el creer obedece a la consigna Wittgensteniana: \enquote*{te enseñaré las diferencias}. A lo largo de su discusión se encuentran diversas distinciones y matizaciones sobre el modo en que empleamos la expresión `creer' cuando decimos que creemos algo que alguien nos ha dicho y también cómo actuamos según ese tipo de creencias.

Una de las primeras distinciones que Anscombe enfatiza en su investigación en \emph{What is it to Believe Someone?} es acerca de los fundamentos de nuestra creencia al recibir un testimonio. Creer a alguien no consiste simplemente en creer lo que alguien me dice o tenerlo por verdadero. El pequeño relato que encabeza el ensayo le sirve para ilustrar esta distinción. El diálogo está construido según una conjunción de premisas que en otro artículo ella llama un `extraño patrón de argumento'\footnote{\cite[Cf.][299]{anscombe2015logic:qpa}: \enquote{The pattern to which my title refers is: $1^{o}$ If $p$, then $q$. $2^{o}$ If $r$, then not (if $p$ then $q$). $3^{o}$ If not $p$ then $r$. $\mathbf{\therefore}$ $p$ and $q$. We get `not $r$' from the first two premises and then `$p$' from `not $r$' and the third; with the first one again this gives us the conclusion}.}. La característica peculiar de este patrón es que es formalmente válido y sus premisas compatibles, pero las premisas dadas no sirven para fundamentar la creencia en la conclusión. El escenario que Anscombe usa como ejemplo culmina con la expresión de Eutidemo: \enquote*{Les creo a todos. Así que infiero que el árbol caerá y el camino quedará obstruido}; entonces Elizabeth propone: \enquote*{¿Qué equivocación tiene Eutidemo?}. La pregunta clave que nos está invitando a considerar ante la inferencia de Eutidemo es: \enquote*{¿cuál es el fundamento real para creer la conclusión?}. Ella explica que: \blockquote[{\Cite[301]{anscombe2015logic:qpa}}: \enquote{The peculiarity of our case is that there doesn't seem to be any difficulty about reasonably judging any of the three premises to be true without having already judged the conclusion or part of it to be true. The difficulty lies in combining them in knowledge, or in a reasonable judgement, unless part of the conclusion is part of the ground for accepting the combination. One wants to say: that you can get this conclusion out of these three propositions is ground for doubting the conjunction of them! But the reason is not that the conclusion is itself false, let alone absurd. It is a perfectly possible proposition, and is objected to only as a conclusion from perfectly possible propositions, which are mutually compatible and from which it does follow}.]{La peculiaridad de este caso es que no parece haber ninguna dificultad para juzgar razonablemente cualquiera de las tres premisas como verdadera sin haber juzgado de antemano la conclusión o parte de ella como verdadera. La dificultad se encuentra al combinarlas como un conocimiento, o un juicio razonable, a no ser que parte de la conclusión sea parte del fundamento para aceptar la combinación. Lo que quiero decir es: ¡el que podamos llegar a esta conclusión desde estas tres proposiciones es fundamento para dudar de la conjunción de ellas! Pero la razón no es que la conclusión misma sea falsa, ni mucho menos absurda. Es una proposición perfectamente posible, y es objetada solo como la conclusión de proposiciones perfectamente posibles, que son mutuamente compatibles y desde las que sí se sigue}.

A esta característica Anscombe la llama `retractabilidad esencial'. Con esto quiere decir que un juicio como el que la conclusión de este argumento expresa, aunque se sigue de la conjunción de sus premisas, es retractable por algún elemento o circunstancia externa que haga irrazonable deducir válidamente la conclusión desde la conjunción de estas premisas\footnote{\cite[Cf.][299]{anscombe2015logic:qpa}: \enquote{Then we have perhaps discovered the special character of (theoretical) hypotheticals whose consequents don't follow logically from their antecedents. We might call this character `essential defeasibility'. This will be the reason why, even though `not $r$' follows from `if $p$ then $q$ and if $r$, then not (if $p$ then $q$)', still it may be highly unreasonable to deduce `not $r$' from that conjunction}.}. ¿Cuál sería el elemento externo que sirve como fundamento para la validez de la creencia en una conclusión en el caso de creer a alguien? Anscombe responde \enquote*{Para creer a $N$ debemos creer que $N$ mismo cree lo que dice}. En el ejemplo de Elizabeth, la inferencia de Eutidemo expresa un juicio basado en la conjunción de las premisas, él podría decir: \enquote*{es razonable juzgar que el árbol caerá e interrumpirá el paso pues esta conclusión se sigue de la conjunción de afirmaciones hechas por $A$, $B$ y $C$}\footnote{Es pertinente recordar aquí que para Anscombe una inferencia valida como conclusión lógica tiene que ser juzgada dentro de la actividad humana: \cite[121]{anscombe1981parmenides:qli}: \enquote{Valid inference, not logical truths, is the subject matter of logic; and a conclusion is justified, not by rules of logic but, in some cases by the truth of its premisses, in some by the steps taken in reaching it, such as making a supposition or drawing a diagram or constructing a table}.}. Ahora bien, al justificar esta inferencia diciendo \enquote*{les creo a todos}, suena como un loco, pues no ha juzgado si $A$ cree lo que ha dicho después de haber escuchado a $B$ y $C$. Está afirmando un juicio que no puede quedar justificado por la conjunción de las premisas, aunque se sigue de esta, y que, según su propia expresión, solo puede tener como fundamento real la creencia de que los tres personajes creen lo que están diciendo. Al no tener en cuenta qué creen $A$, $B$ y $C$, su inferencia queda sin fundamento válido.

La manera en que Anscombe establece esta distinción parece extraña, sin embargo es útil, puesto que sirve para describir con mayor claridad la disposición que alguien tiene cuando cree un testimonio. Elizabeth añade que hay un gran número de juicios que siguen este tipo de patrón\footnote{\cite[Cf.][302]{anscombe2015logic:qpa}: \enquote{There are large numbers of hypothetical judgements that are like this. It is an interesting and important observation that there is a whole class of judgements such that when we make them we are not implicitly dismissing as false everything that would falsify them. In contrast, when I make a categorical statement with appropiate confidence, it is very often the case that I can straightway rule out as false what would falsify it\,---\,just because I know that \emph{it} is true}.}, incluso, su peculiar carácter no solo se encuentra relacionado con la dinámica de creer a alguien en el sentido de `fe humana', sino que también se le puede encontrar en el `creer a Dios'\footnote{Otro de los ejemplos de argumento que siguen el patrón que Anscombe discute en el artículo \emph{On a Queer Pattern of Argument} es un razonamiento hipotético de Isaac al conocer que él era el sacrificio a ser ofrecido por Abrahám, el argumento, dice: \cite[Cf.][309]{anscombe2015logic:qpa}: \enquote{might be produced by a less evasive and tortous Johannes de Silentio picturing Isaac in the interval in which he has realised that \emph{he} is the intended sacrifice, and before Abraham's hand is stayed. Isaac reasons: $1''''$  If God has promised my father that he will be the father of a great nation through me, then my father will be. $2''''$  If my father kills me, it's not true that if God has promised him he will be the father of a great nation through me, then he will be. (\emph{Therefore he is not going to kill me}.) $3''''$  If God has not promised my father that he will be the father of a great nation through me, my father is going to kill me. $\mathbf{\therefore}$  God has promised that to my father and it will be fulfilled. This argument differs from all the other in that in the first proposition the consequent necessarily follows from the antecedent}.}.

Otros ejemplos que Elizabeth usa para insistir en que al creer a alguien, la disposición que la palabra `creer' expresa es la intención de tener por verdadero que \enquote*{$N$ cree lo que me dice} son: `creer' con un objeto personal no puede ser reflexivo, es decir, podemos `decirnos algo' a nosotros mismos, pero no podemos decir que `nos creemos a nosotros mismos' sobre algo; también sugiere que decir a alguien \enquote*{te creo} cuando la información es algo de conocimiento común (p. ej. Napoleón perdió la batalla de \emph{Waterloo}), la declaración suena a chiste; también sonaría a chiste decir que creo a alguien en el caso de que crea lo que me diga, pero porque estoy convencido de que me miente y además está equivocado en lo que cree y por ese cálculo creo lo que me dice porque me lo ha dicho, pero no le creo fiable.

Otra distinción que Elizabeth establece es entre las `presuposiciones' ---que son las creencias adicionales involucradas en creer a alguien--- y aquello que se cree porque se cree a alguien, es decir, el contenido de la comunicación. Esta distinción juega un papel importante en su descripción de lo que es `fe' en el artículo \emph{Faith}. Allí recordaba que el carácter de racionalidad que se le atribuía a las creencias de la fe había sido justificado en una época sobre los llamados `preámbulos' de la fe y el paso de estos a la fe misma, sin embargo, ella propone que la designación adecuada para al menos parte de estos es más bien `presuposiciones'.

Con este cambio, confiere a estas otras creencias involucradas en el `creer a alguien que $p$', o `creer a Dios que $p$' el papel de justificar el carácter de racionalidad que puede atribuírsele a la fe. Anscombe añade que en sentido estricto las presuposiciones no forman parte del contenido de lo que se cree por la fe. Esto lo afirma en el ejemplo de la carta de Jones, o de la carta que recibe el prisionero. Creer que la carta viene de Jones no es una decisión que se toma teniendo como garantía la credibilidad de Jones, lo mismo ocurre con creer que $N$, el que envía la carta al prisionero, existe; la creencia en su existencia y la creencia en el contenido de la carta son lógicamente diferentes.

Las creencias involucradas en las presuposiciones que Elizabeth analiza son principalmente tres: al decir que creemos a alguien tenemos como presupuesto que la comunicación \emph{es de alguien}, que lo que quiere decir \emph{es esto} y que la comunicación \emph{está dirigida a alguien}. Estas creencias caracterizan nuestra disposición ante la comunicación misma y, como se ha insistido, no constituyen lo que en sentido estricto Anscombe llama fe, sino que son presupuestos relacionados con ella.

Dentro de la primera de estas tres creencias Anscombe destaca otros matices que ofrecen más elementos para describir el valor epistemológico del testimonio. Anscombe explica que al creer que una comunicación es de alguien se cree a una persona que puede tener diversos grados de autoridad. Dos ejemplos distintos de autoridad que ella presenta son el caso del testigo y el maestro. Cuando habla de un testigo se refiere a él como uno que es una autoridad original en el sentido de que contribuye algo. El testigo no solo transmite información recibida, aunque generalmente su testimonio está influenciado o compuesto por la información que él ha recibido. Adicionalmente, un testigo puede considerarse como una autoridad \emph{totalmente} original cuando su testimonio sobre una realidad específica no se apoya sobre información recibida.

Es distinta la situación peculiar del que se comunica como maestro. Este caso no es el mismo que cuando el productor inmediato de la comunicación es un interprete o mensajero. Creer lo que estos dicen implica creer a su principal, que es el que habla. El interprete no se equivoca si lo que dice no es verdad, siempre y cuando que comunique lo que su principal ha dicho. El maestro sí se equivoca cuando lo que dice no es verdad. Esto tiene que ver con que cuando sus alumnos creen lo que enseña le creen a él. Se tiene en cuenta su credibilidad como fundamento para creer lo que comunica, aún cuando no sea una autoridad original de lo que enseña, como ocurre en el caso del testigo. La autoridad que tiene la enseñanza del maestro recae sobre el sistema de enseñanza y la tradición de conocimiento del que forma parte\footnote{\cite[Cf.][214]{teichmann2008ans}: \enquote{we all believe, things taught\,---\,not because we have established the reliability of the teacher, but because of the set-up of teaching and learning}.}.

El último elemento de la discusión de Anscombe para tener en cuenta aquí es la cuestión con la que cierra el ensayo \emph{What is it to Believe Someone?}. Ella compara dos `cálculos' que podemos encontrarnos haciendo ante una comunicación de $NN$ sobre $p$; en uno creemos lo que $NN$ dice como resultado del cálculo de que miente y se equivoca, en el otro, creemos lo que dice porque calculamos que es veraz y está en lo correcto. Ante esto plantea la duda: ¿Por qué estamos dispuestos a decir que creemos a $NN$ solo cuando creemos que está en lo correcto y es veraz en su intención?, ¿cuál es la diferencia entre los dos casos, dado que ambos culminan en la creencia que $p$ porque $NN$ ha dicho que $p$?
\label{subsec:verdad}

Detrás de este cuestionamiento está toda la fuerza de las consideraciones del \emph{Tractatus} sobre la verdad y la negación. Anscombe misma advierte en su análisis de la negación en el \emph{Tractatus} que \blockquote[{\Cite[19]{anscombe1959iwt}}: \enquote{`not', which is so simple to use, is utterly mystifying to think about; no theory of thought or judgment which does not give an account of it can hope to be adequate}.]{el `no', que es tan simple de emplear, es totalmente desconcertante cuando pensamos sobre él; ninguna teoría sobre el pensamiento o el juicio puede aspirar a ser adecuada si no ofrece una descripción de él}. El objetivo de Anscombe es ofrecer una descripción adecuada sobre el juicio que se realiza al creer a alguien, y así no ha de causar extrañeza que se cuestione sobre nuestra disposición ante una creencia que adquirimos por un cálculo basado en la falsedad y la negación. Es decir, la discusión no está completa si no pensamos por qué no llamamos `creer a alguien' cuando es el caso que podríamos decir \enquote*{creo esto porque $NN$ lo ha dicho y juzgo que lo que dice es falso y $NN$ no es veraz}.

Este asunto quedó sin respuesta en este artículo, sin embargo Elizabeth lo desarrolla en otros dos lugares: la ponencia presentada en la Universidad de Navarra en 1983 con el título \emph{Truth} y otra lección ofrecida en \emph{John Hopkins University} en 1987 titulada: \emph{Truth, Sense and Assertion}. En la primera discusión Anscombe trabaja la pregunta \enquote*{¿cuál es la primacía de la verdad sobre la falsedad?} y para su análisis indaga en las aportaciones de Wittgenstein y de San Anselmo, a quienes considera `hermanos intelectuales' en esta materia\footnote{\cite[Cf.][73]{anscombe2011plato:truth}: \enquote{`$p$' and `${\sim}p$' are opposite in sense, but to them corresponds just one reality. What reality? Well, the fact, the \emph{res enunciata} by the true one. This comes so close to saying that truth and falsehood are a sort of equal relations between sign and thing signified, and that one proposition ---whichever of the two it is--- signifies in the true way what the other signifies in the false way, that we wonder: what then \emph{is} unequal about them? What \emph{is} the primacy of truth? Wittgenstein is also \emph{épris} with this, and he and Anselm are intellectual brothers on the subject}.}. En la segunda reflexión Elizabeth incluye en este debate a los sofistas y sus ideas sobre `pensar falsamente'\footnote{\cite[264]{anscombe2015logic:tsa}: \enquote{\textins{Protagoras} didn't believe there was any such thing as false opinion\,---\,anything anyone thinks is true, it's like perception, it's how things appear to him}.}.

En Wittgenstein encontramos la cuestión planteada en el \emph{Tractatus}. Se pregunta: dado que las proposiciones son capaces de significar tanto si son falsas como cuando son verdaderas y teniendo en cuenta que en ambos casos se refieren a una misma realidad, \blockquote[{\Cite[\S4.062]{wittgenstein1922tractatus}}: \enquote{Can we not make ourselves understood by means of false propositions as hitherto with true ones, so long as we know that they are meant to be false?}]{¿Acaso no podríamos hacernos entender usando proposiciones falsas tal como hemos hecho hasta ahora por medio de las verdaderas, siempre y cuando sepamos que están significadas falsamente?}. Su respuesta es: \blockquote[{\Cite[\S4.062]{wittgenstein1922tractatus}}: \enquote{No! For a proposition is true, if what we assert by means of it is the case; and if by ``$p$'' we mean ${\sim}p$, and what we mean is the case, then ``$p$'' in the new conception is true and not false}.]{¡No! Pues una proposición es verdadera, si aquello que enunciamos por medio de ella es de hecho; y si por ``$p$'' queremos decir ${\sim}p$, y las cosas son como queremos decir que son, entonces ``$p$'' es verdadero en nuestro nuevo modo de tomarlo y no falso}. Anscombe ve en esto el comienzo de una respuesta. Es útil distinguir entre la proposición y la `aserción' o `enunciación' de lo que la proposición significa. Al enunciar una proposición falsa para afirmar algo que es de hecho esta proposición es concebida como la enunciación o aseveración de una verdad. Esto es aceptable para Anscombe: \blockquote[{\Cite[75]{anscombe2011plato:truth}}: \enquote{Thus true and false are supposed \emph{not} to be `equally justified relations' because the false could not take over the role of the true in assertion and thought. This we can accept}.]{De este modo se supone que verdadero y falso \emph{no} tienen `relaciones igualmente justificadas' porque falso no podría reemplazar el rol de verdadero en la aserción y el pensar. Esto lo podemos aceptar}. Sin embargo, objeta que esto no termina de atender el problema: \blockquote[{\Cite[75]{anscombe2011plato:truth}}: \enquote{But lies are possible. With a lie one means to assert as being the case what is not the case. Also error is possible. When one's assertions are mistaken, what one means to assert as being the case is again not the case. The general impossibility of exchanging the roles of true and false does not exclude either lies or error. Does the general impossibility then contain the whole substance of the `not equally justified relations'? It may give a primacy to truth over flasehood in theory of meaning; but why should that be called a more \emph{justified} relation because of that?}]{Pero las mentiras son posibles. Con una mentira tenemos la intención de enunciar como siendo de hecho algo que no es de hecho. También es posible el error. Cuando nuestras aserciones están equivocadas, aquello que tenemos la intención de afirmar como siendo de hecho, nuevamente, no es de hecho. La imposibilidad general de intercambiar los roles de verdadero y falso no excluye ni las mentiras ni el error. Entonces, ¿acaso esta imposibilidad general contiene toda la sustancia de las `relaciones no igualmente justificadas'? Puede que otorgue a la verdad cierta primacía sobre la falsedad en la teoría del significado; pero, ¿por qué habría de ser motivo para considerarla una relación más \emph{justificada}?}

Elizabeth considera lo que San Anselmo tiene que decir sobre el tema. La pregunta clave para esta discusión es: \enquote*{¿Cuál es el fin de la afirmación?}. El cuestionamiento surge dentro del diálogo entre un discípulo y su maestro. El maestro ha preguntado: \enquote*{¿Cuál te parece ser aquí la verdad?} y la respuesta del discípulo ha sido \enquote*{No sé más que, cuando significa existir lo que existe realmente, está en ella la verdad y es verdadera}. Y es ante esta respuesta que el maestro dirige la atención hacia la finalidad de la afirmación. El argumento de Anselmo llevará a la conclusión de que la verdad del enunciado no es la \emph{res enunciata} por una proposición verdadera, tampoco está en la significación, o en cualquier cosa perteneciente a la definición, sino que cuando una afirmación hace aquello para lo que es, la significación (\emph{significatio}) está hecha rectamente y esta rectitud es lo que la verdad es\footnote{El fragmento del diálogo se desarrolla como sigue: \cite[495]{anselm1952obras:deveritate}: \enquote{\emph{Maestro}---¿Cuál es el fin de la afirmación?  \emph{Discípulo}---Expresar lo que es.   \emph{M.}---¿Debe, pues, hacerlo? \emph{D.}---Ciertamente.  \emph{M.}---Por consiguiente, cuando expresa la existencia de lo que existe, expresa lo que debe.  \emph{D.}---Es evidente.  \emph{M.}---Y cuando expresa lo que debe, expresa con exactitud.  \emph{D.}---Así es. \emph{M.}---Pero cuando expresa con rectitud, ¿su significación es exacta?  \emph{D.}---Sin duda ninguna.  \emph{M.}---Cuando expresa la existencia de lo que es, ¿la significación es recta?  \emph{D.}---Es una conclusión que se impone.  \emph{M.}---Igualmente, cuando significa la existencia de lo que existe, su significado es verdadero. \emph{D.}---Ciertamente es a la vez verdadera y recta cuando expresa la existencia de lo que es.  \emph{M.}---¿Entonces es una misma y única cosa para ella el ser recta y verdadera, es decir, manifestar la existencia de lo que es?  \emph{D.}---Es una sola y misma cosa.  \emph{M.}---Por consiguiente, para ella, la verdad no es otra cosa que la rectitud.  \emph{D.}---Sí; veo con claridad que la verdad no es más que esta rectitud.  \emph{M.}---Lo mismo hay que decir cuando la enunciación expresa la no existencia de lo que existe}.}. El discípulo reacciona diciendo que ve cómo la verdad es esta rectitud y entonces lanza ---en palabras de Anscombe--- \blockquote[{\Cite[75]{anscombe2011plato:truth}}: \enquote{a bomb of a question}]{una bomba de pregunta} que consiste en: \enquote*{Cuando una expresión significa que es algo que no es, ¿se puede decir que está significando lo que debe?}. La respuesta del maestro no deja de ser menos sorprendente: \blockquote[{\Cite[494]{anselm1952obras:deveritate}}.]{veritatem tamen et rectitudinem habet, quia facit quod debet}. Una expresión falsa hace lo que debe en significar aquello que le ha sido dado significar, hace aquello para lo que la expresión es. Sin embargo, teniendo este modo de ser verdadera, no solemos llamarla verdadera pues habitualmente decimos que la expresión es verdadera y correcta solo cuando significa que es aquello que es y no cuando significa que es aquello que no es, pues tiene mayor deber de hacer aquello para lo que se le ha dado significar que para lo que no se le ha dado. Es sorprendente que el maestro no rechace la descripción del discípulo, más aún que la reitere. La objeción presentada no supone un impedimento para sostener esta descripción de la verdad. El maestro retiene su explicación apoyada en que la verdad de un enunciado es que hace lo que debe\footnote{\cite[Cf.][76]{anscombe2011plato:truth}: \enquote{This doing what it ought lies precisely in signifying what it does, i.e. in signifying what it's been given it to signify. But it's customarily called right and true only when it signifies the being so of what it is so, not when it signifies that something is so when it isn't. For it ought more to do what it's been given signification for than what it wasn't given it for. With this he retains the explanation starting from the question `What is affirmation \emph{for}?'}.}. ¿En qué consiste, entonces, la primacía de la verdad según San Anselmo? La proposición verdadera hace lo que debe de dos maneras: significa justo aquello que se le ha dado significar ---independientemente de si es el caso que es de hecho o no--- y significa aquello para lo que se le ha dado esa significación, esto es, afirmar como que es de hecho lo que \emph{es} el caso. Calificamos de justa y verdadera la proposición en virtud de ese hacer doblemente lo que debe, es decir, por su rectitud y verdad\footcite[Cf.][497]{anselm1952obras:deveritate}. Esta descripción de la verdad que Anselmo comienza aquí le llevará por medio de consideraciones sobre la verdad en el pensamiento, la voluntad, la acción y el ser de las cosas a su conocida definición de la verdad como \emph{veritas est rectitudo sola mente perceptibilis}\footcite[522]{anselm1952obras:deveritate}.

Anscombe encuentra la última pieza de su respuesta a toda esta cuestión en las ideas de los sofistas. En esta ocasión ella misma formula la pregunta, que expone diciendo: \blockquote[{\Cite[271]{anscombe2015logic:tsa}}: \enquote{Is enuntiation the same as signification?}]{¿Es la enunciación lo mismo que la significación?}. El sentido de un enunciado es el mismo cuando este es verdadero o falso, pero ¿se puede decir lo mismo de la enunciación en sí? La proposición verdadera tiene una \emph{res enuntiata}, ¿hay algo enunciado cuando una proposición es falsa? Para el sofista todo lo que opina cualquier persona es verdad, lo que viene al pensamiento es como la percepción, es el modo en que las cosas se presentan a cada uno. Desde esta idea, el sofista inventa el argumento de que \blockquote[{\Cite[264]{anscombe2015logic:tsa}}: \enquote{`He who thinks what is false thinks what is not; but what is not isn't anything; so he who thinks what is false isn't thinking \emph{anything}, but if he isn't thinking anything, he isn't thinking.'}.]{Aquel que piensa lo que es falso piensa lo que no es; pero lo que no existe no es nada; así que el que piensa lo que es falso no está pensando nada, pero si no está pensando nada, no está pensando}. Anscombe propone entonces lo que considera \blockquote[{\Cite[271]{anscombe2015logic:tsa}}: \enquote{the last bit, the keystone of the arch representing the relations of truth, sense and assertion}.]{el último pedazo, la piedra angular del arco que representa las relaciones entre verdad, sentido y aserción}, dice: \blockquote[{\Cite[271]{anscombe2015logic:tsa}}: \enquote{Where the Sophists were right is reached in my present formulation: the false proposition, while it does \emph{say something}, does not, being believed, \emph{tell} its believers anything. So: he who thinks what is false thinks what is not; he thinks something which tells him nothing; but that does not mean he thinks nothing, i.e. does not think anything}.]{Se llega a donde los Sofistas estaban en lo correcto en mi presente formulación: la proposición falsa, mientras que sí \emph{dice algo}, no es el caso que, al ser creída, \emph{enuncie} a sus creyentes cosa alguna. Así: aquel que piensa lo que es falso piensa lo que no es; piensa algo que le dice nada; pero esto no significa que piense nada, es decir, que no esté pensando en nada}. Según Anscombe una proposición verdadera refleja la existencia de lo que sí es, mientras que la situación análoga en la proposición falsa es que refleja la existencia de aquello que no es; ambos, la existencia reflejada y aquello que no es, son nada\footnote{\cite[271]{anscombe2015logic:tsa}: \enquote{a proposition believed \emph{tells} its believer something.\,---\,But only if it is true. For then it reflects the being so of what \emph{is} so. But the analogue of this, for a false proposition, would be that it reflects the being so of what is not so. And there is \emph{no} such thing as either}.}. En ese sentido, la proposición falsa, aunque dice o expresa un signo, no transmite o informa nada, puesto que lo que refleja no es. Esto también nos permite tener en cuenta que una aserción no solo tiene como objeto la proposición afirmada, sino que además tiene un sujeto personal. La persona usa la proposición para afirmar lo que la proposición significa. La proposición cumple con la tarea de significar siendo falsa o cierta, la persona que la usa para afirmar, en este sentido, tiene un deber mayor de emplearla para significar la existencia de lo que sí es\footnote{\cite[Cf.][267]{anscombe2015logic:tsa}: \enquote{a proposition, true or false, performs the task of signifying what it does, and the person who asserts it also uses it to signify what it does, but there is a further duty, on the part of one asserting, of signifying as being the case only what is the case. He can use the proposition so, because if it is the complete thing that is said, that is properly what it is for}.}. Hecha esta distinción, se puede decir que una persona enuncie una falsedad, pero esta proposición, si es creída, no informa a su creyente. El pensamiento que se construya desde esa creencia dice algo que no informa de nada\footnote{\cite[Cf.][271]{anscombe2015logic:tsa}: \enquote{A true proposition tells one something if one believes it. A false proposition believed still tells its believer nothing. A \emph{person} may tell one a falsehood; but, just as we say that a proposition as well as a person \emph{says} such-and-such, so we may also say that a proposition believed \emph{tells} its believer something.\,---\,But only if it is true}.}. Una paradoja, por otra parte, no solo no informa o enuncia, sino que no dice o expresa nada\footnote{\cite[Cf.][271]{anscombe2015logic:tsa}: \enquote{A paradox, on the other hand, does not say \emph{anything}}.}.

¿Por qué decimos que creemos a alguien solo cuando juzgamos que es veraz y dice la verdad? Cuando se cree a alguien se está haciendo un juicio del significado de su comunicación y la \emph{res enuntiata} que expresa. Sin embargo este juicio no establece la veracidad de la comunicación. Para eso el creyente juzga la rectitud del que se comunica y de su afirmación y es sobre esta que se establece la veracidad. La persona que usa la proposición para afirmar lo que es de hecho está empleando la aserción rectamente. Esta rectitud perceptible a la mente del creyente es la que permite hacer un juicio sobre la verdad. Este tipo de `cálculo' o juicio tiene primacía sobre un juicio fundado sobre la negación y la falsedad, no solo porque la falsedad no puede reemplazar el rol de la verdad en la enunciación, sino además porque la proposición falsa o la persona que dice una falsedad no comunica con la voluntad o intención de informar, sino que expresa un signo que no informa nada.

Siguiendo a Conesa hemos considerado el valor epistemológico de la fe en el pensamiento de Anscombe como `saber por testimonio' y esto supone que el testimonio cuenta con un valor espistemológico que caracteriza el saber que podemos atribuirle a lo que creemos por la fe. El estudio de la posibilidad de valorar lo que creemos por testimonio como creencia verdadera justificada dentro de la obra de Anscombe ha tenido como punto de partida el análisis de lo que ella llama la `estructura' del creer en el testimonio; un aspecto de esta estructura es que la naturaleza de la creencia en el testimonio puede ser descrita como `creer a $x$ que $p$'. `Creer que $p$' en este caso implica como presupuestos las creencias de que la comunicación viene de alguien, dice esto y va dirigida a alguien. Cuando no hay duda respecto de estos presupuestos estamos en la situación de elegir creer a $x$ o suspender el juicio ante $x$. Creerle consistiría en confiar en $x$ acerca de la verdad de $p$ en particular. Confiar en la verdad implica que se juzga que $x$ cree que $p$ y que $x$ actúa rectamente al enunciar que $p$ con la intención de afirmar cómo son las cosas de hecho.

Cabe añadir que Anscombe detalla que el receptor de la comunicación puede \emph{fallar en creer}, si no nota la comunicación o no la entiende como lenguaje o no la toma como dirigida a él o la malinterpreta o no cree que viene de quien se comunica. En este caso no podemos decir que la persona ha dudado o no creído la comunicación, sino que no ha llegado a estar en la situación de realizar ese juicio.

Como consideración final en este punto podemos tener en cuenta una última distinción que Elizabeth propone. Ella dice: \blockquote[{\Cite[175]{anscombe2015logic:bt}}: \enquote{Belief, and even conviction and certainty, are states \textelp{} `belief' signifies a state of the believing subject. So much seems clear at first, however difficult it may be to give an account of that state}.]{Creer, e incluso la convicción y la certeza, son estados \textelp{} `creer' siginifica un estado en el que se encuentra el sujeto creyente. Al menos esto queda claro en la primera impresíon, independientemente de la dificultad que pueda haber de ofrecer una descripción de ese estado}. En el caso de creer a alguien nuestro lenguaje nos sugiere pensar sobre el creer como un acto, sin embargo, Anscombe se inclina más a hablar del creer como una disposición: \blockquote[{\Cite[154]{anscombe2015logic:bt}}: \enquote{In innumerable cases, I believe something I am told. When? Well, when I am told. That again makes it look as if `I believed it' were the report of an act which took place at the time. But \textelp{a} question of duration shows it is not so: for the duration of belief is not the duration of any action. \textelp{} the question `\emph{How long} did you believe there was a step there?' is quite inappropriate}.]{En innumerables casos, creemos algo que se nos ha dicho. ¿Cuándo? Bueno, cuando se nos ha dicho. Esto de nuevo hace que parezca que `He creído esto' es un informe de un acto que ocurrió en un momento dado. Sin embargo \textelp{una} pregunta sobre la duración mostraría que no es así: pues la duración de la creencia no es la duración de ninguna acción. \textelp{Si tuviera un traspié porque me equivocara en creer que tenía un escalón delante, por ejemplo,} la pregunta `¿\emph{Durante cuánto tiempo} creíste que había un escalón ahí?' Sería completamente inapropiada}. En este sentido `creer a alguien' no se refiere a una acción en el tiempo, sino a una disposición o estado. Sin embargo, el creer puede venir acompañado o iniciado por un acto: \blockquote[{\Cite[155]{anscombe2015logic:bt}}: \enquote{When I do suddenly believe something; or believe it when I am told it, my belief is not an act; but does it perhaps \emph{begin} with an act? \textelp{} Here one is inclined to postulate an inner assent, or act of acceptance}.]{Cuando creemos algo repentinamente; o cuando creemos lo que se nos ha dicho, nuestro creer no es un acto; pero, ¿quizás sí \emph{empieza} con un acto? \textelp{} Aquí podemos estar inclinados a postular un asentimiento interior, o acto de aceptación}. En esto, Elizabeth ve una especie de paralelismo con la intención; así como la intención puede comenzar con una decisión, el creer puede iniciarse con un acto de asentimiento. De este modo, aunque cuando decimos que `creemos a alguien' parece que `creer' consiste en el `episodio' de una actividad, la acción denominada es más bien la del asentimiento que está vinculado al inicio de la creencia: \blockquote[{\Cite[157]{anscombe2015logic:bt}}: \enquote{There is however no such thing as an act of belief; in the `episodic' case, the act is that of assent, conjoined to a thought which is either actively produced or passively received into the mind}.]{En cualquier caso, no hay tal cosa como un acto de creer; en un caso `episódico', el acto es de asentimiento, unido a un pensamiento que es producido activamente o pasivamente recibido en la mente}. Podemos concluir, junto con Anscombe, con una noción del proceso al que el asentimiento se refiere: \blockquote[{\Cite[157]{anscombe2015logic:bt}}: \enquote{Assent from one person to a proposition formulated by another gives us the picture of two procedures: the formulation of something assertible ---what Frege calls a `judgeable content'--- and the assent to, or inward assertion of that content. When someone thinks within himself that such-and-such is the case, he has inwardly done both things}.]{El asentimiento de parte de una persona ante una proposición formulada por otra nos da la representación de dos procesos: la formulación de algo que puede ser aseverado ---lo que Frege llama `contenido juzgable'--- y el asentimiento a, o aserción interna, de este contenido. Cuando alguien piensa dentro de si que algo es el caso, ha realizado interiormente ambas cosas}.

Con todas estas consideraciones hemos querido componer una descripción de la estructura del creer en el testimonio como la creencia que se tiene cuando creemos a alguien que nos comunica algo. Para completar esta descripción ahora tendremos en cuenta otro aspecto de los fundamentos que tienen estas creencias que sostenemos apoyados en lo que se nos ha dicho.

\subsubsection{`Conocimiento Tradicional'}

\label{subsec:testtrad}
En la discusión de \emph{Hume and Julius Caesar} Anscombe estableció una cuestión que juzgó de gran importancia: ¿por qué las cosas que se nos dicen y los escritos que vemos son el punto de partida para creer en eventos distantes y en la linea de transmisión de estos eventos? Ella recurre al pensamiento tardío de Wittgenstein, específicamente el que se encuentra en \emph{Sobre la Certeza}, para dar respuesta; sin embargo, el planteamiento de la cuestión se encuentra en Hume. El tema central de lo que Anscombe trabaja aquí es el fundamento último que justifica las creencias que nosotros usamos como premisas en nuestros argumentos. Su respuesta final es que estos se apoyan sobre lo que podemos llamar `conocimiento tradicional'. Nuestra pregunta respecto de esto es ¿qué relación guarda este tipo de fundamento con el saber por testimonio?
%Anscombe plantea la cuestión  como sigue: \blockquote[{\Cite[121-122]{anscombe2011plato:humecaus}}: \enquote{To my mind the interest of Hume lies primarily in the problems he consciously or unconsciously discovers to us. Here there is a problem unconsciously raised. For Hume judges that we believe Caesar was killed in the Senate House from the testimony of historians. (Is that \emph{testimony?}) And he thinks that this belief is explained as our reasoning from our perception of `certain characters and letters', through succesive steps referring to intermediate records, back to the perception of eyewitnesses and through that to the event. He supposes that the record before our eyes is our reason for believing in the intermediate records, which are in turn our reason for believing in the original event. He must suppose this, otherwise it would not be possible for him, however confusedly, to cite the chain of record back to the eyewitnesses as an illustration of the chain of causes and effects with which we cannot run up \emph{in infinitum}, but must eventually bring to an end with our present perception or memory of written documents.}]{A mi entender, el interés en Hume radica primordialmente en los problemas que él nos descubre inconsciente o conscientemente. Aquí hay un problema establecido inconscientemente. Pues Hume juzga que creemos que César fue asesinado en el Senado apoyados en el testimonio de los historiadores. (¿Eso es \emph{testimonio}?) Y piensa que esta creencia queda explicada como un razonamiento nuestro desde la percepción de `ciertos caracteres y letras', a través de pasos sucesivos de referencia en informes intermediarios, hasta llegar de vuelta a la percepción de testigos presenciales y, a través de esta, al evento mismo. El presupone que el informe ante nuestros ojos es nuestra razón para creer en los informes intermediarios, que son, a su vez, nuestra razón para creer en el evento original. Tiene que suponer esto, de otro modo no sería posible para él, aún de manera confusa, citar la cadena de informes de vuelta a los testigos presenciales como una ilustración de la cadena de causas y efectos que no puede recorrerse \emph{in infinitum}, sino que tiene que llegar a un final con nuestra percepción o memoria presente de los documentos escritos}.

Como vimos (\S\ref{subsec:notchain}, p.~\pageref{subsec:notchain}), Hume describió la razón por la que podemos sostener la creencia en el asesinato del César como una serie de inferencias desde nuestra percepción hasta la de los testigos del hecho.
Y aquí con Anscombe podríamos cuestionar, \enquote{¿el fundamento de nuestra creencia es una cadena de testimonios que conecta nuestra percepción presente con la percepción de los testigos del hecho?} Ella respondió que no. Si tomamos los informes que vemos en el presente como fundamento de nuestra creencia, estos son fundamentos para creer en el hecho que narran y la creencia en el hecho es entonces fundamento para creer en la transmisión intermedia. Pero esto no termina de explicar el fundamento de una creencia como esta. Ella añade: \blockquote[{\Cite[182]{anscombe2015logic:grounds}}: \enquote{Grounds, we think, are premises for arguments. But who argues from the characters and letter in texts that he may produce that Julius Caesar existed in ancient Rome and was killed? That it was so, and that these texts, for example, go back so-and-so far, is a piece of traditional knowledge which we acquire by being told it together with many other facts belonging to the general sketch of history}.]{Los fundamentos, pensamos, son premisas de argumentos. Sin embargo, ¿quién argumenta desde los caracteres y letras presentes en los textos que podemos producir la noción de que Julio Cesar existió en Roma y fue asesinado? Que esto ocurrió, y que estos textos, por ejemplo, tienen este alcance hacia el pasado, es un pedazo de conocimiento tradicional que adquirimos porque se nos ha dicho junto con muchos otros datos correspondientes a la imagen general de la historia}. Hay dos aspectos importantes en este ejemplo del dato histórico de Julio Cesar: su existencia no es una teoría que pretenda explicar ningún fenómeno, y en cuanto que dato histórico forma parte de la infraestructura del conocimiento común de nuestra cultura. En un caso como este, la pregunta \enquote*{¿por qué creemos esto?} se responde diciendo \enquote*{porque nos lo han enseñado}. Y, ¿no pueden haber errores en estas enseñanzas? Ciertamente, pero cuando se rechaza una enseñanza como esta como no verdadera ¿por qué lo hacemos?: \blockquote[{\Cite[182]{anscombe2015logic:grounds}}: \enquote{Because it is incompatible with what else we have in our picture. That means: we take other things as fixed points by which we judge this ostensible record. Why do we accept them?\,---\,They are `traditional knowledge' and they hang together}.]{Porque es incompatible con todo lo demás que tenemos en nuestra imagen. Esto significa: tomamos otras cosas como puntos fijos desde los que juzgamos lo que aparece como una información. ¿Por qué los aceptamos?\,---\,Son `conocimiento tradicional' y se apoyan mutuamente}.

Ahora bien, hay más ejemplos de este tipo de conocimiento común que forma parte de la infraestructura o sistema de nuestra tradición. Otro que Elizabeth usa es el conocimiento del lugar donde habitamos. Sobre esto dice: \blockquote[{\Cite[187-188]{anscombe2015logic:grounds}}: \enquote{My knowledge of the things among which and the places in which I live is not so much `theory laden' as `common-knowledge laden'. I wish to say: it is a falsification here to speak of testimony: to say, for example, that it is by testimony that I know I was born. There is something else, not testimony, though acquired by education from human beings, which is, so to speak, \emph{thicker} than testimony}.]{Mi conocimiento de las cosas entre las cuales y los lugares en los que vivo no está `repleto de teoría' sino `repleto de conocimiento común'. Lo que quiero decir es: es una falsificación aquí hablar de testimonio: decir, por ejemplo, que es por testimonio que sé que he nacido. Hay algo más, no testimonio, aunque recibido por la enseñanza de otros seres humanos, que es, por así decirlo, \emph{más denso} que el testimonio}. Poco a poco se puede ver qué tipo de distinción Anscombe está haciendo entre testimonio y conocimiento tradicional. Es importante no perder de vista que la discusión trata del fundamento de ciertas creencias que forman parte de nuestro sistema de conocimiento o imagen del mundo. Aquí estamos de lleno en el terreno de \emph{Sobre la certeza}. Elizabeth ofrece una descripción que nos puede ayudar a completar esta noción: \blockquote[{\Cite[189]{anscombe2015logic:grounds}}: \enquote{The work of determining England and fixing the meaning of the name \emph{would} depend on testimony\,---\,the testimony of many different people for different parts of it. The work done, people could be taught what Engalnd was (no doubt still disputing some regions). Now those who learned thereafter can hardly be said to have knowledge by testimony. They were taught to \emph{call} something `England'\,---\,something indeed which could in large part only be defined for them by hearsay; and they so taught those who came after them. I am an heir of this tradition. Now, I know I live in England. But by testimony? Some would say so. But there is something queer about it. \emph{What} do I know? That the world is divided up into countries which have names, and that the one I live in is called England and is here on the map of the globe. This involves understanding the use of the globe to represent the earth. It is rather as if I had been taught to join in \emph{doing} something, than to believe something\,---\,but because everyone is taught to do such things, an object of belief is generated. The belief is so certainly correct (for it follows the practice) that it is knowledge; for here knowledge is no other that certainly correct belief in pursuit of a practice. But the connection with testimony is remote and indirect}.]{El trabajo de determinar Inglaterra y fijar el significado del nombre \emph{sí} dependería en el testimonio\,---\,el testimonio de muchas personas de diferentes partes de ella. Realizada la obra, a la gente podría enseñársele qué es Inglaterra (sin duda debatiendo sobre algunas regiones). Ahora, esos que aprendieron a partir de ese momento difícilmente podría decirse que tienen conocimiento por testimonio. Se les enseñó a \emph{llamar} algo `Inglaterra'\,---\,ciertamente algo que en gran parte solo podría quedar definido para ellos por referencia de otros; y así estos lo enseñarían a los que vinieron después. Yo soy heredera de esta tradición. Así, yo sé que vivo en Inglaterra. Pero, ¿por testimonio? Algunos lo dirían. Pero hay algo extraño sobre eso. ¿\emph{Qué} es lo que sé? Que el mundo está dividido en países que tienen nombres, y que ese en el que yo vivo se llama Inglaterra y está aquí en este lugar del mapa del globo. Esto involucra la comprensión del uso de un globo para representar la tierra. Es más bien como si se me hubiera enseñado a unirme en \emph{hacer} algo, más que a creer algo\,---\,pero como a todos se les enseña a hacer este tipo de cosas, queda generado un objeto de creencia. La creencia es tan ciertamente correcta (pues sigue la práctica) que constituye conocimiento; pues aquí conocer no es otra cosa que la creencia ciertamente correcta en la consecución de una práctica. Pero la conexión con el testimonio es remota e indirecta}.

Las consideraciones de Elizabeth van componiendo una imagen en la que testimonio y tradición interactúan pero no se identifican. Para entenderla puede ser útil recurrir a la distinción entre testigo y maestro. El testigo es una autoridad original porque contribuye algo, aún cuando su testimonio pueda estar informado o compuesto por testimonios y enseñanzas que él haya recibido. Podríamos considerarlo como autoridad absolutamente original cuando no depende de información recibida. Al maestro se le cree esperando que enseñe la verdad, pero no como un mero mensajero, sino respaldado por la autoridad que tiene una tradición y sistema de enseñanza del que es portavoz.

Es posible ver cómo una tradición se construye con la aportación de los testigos y cómo el testimonio se nutre de lo que la tradición comunica. En ese sentido podemos decir que la creencia del testimonio de alguien que es una autoridad original es un caso de `creer a alguien' y eso describe un aspecto del tipo de fundamento que justifica esta creencia; adicionalmente, cuando el testigo se apoya en una tradición para su testimonio, la creencia en su comunicación no queda justificada por una cadena de testimonios, sino por el sistema de conocimiento tradicional del que forma parte.
