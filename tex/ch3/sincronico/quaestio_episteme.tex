\subsection{¿Cuál es el valor epistemológico del testimonio?}

\emph{Creer y Conocer}, de Francisco Conesa, es una investigación sobre el valor cognoscitivo de la fe en la filosofía analítica. En su estudio, Conesa sitúa a Anscombe entre los autores que \blockquote[{\cite[84]{conesa1994cc}}]{entienden la fe primordialmente como un saber por testimonio.} El análisis que el autor ofrece como fundamento para este modo de entender la perspectiva de Anscombe se enfoca en dos puntos. El primero es que para Anscombe el significado de la palabra `fe' es `creer a Dios'. Conesa resume este punto refiriéndose a la discusión del artículo \emph{Faith}: \blockquote[{\cite[87--88]{conesa1994cc}}]{<<En la tradición donde el concepto tiene su origen, \emph{fe} es una abreviación de \emph{fe divina} y significa \emph{creer a Dios}>>. Y ¿qué puede significar \emph{creer a Dios}? Todos los casos de <<creer a ``$x$''>> suponen que ``$x$'' habla. Que alguien tiene fe quiere decir que cree que algo es palabra de Dios: <<fe es la creencia que él presta a esa palabra>>.} El segundo tema que compone la explicación de Conesa es del artículo \emph{Hume and Julius Caesar}: \blockquote[{\cite[88]{conesa1994cc}}]{Creer en el testimonio es muy distinto de creer en causas y efectos. Este punto es desarrollado por la filósofa al estudiar el conocimiento histórico: <<Creer en un relato histórico es absolutamente creer que ha habido una cadena de tradición de relatos y documentos que llega hasta el conocimiento contemporáneo; no es creer en los hechos históricos mediante una inferencia que vaya siguiendo cada nudo de esa cadena>>.}

Desde el punto de vista de Conesa, este modo de entender `fe', como `saber por testimonio', sirve para caracterizar el valor cognoscitivo que tienen las creencias que se sostienen sobre el fundamento de la fe. Su propuesta es que: \blockquote[{\cite[88]{conesa1994cc}}]{Desde esta perspectiva comprendemos el valor epistemológico de la fe religiosa, que consiste en \emph{creer a Dios}. Ella forma parte de ese conocimiento que depende del testimonio de otros. En este caso, además, creemos a alguien que conoce. Entonces es claro que accedemos a su conocimiento.} Aquí el autor afirma que el valor epistemológico que tiene la fe es el del `saber por testimonio', y en pocas palabras describe el valor epistemológico de este saber como el conocimiento al que accedemos cuando creemos a alguien que conoce, en este caso a Dios. En este apartado veremos con más detalle cómo Anscombe describe el valor epistemológico de estas creencias que sostenemos por el testimonio que hemos recibido. Las dos cuestiones que Conesa tiene en cuenta al valorar el pensamiento de Elizabeth nos servirán como marco de referencia para esta discusión.

\subsubsection{La `estructura' de creer en el testimonio}

Mary Geach, hija de Anscombe, cuenta una anécdota que le escuchó a su madre; cuando Elizabeth estaba en sus estudios universitarios se topó con un pasaje de Russell en su comentario de Leibniz que sostenía que un argumento construido desde los datos del mundo no sería válido para afirmar la existencia de Dios, pues no es posible deducir una conclusión necesaria desde una premisa contingente. En ese momento Anscombe no sabía qué hay de equivocado en la noción de que las necesidades solamente pueden ser deducidas de premisas necesarias, sin embargo, sí sabía que el negar la posibilidad de conocer de la existencia de Dios por medio de las cosas creadas a la luz de la razón era negar una doctrina de fe definida por la enseñanza de la Iglesia. Deicidió, entonces, ir a una iglesia y hacer un acto de fe. Más tarde en su carrera filosófica llegó a ver cómo argumentar que pueden deducirse conclusiones necesarias de premisas contingentes, pero en aquel momento su acto de fe le evitó caer en un error. Mary destaca un aspecto interesante de esta anécdota: \blockquote[{\cite[xvi--xvii]{anscombe2008faith}}: \enquote{Faith, \textelp{} is believing God, but this story shows how public she believed the voice of God could be, speaking as it has done in the teaching of the Church.}]{La fe, \textelp{} es creer a Dios, y esta historia muestra cuán pública ella creía que la voz de Dios puede ser, hablando como lo hace en la enseñanza de la Iglesia.} Es difícil entender bien el modo en que Elizabeth habla de la fe si no se tiene en cuenta esta creencia suya. Anscombe habla de Dios como uno que está involucrado en la actividad humana del lenguaje, tiene una `voz pública'. Mary Geach añade a su reflexión de la anécdota que \blockquote[{\cite[xvii]{anscombe2008faith}}: \enquote{philosophers nowadays accept on authority much that they do not themselves have the expertise to know firsthand, and they do not see it as a limitation on their freedom}]{hoy en día los filósofos aceptan mucho que ellos mismos no tienen la capacidad para cononcer de primera mano, y esto no lo ven como una limitación de su libertad}. Aceptamos creencias apoyados en la autoridad de peritos y, si están en lo correcto, esta aceptación no implica una limitación de nuestra libertad. Lo mismo se puede considerar respecto de la enseñanza de la Iglesia: \blockquote[{\cite[xvi--xvii]{anscombe2008faith}}: \enquote{To proceed on the assumption that this teaching is true is seen by some as a limitation on one's freedom, but this is only the case if the Church does not have the teaching authority she claims to have.}]{Proceder con el presupuesto de que esta enseñanza es verdadera es visto por algunos como una limitación a nuestra libertad, pero esto solo es el caso si la Iglesia no tiene la autoridad para enseñar que declara tener.}

El nexo entre la discusión tratada en los artículos \emph{What is it to believe someone?} y \emph{Faith} es ese dato, `fe' como la creencia depositada en lo que se nos comunica ---apoyados, entre otras cosas, en la autoridad del que comunica--- y estas creencias como componentes `no desprendibles' de nuestro conocimiento de la realidad más allá de nuestra experiencia personal. Anscombe parte de la descripción de Hume: la justificación para que sea razonable creer el testimonio consiste en la inferencia que hacemos de que al testimonio se sigue la verdad como se siguen los efectos de las causas. Tras expresar su desacuerdo, ella propone en cambio que hemos de reconocer al testimonio como un medio que nos da acceso a una visión más amplia del mundo del mismo modo, o incluso en mayor grado que la relación causa y efecto. A esto añade que \enquote*{creerlo es muy distinto en estructura que la creencia en causas y efectos}. Este comentario sugiere la pregunta: ¿en qué consiste, desde su perspectiva, la `estructura' de la creencia en el testimonio?

Podemos situar la respuesta de Anscombe en torno a dos argumentaciones principales. La primera se encuentra en la descripción que ella hace de lo que significa creer a alguien. Su propuesta es que una persona está en la situación de atender la pregunta acerca de creer o dudar (suspender el juicio ante) alguien cuando están dadas toda una serie de presuposiciones; entonces, libre de confusiones por las preguntas que podrían surgir relacionadas con estos presupuestos, creer a alguien acerca de algo en particular es confiar en esa persona sobre la verdad de ese asunto en particular. La segunda argumentación relacionada con esto está en su discusión sobre el conocimiento histórico. En efecto, como piensa Hume, el hecho de que tenemos creencias justificadas sobre fundamentos que se consideran premisas de argumentos, presupone que hay creencias sin fundamento, o al menos, que no tienen como fundamento algo que pueda considerarse como premisa de un argumento. Para Hume estos fundamentos últimos son las impresiones de nuestros sentidos. Anscombe no piensa así. Se pregunta: ¿por qué las cosas que se nos dicen y los escritos que vemos \emph{son} el punto de partida para nuestro creer en eventos distantes y también en la cadena de transmisión de esta información?, ¿por qué creemos los testimonios e informes que recibimos de estos hechos? Su respuesta es que los fundamentos últimos de estas creencias se encuentran en el conocimiento tradicional o común, aquellas creencias de las cuales diríamos \enquote*{¡Todo el mundo sabe eso!}. Examinaremos ambas cuestiones más detenidamente.

\subsubsection{¿Qúe es creer a alguien?}

Se podría decir que la actitud que caracteriza la reflexión de Asncombe sobre el creer obedece a la consigna Wittgensteniana: \enquote*{te enseñaré diferencias}. A lo largo de su discusión se encuentran diversas distinciones y matizaciones sobre el modo en que empleamos la expresión `creer' cuando decimos que creemos algo que alguien nos ha dicho y también cómo actuamos según ese tipo de creencias.

Una de las primeras distinciones que Anscombe enfatiza en su investigación en \emph{What is it to Believe Someone?} es acerca de los fundamentos de nuestra creencia al recibir un testimonio. Creer a alguien no consiste simplemente en creer lo que alguien me dice o tenerlo por verdadero. El pequeño relato que encabeza el ensayo le sirve para ilustrar esta distinción. El diálogo está construido según una conjunción de premisas que en otro artículo ella llama un `extraño patrón de argumento'.\footnote{\cite[Cf.~][299]{anscombe2015logic:qpa}: \enquote{The pattern to which my title refers is: $1^{o}$ If $p$, then $q$. $2^{o}$ If $r$, then not (if $p$ then $q$). $3^{o}$ If not $p$ then $r$. $\mathbf{\therefore}$ $p$ and $q$. We get `not $r$' from the first two premises and then `$p$' from `not $r$' and the third; with the first one again this gives us the conclusion.}} La característica peculiar de este patrón es que es formalmente válido y sus premisas compatibles, pero las premisas dadas no sirven para fundamentar la creencia en la conclusión. El escenario que Anscombe usa como ejemplo culmina con la expresión de Eutidemo: \enquote*{Les creo a todos. Así que infiero que el árbol caerá y el camino quedará obstruido}; entonces Elizabeth propone: \enquote*{¿Qué equivocación tiene Eutidemo?}. La pregunta clave que nos está invitando a considerar ante la inferencia de Eutidemo es: \enquote*{¿cuál es el fundamento real para creer la conclusión?}. Ella explica que: \blockquote[{\cite[301]{anscombe2015logic:qpa}}: \enquote{The peculiarity of our case is that there doesn't seem to be any difficulty about reasonably judging any of the three premises to be true without having already judged the conclusion or part of it to be true. The difficulty lies in combining them in knowledge, or in a reasonable judgement, unless part of the conclusion is part of the ground for accepting the combination. One wants to say: that you can get this conclusion out of these three propositions is ground for doubting the conjunction of them! But the reason is not that the conclusion is itself false, let alone absurd. It is a perfectly possible proposition, and is objected to only as a conclusion from perfectly possible propositions, which are mutually compatible and from which it does follow.}]{La peculiaridad de este caso es que no parece haber ninguna dificultad para juzgar razonablemente cualquiera de las tres premisas como verdadera sin haber juzgado de antemano la conclusión o parte de ella como verdadera. La dificultad se encuentra al combinarlas como un conocimiento, o un juicio razonable, a no ser que parte de la conclusión sea parte del fundamento para aceptar la combinación. Quisieramos decir: ¡que podamos llegar a esta conclusión desde estas tres proposiciones es fundamento para dudar de la conjunción de estas! Pero la razón no es que la conclusión misma sea falsa, ni mucho menos absurda. Es una proposición perfectamente posible, y es objetada solo como la conclusión de proposiciones perfectamente posibles, que son mutuamente compatibles y desde las que sí se sigue.}

A esta característica Anscombe la llama `retractabilidad esencial'. Con esto quiere decir que un juicio como el que la conclusión de este argumento expresa, aunque se sigue de la conjunción de sus premisas, es retractable por algún elemento o circunstancia externa que haga irrazonable deducir válidamente la conclusión desde la conjunción de estas premisas.\footnote{\cite[Cf.~][299]{anscombe2015logic:qpa}: \enquote{Then we have perhaps discovered the special character of (theoretical) hypotheticals whose consequents don't follow logically from their antecedents. We might call this character `essential defeasibility'. This will be the reason why, even though `not $r$' follows from `if $p$ then $q$ and if $r$, then not (if $p$ then $q$)', still it may be highly unreasonable to deduce `not $r$' from that conjunction.}} ¿Cuál sería el elemento externo que sirve como fundamento para la validez de la creencia en una conclusión en el caso de creer a alguien? Anscombe responde \enquote*{Para creer a $N$ debemos creer que $N$ mismo cree lo que dice}. En el ejemplo de Elizabeth, la inferencia de Eutidemo expresa un juicio basado en la conjunción de las premisas, él podría decir: \enquote*{es razonable juzgar que el árbol caerá e interrumpirá el paso pues esta conclusión se sigue de la conjunción de afirmaciones hechas por $A$, $B$ y $C$}.\footnote{Es pertinente recordar aquí que para Anscombe una inferencia valida como conclusión lógica tiene que ser juzgada dentro de la actividad humana: \cite[121]{anscombe1981parmenides:qli} \enquote{Valid inference, not logical truths, is the subject matter of logic; and a conclusion is justified, not by rules of logic but, in some cases by the truth of its premisses, in some by the steps taken in reaching it, such as making a supposition or drawing a diagram or constructing a table.}} Ahora bien, al justificar esta inferencia diciendo \enquote{les creo a todos}, suena como un loco, pues no ha juzgado si $A$ cree lo que ha dicho después de haber escuchado a $B$ y $C$. Esta afirmando un juicio que no puede quedar justificado por la conjunción de las premisas y que, según su propia expresión, solo puede tener como fundamento real la creencia de que los tres personajes creen lo que están diciendo. Al no tener en cuenta qué creen $A$, $B$ y $C$, su inferencia queda sin fundamento válido.

La manera en que Anscombe establece esta distinción parece extraña, sin embargo es útil, puesto que sirve para describir con mayor claridad la disposición que alguien tiene cuando cree un testimonio. Elizabeth añade, que hay un gran número de juicios que siguen este tipo de patrón,\footnote{\cite[Cf.~][302]{anscombe2015logic:qpa}: \enquote{There are large numbers of hypothetical judgements that are like this. It is an interesting and important observation that there is a whole class of judgements such that when we make them we are not implicitly dismissing as false everything that would falsify them. In contrast, when I make a categorical statement with appropiate confidence, it is very often the case that I can straightway rule out as false what would falsify it\,---\,just because I know that \emph{it} is true.}} incluso, su peculiar carácter no solo se encuentra relacionado con la dinámica de creer a alguien en el sentido de `fe humana', sino que también se le puede encontrar en el `creer a Dios'.\footnote{Otro de los ejemplos de argumento que siguen el patrón que Anscombe discute en el artículo \emph{On a Queer Pattern of Argument} es un razonamiento hipotético de Isaac al conocer que él era el sacrificio a ser ofrecido por Abrahám, el argumento, dice: \cite[Cf.~][309]{anscombe2015logic:qpa}: \enquote{might be produced by a less evasive and tortous Johannes de Silentio picturing Isaac in the interval in which he has realised that \emph{he} is the intended sacrifice, and before Abraham's hand is stayed. Isaac reasons: $1''''$  If God has promised my father that he will be the father of a great nation through me, then my father will be. $2''''$  If my father kills me, it's not true that if God has promised him he will be the father of a great nation through me, then he will be. (\emph{Therefore he is not going to kill me}.) $3''''$  If God has not promised my father that he will be the father of a great nation through me, my father is going to kill me. $\mathbf{\therefore}$  God has promised that to my father and it will be fulfilled. This argument differs from all the other in that in the first proposition the consequent necessarily follows from the antecedent.}} Otros ejemplos que Elizabeth usa para insistir en que al creer a alguien, la disposición que la palabra `creer' expresa es la intención de tener por verdadero que \enquote*{$N$ creer lo que me dice} son: `creer' con un objeto personal no puede ser reflexivo, es decir, podemos `decirnos algo' a nosotros mismos, pero no podemos decir que `nos creemos a nosostros mismos' sobre algo; también sugiere que decir a alguien \enquote*{te creo} cuando la información es algo de conocimiento común (p. ej. Napoleón perdió la batalla de \emph{Waterloo}), la declaración suena a chiste; también sonaría a chiste decir que creo a alguien en el caso de que crea lo que me diga, pero porque estoy convencido de que me miente y además está equivocado en lo que cree y por ese cálculo creo lo que me dice porque me lo ha dicho, pero no le creo fiable.

Otra distinción que Elizabeth establece es entre las `presuposiciones', que son las creencias adicionales involucradas en creer a alguien, y aquello que se cree porque se cree a alguien, es decir, el contenido de la comunicación. Esta distinción juega un papel importante en su descripción de lo que es `fe' en el artículo \emph{Faith}. Allí recordaba que el carácter de racionalidad que se le atribuía a las creencias de la fe había sido justificado en una época sobre los llamados `preámbulos' de la fe y el paso de estos a la fe misma, sin embargo, ella propone que la designación adecuada para al menos parte de estos es más bien `presuposiciones'. En este sentido, estas otras creencias involucradas en el `creer a alguien que $p$', o `creer a Dios que $p$' tienen que ver con el carácter de racionalidad que puede atribuírsele a la fe. Anscombe añade que en sentido estricto las presuposiciones no forman parte del contenido de lo que se cree por la fe. Esto lo afirma en el ejemplo de la carta de Jones, o de la carta que recibe el prisionero, creer que la carta viene de Jones no es una decisión que se toma teniendo como garantía la credibilidad de Jones, lo mismo ocurre con creer que $N$, el que envía la carta al prisionero, existe, la creencia en su existencia y la creencia en el contenido de la carta son lógicamente diferentes.

El modo en el que describe estas presuposiciones las pone en relación con la comunicación


 This is the sense in which the presuppostions are not themselves part of ....

this brings out the difference between presuppostions of believing N and believing such....



-> otra distinción es la de las presuposiciones, para esto usa los ejemplos de una comunicación escrita

-> negación: iwt 19 `not', which is so simple to use, is utterly mystifying to think about; no theory of thought or judgment which does not give an account of it can hope to be adequate.
trae el tema de la negación: al final truth over falsehood to talk about....

está la propouesta de que

Anscombe nos ofrece un
what is it to believe.... presuppostions

defeseability theory

original authority

hume and.... believe in the chain because we believe in the event


Anscombe dice que creer en el testimonio es un creer bastante distinto en estructura
que creer en causas y efectos.

Parece que habla de esto en hume and julius caesar y en grounds for belief

puede decirse lo siguiente?

la estructura de creer en el testimonio es la estructura de creer en alguien
la estructura de creer en alguien es

dados los presupuestos
A Nota la comunicación
A Toma la comunicación como lenguaje
A Toma la comunicación como dirigda a él
A Interpreta la comunicación correctamente
A Cree que viene de NN

confiar en NN acerca de la verdad de x cuando una comunicación de NN llega a A por medio de un productor inmediato.

La estructura de creer en el testimonio de alguien
si entendemos creer en el testimonio de alguien como
creer a x que p

es la estructura de la fe tambien

cual es esa estructura?
dados ciertos presupuestos
x confia en NN acerca de la verdad de p

podemos sacar una descripción de
la categoría del testimonio

de las interconexiones que Anscombe describe
en el "arco" de la verdad, el sentido y la aserción
enunciar y significar son distintos
la rectitud propia de lo que la verdad es aplica tanto a la persona que enuncia como al enunciado

la persona puede mentir
el enunciado falso cuando es creido significa algo pero no enuncia nada.

la paradoja, distinto de el enunciado falso no significa nada.

el enunciado verdadero hace rectamente aquello para lo que se creó
la persona que dice una proposición verdadera actua rectamente

creer a alguien que dice una proposición verdadera es reconocer la rectitud de la aserción y reconocer la rectitud de la persona que habla

hay, por tanto un modo de conocer la verdad que se puede describir como

dados los presupuestos
confiar en NN acerca de la verdad de una proposición
cuando la proposición es verdadera tiene rectitud perceptible a la mente
NN actua con rectitud
cuando la proposición es falsa aunque signifique algo no dice nada
cuando la proposición es una paradoja no significa ni comunica nada

la rectitud es perceptible a la mente sin tener que acudir a la experiencia



En un argumento que tiene caracter de revocabilidad esencial la razonabilidad de un juicio o conocimiento formado a partir de éste reclama un apoyo externo a sus premisas. En el caso del creer a alguien que p el que alguien crea lo que dice es este apoyo

Anscombe habla del testigo como una autoridad original, `que él mismo contribuye algo' en oposición a simplemente transmitir una información recibida sin embargo se puede decir que se cree a x que p sin que este sea una autoridad original

Se abren tres rutas desde aquí sobre la primacia de la verdad la fe como creer a Dios que p sobre la estructura del testimonio de la estructura del testimonio se pasa a Hume on Miracles, de ahí a Prophecy and Miracles

\subsubsection{`Conocimiento Tradicional'}

La creencia en el testimonio no está justificada en otro testimonio como una cadena... sino que tiene su fundamento en otras creencias que forman parte del conocimiento tradicional.

\blockquote[{\cite[121--122]{anscombe2015logic:grounds}}: \enquote{Grounds, we think, are premises for arguments. But who argues from the characters and letter in texts that he may produce that Julius Caesar existed in ancient Rome and was killed? That it was so, and that these texts, for example, go back so-and-so far, is a piece of traditional knowledge which we acquire by being told it tofether with many other facts belonging to the general sketch of history.}]{}

Why do we believe these things? there is nothing to say but we were taught to do so

are there never mistakes? certainly

why is something rejected as untrue? because it is incompatible with what else we have in our picture. we take other things as fixed points by which we judge this ostensible record

why do we accept them? they are traditional knowledge and the jang together

belief on grounds which can be considered as premises for arguments presupposes belief without grounds, or at any rate without grounds that can be so considered



\blockquote[{\cite[121--122]{anscombe2011plato:humecaus}}: \enquote{To my mind the interest of Hume lies primarily in the problems he consciously or unconsciously discovers to us. Here there is a problem unconsciously raised. For Hume judges that we believe Caesar was killed in the Senate House from the testimony of historians. (Is that \emph{testimony?}) And he thinks that this belief is explained as our reasoning from our perception of `certain characters and letters', through succesive steps referring to intermediate records, back to the perception of eyewitnesses and through that to the event. He supposes that the record before our eyes is our reason for believing in the intermediate records, which are in turn our reason for believing in the original event. He must suppose this, otherwise it would not be possible for him, however confusedly, to cite the chain of record back to the eyewitnesses as an illustration of the chain of causes and effects with which we cannot run up \emph{in infinitum}, but must eventually bring to an end with our present perception or memory of written documents.}]{A mi entender, el interés en Hume radica primordialmente en los problemas que él nos descubre inconsciente o conscientemente. Aquí hay un problema establecido inconscientemente. Pues Hume juzga que creemos que César fue asesinado en el Senado apoyados en el testimonio de los historiadores. (¿Eso es \emph{testimonio}?) Y piensa que esta creencia queda explicada como un razonamiento nuestro desde la percepción de `ciertos caracteres y letras', a través de pasos sucesivos de referencia en informes intermediarios, hasta llegar de vuelta a la percepción de testigos presenciales y, a través de esta, al evento mismo. El presupone que el informe ante nuestros ojos es nuestra razón para creer en los informes intermediarios, que son, a su vez, nuestra razón para creer en el evento original. Tiene que suponer esto, de otro modo no sería posible para él, aún de manera confusa, citar la cadena de informes de vuelta a los testigos presenciales como una ilustración de la cadena de causas y efectos que no puede recorrerse \emph{in infinitum}, sino que tiene que llegar a un final con nuestra percepción o memoria presente de los documentos escritos.}

\blockquote[{\cite[121--122]{anscombe2011plato:humecaus}}: \enquote{But it is not like this at all. If the written records that we see are our grounds for belief, they are first and foremost grounds for belief in the original event, and then our belief in the original event is a ground for belief in the intermediate transmission.}]{Pero no es así para nada. Si los informes escritos que vemos son los fundamentos para creer, son primero y ante todo fundamentos para creer en el evento original, y entonces nuestro creer en el evento original es fundamento para creer en la transmisión intermedia.}

Esto lo compara con la creencia en la continuidad espacio-temporal de una persona que identificamos ahora como alguien que vimos la semana pasada. Se cree en que esta es la misma persona, no porque observamos la continuidad de un patrón humano desde la semana pasada a la persona que vemos ahora. Más bien creemos en la continuidad espacio-temporal porque creemos que esta persona es la misma que la de la semana pasada.

\blockquote[{\cite[121--122]{anscombe2011plato:humecaus}}: \enquote{Our belief in recorded history is a belief \emph{that there has been} a chain of tradition of reports and records: it is not a belief in the historical facts \emph{through} the links of such a chain. At most, that can \emph{very seldom} be the case.}]{Nuestro creer en la historia registrada es una creer \emph{que ha habido} una cadena de tradición de informes y registros: no es creer en los hechos históricos \emph{por medio} de los eslabones de una cadena de ese tipo. Como mucho, eso podría \emph{muy raramente} ser el caso.}

\blockquote[{\cite[121--122]{anscombe2011plato:humecaus}}: \enquote{The interesting problem that arises, then, is why the things we are told and the wirtings that we see \emph{are} the starting points for our belief in the far distant events and so in the intermediate chain of record. This is a question of vast importance.}]{El problema interesante que surge, entonces, es por qué las cosas que se nos dicen y los escritos que vemos \emph{son} el punto de partida para nuestro creer en los eventos distantes y así también en la cadena intermedia de informes. Esta es una pregunta de amplia importancia.}




  The work of determining England and fixing the meaning of the name \emph{would} depend on testimony\,---\,the testimony of many different people for different parts of it. The work done, people could be taught what Engalnd was (no doubt still disputing some regions). Now those who learned thereafter can hardly be said to have knowledge by testimony. They were taught to \emph{call} something `England'\,---\,something indeed which could in large part only be defined for them by hearsay; and they so taught those who came after them. I am an heir of this tradition. Now, I know I live in England. But by testimony? Some would say so. But there is something queer about it. \emph{What} do I know? That the world is divided up into countries which have names, and that the one I live in is called England and is here on the map of the globe. This involves understanding the use of the globe to represent the earth. It is rather as if I had been taught to join in \emph{doing} something, than to believe something\,---\,but because everyone is taught to do such things, an object of belief is generated. The belief is so certainly correct (for it follows the practice) that it is knowledge; for here knowledge is no other that certainly correct belief in pursuit of a practice. But the connection with testimony is remote and indirect.
