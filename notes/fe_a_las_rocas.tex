\documentclass[10pt]{article}

%LANGUAGE-----------------------------
\usepackage[main=spanish,english,greek]{babel}

\usepackage[backend=biber,style=verbose-ibid]{biblatex}
\addbibresource{ps_biblio.bib} \usepackage{hyperref} \usepackage{blindtext}
\usepackage{titling}

\title{Grandes Decisiones y la propuesta de Jesús: Vive y Arriésgate. O, Jesús,
  Universal Concreto} \author{Joel Del Cueto Santiago} \date{}

\pretitle{\begin{flushleft}\fontsize{19bp}{19bp}\normalfont\sffamily\bfseries\selectfont}
  \posttitle{\par\end{flushleft}}

\preauthor{\begin{flushleft}\fontsize{11bp}{11bp}\normalfont\sffamily\fontseries{CndB}\selectfont}
  \postauthor{\par\end{flushleft}}

\predate{\begin{flushright}\fontsize{11bp}{11bp}\normalfont\sffamily\fontseries{CndB}\selectfont}
  \postdate{\par\end{flushright}}

\usepackage{fontspec,xltxtra,xunicode}


\defaultfontfeatures{Ligatures={TeX,Common}}


\setmainfont[ Path =
/home/blest/Bureau/toolkit/texlive/texmf-dist/fonts/truetype/sorkin/merriweather/,
Extension = .ttf, UprightFont = *-Regular, ItalicFont = *-Italic, BoldFont =
*-Bold, BoldItalicFont = *-BoldIt, ]{Merriweather}

\setsansfont[ Path =
/home/blest/Bureau/toolkit/texlive/texmf-dist/fonts/opentype/impallari/cabin/,
Extension = .otf, UprightFont = *-Regular, ItalicFont = *-RegularItalic,
BoldFont = *-Bold, BoldItalicFont = *-BoldItalic, FontFace={SB}{n}{*-SemiBold},
FontFace={SBIt}{n}{*-SemiBoldItalic}, FontFace={M}{n}{*-Medium},
FontFace={MIt}{n}{*-MediumItalic}, FontFace={CndR}{n}{*Condensed-Regular},
FontFace={CndRIt}{n}{*Condensed-RegularItalic},
FontFace={CndB}{n}{*Condensed-Bold},
FontFace={CndBIt}{n}{*Condensed-BoldItalic},
FontFace={CndSB}{n}{*Condensed-SemiBold},
FontFace={CndSBIt}{n}{*Condensed-SemiBoldItalic},
FontFace={CndM}{n}{*Condensed-Medium},
FontFace={CndMIt}{n}{*Condensed-MediumItalic}, ]{Cabin}


\usepackage{sectsty} \allsectionsfont{\normalfont\sffamily\bfseries}
\paragraphfont{\normalfont\sffamily\fontseries{CndB}\selectfont}


\usepackage{changepage} % for the adjustwidth environment

\newcommand{\citalitlar}[1]{
\begin{adjustwidth}{1.2cm}{}
\emph{#1}
\end{adjustwidth}
}

% Custom Titles
\usepackage[explicit]{titlesec}

% Los títulos de inicio del capítulo/sección centrados, en mayúsculas, y a 12pt
% \titleformat{\chapter} {\normalfont\filcenter} {\thechapter} {12pt}
% {\MakeUppercase{#1}} [\vspace{1ex}%
% {\titlerule[2pt]}]

% \titlespacing*{\chapter}
% {0pt}{.42cm}{0.84cm}

% Los títulos del primer nivel irán en negrita y sin sangría

% \titleformat{\section}
% {\normalfont\bfseries}
% {\thesection}{10pt}
% {#1}
% \titleformat{name=\section,numberless}[block]
% {\normalfont\bfseries}
% {#1}
% {0pt}
% {}

\titlespacing*{\section} {0pt}{0.42cm}{0.21cm}

% los del segundo nivel en negrita y con sangría de 0’5cm

% \titleformat{\subsection}
% {\normalfont\bfseries}
% {\hspace{0.5cm}\thesubsection}{10pt}
% {#1}

% \titleformat{name=\subsection,numberless}[block]
% {\normalfont\bfseries\hspace{0.5cm}}
% {#1}
% {0pt}
% {}

\titlespacing*{\subsection} {0pt}{0.42cm}{0.21cm}

% los de tercer y cuarto nivel, irán sin negrita y con sangría de 0,5cm

% \titleformat{\subsubsection}
% {\normalfont}
% {\hspace{0.5cm}\thesubsubsection}
% {10pt}
% {#1}

% \titleformat{name=\subsubsection,numberless}[block]
% {\normalfont\hspace{0.5cm}}
% {#1}
% {0pt}
% {}

\titlespacing*{\subsubsection} {0pt}{0.42cm}{0.21cm}


%----------------------------------------------------------------------

\usepackage{setspace} \setstretch{1}


% Paragraphs
\setlength{\parindent}{1.25cm} \setlength{\parskip}{0.21cm}

\usepackage[a4paper,inner=3.5cm, outer=3.5cm, top=3.5cm, bottom=3.5cm,
marginparwidth=5cm, marginparsep=1cm]{geometry}

\setcounter{secnumdepth}{-1}


\begin{document}

\maketitle
%\noindent\rule{\textwidth}{1pt}\\

Cuando [Jesús] oyó que Juan había sido entregado, se retiró a Galilea. Y dejando
Nazarét, vino a residir en Cafarnaúm junto al mar, en el término de
Zabulón y Neftalí; para que se cumpliera el oráculo del profeta Isaías:\\\\
\emph{¡Tierra de Zabulón, tierra de Neftalí, camino del mar, allende el Jordán,
  Galilea de los gentiles! El pueblo que habitaba en tinieblas ha visto una gran
  luz; a los que habitaban en paraje de sombras de muerte una luz les ha
  amanecido.}\\\\ Desde entonces comenzó Jesús a predicar y decir:\\\\
\emph{<<Convertíos, porque el Reino de los Cielos ha llegado.>>}\\\\ Caminando
por la ribera del mar de Galilea vio a dos hermanos, Simón, llamado Pedro, y su
hermano Andrés, echando la red en el mar, pues eran pescadores, y les dice:\\\\
\emph{<<Venid conmigo, y os haré pescadores de hombres.>>}\\\\ Y ellos al
instante, dejando las redes, le siguieron. Caminando adelante, vio a otros dos
hermanos, Santiago el de Zebedeo y su hermano Juan, que estaban en la barca con
su padre Zebedeo arreglando sus redes; y los llamó. Y ellos al instante, dejando
la barca y a su padre, le siguieron.
    \begin{flushright}
    Mt 4, 12--22\end
    {flushright}

    \paragraph{Esta discusión tiene como objetivo suscitar un espacio de
      discernimiento en dónde se le plantea a los jóvenes que en un contexto
      de precareidad y fluidez ellos juegan un papel activo en el escenario
      social, y este papel tiene que ver con la tarea del Reino.}

\section{Presupuestos}
  \paragraph{Nuestra discusión da por ciertas varias ideas.\\}

  \subsection{<<Quizá sea verdad>>}
  En una entrevista de Stephen Colbert a Ricky Gervais transmitida en su programa
  el 1ro de este mes, Colbert, que es católico, propuso el tema de la religión a
  Gervais, ateo agnóstico. En su discusíon Ricky describió el ateísmo agnóstico
  con palabras que podemos resumir de esta manera:

  El agnostico afirma que no conocemos que Dios sea verdad. El agnostico ateo
  afirma que dado que es imposible ofrecer pruebas que sustenten el conocimiento
  de la verdad de Dios, esta verdad no es creible.

  El diálogo de los comediantes me recuerda una relfexión de Martin Buber relatada
  por Joseph Ratzinger en su libro ``Introducción al Cristianismo''. Cuenta lo
  siguiente:

  \citalitlar{Un racionalista, un hombre muy entendido, fue un día a disputar con
    un Zaddik (Maestro Judio) con la idea de destruir sus viejas pruebas en favor
    de la verdad de su fe. Cuando entró en su aposento, lo vio pasear por la
    habitación con un libro en las manos y sumido en profunda meditación. Ni
    siquiera se dio cuenta de que había
    llegado alguien. Por fin, lo miró de soslayo y le dijo: <<Quizá sea verdad>>.\\
    El hombre instruido intentó en vano conservar la serenidad: el Zaddik le
    parecía tan terrible, su frase le pareció tan tremenda, que empezaron a
    temblarle las piernas. El rabí Levi Jizchak se volvió hacia él, le miró fija y
    tranquilamente, y le dijo: <<Amigo mío, los grandes de la Tora, con los que
    has disputado, se han prodigado en palabras; tú te has echado a reír. Ni ellos
    ni yo podemos poner ni a Dios y ni a su Reino encima de la mesa. Pero piensa
    en esto: quizá sea verdad>> El racionalista movilizó todas sus fuerzas para
    contrarrestar el ataque; pero aquel <<quizá>>, que de vez en cuando retumbaba
    en sus oídos, oponía resistencia.}
    
  Ratzinger añade:
  \citalitlar{Es ley fundamental del destino humano encontrar lo decisivo de su
    existencia en la perpetua rivalidad entre la duda y la fe, entre la
    impugnación y la certidumbre. La duda impide que ambos se encierren
    herméticamente en su yo y tiende al mismo tiempo un puente que los comunica.
    Impide a ambos que se cierren en sí mismos: al creyente lo acerca al que duda
    y al que duda lo lleva al creyente; para uno es participar en el destino del
    no creyente; para el otro la duda es la forma en la que la fe, a pesar de
    todo, subsiste en él como reto.}

\subsubsection{Primer Presupuesto:}
\begin{itemize}
\item Tanto el creyente como el no creyente tienen que lidiar con la duda, ante
  ese reto común tendremos por cierto que el <<quizá sea verdad>> es un
  postulado que genera, si no atracción, al menos inquietud.
\end{itemize}

\subsection{La Fe pretende ser un modo de conocer}
En esta discusión haremos referencia a premisas que se apoyan en la fe. Por fe
no entendemos ``conocimiento sin pruebas'', sino más bien, ``conocimiento
adquirido por la confianza en aquello que entendemos que es palabra de
alguien''. Según esta definición, hablar de la fe respecto de Dios se refiere a
aquello que conocemos por la confianza que tenemos en aquello que entendemos que
es palabra de Dios. Entendemos sin embargo la dificultad que trae decir que Dios
habla,y que no es una verdad evidente, sin embargo, aquel que dice tener fe
afirma precisamente eso, que ha ``escuchado a Dios''.

\subsubsection{Segundo Presupuesto:}
\begin{itemize}
\item Tenemos por cierto que por fe se denomina una experiencia humana que
  pretende ser medio o camino de conocimiento de la verdad. El tipo de verdad
  que pretende dar a conocer es la verdad de un ser personal.
\end{itemize}

\subsection{Dios no es el objeto de tal o cual culto}
 
Definir a Dios como ``objeto de culto'' no nos sirve, porque tendríamos que
definir culto como ``honor que se ofrece a la deidad''. Es decir, el culto
divino es el tipo especial de honor que se pretende que sea rendido a una
deidad, sin embargo, este honor puede ser ofrecido a algo o alguien que no sea
una deidad. Por Dios entendemos no un nombre propio, sino una descripción
definitiva, es decir, equivale a `el único dios verdadero', o `aquella deidad
que sí es verdad'.

\subsubsection{Tercer Presupuesto:}
\begin{itemize}
\item Tenemos por cierto que por el término Dios hemos de entender 'el único y
  verdadero Dios'.
\end{itemize}

\section{Cuestión Central}
       \paragraph{Expuestos los presupuestos, planteamos la cuestión a ser
         discutida.}

         \subsection{El absurdo}
         Quisiera comenzar con algo un poco absurdo\ldots que nos ayude a
         plantearnos nuestra pregunta. Douglas Adams comienza la historia en ``Hitchhikers guide
         to the galaxy'' (1978) con la destrucción del planeta tierra:

         \citalitlar{Este planeta tiene --o tenía-- un problema, que era este:
           la mayoría de la gente que vivía en él se sentían infelices casi todo
           el tiempo. Muchas soluciones fueron sugeridas para este problema,
           pero la mayoría de éstas estaban grandemente relacionadas con el
           movimiento de pequeños pedazos de papel verde, lo que resulta extraño
           dado que no eran los pedazos de papel verde los que eran infelices.}
         \citalitlar{Y así el problema permanecía; muchas de las personas eran
           crueles, y la mayoría miserables, hasta los que tenían relojes
           digitales. [\ldots] Entonces, un jueves, casi dos mil años después de
           que un hombre fuera clavado a un árbol por decir cuán grandioso sería
           ser amables para variar, una joven sentada en un pequeño café realizó
           de repente qué era lo que había estado mal todo este tiempo, y
           finalmente entendió cómo el mundo podía ser hecho un lugar bueno y
           feliz. Esta vez funcionará --pensó--. Sin embargo, antes de que
           pudiera contarlo a nadie una estúpida y terrible catástrofe ocurrió.}
         \citalitlar{<<Como ustedes sin duda sabrán, los planes para el
           desarrollo de las regiones exteriores de la galaxia requieren la
           construcción de una autopista hyperespacial que cruzará a través de
           sus sitema solar y desafortunadamente su planeta es uno de los que ha
           sido planificado para demolición.>>} \citalitlar{<<No tiene sentido
           que reaccionen sorprendidos. Todas las tablas de planificación y
           ordenes de demolición han sido publicadas en el departamento que les
           corresponde an Alpha Centauri durante 50 de sus años, así que han
           tenido tiempo de sobra para presentar una querella formal, y es
           demasiado tarde ahora para quejarse.>>} \citalitlar{<<¿Cómo que no
           han ido nunca a Alpha Centauri? ¡Si está solo a 4 años luz! Lo
           sentimos, pero si no se molestan en interesarse por las cuestiones
           locales, es su problema.>>}

       Nuestro mundo, de valor incalculable para cada uno de nosotros, queda
       reducido a una cosa insignificante, destruido por una gestión
       burocrática. Expresa así una cuestión prominente en la obra,
         \begin{itemize}
         \item \emph{en esta vida absurda, ¿qué puede darle valor a nuestra
             existencia?}
         \end{itemize}


       Voy todavía un poco mas allá con esta cuestión. Esta vez inspirado en una
       escena de la película ``The Meaning of Life'' (1983) de Monty Python.
       \begin{adjustwidth}{1.2cm}{}
         Dos trabajadores tocan la puerta de una casa y les recibe un hombre.
         Inmediatamente le asaltan con la pregunta:``Buenas. Ehh, ¿nos donaria
         usted su higado?'' Despues de una breve discusión fuerzan al hombre a
         donarles el higado en una escena estraflaria.

         La esposa del donante entra en escena y sin manifestar ninguna sorpresa
         por lo que allí ocurre le ofrece té a los trabajadores. Uno de ellos la
         acompaña a la cocina y allí le pregunta a la Señora:

         \noindent\emph{Trabajador.}~---¿Me pregunto\ldots tendrá algún plan o
         pretendiente despues de esto?

         \noindent\emph{Mrs. Brown.}~---No, ya no estoy para eso, no creo que
         vuelva a juntarme con nadie.

         \noindent\emph{Trabajador.}~---¿Segura?

         \noindent\emph{Mrs. Brown.}~---Segura.

         \noindent\emph{Trabajador.}~---¿Donaría usted su hígado entonces?

         \noindent\emph{Mrs. Brown.}~---No, ¡oh no! Me daría miedo\ldots

         \noindent\emph{Trabajador.}~---Muy bien, escuche entonces esto\ldots

         Y de repente sale de la never un hombre que comienza una espectacular
         presentación musical\ldots y canta\ldots

         \emph{Solo recuerde que está en un planeta que evoluciona y rota a
           9,000 millas por hora, y está en órbita a 19,000 millas por segundo
           alrededor del sol, y éste y usted y yo y las estrellas que podemos
           ver nos movemos millones de millas al día en esta galaxia que
           llamamos la vía lactea\ldots Y esta galaxia es sólo una de millones
           de billones en este maravilloso y grandioso universo\ldots}

         \noindent\emph{Mrs. Brown}~---Le hace sentir a uno así como
         insignificante, ¿no?

         \noindent\emph{Trabajador.}~---Sí, sí. ¿Me dona su higado entonces?

         \noindent\emph{Mrs. Brown}~---Sí, muy bien\ldots Sí que me
         convenció\ldots

       \end{adjustwidth}

       Aparece aquí un misma consideración de la existencia como una
       tragicomedia absurda. La vida como una particula insignificante. Y como
       resultado una idea macabra de lo que significa donar la vida. Podríamos
       desde aquí plantear la pregunta:

       \begin{itemize}
       \item \emph{¿Existe alguna causa a la que no sea un error comprometerle
           la vida por entero?}
       \end{itemize}

\section{Una cuestión existencial(ista)}
    Todo esto nos ha dejado con un sabor existencialista en la boca.

    El planteamiento de que <<no hay una teleología que de orden al mundo\ldots ~La
    existencia no tiene detrás una intencionalidad>> se convierte en motivo de
    angustia y urgencia. Nada tiene intención, o sentido, pero entonces puedes
    dárselo tu a cualquier cosa.

    Un testimonio al que me sigo remitiendo y que ilustra esta noción es el de
    Victor Frankl y su experiencia en el campo de concentración: 

    <<en realidad no importa lo que esperemos de la vida, sino si la vida
    espera algo de nosotros. Tenemos que dejar de hacernos preguntas sobre el
    significado de la vida y, en vez de ello, pensar en nosotros como en seres
    a quienes la vida les inquiriera continua e incesantemente. Nuestra
    contestación tiene que estar hecha no de palabras ni tampoco de
    meditación, sino de una conducta y una actuación rectas. En última
    instancia, vivir significa asumir la responsabilidad de encontrar la
    respuesta correcta a los problemas que ello plantea y cumplir las tareas
    que la vida asigna continuamente a cada individuo.>>

    El mismo Frankl plantea, sin embargo:

    <<La apatía, el principal síntoma de la segunda fase, era un mecanismo
    necesario de autodefensa.>>

¿Por qué invocar todas estas cuestiones?

El siglo pasado estuvo lleno de experiencias estremecedoras para nuestra
cultura. En este sentido Frankl sería portavoz de un sentimiento generalizado,
la realidad que nos rodea es tan atroz que es dificil argumentar que haya un
propósito detrás de todo.

Nuestro siglo nos sigue planteando nuevas desiluciones, los sistemas políticos,
económicos e idelógicos pasan por crisis que nos hacen cuestionar su valor.

¿Cual puede ser nuestra respuesta? Me resuenan las palabras de Frankl, la apatía
sería un mecanismo de autodefensa. Dejar la vida en suspenso, perder la audacia
de las grandes propuestas. Sería un refugio.

Albert Camus(1913-1960) en su obra ``El mito de Sísifo'' recuerda el mito de
aquel hombre condenado por los dioses a subir sin cesar una roca hasta la cima
de una montaña desde donde la piedra volvía a caer por su propio peso. Sísifo es
feliz, sin embargo, dice Camus. Aún en su trabajo absurdo, su destino le
pertenece, por eso es feliz. El que trabaja hoy ha de juzgar que todo está bien,
y creer que su destino está en sus manos?

¿Esto es todo lo que se puede esperar?

\subsection{El Reino de los cielos ha llegado}

    Es en este contexto que quisiera evocar la causa de Jesús de Nazaret. Dialogar
    con él.

    Recuerdo cuando estaba en escuela superior y me puse a leer los evangelios de
    corrido por primera vez. Empecé por Mateo. No tardé en toparme con un asunto
    que me llamó la atención.

    Al narrar los inicios de la misión de Jesús de Nazarét el evangelista lo
    describe proponiendo el siguiente mensaje: <<Convertíos, porque el Reino de
    los cielos ha llegado>>.

    Recuerdo que no entendí eso del Reino. Pensaba que entendía el mensaje
    cristiano, pero no entendí a qué se refería esta expresión. Pienso que lo
    confuso de la idea en el momento era pensar que el mensaje propuesto por Jesús
    no consistía en enseñanzas para juzgar o aplicarme en mi conciencia o
    interioridad, sino que invitaba además a juzgar y entender la situación que me
    rodea.

    El ``Reino de Dios'' no se limita a describir una situación de ``mi mundo'',
    sino que describe ``nuestro mundo''.

    Quisiera compartirles el relato completo. (Arriba)

    Puntos del relato: 

    -Un hecho histórico adquiere un valor trascendente o universal para la vida
    de los que lo han experimentado. 

    -Son invitados a entablar una nueva relación con los seres humanos y Dios:
    Pescadores de hombres

    -Se valora la enseñanza de Jesús desde la fe. Se descubre en ella la palabra
    de Dios.

    Para el contexto judío al que se dirige el evangelio según san Mateo esta
    idea no resulta tan problemática. El Malkut de Dios o Reino de Dios es un
    concepto conocido en ese contexto.

    Cualquier maestro judío podía haber dicho: ``Que el Reino de Dios llegue en
    nuestros días.'' o ``Si se arrepienten y se comprometen a observar la Torá
    habrán aceptado el reino de Dios.''

    Sin embargo el mensaje de Jesús va más allá. Dice, por ejemplo: si yo
    expulso demonios por el dedo de Dios es que el reino de Dios ha llegado a
    ustedes.

    No se trata de tener a Dios por rey en el sentido de obedecer sus
    mandamientos, sino de ser confrontados cone el poder de Dios que actúa
    en el mundo.

    Jesús habla del reino como un hecho presente que ha de ser reconocido tanto por
    los que lo rechazan como por los que lo aceptan con sus acciones. Nos lo
    presenta como un hecho histórico que no depende de la actitud de los seres
    humanos.

\subsection{La forma pesada de la fe}
    Las propuestas de la fe se hacen confusas cuando:
  \begin{itemize}
  \item se confunde la actitud de la fe con actitudes mágicas o
    supersticiosas se

  \item evaluan los postulados de la fe como parte de un sistema
    ideológico o como una doctrina que naciera con la intención de
    fundar una ideología específica.

  \item se desliga de su finalidad interrelacional

  \item se anteponen discursos que pretenden explicar y que hay que
    admitir antes de atender el contenido de la fe
    \end{itemize}

    Lo que nos molesta en la fe cristiana es sobre todo la carga de excesivos
    enunciados, que se han amontonado a lo largo de la historia, y se
    presentan ahora todos ellos ante nosotros exigiendo nuestra fe.

    Cuando se oye continuamente que esta o aquella conferenci, este o aquel
    libro, han sido liberadores, entonces resulta claro que los seres humanos
    sienten hoy comouna carga la forma de la fe, pero que al mismo tiempo
    están animados por la exigencias de ser creyentes.

    Habra momentos en la vida en que, en la múltiple oscuridad de la fe,
    tendremos que concentrarnos realmente en el simple sí: creo en tí Jesus de
    Nazaret; confío en que en tí se ha mostrado el sentido divino por el cual
    puedo vivir mi vida paciente y animoso. Mientras esté presente este
    centro, el ser humano está en la fe aunque muchos de los enunciados
    concretos de ésta le resulten oscuros y por el momento no practicables.

    La fe cristiana es encontrar un Tú que me sostiene y que, a pesar de la
    imperfección y del carácter intrínsdsecamente incompleto de tod encuentro
    humano, regala la promesa de un amor indestructible que no solo aspira a
    la eternidad, sino que la otroga.

\subsection{Síntesis de las Parábolas del Reino}

    Jesús usa sus palabras, sus parábolas, para ayudar a los seres humanos que en
    los acontecimientos que tienen ante sus ojos --desde sus acciones por el bien
    de los demás, hasta, el trágico conflicto de la cruz-- Dios les sale al paso
    en su reino, su poder y su gloria. Este mundo se ha convertido en el escenario
    de un drama divino en el que las decisiones eternas quedan al descubierto.

    - Ha llegado la hora del cumplimiento, Dios actua con poder en la historia. No
    se trata sin embargo de una exhibición espectacular, sino que lo que se tiene
    ante los ojos es a un joven carpintero de Galilea, los indiferentes dicen:
    vino Juan, que estaba loco, ahora viene Jesús, que no merece respeto. Jesús
    sin embargo insiste, el reino de Dios ha llegado, porque hay una fuerza que
    actua desde dentro, como la levadura en la masa, y ésta no se detiene.

    -Jesús no ha venido como un reformador religioso a poner un remiendo en la
    desgarrada vestidura del judaísmo farisaico. Sino a ofrecer un nuevo punto de
    partida en la relación entre Dios y el ser humano. Jesús se muestra satisfecho
    de se conocido como el amigo de publicanos y pecadores. Su mensaje fue
    acogido. Entre judios y no judios el llamamiento de Jesús tuvo éxito.

    - El llamamiento y el éxito de Jesús produjeron escándalo. ¿Cómo puede ser eso
    del Reino de Dios, cuando todas las salvaguardias morales, laboriosamente
    levantadas por los maestros de la ley, eran dejadas a un lado y la gente de
    mal vivir era bien recibida entre los discípulos? Se acoje al que dijo no y
    después dice sí, a los que acogen la invitación, se da el jornal entero por
    una hora de trabajo.

    -En el reino se juzga contra los fariseos, a los que viven con cautela
    egoísta, a los exclusivistas, a los negligentes con sus responsabilidades y
    ciegos a Dios y su obra. A la sal desabrida.

    -El Reino exige una gran respuesta, una decisión que determine la orientación
    de la vida. Aceptar el Reino significa arriegar la vida. Muchedumbres fueron
    atraidas por la predicación de Jesús. Aquellas personas se juzgaron a sí
    mismas no por un examen introspectivo, sino por la reacción ante la situación
    concreta.

\subsection{Discernimiento, amistad, comunidad}
  Como punto final quisiera compartirles una llamada a construir comunidad. ¿Tú y
  qué otros once? ¿Cuál es tu comunidad en donde disciernes y construyes.

Salmo 1:

Dichoso el hombre
que no sigue el consejo de los impíos,
ni entra por la senda de los pecadores,
ni se sienta en la reunión de los cínicos;
 sino que su gozo es la ley del Señor,
y medita su ley día y noche.


Será como un árbol 
plantado al borde de la acequia:
da fruto en su sazón
y no se marchitan sus hojas;
y cuanto emprende tiene buen fin.

\end{document}
