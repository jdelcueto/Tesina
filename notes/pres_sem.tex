\documentclass[10pt]{article}

%LANGUAGE-----------------------------
\usepackage[main=spanish,english,greek]{babel}

\usepackage[backend=biber,style=verbose-ibid]{biblatex}
\addbibresource{ps_biblio.bib}
\usepackage{hyperref}
\usepackage{blindtext}
\usepackage{titling}

\title{Apuntes sobre el lenguaje de la fe y su transmisión}
\author{Joel Del Cueto Santiago}
\date{}

\pretitle{\begin{flushleft}\fontsize{19bp}{19bp}\normalfont\sffamily\bfseries\selectfont}
\posttitle{\par\end{flushleft}}

\preauthor{\begin{flushleft}\fontsize{11bp}{11bp}\normalfont\sffamily\fontseries{CndB}\selectfont}
\postauthor{\par\end{flushleft}}

\predate{\begin{flushright}\fontsize{11bp}{11bp}\normalfont\sffamily\fontseries{CndB}\selectfont}
\postdate{\par\end{flushright}}

\usepackage{fontspec,xltxtra,xunicode} 


\defaultfontfeatures{Ligatures={TeX,Common}}


\setmainfont[
     Path = /home/blest/Bureau/toolkit/texlive/texmf-dist/fonts/truetype/sorkin/merriweather/,
     Extension      = .ttf,
     UprightFont    = *-Regular,
     ItalicFont     = *-Italic,
     BoldFont       = *-Bold,
     BoldItalicFont = *-BoldIt,
     ]{Merriweather}

\setsansfont[
     Path = /home/blest/Bureau/toolkit/texlive/texmf-dist/fonts/opentype/impallari/cabin/,
     Extension      = .otf,
     UprightFont    = *-Regular,
     ItalicFont     = *-RegularItalic,
     BoldFont       = *-Bold,
     BoldItalicFont = *-BoldItalic,
     FontFace={SB}{n}{*-SemiBold},
     FontFace={SBIt}{n}{*-SemiBoldItalic},
     FontFace={M}{n}{*-Medium},
     FontFace={MIt}{n}{*-MediumItalic},
     FontFace={CndR}{n}{*Condensed-Regular},
     FontFace={CndRIt}{n}{*Condensed-RegularItalic},
     FontFace={CndB}{n}{*Condensed-Bold},
     FontFace={CndBIt}{n}{*Condensed-BoldItalic},
     FontFace={CndSB}{n}{*Condensed-SemiBold},
     FontFace={CndSBIt}{n}{*Condensed-SemiBoldItalic},
     FontFace={CndM}{n}{*Condensed-Medium},
     FontFace={CndMIt}{n}{*Condensed-MediumItalic},
     ]{Cabin}


\usepackage{sectsty}
\allsectionsfont{\normalfont\sffamily\bfseries}
\paragraphfont{\normalfont\sffamily\fontseries{CndB}\selectfont}


\usepackage{changepage}   % for the adjustwidth environment

\newcommand{\citalitlar}[1]{
\begin{adjustwidth}{1.2cm}{}
\emph{#1}
\end{adjustwidth}
}

%Custom Titles
\usepackage[explicit]{titlesec}

%Los títulos de inicio del capítulo/sección centrados, en mayúsculas, y a 12pt
%\titleformat{\chapter}
%  {\normalfont\filcenter}
%  {\thechapter}
%  {12pt}
%  {\MakeUppercase{#1}}
%  [\vspace{1ex}%
%  {\titlerule[2pt]}]

%\titlespacing*{\chapter}
%{0pt}{.42cm}{0.84cm}

%Los títulos del primer nivel irán en negrita y sin sangría

%\titleformat{\section}
%  {\normalfont\bfseries}
%  {\thesection}{10pt}
%  {#1}
%
%\titleformat{name=\section,numberless}[block]
%  {\normalfont\bfseries}
%  {#1}
%  {0pt}
%  {}

\titlespacing*{\section}
{0pt}{0.42cm}{0.21cm}

%los del segundo nivel en negrita y con sangría de 0’5cm

%\titleformat{\subsection}
%  {\normalfont\bfseries}
%  {\hspace{0.5cm}\thesubsection}{10pt}
%  {#1}

%\titleformat{name=\subsection,numberless}[block]
%  {\normalfont\bfseries\hspace{0.5cm}}
%  {#1}
%  {0pt}
%  {}

\titlespacing*{\subsection}
{0pt}{0.42cm}{0.21cm}

%los de tercer y cuarto nivel, irán sin negrita y con sangría de 0,5cm

%\titleformat{\subsubsection}
%  {\normalfont}
%  {\hspace{0.5cm}\thesubsubsection}
%  https://inbox.google.com/ | Inbox – jdelcueto@gmail.com
https://store.google.com/config/pixel_phone?hl=en_US | Pixel - The First Phone by Google - Google Store
https://calendar.google.com/calendar/render#main_7 | Google Calendar - Week of Feb 19, 2017
https://jezebel.com/bill-maher-is-a-monster-1792516121 | Bill Maher Is a Monster
https://themuse.jezebel.com/a-loose-and-even-enjoyable-reading-list-to-help-explain-1791694997 | A Loose and Even Enjoyable Reading List to Help Explain How We Got Here
https://www.bloomberg.com/features/2016-america-divided/milo-yiannopoulos/ | Milo Yiannopoulos Is the Pretty, Monstrous Face of the Alt-Right
https://www.bloomberg.com/ | Bloomberg.com
https://www.nytimes.com/2014/10/09/opinion/bill-maher-isnt-the-only-one-who-misunderstands-religion.html | Bill Maher Isn’t the Only One Who Misunderstands Religion - The New York Times
http://www.theverge.com/2015/11/17/9748958/stephen-colberts-bill-maher-interview | Stephen Colbert's interview with Bill Maher is refreshingly antagonistic - The Verge
http://www.theverge.com/ | The Verge
https://www.reddit.com/r/ffxiv/ | Final Fantasy X|V: Heavensward
https://www.wikiwand.com/en/Monty_Python%27s_The_Meaning_of_Life | Monty Python's The Meaning of Life - Wikiwand
http://www.philfilms.utm.edu/1/meaning.htm | Monty Python's The Meaning of Life (Philosophical Films)
http://imago.yolasite.com/resources/Kant,%20Immanuel%20-%20Cr%C3%ADtica%20del%20juicio.pdf | Microsoft Word - Crítica del jucio - Observaciones sobre lo bello y lo subl–
http://www.tesisenred.net/bitstream/handle/10803/310600/mro1de1.pdf?sequence=1 | mro1de1.pdf
http://www.scielo.cl/scielo.php?script=sci_arttext&pid=S0718-43602009000100008 | EL ARGUMENTO COSMOLÓGICO DE ZUBIRI
https://www.wikiwand.com/es/Epoj%C3%A9 | Epojé - Wikiwand
http://192.168.0.1/?wan_dynamic | 192.168.0.1/?wan_dynamic
https://encrypted.google.com/search?hl=en&q=la%20caida%20de%20las%20grandes%20ideolog%C3%ADas%20guerra%20mundial | la caida de las grandes ideologías guerra mundial - Google Search
http://elnacional.com.do/la-guerra-fria-de-las-ideologias/ | La guerra fría de las ideologías - El Nacional
https://books.google.com.pr/books?id=YVmM_o77XC4C&pg=PT82&lpg=PT82&dq=la+caida+de+las+grandes+ideolog%C3%ADas+guerra+mundial&source=bl&ots=PsHlWo7mjJ&sig=B8myerRByITJ2AwYxALBj8UbEzM&hl=en&sa=X&redir_esc=y#v=onepage&q=la%20caida%20de%20las%20grandes%20ideolog%C3%ADas%20guerra%20mundial&f=false | Pensamiento Critico - Google Books
https://books.google.com.pr/books?id=YVmM_o77XC4C&dq=la+caida+de+las+grandes+ideolog%C3%ADas+guerra+mundial&source=gbs_navlinks_s | Pensamiento Critico - Google Books
https://encrypted.google.com/search?hl=en&q=google%20pixel%20out%20of%20stock#q=google+pixel+out+of+stock&safe=off&hl=en&tbs=qdr:w | google pixel out of stock - Google Search
http://www.nowinstock.net/electronics/mobilephones/unlocked/googlepixel/ | Google Pixel Phone In Stock Tracker - Pixel, Pixel XL - NowInStock.net
http://www.elnuevodia.com/ | Portada | El Nuevo Día
https://www.wikiwand.com/es/Movimiento_15-M | Movimiento 15-M - Wikiwand

http://tutex.tug.org/pracjourn/2005-4/hefferon/hefferon.pdf | hefferon.pdf
http://tutex.tug.org/fonts/ | Fonts and TeX - TeX Users Group
http://www.creativebloq.com/design/designers-guide-golden-ratio-12121546 | The designer's guide to the Golden Ratio | Creative Bloq
https://tex.stackexchange.com/questions/55502/modern-book-design-margins-and-typed-area | typography - Modern book design, margins and typed area - TeX - LaTeX Stack Exchange
http://retinart.net/graphic-design/secret-law-of-page-harmony/ | The Secret Law of Page Harmony - Retinart
https://texblog.org/2011/02/26/generating-dummy-textblindtext-with-latex-for-testing/ | Generating dummy text/blindtext with LaTeX for testing – texblog
https://tex.stackexchange.com/questions/42050/defining-page-margins-with-geometry-for-two-sided-documents | Defining page margins with \geometry for two-sided documents - TeX - LaTeX Stack Exchange
http://andrew.hedges.name/experiments/aspect_ratio/ | Aspect Ratio Calculator (ARC)
https://www.tug.org/TUGboat/tb29-1/tb91walczak.pdf | tb91walczak.pdf
https://tex.stackexchange.com/questions/20425/z-level-in-tikz | "Z-level" in TikZ - TeX - LaTeX Stack Exchange
http://ctan.uniminuto.edu/graphics/pgf/base/doc/pgfmanual.pdf | pgfmanual.pdf
https://tex.stackexchange.com/questions/230861/place-tikz-picture-over-top-of-text | xetex - Place TikZ picture over top of text - TeX - LaTeX Stack Exchange
https://tex.stackexchange.com/questions/41382/place-a-tikz-picture-on-every-page | header footer - Place a TikZ picture on every page - TeX - LaTeX Stack Exchange
https://texblog.org/2013/02/13/latex-documentclass-options-illustrated/ | LaTeX documentclass options illustrated – texblog
https://www.latex-tutorial.com/tutorials/advanced/latex-tikz/ | Draw pictures in LaTeX - With tikz/pgf
https://www.sharelatex.com/blog/2013/08/27/tikz-series-pt1.html | Basic Drawing Using TikZ - ShareLaTeX, Online LaTeX Editor
https://en.wikibooks.org/wiki/LaTeX/PGF/TikZ#Specifying_Coordinates | LaTeX/PGF/TikZ - Wikibooks, open books for an open world
https://www.tug.org/TUGboat/tb29-1/tb91walczak.pdf | tb91walczak.pdf
http://latex-community.org/forum/viewtopic.php?f=45&t=23255 | Aperiodic Grid
http://ctan.uniminuto.edu/macros/latex/contrib/background/background.pdf | Background material
https://tex.stackexchange.com/questions/219232/calculating-with-different-types-of-variables | tikz pgf - Calculating with different types of variables - TeX - LaTeX Stack Exchange
https://duckduckgo.com/?q=latex+geometry+margins&atb=v34-1a_&ia=web | latex geometry margins at DuckDuckGo
https://kb.mit.edu/confluence/pages/viewpage.action?pageId=3907057 | How can I change the margins in LaTeX? - IS&T Contributions - Hermes
https://en.wikibooks.org/wiki/LaTeX/Page_Layout | LaTeX/Page Layout - Wikibooks, open books for an open world
https://www.sharelatex.com/learn/Page_size_and_margins | Page size and margins - ShareLaTeX, Online LaTeX Editor
https://encrypted.google.com/search?hl=en&q=latex%20fp%20%5Cpaperwidth#safe=off&hl=en&q=latex+fp+macro | latex fp macro - Google Search
https://tex.stackexchange.com/questions/48213/macros-that-wont-accept-other-macros-as-arguments | fp - macros that won't accept other macros as arguments - TeX - LaTeX Stack Exchange
https://tex.stackexchange.com/questions/222425/tex-latex-macro-for-getting-a-length-as-number | expansion - TeX/LaTeX: Macro for getting a length as number - TeX - LaTeX Stack Exchange
https://tex.stackexchange.com/questions/154224/fp-calculation-variable | fp calculation variable - TeX - LaTeX Stack Exchange
http://latex-community.org/forum/viewtopic.php?t=2712 | How to print out the value of \textwidth?
https://www.ctan.org/pkg/fp?lang=en | CTAN: Package fp
http://www.tug.dk/FontCatalogue/raleway/ | The LaTeX Font Catalogue – Raleway
http://fontpair.co/ | Font Pair - Helps designers pair Google Fonts together. Beautiful Google Font combinations and pairs.
https://www.typewolf.com/google-fonts | The 40 Best Google Fonts—A Curated Collection for 2016 · Typewolf
https://www.typewolf.com/site-of-the-day/fonts/roboto-slab | Roboto Slab Font Combinations & Similar Fonts · Typewolf
https://www.typewolf.com/site-of-the-day/fonts/raleway | Raleway Font Combinations & Similar Fonts · Typewolf
https://www.typewolf.com/site-of-the-day/jan-paul-koudstaal | Jan-Paul Koudstaal · Typewolf
https://en.wikibooks.org/wiki/LaTeX/Fonts | LaTeX/Fonts - Wikibooks, open books for an open world
https://tex.stackexchange.com/questions/109186/making-section-headings-bold-and-sans-serif | fonts - Making section headings bold and sans serif - TeX - LaTeX Stack Exchange
https://tex.stackexchange.com/questions/296802/coloring-section-rule-using-sectsty-package | sectioning - coloring section rule using sectsty package - TeX - LaTeX Stack Exchange
http://ctan.uniminuto.edu/macros/latex/contrib/sectsty/sectsty.pdf | sectsty.pdf
https://en.wikibooks.org/wiki/LaTeX/Colors | LaTeX/Colors - Wikibooks, open books for an open world
https://texblog.org/2007/11/07/headerfooter-in-latex-with-fancyhdr/ | Header/Footer in LaTeX with Fancyhdr – texblog
https://en.wikibooks.org/wiki/LaTeX/Page_Layout | LaTeX/Page Layout - Wikibooks, open books for an open world
http://ctan.uniminuto.edu/macros/latex/contrib/fancyhdr/fancyhdr.pdf | fancyhdr.pdf
http://ctan.uniminuto.edu/macros/latex/contrib/fancyhdr/fancyhdr.pdf | fancyhdr.pdf
https://www.sharelatex.com/learn/Multiple_columns | Multiple columns - ShareLaTeX, Online LaTeX Editor
https://tex.stackexchange.com/questions/37879/how-to-set-the-header-font-size-using-fancyhdr-package | fontsize - How to set the header font size using fancyhdr package? - TeX - LaTeX Stack Exchange
https://stackoverflow.com/questions/890127/how-to-set-latex-font-size-in-millimeter | how to set LaTeX font size in millimeter? - Stack Overflow
http://ctan.uniminuto.edu/macros/latex/contrib/oberdiek/ifdraft.pdf | The ifdraft package
https://www.typewolf.com/site-of-the-day/fonts/cabin | Cabin Font Combinations & Similar Fonts · Typewolf
https://encrypted.google.com/search?hl=en&q=latex%20multiple%20fonts | latex multiple fonts - Google Search
https://tex.stackexchange.com/questions/47546/using-multiple-font-types | Using multiple font types - TeX - LaTeX Stack Exchange
https://www.tug.org/pracjourn/2006-1/schmidt/schmidt.pdf | schmidt.pdf
https://encrypted.google.com/search?hl=en&q=latex%20koma%20classes | latex koma classes - Google Search
https://tex.stackexchange.com/questions/73141/why-should-i-not-use-the-koma-script-classes#73146 | Why should I *not* use the KOMA-Script classes? - TeX - LaTeX Stack Exchange
https://www.ctan.org/pkg/typearea | CTAN: Package typearea
http://ctan.uniminuto.edu/macros/latex/contrib/koma-script/doc/scrguien.pdf | KOMA-Script
http://ctan.uniminuto.edu/macros/latex/contrib/koma-script/doc/scrguien.pdf | KOMA-Script
https://tex.stackexchange.com/questions/7742/what-are-the-strengths-and-weaknesses-of-koma-script-and-memoir | What are the strengths and weaknesses of KOMA-Script and memoir? - TeX - LaTeX Stack Exchange
https://tex.stackexchange.com/questions/165230/using-koma-script-packages-with-other-classes | Using KOMA-Script packages with other classes - TeX - LaTeX Stack Exchange
{10pt}
%  {#1}

%\titleformat{name=\subsubsection,numberless}[block]
%  {\normalfont\hspace{0.5cm}}
%  {#1}
%  {0pt}
%  {}

\titlespacing*{\subsubsection}
{0pt}{0.42cm}{0.21cm}


%----------------------------------------------------------------------

\usepackage{setspace}
\setstretch{1}


%Paragraphs
\setlength{\parindent}{1.25cm}
\setlength{\parskip}{0.21cm}

\usepackage[a4paper,inner=3.5cm, outer=3.5cm, top=3.5cm, bottom=3.5cm,
marginparwidth=5cm, marginparsep=1cm]{geometry}

\setcounter{secnumdepth}{-1}


\begin{document}

\maketitle
%\noindent\rule{\textwidth}{1pt}\\

\emph{Lo que existía desde el principio, lo que hemos oído,
lo que hemos visto con nuestros ojos,
lo que contemplamos
y palparon nuestras manos
acerca de la Palabra de vida
---pues la Vida se manifestó,
y nosotros la hemos visto y damos testimonio
y os anunciamos la Vida eterna,
que estaba junto al Padre y que se nos maifestó---
lo que hemos visto y oído os lo anunciamos,
para que también vosotros estéis en comunión con nosotros.
Y nosotros estamos en comunión con el Padre y con su Hijo Jesucristo.
Os escribimos esto para que nuestro gozo sea completo.}
\begin{flushright}
1~Jn 1, 1--4\end
{flushright}

\section{Anunciar la Palabra de la Vida}
\paragraph{``Quien intente hoy día hablar del problema de la fe cristiana a los
  hombres que ni por vocación ni por convicción se hallan dentro de la temática
  eclesial, notará al punto la ardua dificultad de tal empresa. Probablemente
  tendrá en seguida la impresión de que su situación ha sido descrita con
  bastante acierto en la conocida narración parabólica de
  Kierkegaard sobre el payaso de la aldea en llamas\ldots''\\\\
}

En 1843 Kierkergaard publicó ``O lo uno o lo otro''. Bajo el pseudónimo de
`Victor Eremita' presenta una ojeada a la vida de dos personas: `el estético'
(A) y Wilhelm (B). El libro se divide así en dos partes; los aforismos
poéticos y afectivos de `A' y cartas mas discursivas y éticas de `B'. Ambas
partes tratan sobre la vida, su sentido y cuál pueda ser la mejor manera de
vivirla.

Es `el estético' quién con cierto desdén por la futilidad de la vida y
cuestionando su sentido cuenta esa breve `parábola' del incendio y los
payasos: \citalitlar{Sucedió una vez en un teatro que se prendió fuego entre
  bastidores. El payaso acudió para avisar al público de lo que ocurría.
  Creyeron que se trataba de un chiste y aplaudieron; aquél lo repitió y ellos
  rieron aún con más fuerza.\\ De igual modo pienso que el mundo se acabará
  con la carcajada general de amenos guasones creyendo que se trata de un
  chiste.\footcite{kierkegaard2006uno}}

Lo más amargo de la historia es que no es ficticia. El 14 de febrero de 1836 una
lámpara mal montada en el escenario comenzó un incendio en el Teatro Lehmann en
San Petersburgo que causó la muerte de cientos de
personas.\footcite{gerhard1896theatre}

En 1969 Joseph Ratzinger comienza ``Introducción al Cristianismo'' haciendo
  alusión a este relato para `pintar un irritante cuadro' que `refleja en cierto
  modo la agobiante situación en que se encuentra el pensamiento teológico
  actual'\footcite{introcrist}. Dice que el teólogo\ldots 
  \citalitlar{Ya puede decir lo que quiera, lleva siempre la etiqueta del papel
    que desempeña. Y, aunque se esfuerce por presentarse con toda seriedad, se
    sabe de antemano lo que es: un payaso. Se conoce lo que dice y se sabe también
    que sus ideas no tienen nada que ver con la realidad. Se le puede escuchar
    confiado, sin temor al peligro de tener que preocuparse seriamente por algo.}

  De este modo el pensamiento teológico actual se encuentra \emph{``en la
    agobiante imposibilidad de romper las formas fijas del pensamiento y del
    lenguaje, y en la de hacer ver que la teología es algo sumamente serio en la
    vida de los hombres''}.

  Consideremos el contexto que tiene esta reflexión, los años del Concilio
  Vaticano II culminado en el 65 y el esfuerzo de la Iglesia por dialogar con la
  cultura moderna. Así no extraña la pregunta que Ratzinger plantea a
  continuación: 
  \citalitlar{¿Es que basta con que nos agarremos al `aggiornamento', que nos
    quitemos el maquillaje y asumamos el aspecto civil de un lenguaje secular o de
    un cristianismo sin religión para que todo se arregle?\\ ¿Es que basta con
    cambiar los vestidos eclesiásticos para que los hombres acudan alegres a
    apagar el fuego que, como dice el teólogo, existe y es un peligro para
    nosotros?}

  Esta respuesta es todavía superficial. Para Ratzinger el problema no es
  simplemente uno de ropajes externos. Así añade:
  \citalitlar{Al resultar el quehacer teológico algo tan insólito para los hombres
    de nuestro tiempo, quien tome la cosa en serio se dará cuenta no sólo de lo
    difícil que es traducir, sino también de lo vulnerable que es su propia fe
    que, al querer creer, experimentará y reconocerá en sí mismo el inquietante
    poder de la incredulidad.}

  La situación del creyente, en este sentido, no es tan distinta de la del
  no-creyente, en ambos operan fuerzas semejantes aunque en modo diverso;
  \emph{``de la misma manera que el creyente se siente continuamente amenazado por
    la incredulidad, que es para él su más seria tentación, así también la fe
    siempre será tentación para el no-creyente''}. Así, \emph{``quien quiera
    escapar de la incertidumbre de la fe, caerá en la incertidumbre de la
    incredulidad, que jamás podrá afirmar de forma cierta y definitiva que la fe
    no sea la verdad.''}. En definitiva: \emph{``nadie puede sustraerse al dilema
    del ser humano''}.

Culmina la reflexión de Ratzinger con una elocuente narración de Martin Buber:
\citalitlar{Un racionalista, un hombre muy entendido, fue un día a disputar
  con un Zaddik con la idea de destruir sus viejas pruebas en favor de la
  verdad de su fe. Cuando entró en su aposento, lo vio pasear por la
  habitación con un libro en las manos y sumido en profunda meditación. Ni
  siquiera se dio cuenta de que había
  llegado alguien. Por fin, lo miró de soslayo y le dijo: <<Quizá sea verdad>>.\\
  El hombre instruido intentó en vano conservar la serenidad: el Zaddik le
  parecía tan terrible, su frase le pareció tan tremenda, que empezaron a
  temblarle las piernas. El rabí Levi Jizchak se volvió hacia él, le miró fija
  y tranquilamente, y le dijo: <<Amigo mío, los grandes de la Tora, con los
  que has disputado, se han prodigado en palabras; tú te has echado a reír. Ni
  ellos ni yo podemos poner ni a Dios y ni a su Reino encima de la mesa. Pero
  piensa en esto: quizá sea verdad>> El racionalista movilizó todas sus
  fuerzas para contrarrestar el ataque; pero aquel <<quizá>>, que de vez en
  cuando retumbaba en sus oídos, oponía resistencia.}

He aquí una propuesta interesante; tanto para el creyente como para el
ilustrado este <<quizá sea verdad>> resuena como una fuerza inquietante y
potente, y esta experiencia común se convierte en punto de encuentro. Así
añade el futuro Benedicto XVI:
\citalitlar{Es ley fundamental del destino humano encontrar lo decisivo de su
  existencia en la perpetua rivalidad entre la duda y la fe, entre la
  impugnación y la certidumbre. La duda impide que ambos se encierren
  herméticamente en su yo y tiende al mismo tiempo un puente que los comunica.
  Impide a ambos que se cierren en sí mismos: al creyente lo acerca al que duda
  y al que duda lo lleva al creyente; para uno es participar en el destino del
  no creyente; para el otro la duda es la forma en la que la fe, a pesar de
  todo, subsiste en él como reto.}

\subsection{Análisis:}
\begin{itemize}
  \item ¿Qué servicio puede ofrecer la teología a nuestro pueblo puertorriqueño?
  \item ¿Qué retos te sugiere el diálogo de la fe en nuestro contexto?
\end{itemize}
\newpage

\section{Perseverar en el ejercicio de la actividad filosófica}
  \paragraph{Hemos considerado anteriormente la teología como servicio. Siguiendo
    la reflexión de ``Introducción al Cristianismo'' hemos planteado que el
    territorio de la duda que la fe supone es uno adecuado para la comunicación
    entre creyentes y no-creyentes. Es preciso ahora examinar la actividad que es
    preguntar sobre la verdad.\\\\}

  \subsection{¿Dónde te dirigirás en medio de esta ignorancia?}
  En el Parménides de Platón el filósofo discurre por un diálogo peculiar respecto
  del resto de su obra. La obra presenta a un joven Socrates en discusión con
  Parménides y Zenón, y son éstos últimos los que cuestionan y ofrecen consejo a
  los planteamientos del joven. En esta ocasión Socrates no hace las preguntas,
  sino que se le dirigen a él. En medio de la discusión se suscita el siguiente
  intercambio:

  \begin{adjustwidth}{1.2cm}{}

    \noindent\emph{PARMÉNIDES.}~---¿Qué partido tomarás con respecto á la filosofía; y a
    dónde te dirigirás en medio de esta ignorancia?

    \noindent\emph{SÓCRATES.}~---En este momento no lo sé.

    \noindent\emph{PARMÉNIDES.}~---En eso consiste, mi querido Sócrates, en que te
    atreves,
    antes de estar suficientemente ejercitado, a definir lo bello, lo justo, lo
    bueno, y las demás ideas. Ya, últimamente, te hice esta observación, oyéndote
    discutir aquí con mi querido Aristóteles. Es muy bello y hasta divino, sírvate
    de gobierno, ver el ardor con que te entregas a las indagaciones filosóficas;
    pero es preciso, mientras que eres joven, poner tu espíritu a prueba, y
    ejercitarte en lo que la multitud juzga inútil y llama una vana palabrería; y de
    no hacerlo así, se te escapará la verdad.

    \noindent\emph{SÓCRATES.}~---¿De qué clase de ejercicio hablas Parménides?

    \noindent[\ldots]

    \noindent\emph{PARMÉNIDES.}~---En una palabra, cualquiera que sea la cosa que
    supongas
    existiendo o no existiendo, o experimentando cualquiera otra modificación, debes
    indagar lo que la sucederá con relación a sí misma, con relación a cada una de
    las otras cosas que quieras considerar, o con relación a muchos o a todos los
    objetos; y después de esto, examinando a su vez las demás cosas, debes también
    indagar lo que las sucederá con relación a sí mismas, y con relación a cualquier
    otro objeto que quieras considerar, ya supongas que tales cosas existen o que no
    existen. Sólo procediendo de este modo, te ejercitarás de una manera completa y
    discernirás claramente la verdad.

    \noindent\emph{SÓCRATES.}~---Es un trabajo muy arduo el que me propones,
    Parménides; y no estoy seguro de comprenderlo bien. Pero ¿por qué no me
    desenvuelves tú alguna hipótesis, para darte mejor á entender?

  \end{adjustwidth}

  El ejercicio que Parménides sugiere a Sócrates puede ser descrito del siguiente
  modo: ``tomar sucesivamente cada idea, y suponiendo, primero, que existe,
  segundo, que no existe; examinar cuáles son las consecuencias de esta doble
  hipótesis, ya con respecto á la idea considerada en sí misma y con relación á
  las otras cosas, ya con respecto a las otras cosas consideradas en sí mismas y
  con relación a la idea. Es imposible que el espíritu no encuentre, en esta
  `gimnasia intelectual', la explicación verdadera de las cosas y de sus
  principios con más firmeza y rectitud.'' 

  \begin{itemize}
  \item ¿Es posible aplicar éste método a las verdades de la fe?
  \end{itemize}

\subsection{¿Puede ser verdadero el lenguaje teológico?}
En el 2015 la universidad de Viena organizó una exibición dedicada al `Círculo
de Viena' con el lema: ``Pensamiento exacto en tiempos dementes''. La exhibición
con tan sugerente `slogan' celebraba un fenómeno en el pensamiento filosófico de
comienzos del siglo pasado. Esta corriente de pensamiento surgió arraigada en
tres vertientes de la cultura filosófica dominante: el interés por llegar a un
univocismo semántico en los términos utilizados por las ciencias; la busqueda de
un rigor lógico-sintáctico en los sistemas científicos; finalmente, un deseo
frenético por la verificación empírica de las proposiciones veritativas.

La pretensíon era buscar una fundamentación sólida y suficiente para todas las
ramas de la ciencia. Esta corriente significó una renovación del positivismo y
empiricismos del siglo XVIII, también recibió importantes influencias del
desarrollo de la lógica ocurridos en la busqueda de la fundamentación de los
principios matemáticos. Esta dimensión lógica de la corriente desembocó en el
criterio principal de unificación de la ciencia: el análisis del
lenguaje.\footcite{anteo}

Para lograr la pretendida unificación de la ciencia, el Círculo de Viena vio la
necesidad de efectuar un análisis lógico del conocimiento científico partiendo
de su expresión lingüística. Este método de análisis fue el contexto que enmarcó
ataques a la posibilidad del lenguaje teológico como uno que comunique alguna
verdad.

A. J. Ayer lo describe de este modo: 
\citalitlar{Si la existencia de tal dios fuese probable, la proposición de que
  existiera sería una hipótesis empírica. Y, en este caso, sería posible deducir
  de ella, y de otras hipótesis científicas, ciertas proposiciones
  experienciales que no fuesen deducibles de esas otras hipótesis solas. Pero,
  en realidad, esto no es posible. [\ldots] Porque decir que ``Dios existe'' es
  realizar una expresión metafísica que no puede ser ni verdadera ni flasa. Y,
  según el mismo criterio, ninguna oración que pretenda describir la naturaleza
  de un dios trascendente puede poseer ninguna significación
  literal.\footcite{ayer}}

En la misma línea Anthony Flew planteó lo que llamó el `desafío falsacionista':
\emph{<<¿Qué tendría que ocurrir o que haber ocurrido para que constituyera una
  prueba en contra del amor o la existencia de Dios?>>} Si no se puede describir
una situación en la que ``Dios existe'' sea falsa entonces tal aserción es
factualmente no significativa. De este modo las proposiciones religiosas como
``Dios tiene un designio'', ``Dios creó el mundo'', ``Dios nos ama como un padre
a sus hijos'', no son falseables y, por tanto, no tienen contenido
empírico.\footcite{cyc}

Esta crítica hecha por el círculo de Viena no se suma al ``Dios a muerto de
Nietzsche'', sino que va más allá\ldots Lo que argumenta que ha muerto es la
misma palabra ``Dios''.\footcite{anteo}

Todo esto tendrá como consecuencia que para el neopositivismo el lenguaje
teológico sea sólo una proyección de sentimientos o deseos, y no el fruto de una
reflexión sobre algo real. Esta postura resulta ser claramente perniciosa, no
sólo por vaciar de contenido las proposiciones teológicas y metafísicas, sino
ademas por separar lo afectivo de lo racional. A modo de recreación de esta
unidad Zubiri diría:
\citalitlar{El sentir humano y la intelección no son dos actos numéricamente
  distintos, cada uno completo en su orden, sino que cosntituyen dos momentos
  de un solo acto de aprehensión sentiente de lo real: es la inteligencia
  sentiente\footcite{zubiri}}

\subsection{Acerca de lo que puede ser dicho}
Habiendo ofrecido un breve descripción del problema del valor proposicional del
lenguaje teológico, sera útil ahora deternenos a una de las afirmaciones que
está a la raiz de estas consideraciones sobre el lenguaje y luego examinaremos
una posible descripción de la fe en este cotnexto.

En el \emph{``Tractatus Logico-Philosophicus''} L. Wittgenstein plantea lo siguiente:
\citalitlar{6.53~El método correcto para la filosofía sería este. No decir nada
  excepto lo que pueda ser dicho, esto es, proposiciones de la ciencia natural,
  es decir, algo que no tiene nada que ver con la filosofía: y luego siempre,
  cuando alguien quiera decir algo metafísico, demostrarle que no ha logrado dar
  significado a ciertos signos en sus proposiciones. Este método sería
  insatisfactorio para la otra persona ---no tendría la impresión de que le
  estuviéramos enseñando filosofía--- pero este método sería el único
  estrictamente correcto.\footcite{tractatus}}

En \emph{``Investigaciones Filosóficas''} el mismo Wittgenstein ofrece este otro
planteamiento: 
\citalitlar{\S353~Preguntar sobre el tipo y la posibilidad de la verificación de
  una proposición es sólo una forma especial de la pregunta ¿Qué quieres decir?
  (`How do you mean?'). La respuesta es una contribución a la gramática de la
  proposición.\footcite{PI}}

\subsection{Fe como `creer a Dios'}
En Oscott College, el seminario de la Archidiócesis de Birmingham, se
  comenzaron a celebrar las conferencias llamadas `Wiseman Lectures'
  en 1971. Para estas lecciones ofrecidas anualmente en memoria de Nicolás
  Wiseman se invitaba un ponente que tratara algún tema relacionado con la
  filosofía de la religión o alguna materia en torno al
  ecumenísmo.\footcite[cf.~][p.~7]{wisemanlects}

  El 27 de octubre de 1975, para la quinta edición de las conferencias, Anscombe
  presentó una lección titulada simplemente `Faith'. Allí planteaba la
  siguiente cuestión:\citalitlar{Quiero decir qué puede ser entendido sobre la
  fe por alguien que no la tenga; alguien, incluso, que no necesariamente crea
  que Dios existe, pero que sea capaz de pensar cuidadosa y honestamente sobre
  ella. Bertrand Russell llamó a la fe ``certeza sin prueba''. Esto parece
  correcto. Ambrose Bierce tiene una definición en su `Devil's
  Dictionary': ``La actitud de la mente de uno que cree sin evidencia a uno que
  habla sin conocimiento cosas sin parangón''.} \citalitlar{¿Qué deberíamos
  pensar de esto?\footcite[p.~115]{faith}} 

¿Qué deberíamos pensar de las definiciones que dan Russell y Bierce? ¿Esto es
todo lo que se puede discernir sobre la fe desde una perspectiva no creyente?

Hubo una época en la que se vivió gran entusiasmo en la Iglesia por la
racionalidad de la fe. Este carácter racional de la fe, sin embargo, estaba
sujeto a los llamados `preámbulos' y el paso de éstos a la fe.

Estos preámbulos eran argumentaciones en cierto modo demostrativas que
ofrecían cierto fundamento racional a las proposiciones de la fe. Para
Anscombe sin embargo estos llamados preámbulos son construcciones ideales.
No en el sentido de que fueran la mejor manera del desarrollo del
pensamiento si ocurriera en un individuo, sino que más bien eran sueños
imaginados a partir de prejuicios sobre lo que debería de ser razonable en
sostener una creencia.

El modo correcto de designar estos llamados preámbulos, no es tal cosa, sino
que al menos parte de ellos, sería más apropiado llamarles `presuposiciones'.

Ahora bien, ¿qué significa que la fe tiene presuposiciones? En el uso moderno
`fe' tiende a significar `creencia religiosa' o `religión'. Se le llama
generalmente `fe', por ejemplo, a la creencia en la existencia de Dios. Sin
embargo \emph{en la tradición donde el concepto tiene su origen, `fe' es la
  forma breve de `fe divina' y significa `creer a Dios'\footcite{belief}.} De
esa manera fue usada la expresión, al menos por los pensadores cristianos. Según
este modo de hablar `fe' se distinguía como humana y divina. Fe humana era creer
a una persona humana, fe divina era creer a Dios. Dice la escritura: <<Abrahám
creyó a Dios y ésto se le contó como justicia.>>\footnote{Gn~15,6} De tal modo
que es llamado `padre de la fe'.\footnote{cfr.~Rm~4~y~Ga 3,7}
Al describir la fe, por tanto, nos referimos a la forma de la expresión creer a
\emph{x} que \emph{p}. Ahora, ¿qué se puede entender por presupuestos de la fe?

Supongamos que recibes una carta de un amigo, llamémosle Juan. En la carta Juan
te cuenta que su esposa ha muerto. Ahora, creer a Juan, creerle que su esposa ha
muerto tiene varias presuposiciones. Creyéndole a tu amigo, tu presupones que tu
amigo Juan existe, que esta carta es verdaderamente de su parte y que esta
comunicación es verdaderamente lo que esta carta te dice.

Estas tres convicciones son presuposiciones lógicas que tiene tu creencia de la
muerte de la esposa de Juan porque le crees a Juan. Estas tres convicciones son
tus creencias, no necesariamente reales, sino presupuestos lógicos.

Fe, como hemos dicho, es creer a Dios. Si los presupuestos son ciertos esto
sería entonces creer en la mejor situación posible a uno que habla con
conocimiento perfecto.

Consideremos el ejemplo de Abrahám. Abrahám tuvo fe porque creyó que su
descendencia sería tan numerosa como las estrellas porque creyó a Dios.

Los presupuestos lógicos de esta creencia de Abrahám son que Dios existe,
que el mensaje sobre su descendencia venía de Dios, que el mensaje dice
efectivamente eso que él entendió. Si estos presupuestos son ciertos
entonces Abrahám creyó apoyado en el mejor fundamento posible. Si sólo las
presuposiciones son dadas sería tonto pensar que Abrahám se hallaba en la
actitud de la mente de uno que ``cree sin evidencia a uno que dice sin
conocimiento cosas sin parangón''. Si las presuposiciones son dadas quedaría
refutado decir que Abrahám ``tenía certeza sin prueba''.

Ahora bien aquí la expresión no deja de ser sorprendente: <<creer a Dios>>. La
expresión misma es dificil de entender. 

El asunto de creer a álguien en sí mismo es bastante complejo, limitémonos aquí
ahora a afirmar que dados los presupuestos una persona puede llegar a estar en
la situación dónde surge la pregunta sobre si creer o dudar (suspender el juicio
acerca de) \emph{N.}. Sin confusiones por las preguntas que puedan surgir por
los presupuestos, podemos decir que creer a alguien (acerca de un caso
particular) es confiar en él para la verdad ---de ese caso particular.

Podemos considerar distintos ejemplos en los cuales puede llegarse a ver
claramente qué es que alguien crea a alguien. Ahora bien ¿puede verse con
claridad qué sería creer a Dios? ¿Puede algún experto, por la autoridad de su
conocimiento, informarme de la evidencia de que Dios ha hablado? El único uso de
sus conocimientos más bien serían para remover los obstaculos que pueda poner a
tal posibilidad.

Una creencia rabínica puede ayudar a ilustrar esto. Llamada `Bath Qol' o `hija
de la voz' que entiende que respecto de la creencia de que Dios está hablando, el
cómo se produce la voz no importa. Como cuando San Agustín escucha el grito del
niño: ``tolle, lege''; no tiene que suponer que la expresión no fue hecha en
medio de alguna conversación que no tiene nada que ver con él, pero la voz le
golpea directo al corazón y le hace actuar en obediencia a ella.

A esto habría que añadir que por escuchar a Dios no nos referimos al dios de
tal o cual culto. Definir a Dios como el objeto de tal o cual culto no nos
sirve, pues puede ofrecerse el honor pretendido para Dios a algo que no lo es.

Por Dios hemos de entender `el único y verdadero dios' el ateo tiene a Dios
entre los dioses que no existen, pues cree que nada es una deidad. Sin
embargo puede reconocer la identidad de `Dios' y `el único y verdadero
dios'.

\emph{Entonces podemos decir esto: la suposición de que alguien tiene fe
  es la suposición de que cree que algo ---sea una voz, o algo que se le ha
  enseñado--- viene como una palabra de Dios. Fe es entonces la creencia que da
  a esta palabra.\\Hasta aquí puede ser discernido por un no-creyente, sea que
  su actitud respecto de este fenómeno sea potencialmente una de reverencia o de
  hostilidad. El cristiano, sin embargo, añade que esta creencia en ocasiones es
  verdadera, y esta consecuente creencia es únicamente lo que \emph{él} llama
  fe.}

\end{document}
