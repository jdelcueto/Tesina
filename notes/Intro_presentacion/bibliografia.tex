\subsubsection{Bibliografía Primaria}
\noindent J. M. Torralba ofrece una recopilación de la Bibliografía en Anscombe en:\\
\emph{\href{http://www.unav.es/filosofia/jmtorralba/anscombe/G.E.M.\_Anscombe\_Bibliography.htm}{http://www.unav.es/filosofia/jmtorralba/anscombe/G.E.M.\_Anscombe\_Bibliography.htm}}

\noindent Escritos de Anscombe:



%\printbibliography[type=collection]

\printbibliography[keyword={anscombe}]

\begin{itemize}
    \item Colecciónes:
      \begin{itemize}
          \item Collected Philosophical Papers:
            \begin{itemize}
                \item \cite{collectedppI}
                \item \fullcite{collectedppII}
                \item \fullcite{collectedppIII}
            \end{itemize}

          \item St. Andrews Studies:
            \begin{itemize}
                \item \cite{hlae}
                \item \fullcite{fhg}
                \item \fullcite{ptow}
            \end{itemize}
      \end{itemize}
    \item Otros Escritos:
      \begin{itemize}
        \item \fullcite{intention}
        \item \fullcite{introtract}
      \end{itemize}
    \item Conferencias:
      \begin{itemize}
        \item \fullcite{torralba}
      \end{itemize}

\end{itemize}

\noindent Estudios sobre el pensamiento de Anscombe:

\begin{itemize}
    \item \fullcite{lcateich}
    \item \fullcite{philteich}
\end{itemize}

\subsubsection{Bibliografía Secundaria}
\subparagraph{Comentarios y escritos sobre Anscombe y su obra}
\subparagraph{Escritos sobre Wittgenstein}
\subparagraph{Escritos sobre filosofía analítica}

