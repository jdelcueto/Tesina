\documentclass[12pt]{article}

\usepackage{hyperref}
\usepackage[T1]{fontenc}
\usepackage[utf8]{inputenc}
\usepackage[scaled]{uarial}

\usepackage[spanish]{babel}

\renewcommand*{\familydefault}{\sfdefault}

\setcounter{secnumdepth}{3}

\setlength{\parindent}{1.25cm}
\setlength{\parskip}{0.39cm}

\usepackage{marginnote,letltxmacro}
\renewcommand*{\marginfont}{\footnotesize}
\LetLtxMacro\mn\marginnote

\usepackage[top=1.5cm, bottom=1.5cm, outer=4.5cm, inner=2cm, heightrounded, marginparwidth=3.5cm, marginparsep=.5cm]{geometry}



\usepackage[
backend=biber,
style=apa,
citestyle=verbose-ibid,
refsection=section
]{biblatex}

\DeclareLanguageMapping{spanish}{spanish-apa}

\setcounter{smartand}{0}
\DefineBibliographyStrings{spanish}{%
    andothers = {et al.},
}

\addbibresource{biblio.bib}
\addbibresource{refs.bib}
\title{Presentación General de la tesina.}
\author{Joel Del Cueto Santiago}
\date{mayo, 2015}

\interfootnotelinepenalty=10000


\begin{document}
%\reversemarginpar

\maketitle

\part*{Título}
\paragraph{\noindent La Categoría del Testimonio en el pensamiento de Elizabeth Anscombe.\\
\emph{Valoración y crítica en perspectiva teológico-fundamental.}}

\part*{Introducción}

Esta introducción pretende exponer brevemente algunos elementos básicos del trabajo. Sin perseguir ser exhaustivo en los temas presentados, quisiera que desarrollar esta presentación como una hoja de ruta para el recorrido de la investigación. 

En la sección dedicada a las motivaciones personales ofrezco las razones detrás del interés en realizar una investigación sobre la categoría del testimonio en el ámbito de la filosofía analítica, específicamente en el pensamiento de G. E. M. Anscombe. Con esto quisiera ofrecer una justificación para este estudio y definir el marco de la reflexión posterior.

Presento en segundo lugar el objetivo de la tesina desde el cual queda orientada la perspectiva del trabajo y la cuestión principal que pretende abordar.

Finalmente enumero las fuentes básicas catalogadas según los temas principales que se trabajarán en la investigación.

\section{Motivación Personal}

\subsection{Categoría del Testimonio}

El interés por la categoría del testimonio tiene su origen en mi propia \textbf{experiencia creyente} y en la vivencia de la revelación divina dentro de la Iglesia. La \textbf{teología de la revelación del evangelista San Juan}, en una cierta relación de circularidad con mi vida de fe, ha sido iluminadora para comprender y profundizar en estas experiencias.

El testimonio es una categoría fundamental que forma parte del lenguaje que hemos recibido del \textbf{Concilio Vaticano II}, como tal, es un elemento importante en la interpretación del contexto histórico en el que vivimos en la Iglesia y de su misión en nuestro mundo. En consonancia con esto interpretamos la renovada llamada que nos hace el Concilio a ser testigos de Cristo en \textbf{diálogo con las múltiples culturas y religiones} que conforman la sociedad contemporánea como una valiosa enseñanza del Espíritu Santo.
\footnote{ 
Cfr. Encuentro con los Obispos que Participaron en el Concilio Vaticano II y un Grupo de Presidentes de Conferencias Episcopales, Discurso del Santo Padre Benedicto XVI en la Sala Clementina el Viernes 12 de octubre de 2012:
\emph{ ``El Concilio fue un tiempo de gracia en que el Espíritu Santo nos enseñó que la Iglesia, en su camino en la historia, debe siempre hablar al hombre contemporáneo, pero esto sólo puede ocurrir por la fuerza de aquellos que tienen raíces profundas en Dios, se dejan guiar por Él y viven con pureza la propia fe; no viene de quien se adapta al momento que pasa, de quien escoge el camino más cómodo. El Concilio lo tenía bien claro, cuando en la constitución dogmática sobre la Iglesia Lumen Gentium, en el número 49, afirmó que todos en la Iglesia están llamados a la santidad según las palabras del Apóstol Pablo: <<Esta es la voluntad de Dios: vuestra santificación>> (1 Tes 4, 3). La santidad muestra el verdadero rostro de la Iglesia, hace entrar el <<hoy>> eterno de Dios en el <<hoy>> de nuestra vida, en el <<hoy>> del hombre de nuestra época.''
}
}

\subsubsection{El testimonio como experiencia humana y dimensión de la vida cristiana}

El testimonio, en cuanto analogía, hace referencia a la experiencia humana para expresar una de las relaciones fundamentales entre el hombre y Dios. Así queda expresado por R. Latourelle:

{\emph{``El testimonio pertenece al grupo de analogías empleadas por la Escritura para introducir al hombre en las riquezas del misterio divino, por ejemplo las categorías de alianza, de palabra, de paternidad y de filiación. Si la revelación misma se apoya en la experiencia humana del testimonio para expresar una de las relaciones fundamentales que unen al hombre con Dios, la reflexión teológica se encuentra entonces autorizada a explorar los datos de esta experiencia.''
}
\footcite[p.~1523]{dicctf}

Aquí se encuentra una linea de interés importante para esta investigación; indagar \mn{\textbf{Linea de estudio}\\\emph{testimonio humano analogía del misterio de la revelación}}en la experiencia humana del testimonio tiene el interés teológico de iluminar el misterio que ésta representa. 

El misterio en el que quedamos introducidos por el testimonio es el de la Revelación de Dios. Por medio del testimonio conocemos a Dios y por medio del testimonio se transmite este conocimiento. Esto le confiere a la experiencia del conocimiento de la verdad de Dios las peculiaridades propias de la dinámica de la fe y de la <<lógica del testimonio>>, como afirma F. Conesa:

\emph{
``Es sabido que, especialmente en el Nuevo Testamento, se acentúa la vinculación entre la fe y el conocimiento de Dios. Esta íntima relación es también experimentada por el creyente, que entiende y vive su fe como un modo singular de conocer a Dios. Así lo percibe de modo singular nuestro clásico castellano en su ``Cantar del alma que se huelga de conoscer a Dios por la fe'': <<Qué bien sé yo ---afirma--- la fonte que mana y corre, aunque es de noche>>.''}
\footcite[p.~15]{cyc}


\emph{
``La reflexión sobre el aspecto o dimensión proposicional de la fe pondrá de relieve que no se trata de algo externo al mismo acto de creer. En la aceptación de las proposiciones de fe opera lo que denominamos <<lógica del testimonio>>: aceptamos las verdades porque las ha dicho aquel en quien ponemos razonablemente la confianza. Creemos algo porque creemos en alguien y le creemos. \textquestiondown Cuál es el valor epistemológico de este asentimiento?''}
\footcite[p.483]{feylogicaconesa}

Esta será otra linea importante en el desarrollo del trabajo; examinar el valor epistemológico del testimonio \mn{\textbf{Linea de estudio}\\\emph{Valor epistemológico del conocimiento por testimonio}}en cuanto que es medio de conocimiento de Dios y de transmisión de su verdad. En esto tenemos en cuenta la distinción que hay entre el conocimiento de Dios que ha dispuesto revelarse a Sí mismo por medio de Cristo, Verbo encarnado\footnote{cfr. DV 2} y el conocimiento de las verdades o proposiciones de fe por medio de testigos. 

\subsubsection{La Teología de la revelación del evangelista San Juan}

Al detenernos en la teología joánica ofrecemos como clave fundamental para el estudio la centralidad de Cristo en la dinámica del testimonio. \mn{\textbf{Clave hermenéutica}\\\emph{Centralidad de Cristo en la dinámica del testimonio de Dios.}}Latourelle afirma:

``En san Juan el testimonio culmina como narración, como confesión, como compromiso y como interiorización. El testigo es Cristo (Ap 1, 5; 3, 14); y para Cristo, atestiguar equivale a manifestar al Padre, a revelar al Padre.''
\footcite[p.~1529]{dicctf}

Al respecto encontramos las siguientes palabras de D. Antiseri que al tratar el tema de la lógica del testimonio dice:

``Ahora bien, en el cristianismo el primer testigo es Cristo, y es a Cristo a quien, de testimonio en testimonio, tendremos que remitirnos. Su persona histórica es ---según la expresión de R. Guardini--- la que, sin resolverse en la historia, constituye la esencia del cristianismo.''
\footcite[p.~168]{antiseri}

Latourelle continúa:

``Cristo es, por tanto, el testigo absoluto, el que lleva en sí mismo la garantía de su testimonio. El hombre, sin embargo, no sería capaz de acoger por la fe este testimonio del absoluto, manifestado en la carne y el lenguaje de Jesús, sin una atracción interior (Jn 6, 44), que es un don del Padre y un testimonio del Espíritu (1Jn 5, 9-10).''
\footcite[p.~1530]{dicctf}

La centralidad de Cristo en el misterio de la Revelación de Dios queda recogida en la imagen de la Palabra encarnada que Juan presenta. En el trasfondo de la tesina estará presente esta imagen; \mn{\textbf{Clave hermenéutica}\\\emph{Cristo es la Palabra a la que se refiere todo testimonio de Dios.}}la Palabra encarnada que es Cristo en donde se revela el rostro del Padre. Es Palabra que se acoge, se interioriza y se comunica.\footnote{
cfr. \cite[p.~1530]{dicctf}: \emph{``...el que cree en Cristo tiene dentro de sí el testimonio de Dios. El testimonio que el creyente posee en sí mismo es el testimonio que el espíritu de del Hijo. Si el testimonio se interioriza es siempre en relación con la palabra de Cristo que exterioriza la intimidad de su diálogo con el Padre.''} El comentario de San Agustín a Jn 1, 1--5 ofrece una reflexión significativa al respecto:
\emph{``A diario, cuando hablamos, las palabras se nos quedan en nada. A fuerza de sonar palabras y desaparecer, su valor se degrada y no nos parecen sino meras palabras. Pero hay en el hombre una palabra que permanece dentro, porque el sonido sale de la boca. Y hay otra palabra que realmente se pronuncia con el espíritu, lo que entiendes por medio del sonido, no el sonido mismo. Cuando yo digo <<Dios>>, pronuncio una palabra. Bien breve es lo que he pronunciado: cuatro letras y una sílaba. \textquestiondown Acaso Dios es en total una sílaba de cuatro letras? \textquestiondown O quizá cuanto menos vale este sonido, tanto más precioso es lo que por él entendemos? \textquestiondown Qué ocurre en mi interior cuando yo digo <<Dios>>? He pensado en un ser supremo, que trasciende toda criatura mudable, carnal y animal. Y si yo te preguntase: <<\textquestiondown Dios es mudable o inmutable?>>, inmediatamente responderías: <<Lejos de mí creer o pensar en Dios como mudable: Dios es inmutable>>. Tu alma, aunque pequeña, quizá carnal todavía, no pudo menos de responderme que Dios en inmutable, puesto que toda criatura es mudable. \textquestiondown De dónde te pudo venir la chispa que te ha iluminado este misterio, para responderme sin titubear que Dios es inmutable? \textquestiondown Qué hay en tu interior, cuando piensas en una sustancia viva, eterna, omnipotente, infinita, presente toda ella en todas partes, y no contenida por límites algunos? Cuando esto piensas, es la Palabra de Dios lo que hay en tu interior. \textquestiondown Es esto aquel sonido que consta de una sílaba y cuatro letras? Todo lo que se pronuncia y desaparece son sonidos, sílabas. La palabra que suena es la que pasa; pero la significada por el sonido está en el pensamiento de quien la dijo, permanece en la inteligencia de quien la ha oído, aunque desaparezcan las palabras.''}(\cite[n.~8]{aguscomentjn})
}
El mismo Juan ha conocido a Cristo y anuncia el testimonio de lo que ha visto y oído para que los que lo acojan estén en comunión con el Padre y el Hijo; ofrece un testimonio de los signos de Cristo para que crean en Él y creyendo tengan vida en su nombre.\footnote{cfr. Jn 20, 30--31 y 1Jn 1,1--4. Además \cite[1530]{dicctf}} 

Juan ofrece el testimonio de lo que ha visto y oído de la Palabra de la Vida que se ha manifestado y al mismo tiempo este anuncio lo ofrece para aquellos que son dichosos porque <<no han visto y han creído>>(cfr. Jn 20, 29). Aquel que cree en el testimonio manifiesto conoce a Cristo y confiesa su misterio: ``Señor mío y Dios mío''(cfr. Jn 20, 28).\footnote{cfr. San Gregorio, In Evang. hom. 26: \emph{``Pero como diga el Apóstol que la fe es la sustancia de cosas que se esperan (Heb 11,1), pero que no se ven evidentemente, se deduce que, en las que están a la vista, no cabe fe, sino conocimiento. Si, pues, Tomás vio y tocó, \textquestiondown por qué se le dice ``Porque me viste, creíste''? Pero una cosa vio y otra creyó; vio al hombre, y confesó a Dios. Mucho alegra lo que sigue: ``Bienaventurados los que no vieron y creyeron''. En esta sentencia estamos especialmente comprendidos, porque Aquel a quien no hemos visto en carne lo vemos por la fe, si la acompañamos con las obras, pues aquel cree verdaderamente que ejecuta obrando lo que cree.''}
}

Esta dinámica de la Palabra en la Revelación según Juan la presenta nos resulta iluminadora para el estudio sobre el testimonio y por tanto será una clave importante en el análisis que se desarrollará en la investigación.

\subsubsection{Teología de la Revelación y misión de la Iglesia en el Concilio Vaticano II}

El estudio se desarrolla dentro de un contexto filosófico y uno teológico. El contexto teológico nos lo ofrece la teología desarrollada a partir del Concilio Vaticano II. \mn{\textbf{Contexto teológico}\\\emph{Importancia de la categoría del testimonio en los documentos del Concilio como categoría teológica y pastoral.}}Valoramos la categoría del testimonio desde su relevancia teológica y pastoral en el Concilio. 

Como categoría teológica aparece como tema recurrente en los documentos conciliares y pone de manifiesto el cambio de perspectivas desde el Vaticano I. Al respecto Latourelle comenta:

``Lo que el Vaticano I entendía por el signo de la Iglesia, se concentra ahora en la categoría de testimonio. Una vez percibida esta trasposición, se constata que el tema del testimonio es uno de los temas principales y privilegiados del Vaticano II. Como un `leitmotiv', aparece en todas las constituciones y en todos los decretos. A los ojos del concilio, atestiguar significa acreditar el evangelio como verdad y salvación del hombre mediante una vida conforme con el evangelio.''\footcite[p.~1532]{dicctf}

Como categoría pastoral \mn{\textbf{Objetivo Pastoral}\\\emph{El testimonio es lenguaje de diálogo con el hombre contemporáneo y misión del pueblo de Dios}}el testimonio se refiere a la vida comprometida de los que forman parte del pueblo de Dios, la cual es el signo de salvación para este mundo. Sobre esto afirma R. Pellitero:

``El Vaticano II redescubrió el enorme valor evangelizador del testimonio cristiano, aún el más sencillo y cotidiano. Sus grandes afirmaciones están en la Constitución Lumen gentium y el Decreto Ad gentes. Entre los documentos postconciliares que se ocupan del tema destaca la Exhortación Evangelii Nuntiandi.''\footcite[pp.~378--379]{ftcpellitero}

El concilio manifiesta su preocupación de hablar al hombre del siglo XX, que rechaza un tipo de santidad platónica y abstracta, y ofrece como respuesta el lenguaje del testimonio de la vida realmente comprometida de los cristianos, de la santidad vivida como compromiso total al servicio de Cristo.\footnote{cfr. \cite[p.~1532 y 1533]{dicctf}} 

El testimonio aparece como misión de toda la la Iglesia; los obispos ofrecen al mundo el rostro de la Iglesia con su trato y trabajo pastoral\footnote{cfr. GS 43}, los presbíteros, creciendo en el amor por el desempeño de su oficio, han de ser un vivo testimonio de Dios\footnote{cfr. LG 41}, los fieles han de dar testimonio de verdad como testigos de la resurrección\footnote{cfr. LG 28 y LG 38}, los religiosos deben ofrecer un testimonio sostenido por la integridad de la fe, por la caridad y el amor a la cruz y la esperanza de la gloria futura\footnote{PC 25}, los profesores han de dar testimonio tanto con su vida como con su doctrina\footnote{cfr. GE 8}, los misioneros han de ofrecer testimonio con una vida enteramente evangélica, con paciencia, longanimidad, suavidad, caridad sincera, y si es necesario hasta con la propia sangre.\footnote{cfr. AG 24}

El afán de hablar con el hombre contemporáneo no deja de ser apremiante, en este estudio se examina el testimonio desde esta preocupación. 

\subsubsection{El diálogo con las culturas y las religiones}
Desde el afán del Concilio Vaticano II se ha presentado un esbozo breve del interés por estudiar el testimonio como la modalidad de hablar con la sociedad actual, sobre este tema quisiera detallar algunas consideraciones metodológicas adicionales.

En el recorrido por la obra de Anscombe servirá para orientarnos lo sugerido en la Reflexión ``Conocer la verdad a través de testigos'' donde, al abordar la empresa del cristianismo de ofrecer el testimonio del Hecho de Cristo como verdad de salvación razonable y creíble en esta circunstancia histórica, se propone:

``Si queremos mostrar que dicha empresa es razonable, por una parte debemos recuperar la legitimidad de la pregunta filosófica y religiosa sobre la verdad... en segundo lugar, se tiene que aclarar la naturaleza original del testimonio cristiano; por último, se tienen que poder demostrar que el testimonio es un modo adecuado de conocer y de transmitir la verdad.''\footcite[p.~267]{pradesmulticr}

Estas tres cuestiones servirán de guía \mn{\textbf{Cuestiones Guía}\\\emph{Pregunta sobre la verdad, naturaleza del testimonio cristiano, el testimonio como modo adecuado de conocer la verdad.}}en el acercamiento al pensamiento de Anscombe. Junto a estos tres planteamientos servirá de orientación el estudio realizado por C. A. J. Coady en `Testimony'. El esquema de su investigación ofrece a esta investigación una buena estructura.  

En el recorrido por el pensamiento de Anscombe el primer tema a tratar será la pregunta sobre la verdad. Seguidamente se tratará el tema de la fe. Ambos temas servirán de base al tratar el resto de sus reflexiones.

Al abordar la pregunta sobre la validez del testimonio en la transmisión y conocimiento de la verdad servirá de guía el estudio de Coady; el análisis del testimonio atenderá primero su estado general y su centralidad y fiabilidad como medio de conocimiento. Este análisis tendrá como guía el que realiza Coady, como él mismo lo describe:

\mn{\textbf{Cuestiones Guía}\\\emph{Examen de la relevancia del testimonio en nuestras vidas, su lugar en el panorama epistemológico, la tradición del debate sobre ese lugar, qué se puede decir para defender la extensa relevancia que se ha afirmado sobre nuestra confianza en el testimonio}}``We have been concerned to examine the broad significance of testimony in our lives, its place in our epistemological landscape, the tradition of debate about that place, and finally, what can be said to vindicate the extensive significance we have claimed for our reliance upon testimony.''\footcite[p.~175]{testcoady}

Aunque el trabajo de Coady sirve de guía, esta investigación no se adhiere estrictamente al mismo desarrollo que él. El motivo de la elección del trabajo de Coady radica en su relación con Anscombe y la atención que le presta a las afirmaciones de la filosofía analítica. El `epistemological landscape' que examina coincide con el contexto en el que trabaja Anscombe.
 
Después de atender las cuestiones generales, la reflexión culminará con las interrogantes sobre tipos específicos de testimonio. Estos últimos interrogantes orientarán hacia el tema de la naturaleza del testimonio cristiano.

El interés expresado por Coady como `to vindicate the extensive significance we have claimed for our reliance upon testimony'\footcite{testcoady}, va en la linea de mostrar que nuestra confianza puesta en el testimonio del Hecho de Cristo es razonable\footnote{cfr. \cite[p.~267]{pradesmulticr}}. Esta finalidad la comparte esta tesina. El objetivo de la investigación es el diálogo.

\subsubsection{Wittgenstein y la filosofía analítica}

La tesina está centrada en el pensamiento de Elizabeth Anscombe, su relación con Wittgenstein y la filosofía analítica\mn{\textbf{Punto de Interés}\\\emph{Filosofía analítica en tanto que es la línea de pensamiento a la que Anscombe pertenece.}} requiere que esta investigación tenga en cuenta algunas consideraciones básicas tanto del Profesor como de la escuela. Además de la relación con Anscombe es útil reconocer otros puntos de interés.

Para el concepto de filosofía analítica nos adscribimos a la definición que ofrece Conesa:

``Por filosofía analítica entiendo aquel enfoque de los problemas filosóficos que pone un énfasis en el análisis lógico y conceptual como método privilegiado \mbox{---aunque} no \mbox{exclusivo---} para resolver estos problemas''\footcite[p.~16]{cyc}

La filosofía analítica nos ofrece un ámbito adecuado para el estudio del testimonio, como afirma el mismo Conesa:

``En el ámbito de la filosofía analítica ---y especialmente por la influencia de Thomas Reid y Ludwig Wittgenstein--- asistimos a una revalorización del testimonio como forma de saber. Formulado en su forma más simple, se viene a decir que <<si alguien sabe que `p' y dice que `p', entonces quien lo escucha también sabe que `p'>>.''\footcite[p.~487]{feylogicaconesa}

Esta revalorización del testimonio en la filosofía analítica explica el interés de ubicar esta investigación dentro de esta escuela. 

Las aportaciones de Wittgenstein en el campo de la filosofía analítica con el Tractatus y más tarde con las Investigaciones Filosóficas son fundamentales. Su influencia en el pensamiento de Anscombe es importante como afirma Teichmann\mn{\textbf{Punto de Interés}\\\emph{Wittgenstein en tanto que está en el trasfondo del pensamiento de Anscombe y es una influencia en su trabajo}}:

``Elizabeth Anscombe's name will be known to many as that of the English translator of a good number of Wittgenstein's Writings, in particular of the summation of his later philosophy, th ePhilosophical Investigations.[...]And it is probably the influence of Wittgenstein's later philosophy that most readers of Anscombe will detect, rather than of his early philosophy. But her most sustained direct discussion of Wittgenstein is of course to be found in An Introduction to Wittgenstein's Tractatus; and it is interesting how themes from the Tractatus crop up in various guises in her articles, troughout her career.''\footcite[p.~191]{teichmann}

Sin embargo, además de su papel en la filosofía analítica y en el pensamiento de Anscombe nos interesan algunas afirmaciones suyas recogidas en reflexiones específicas de Elizabeth. Un ejemplo de esto es ``Mysticism and Solipsism'' en ``An Introduction to Wittgenstein's Tractatus''.

En la investigación se tendrá en cuenta la figura y pensamiento de Wittgenstein en cuanto que es examinado en las reflexiones de Anscombe y es una clara influencia en su trabajo.

\subsection{G. E. M. Anscombe}

Francisco Conesa ubica a Anscombe \mn{\textbf{Presupuesto a examinar}\\\emph{Anscombe entiende la fe como `saber por testimonio'.}}dentro de ``aquellos autores que entienden la fe primordialmente como un saber por testimonio''\footcite[p.~84]{cyc}, esto la sitúa en el centro de nuestra investigación. 

Anscombe se convirtió a la fe católica en su juventud temprana y esta fe juega un papel importantísimo en su vida, pensamiento y filosofía. Este dato es llamativo cuando se considera su estrecha relación con Wittgenstein y la filosofía analítica, ella misma, sin embargo explica:

``Analytical philosophy is more characterized by styles of argument and investigation than by doctrinal content. It is thus possible for people of widely different beliefs to be practitioners of this sort of philosophy. It ought not to surprise anyone that a seriously beleiving Catholic Christian should also be an analytical philosopher.''\footcite[p.~66]{opinionsanscombe}

Una escena de su vida puede resumir el interés por su pensamiento en este estudio. Su hija, Mary Geach, nos narra:

``She told me how one day ---I think it was as an undergraduate--- she had come across a passage in Russell to the effect that an argument from the facts about the world to the existence of God could not be valid, as one could not deduce a necessary conclusion from a contingent premiss. She had not at the time been able to see what was wrong with the notion that necessities can only be deduced from necessities, but she had known that to deny the possibility of moving by reason from the facts about the world to a knowledge of existence of God was to deny a doctrine defined as of faith by an ecumenical council. She went then to church and made an act of faith [...]. She realized later that of course one can derive necessary conclusions from contingent premises. If she had relied on her own understanding, she would have lost her faith for a falsehood.''

La actitud de Anscombe hacia la Iglesia y el misterio de la Revelación tanto como su lugar dentro de la filosofía analítica son muy relevantes para el tipo de reflexión teológica que quisiéramos desarrollar. Más aún son relevantes para una reflexión sobre el testimonio en nuestro contexto contemporáneo.

\subsection{Perspectiva Teológico-Fundamental}

Añadimos unas últimas consideraciones sobre la elección de la perspectiva teológico-fundamental del trabajo. 

La reflexión ``La Identidad de la Teología Fundamental'' de Salvador Pié-Ninot nos da una buena pauta sobre la misión de la Teología Fundamental y, en consecuencia, sobre la perspectiva que ha de tener esta investigación. 

Al exponer la nueva imagen de la teología fundamental que se va desarrollando en la etapa entre 1980 y 1998 observa que ``esta nueva imagen se puede tipificar en torno a dos grandes modelos con sus aspectos decisivos: el epistemológico, que proviene de la concepción subyacente de la credibilidad, y el sistemático, que manifiesta la estructuración y contenido de la Teología Fundamental en esta etapa''\footcite[p.~29]{ninotTF}.

Sobre la epistemología comenta:
``La renovación de la TF va muy ligada a las componentes epistemológicas ``humanas'' del acto de fe: en definitiva la función de la credibilidad y la de los signos de ella.''\footcite[p.~31]{ninotTF}

En cuanto al modelo sistemático (monstratio religiosa, christiana y catholica) al abordar la monstratio catholica aporta su propia reflexión:

``el tratado sobre la Iglesia, ni es sólo la tercera monstratio teológico fundamental, ni es sólo el máximo signo de revelación como Cristo-en-la-Iglesia, sino que además es el marco englobante y significativo de toda la Teología Fundamental. Y esto a partir de la categoría testimonio en su doble función: la fundacional-hermenéutica ``ad-intra'' y la apologética-misionera ``ad-extra''. ''

Añade que esta categoría conlleva una mutua circularidad entre la dimensión externa del testimonio eclesial, la dimensión interiorizada que es el testimonio creyente vivido y la dimensión interior e interiorizada que es el testimonio constante del Espíritu. De esta circularidad emerge la función del testimonio eclesial como camino de credibilidad que es invitación externa e interna a la vez. \footnote{cfr. \cite[p.~40]{ninotTF}}

Al llegar a la propuesta para la teología fundamental que se encuentra en la Fides et Ratio distingue tres ejes importantes planteados para esta disciplina en el n. 67\footcite[p.~49]{ninotTF}:\\1. Disciplina que da razón de la fe (cf. 1 Pe 3, 15).\\
2. Justifica y explicita la relación entre la fe y la reflexión filosófica.\\
3. Estudia la Revelación y su credibilidad, con el acto de fe.

Finalmente ofrece su propuesta:

``La teología fundamental, ``como disciplina del Dar razón de la fe'', tiene como identidad: fundar y justificar la pretensión de verdad de la revelación cristiana como propuesta sensata de credibilidad.''\footcite[p.~72]{ninotTF}

Desde esta identidad denomina dos dimensiones\mn{\textbf{Perspectiva Teológico-Fundamental}\\\emph{Valoración en clave dogmático-fundacional y en clave apologético-misionera}}: la tarea dogmático-fundacional y la tarea apologético-misionera. De este modo ofrece una concepción de la Teología Fundamental que integra una doble tarea y se orienta por un lado hacia una `martiría' como expresión de su dimensión apologética y por otro lado se muestra inteligente como manifestación de su dimensión fundante en la perspectiva de la esperanza cristiana.\footnote{cfr. \cite[p.~72]{ninotTF}}

Esta doble tarea es la que que asume esta investigación al pretender orientarse desde una perspectiva teológico-fundamental. La valoración de la categoría del testimonio en el pensamiento de Elizabeth Anscombe se realizará en clave de ``teología fundante'' que busca arrojar luz sobre el testimonio que es componente epistemológico en la experiencia humana y categoría teológica con una función fundacional-hermenéutica. Es también una valoración en clave de ``teología apologética'' que busca describir la categoría del testimonio en su función apologética-misionera y estudiar su valor como elemento de credibilidad de la Revelación.


\printbibliography


\section{Objetivo de la Tesina}

\emph{Dada la valoración de la categoría del testimonio en la escritura, la experiencia creyente y el Concilio Vaticano II.}
\begin{itemize}
  \item Ofrecer una reflexión sobre la categoría del testimonio como un camino de conocimiento de la verdad de Dios en clave fundacional-hermenéutica y apologética-misionera.
  \item Ofrecer una reflexión fundada en la aportación de G. E. M. Anscombe sobre la categoría teológica del testimonio.
\end{itemize}

\section{Fuentes básicas y bibliografía elemental que se van a consultar}

\subsubsection{Bibliografía Primaria}
\noindent J. M. Torralba ofrece una recopilación de la Bibliografía en Anscombe en:\\
\emph{\href{http://www.unav.es/filosofia/jmtorralba/anscombe/G.E.M.\_Anscombe\_Bibliography.htm}{http://www.unav.es/filosofia/jmtorralba/anscombe/G.E.M.\_Anscombe\_Bibliography.htm}}

\noindent Escritos de Anscombe:

\begin{itemize}
    \item Colecciónes:
      \begin{itemize}
          \item Collected Philosophical Papers:
            \begin{itemize}
                \item \fullcite{collectedppI}
                \item \fullcite{collectedppII}
                \item \fullcite{collectedppIII}
            \end{itemize}

          \item St. Andrews Studies:
            \begin{itemize}
                \item \fullcite{hlae}
                \item \fullcite{fhg}
                \item \fullcite{ptow}
            \end{itemize}
      \end{itemize}
    \item Otros Escritos:
      \begin{itemize}
        \item \fullcite{intention}
        \item \fullcite{introtract}
      \end{itemize}
    \item Conferencias:
      \begin{itemize}
        \item \fullcite{torralba}
      \end{itemize}

\end{itemize}

\noindent Estudios sobre el pensamiento de Anscombe:

\begin{itemize}
    \item \fullcite{lcateich}
    \item \fullcite{philteich}
\end{itemize}

\subsubsection{Bibliografía Secundaria}
\subparagraph{Comentarios y escritos sobre Anscombe y su obra}
\subparagraph{Escritos sobre Wittgenstein}
\subparagraph{Escritos sobre filosofía analítica}



\end{document}
