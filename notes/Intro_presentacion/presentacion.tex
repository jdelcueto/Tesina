
\documentclass[11pt]{article}

\usepackage[spanish]{babel}
\usepackage{hyperref}

\usepackage[T1]{fontenc}
\usepackage[scaled]{helvet}
\usepackage[utf8]{inputenc}
\renewcommand*{\familydefault}{\sfdefault}
%\usepackage{fontspec,xltxtra,xunicode} 
%\defaultfontfeatures{Mapping=tex-text} 

%\setmainfont[
%  Ligatures=TeX,
%  BoldFont=Arial Unicode MS,
%  BoldFeatures={FakeBold=3},
%  ItalicFont=Arial Unicode MS,
%  BoldItalicFont=Arial Unicode MS,
%  BoldItalicFeatures={FakeBold=3},
%  AutoFakeSlant=0.25,
%  Mapping=tex-text,
%  Scale=1.0
%]{Arial Unicode MS}

%\newcommand\fauxsc[1]{\fauxschelper#1 \relax\relax}
%\def\fauxschelper#1 #2\relax{%
%  \fauxschelphelp#1\relax\relax%
%  \if\relax#2\relax\else\ \fauxschelper#2\relax\fi%
%}
%\def\Hscale{.83}\def\Vscale{.72}\def\Cscale{1.00}
%\def\fauxschelphelp#1#2\relax{%
%  \ifnum`#1>``\ifnum`#1<`\{\scalebox{\Hscale}[\Vscale]{\uppercase{#1}}\else%
%    \scalebox{\Cscale}[1]{#1}\fi\else\scalebox{\Cscale}[1]{#1}\fi%
%  \ifx\relax#2\relax\else\fauxschelphelp#2\relax\fi}

%\renewcommand{\textsc}[1]{\fauxsc{#1}}


\usepackage[top=2cm, bottom=2cm, outer=3cm, inner=3cm, heightrounded, marginparwidth=3.5cm, marginparsep=.5cm]{geometry}

\setcounter{secnumdepth}{3}
\setlength{\parindent}{.8cm}
\setlength{\parskip}{0.39cm}

\usepackage{tabto}
\NumTabs{12}

\usepackage{enumitem}

\usepackage{hyphenat}

\usepackage{setspace}

\usepackage[raggedright]{titlesec}

\usepackage{marginnote,letltxmacro}
\renewcommand*{\marginfont}{\footnotesize}
\LetLtxMacro\mn\marginnote

\interfootnotelinepenalty=10000

\usepackage[
backend=biber,
bibstyle=authortitle-custom,
citestyle=verbose-ibid-custom,
refsection=section,
useprefix=true,
block=none,
firstinits=true,
dashed=false,
labeldate=false,
sorting=nyvt,
]{biblatex}

\DefineBibliographyStrings{spanish}{%
    andothers = {et al.},
    in = {\lowercase{e}n:},
    editor = {(\lowercase{e}d.),},
    editors = {(\lowercase{e}ds.),},
}

\addbibresource{biblio.bib}
\addbibresource{refs.bib}

\title{Presentación General de la tesina.}
\author{Joel Del Cueto Santiago}
%\date{mayo, 2015}

\begin{document}

\maketitle

\part*{Título}
\paragraph{\noindent La Categoría del Testimonio en el pensamiento de Elizabeth Anscombe.\\
\emph{Valoración y crítica en perspectiva teológico-fundamental.}}

\part*{Introducción}

Esta presentación pretende exponer los elementos básicos del trabajo y la ruta a recorrer para el desarrollo de la investigación.

En la sección dedicada a las motivaciones personales ofrezco las razones detrás del interés en realizar una investigación sobre la categoría del testimonio, luego el interés de realizar este estudio en el ámbito de la filosofía analítica y por qué específicamente en el pensamiento de G. E. M. Anscombe. Con esto quisiera ofrecer una justificación para este estudio y definir el marco de la reflexión posterior. Presento en segundo lugar el objetivo de la tesina desde el cual queda orientada la perspectiva del trabajo y la cuestión principal que pretende abordar. En tercer lugar enumero las fuentes básicas catalogadas según los temas principales que se trabajarán en la investigación. Finalmente describo el esquema que da estructura al estudio.

\section{Motivación Personal}

\subsection{Categoría del Testimonio}

Mi interés por la categoría del testimonio tiene su origen en la enseñanza de la Escritura, especialmente en los escritos del \textbf{evangelista San Juan}. El testimonio de Cristo que nos da el evangelista, en una cierta relación de circularidad con mi propia \textbf{experiencia creyente} dentro de la Iglesia, ha sido fundamento claro de mi comprensión de la fe y la Revelación de Dios.

La categoría del testimonio es además un concepto fundamental del lenguaje que hemos recibido del \textbf{Concilio Vaticano II}. Es una \emph{clave de interpretación} de los desafios históricos a los que responde la Iglesia en su misión evangelizadora y expresa además la \emph{vocación} del pueblo de Dios a proclamar el Evangelio con el testimonio de nuestra vida en \textbf{diálogo con las múltiples culturas y religiones} que conforman la sociedad. Esta llamada a vivir la fe en diálogo con el hombre contemporaneo es una valiosa enseñanza del Espíritu Santo para la Iglesia.
\footnote{ 
\emph{
``El Concilio fue un tiempo de gracia en que el Espíritu Santo nos enseñó que la Iglesia, en su camino en la historia, debe siempre hablar al hombre contemporáneo, pero esto sólo puede ocurrir por la fuerza de aquellos que tienen raíces profundas en Dios, se dejan guiar por Él y viven con pureza la propia fe; no viene de quien se adapta al momento que pasa, de quien escoge el camino más cómodo. El Concilio lo tenía bien claro, cuando en la constitución dogmática sobre la Iglesia Lumen Gentium, en el número 49, afirmó que todos en la Iglesia están llamados a la santidad según las palabras del Apóstol Pablo: <<Esta es la voluntad de Dios: vuestra santificación>> (1 Tes 4, 3). La santidad muestra el verdadero rostro de la Iglesia, hace entrar el <<hoy>> eterno de Dios en el <<hoy>> de nuestra vida, en el <<hoy>> del hombre de nuestra época.'' (\cite{benxvi})
}
}
La \textbf{Nueva Evangelización} es la respuesta de la Iglesia de nuestra época a esta llamada del Espíritu y en esta tarea la categoría del testimonio sigue jugando un papel fundamental. 

En los siguientes puntos quisiera detenerme con más detalle en estas motivaciones para el estudio de la categoría del testimonio.

\subsubsection{Revelación y testimonio}

\emph{Lo que existía desde el principio, lo que hemos oído,
lo que hemos visto con nuestros ojos,
lo que contemplamos
y palparon nuestras manos
acerca de la \textbf{Palabra} de vida
--pues la Vida se manifestó,
y nosotros la hemos visto y damos \textbf{testimonio}
y os anunciamos la Vida eterna,
que estaba junto al Padre y que se nos maifestó--
lo que hemos visto y oído os lo anunciamos,
para que también vosotros estéis en comunión con nosotros.
Y nosotros estamos en comunión con el Padre y con su Hijo Jesucristo.
Os escribimos esto para que nuestro gozo sea completo.} (1Jn 1, 1--4)

El comienzo de la primera Carta de San Juan describe la Revelación en clave testimonial. El apóstol da testimonio de la Palabra de la vida que estaba junto al Padre y se manifestó. Este anuncio acogido introduce en la comunión que el apóstol tiene con el Padre y su Hijo y es motivo de gozo. En el centro de la dinámica de la Revelación está Cristo que es el testigo del Padre. R. Latourelle afirma al respecto:

\emph{
``En San Juan el testimonio culmina como narración, como confesión, como compromiso y como interiorización. El testigo es Cristo (Ap 1, 5; 3, 14); y para Cristo, atestiguar equivale a manifestar al Padre, a revelar al Padre.''
}\footcite[1529]{dicctf}

El testimonio de Cristo es acogido por el hombre y transmitido. Así lo vemos en las expresiones de Juan. Esta acogida es fruto de la acción del Dios trinitario, Latourelle continúa:

\emph{
``Cristo es, por tanto, el testigo absoluto, el que lleva en sí mismo la garantía de su testimonio. El hombre, sin embargo, no sería capaz de acoger por la fe este testimonio del absoluto, manifestado en la carne y el lenguaje de Jesús, sin una atracción interior (Jn 6, 44), que es un don del Padre y un testimonio del Espíritu (1Jn 5, 9--10).''
}\footcite[1530]{dicctf}

Así el que cree en Cristo lleva en sí el testimonio de Dios, Latourelle añade:

 \emph{``\ldots{}el que cree en Cristo tiene dentro de sí el testimonio de Dios. El testimonio que el creyente posee en sí mismo es el testimonio que el espíritu da del Hijo.\footnote{El comentario de San Agustín a Jn 1, 1--5 ofrece una reflexión significativa al respecto:
\emph{``El diario hablar hace que pierdan estima las palabras. Como suenan y desaparecen, han perdido su valor y ya no parecen ser sino palabras. Pero hay también en el hombre una palabra que ayuda dentro; el sonido sale por la boca. Existe una palabra que se pronuncia realmente por el espíritu: es lo que entiendes por el sonido, no el sonido mismo. Cuando digo Dios, pronuncio una palabra. Son cuatro letras y una sílaba. \textquestiondown{}Todo esto y nada más es Dios? \textquestiondown{}Cuatro letras y una sílaba? \textquestiondown{}O se dirá acaso que, cuanto menos digno de aprecio es el sonido exterior, tanto es más apreciable su significado? Instantáneamente nos viene el pensamiento de un grande y supremo ser que trasciende la mudable criatura carnal y animal. Y a mi pregunta: \textquestiondown{}Dios es mudable o inmutable?, al instante contestas: Lejos de mí ni aún el pensamiento de que Dios sea mudable; Dios es inmutable. Tu alma, aunque pequeña, aunque tal vez carnal todavía, sin poderlo resistir, confiesa la inmutabilidad de Dios y la mutabilidad de la criatura. \textquestiondown{}Con qué luz y de qué forma has podido ver lo que trasciende todo lo criado, que con certeza me contestas que Dios es inmutable? \textquestiondown{}Qué hay en tu corazón cuando te representas un ser viviente, eterno, omnipotente e infinito y cuya presencia está en todo y todo Él en todo lugar y sin que pueda por ninguno ser limitado? Esta representación es el Verbo de Dios en tu corazón. No es el sonido compuesto de cuatro letras y dos sílabas, Lo que se pronuncia desaparece: son sonidos, letras, sílabas. Lo que pasa es la palabra que suena. Lo que la palabra significa y existe en el ser pensante, que habla, y en el inteligente, que oye, permanece aun desaparecido el sonido.''} (\cite[Tratado I, n.~1, 8]{aguscomentjn}) } Si el testimonio se interioriza es siempre en relación con la palabra de Cristo que exterioriza la intimidad de su diálogo con el Padre.''}\footnote{\cite[1530]{dicctf}}

Juan ha conocido a Cristo y anuncia el testimonio de lo que ha visto y oído para que los que lo acojan estén en comunión con el Padre y el Hijo; ofrece un testimonio de los signos de Cristo para que crean en Él y creyendo tengan vida en su nombre.\footnote{Jn 20, 30--31 y 1 Jn 1,1--4. Además \cite[1530]{dicctf}} 

Este testimonio del apóstol no sólo ofrece la narración de lo que ha visto, sino que confiesa el misterio de Dios. Da testimonio de la palabra manifiesta que ha palpado, y anuncia la Vida eterna que estaba junto al Padre. Esta confesión nos recuerda al apóstol que ve las llagas y proclama: ``Señor mío y Dios mío''(Jn 20, 28); ``vio al hombre, y confesó a Dios.''\footnote{Nos referimos a la reflexión de S. Gregorio: \emph{``Pero como diga el Apóstol que la fe es la sustancia de cosas que se esperan (Heb 11,1), pero que no se ven evidentemente, se deduce que, en las que están a la vista, no cabe fe, sino conocimiento. Si, pues, Tomás vio y tocó, \textquestiondown{}por qué se le dice ``Porque me viste, creíste''? Pero una cosa vio y otra creyó; vio al hombre, y confesó a Dios. Mucho alegra lo que sigue: ``Bienaventurados los que no vieron y creyeron''. En esta sentencia estamos especialmente comprendidos, porque Aquel a quien no hemos visto en carne lo vemos por la fe, si la acompañamos con las obras, pues aquel cree verdaderamente que ejecuta obrando lo que cree.''}\cite[Homilía 26]{greg}
}

La categoría del testimonio como clave hermenéutica de la dinámica de la Revelación, por tanto, despierta gran interés para nuestra reflexión teológica. En palabras de Latourelle:

\emph{
``El testimonio pertenece al grupo de analogías empleadas por la Escritura para introducir al hombre en las riquezas del misterio divino, por ejemplo las categorías de alianza, de palabra, de paternidad y de filiación. Si la revelación misma se apoya en la experiencia humana del testimonio para expresar una de las relaciones fundamentales que unen al hombre con Dios, la reflexión teológica se encuentra entonces autorizada a explorar los datos de esta experiencia.''
}\footcite[1523]{dicctf}

El hombre conoce al Dios revelado en Cristo creyendo en él. Esta relación fundamental que une al hombre con Dios se expresa en la categoría del testimonio. Una reflexión teológica que examine esta categoría se sitúa en la vinculación entre fe y conocimiento de Dios. F. Conesa describe esta relación como sigue:

\emph{
``Es sabido que, especialmente en el Nuevo Testamento, se acentúa la vinculación entre la fe y el conocimiento de Dios. Esta íntima relación es también experimentada por el creyente, que entiende y vive su fe como un modo singular de conocer a Dios. Así lo percibe de modo singular nuestro clásico castellano en su ``Cantar del alma que se huelga de conoscer a Dios por la fe'': <<Qué bien sé yo ---afirma--- la fonte que mana y corre, aunque es de noche>>.''
}\footcite[15]{cyc}

Nuestra fe en Cristo y su testimonio nos permite conocer a Dios. \textquestiondown{}{}Qué significa aquí conocer? El testimonio de Cristo nos llega por el anuncio apostólico. \textquestiondown{}{}Qué valor tiene este anuncio? \textquestiondown{}{}Qué valor tienen las proposiciones que conocemos por la fe? A esto Conesa añade:

\emph{
``La reflexión sobre el aspecto o dimensión proposicional de la fe pondrá de relieve que no se trata de algo externo al mismo acto de creer. En la aceptación de las proposiciones de fe opera lo que denominamos <<lógica del testimonio>>: aceptamos las verdades porque las ha dicho aquel en quien ponemos razonablemente la confianza. Creemos algo porque creemos en alguien y le creemos. \textquestiondown{}{}Cuál es el valor epistemológico de este asentimiento?''
}\footcite[p.483]{feylogicaconesa}

D. Antiseri coloca a Cristo en el centro de esta <<lógica del testimonio>> como sigue:

\emph{
``Ahora bien, en el cristianismo el primer testigo es Cristo, y es a Cristo a quien, de testimonio en testimonio, tendremos que remitirnos. Su persona histórica es ---según la expresión de R. Guardini--- la que, sin resolverse en la historia, constituye la esencia del cristianismo.''
}\footcite[168]{antiseri}

Con esta investigación quisiera estudiar la categoría del testimonio como clave que expresa y que forma parte de la dinámica de la Revelación de Dios y de su transmisión. Al examinar la ``lógica'' que opera en las proposiciones que conocemos por testimonio quisiera indagar sobre el valor epistemológico de las proposiciones de fe y el conocimiento de Dios. El objetivo de esta valoración es el diálogo intercultural e interreligioso que forma parte de la misión evangelizadora de la Iglesia. La reflexión, por tanto, tiene una finalidad pastoral en el espíritu del Concilio Vaticano II y de la Nueva Evangelización.

\subsubsection{Testimonio como misión de la Iglesia en el Concilio Vaticano II}

El estudio está situado dentro de un contexto filosófico y uno teológico. El contexto teológico nos lo ofrece la teología desarrollada a partir del Concilio Vaticano II. Valoramos la categoría del testimonio desde su relevancia teológica y pastoral en el Concilio. Al respecto Latourelle comenta:

\emph{
``Lo que el Vaticano I entendía por el signo de la Iglesia, se concentra ahora en la categoría de testimonio. Una vez percibida esta trasposición, se constata que el tema del testimonio es uno de los temas principales y privilegiados del Vaticano II. Como un `leitmotiv', aparece en todas las constituciones y en todos los decretos. A los ojos del concilio, atestiguar significa acreditar el evangelio como verdad y salvación del hombre mediante una vida conforme con el evangelio.''}\footcite[1532]{dicctf}

Como categoría pastoral el testimonio se refiere a la vida comprometida de los que forman parte del pueblo de Dios, la cual es el signo de salvación para este mundo. Sobre esto afirma R. Pellitero:

\emph{
``El Vaticano II redescubrió el enorme valor evangelizador del testimonio cristiano, aún el más sencillo y cotidiano. Sus grandes afirmaciones están en la Constitución Lumen gentium y el Decreto Ad gentes. Entre los documentos postconciliares que se ocupan del tema destaca la Exhortación Evangelii Nuntiandi.''}\footcite[378--379]{ftcpellitero}

El concilio manifiesta su preocupación de hablar al hombre del siglo XX, que rechaza un tipo de santidad platónica y abstracta, y ofrece como respuesta el lenguaje del testimonio de la vida realmente comprometida de los cristianos, de la santidad vivida como compromiso total al servicio de Cristo.\footnote{\cite[1532 y 1533]{dicctf}} 

El testimonio es misión de toda la la Iglesia; los obispos ofrecen al mundo el rostro de la Iglesia con su trato y trabajo pastoral\footnote{GS 43}, los presbíteros, creciendo en el amor por el desempeño de su oficio, han de ser un vivo testimonio de Dios\footnote{LG 41}, los fieles han de dar testimonio de verdad como testigos de la resurrección\footnote{LG 28 y LG 38}, los religiosos deben ofrecer un testimonio sostenido por la integridad de la fe, por la caridad y el amor a la cruz y la esperanza de la gloria futura\footnote{PC 25}, los profesores han de dar testimonio tanto con su vida como con su doctrina\footnote{GE 8}, los misioneros han de ofrecer testimonio con una vida enteramente evangélica, con paciencia, longanimidad, suavidad, caridad sincera, y si es necesario hasta con la propia sangre.\footnote{AG 24}

El afán de hablar con el hombre contemporáneo es una motivación importante para este estudio y es una preocupación aprendida del Concilio. El testimonio que aparece como 'leitmotiv' en los documentos del Concilio está presente del mismo modo en la Nueva Evangelización, es preciso, por tanto, tener en cuenta este proceso actual que se desarrolla en continuidad con la misión del Concilio. 

\subsubsection{Testimonio y Nueva Evangelización}
En la homilía de las vísperas de los apóstoles Pedro y Pablo el Papa Benedicto XVI reflexionaba sobre la Nueva Evangelización y anunciaba su deseo de crear un Consejo Pontificio dedicado a la promoción de ésta. En la homilía expresaba como sigue el espíritu de este renovado impulso misionero:

\emph{
``No hay palabras para explicar cómo el venerable Juan Pablo II, en su largo pontificado, desarrolló esta proyección misionera, que —conviene recordar siempre— responde a la naturaleza misma de la Iglesia, la cual, con san Pablo, puede y debe repetir siempre: <<Si anuncio el Evangelio, no lo hago para gloriarme: al contrario, es para mí una necesidad imperiosa. \textexclamdown{}Ay de mí si no predicara el Evangelio!>> (1 Co 9, 16). El Papa Juan Pablo II representó <<en vivo>> la naturaleza misionera de la Iglesia, con los viajes apostólicos y con la insistencia de su magisterio en la urgencia de una <<nueva evangelización>>: <<nueva>> no en los contenidos, sino en el impulso interior, abierto a la gracia del Espíritu Santo, que constituye la fuerza de la ley nueva del Evangelio y que renueva siempre a la Iglesia; <<nueva>> en la búsqueda de modalidades que correspondan a la fuerza del Espíritu Santo y sean adecuadas a los tiempos y a las situaciones; <<nueva>> porque es necesaria incluso en países que ya han recibido el anuncio del Evangelio. A todos es evidente que mi Predecesor dio un impulso extraordinario a la misión de la Iglesia, no sólo —repito— por las distancias que recorrió, sino sobre todo por el genuino espíritu misionero que lo animaba y que nos dejó en herencia al alba del tercer milenio.''}\footcite{benxvi_ne}

En la Carta apostólica Porta Fidei expresó la importancia del testimonio en la renovación de la Iglesia:

\emph{``La renovación de la Iglesia pasa también a través del testimonio ofrecido por la vida de los creyentes: con su misma existencia en el mundo, los cristianos están llamados efectivamente a hacer resplandecer la Palabra de verdad que el Señor Jesús nos dejó.''}\footcite[n. 6]{PF}

Y en la misa celebrada en octubre del 2011 por la nueva evangelización el mismo Benedicto afirmaba:

\emph{``Los nuevos evangelizadores están llamados a ser los primeros en avanzar por este camino que es Cristo, para dar a conocer a los demás la belleza del Evangelio que da la vida. Y en este camino, nunca avanzamos solos, sino en compañía: una experiencia de comunión y de fraternidad que se ofrece a cuantos encontramos, para hacerlos partícipes de nuestra experiencia de Cristo y de su Iglesia. Así, el testimonio unido al anuncio puede abrir el corazón de quienes están en busca de la verdad, para que puedan descubrir el sentido de su propia vida.''}\footcite{benxvi_ne2}

En el 2012 el sínodo de los Obispos dedicó su XIII asamblea a la nueva evangelización para la transmisión de la fe cristiana. En el instrumento de trabajo la categoría del testimonio aparece con frecuencia y el mensaje final se apoya en su punto de partida sobre la imagen de la mujer samaritana que tras su encuentro con Jesús se convierte en testigo:

\emph{``Sólo Jesús es capaz de leer hasta lo más profundo del corazón y desvelarnos nuestra verdad: <<Me ha dicho todo lo que he hecho>>, confiesa la mujer a sus vecinos. Esta palabra de anuncio --a la que se une la pregunta que abre a la fe: <<\textquestiondown{}Será Él el Cristo?>>-- muestra que quien ha recibido la vida nueva del encuentro con Jesús, a su vez no puede hacer menos que convertirse en anunciador de verdad y esperanza para los demás. La pecadora convertida deviene mensajera de salvación y conduce a toda la ciudad hacia Jesús. De la acogida del testimonio la gente pasará después a la experiencia personal del encuentro: <<Ya no creemos por lo que tú has dicho; nosotros mismos lo hemos oído y sabemos que Él es verdaderamente el Salvador del mundo>>.''}\footcite{sinod}

En la Evangelii Gaudium el Papa Francisco recoge las palabras de Pablo VI Evangelii nuntiandi sobre la necesidad del testimonio en el anuncio del Evangelio en nuestra época:

\emph{``También en esta época la gente prefiere escuchar a los testigos: <<tiene sed de autenticidad [\ldots] Exige a los
evangelizadores que le hablen de un Dios a quien ellos conocen y tratan familiarmente como si lo
estuvieran viendo>>.''}\footcite[n. 150]{EG}

La convocación al jubileo de la misericordia se expresa también en clave testimonial:

\emph{``Queridos hermanos y hermanas, he pensado con frecuencia de qué forma la Iglesia puede hacer más evidente su misión de ser testigo de la misericordia. Es un camino que inicia con una conversión espiritual; y tenemos que recorrer este camino. Por eso he decidido convocar un Jubileo extraordinario que tenga en el centro la misericordia de Dios. Será un Año santo de la misericordia. Lo queremos vivir a la luz de la Palabra del Señor: <<Sed misericordiosos como el Padre>> (cf. Lc 6, 36). Esto especialmente para los confesores: \textexclamdown{}mucha misericordia!''}\footcite{fran_mis}

Las enseñanzas del Concilio Vaticano II y el entusiasmo por la Nueva Evangelización motivan el deseo de comprender profundamente la dinámica del testimonio para conocer también en profundidad nuestra misión de testigos. 

\subsubsection{El diálogo con las culturas y las religiones}
Habiendo considerado las enseñanzas del Concilio Vaticano II y de la Nueva Evangelización he querido explicar el interés por estudiar el testimonio como la modalidad de hablar con la sociedad actual. Ahora quisiera detallar algunas consideraciones metodológicas adicionales sobre este tema.

En el recorrido por la obra de Anscombe servirá para orientarnos lo sugerido en la Reflexión ``Conocer la verdad a través de testigos'' donde, al abordar la empresa del cristianismo de ofrecer el testimonio del Hecho de Cristo como verdad de salvación razonable y creíble en nuestra circunstancia histórica, se propone:

\emph{
``Si queremos mostrar que dicha empresa es razonable, por una parte debemos recuperar la legitimidad de la pregunta filosófica y religiosa sobre la verdad\ldots en segundo lugar, se tiene que aclarar la naturaleza original del testimonio cristiano; por último, se tiene que poder demostrar que el testimonio es un modo adecuado de conocer y de transmitir la verdad.''}\footcite[267]{pradesmulticr}

Estas tres cuestiones servirán de guía en el acercamiento al pensamiento de Anscombe. Junto a estos tres planteamientos servirá de orientación el estudio realizado por C. A. J. Coady en `Testimony'. Las cuestiones que Coady plantea ofrecen a esta investigación una buena guía. Coady describe como sigue las etapas de su reflexión:

\emph{``We have been concerned to examine the broad significance of testimony in our lives, its place in our epistemological landscape, the tradition of debate about that place, and finally, what can be said to vindicate the extensive significance we have claimed for our reliance upon testimony.''}\footcite[175]{testcoady}

La elección de estas cuestiones responden a la finalidad del trabajo. Al preguntarse sobre el testimonio como concepto epistémico Coady se cuestiona más sobre cómo \emph{nosotros} conocemos que el cómo \emph{yo} conozco. Igualmente en la reflexión de J. Prades el testimonio se examina en el contexto del diálogo. Al seguir estas cuestiones quisiera ubicar esta investigación dentro de este mismo contexto social y de diálogo.

\subsubsection{Wittgenstein y la filosofía analítica}

La tesina está centrada en el pensamiento de Elizabeth Anscombe, sin embargo, su relación con Wittgenstein y la filosofía analítica requiere que esta investigación tenga en cuenta algunas consideraciones básicas tanto del Profesor como de la escuela.

Para el concepto de filosofía analítica nos adscribimos por el momento a la definición que ofrece Conesa:

\emph{
``Por filosofía analítica entiendo aquel enfoque de los problemas filosóficos que pone un énfasis en el análisis lógico y conceptual como método privilegiado \mbox{---aunque} no \mbox{exclusivo---} para resolver estos problemas''}\footcite[16]{cyc}

La filosofía analítica nos ofrece un ámbito adecuado para el estudio del testimonio, como afirma el mismo Conesa:

\emph{
``En el ámbito de la filosofía analítica ---y especialmente por la influencia de Thomas Reid y Ludwig Wittgenstein--- asistimos a una revalorización del testimonio como forma de saber. Formulado en su forma más simple, se viene a decir que <<si alguien sabe que `p' y dice que `p', entonces quien lo escucha también sabe que `p'>>.''}\footcite[487]{feylogicaconesa}

Esta revalorización del testimonio en la filosofía analítica explica el interés de realizar esta investigación dentro de esta escuela. 

Las aportaciones de Wittgenstein en el campo de la filosofía analítica con el Tractatus y más tarde con las Investigaciones Filosóficas son fundamentales. Su influencia en el pensamiento de Anscombe es importante, así afirma Teichmann:

\emph{
``Elizabeth Anscombe's name will be known to many as that of the English translator of a good number of Wittgenstein's Writings, in particular of the summation of his later philosophy, the Philosophical Investigations.[\ldots]And it is probably the influence of Wittgenstein's later philosophy that most readers of Anscombe will detect, rather than of his early philosophy. But her most sustained direct discussion of Wittgenstein is of course to be found in An Introduction to Wittgenstein's Tractatus; and it is interesting how themes from the Tractatus crop up in various guises in her articles, troughout her career.''}\footcite[191]{teichmann}

Aunque la relación entre Anscombe y Wittgenstein era muy estrecha, sus pensamientos y actitudes eran distintos en diversos temas. Una cuestión en la que ambos filósofos difieren y que será interesante para nuestro estudio es el tema de la razonabilidad de la fe. Al recorrer la biografía de Anscombe y su relación con Wittgenstein abordaré esta cuestión como punto de partida en el estudio de Anscombe. Tambíen nos servira de marco el pensamiento de Wittgenstein al tratar el tema del misterio y el sinsentido al final del recorrido por el pensamiento de Elizabeth.

\subsection{G. E. M. Anscombe}

Francisco Conesa ubica a Anscombe dentro de ``aquellos autores que entienden la fe primordialmente como un saber por testimonio''\footcite[84]{cyc}, esto la sitúa en el centro de nuestra investigación. 

Anscombe se convirtió a la fe católica en su juventud temprana y esta fe juega un papel importantísimo en su vida, pensamiento y filosofía. Este dato es llamativo cuando se considera su estrecha relación con Wittgenstein y la filosofía analítica, ella misma, sin embargo explica:

\emph{
``Analytical philosophy is more characterized by styles of argument and investigation than by doctrinal content. It is thus possible for people of widely different beliefs to be practitioners of this sort of philosophy. It ought not to surprise anyone that a seriously beleiving Catholic Christian should also be an analytical philosopher.''}\footcite[66]{opinionsanscombe}

Una escena de su vida puede resumir el interés por su pensamiento en este estudio. Su hija, Mary Geach, nos narra:

\emph{
``She told me how one day ---I think it was as an undergraduate--- she had come across a passage in Russell to the effect that an argument from the facts about the world to the existence of God could not be valid, as one could not deduce a necessary conclusion from a contingent premiss. She had not at the time been able to see what was wrong with the notion that necessities can only be deduced from necessities, but she had known that to deny the possibility of moving by reason from the facts about the world to a knowledge of existence of God was to deny a doctrine defined as of faith by an ecumenical council. She went then to church and made an act of faith [\ldots]. She realized later that of course one can derive necessary conclusions from contingent premises. If she had relied on her own understanding, she would have lost her faith for a falsehood.''}

La actitud de Anscombe hacia la Iglesia y el misterio de la Revelación tanto como su lugar dentro de la filosofía analítica son muy relevantes para el tipo de reflexión teológica que quisiéramos desarrollar. Más aún son relevantes para una reflexión sobre el testimonio en nuestro contexto contemporáneo.

\subsection{Perspectiva Teológico-Fundamental}

En el primer punto resaltaba la relación entre testimonio y Revelación, en el examen de esta relación anticipaba ya la perspectiva teológico-fundamental del trabajo. A esto quisiera añadir algunas consideraciones más.

La reflexión ``La Identidad de la Teología Fundamental'' de Salvador Pié-Ninot nos da una buena pauta sobre la misión de la Teología Fundamental y, en consecuencia, sobre la perspectiva que ha de tener esta investigación. 

Al exponer la nueva imagen de la teología fundamental que se va desarrollando en la etapa entre 1980 y 1998 observa que \emph{``esta nueva imagen se puede tipificar en torno a dos grandes modelos con sus aspectos decisivos: el epistemológico, que proviene de la concepción subyacente de la credibilidad, y el sistemático, que manifiesta la estructuración y contenido de la Teología Fundamental en esta etapa''}\footcite[29]{ninotTF}.

Sobre la epistemología comenta:

\emph{
``La renovación de la TF va muy ligada a las componentes epistemológicas ``humanas'' del acto de fe: en definitiva la función de la credibilidad y la de los signos de ella.''}\footcite[31]{ninotTF}

En cuanto al modelo sistemático (monstratio religiosa, christiana y catholica) al abordar la monstratio catholica aporta su propia reflexión:

\emph{
``el tratado sobre la Iglesia, ni es sólo la tercera monstratio teológico fundamental, ni es sólo el máximo signo de revelación como Cristo-en-la-Iglesia, sino que además es el marco englobante y significativo de toda la Teología Fundamental. Y esto a partir de la categoría testimonio en su doble función: la fundacional-hermenéutica ``ad-intra'' y la apologética-misionera ``ad-extra''. ''}

Añade que esta categoría conlleva una mutua circularidad entre la dimensión externa del testimonio eclesial, la dimensión interiorizada que es el testimonio creyente vivido y la dimensión interior e interiorizada que es el testimonio constante del Espíritu. De esta circularidad emerge la función del testimonio eclesial como camino de credibilidad que es invitación externa e interna a la vez. \footnote{\cite[40]{ninotTF}}

Al llegar a la propuesta para la teología fundamental que se encuentra en la Fides et Ratio distingue tres ejes importantes planteados para esta disciplina en el n. 67\footcite[49]{ninotTF}:\\1. Disciplina que da razón de la fe (cf. 1 Pe 3, 15).\\
2. Justifica y explicita la relación entre la fe y la reflexión filosófica.\\
3. Estudia la Revelación y su credibilidad, con el acto de fe.

Finalmente ofrece su propuesta:

\emph{
``La teología fundamental, ``como disciplina del Dar razón de la fe'', tiene como identidad: fundar y justificar la pretensión de verdad de la revelación cristiana como propuesta sensata de credibilidad.''}\footcite[72]{ninotTF}

Desde esta doble identidad identifica una doble tarea: la tarea de fundar y justificar la denomina ``tarea dogmático-fundacional'' y la pretensión de la verdad como propuesta sensata de credibilidad la llama ``tarea apologético-misionera''. De este modo ofrece una concepción de la Teología Fundamental que integra una doble tarea y se orienta por un lado hacia una `martiría' como expresión de su dimensión apologética y por otro lado se muestra inteligente como manifestación de su dimensión fundante en la perspectiva de la esperanza cristiana.\footnote{\cite[72]{ninotTF}}

Esta doble tarea es la que que asume esta investigación al pretender orientarse desde una perspectiva teológico-fundamental. La valoración de la categoría del testimonio en el pensamiento de Elizabeth Anscombe se realizará en clave de ``teología fundante'' que busca arrojar luz sobre el testimonio que es componente epistemológico en la experiencia humana y categoría teológica con una función fundacional-hermenéutica. Es también una valoración en clave de ``teología apologética'' que busca describir la categoría del testimonio en su función apologética-misionera y estudiar su valor como elemento de credibilidad de la Revelación.


\printbibliography[title={Referencias usadas en la presentación:},keyword=pres]

\part*{Objetivo de la Tesina}

\emph{Dada la valoración de la categoría del testimonio en la escritura, la experiencia creyente y el Concilio Vaticano II.}
\begin{itemize}
  \item Ofrecer una reflexión sobre la categoría del testimonio como un camino de conocimiento de la verdad de Dios en clave fundacional-hermenéutica y apologética-misionera.
  \item Ofrecer una reflexión fundada en la aportación de G. E. M. Anscombe sobre la categoría teológica del testimonio.
\end{itemize}

\part*{Fuentes básicas y bibliografía elemental que se va a consultar}

%\subsubsection{Bibliografía Primaria}
\noindent J. M. Torralba ofrece una recopilación de la Bibliografía en Anscombe en:\\
\emph{\href{http://www.unav.es/filosofia/jmtorralba/anscombe/G.E.M.\_Anscombe\_Bibliography.htm}{http://www.unav.es/filosofia/jmtorralba/anscombe/G.E.M.\_Anscombe\_Bibliography.htm}}

\noindent Escritos de Anscombe:

\begin{itemize}
    \item Colecciónes:
      \begin{itemize}
          \item Collected Philosophical Papers:
            \begin{itemize}
                \item \fullcite{collectedppI}
                \item \fullcite{collectedppII}
                \item \fullcite{collectedppIII}
            \end{itemize}

          \item St. Andrews Studies:
            \begin{itemize}
                \item \fullcite{hlae}
                \item \fullcite{fhg}
                \item \fullcite{ptow}
            \end{itemize}
      \end{itemize}
    \item Otros Escritos:
      \begin{itemize}
        \item \fullcite{intention}
        \item \fullcite{introtract}
      \end{itemize}
    \item Conferencias:
      \begin{itemize}
        \item \fullcite{torralba}
      \end{itemize}

\end{itemize}

\noindent Estudios sobre el pensamiento de Anscombe:

\begin{itemize}
    \item \fullcite{lcateich}
    \item \fullcite{philteich}
\end{itemize}

\subsubsection{Bibliografía Secundaria}
\subparagraph{Comentarios y escritos sobre Anscombe y su obra}
\subparagraph{Escritos sobre Wittgenstein}
\subparagraph{Escritos sobre filosofía analítica}



\section{Bibliografía Primaria}

\subsection {Escritos de Anscombe:}
\noindent J. M. Torralba ofrece una recopilación de la Bibliografía en Anscombe en:\\
\emph{\href{http://www.unav.es/filosofia/jmtorralba/anscombe/G.E.M.\_Anscombe\_Bibliography.htm}{http://www.unav.es/filosofia/jmtorralba/anscombe/G.E.M.\_Anscombe\_Bibliography.htm}}

\subsubsection{Colecciones}

\textbf{Collected Philosophical Papers:}
\nocite{collectedppIref}
\nocite{collectedppIIref}
\nocite{collectedppIIIref}
\printbibliography[heading=none,keyword=anscombe]


\textbf{St. Andrews Studies}
\nocite{hlaeref}
\nocite{fhgref}
\nocite{ptowref}
\printbibliography[heading=none,keyword=standrews]

\subsubsection{Otros Escritos}
\nocite{intentionref}
\nocite{introtractref}
\printbibliography[heading=none,keyword=otros]

\subsubsection{Conferencias}
\nocite{torralbaref}
\printbibliography[heading=none,keyword=conferencias]

\subsection{Estudios sobre el pensamiento de Anscombe:}

\nocite{lcateichref}
\nocite{philteichref}
\nocite{oncertref}
\printbibliography[heading=none,keyword=estudios]

\section{Bibliografía Secundaria}
\subsection{Wittgenstein}
\nocite{tractatusref}
\nocite{philinvref}
\nocite{oncertref}
\nocite{notebooksref}
\printbibliography[heading=none,keyword=witt]
\subsection{Escritos sobre la fe y filosofía analítica}
\nocite{antiseriref}
\nocite{cycref}
\printbibliography[heading=none,keyword=fefilo]
\subsection{Escritos sobre testimonio}
\nocite{testcoadyref}
\nocite{dicctfref}
\nocite{ftcpelliteroref}
\nocite{callaghanref}
\nocite{feylogicaconesaref}
\printbibliography[heading=none,keyword=testimonio]
\subsection{Escritos sobre teología fundamental}
\nocite{pradesmulticrref}
\nocite{CITFref}
\printbibliography[heading=none,keyword=tfund]

\subsection{Documentos Magisteriales}

\nocite{dvref}
\nocite{lgref}
\nocite{gsref}
\nocite{frref}
\printbibliography[heading=none,keyword=mag]


\part*{Esquema General}
\nohyphens{
\begin{spacing}{1.5}

\newcounter{esqchapter}
\newcounter{esqsection}[esqchapter]
\newcounter{esqsubsection}[esqsection]
\newcounter{esqsubsubsection}[esqsubsection]
\addtocounter{esqchapter}{1}
\addtocounter{esqsection}{1} % set them to some other numbers than 0
\addtocounter{esqsubsection}{1} % same
\addtocounter{esqsubsubsection}{1} % same


{\LARGE Capítulo \Roman{esqchapter}: 
Naturaleza del Testimonio Cristiano}
\begin{spacing}{1}
En este primer capítulo nos preguntamos \textquestiondown{}Cuál testimonio es Revelación de Dios? \textquestiondown{}Qué papel juega el testimonio en la Dinámica de la Revelación y su transmisión? Como referencia principal tendremos el artículo de Latourelle sobre el Testimonio en el Diccionario de Teología Fundamental. 
\end{spacing}


\tab {\Large \textbf{\Alph{esqsection}.} 
El testimonio en la dinámica de la Revelación}

\tab \tab {\large \Alph{esqsection}. 
\arabic{esqsubsection}. 
Cristo Testigo}

\stepcounter{esqsubsection}
\tab \tab {\large \Alph{esqsection}. 
\arabic{esqsubsection}. 
La transmisión de la revelación: el Testimonio apostólico}

\stepcounter{esqsubsection}
\tab \tab {\large \Alph{esqsection}. 
\arabic{esqsubsection}. 
El testimonio de la fe en el martirio}

\stepcounter{esqsection}
\tab {\Large \textbf{\Alph{esqsection}.}
La categoría del testimonio en el contexto actual de la Teología Fundamental}

\stepcounter{esqsubsection}
\tab \tab {\large \Alph{esqsection}. 
\arabic{esqsubsection}. 
Revalorización del testimonio a partir del Concilio Vaticano II}

\tab \tab \tab \Alph{esqsection}. 
\arabic{esqsubsection}. 
\stepcounter{esqsubsubsection}
(\roman{esqsubsubsection})
Vaticano I

\tab \tab \tab \Alph{esqsection}. 
\arabic{esqsubsection}. 
\stepcounter{esqsubsubsection}
(\roman{esqsubsubsection})
Vaticano II

\tab \tab \tab \Alph{esqsection}. 
\arabic{esqsubsection}. 
\stepcounter{esqsubsubsection}
(\roman{esqsubsubsection})
Magisterio Post-Conciliar

\tab \tab \tab \Alph{esqsection}. 
\arabic{esqsubsection}. 
\stepcounter{esqsubsubsection}
(\roman{esqsubsubsection})
Nueva Evangelización

\stepcounter{esqsubsection}
\tab \tab {\large \Alph{esqsection}. 
\arabic{esqsubsection}. 
El Testimonio como categoría hermeneútica}

\stepcounter{esqsubsection}
\tab \tab {\large \Alph{esqsection}. 
\arabic{esqsubsection}. 
El Testimonio como misión}

\stepcounter{esqchapter}
{\LARGE Capítulo \Roman{esqchapter}: 
El Valor Cognoscitvo de la Fe y el problema del lenguaje religioso en la Filosofía Analítica}

\begin{spacing}{1}
Este capítulo sigue el estudio de F. Conesa en Creer y Conocer. En él se plantean las cuestiones y problemáticas fundamentales relacionadas con el valor cognoscitivo de la fe en el ámbito de la filosofía analítica.  \textquestiondown{}Cuál es la significatividad del lenguaje religioso? Si se admite que no carece de significado, \textquestiondown{}Qué valor cognoscitivo tiene? Si se afirma que es susceptible de ser verdadero o falso \textquestiondown{}existe un conocimiento religioso? \textquestiondown{}Cuál es su valor? Esta cuestión nos lleva a la afirmación que estudiaremos en Anscombe: el valor cognoscitivo de la Fe es el del saber por testimonio.
\end{spacing}

\stepcounter{esqsection}
\tab {\Large \textbf{\Alph{esqsection}.}
Significatividad del Lenguaje Religioso}

\stepcounter{esqsection}
\tab {\Large \textbf{\Alph{esqsection}.}
El Valor Cognoscitivo del Lenguaje Religioso}

\stepcounter{esqsection}
\tab {\Large \textbf{\Alph{esqsection}.}
El Valor Cognoscitivo de la Fe Religiosa}

\stepcounter{esqsection}
\tab {\Large \textbf{\Alph{esqsection}.}
La Fe como Saber Por Testimonio}

\stepcounter{esqchapter}
{\LARGE Capítulo \Roman{esqchapter}: 
La Categoría del Testimonio en el Pensamiento de Elizabeth Anscombe}
\begin{spacing}{1}
Este es el capítulo central del trabajo. El recorrido que haremos a lo largo del pensamiento de Anscombe está orientado según las siguientes cuestiones:

\noindent- La legitimidad de la pregunta filosófica y religiosa sobre la verdad.\\
- Aclarar la naturaleza original del testimonio cristiano.\\
- Demostrar que el testimonio es un modo adecuado de conocer y de transmitir la verdad\\
- Examen de la relevancia del testimonio en nuestras vidas\\
- Lugar del testimonio en el panorama epistemológico y la tradición del debate sobre ese lugar\\
- Qué se puede decir para defender la extensa relevancia que se ha afirmado sobre nuestra confianza en el testimonio

Anscombe no crea un sistema o teoría general en el acercamiento a los problemas filosóficos, más bien toma cada caso en sus propios méritos. Esta evasión de sistema complica la tarea de estudiar su pensamiento. Como respuesta a esta dificultad es útil la metodología empleada por Teichmann en el estudio de la filosofía de Anscombe que el mismo describe en tres momentos: ``read thoroughly'', ``bring out the manifold connections between her thoughts on different topics'' y ``engage with what Anscombe says''.

El pensamiento de Anscombe sobre la categoría del testimonio se estudiará como sigue:

1. Examinar su historia y relación con Wittgenstein, en este examen nos preguntamos: Nos debemos preocupar por que la fe sea razonable? Qué cuestiones plantea Wittgenstein al conocimiento por la fe y el lenguaje religioso?
2. 


\end{spacing}
\stepcounter{esqsection}
\tab {\Large \textbf{\Alph{esqsection}.}
Wittgenstein y Anscombe: La Razonabilidad de la Fe}
\footnote{
Ludwig Wittgenstein, 
Wittgenstein on Rules and Private Language, 
Wittgenstein, Frege and Ramsey, 
Wittgenstein: Whose Philosopher?, 
Wittgenstein's 'two cuts' in the history of philosophy, 
Consequences of the Picture Theory, 
On the form of Wittgsenstein's writing, 
Was Wittgenstein a conventionalist?, 
The Simplicity of the Tractatus, 
An Introduction to Wittgenstein's Tractatus
}

\stepcounter{esqsection}
\tab {\Large \textbf{\Alph{esqsection}.}
La pregunta sobre la Verdad}
\stepcounter{esqsubsection}
\footnote{
Truth: Anselm and Wittgenstein, 
Truth: Anselm or Thomas?, 
Anselm and the Unity of Truth, 
A theory of Language?, 
Necessity and Truth, 
Thought and Action in Aristotle: What is Practical Truth?, 
Practical Truth
}

\stepcounter{esqsection}
\tab {\Large \textbf{\Alph{esqsection}.}
Valor Cognoscitivo de la fe}

\tab \tab \tab \Alph{esqsection}. 
\arabic{esqsubsection}. 
\stepcounter{esqsubsubsection}
(\roman{esqsubsubsection})
Proposiciones de fe y lenguaje religioso
\footnote{
Faith, 
A Reply to Mr. C. S. Lewis's Argument that “Naturalism” is Self- Refuting, 
Has Mankind One Soul: An Angel Distributed among many Bodies?, 
Human Essence, 
La esencia Humana, 
Plato, Soul and 'the Unity of Apperception', 
Why Anselm's Proof in the Proslogion in not an onthological argument, 
On the Hatred of God, 
On Attachment to Things and Obedience to God, 
On being on Good Faith, 
On Humanae Vitae, 
Philosophers and Economists: Two Philosphers' Objections to Usury, 
Retractation, 
Sin: the McGivney Lectures, 
The Inmortality of the Soul, 
Two Moral Theologians, 
You Can Have Sex without Children: Christianity and the New Offer, 
Morality, 
Modern Moral Philosophy
}

\stepcounter{esqsubsection}
\tab \tab {\large \Alph{esqsection}. 
\arabic{esqsubsection}. 
La fe es un modo de conocimiento, es saber por testimonio}
\footnote{
Faith, 
What is to believe someone?
}

\stepcounter{esqsection}
\tab {\Large \textbf{\Alph{esqsection}.}
Legitimidad de la confianza puesta en el conocimento por testimonio}


\stepcounter{esqsubsection}
\tab \tab {\large \Alph{esqsection}. 
\arabic{esqsubsection}. 
La Tradición: Hume sobre el valor epistemológico del testimonio}
\footnote{
Hume and Julius Caesar, 
Hume on causality: introductory, 
The Reality of the Past, 
Causality and Determination, 
Causality and Extensionality, 
“Whatever has a beginning of existence must have a cause”: Hume's Argument Exposed, 
Times, Beginnings and Causes, 
Before and After, 
The Causation of Action, 
Chisolm on Action, 
Action, Intention and 'Double Effect', 
Part Three: Causality and time
}

\tab \tab \tab \Alph{esqsection}. 
\arabic{esqsubsection}. 
\stepcounter{esqsubsubsection}
(\roman{esqsubsubsection})
El Conocimiento de la historia
\footnote{
The Reality of the Past, 
Hume and Julius Caesar
}

\tab \tab \tab \Alph{esqsection}. 
\arabic{esqsubsection}. 
\stepcounter{esqsubsubsection}
(\roman{esqsubsubsection})
Las narraciones extraordinarias
\footnote{
Aristotle and the Sea Battle: De Interpretatione, Chapter IX, 
Prophecy and Miracles, 
Hume on Miracles, 
Modern Moral Philosophy, 
Good and Bad Human Action
}

\stepcounter{esqsubsection}
\tab \tab {\large \Alph{esqsection}. 
\arabic{esqsubsection}. 
Lenguaje epistémico y lógica del testimonio}
\footnote{
On Wisdom, 
Knowledge and Certainty, 
Knowledge and Reverence for Human Life, 
'The General Form of Proposition', 
Comments on Professor R. L. Gregory's Paper on Perception, 
On Brute Facts, 
Will and Emotion, 
Memory, 'Experience' and Causation, 
Understanding Proofs: Meno, 85d9 – 86c2, 
Subjunctive Conditionals, 
What is it to Believe Someone?, 
The Intentionality of Sensation, 
Substance, 
The Subjectivity of Sensation, 
Events in the mind, 
On Sensations of Position, 
Intention, 
Pretending, 
Practical Inference
}


\tab \tab \tab \Alph{esqsection}. 
\arabic{esqsubsection}. 
\stepcounter{esqsubsubsection}
(\roman{esqsubsubsection})
Creer
\footnote{
What is it to Believe Someone?
}

\tab \tab \tab \Alph{esqsection}. 
\arabic{esqsubsection}. 
\stepcounter{esqsubsubsection}
(\roman{esqsubsubsection})
Autoridad
\footnote{
Authority in Morals, 
On the Source of the Authority of the State, 
The Moral Enviroment of the Child, 
On Promising and its justice, and Whether it Need be Respected in Foro Interno, 
Rules, Rights and Promises, 
The Two Kinds of error in action
}

\tab \tab \tab \Alph{esqsection}. 
\arabic{esqsubsection}. 
\stepcounter{esqsubsubsection}
(\roman{esqsubsubsection})
Percepción, memoria, entendimiento y testimonio

\stepcounter{esqsubsection}
\tab \tab {\large \Alph{esqsection}. 
\arabic{esqsubsection}. 
Sense, Nonsense and Mystery}
\footnote{
`Mysticism' and Solipsism, 
Analytical Philosophy and the Sipirituality of Man, 
On Transubstantiation, 
Parmenides, Mystery and Contradiction, 
The Question of Linguistic Idealism, 
Paganism, Superstition and Philosophy, 
On Piety, or: Plato's Euthyphro.
}

\stepcounter{esqchapter}
{\LARGE Capítulo \Roman{esqchapter}: 
Valoración y Crítica}

\stepcounter{esqsection}
\tab {\Large \textbf{\Alph{esqsection}.}
Fundamento de Credibilidad}

\stepcounter{esqsection}
\tab {\Large \textbf{\Alph{esqsection}.}
Autoridad Moral}


\end{spacing}
}

\end{document}
