\documentclass[12pt]{article}

\usepackage{hyperref}
\usepackage[T1]{fontenc}
\usepackage[utf8]{inputenc}
\usepackage[scaled]{uarial}

\usepackage[spanish]{babel}

\renewcommand*{\familydefault}{\sfdefault}

\setcounter{secnumdepth}{0}

\setlength{\parindent}{1.25cm}
\setlength{\parskip}{0.39cm}

\begin{document}

\section{Introducción}
\subsection{Motivación Personal}
\subsubsection{1Jn 1-4}
Lo que existía desde el principio, lo que hemos oído, lo que hemos visto con nuestros ojos, lo que contemplamos y palparon nuestras manos acerca de la \textbf{Palabra} de vida ---\emph{pues la Vida se manifestó, y nosotros la hemos visto y damos \textbf{testimonio} y os anunciamos la Vida eterna, que estaba junto al Padre y que se nos maifestó}--- lo que hemos visto y oído os lo anunciamos, para que también vosotros \emph{estéis en comunión con nosotros}. Y \emph{nosotros estamos en comunión con el Padre y con su Hijo Jesucristo}. \textbf{Os escribimos esto para que nuestro gozo sea completo}.

\subsubsection{Experiencia Creyente}
Conocer a Dios para vivir en comunión con Él. 

\subsubsection{Los retos del diálogo}
El Diálogo con la cultura. El diálogo interreligioso. 

\subsection{Objetivo de la Tesina}
\subsubsection{Reflexionar sobre la categoría del testimonio como un camino de conocimiento de la verdad de Dios.}
\subsubsection{Presentar la aportación de G. E. M. Anscombe sobre la catégoría teológica del testimonio.}

\subsection{Fuentes básicas y bibliografía elemental que se van a consultar}
\subsubsection{Bibliografía Primaria}
\href{http://www.unav.es/filosofia/jmtorralba/anscombe/G.E.M.\_Anscombe\_Bibliography.htm}{Bibliografía de Elizabeth Anscombe recopilada por J. M. Torralba}
\subsubsection{Bibliografía Secundaria}
\subparagraph{Comentarios y escritos sobre Anscombe y su obra}
\subparagraph{Escritos sobre Wittgenstein}
\subparagraph{Escritos sobre filosofía analítica}

\end{document}